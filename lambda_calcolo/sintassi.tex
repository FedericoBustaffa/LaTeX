\section{Sintassi}
Il \textbf{lambda calcolo} può essere visto come il primo \textbf{linguaggio funzionale} e consente di definire
solo funzioni anonime con una sintassi minimale e molto semplice.

Un programma nel lambda calcolo è di fatto un'espressione, detta \textbf{lambda espressione}, la cui sintassi ha
3 componenti principali:
\begin{itemize}
	\item \textbf{Identificatori}
	      \[ e ::= x \]
	      In questo modo l'\textbf{identificatore} $x$ assume come valore l'espressione $e$.
	\item \textbf{Astrazione funzionale}
	      \[ \lambda x.e \]
	      Indica una funzione (anonima) con parametro $x$ e corpo $e$.
	\item \textbf{Applicazione funzionale}
	      \[ e_1 \; e_2 \]
	      In questo modo si applica l'espressione $e_1$ al parametro $e_2$.
\end{itemize}
Una funzione può essere definita e applicata scrivendo tutto nella solita riga.
\[ (\lambda x.(\lambda y.xy)) \; (\lambda z.z) \]

\subsection{Costruttore lambda}
Lo \textbf{scope} del lambda si estende il più a destra possibile. Scrivere
\[ \lambda x. \lambda y. x y \]
equivale a
\[ \lambda x. (\lambda y. (x y)) \]
In entrambi i casi stiamo definendo una funzione che prende $x$ come parametro e come corpo ha una funzione con
parametro $y$ e corpo l'applicazione di $x$ a $y$, ovvero $xy$.

\subsection{Applicazione funzionale}
L'\textbf{applicazione funzionale} associa a sinistra. Scrivere
\[ e_1 \; e_2 \; e_3 \]
è come scrivere
\[ (e_1 \; e_2) \; e_3 \]
In entrambi i casi stiamo applicando $e_3$ all'applicazione di $e_1$ su $e_2$.

\subsection{Variabili legate e libere}
Tutte le altre variabili che non sono associate a una dichiarazione con un qualche $\lambda$ sono \textbf{libere}.

Le variabili in una lambda espressione che sono introdotte (dichiarate) in un qualche $\lambda$ sono
\textbf{legate} da quel $\lambda$.

\subsubsection{Insieme delle variabili libere}
L'\textbf{insieme delle variabili libere} di una lambda espressione $e$, denotato come $FV(e)$, è definito
ricorsivamente sulla struttura di $e$ come segue:
\begin{gather*}
	FV(x) = \{ x \} \\
	FV(\lambda x.e) = FV(e) \; \backslash \; \{ x \} \\
	FV(e_1 \; e_2) = FV(e_1) \cup FV(e_2)
\end{gather*}
In altre parole il calcolo di $FV(e)$ scende ricorsivamente accumulando tramite l'unione tutte le variabili che
incontra nelle sottoespressioni. Nel momento in cui si incontra una $\lambda$-astrazione la variabile legata
viene rimossa dall'insieme delle variabili libere.

\subsection{Alpha equivalenza}
Due espressioni $e_1$ ed $e_2$ sono \textbf{$\alpha$-equivalenti} ($e_1 \equiv_\alpha e_2$) se l'una può essere
ottenuta dall'altra rinominando una o più variabili legate, senza interferire con le variabili libere. Le due
espressioni
\[ \lambda a.ac \quad \text{e} \quad \lambda b.bc \]
sono \textbf{$\alpha$-equivalenti}.

Espressioni $\alpha$-equivalenti rappresentano lo stesso programma.

\subsubsection{Alpha conversione}
La sostituzione di una variabile legata con un'altra opportuna variabile \emph{fresca} costituisce una
trasformazione sintattica di programmi che, solitamente in matematica, viene chiamata \textbf{$\alpha$-conversione}.

Nel $\lambda$-calcolo, un'espressione $e_2$, risultato dell'$\alpha$-conversione di un'espressione $e_1$ è
$\alpha$-equivalente a $e_1$.