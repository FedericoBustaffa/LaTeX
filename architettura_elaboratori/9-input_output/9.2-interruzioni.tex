\section{Interruzioni}
Vogliamo quindi ottimizzare l'interazione con i dispositivi di I/O e per farlo entrano in gioco le
\textbf{interruzioni}. Un modo per permettere a tali dispositivi di notificare al processore che
sono passati in uno stato di \emph{pronto}.

Fino ad ora abbiamo eseguito le varie operazioni in \textbf{modalità utente} ma ogni processore ha
a disposizione diversi \textbf{stati di esecuzione} che differiscono sostanzialmente per i alcuni
privilegi di cui possono godere o meno.

Quando si verifica un'interruzione il processore, il processore passa nello in uno stato detto
\textbf{stato interruzione} che interrompe ciò che il processore stava facendo fino a quel momento
per permettergli di gestire l'interruzione.

Quando il processore entra in questa modalità è in grado di usare una copia dei registri generali
in modo da non sporcare i veri registri generali ed inoltre è in grado di utilizzare il Memory
Mapped I/O che in modalità utente non è possibile usare.

