\documentclass[11pt]{report}

% --------------- PACKAGES ---------------
\usepackage{inputenc}
\usepackage[T1]{fontenc}
\usepackage[italian]{babel}
\usepackage[hidelinks]{hyperref}

\hypersetup{
	colorlinks=true,
	linkcolor=blue!70!black
}

% --------------- STYLE ---------------
\usepackage[margin=1.25in]{geometry}
\usepackage[most]{tcolorbox}

% Font
\usepackage{sansmath}

\renewcommand{\familydefault}{\sfdefault}
\sansmath

% page style
\usepackage{fancyhdr}
\usepackage[Sonny]{fncychap}

\pagestyle{fancy}
\setlength{\headheight}{15pt}
\rhead{\thepage}
\cfoot{\thepage}

% --------------- MATH ---------------
\usepackage{amsmath}
\usepackage{amssymb}
\usepackage{amsthm}
\usepackage{amsfonts}
\usepackage{mathtools}
\usepackage{mdframed}

\newcommand{\N}{\mathbb{N}}
\newcommand{\Z}{\mathbb{Z}}
\newcommand{\R}{\mathbb{R}}
\newcommand{\C}{\mathbb{C}}
\newcommand{\E}{\mathbb{E}}

\newcommand{\F}{\mathcal{F}}

\DeclareMathOperator{\Var}{Var}
\DeclareMathOperator{\Cov}{Cov}

% Boxes for theorem, definitions and examples
\newtheoremstyle{math_box}
{0pt}
{0pt}
{\normalfont}
{}
{\color{orange}}
{\;}
{0.25em}
{\thmname{\textbf{#1}}\thmnumber{ \textbf{#2}}{\color{black}\thmnote{\textbf{ -- #3.}}}}

\newmdenv[
	rightline=false,
	leftline=true,
	topline=false,
	bottomline=false,
	linecolor=orange!40,
	innerleftmargin=5pt,
	innerrightmargin=5pt,
	innertopmargin=0pt,
	innerbottommargin=0pt,
	leftmargin=0cm,
	rightmargin=0cm,
	linewidth=3pt
]{dBox}

\newmdenv[
	rightline=false,
	leftline=false,
	topline=false,
	bottomline=false,
	backgroundcolor=orange!15,
	innerleftmargin=5pt,
	innerrightmargin=5pt,
	innertopmargin=5pt,
	innerbottommargin=5pt,
	leftmargin=0cm,
	rightmargin=0cm,
]{pBox}

\theoremstyle{math_box}
\newtheorem{theoremeT}{Teorema}[chapter]
\newtheorem{definitionT}{Definizione}[chapter]
\newtheorem{propositionT}{Proposizione}[chapter]
\newtheorem{corollary}{Corollario}[chapter]
\newtheorem{lemma}{Lemma}[chapter]
\newtheorem{observation}{Osservazione}[chapter]
\newtheorem{exampleT}{Esempio}[section]

\newenvironment{theorem}{\begin{pBox}\begin{theoremeT}}{\end{theoremeT}\end{pBox}}
\newenvironment{definition}{\begin{dBox}\begin{definitionT}}{\end{definitionT}\end{dBox}}
\newenvironment{proposition}{\begin{pBox}\begin{propositionT}}{\end{propositionT}\end{pBox}}
\newenvironment{example}{\begin{dBox}\begin{exampleT}}{\end{exampleT}\end{dBox}}

\usepackage{tikz, pgfplots, pgf-pie}
\usepackage{caption, subcaption}
\usepackage{scalerel}
\usepackage{pict2e}
\usepackage{tkz-euclide}
\usepackage{pgfplots, pgfplotstable, pgf-pie}
\usepackage{circuitikz}

\usetikzlibrary{calc}
\usetikzlibrary{patterns, arrows}
\usetikzlibrary{shadows}
\usetikzlibrary{external}

\pgfplotsset{compat=newest}
\usepgfplotslibrary{statistics, fillbetween}

\tikzstyle{branch}=[
fill,
shape=circle,
minimum size=3pt,
inner sep=0pt
]


\usepackage{minted}
\usepackage{diagbox}
\usepackage[label=corner]{karnaugh-map}

\definecolor{minted_bg}{rgb}{0.9, 0.9, 0.9}
\usemintedstyle{borland}

\setminted[c]{
	tabsize=4,
	% linenos=true,
	bgcolor=minted_bg,
	fontsize=\small,
	mathescape=true
}

\setminted[gas]{
	tabsize=4,
	% linenos=true,
	bgcolor=minted_bg,
	fontsize=\small,
	mathescape=true
}

\title{Architettura degli elaboratori}
\author{Federico Bustaffa}
\date{13/11/2023}

\begin{document}

\maketitle
\tableofcontents

\chapter{Introduzione}
L'obbiettivo del corso è quello di progettare sistemi robusti che non crollino al primo attacco o davanti al
primo utente ingenuo che ne fa uso.

Gran parte degli oggetti di uso quotidiano sono stati resi "intelligenti" dall'aggiunta di una componente
informatica al loro interno. Tali oggetti, anche se dall'esterno possono sembrare estremamente limitati, sono
sistemi completi e flessibili in grado di eseguire qualsiasi software.

L'aggiornamento online del software all'interno di tali sistemi è un mezzo per mantenere l'oggetto all'avanguardia
ma implica che esso possa essere manipolato da remoto per eseguire qualsiasi software e dunque per modificare
il comportamento dell'oggetto a proprio piacimento.

\section{Proprietà di sicurezza}
Un sistema informatico è, ad ogni livello di implementazione, formato da un insieme di moduli connessi, ognuno dei
quali offre un certo numero di operazioni.

Queste operazioni permettono di leggere e manipolare informazioni, che hanno poi un impatto sul mondo esterno.

In ogni sistema informatico ci sono regole (\textbf{politica di sicurezza}) che definiscono chi può invocare una
certa operazione e quindi ha il diritto di leggere o manipolare informazioni. Tali regole vengono implementate da
un sottoinsieme dei moduli del sistema informatico.

In questo contesto le tre principali proprietà che ci interessano sono:
\begin{itemize}
	\item \textbf{Confidenzialità}: solo chi ha il diritto di leggere una certa informazione può farlo.
	\item \textbf{Integrità}: solo chi ha il diritto di aggiornare una certa informazione può farlo.
	\item \textbf{Disponibilitò}: chi ha un diritto e vuole esercitarlo riesce a farlo in un tempo finito.
\end{itemize}
Da queste tre proprietà fondamentali possiamo derivarne altre secondarie:
\begin{itemize}
	\item \textbf{Tracciabilità}: individuare chi ha invocato un'operazione.
	\item \textbf{Accountability}: addebitare l'uso delle risorse.
	\item \textbf{Auditability}: verificare l'efficacia dei meccanismi di \emph{enforcement} di una politica (come
	      viene realizzata).
	\item \textbf{Forensics}: provare che certe azioni hanno avuto luogo.
	\item \textbf{Privacy/GDPR}: individuare chi, come e se un utente può usare informazioni personali.
\end{itemize}

\section{Politiche di sicurezza}
Una \textbf{politica di sicurezza} è un insieme di regole definite dal proprietario del sistema o del processo
aziendale per decidere gli utenti che possono invocare un'operazione e quando possono farlo.

Esistono diverse categorie di politiche descrivibili come il risultato di due scelte indipendenti:
\begin{itemize}
	\item La prima scelta è relativa al come la politica viene descritta:
	      \begin{itemize}
		      \item \textbf{Default allow}: operazioni vietate.
		      \item \textbf{Default deny}: operazioni permesse.
	      \end{itemize}
	\item La seconda scelta definisce vincoli sul proprietario del sistema:
	      \begin{itemize}
		      \item \textbf{Discretionary Access Control}: decide il proprietario.
		      \item \textbf{Mandatory Access Control}: esistono vincoli globali a tutto il sistema che nemmeno il
		            proprietario può violare.
	      \end{itemize}
\end{itemize}

\subsection{Soggetti e oggetti}
Relativamente alla seconda caratteristica che una politica deve avere, si cerca di modellare le risorse condivise
come \textbf{oggetti} e gli utenti come \textbf{soggetti}.

I soggetti invocano le operazioni definite dagli oggetti, se un oggetto invoca le operazioni definite da altri
oggetti allora diventa esso stesso un soggetto. In sintesi
\begin{itemize}
	\item \textbf{Soggetto}: in genere è un utente, un processo, un thread, un'istruzione.
	\item \textbf{Oggetto}: in genere si tratta di tipi di dati astratti, procedure, parametri, risorse logiche o
	      fisiche.
\end{itemize}

\subsection{Discretionary Access Control - DAC}
In questo modello per ogni oggetto esiste un \textbf{proprietario} (del sistema o del processo), il quale decide i
diritti dei vari soggetti mentre lui non ha vincoli di alcun tipo. Questo modello è tipico del mondo industriale.

\subsection{Mandatory Access Control - MAC}
Questo modello prevede la divisione di soggetti (utenti) e oggetti (risorse) in \textbf{classi}. Le classi sono
ordinate parzialmente in \textbf{livelli} (1, 2, 3 e così via).

Il livello di un soggetto esprime il grado di libertà che vogliamo lasciare a tale soggetto. Tanto più alto è il
livello di un soggetto tanto maggiore sarà il livello degli oggetti con cui esso può interagire.

Il livello di un oggetto esprime invece il grado di importanza di tale oggetto. Tanto più alto è il livello di un
oggetto tanto maggiore dev'essere è il livello del soggetto perché esso possa interagire con tale oggetto.

\subsubsection{Information Flow I}
In questo tipo di politica un utente può
\begin{itemize}
	\item Leggere tutti i file che hanno classe minore o uguale alla sua.
	\item Modificare i record dei file che hanno classe uguale alla sua.
	\item Appendere un record ad un file che ha classe maggiore della sua.
\end{itemize}
Per compiere queste operazioni è necessario il permesso dell'\emph{owner} che può solo restringere ulteriormente ciò
che un utente può fare. Si tratta di una politica MAC di tipo \textbf{no write down} e privilegia la confidenzialità.

\subsubsection{Information Flow II}
In questo tipo di politica un utente può
\begin{itemize}
	\item Scrivere tutti i file di una classe minore o uguale alla sua.
	\item Leggere tutti i file di una classe maggiore o uguale.
\end{itemize}
Questo implica che un utente poco affidabile, ossia di basso livello, non può andare a modificare dati critici. Si
tratta di una politica MAC di tipo \textbf{no write up} e privilegia l'integrità.

\subsubsection{Chinese Wall}
Gli oggetti del sistema sono partizionati in sottoinsiemi. L'utente che ha operato su un oggetto di un insieme non
può operare su oggetti di un altro insieme.

Questa politica \textbf{dinamica} permette di gestire conflitti di interesse ed è compatibile con la politica MAC
poiché aggiunge vincoli.

\subsubsection{Watermark}
Questa politica non prevede che un soggetto abbia un livello fissato ma che varia in base alle operazioni che esso
compie sui vari oggetti e dal livello di questi ultimi.

Il livello di un soggetto aumenta dopo che questo legge un oggetto di livello più alto del suo, rimane invece invariato
se il soggetto legge un oggetto di livello più basso.

\section{Matrice di controllo degli accessi}
Si tratta di una matrice con un comportamento molto dinamico che ha tante righe quanti sono i soggetti e tante colonne
quanti sono gli oggetti.

Nella posizione identificata dal soggetto $i$ e dall'oggetto $j$ si trovano i \textbf{diritti} che quel soggetto ha
su quell'oggetto. In generale è bene ai fini di sicurezza che la matrice contenga pochi diritti e che dunque appaia
sostanzialmente vuota.

Una rappresentazione concreta di tale matrice è necessaria ma non sufficiente per il rispetto della politica.

\subsection{Linux}
Nei sistemi operativi Linux vi è una rappresentazione concreta di tale matrice nel file system. Per ogni file, Linux
fornisce una sequenza di bit (bitmask) che indica i diritti che l'utente ha su tali file (lettura, scrittura,
esecuzione e così via).

\section{Trusted Computing Base - TCB}
Una caratteristica dei sistemi informatici è quella di avere al loro interno componenti informatici per implementare
e gestire la loro politica di sicurezza interna.

Se uno di questi componenti ha un errore o c'è un errore nei componenti che esso utilizza, allora l'implementazione
della politica non è corretta e quasi sicuramente un soggetto potrebbe invocare operazioni per le quali non ha i
diritti necessari.

\subsection{Dimensioni del TCB}
Per quanto riguarda le dimensioni del TCB possiamo dire che, tanto minore è il numero delle sue componenti, tanto
maggiore è la sicurezza del sistema.

Un TCB con dimensioni contenute permette anche una dimostrazione matematica relativamente semplice della sua
correttezza.

La dimensione del TCB è anche un criterio che permette di confrontare strategie alternative nella realizzazione della
politica di sicurezza.

\section{Vulnerabilità}
Quando si parla di \textbf{vulnerabilità} si vuole indicare un \emph{difetto}
\begin{itemize}
	\item Hardware
	\item Software
	\item Nell'utente
	\item Nelle regole della politica
\end{itemize}
che permette di violare la politica di sicurezza del sistema permettendo ad un soggetto di compiere un'operazione per
la quale non ha diritti.

L'obbiettivo della sicurezza informatica è quello di costruire sistemi che funzionano anche con delle vulnerabilità
e non quello di costruire sistemi senza vulnerabilità. In genere le varie vulnerabilità dei vari componenti vengono
\emph{compensate} in qualche modo da altri componenti.

La vulnerabilità più frequente, spesso non è nel codice, ma nel fatto che chi implementa il sistema dia per scontato
che l'utente non faccia errori nell'interfacciarsi con esso.

\subsection{Intrusioni}
Un'\textbf{intrusione} è una sequenza di \textbf{azioni} e \textbf{attacchi} per riuscire a controllare gli oggetti
del sistema.

Non tutte le azioni che l'attaccante fa sono \emph{illegali}, alcune di esse possono essere perfettamente lecite ma
sfruttate poi per violare il sistema.

In un'intrusione si sfrutta una o più vulnerabilità, usando anche programmi automatizzati (\textbf{exploit}) per ognuna
di esse, per riuscire a sostituirsi all'\emph{owner} del sistema e dunque avere la possibilità di
\begin{itemize}
	\item Raccogliere informazioni
	\item Modificare informazioni
	\item Impedire ad altri di accedere alle informazioni
\end{itemize}

\subsubsection{Fasi di un'intrusione}
Vediamo nello specifico cosa fa un hacker quando tenta un'intrusione:
\begin{enumerate}
	\item Raccoglie di informazioni iniziali sul sistema.
	\item Individuazione delle vulnerabilità del sistema per compiere un accesso iniziale.
	\item Ripetizione di una sequenza di operazioni finché non ha successo:
	      \begin{itemize}
		      \item Raccolta di informazioni sul sistema.
		      \item Scoperta di vulnerabilità.
		      \item Costruzione di exploit.
		      \item Attacco: si eseguono gli exploit ed eventuali azioni manuali.
	      \end{itemize}
	\item Installazione di strumenti per il controllo.
	\item Cancellazione delle tracce dell'attacco.
	\item Accesso, modifica ecc. ad un \emph{sottoinsieme} delle informazioni del sistema o si compiono altri
	      tipi attacchi:
	      \begin{itemize}
		      \item Furto di informazioni.
		      \item Cifratura di dati per chiedere riscatto.
	      \end{itemize}
\end{enumerate}
Un'intrusione può essere vista come un'\textbf{escalation} nell'acquisizione di diritti e nella raccolta di
informazioni tramite vari attacchi ripetuti.

Un attaccante in genere cerca di acquisire delle informazioni che gli permettano di avere nuovi diritti. Una volta
ottenuti i diritti è in grado di acquisire nuove informazioni e così via finché non raggiunge il proprio obbiettivo.

\subsection{Approcci alla sicurezza}
Le intrusioni sono dunque possibili grazie alle vulnerabilità e questo ci porta a definire due approcci alla sicurezza:
\begin{itemize}
	\item \textbf{Sicurezza incondizionale}: si assume che qualsiasi sia la vulnerabilità nel sistema, esista qualcuno
	      interessato ad usarla ed è quindi necessario eliminarle tutte.
	\item \textbf{Sicurezza condizionale}: in questo tipo di approccio si fa un'analisi in cui si cerca di capire quali
	      siano le reali minacce per il sistema e si eliminano solo le vulnerabilità che tali minacce potrebbero usare
	      per attaccare il sistema.
\end{itemize}

\subsubsection{Analisi del rischio}
Il primo approccio comporta costi molto elevati e spesso inaccettabili, inoltre richiede una quantità di lavoro enorme
e spesso inutile.

Con il secondo approccio invece si cerca di capire quali componenti del sistema si possono difendere e soprattutto
quali componenti \emph{conviene} difendere.

Per capirlo è necessaria un'\textbf{analisi del rischio} con la quale si cerca di individuare la tipologia di attacco
più probabile in relazione al sistema che stiamo cercando di proteggere.

\chapter{Aritmetica binaria e reti logiche}
Passiamo ora ad un breve richiamo sull'aritmetica binaria e sulla logica booleana per capire meglio
cosa verrà utilizzato dal calcolatore per svolgere i calcoli.

\section{Aritmetica binaria}
Il sistema di numerazione binario, così come quelli decimale ed esadecimale, è detto
\textbf{posizionale}. Si assegna cioè un \textbf{peso} alle possibili posizioni in cui può cadere
una delle cifre a cui poi verrà moltiplicata la cifra effettiva.

Il numero $123_{10}$ si ottiene dalla somma delle sue singole cifre moltiplicate per il peso della
posizione in cui si trovano. Nello specifico, se la base di numerazione è $b$, partendo dalla
posizione più a destra, il peso della prima posizione sarà $b^0$ mentre il peso dell'$n$-esima
posizione sarà $b^{n-1}$.
\[ 123_{10} = 1 \cdot 10^2 + 2 \cdot 10^1 + 3 \cdot 10^0 = 100 + 20 + 3 \]
Se ad esempio volessimo fare lo stesso per il numero $1010_2$ otterremmo
\[ 1010_2 = 1 \cdot 2^3 + 0 \cdot 2^2 + 1 \cdot 2^1 + 0 \cdot 2^0 = 8 + 2 = 10_{10} \]
Come possiamo vedere abbiamo anche trovato un modo semplice per la conversione da binario a
decimale. In genere per la rappresentazione in macchina viene usata la notazione
\textbf{esadecimale} poiché, avendo 16 simboli a disposizione, fornisce una rappresentazione più
compatta di numeri binari altrimenti molto lunghi e difficilmente leggibili.
\[ 1F0_{16} = 1 \cdot 16^2 + 15 \cdot 16^1 + 0 \cdot 16^0 = 256 + 240 = 496_{10} \]
L'operazioni più semplice che possiamo fare tra numeri binari è ovviamente la \textbf{somma} che
funziona in modo analogo
\begin{center}
	\begin{tabular}{c | c c}
		$2_{10}$ & $0010$ & + \\
		$3_{10}$ & $0011$ & = \\ \hline
		$5_{10}$ & $0101$
	\end{tabular}
\end{center}

Per la \textbf{conversione} da decimale a binario possiamo
\begin{enumerate}
	\item Prendere la più grande potenza di 2 minore o uguale del numero che vogliamo convertire.
	\item Sottraiamo al numero la potenza di 2 trovata.
	\item Ripetiamo il procedimento con la differenza ottenuta finché non si ottiene 0.
	\item Quando si ottiene 0 si va a mettere un 1 in ognuna delle posizioni relative alle potenze
	      trovate.
\end{enumerate}
Se ad esempio volessimo rappresentare $17_{10}$ troveremmo che $2^4 = 16$ è la più grande potenza
di 2 minore o uguale di 17. Quindi eseguiamo $17 - 16 = 1$ a questo punto abbiamo che $2^0 = 1$ è
la più grande potenza minore o uguale di 1. Eseguiamo $1-1 = 0$ e concludiamo. Il risultato sarà
\[ 17_{10} = 10001_2 \]
dove gli 1 si trovano rispettivamente in posizione 1 e 5 ossia le posizioni di peso $2^0$ e $2^4$.

\section{Rappresentazione in macchina}
Se volessimo rappresentare numeri in macchina avendo a disposizione registri di $n$ bit potremmo
rappresentare $2^n$ possibili numeri (in particolare in numeri da 0 a $2^n - 1$).

Si ha però un problema nel momento in cui si vogliono rappresentare numeri negativi. Supponiamo di
voler rappresentare i numeri da $-x$ a $+x$, il primo metodo di rappresentazione dei numeri
negativi è il cosiddetto \textbf{modulo e segno}.

Questo metodo utilizza un bit per il segno e i restanti vengono usati per la rappresentazione come
abbiamo visto fin ad ora.

Per quanto riguarda i numeri positivi il bit di segno sarà a 0 e per i bit negativi sarà a 1. La
restante rappresentazione del numero rimane invariata. Notiamo che adesso, con registri da $n$ bit
ne possiamo usare $n-1$ per la rappresentazione e dunque avremo a disposizione i numeri da
$-2^{n-1}$ a $+2^{n-1}$.

Questa rappresentazione si porta dietro due problemi principali: il primo è che rappresenta due
volte lo 0 (-0 e +0) e poi introduce la necessità di una componente in grado di fare le sottrazioni
(necessaria quando si sottrae un numero più grande ad un numero più piccolo).

\subsection{Rappresentazione in complemento a 2}
Si è quindi passati alla rappresentazione in \textbf{complemento a 2}, la quale prevede che, per i
numeri positivi si utilizzino i primi $n-1$ bit per la rappresentazione e l'$n$-esimo bit messo a 0.
Per i numeri negativi si deve invece rappresentare prima di tutto il modulo del numero, dopodiché
si negano tutti i bit e infine si somma 1.

Se ad esempio volessimo rappresentare $-3$ in complemento a 2 su 8 bit il risultato sarebbe il
seguente:
\begin{enumerate}
	\item Rappresentazione 3 in base 2: $00000011_2$.
	\item Negazione di tutti i bit $00000011_2 \to 11111100_2$.
	\item Sommiamo 1 al risultato $11111100 + 1 = 11111101$.
\end{enumerate}
Come possiamo vedere otteniamo una rappresentazione in cui il numero più a sinistra è 1, e quindi
capiamo che il numero è negativo. Se volessimo sapere di quale numero si tratta ci basta fare il
procedimento inverso negando tutti i bit
\[ 11111101 \to 00000010 \]
e sommandoci 1
\begin{center}
	\begin{tabular}{r c}
		00000010 & + \\
		1        & = \\ \hline
		00000011
	\end{tabular}
\end{center}
che infatti è 3. Questo metodo ci permette di effettuare sottrazioni semplicemente facendo somme
con numero negativi. Se per esempio volessimo effettuare $2 - 3$ otterremmo
\begin{center}
	\begin{tabular}{r c}
		00000010 & + \\
		11111101 & = \\ \hline
		11111111
	\end{tabular}
\end{center}
Che se convertiamo in modulo diventa $11111111 \to 00000000 + 1 = 00000001$ ovvero 1 e quindi
deduciamo facilmente che il risultato della somma appena fatta è $-1$.

Altro fatto importante è che l'unica rappresentazione dello 0 è quella con tutti i bit messi a 0,
mentre la rappresentazione
\[ 10000000 \]
non è valida o comunque non viene mai generata da possibili calcoli in aritmetica binaria.

\subsection{Shift dei bit}
Quando si parla di moltiplicazioni e divisioni, un caso particolare è quello in cui si moltiplica
o si divide un numero per la base di numerazione o per una sua potenza. Se ad esempio volessimo
svolgere $123 \times 10$ è immediato riconoscere che il risultato è $1230$ poiché come sappiamo
basta aggiungere uno zero a destra.

Questo però implica uno \textbf{shift} delle cifre dalla lora posizione ad una posizione a peso
maggiore (a sinistra). Allo stesso modo, vale $123 / 10 = 12.3$, ottenuto tramite un shift a destra
delle cifre.

In modo del tutto analogo questo è anche possibile in aritmetica binaria ma invece di moltiplicare
o dividere per 10, moltiplichiamo o dividiamo per 2. Per esempio se dividiamo $10_{10} = 1010_2$
per 2 abbiamo, in decimale che il risultato è 5 e in binario, dato che dividiamo per la base ci
basterà shiftare tutti i bit a destra di una posizione ottenendo
\[ 0101 = 2^2 + 2^0 = 4 + 1 = 5_{10} \]
Se invece volessimo moltiplicare o dividere per una potenza di 2 dovremmo effettuare tanti shift
(a destra o a sinistra) quanto vale l'esponente. Per esempio $10 \times 2^2$ equivale ad effettuare
un doppio shift verso sinistra del numero $1010$, ottenendo così
\[ 101000 = 2^5 + 2^3 = 32 + 8 = 40_{10} \]
Questo ci sarà molto utile nell'indirizzamento della memoria che vedremo più avanti in quanto la
memoria del computer è partizionata in un numero di blocchi pari ad una potenza di due. I blocchi
sono a loro volta suddivisi in un numero di pagine sempre pari ad una potenza di 2. Questi ci
permette di muoverci tra i blocchi o le pagine della memoria in modo molto più agevole e veloce
tramite le operazioni di shift.

Per quanto riguarda i numeri negativi rappresentati in complemento a 2 dobbiamo avere una piccola
accortezza. Prendiamo per esempio l'operazione
\[ -8 / 2^2 = -2 \]
che in binario equivale a shiftare a destra di due posizioni il numero $11111000$ (-8 in
complemento a 2). Dato che però, tramite un normale shift a destra otterremmo $00111110$, cioè un
numero positivo, è chiaro che non è il risultato corretto. Semplicemente negli shift a destra di
numeri negativi le posizioni a sinistra vengono rimpiazzate da 1 invece che da 0, otteniamo quindi
$11111110$ che, se effettuiamo la conversione in complemento a 2, equivale a $00000010_2 = 2_{10}$.

Per quanto riguarda invece gli shift a sinistra di numeri negativi in complemento a 2 si effettua
un normale shift aggiungendo tanti 0 a destra quanto vale l'esponente della potenza per cui si
vuole moltiplicare. Per esempio
\[ -8 \times 2^2 = -32 \]
per ottenere il calcolo in binario dobbiamo shiftare a sinistra il numero $11111000$. Otteniamo
quindi $11100000$ che se convertito in modulo diventa $00100000_2 = 2^5 = 32_{10}$

\subsection{Rappresentazione in virgola mobile}
La rappresentazione in virgola mobile ha subito diverse variazioni negli anni fino ad arrivare ad
uno standard che ad oggi prende il nome di IEEE 754, il quale prevede l'utilizzo di
\begin{itemize}
	\item Un bit per il \textbf{segno}.
	\item Un certo numero di bit per l'\textbf{esponente}.
	\item I restanti bit per la \textbf{mantissa}.
\end{itemize}
Il numero di bit per esponente e mantissa dipende dalla lunghezza dei registri che utilizziamo (se
a 32 o a 64 bit).
\begin{itemize}
	\item Se il bit del segno è 0 il numero è positivo, se invece è 1 allora il numero è negativo.
	\item L'esponente indica il numero a cui elevare la base.
	\item La mantissa rappresenta il numero con una virgola in posizione \emph{"fissa"}.
\end{itemize}
Il numero $x$ cercato è rappresentato come
\[ x = \text{sign}(x) \cdot b^e \cdot m \]
L'esponente è rappresentato in \textbf{eccesso a k}, ovvero se l'esponente è $e$, il calcolatore
rappresenta $e+k$ in modo che numeri più grandi di $k$ rappresentano numeri positivi mentre numeri
più piccoli di $k$ rappresentano numeri negativi. Per lo standard IEEE 754, nel caso di numeri in
virgola mobile a
\begin{itemize}
	\item 32 bit, l'esponente è rappresentato con 8 bit in eccesso a 127 e la mantissa ha a
	      disposizione 23 bit.
	\item 64 bit, l'esponente è rappresentato con 11 bit in eccesso a 1023 e la mantissa ha a
	      disposizione 52 bit.
\end{itemize}
Aggiungiamo inoltre che lo standard vuole che la prima cifra per una rappresentazione in virgola
mobile sia non nulla e tale che alla sua sinistra vi siano solo cifre non significative
\[ 123.45 = 1.2345 \times 10^2 \]
Questo implica che le operazioni tra numeri rappresentati in virgola mobile non sono così immediate
nel caso in cui questi abbiano ordini di grandezza diversi.

Quel che viene fatto è passarli ad un \textbf{incolonnatore} che, tramite alcune operazioni riesce
a spostare la virgola e cambiare l'esponente dei due operandi in modo che il calcolo sia eseguibile
nel modo classico. Una volta effettuato il calcolo, il risultato passa da un \textbf{normalizzatore}
che riporta il risultato ad una rappresentazione che soddisfa lo standard.

Se vogliamo implementare la possibilità di eseguire operazioni di addizione tra numeri interi e tra
numeri in virgola mobile, abbiamo bisogno di una componente detta \textbf{ALU}. Per le operazioni
tra interi si parla di \textbf{ALU-I}, mentre per le operazioni tra numeri in virgola mobile si
parla di \textbf{ALU-FP}.

\subsection{Rappresentazione di caratteri}
In ultima battuta parliamo della rappresentazione di documenti testuali, i quali non sono altro
che \textbf{sequenze di caratteri}. Tali caratteri seguono la rappresentazione \textbf{ASCII} che
fa utilizzo di 8 o 16 bit.

Per sapere cosa rappresenta la sequenza di bit si fa utilizzo di una \textbf{tabella ASCII} che
mette in corrispondenza il numero rappresentato dalla sequenza di bit ad un carattere nella tabella.

In questo ambito è possibile effettuare piccole operazioni come \textbf{ordinamenti lessicografici}
poiché, per esempio, le lettere da "A" a "Z" hanno un codice che le mette in ordine una dopo
l'altra.
\section{Circuiti logici}
Nel corso non andremo a trattare l'ultimo livello di astrazione, ossia quello più basso, ma andremo
a trattare lo strato soprastante, che tramite delle \textbf{porte logiche} e delle operazioni
aritmetiche binarie riesce a rappresentare quello che succede. Le tre operazioni implmentate dalle
porte logiche sono
\begin{itemize}
	\item \verb|AND(x,y)|: 1 se \verb|x=y=1|, 0 altrimenti.
	\item \verb|OR(x,y)|: 0 se \verb|x=y=0|, 1 altrimenti.
	\item \verb|NOT(x)|: 1 se \verb|x=0|, 0 se \verb|x=1|.
\end{itemize}
Altro strumento utile per capire meglio come funzionano tali porte e per vedere come funzionano
altre porte che risultano essere una combinazione di esse, sono le \textbf{tabelle di verità}.
Nelle tabelle di verità immettiamo tutti i possibili valori di input e calcoliamo i relativi output.
Per esempio, la tabella di verità di una porta logica \verb|AND| è la seguente
\begin{center}
	\begin{tabular}{c c | c}
		x & y & z \\ \hline
		0 & 0 & 0 \\
		0 & 1 & 0 \\
		1 & 0 & 0 \\
		1 & 1 & 1
	\end{tabular}
\end{center}
Per una questione legata alla circuiteria sottostante e alla leggi fisiche che regolano il
funzionamento dei transistor, il numero di ingressi delle porte è, in genere, al più 8 poiché
averne di più introduce troppo ritardo nell'elaborazione dei segnali.

Quello che useremo d'ora in poi saranno delle funzioni che hanno un certo numero di ingressi e
uscite booleani che realizzeremo come \textbf{reti combinatorie}, ossia composizioni di porte
\verb|AND|, \verb|OR| e \verb|NOT| a seconda delle necessità.

Supponiamo ad esempio di voler calcolare il numero di bit a 1 su 2 ingressi, ciascuno da 1 bit. In
questo caso i possibili valori di output sono 3 (0, 1 e 2) e abbiamo quindi bisogno di un numero
di uscite pari a $\lceil \log_2 (3) \rceil = 2$ uscite. La tabella di verità del nostro circuito
avrà la seguente tabella di verità
\begin{center}
	\begin{tabular}{c c | c c}
		$x_0$ & $x_1$ & $z_0$ & $z_1$ \\ \hline
		0     & 0     & 0     & 0     \\
		0     & 1     & 0     & 1     \\
		1     & 0     & 0     & 1     \\
		1     & 1     & 1     & 0
	\end{tabular}
\end{center}
Per trovare il circuito desiderato c'è una procedura standard, la quale utilizza il fatto che un
\verb|AND| logico corrisponde al prodotto tra due numeri mentre l'\verb|OR| logico corrisponde alla
somma:
\begin{enumerate}
	\item Per ogni riga in cui una delle colonne d'uscita presenta almeno un 1 mettiamo in
	      \verb|AND| gli ingressi, negandoli se uguali a 0.
	\item Per ogni colonna si mettono in \verb|OR| tutti i risultati ottenuti al passo precedente.
\end{enumerate}
Nel nostro caso la colonna $z_0$ ha un 1 sull'ultima riga e i relativi valori di $x_0$ e $x_1$ sono
entrambi 1 quindi possiamo dire che
\[ z_0 = x_0 \cdot x_1 \]
ossia
\begin{center}
	\verb|z0 = AND(x0, x1)|
\end{center}
Per quanto riguarda invece la colonna $z_1$ abbiamo due 1 e in corrispondenza della seconda e terza
riga. Ma in entrambi i casi uno dei due valori in ingresso è 0 e l'altro è 1 e dunque il risultato
finale è
\[ z_1 = \bar{x_0} \cdot x_1 + x_0 \cdot \bar{x_1} \]
ossia
\begin{center}
	\verb|z1 = OR(AND(NOT(x0), x1), AND(x0, NOT(x1)))|
\end{center}
Il circuito logico che ne deriva è il seguente

% \begin{tikzpicture}[label distance=2mm]

% 	\node (x3) at (0,0) {$x_3$};
% 	\node (x2) at (1,0) {$x_2$};
% 	\node (x1) at (2,0) {$x_1$};
% 	\node (x0) at (3,0) {$x_0$};

% 	\node[not gate US, draw, rotate=-90] at ($(x2)+(0.5,-1)$) (Not2) {};
% 	\node[not gate US, draw, rotate=-90] at ($(x1)+(0.5,-1)$) (Not1) {};
% 	\node[not gate US, draw, rotate=-90] at ($(x0)+(0.5,-1)$) (Not0) {};

% 	\node[or gate US, draw, logic gate inputs=nnn] at ($(x0)+(2,-2)$) (Or1) {};
% 	\node[or gate US, draw, logic gate inputs=nnnn] at ($(Or1)+(0,-1)$) (Or2) {};
% 	\node[or gate US, draw, logic gate inputs=nnn] at ($(Or2)+(0,-1)$) (Or3) {};
% 	\node[xor gate US, draw, logic gate inputs=nn] at ($(Or3)+(0,-1)$) (Xor1) {};
% 	\node[and gate US, draw, logic gate inputs=nn, anchor=input 1] at ($(Or3.output)+(1,0)$) (And1) {};
% 	\node[nor gate US, draw, logic gate inputs=nn, anchor=input 1] at ($(Or2.output -| And1.output)+(1,0)$) (Nor1) {};
% 	\node[and gate US, draw, logic gate inputs=nn, anchor=input 1] at ($(Or1.output -| Nor1.output)+(1,0)$) (And2) {};

% 	\foreach \i in {2,1,0}
% 		{
% 			\path (x\i) -- coordinate (punt\i) (x\i |- Not\i.input);
% 			\draw (punt\i) node[branch] {} -| (Not\i.input);
% 		}
% 	\draw (x3) |- (Or2.input 1);
% 	\draw (x3 |- Or1.input 1) node[branch] {} -- (Or1.input 1);
% 	\draw (x2) |- (Xor1.input 1);
% 	\draw (x2 |- Or3.input 1) node[branch] {} -- (Or3.input 1);
% 	\draw (Not2.output) |- (Or2.input 2);
% 	\draw (x1) |- (Or3.input 2);
% 	\draw (x1 |- Or1.input 2) node[branch] {} -- (Or1.input 2);
% 	\draw (Not1.output) |- (Xor1.input 2);
% 	\draw (Not1.output |- Or2.input 3) node[branch] {} -- (Or2.input 3);
% 	\draw (x0) |- (Or2.input 4);
% 	\draw (Not0.output) |- (Or3.input 3);
% 	\draw (Not0.output |- Or1.input 3) node[branch] {} -- (Or1.input 3);
% 	\draw (Or3.output) -- (And1.input 1);
% 	\draw (Xor1.output) -- ([xshift=0.5cm]Xor1.output) |- (And1.input 2);
% 	\draw (Or2.output) -- (Nor1.input 1);
% 	\draw (And1.output) -- ([xshift=0.5cm]And1.output) |- (Nor1.input 2);
% 	\draw (Or1.output) -- (And2.input 1);
% 	\draw (Nor1.output) -- ([xshift=0.5cm]Nor1.output) |- (And2.input 2);
% 	\draw (And2.output) -- ([xshift=0.5cm]And2.output) node[above] {$f_1$};

% \end{tikzpicture}


\begin{center}
	\begin{circuitikz}
		% gate
		\node[and port] (and1) at (3.5, -1) {};
		\node[and port] (and2) at (3.5, -2.5) {};
		\node[and port] (and3) at (3.5, -4) {};
		\node[or port] (or) at (5.5, -3.25) {};

		% connessioni
		\draw (0, 0) node[label=above:$x_0$] {} to[short, -*] (0, 52 |- and1.in 1) -- (and1.in 1);
		\draw (0, 52 |- and1.in 1) to[short, -*] (0, 52 |- and2.in 1) to[short, -o] (and2.in 1);
		\draw (0, 52 |- and2.in 1) to[short, -*] (0, 52 |- and3.in 1) -- (and3.in 1);

		\draw (1, 0) node[label=above:$x_1$] {} to[short, -*] (1, 52 |- and1.in 2) -- (and1.in 2);
		\draw (1, 52 |- and1.in 2) to[short, -*] (1, 52 |- and2.in 2) -- (and2.in 2);
		\draw (1, 52 |- and2.in 2) to[short, -*] (1, 52 |- and3.in 2) to[short, -o] (and3.in 2);

		\draw (and2.out) |- (or.in 1);
		\draw (and3.out) |- (or.in 2);

		\draw (and1.out) -- (6.5, 52 |- and1.out) node[label=above:$z_0$] {};
		\draw (or.out) -- (6.5, 52 |- or.out) node[label=above:$z_1$] {};
	\end{circuitikz}
\end{center}
sul quale è possibile provare ad inserire vari input di $x_0$ e $x_1$ per verificarne la
correttezza.

Supponiamo ora di dover scegliere uno tra due ingressi possibili a seconda di un ingresso di
controllo regolato da un \textbf{multiplexer} che ha una forma di questo tipo
\begin{center}
	\begin{circuitikz}
		\draw[thick] (0, -1) -- (0, 1) -- (1, 0.5) -- (1, -0.5) -- cycle;
		\draw (-1, 0.5) node[label=left:$x_0$] {} -- (0, 0.5);
		\draw (-1, -0.5) node[label=left:$x_1$] {} -- (0, -0.5);
		\draw (0.5, 1.5) node[label=above:$c$] {} -- (0.5, 0.75);
		\draw (1, 0) -- (2, 0) node[label=right:$z$] {};
	\end{circuitikz}
\end{center}
Di fatto dobbiamo implementare un circuito che da come risultato il valore di $x_0$ quando $c=0$ e
da come risultato il valore di $x_1$ quando $c=1$.

In questo caso abbiamo tre ingressi e un'uscita, dovremmo quindi scrivere una tabella di verità con
8 righe, ma dato che uno dei valori viene scartato a seconda del valore di $c$ il risultato è una
tabella più compatta.

Avere una tabella più compatta significa anche avere un circuito più compatto e con meno componenti.
Questo si traduce in un minor numero di nodi di calcolo e quindi una computazione più veloce, ma
anche in un minor consumo di energia e minor bisogno di spazio sul processore.
\begin{center}
	\begin{tabular}{c c c | c}
		$x_0$ & $x_1$ & $c$ & $z$ \\ \hline
		0     & -     & 0   & 0   \\
		1     & -     & 0   & 1   \\
		-     & 0     & 1   & 0   \\
		-     & 1     & 1   & 1
	\end{tabular}
\end{center}
Svolgiamo lo stesso procedimento di prima e ricaviamo un circuito di questo tipo
\begin{center}
	\begin{circuitikz}
		\node[and port] (and1) at (3.5, 0.75) {};
		\node[and port] (and2) at (3.5, -0.75) {};
		\node[or port] (or) at (5.5, 0) {};

		% connessioni
		\draw (0, 1.5) node[label=above:$x_0$] {} to[short, -*] (0, 52 |- and1.in 1) -- (and1.in 1);

		\draw (0.5, 1.5) node[label=above:$x_1$] {} to[short, -*] (0.5, 52 |- and2.in 1) -- (and2.in 1);

		\draw (1, 1.5) node[label=above:$c$] {} to[short, -*] (1, 52 |- and1.in 2) to[short, -o] (and1.in 2);
		\draw (1, 52 |- and1.in 2) to[short, -*] (1, 52 |- and2.in 2) -- (and2.in 2);

		\draw (and1.out) |- (or.in 1);
		\draw (and2.out) |- (or.in 2);

		\draw (or.out) -- (6.5, 52 |- or.out) node[label=above:$z$] {};
	\end{circuitikz}
\end{center}
che calcola esattamente
\[ z = x_0 \cdot \bar{c} + x_1 \cdot c \]
ossia il valore del canale scelto dal multiplexer.
\section{Algebra di Boole}
Per introdurre l'\textbf{algebra di Boole} introduciamo i seguenti \textbf{assiomi} che determinano
il comportamento dell'alfabeto $\{0, 1\}$ in relazione a delle operazioni di base che possiamo
fare con i suoi elementi.
\begin{gather*}
	a = 0 \implies a \neq 1 \quad \land \quad a = 1 \implies a \neq 0 \\
	a = 0 \implies \bar{a} = 1 \quad \land \quad a = 1 \implies \bar{a} = 0 \\
	0 \cdot 1 = 0 \quad \land \quad 1 \cdot 0 = 0 \\
	0 \cdot 0 = 0 \quad \land \quad 1 \cdot 1 = 1 \\
	0 + 1 = 1 \quad \land \quad 1 + 0 = 1 \\
	0 + 0 = 0 \quad \land \quad 1 + 1 = 1
\end{gather*}
Da questi deduciamo anche che
\begin{gather*}
	A \cdot 1 = A \quad \land \quad A + 0 = A \\
	A \cdot 0 = 0 \quad \land \quad A + 1 = 1 \\
	A \cdot A = A \quad \land \quad A + A = A \\
	A \cdot \bar{A} = 0 \quad \land \quad A + \bar{A} = 1 \\
	\bar{\bar{A}} = A
\end{gather*}
Le altre proprietà fondamentali per le operazioni di \verb|AND| e \verb|OR| nell'algebra booleana
sono
\begin{itemize}
	\item \textbf{Commutatività} per l'\verb|AND|: $A \cdot B = B \cdot A$
	\item \textbf{Commutatività} per l'\verb|OR|: $A + B = B + A$
	\item \textbf{Distributività}: $A \cdot (B + C) = A \cdot B + A \cdot C$ e la formula duale
	      $A + (B \cdot C) = (A \cdot B) + (A \cdot C)$
	\item \textbf{De Morgan}: $\overline{A \cdot B} = \bar{A} + \bar{B}$ e la formula duale
	      $\overline{A + B} = \bar{A} \cdot \bar{B}$
\end{itemize}
Con queste proprietà è possibile semplificare alcune delle formule generate da alcune tabelle di
verità come abbiamo fatto nel caso del multiplexer. Supponiamo che per un qualche motivo otteniamo
una funzione di $a$, $b$ e $c$ tale che
\[ f(a,b,c) = \bar{a} \bar{b} \bar{c} + a \bar{b} \bar{c} + a \bar{b} c \]
Se volessimo implementare questa formula tramite un circuito avremmo bisogno di tre porte
\verb|AND3| e di 1 porta \verb|OR3|. Usando le proprietà possiamo ottenere
\[
	\bar{a} \bar{b} \bar{c} + a \bar{b} \bar{c} + a \bar{b} c
	= \bar{b} \bar{c} (\bar{a} + a) + a \bar{b} c
	= \bar{b} \bar{c} + a \bar{b} c
\]
Passando così ad una formula che ci permette di implementare un circuito tramite due porte
\verb|AND3| e una porta \verb|OR3|. Proviamo un altro modo di procedere
\begin{align*}
	\bar{a} \bar{b} \bar{c} + a \bar{b} \bar{c} + a \bar{b} c
	 & = \bar{a} \bar{b} \bar{c} + a \bar{b} \bar{c} + a \bar{b} \bar{c} + a \bar{b} c         \\
	 & = \bar{b} \bar{c} (\bar{a} + a) + a \bar{b} (c + \bar{c}) = \bar{b} \bar{c} + a \bar{b}
\end{align*}
ottenendo così la possibilità di implementare un circuito tramite due porte \verb|AND2| e una porta
\verb|OR2|. Come possiamo vedere, a seconda di come usiamo queste proprietà, è possibile diminuire
notevolmente la dimensione dei circuiti e quindi la complessità di ciò che stiamo calcolando.
\begin{center}
	\begin{circuitikz}
		% gates
		\node[and port] (and1) at (3.5, 1) {};
		\node[and port] (and2) at (3.5, -1) {};
		\node[or port] (or) at (5.5, 0) {};

		\draw (0, 2) node[label=above:$a$] {} to[short, -*] (0, 52 |- and2.in 1) -- (and2.in 1);
		\draw (0.5, 2) node[label=above:$b$] {} to[short, -*] (0.5, 52 |- and1.in 1) to[short, -o] (and1.in 1);
		\draw (0.5, 52 |- and1.in 1) to[short, -*] (0.5, 52 |- and2.in 2) to[short, -o] (and2.in 2);
		\draw (1, 2) node[label=above:$c$] {} to[short, -*] (1, 52 |- and1.in 2) to[short, -o] (and1.in 2);

		\draw (and1.out) |- (or.in 1);
		\draw (and2.out) |- (or.in 2);
	\end{circuitikz}
\end{center}
A questo punto sarebbe possibile semplificare ulteriormente la formula raccogliendo $\bar{b}$ e
implementando il circuito descritto da
\[ \bar{b} \cdot (\bar{c} + a) \]
ma questo introduce un problema in quanto il circuito generato è asimmetrico, ossia i segnali in
ingresso non attraversano tutti lo stesso numero di porte come possiamo vedere in figura
\begin{center}
	\begin{circuitikz}
		\node[or port] (or) at (3.5, 1) {};
		\node[and port] (and) at (5.5, 0) {};

		\draw (0, 2) node[label=above:$a$] {} to[short, -*] (0, 52 |- or.in 1) -- (or.in 1);
		\draw (0.5, 2) node[label=above:$b$] {} to[short, -*] (0.5, 52 |- and.in 2) to[short, -o] (and.in 2);
		\draw (1, 2) node[label=above:$c$] {} to[short, -*] (1, 52 |- or.in 2) to[short, -o] (or.in 2);

		\draw (or.out) |- (and.in 1);
	\end{circuitikz}
\end{center}
Questo si traduce in un intervallo di tempo in cui la porta \verb|AND| riceve, da una parte il
vecchio segnale trasmesso dalla porta \verb|OR| prodotto al calcolo precedente, dall'altra l'ultimo
segnale prodotto dall'ingresso $b$.

Fino a che la porta \verb|OR| non finisce di elaborare i segnali in arrivo da $a$ e $c$ la porta
\verb|AND| potrebbe produrre risultati errati, dovuti a quello che viene chiamato \textbf{glitch}.

\subsection{Mappe di Karnaugh}
Come abbiamo appena visto, non sempre ridurre la complessità della nostra formula in modo
\emph{monotòno} ci porta alla migliore ottimizzazione. A volte conviene aumentare la complessità
per poi giungere ad un modello migliore.

Le \textbf{mappe di Karnaugh} forniscono un metodo grafico per riuscire a semplificare le formule
booleane senza però garantire la miglior minimizzazione di quest'ultime. Nell'esempio di prima
abbiamo una funzione booleana con la seguente tabella di verità
\begin{center}
	\begin{tabular}{c c c | c}
		$a$ & $b$ & $c$ & $f(a,b,c)$ \\ \hline
		0   & 0   & 0   & 1          \\
		0   & 0   & 1   & 0          \\
		0   & 1   & 0   & 0          \\
		0   & 1   & 1   & 0          \\
		1   & 0   & 0   & 1          \\
		1   & 0   & 1   & 1          \\
		1   & 1   & 0   & 0          \\
		1   & 1   & 1   & 0
	\end{tabular}
\end{center}
Da questa tabella possiamo ricavare una mappa di Karnaugh prendendo tutti i possibili valori di $a$
e mettendoli nella prima colonna e poi prendendo tutti i possibili valori della coppia $bc$ e
mettendoli sulla prima riga, disponendoli in modo che ogni valore differisca dal precedente al più
di un bit.

Il nostro obbiettivo è quello di individuare i quadrati o rettangoli contenenti un numero di 1 pari
ad una potenza di 2 e raggrupparli. Per tale raggruppamento è possibile
\begin{itemize}
	\item Uscire dalla tabella e rientrare dall'altra parte se ho degli 1 agli estremi.
	\item Includere degli 1 già raccolti in un precedente raggruppamento.
\end{itemize}
Nel nostro caso abbiamo due rettangoli da due 1: il primo verticale che prende la prima colonna per
intero e il secondo orizzontale che prende la prima metà della seconda riga.
\begin{center}
\begin{karnaugh-map}[4][2][1][$c$][$b$][$a$]
\maxterms{1, 2, 3, 6, 7}
\minterms{0, 4, 5}
\implicant{0}{4}
\implicant{4}{5}
\end{karnaugh-map}
\end{center}
A questo punto siamo
in grado di semplificare la formula di partenza
\begin{enumerate}
	\item Mettendo in \verb|AND| le variabili facenti parte dello stesso raggruppamento che
	      rimangono costanti e negando quelle con valore 0.
	\item Sommando tra di loro i raggruppamenti.
\end{enumerate}
Otteniamo così la formula ottenuta in precedenza con le proprietà dell'algebra booleana
\[ \bar{b} \bar{c} + a \bar{b} \]
in modo meccanico. Il primo termine della somma è ottenuto prendendo in considerazione il
raggruppamento verticale di 1 e considerando che $b$ e $c$ non variano ed essendo a 0 vengono
negati. Il secondo termini si ottiene similmente notando che $a$ e $b$ sono la parte costante del
raggruppamento ed inoltre $b$ è a 0 e dunque deve essere negato.

Prendiamo ora come esempio un \textbf{sommatore} di 2 bit con riporto, il cui funzionamento dipende
da tre parametri di ingresso: $x_1$ e $x_2$ i bit che vogliamo sommare e $r_0$ il possibile riporto
da aggiungere. Abbiamo inoltre due uscite: il risultato $s$ della somma e il possibile riporto $r_1$
generato da essa.
\begin{center}
	\begin{tikzpicture}
		\draw[thick] (0, 0) rectangle (2, 1.5);
		\node (add) at (1, 0.75) {ADD};

		\draw (0.5, 2) node[label=above:$x_1$] {} to[short, o-] (0.5, 1.5);
		\draw (1.5, 2) node[label=above:$x_2$] {} to[short, o-] (1.5, 1.5);
		\draw (2.5, 0.75) node[label=right:$r_0$] {} to[short, o-] (2, 0.75);
		\draw (0, 0.75) -- (-0.5, 0.75) -- (-0.5, -0.75) node[label=below:$r_1$] {};
		\draw (1, 0) -- (1, -0.75) node[label=below:$s$] {};
	\end{tikzpicture}
\end{center}
In questo caso la tabella di verità di tale oggetto è
\begin{center}
	\begin{tabular}{c c c | c | c }
		$x_1$ & $x_2$ & $r_0$ & $s$ & $r_1$ \\ \hline
		0     & 0     & 0     & 0   & 0     \\
		0     & 0     & 1     & 1   & 0     \\
		0     & 1     & 0     & 1   & 0     \\
		0     & 1     & 1     & 0   & 1     \\
		1     & 0     & 0     & 1   & 0     \\
		1     & 0     & 1     & 0   & 1     \\
		1     & 1     & 0     & 0   & 1     \\
		1     & 1     & 1     & 1   & 1
	\end{tabular}
\end{center}
Le mappe di Karnaugh per $s$ ed $r_1$ risultano le seguenti
\begin{center}
\begin{figure}[h!] \centering
\begin{subfigure}[b]{0.4\textwidth}
\begin{karnaugh-map}[4][2][1][$c$][$b$][$a$]
\minterms{1,2,4,7}
\maxterms{0,3,5,6}
\implicant{1}{1}
\implicant{2}{2}
\implicant{4}{4}
\implicant{7}{7}
\end{karnaugh-map}
\end{subfigure}
\begin{subfigure}[b]{0.4\textwidth}
\begin{karnaugh-map}[4][2][1][$c$][$b$][$a$]
\minterms{3,5,7,6}
\maxterms{0,1,2,4}
\implicant{3}{7}
\implicant{7}{6}
\implicant{5}{6}
\end{karnaugh-map}
\end{subfigure}
\end{figure}
\end{center}
Da tali mappe di Karnaugh ricaviamo le seguenti formule per $s$ ed $r_1$
\begin{align*}
	s   & = r_0 \bar{x_1} \bar{x_2} + \bar{r_0} \bar{x_1} x_2 + r_0 x_1 x_2 + \bar{r_0} x_1 \bar{x_2} \\
	r_1 & = x_1 x_2 + r_0 x_2 + r_0 x_1
\end{align*}
Di seguito raffiguriamo il circuito ricavato dalla formula per $r_1$.
\begin{center}
	\begin{circuitikz}
		\node[and port] (and1) at (3.5, 1.5) {};
		\node[and port] (and2) at (3.5, 0) {};
		\node[and port] (and3) at (3.5, -1.5) {};
		\node[or port, number inputs=3] (or) at (5.5, 0) {};

		\draw (0, 2) node[label=above:$x_1$] {} to[short, -*] (0, 52 |- and1.in 1) -- (and1.in 1);
		\draw (0, 52 |- and1.in 1) to[short, -*] (0, 52 |- and3.in 2) -- (and3.in 2);

		\draw (0.5, 2) node[label=above:$x_2$] {} to[short, -*] (0.5, 52 |- and1.in 2) -- (and1.in 2);
		\draw (0.5, 52 |- and1.in 2) to[short, -*] (0.5, 52 |- and2.in 2) -- (and2.in 2);

		\draw (1, 2) node[label=above:$r_0$] {} to[short, -*] (1, 52 |- and2.in 1) -- (and2.in 1);
		\draw (1, 52 |- and2.in 1) to[short, -*] (1, 52 |- and3.in 1) -- (and3.in 1);

		\draw (and1.out) -- (or.in 1);
		\draw (and2.out) -- (or.in 2);
		\draw (and3.out) -- (or.in 3);
		\draw (or.out) --++ (0.5, 0) node[label=right:$r_1$] {};
	\end{circuitikz}
\end{center}
Per riassumere possiamo usare sia le regole e gli assiomi dell'algebra booleana per semplificare le
formule ma questo potrebbe portarci sia alla minima forma possibile sia ad un'espressione più
complessa. Con le mappe di Karnaugh non abbiamo la certezza di ottenere la miglior minimizzazione
ma ci offre un modo meccanico per ridurre la complessità.

Vogliamo ora implementare un \textbf{moltiplicatore} che moltiplica due sequenze da 2 bit dando
come risultato una sequenza da 4 bit. Per capire come calcolare tale sequenza possiamo procedere
tramite una tabella di verità che però, avendo 4 bit di ingresso risulta avere 16 righe. Cerchiamo
quindi di rappresentare solo le righe significative.
\begin{center}
	\begin{tabular}{c c c c | c c c c}
		$x_1$ & $x_2$ & $y_1$ & $y_2$ & $z_1$ & $z_2$ & $z_3$ & $z_4$ \\ \hline
		0     & 0     & -     & -     & 0     & 0     & 0     & 0     \\
		-     & -     & 0     & 0     & 0     & 0     & 0     & 0     \\ \hline
		0     & 1     & 0     & 1     & 0     & 0     & 0     & 1     \\
		      &       & 1     & 0     & 0     & 0     & 1     & 0     \\
		      &       & 1     & 1     & 0     & 0     & 1     & 1     \\ \hline
		1     & 0     & 0     & 1     & 0     & 0     & 1     & 0     \\
		      &       & 1     & 0     & 0     & 1     & 0     & 0     \\
		      &       & 1     & 1     & 0     & 1     & 1     & 0     \\ \hline
		1     & 1     & 0     & 1     & 0     & 0     & 1     & 1     \\
		      &       & 1     & 0     & 0     & 1     & 1     & 0     \\
		      &       & 1     & 1     & 1     & 0     & 0     & 1     \\
	\end{tabular}
\end{center}
A questo punto possiamo disegnare una mappa di Karnaugh per ogni uscita $z_i$ che abbiamo,
limitiamoci per il momento a disegnare solo quelle di $z_1$ e $z_3$.
\begin{center}
\begin{figure}[h!]\centering
\begin{subfigure}[b]{0.4\textwidth}
\begin{karnaugh-map}[4][4][1][$y_2$][$y_1$][$x_2$][$x_1$]
\minterms{15}
\maxterms{0,1, 3, 2, 4, 5, 7, 6, 12, 13, 14, 8, 9, 11, 10}
\implicant{15}{15}
\end{karnaugh-map}
\end{subfigure}
\begin{subfigure}[b]{0.4\textwidth}
\begin{karnaugh-map}[4][4][1][$y_2$][$y_1$][$x_2$][$x_1$]
\minterms{6, 7, 15, 13, 9, 10}
\maxterms{0, 1, 3, 2, 4, 5, 12, 14, 8, 11}
\implicant{7}{15}
\implicant{7}{6}
\implicant{13}{9}
\implicant{10}{10}
\end{karnaugh-map}
\end{subfigure}
\end{figure}
\end{center}
Come possiamo vedere anche dalle formule che seguono, per il calcolo di $z_1$ è sufficiente una
porta \verb|AND| mentre per il calcolo di $z_3$ sono necessarie 4 porte \verb|AND| e 1 porta
\verb|OR|. C'è quindi un ritardo tra il calcolo di $z_1$ e $z_3$ e il ritardo complessivo è dovuto
al passaggio del calcolo da 2 livelli di porte logiche.
\begin{align*}
	z_1 & = x_1 x_2 y_1 y_2                                                                   \\
	z_3 & = x_1 \bar{y_1} y_2 + x_2 y_1 y_2 + \bar{x_1} x_2 y_1 + x_1 \bar{x_2} y_1 \bar{y_2}
\end{align*}
In alternativa, considerando che una moltiplicazione tra due sequenze di 2 bit si svolge in questo
modo
\begin{center}
	\begin{tabular}{c c c c}
		  & 1 & 1 & $\times$ \\
		  & 1 & 0 & =        \\ \hline
		  & 0 & 0 & +        \\
		1 & 1 & - & =        \\ \hline
		1 & 1 & 0
	\end{tabular}
\end{center}
possiamo notare che se il bit al moltiplicatore è 0 allora avremo tutti 0 mentre se abbiamo 1 il
risultato sarà esattamente il moltiplicando. Possiamo quindi calcolarci separatamente
$x_1 x_2 \cdot y_1$ e $x_1 x_2 \cdot y_2$ tramite due multiplexer di questo tipo
\begin{center}
	\begin{tikzpicture}
		\draw[thick] (-1.25, 1) -- (1.25, 1) -- (0.75, 0) -- (-0.75, 0) -- cycle;
		\node (mux) at (0, 0.5) {MUX};
		\draw (-0.5, 1.5) node[label=left:$x_1 x_2$] {} to[short, o-] (-0.5, 1);
		\draw (0.5, 1.5) node[label=right:$00$] {} to[short, o-] (0.5, 1);
		\draw (-1.75, 0.5) node[label=left:$y_i$] {} to[short, o-] (-1, 0.5);
		\draw (0, 0) -- (0, -0.75);
	\end{tikzpicture}
\end{center}
Dove $x_1 x_2$ è un ingresso da 2 bit e dove l'altro ingresso è la costante 00. Questo multiplexer
effettua esattamente la scelta di cui abbiamo parlato prima: se $y_i = 0$ dà come risultato 00, se
invece $y_i = 1$ dà come risultato $x_1 x_2$.

Per implementare un moltiplicare $2 \times 2$ dobbiamo sostanzialmente affiancare due di questi
multiplexer, uno per $y_1$ e uno per $y_2$, aggiungere degli zeri dove necessario e poi effettuare
una somma con un bit di riporto $R$ che viene messo in cima alla sequenza generata. Il circuito che
ne risulta è un qualcosa di questo tipo
\begin{center}
	\begin{tikzpicture}
		\draw[thick] (-4.25, 1) -- (-1.75, 1) -- (-2.25, 0) -- (-3.75, 0) -- cycle;
		\node (mux) at (-3, 0.5) {MUX};
		\draw (-3.5, 1.5) node[label=left:$x_1 x_2$] {} to[short, o-] (-3.5, 1);
		\draw (-2.5, 1.5) node[label=right:$00$] {} to[short, o-] (-2.5, 1);
		\draw (-4.75, 0.5) node[label=left:$y_1$] {} to[short, o-] (-4, 0.5);
		\draw (-3, 0) -- (-3, -1) node[label=above left:$z_1 z_2$] {};

		\draw[thick] (4.25, 1) -- (1.75, 1) -- (2.25, 0) -- (3.75, 0) -- cycle;
		\node (mux) at (3, 0.5) {MUX};
		\draw (2.5, 1.5) node[label=left:$x_1 x_2$] {} to[short, o-] (2.5, 1);
		\draw (3.5, 1.5) node[label=right:$00$] {} to[short, o-] (3.5, 1);
		\draw (4.75, 0.5) node[label=right:$y_2$] {} to[short, o-] (4, 0.5);
		\draw (3, 0) -- (3, -1) node[label=above right:$z_3 z_4$] {};

		\draw (-1.5, -0.5) node[label=right:$0$] {} to[short, o-] (-1.5, -1) to[short, -*] (-2.25, -1);
		\draw (-3, -1) -- (-2.25, -1) -- (-2.25, -1.75) -- (-0.5, -1.75) -- (-0.5, -2.5);

		\draw (1.5, -0.5) node[label=left:$0$] {} to[short, o-] (1.5, -1) to[short, -*] (2.25, -1);
		\draw (3, -1) -- (2.25, -1) -- (2.25, -1.75) -- (0.5, -1.75) -- (0.5, -2.5);

		\draw[thick] (-1, -2.5) rectangle (1, -3.75);
		\node (add) at (0, -3.125) {ADD};
		\draw (0, -3.75) -- (0, -4.25);
		\draw (-1, -3.125) -- (-1.5, -3.125) -- node[label=left:$r$] {} (-1.5, -4.25);
	\end{tikzpicture}
\end{center}
In questo modo, dato che, come abbiamo visto in precendenza, sia il multiplexer che l'addizionatore
sono implementati tramite un circuito a due livelli di porte logiche abbiamo in totale un circuito
costituito da quattro livelli di porte logiche.

Questo si traduce in un maggior numero di componenti e in un maggior tempo di elaborazione ma in
compenso facciamo uso di due componenti standard che abbiamo già implementato e non dobbiamo
ricorrere alla costruzione di tabelle di verità, mappe di Karnaugh ecc.

Altro circuito considerato standard è il \textbf{codificatore} il quale ha $n$ bit in ingresso e
$2^n$ bit in uscita, e mette a 1 solo l'$n$-esimo bit lasciando tutti gli altri a 0. Se ad esempio
avessimo in ingresso 2 bit, avremo di conseguenza un circuito di questo tipo
\begin{center}
	\begin{tikzpicture}
		\draw[thick] (-0.8, 1) -- (0.8, 1) -- (1.5, 0) -- (-1.5, 0) -- cycle;
		\node (mux) at (0, 0.5) {ENCODER};

		\draw (-0.5, 1.5) node[label=above:$x_1$] {} to[short, o-] (-0.5, 1);
		\draw (0.5, 1.5) node[label=above:$x_2$] {} to[short, o-] (0.5, 1);

		\draw (-1.25, 0) -- (-1.25, -0.5) node[label=below:$z_1$] {};
		\draw (-0.5, 0) -- (-0.5, -0.5) node[label=below:$z_2$] {};
		\draw (0.5, 0) -- (0.5, -0.5) node[label=below:$z_3$] {};
		\draw (1.25, 0) -- (1.25, -0.5) node[label=below:$z_4$] {};
	\end{tikzpicture}
\end{center}
e la tabella di verità corrispondente sarebbe la solita
\begin{center}
	\begin{tabular}{c c | c c c c}
		$x_1$ & $x_2$ & $z_1$ & $z_2$ & $z_3$ & $z_4$ \\ \hline
		0     & 0     & 1     & 0     & 0     & 0     \\
		0     & 1     & 0     & 1     & 0     & 0     \\
		1     & 0     & 0     & 0     & 1     & 0     \\
		1     & 1     & 0     & 0     & 0     & 1
	\end{tabular}
\end{center}
Le formule risultanti da questa tabella sono
\begin{align*}
	z_1 & = \bar{x_1} \bar{x_2} \\
	z_2 & = \bar{x_1} x_2       \\
	z_3 & = x_1 \bar{x_2}       \\
	z_4 & = x_1 x_2
\end{align*}


% \chapter{Reti logiche}
% \chapter{Unità firmware}
% \chapter{Linguaggio assembler}
% \chapter{Microarchitettura}
% \chapter{Gestione della memoria}
% \chapter{Gestione I/O}
% \chapter{Architetture avanzate}

% \begin{figure}[!h]\centering
% 	\begin{circuitikz}
% 		% logic gates
% 		\node[and port] (and1) at (0, 2)  {};
% 		\node[or port] (or) at (0, 0)  {};
% 		\node[and port]	(and2) at (0, -2) {};

% 		\node[xnor port] (xnor) at (2.5, 1) {};
% 		\node[not port] (not) at (2.5, -1) {};

% 		\node[xor port] (xor) at (5, 0) {};

% 		% connections
% 		\draw (and1.out) |- (xnor.in 1);
% 		\draw (or.out)   |- (xnor.in 2);
% 		\draw (and2.out) |- (not.in);

% 		\draw (xnor.out) |- (xor.in 1);
% 		\draw (not.out)  |- (xor.in 2);
% 	\end{circuitikz}
% \end{figure}

% \begin{tabular}{c|c}
% 	Pari & Dispari \\ \hline
% 	0    & 1       \\
% 	2    & 3       \\
% 	4    & 5       \\
% 	6    & 7       \\
% 	8    & 9
% \end{tabular}

\end{document}
