\documentclass[11pt]{report}

% --------------- PACKAGES ---------------
\usepackage{inputenc}
\usepackage[T1]{fontenc}
\usepackage[italian]{babel}
\usepackage[hidelinks]{hyperref}

\hypersetup{
	colorlinks=true,
	linkcolor=blue!70!black
}

% --------------- STYLE ---------------
\usepackage[margin=1.25in]{geometry}
\usepackage[most]{tcolorbox}

% Font
\usepackage{sansmath}

\renewcommand{\familydefault}{\sfdefault}
\sansmath

% page style
\usepackage{fancyhdr}
\usepackage[Sonny]{fncychap}

\pagestyle{fancy}
\setlength{\headheight}{15pt}
\rhead{\thepage}
\cfoot{\thepage}

% --------------- MATH ---------------
\usepackage{amsmath}
\usepackage{amssymb}
\usepackage{amsthm}
\usepackage{amsfonts}
\usepackage{mathtools}
\usepackage{mdframed}

\newcommand{\N}{\mathbb{N}}
\newcommand{\Z}{\mathbb{Z}}
\newcommand{\R}{\mathbb{R}}
\newcommand{\C}{\mathbb{C}}
\newcommand{\E}{\mathbb{E}}

\newcommand{\F}{\mathcal{F}}

\DeclareMathOperator{\Var}{Var}
\DeclareMathOperator{\Cov}{Cov}

% Boxes for theorem, definitions and examples
\newtheoremstyle{math_box}
{0pt}
{0pt}
{\normalfont}
{}
{\color{orange}}
{\;}
{0.25em}
{\thmname{\textbf{#1}}\thmnumber{ \textbf{#2}}{\color{black}\thmnote{\textbf{ -- #3.}}}}

\newmdenv[
	rightline=false,
	leftline=true,
	topline=false,
	bottomline=false,
	linecolor=orange!40,
	innerleftmargin=5pt,
	innerrightmargin=5pt,
	innertopmargin=0pt,
	innerbottommargin=0pt,
	leftmargin=0cm,
	rightmargin=0cm,
	linewidth=3pt
]{dBox}

\newmdenv[
	rightline=false,
	leftline=false,
	topline=false,
	bottomline=false,
	backgroundcolor=orange!15,
	innerleftmargin=5pt,
	innerrightmargin=5pt,
	innertopmargin=5pt,
	innerbottommargin=5pt,
	leftmargin=0cm,
	rightmargin=0cm,
]{pBox}

\theoremstyle{math_box}
\newtheorem{theoremeT}{Teorema}[chapter]
\newtheorem{definitionT}{Definizione}[chapter]
\newtheorem{propositionT}{Proposizione}[chapter]
\newtheorem{corollary}{Corollario}[chapter]
\newtheorem{lemma}{Lemma}[chapter]
\newtheorem{observation}{Osservazione}[chapter]
\newtheorem{exampleT}{Esempio}[section]

\newenvironment{theorem}{\begin{pBox}\begin{theoremeT}}{\end{theoremeT}\end{pBox}}
\newenvironment{definition}{\begin{dBox}\begin{definitionT}}{\end{definitionT}\end{dBox}}
\newenvironment{proposition}{\begin{pBox}\begin{propositionT}}{\end{propositionT}\end{pBox}}
\newenvironment{example}{\begin{dBox}\begin{exampleT}}{\end{exampleT}\end{dBox}}

\usepackage{tikz, pgfplots, pgf-pie}
\usepackage{caption, subcaption}
\usepackage{scalerel}
\usepackage{pict2e}
\usepackage{tkz-euclide}
\usepackage{pgfplots, pgfplotstable, pgf-pie}
\usepackage{circuitikz}

\usetikzlibrary{calc}
\usetikzlibrary{patterns, arrows}
\usetikzlibrary{shadows}
\usetikzlibrary{external}

\pgfplotsset{compat=newest}
\usepgfplotslibrary{statistics, fillbetween}

\tikzstyle{branch}=[
fill,
shape=circle,
minimum size=3pt,
inner sep=0pt
]


\usepackage{minted}
\usepackage{diagbox}
\usepackage[label=corner]{karnaugh-map}

\definecolor{minted_bg}{rgb}{0.9, 0.9, 0.9}
\usemintedstyle{colorful}

\setminted[c]{
	tabsize=4,
	% linenos=true,
	bgcolor=minted_bg,
	fontsize=\small,
	mathescape=true
}

\setminted[gas]{
	tabsize=4,
	% linenos=true,
	bgcolor=minted_bg,
	fontsize=\small,
	mathescape=true
}

\setminted[bash]{
	tabsize=4,
	% linenos=true,
	bgcolor=minted_bg,
	fontsize=\small,
	mathescape=true
}

\setminted[verilog]{
	tabsize=4,
	% linenos=true,
	bgcolor=minted_bg,
	fontsize=\small,
	mathescape=true
}

\title{Architettura degli elaboratori}
\author{Federico Bustaffa}
\date{13/11/2023}

\begin{document}

\maketitle
\tableofcontents

\chapter{Introduzione}
L'obbiettivo del corso è quello di progettare sistemi robusti che non crollino al primo attacco o davanti al
primo utente ingenuo che ne fa uso.

Gran parte degli oggetti di uso quotidiano sono stati resi "intelligenti" dall'aggiunta di una componente
informatica al loro interno. Tali oggetti, anche se dall'esterno possono sembrare estremamente limitati, sono
sistemi completi e flessibili in grado di eseguire qualsiasi software.

L'aggiornamento online del software all'interno di tali sistemi è un mezzo per mantenere l'oggetto all'avanguardia
ma implica che esso possa essere manipolato da remoto per eseguire qualsiasi software e dunque per modificare
il comportamento dell'oggetto a proprio piacimento.

\section{Proprietà di sicurezza}
Un sistema informatico è, ad ogni livello di implementazione, formato da un insieme di moduli connessi, ognuno dei
quali offre un certo numero di operazioni.

Queste operazioni permettono di leggere e manipolare informazioni, che hanno poi un impatto sul mondo esterno.

In ogni sistema informatico ci sono regole (\textbf{politica di sicurezza}) che definiscono chi può invocare una
certa operazione e quindi ha il diritto di leggere o manipolare informazioni. Tali regole vengono implementate da
un sottoinsieme dei moduli del sistema informatico.

In questo contesto le tre principali proprietà che ci interessano sono:
\begin{itemize}
	\item \textbf{Confidenzialità}: solo chi ha il diritto di leggere una certa informazione può farlo.
	\item \textbf{Integrità}: solo chi ha il diritto di aggiornare una certa informazione può farlo.
	\item \textbf{Disponibilitò}: chi ha un diritto e vuole esercitarlo riesce a farlo in un tempo finito.
\end{itemize}
Da queste tre proprietà fondamentali possiamo derivarne altre secondarie:
\begin{itemize}
	\item \textbf{Tracciabilità}: individuare chi ha invocato un'operazione.
	\item \textbf{Accountability}: addebitare l'uso delle risorse.
	\item \textbf{Auditability}: verificare l'efficacia dei meccanismi di \emph{enforcement} di una politica (come
	      viene realizzata).
	\item \textbf{Forensics}: provare che certe azioni hanno avuto luogo.
	\item \textbf{Privacy/GDPR}: individuare chi, come e se un utente può usare informazioni personali.
\end{itemize}

\section{Politiche di sicurezza}
Una \textbf{politica di sicurezza} è un insieme di regole definite dal proprietario del sistema o del processo
aziendale per decidere gli utenti che possono invocare un'operazione e quando possono farlo.

Esistono diverse categorie di politiche descrivibili come il risultato di due scelte indipendenti:
\begin{itemize}
	\item La prima scelta è relativa al come la politica viene descritta:
	      \begin{itemize}
		      \item \textbf{Default allow}: operazioni vietate.
		      \item \textbf{Default deny}: operazioni permesse.
	      \end{itemize}
	\item La seconda scelta definisce vincoli sul proprietario del sistema:
	      \begin{itemize}
		      \item \textbf{Discretionary Access Control}: decide il proprietario.
		      \item \textbf{Mandatory Access Control}: esistono vincoli globali a tutto il sistema che nemmeno il
		            proprietario può violare.
	      \end{itemize}
\end{itemize}

\subsection{Soggetti e oggetti}
Relativamente alla seconda caratteristica che una politica deve avere, si cerca di modellare le risorse condivise
come \textbf{oggetti} e gli utenti come \textbf{soggetti}.

I soggetti invocano le operazioni definite dagli oggetti, se un oggetto invoca le operazioni definite da altri
oggetti allora diventa esso stesso un soggetto. In sintesi
\begin{itemize}
	\item \textbf{Soggetto}: in genere è un utente, un processo, un thread, un'istruzione.
	\item \textbf{Oggetto}: in genere si tratta di tipi di dati astratti, procedure, parametri, risorse logiche o
	      fisiche.
\end{itemize}

\subsection{Discretionary Access Control - DAC}
In questo modello per ogni oggetto esiste un \textbf{proprietario} (del sistema o del processo), il quale decide i
diritti dei vari soggetti mentre lui non ha vincoli di alcun tipo. Questo modello è tipico del mondo industriale.

\subsection{Mandatory Access Control - MAC}
Questo modello prevede la divisione di soggetti (utenti) e oggetti (risorse) in \textbf{classi}. Le classi sono
ordinate parzialmente in \textbf{livelli} (1, 2, 3 e così via).

Il livello di un soggetto esprime il grado di libertà che vogliamo lasciare a tale soggetto. Tanto più alto è il
livello di un soggetto tanto maggiore sarà il livello degli oggetti con cui esso può interagire.

Il livello di un oggetto esprime invece il grado di importanza di tale oggetto. Tanto più alto è il livello di un
oggetto tanto maggiore dev'essere è il livello del soggetto perché esso possa interagire con tale oggetto.

\subsubsection{Information Flow I}
In questo tipo di politica un utente può
\begin{itemize}
	\item Leggere tutti i file che hanno classe minore o uguale alla sua.
	\item Modificare i record dei file che hanno classe uguale alla sua.
	\item Appendere un record ad un file che ha classe maggiore della sua.
\end{itemize}
Per compiere queste operazioni è necessario il permesso dell'\emph{owner} che può solo restringere ulteriormente ciò
che un utente può fare. Si tratta di una politica MAC di tipo \textbf{no write down} e privilegia la confidenzialità.

\subsubsection{Information Flow II}
In questo tipo di politica un utente può
\begin{itemize}
	\item Scrivere tutti i file di una classe minore o uguale alla sua.
	\item Leggere tutti i file di una classe maggiore o uguale.
\end{itemize}
Questo implica che un utente poco affidabile, ossia di basso livello, non può andare a modificare dati critici. Si
tratta di una politica MAC di tipo \textbf{no write up} e privilegia l'integrità.

\subsubsection{Chinese Wall}
Gli oggetti del sistema sono partizionati in sottoinsiemi. L'utente che ha operato su un oggetto di un insieme non
può operare su oggetti di un altro insieme.

Questa politica \textbf{dinamica} permette di gestire conflitti di interesse ed è compatibile con la politica MAC
poiché aggiunge vincoli.

\subsubsection{Watermark}
Questa politica non prevede che un soggetto abbia un livello fissato ma che varia in base alle operazioni che esso
compie sui vari oggetti e dal livello di questi ultimi.

Il livello di un soggetto aumenta dopo che questo legge un oggetto di livello più alto del suo, rimane invece invariato
se il soggetto legge un oggetto di livello più basso.

\section{Matrice di controllo degli accessi}
Si tratta di una matrice con un comportamento molto dinamico che ha tante righe quanti sono i soggetti e tante colonne
quanti sono gli oggetti.

Nella posizione identificata dal soggetto $i$ e dall'oggetto $j$ si trovano i \textbf{diritti} che quel soggetto ha
su quell'oggetto. In generale è bene ai fini di sicurezza che la matrice contenga pochi diritti e che dunque appaia
sostanzialmente vuota.

Una rappresentazione concreta di tale matrice è necessaria ma non sufficiente per il rispetto della politica.

\subsection{Linux}
Nei sistemi operativi Linux vi è una rappresentazione concreta di tale matrice nel file system. Per ogni file, Linux
fornisce una sequenza di bit (bitmask) che indica i diritti che l'utente ha su tali file (lettura, scrittura,
esecuzione e così via).

\section{Trusted Computing Base - TCB}
Una caratteristica dei sistemi informatici è quella di avere al loro interno componenti informatici per implementare
e gestire la loro politica di sicurezza interna.

Se uno di questi componenti ha un errore o c'è un errore nei componenti che esso utilizza, allora l'implementazione
della politica non è corretta e quasi sicuramente un soggetto potrebbe invocare operazioni per le quali non ha i
diritti necessari.

\subsection{Dimensioni del TCB}
Per quanto riguarda le dimensioni del TCB possiamo dire che, tanto minore è il numero delle sue componenti, tanto
maggiore è la sicurezza del sistema.

Un TCB con dimensioni contenute permette anche una dimostrazione matematica relativamente semplice della sua
correttezza.

La dimensione del TCB è anche un criterio che permette di confrontare strategie alternative nella realizzazione della
politica di sicurezza.

\section{Vulnerabilità}
Quando si parla di \textbf{vulnerabilità} si vuole indicare un \emph{difetto}
\begin{itemize}
	\item Hardware
	\item Software
	\item Nell'utente
	\item Nelle regole della politica
\end{itemize}
che permette di violare la politica di sicurezza del sistema permettendo ad un soggetto di compiere un'operazione per
la quale non ha diritti.

L'obbiettivo della sicurezza informatica è quello di costruire sistemi che funzionano anche con delle vulnerabilità
e non quello di costruire sistemi senza vulnerabilità. In genere le varie vulnerabilità dei vari componenti vengono
\emph{compensate} in qualche modo da altri componenti.

La vulnerabilità più frequente, spesso non è nel codice, ma nel fatto che chi implementa il sistema dia per scontato
che l'utente non faccia errori nell'interfacciarsi con esso.

\subsection{Intrusioni}
Un'\textbf{intrusione} è una sequenza di \textbf{azioni} e \textbf{attacchi} per riuscire a controllare gli oggetti
del sistema.

Non tutte le azioni che l'attaccante fa sono \emph{illegali}, alcune di esse possono essere perfettamente lecite ma
sfruttate poi per violare il sistema.

In un'intrusione si sfrutta una o più vulnerabilità, usando anche programmi automatizzati (\textbf{exploit}) per ognuna
di esse, per riuscire a sostituirsi all'\emph{owner} del sistema e dunque avere la possibilità di
\begin{itemize}
	\item Raccogliere informazioni
	\item Modificare informazioni
	\item Impedire ad altri di accedere alle informazioni
\end{itemize}

\subsubsection{Fasi di un'intrusione}
Vediamo nello specifico cosa fa un hacker quando tenta un'intrusione:
\begin{enumerate}
	\item Raccoglie di informazioni iniziali sul sistema.
	\item Individuazione delle vulnerabilità del sistema per compiere un accesso iniziale.
	\item Ripetizione di una sequenza di operazioni finché non ha successo:
	      \begin{itemize}
		      \item Raccolta di informazioni sul sistema.
		      \item Scoperta di vulnerabilità.
		      \item Costruzione di exploit.
		      \item Attacco: si eseguono gli exploit ed eventuali azioni manuali.
	      \end{itemize}
	\item Installazione di strumenti per il controllo.
	\item Cancellazione delle tracce dell'attacco.
	\item Accesso, modifica ecc. ad un \emph{sottoinsieme} delle informazioni del sistema o si compiono altri
	      tipi attacchi:
	      \begin{itemize}
		      \item Furto di informazioni.
		      \item Cifratura di dati per chiedere riscatto.
	      \end{itemize}
\end{enumerate}
Un'intrusione può essere vista come un'\textbf{escalation} nell'acquisizione di diritti e nella raccolta di
informazioni tramite vari attacchi ripetuti.

Un attaccante in genere cerca di acquisire delle informazioni che gli permettano di avere nuovi diritti. Una volta
ottenuti i diritti è in grado di acquisire nuove informazioni e così via finché non raggiunge il proprio obbiettivo.

\subsection{Approcci alla sicurezza}
Le intrusioni sono dunque possibili grazie alle vulnerabilità e questo ci porta a definire due approcci alla sicurezza:
\begin{itemize}
	\item \textbf{Sicurezza incondizionale}: si assume che qualsiasi sia la vulnerabilità nel sistema, esista qualcuno
	      interessato ad usarla ed è quindi necessario eliminarle tutte.
	\item \textbf{Sicurezza condizionale}: in questo tipo di approccio si fa un'analisi in cui si cerca di capire quali
	      siano le reali minacce per il sistema e si eliminano solo le vulnerabilità che tali minacce potrebbero usare
	      per attaccare il sistema.
\end{itemize}

\subsubsection{Analisi del rischio}
Il primo approccio comporta costi molto elevati e spesso inaccettabili, inoltre richiede una quantità di lavoro enorme
e spesso inutile.

Con il secondo approccio invece si cerca di capire quali componenti del sistema si possono difendere e soprattutto
quali componenti \emph{conviene} difendere.

Per capirlo è necessaria un'\textbf{analisi del rischio} con la quale si cerca di individuare la tipologia di attacco
più probabile in relazione al sistema che stiamo cercando di proteggere.

% PARTE 1: Reti logiche
\part{Reti logiche}

\chapter{Aritmetica binaria e algebra booleana}
Passiamo ora ad un breve richiamo sull'\textbf{aritmetica binaria} e sulla \textbf{logica booleana}
per capire meglio le regole seguite da un calcolatore per il suo funzionamento.

\section{Aritmetica binaria}
Il sistema di numerazione binario, così come quelli decimale ed esadecimale, è detto
\textbf{posizionale}. Si assegna cioè un \textbf{peso} alle possibili posizioni in cui può cadere
una delle cifre a cui poi verrà moltiplicata la cifra effettiva.

Il numero $123_{10}$ si ottiene dalla somma delle sue singole cifre moltiplicate per il peso della
posizione in cui si trovano. Nello specifico, se la base di numerazione è $b$, partendo dalla
posizione più a destra, il peso della prima posizione sarà $b^0$ mentre il peso dell'$n$-esima
posizione sarà $b^{n-1}$.
\[ 123_{10} = 1 \cdot 10^2 + 2 \cdot 10^1 + 3 \cdot 10^0 = 100 + 20 + 3 \]
Se ad esempio volessimo fare lo stesso per il numero $1010_2$ otterremmo
\[ 1010_2 = 1 \cdot 2^3 + 0 \cdot 2^2 + 1 \cdot 2^1 + 0 \cdot 2^0 = 8 + 2 = 10_{10} \]
Come possiamo vedere abbiamo anche trovato un modo semplice per la conversione da binario a
decimale. In genere per la rappresentazione in macchina viene usata la notazione
\textbf{esadecimale} poiché, avendo 16 simboli a disposizione, fornisce una rappresentazione più
compatta di numeri binari altrimenti molto lunghi e difficilmente leggibili.
\[ 1F0_{16} = 1 \cdot 16^2 + 15 \cdot 16^1 + 0 \cdot 16^0 = 256 + 240 = 496_{10} \]
L'operazioni più semplice che possiamo fare tra numeri binari è ovviamente la \textbf{somma} che
funziona in modo analogo
\begin{center}
	\begin{tabular}{c | c c}
		$2_{10}$ & $0010$ & + \\
		$3_{10}$ & $0011$ & = \\ \hline
		$5_{10}$ & $0101$
	\end{tabular}
\end{center}

Per la \textbf{conversione} da decimale a binario possiamo
\begin{enumerate}
	\item Prendere la più grande potenza di 2 minore o uguale del numero che vogliamo convertire.
	\item Sottraiamo al numero la potenza di 2 trovata.
	\item Ripetiamo il procedimento con la differenza ottenuta finché non si ottiene 0.
	\item Quando si ottiene 0 si va a mettere un 1 in ognuna delle posizioni relative alle potenze
	      trovate.
\end{enumerate}
Se ad esempio volessimo rappresentare $17_{10}$ troveremmo che $2^4 = 16$ è la più grande potenza
di 2 minore o uguale di 17. Quindi eseguiamo $17 - 16 = 1$ a questo punto abbiamo che $2^0 = 1$ è
la più grande potenza minore o uguale di 1. Eseguiamo $1-1 = 0$ e concludiamo. Il risultato sarà
\[ 17_{10} = 10001_2 \]
dove gli 1 si trovano rispettivamente in posizione 1 e 5 ossia le posizioni di peso $2^0$ e $2^4$.

\subsection{Rappresentazione in macchina}
Se volessimo rappresentare numeri in macchina avendo a disposizione registri di $n$ bit potremmo
rappresentare $2^n$ possibili numeri (in particolare in numeri da 0 a $2^n - 1$).

Si ha però un problema nel momento in cui si vogliono rappresentare numeri negativi. Supponiamo di
voler rappresentare i numeri da $-x$ a $+x$, il primo metodo di rappresentazione dei numeri
negativi è il cosiddetto \textbf{modulo e segno}.

Questo metodo utilizza un bit per il segno e i restanti vengono usati per la rappresentazione come
abbiamo visto fin ad ora.

Per quanto riguarda i numeri positivi il bit di segno sarà a 0 e per i bit negativi sarà a 1. La
restante rappresentazione del numero rimane invariata. Notiamo che adesso, con registri da $n$ bit
ne possiamo usare $n-1$ per la rappresentazione e dunque avremo a disposizione i numeri da
$-2^{n-1}$ a $+2^{n-1}$.

Questa rappresentazione si porta dietro due problemi principali: il primo è che rappresenta due
volte lo 0 (-0 e +0) e poi introduce la necessità di una componente in grado di fare le sottrazioni
(necessaria quando si sottrae un numero più grande ad un numero più piccolo).

\subsubsection{Rappresentazione in complemento a 2}
Si è quindi passati alla rappresentazione in \textbf{complemento a 2}, la quale prevede che, per i
numeri positivi si utilizzino i primi $n-1$ bit per la rappresentazione e l'$n$-esimo bit messo a 0.
Per i numeri negativi si deve invece rappresentare prima di tutto il modulo del numero, dopodiché
si negano tutti i bit e infine si somma 1.

Se ad esempio volessimo rappresentare $-3$ in complemento a 2 su 8 bit il risultato sarebbe il
seguente:
\begin{enumerate}
	\item Rappresentazione 3 in base 2: $00000011_2$.
	\item Negazione di tutti i bit $00000011_2 \to 11111100_2$.
	\item Sommiamo 1 al risultato $11111100 + 1 = 11111101$.
\end{enumerate}
Come possiamo vedere otteniamo una rappresentazione in cui il numero più a sinistra è 1, e quindi
capiamo che il numero è negativo. Se volessimo sapere di quale numero si tratta ci basta fare il
procedimento inverso negando tutti i bit
\[ 11111101 \to 00000010 \]
e sommandoci 1
\begin{center}
	\begin{tabular}{r c}
		00000010 & + \\
		1        & = \\ \hline
		00000011
	\end{tabular}
\end{center}
che infatti è 3. Questo metodo ci permette di effettuare sottrazioni semplicemente facendo somme
con numero negativi. Se per esempio volessimo effettuare $2 - 3$ otterremmo
\begin{center}
	\begin{tabular}{r c}
		00000010 & + \\
		11111101 & = \\ \hline
		11111111
	\end{tabular}
\end{center}
Che se convertiamo in modulo diventa $11111111 \to 00000000 + 1 = 00000001$ ovvero 1 e quindi
deduciamo facilmente che il risultato della somma appena fatta è $-1$.

Altro fatto importante è che l'unica rappresentazione dello 0 è quella con tutti i bit messi a 0,
mentre la rappresentazione
\[ 10000000 \]
non è valida o comunque non viene mai generata da possibili calcoli in aritmetica binaria.

\subsubsection{Rappresentazione in virgola mobile}
La rappresentazione in virgola mobile ha subito diverse variazioni negli anni fino ad arrivare ad
uno standard che ad oggi prende il nome di IEEE 754, il quale prevede l'utilizzo di
\begin{itemize}
	\item Un bit per il \textbf{segno}.
	\item Un certo numero di bit per l'\textbf{esponente}.
	\item I restanti bit per la \textbf{mantissa}.
\end{itemize}
Il numero di bit per esponente e mantissa dipende dalla lunghezza dei registri che utilizziamo (se
a 32 o a 64 bit).
\begin{itemize}
	\item Se il bit del segno è 0 il numero è positivo, se invece è 1 allora il numero è negativo.
	\item L'esponente indica il numero a cui elevare la base.
	\item La mantissa rappresenta il numero con una virgola in posizione \emph{"fissa"}.
\end{itemize}
Il numero $x$ cercato è rappresentato come
\[ x = \text{sign}(x) \cdot b^e \cdot m \]
L'esponente è rappresentato in \textbf{eccesso a k}, ovvero se l'esponente è $e$, il calcolatore
rappresenta $e+k$ in modo che numeri più grandi di $k$ rappresentano numeri positivi mentre numeri
più piccoli di $k$ rappresentano numeri negativi. Per lo standard IEEE 754, nel caso di numeri in
virgola mobile a
\begin{itemize}
	\item 32 bit, l'esponente è rappresentato con 8 bit in eccesso a 127 e la mantissa ha a
	      disposizione 23 bit.
	\item 64 bit, l'esponente è rappresentato con 11 bit in eccesso a 1023 e la mantissa ha a
	      disposizione 52 bit.
\end{itemize}
Aggiungiamo inoltre che lo standard vuole che la prima cifra per una rappresentazione in virgola
mobile sia non nulla e tale che alla sua sinistra vi siano solo cifre non significative
\[ 123.45 = 1.2345 \times 10^2 \]
Questo implica che le operazioni tra numeri rappresentati in virgola mobile non sono così immediate
nel caso in cui questi abbiano ordini di grandezza diversi.

Quel che viene fatto è passarli ad un \textbf{incolonnatore} che, tramite alcune operazioni riesce
a spostare la virgola e cambiare l'esponente dei due operandi in modo che il calcolo sia eseguibile
nel modo classico. Una volta effettuato il calcolo, il risultato passa da un \textbf{normalizzatore}
che riporta il risultato ad una rappresentazione che soddisfa lo standard.

Se vogliamo implementare la possibilità di eseguire operazioni di addizione tra numeri interi e tra
numeri in virgola mobile, abbiamo bisogno di una componente detta \textbf{ALU}. Per le operazioni
tra interi si parla di \textbf{ALU-I}, mentre per le operazioni tra numeri in virgola mobile si
parla di \textbf{ALU-FP}.

\subsubsection{Rappresentazione di caratteri}
In ultima battuta parliamo della rappresentazione di documenti testuali, i quali non sono altro
che \textbf{sequenze di caratteri}. Tali caratteri seguono la rappresentazione \textbf{ASCII} che
fa utilizzo di 8 o 16 bit.

Per sapere cosa rappresenta la sequenza di bit si fa utilizzo di una \textbf{tabella ASCII} che
mette in corrispondenza il numero rappresentato dalla sequenza di bit ad un carattere nella tabella.

In questo ambito è possibile effettuare piccole operazioni come \textbf{ordinamenti lessicografici}
poiché, per esempio, le lettere da "A" a "Z" hanno un codice che le mette in ordine una dopo
l'altra.

\subsection{Shift dei bit}
Quando si parla di moltiplicazioni e divisioni, un caso particolare è quello in cui si moltiplica
o si divide un numero per la base di numerazione o per una sua potenza. Se ad esempio volessimo
svolgere $123 \times 10$ è immediato riconoscere che il risultato è $1230$ poiché come sappiamo
basta aggiungere uno zero a destra.

Questo però implica uno \textbf{shift} delle cifre dalla lora posizione ad una posizione a peso
maggiore (a sinistra). Allo stesso modo, vale $123 / 10 = 12.3$, ottenuto tramite un shift a destra
delle cifre.

In modo del tutto analogo questo è anche possibile in aritmetica binaria ma invece di moltiplicare
o dividere per 10, moltiplichiamo o dividiamo per 2. Per esempio se dividiamo $10_{10} = 1010_2$
per 2 abbiamo, in decimale che il risultato è 5 e in binario, dato che dividiamo per la base ci
basterà shiftare tutti i bit a destra di una posizione ottenendo
\[ 0101 = 2^2 + 2^0 = 4 + 1 = 5_{10} \]
Se invece volessimo moltiplicare o dividere per una potenza di 2 dovremmo effettuare tanti shift
(a destra o a sinistra) quanto vale l'esponente. Per esempio $10 \times 2^2$ equivale ad effettuare
un doppio shift verso sinistra del numero $1010$, ottenendo così
\[ 101000 = 2^5 + 2^3 = 32 + 8 = 40_{10} \]
Questo ci sarà molto utile nell'indirizzamento della memoria che vedremo più avanti in quanto la
memoria del computer è partizionata in un numero di blocchi pari ad una potenza di due. I blocchi
sono a loro volta suddivisi in un numero di pagine sempre pari ad una potenza di 2. Questi ci
permette di muoverci tra i blocchi o le pagine della memoria in modo molto più agevole e veloce
tramite le operazioni di shift.

Per quanto riguarda i numeri negativi rappresentati in complemento a 2 dobbiamo avere una piccola
accortezza. Prendiamo per esempio l'operazione
\[ -8 / 2^2 = -2 \]
che in binario equivale a shiftare a destra di due posizioni il numero $11111000$ (-8 in
complemento a 2). Dato che però, tramite un normale shift a destra otterremmo $00111110$, cioè un
numero positivo, è chiaro che non è il risultato corretto. Semplicemente negli shift a destra di
numeri negativi le posizioni a sinistra vengono rimpiazzate da 1 invece che da 0, otteniamo quindi
$11111110$ che, se effettuiamo la conversione in complemento a 2, equivale a $00000010_2 = 2_{10}$.
Questo prende il nome di \textbf{shift aritmetico}.

Per quanto riguarda invece gli shift a sinistra di numeri negativi in complemento a 2 si effettua
un normale shift aggiungendo tanti 0 a destra quanto vale l'esponente della potenza per cui si
vuole moltiplicare. Per esempio
\[ -8 \times 2^2 = -32 \]
per ottenere il calcolo in binario dobbiamo shiftare a sinistra il numero $11111000$. Otteniamo
quindi $11100000$ che se convertito in modulo diventa $00100000_2 = 2^5 = 32_{10}$.
\section{Algebra booleana}
Nel corso non andremo a trattare l'ultimo livello di astrazione, ossia quello più basso, ma andremo
a trattare lo strato soprastante, che tramite delle \textbf{porte logiche} e delle operazioni
aritmetiche binarie riesce a rappresentare quello che succede. Le tre operazioni implementate dalle
porte logiche sono
\begin{itemize}
	\item \verb|AND(x,y)|: 1 se $x = y = 1$, 0 altrimenti.
	\item \verb|OR(x,y)|: 0 se $x = y = 0$, 1 altrimenti.
	\item \verb|NOT(x)|: 1 se $x=0$, 0 se $x=1$.
\end{itemize}
Altro strumento utile per capire meglio come funzionano tali porte e per vedere come funzionano
altre porte che risultano essere una combinazione di esse, sono le \textbf{tabelle di verità}.
Nelle tabelle di verità immettiamo tutti i possibili valori di input e calcoliamo i relativi output.
Per esempio, la tabella di verità di una porta logica \verb|AND| è la seguente
\begin{center}
	\begin{tabular}{c c | c}
		x & y & z \\ \hline
		0 & 0 & 0 \\
		0 & 1 & 0 \\
		1 & 0 & 0 \\
		1 & 1 & 1
	\end{tabular}
\end{center}
Per una questione legata alla circuiteria sottostante e alla leggi fisiche che regolano il
funzionamento dei transistor, il numero di ingressi delle porte è, in genere, al più 8 poiché
averne di più introduce troppo ritardo nell'elaborazione dei segnali.

L'algebra di booleana, tramite gli \textbf{assiomi} che determinano il comportamento dell'alfabeto
$\{0, 1\}$ in relazione a delle operazioni di base che possiamo fare con i suoi elementi, ci
permette di semplificare o più in generale di manipolare le espressioni dell'algebra booleana per
andare quindi a modellare anche i nostri circuiti. Di seguito andremo ad elencarli
\begin{gather*}
	a = 0 \implies a \neq 1 \quad \land \quad a = 1 \implies a \neq 0 \\
	a = 0 \implies \bar{a} = 1 \quad \land \quad a = 1 \implies \bar{a} = 0 \\
	0 \cdot 1 = 0 \quad \land \quad 1 \cdot 0 = 0 \\
	0 \cdot 0 = 0 \quad \land \quad 1 \cdot 1 = 1 \\
	0 + 1 = 1 \quad \land \quad 1 + 0 = 1 \\
	0 + 0 = 0 \quad \land \quad 1 + 1 = 1
\end{gather*}
Da questi deduciamo anche che
\begin{gather*}
	A \cdot 1 = A \quad \land \quad A + 0 = A \\
	A \cdot 0 = 0 \quad \land \quad A + 1 = 1 \\
	A \cdot A = A \quad \land \quad A + A = A \\
	A \cdot \bar{A} = 0 \quad \land \quad A + \bar{A} = 1 \\
	\bar{\bar{A}} = A
\end{gather*}
Le operazioni di \verb|AND|, \verb|OR| e \verb|NOT| dell'algebra booleana godono di alcune
proprietà molto utili per la manipolazione delle espressioni boooleane:
\begin{itemize}
	\item \textbf{Commutatività} per l'\verb|AND|: $A \cdot B = B \cdot A$
	\item \textbf{Commutatività} per l'\verb|OR|: $A + B = B + A$
	\item \textbf{Distributività}: $A \cdot (B + C) = A \cdot B + A \cdot C$ e la formula duale
	      $A + (B \cdot C) = (A \cdot B) + (A \cdot C)$
	\item \textbf{De Morgan}: $\overline{A \cdot B} = \bar{A} + \bar{B}$ e la formula duale
	      $\overline{A + B} = \bar{A} \cdot \bar{B}$
\end{itemize}
Con queste proprietà è possibile semplificare alcune delle formule generate da alcune tabelle di
verità come abbiamo fatto nel caso del multiplexer.

\chapter{Reti combinatorie}
Le \textbf{reti combinatorie} sono delle funzioni che hanno un certo numero di ingressi e un certo
numero di uscite, formate da una composizione di porte \verb|AND|, \verb|OR| e \verb|NOT|.

Vedremo nello specifico alcune tecniche per riuscire a derivare dei circuiti logici da tabelle di
verità e formule booleane in modo che questi rappresentino un'implementazione ragionevole.

\section{Derivazione dalle tabelle di verità}
Supponiamo ad esempio di voler implementare una rete combinatoria in grado di calcolare il numero
di bit a 1 su 2 ingressi ciascuno da 1 bit. In questo caso i possibili valori di output sono 3 (0,
1 e 2) e abbiamo quindi bisogno di un numero di uscite pari a $\lceil \log_2 (3) \rceil = 2$. La
tabella di verità del nostro circuito sarà la seguente
\begin{center}
	\begin{tabular}{c c | c c}
		$x_0$ & $x_1$ & $z_0$ & $z_1$ \\ \hline
		0     & 0     & 0     & 0     \\
		0     & 1     & 0     & 1     \\
		1     & 0     & 0     & 1     \\
		1     & 1     & 1     & 0
	\end{tabular}
\end{center}
Per trovare il circuito desiderato c'è una procedura standard, la quale utilizza il fatto che un
\verb|AND| logico corrisponde al prodotto tra due numeri mentre l'\verb|OR| logico corrisponde alla
somma:
\begin{enumerate}
	\item Per ogni riga in cui una delle colonne d'uscita presenta almeno un 1 mettiamo in
	      \verb|AND| gli ingressi, negandoli se uguali a 0.
	\item Per ogni colonna si mettono in \verb|OR| tutti i risultati ottenuti al passo precedente.
\end{enumerate}
Nel nostro caso la colonna $z_0$ ha un 1 sull'ultima riga e i relativi valori di $x_0$ e $x_1$ sono
entrambi 1 quindi possiamo dire che
\[ z_0 = x_0 \cdot x_1 \]
Per quanto riguarda invece la colonna $z_1$ abbiamo due 1 e in corrispondenza della seconda e terza
riga. Ma in entrambi i casi uno dei due valori in ingresso è 0 e l'altro è 1 e dunque il risultato
finale è
\[ z_1 = \bar{x_0} \cdot x_1 + x_0 \cdot \bar{x_1} \]
Il circuito logico che ne deriva è il seguente
\begin{center}
	\begin{circuitikz}
		% gate
		\node[and port] (and1) at (3.5, -1) {};
		\node[and port] (and2) at (3.5, -2.5) {};
		\node[and port] (and3) at (3.5, -4) {};
		\node[or port] (or) at (5.5, -3.25) {};

		% connessioni
		\draw (0, 0) node[label=above:$x_0$] {} to[short, -*] (0, 52 |- and1.in 1) -- (and1.in 1);
		\draw (0, 52 |- and1.in 1) to[short, -*] (0, 52 |- and2.in 1) to[short, -o] (and2.in 1);
		\draw (0, 52 |- and2.in 1) to[short, -*] (0, 52 |- and3.in 1) -- (and3.in 1);

		\draw (1, 0) node[label=above:$x_1$] {} to[short, -*] (1, 52 |- and1.in 2) -- (and1.in 2);
		\draw (1, 52 |- and1.in 2) to[short, -*] (1, 52 |- and2.in 2) -- (and2.in 2);
		\draw (1, 52 |- and2.in 2) to[short, -*] (1, 52 |- and3.in 2) to[short, -o] (and3.in 2);

		\draw (and2.out) |- (or.in 1);
		\draw (and3.out) |- (or.in 2);

		\draw (and1.out) -- (6.5, 52 |- and1.out) node[label=above:$z_0$] {};
		\draw (or.out) -- (6.5, 52 |- or.out) node[label=above:$z_1$] {};
	\end{circuitikz}
\end{center}
Su tale circuito è possibile provare ad inserire vari input di $x_0$ e $x_1$ per verificarne la
correttezza.

Supponiamo ora di dover scegliere uno tra due ingressi possibili a seconda di un ingresso di
controllo regolato da un \textbf{multiplexer} che ha una forma di questo tipo
\begin{center}
	\begin{tikzpicture}
		\draw[thick] (-1.25, 1) -- (1.25, 1) -- (0.75, 0) -- (-0.75, 0) -- cycle;
		\node (mux) at (0, 0.5) {MUX};
		\draw (-0.5, 1.5) node[label=left:$x_0$] {} to[short, o-] (-0.5, 1);
		\draw (0.5, 1.5) node[label=right:$x_1$] {} to[short, o-] (0.5, 1);
		\draw (-1.75, 0.5) node[label=left:$c$] {} to[short, o-] (-1, 0.5);
		\draw (0, 0) -- (0, -0.75) node[label=right:$z$] {};
	\end{tikzpicture}
\end{center}
Di fatto dobbiamo implementare un circuito che da come risultato il valore di $x_0$ quando $c=0$ e
da come risultato il valore di $x_1$ quando $c=1$.

In questo caso abbiamo tre ingressi e un'uscita, dovremmo quindi scrivere una tabella di verità con
8 righe, ma dato che uno dei valori viene scartato a seconda del valore di $c$ il risultato è una
tabella più compatta.

Avere una tabella più compatta significa anche avere un circuito più compatto e con meno componenti.
Questo si traduce in un minor numero di nodi di calcolo e quindi una computazione più veloce, ma
anche in un minor consumo di energia e minor bisogno di spazio sul processore.
\begin{center}
	\begin{tabular}{c c c | c}
		$x_0$ & $x_1$ & $c$ & $z$ \\ \hline
		0     & -     & 0   & 0   \\
		1     & -     & 0   & 1   \\
		-     & 0     & 1   & 0   \\
		-     & 1     & 1   & 1
	\end{tabular}
\end{center}
Svolgiamo lo stesso procedimento di prima e ricaviamo un circuito di questo tipo
\begin{center}
	\begin{circuitikz}
		\node[and port] (and1) at (3.5, 0.75) {};
		\node[and port] (and2) at (3.5, -0.75) {};
		\node[or port] (or) at (5.5, 0) {};

		% connessioni
		\draw (0, 1.5) node[label=above:$x_0$] {} to[short, -*] (0, 52 |- and1.in 1) -- (and1.in 1);

		\draw (0.5, 1.5) node[label=above:$x_1$] {} to[short, -*] (0.5, 52 |- and2.in 1) -- (and2.in 1);

		\draw (1, 1.5) node[label=above:$c$] {} to[short, -*] (1, 52 |- and1.in 2) to[short, -o] (and1.in 2);
		\draw (1, 52 |- and1.in 2) to[short, -*] (1, 52 |- and2.in 2) -- (and2.in 2);

		\draw (and1.out) |- (or.in 1);
		\draw (and2.out) |- (or.in 2);

		\draw (or.out) -- (6.5, 52 |- or.out) node[label=above:$z$] {};
	\end{circuitikz}
\end{center}
che calcola esattamente
\[ z = x_0 \cdot \bar{c} + x_1 \cdot c \]
ossia il valore del canale scelto dal multiplexer.


\section{Derivazione tramite algebra booleana}
Supponiamo che per un qualche motivo otteniamo una funzione di $a$, $b$ e $c$ tale che
\[ f(a,b,c) = \bar{a} \bar{b} \bar{c} + a \bar{b} \bar{c} + a \bar{b} c \]
Se volessimo implementare questa formula tramite un circuito avremmo bisogno di tre porte
\verb|AND3| e di 1 porta \verb|OR3|. Usando però le proprietà dell'algebra booleana otteniamo
\[
	\bar{a} \bar{b} \bar{c} + a \bar{b} \bar{c} + a \bar{b} c
	= \bar{b} \bar{c} (\bar{a} + a) + a \bar{b} c
	= \bar{b} \bar{c} + a \bar{b} c
\]
Passando così ad una formula che ci permette di implementare un circuito tramite due porte
\verb|AND3| e una porta \verb|OR3|. Se procedessimo invece in questo modo
\begin{align*}
	\bar{a} \bar{b} \bar{c} + a \bar{b} \bar{c} + a \bar{b} c
	 & = \bar{a} \bar{b} \bar{c} + a \bar{b} \bar{c} + a \bar{b} \bar{c} + a \bar{b} c         \\
	 & = \bar{b} \bar{c} (\bar{a} + a) + a \bar{b} (c + \bar{c}) = \bar{b} \bar{c} + a \bar{b}
\end{align*}
otterremmo la possibilità di implementare un circuito tramite due porte \verb|AND2| e una porta
\verb|OR2|. Come possiamo vedere, a seconda di come usiamo queste proprietà, è possibile diminuire
notevolmente la dimensione dei circuiti e quindi la complessità di ciò che stiamo calcolando.
\begin{center}
	\begin{circuitikz}
		% gates
		\node[and port] (and1) at (3.5, 1) {};
		\node[and port] (and2) at (3.5, -1) {};
		\node[or port] (or) at (5.5, 0) {};

		\draw (0, 2) node[label=above:$a$] {} to[short, -*] (0, 52 |- and2.in 1) -- (and2.in 1);
		\draw (0.5, 2) node[label=above:$b$] {} to[short, -*] (0.5, 52 |- and1.in 1) to[short, -o] (and1.in 1);
		\draw (0.5, 52 |- and1.in 1) to[short, -*] (0.5, 52 |- and2.in 2) to[short, -o] (and2.in 2);
		\draw (1, 2) node[label=above:$c$] {} to[short, -*] (1, 52 |- and1.in 2) to[short, -o] (and1.in 2);

		\draw (and1.out) |- (or.in 1);
		\draw (and2.out) |- (or.in 2);
	\end{circuitikz}
\end{center}
A questo punto sarebbe possibile semplificare ulteriormente la formula raccogliendo $\bar{b}$ e
implementando il circuito descritto da
\[ \bar{b} \cdot (\bar{c} + a) \]
ma questo introduce un problema in quanto il circuito generato è asimmetrico e dunque i segnali in
ingresso non attraversano tutti lo stesso numero di porte come possiamo vedere in figura
\begin{center}
	\begin{circuitikz}
		\node[or port] (or) at (3.5, 1) {};
		\node[and port] (and) at (5.5, 0) {};

		\draw (0, 2) node[label=above:$a$] {} to[short, -*] (0, 52 |- or.in 1) -- (or.in 1);
		\draw (0.5, 2) node[label=above:$b$] {} to[short, -*] (0.5, 52 |- and.in 2) to[short, -o] (and.in 2);
		\draw (1, 2) node[label=above:$c$] {} to[short, -*] (1, 52 |- or.in 2) to[short, -o] (or.in 2);

		\draw (or.out) |- (and.in 1);
	\end{circuitikz}
\end{center}
Questo si traduce in un intervallo di tempo in cui la porta \verb|AND| riceve, da una parte il
vecchio segnale trasmesso dalla porta \verb|OR| prodotto al calcolo precedente, dall'altra l'ultimo
segnale prodotto dall'ingresso $b$.

Fino a che la porta \verb|OR| non finisce di elaborare i segnali in arrivo da $a$ e $c$ la porta
\verb|AND| potrebbe produrre risultati errati, dovuti a quello che viene chiamato \textbf{glitch}.
\section{Mappe di Karnaugh}
Come abbiamo appena visto, non sempre ridurre la complessità della nostra formula in modo
\emph{monotòno} ci porta alla migliore ottimizzazione. A volte conviene aumentare la complessità
per poi giungere ad un modello migliore.

Le \textbf{mappe di Karnaugh} forniscono un metodo grafico per riuscire a semplificare le formule
booleane senza però garantire la miglior minimizzazione di quest'ultime. Nell'esempio di prima
abbiamo una funzione booleana con la seguente tabella di verità
\begin{center}
	\begin{tabular}{c c c | c}
		$a$ & $b$ & $c$ & $f(a,b,c)$ \\ \hline
		0   & 0   & 0   & 1          \\
		0   & 0   & 1   & 0          \\
		0   & 1   & 0   & 0          \\
		0   & 1   & 1   & 0          \\
		1   & 0   & 0   & 1          \\
		1   & 0   & 1   & 1          \\
		1   & 1   & 0   & 0          \\
		1   & 1   & 1   & 0
	\end{tabular}
\end{center}
Da questa tabella possiamo ricavare una mappa di Karnaugh prendendo tutti i possibili valori di $a$
e mettendoli nella prima colonna e poi prendendo tutti i possibili valori della coppia $bc$ e
mettendoli sulla prima riga, disponendoli in modo che ogni valore differisca dal precedente al più
di un bit.

Il nostro obbiettivo è quello di individuare gli \textbf{implicanti}, ossia quei quadrati o
rettangoli contenenti un numero di 1 pari ad una potenza di 2 e raggrupparli. Per tale
raggruppamento è possibile
\begin{itemize}
	\item Uscire dalla tabella e rientrare dall'altra parte se ho degli 1 agli estremi.
	\item Includere degli 1 già raccolti in un precedente raggruppamento.
\end{itemize}
Nel nostro caso abbiamo due rettangoli da due 1:
\begin{center}
\begin{karnaugh-map}[4][2][1][$c$][$b$][$a$]
\maxterms{1, 2, 3, 6, 7}
\minterms{0, 4, 5}
\implicant{0}{4}
\implicant{4}{5}
\end{karnaugh-map}
\end{center}
A questo punto siamo
in grado di semplificare la formula di partenza
\begin{enumerate}
	\item Mettendo in \verb|AND| le variabili facenti parte dello stesso raggruppamento che
	      rimangono costanti e negando quelle con valore 0.
	\item Sommando tra di loro i raggruppamenti.
\end{enumerate}
Otteniamo così la formula ottenuta in precedenza con le proprietà dell'algebra booleana
\[ \bar{b} \bar{c} + a \bar{b} \]
in modo meccanico. Il primo termine della somma è ottenuto prendendo in considerazione il
raggruppamento verticale di 1 e considerando che $b$ e $c$ non variano ed essendo a 0 vengono
negati. Il secondo termine si ottiene similmente notando che $a$ e $b$ sono la parte costante del
raggruppamento ed inoltre $b$ è a 0 e dunque deve essere negato.

Prendiamo ora come esempio un \textbf{addizionatore} di 2 bit con riporto, il cui funzionamento
dipende da tre parametri di ingresso: $x_1$ e $x_2$, i bit che vogliamo sommare, e $r_0$, il
possibile riporto da aggiungere. Abbiamo inoltre due uscite: il risultato $s$ della somma e il
possibile riporto $r_1$ generato da essa.
\begin{center}
	\begin{tikzpicture}
		\draw[thick] (0, 0) rectangle (2, 1.5);
		\node (add) at (1, 0.75) {ADD};

		\draw (0.5, 2) node[label=above:$x_1$] {} to[short, o-] (0.5, 1.5);
		\draw (1.5, 2) node[label=above:$x_2$] {} to[short, o-] (1.5, 1.5);
		\draw (2.5, 0.75) node[label=right:$r_0$] {} to[short, o-] (2, 0.75);
		\draw (0, 0.75) -- (-0.5, 0.75) -- (-0.5, -0.75) node[label=below:$r_1$] {};
		\draw (1, 0) -- (1, -0.75) node[label=below:$s$] {};
	\end{tikzpicture}
\end{center}
Le mappe di Karnaugh per $s$ ed $r_1$ risultano le seguenti
\begin{center}
\begin{figure}[h!] \centering
\begin{subfigure}[b]{0.4\textwidth}
\begin{karnaugh-map}[4][2][1][$c$][$b$][$a$]
\minterms{1,2,4,7}
\maxterms{0,3,5,6}
\implicant{1}{1}
\implicant{2}{2}
\implicant{4}{4}
\implicant{7}{7}
\end{karnaugh-map}
\end{subfigure}
\begin{subfigure}[b]{0.4\textwidth}
\begin{karnaugh-map}[4][2][1][$c$][$b$][$a$]
\minterms{3,5,7,6}
\maxterms{0,1,2,4}
\implicant{3}{7}
\implicant{7}{6}
\implicant{5}{6}
\end{karnaugh-map}
\end{subfigure}
\end{figure}
\end{center}
Da tali mappe di Karnaugh ricaviamo le seguenti formule per $s$ ed $r_1$
\begin{align*}
	s   & = r_0 \bar{x_1} \bar{x_2} + \bar{r_0} \bar{x_1} x_2 + r_0 x_1 x_2 + \bar{r_0} x_1 \bar{x_2} \\
	r_1 & = x_1 x_2 + r_0 x_2 + r_0 x_1
\end{align*}
Di seguito raffiguriamo il circuito ricavato dalla formula per $r_1$.
\begin{center}
	\begin{circuitikz}
		\node[and port] (and1) at (3.5, 1.5) {};
		\node[and port] (and2) at (3.5, 0) {};
		\node[and port] (and3) at (3.5, -1.5) {};
		\node[or port, number inputs=3] (or) at (5.5, 0) {};

		\draw (0, 2) node[label=above:$x_1$] {} to[short, -*] (0, 52 |- and1.in 1) -- (and1.in 1);
		\draw (0, 52 |- and1.in 1) to[short, -*] (0, 52 |- and3.in 2) -- (and3.in 2);

		\draw (0.5, 2) node[label=above:$x_2$] {} to[short, -*] (0.5, 52 |- and1.in 2) -- (and1.in 2);
		\draw (0.5, 52 |- and1.in 2) to[short, -*] (0.5, 52 |- and2.in 2) -- (and2.in 2);

		\draw (1, 2) node[label=above:$r_0$] {} to[short, -*] (1, 52 |- and2.in 1) -- (and2.in 1);
		\draw (1, 52 |- and2.in 1) to[short, -*] (1, 52 |- and3.in 1) -- (and3.in 1);

		\draw (and1.out) -- (or.in 1);
		\draw (and2.out) -- (or.in 2);
		\draw (and3.out) -- (or.in 3);
		\draw (or.out) --++ (0.5, 0) node[label=right:$r_1$] {};
	\end{circuitikz}
\end{center}

\begin{tcolorbox}
	Possiamo quindi usare sia le regole e gli assiomi dell'algebra booleana per semplificare le
	formule ma questo potrebbe portarci sia alla minima forma possibile sia ad un'espressione più
	complessa. Con le mappe di Karnaugh non abbiamo la certezza di ottenere la miglior minimizzazione
	ma ci offrono un modo meccanico per ridurre la complessità.
\end{tcolorbox}

Vogliamo ora implementare un \textbf{moltiplicatore} che moltiplica due sequenze da 2 bit dando
come risultato una sequenza da 4 bit. Per capire come calcolare tale sequenza possiamo procedere
tramite una tabella di verità che però, avendo 4 bit di ingresso risulta avere 16 righe. Cerchiamo
quindi di rappresentare solo le righe significative.
\begin{center}
	\begin{tabular}{c c c c | c c c c}
		$x_1$ & $x_2$ & $y_1$ & $y_2$ & $z_1$ & $z_2$ & $z_3$ & $z_4$ \\ \hline
		0     & 0     & -     & -     & 0     & 0     & 0     & 0     \\
		-     & -     & 0     & 0     & 0     & 0     & 0     & 0     \\ \hline
		0     & 1     & 0     & 1     & 0     & 0     & 0     & 1     \\
		      &       & 1     & 0     & 0     & 0     & 1     & 0     \\
		      &       & 1     & 1     & 0     & 0     & 1     & 1     \\ \hline
		1     & 0     & 0     & 1     & 0     & 0     & 1     & 0     \\
		      &       & 1     & 0     & 0     & 1     & 0     & 0     \\
		      &       & 1     & 1     & 0     & 1     & 1     & 0     \\ \hline
		1     & 1     & 0     & 1     & 0     & 0     & 1     & 1     \\
		      &       & 1     & 0     & 0     & 1     & 1     & 0     \\
		      &       & 1     & 1     & 1     & 0     & 0     & 1     \\
	\end{tabular}
\end{center}
A questo punto possiamo disegnare una mappa di Karnaugh per ogni uscita $z_i$ che abbiamo.
Limitiamoci per il momento a disegnare solo quelle di $z_1$ e $z_3$.

\begin{figure}[!h]\centering
\resizebox{0.55\textwidth}{!}{
	\begin{subfigure}[b]{0.4\textwidth}
	\begin{karnaugh-map}[4][4][1][$y_2$][$y_1$][$x_2$][$x_1$]
	\minterms{15}
	\maxterms{0,1, 3, 2, 4, 5, 7, 6, 12, 13, 14, 8, 9, 11, 10}
	\implicant{15}{15}
	\end{karnaugh-map}
	\end{subfigure}
	\begin{subfigure}[b]{0.4\textwidth}
	\begin{karnaugh-map}[4][4][1][$y_2$][$y_1$][$x_2$][$x_1$]
	\minterms{6, 7, 15, 13, 9, 10}
	\maxterms{0, 1, 3, 2, 4, 5, 12, 14, 8, 11}
	\implicant{7}{15}
	\implicant{7}{6}
	\implicant{13}{9}
	\implicant{10}{10}
	\end{karnaugh-map}
	\end{subfigure}
}
\end{figure}

Come possiamo vedere anche dalle formule che seguono, per il calcolo di $z_1$ è sufficiente una
porta \verb|AND| mentre per il calcolo di $z_3$ sono necessarie 4 porte \verb|AND| e 1 porta
\verb|OR|. C'è quindi un ritardo tra il calcolo di $z_1$ e $z_3$ e il ritardo complessivo è dovuto
al passaggio del calcolo da 2 livelli di porte logiche.
\begin{align*}
	z_1 & = x_1 x_2 y_1 y_2                                                                   \\
	z_3 & = x_1 \bar{y_1} y_2 + x_2 y_1 y_2 + \bar{x_1} x_2 y_1 + x_1 \bar{x_2} y_1 \bar{y_2}
\end{align*}
In alternativa, considerando che una moltiplicazione tra due sequenze di 2 bit si svolge in questo
modo
\begin{center}
	\begin{tabular}{c c c c}
		  & 1 & 1 & $\times$ \\
		  & 1 & 0 & =        \\ \hline
		  & 0 & 0 & +        \\
		1 & 1 & - & =        \\ \hline
		1 & 1 & 0
	\end{tabular}
\end{center}
possiamo notare che se il bit al moltiplicatore è 0 allora avremo tutti 0 mentre se abbiamo 1 il
risultato sarà esattamente il moltiplicando. Possiamo quindi calcolarci separatamente
$x_1 x_2 \cdot y_1$ e $x_1 x_2 \cdot y_2$ tramite due multiplexer di questo tipo
\begin{center}
	\begin{tikzpicture}
		\draw[thick] (-1.25, 1) -- (1.25, 1) -- (0.75, 0) -- (-0.75, 0) -- cycle;
		\node (mux) at (0, 0.5) {MUX};
		\draw (-0.5, 1.5) node[label=left:$x_1 x_2$] {} to[short, o-] (-0.5, 1);
		\draw (0.5, 1.5) node[label=right:$00$] {} to[short, o-] (0.5, 1);
		\draw (-1.75, 0.5) node[label=left:$y_i$] {} to[short, o-] (-1, 0.5);
		\draw [->, >=Stealth] (0, 0) -- (0, -0.625);
	\end{tikzpicture}
\end{center}
Dove $x_1 x_2$ è un ingresso da 2 bit e dove l'altro ingresso è la costante 00. Questo multiplexer
effettua esattamente la scelta di cui abbiamo parlato prima: se $y_i = 0$ dà come risultato 00, se
invece $y_i = 1$ dà come risultato $x_1 x_2$.

Per implementare un moltiplicare $2 \times 2$ dobbiamo sostanzialmente affiancare due di questi
multiplexer, uno per $y_1$ e uno per $y_2$, aggiungere degli zeri dove necessario e poi effettuare
una somma con un bit di riporto $r$ che viene messo in cima alla sequenza generata. Il circuito che
ne risulta è un qualcosa di questo tipo
\begin{center}
	\begin{tikzpicture}
		\draw[thick] (-4.25, 1) -- (-1.75, 1) -- (-2.25, 0) -- (-3.75, 0) -- cycle;
		\node (mux) at (-3, 0.5) {MUX};
		\draw (-3.5, 1.5) node[label=left:$x_1 x_2$] {} to[short, o-] (-3.5, 1);
		\draw (-2.5, 1.5) node[label=right:$00$] {} to[short, o-] (-2.5, 1);
		\draw (-4.75, 0.5) node[label=left:$y_1$] {} to[short, o-] (-4, 0.5);
		\draw (-3, 0) -- (-3, -1) node[label=above left:$z_1 z_2$] {};

		\draw[thick] (4.25, 1) -- (1.75, 1) -- (2.25, 0) -- (3.75, 0) -- cycle;
		\node (mux) at (3, 0.5) {MUX};
		\draw (2.5, 1.5) node[label=left:$x_1 x_2$] {} to[short, o-] (2.5, 1);
		\draw (3.5, 1.5) node[label=right:$00$] {} to[short, o-] (3.5, 1);
		\draw (4.75, 0.5) node[label=right:$y_2$] {} to[short, o-] (4, 0.5);
		\draw (3, 0) -- (3, -1) node[label=above right:$z_3 z_4$] {};

		\draw (-1.5, -0.5) node[label=right:$0$] {} to[short, o-] (-1.5, -1) to[short, -*] (-2.25, -1);
		\draw (-3, -1) -- (-2.25, -1) -- (-2.25, -1.75) -- (-0.5, -1.75) -- (-0.5, -2.5);

		\draw (1.5, -0.5) node[label=left:$0$] {} to[short, o-] (1.5, -1) to[short, -*] (2.25, -1);
		\draw (3, -1) -- (2.25, -1) -- (2.25, -1.75) -- (0.5, -1.75) -- (0.5, -2.5);

		\draw[thick] (-1, -2.5) rectangle (1, -3.75);
		\node (add) at (0, -3.125) {ADD};
		\draw (0, -3.75) -- (0, -4.25);
		\draw (-1, -3.125) -- (-1.5, -3.125) -- node[label=left:$r$] {} (-1.5, -4.25);
	\end{tikzpicture}
\end{center}
In questo modo, dato che, come abbiamo visto in precendenza, sia il multiplexer che l'addizionatore
sono implementati tramite un circuito a due livelli di porte logiche abbiamo in totale un circuito
costituito da quattro livelli di porte logiche.

Questo si traduce in un maggior numero di componenti e in un maggior tempo di elaborazione ma in
compenso facciamo uso di due componenti standard che abbiamo già implementato e non dobbiamo
ricorrere alla costruzione di tabelle di verità, mappe di Karnaugh ecc.


\chapter{Reti sequenziali}
Le \textbf{reti sequenziali} ci servono ad implementare \textbf{macchine con stato} o
\textbf{automi}, i quali hanno bisogno di una componente di \textbf{memoria} che ci permetta di
salvare per l'appunto un certo \textbf{stato}.

Per riuscire ad implementare un automa abbiamo prima bisogno di un componente in grado di
memorizzare lo stato e questo è detto \textbf{registro}.

\section{Registri}
Per implementare un registro in grado di salvare lo stato di uno o più bit abbiamo prima bisogno
di implementare altre componenti.

\subsection{Latch SR}
Il primo oggetto di cui andiamo a parlare è chiamato \textbf{latch SR} dove SR sta per \emph{Set} e
\emph{Reset} ed è implementato in questo modo
\begin{center}
	\begin{circuitikz}
		\node[nor port] (or1) at (0, 1) {};
		\node[nor port] (or2) at (0, -1) {};

		\draw (-2.5, 52 |- or1.in 1) node[label=left:$R$] {} -- (or1.in 1);
		\draw (-2.5, 52 |- or2.in 2) node[label=left:$S$] {} -- (or2.in 2);

		\draw (or2.in 1) --++ (0, 0.5) -- (0.5, 0.5) |- (or1.out);
		\draw (or1.in 2) --++ (0, -0.5) -- (0.5, -0.5) |- (or2.out);

		\draw (0.5, 52 |- or1.out) to[short, *-] (1.5, 52 |- or1.out) node[label=right:$Q$]{};
		\draw (0.5, 52 |- or2.out) to[short, *-] (1.5, 52 |- or2.out) node[label=right:$\bar{Q}$]{};
	\end{circuitikz}
\end{center}
Quest'implementazione è pensata per avere l'uscita $Q$ a 1 quando l'ingresso $S$ è messo a 1 e,
finché non si imposta l'ingresso $R$ a 1 l'uscita $Q$ dovrebbe rimanere a 1. In questo modo, una
volta che impostiamo $S = 1$, l'uscita $Q$ rimane 1 anche se cambiamo l'ingresso $S$ a 0. Questo
metodo di memorizzazione presenta due problemi:
\begin{enumerate}
	\item Necessità di effettuare "manualmente" un reset ogni volta che vogliamo cancellare il
	      contenuto del registro.
	\item Ambiguità nel caso $S$ ed $R$ vadano nello stesso momento a 1: in questo caso si vuole
	      sia settare il bit a 1 sia resettarlo a 0. Come risultato otterremo entrambe le uscite a
	      0 in quanto entrambi gli ingressi a 1 fanno uscire 0 dalle due porte \verb|NOR|.
\end{enumerate}
Dato che $Q$ e $\bar{Q}$ dovrebbero essere l'uno l'opposto dell'altro e che si crea questa
situazione di ambiguità si è passati al D latch.

\subsection{D latch}
Il \textbf{D latch} introduce un \textbf{segnale di clock} costante e riduce il numero di ingressi
significativi ad uno, ossia $D$.

L'idea è avere un qualcosa che memorizza il valore di $D$ ogni volta che questo cambia. Quello che
succede è che, ad ogni ciclo di clock, all'interno del latch SR si effettua un reset e si salva il
valore di $S$ nel circuito.
\begin{center}
	\begin{circuitikz}
		\node[and port] (and1) at (0, 1) {};
		\node[and port] (and2) at (0, -1) {};
		\draw[thick] (1.5, -0.75) rectangle (3, 0.75);
		\node (latchSR) at (2.25, 0) {latch SR};

		\draw[dashed] (-3, 52 |- and1.in 1) node[label=left:clock] {} -- (and1.in 1);
		\draw[dashed] (-2.5, 52 |- and1.in 1) to[short, *-] (-2.5, 52 |- and2.in 1) -- (and2.in 1);

		\draw (-3, 52 |- and2.in 2) node[label=left:$D$] {} -- (and2.in 2);
		\draw (-2, 52 |- and2.in 2) to[short, *-] (-2, 52 |- and1.in 2) to[short, -o] (and1.in 2);

		\draw (and1.out) -- (1.5, 0.5) node[label=above left:$R$]{};
		\draw (and2.out) -- (1.5, -0.5) node[label=below left:$S$]{};

		\draw (3, 0.5) -- (4, 0.5) node[label=right:$Q$] {};
		\draw (3, -0.5) -- (4, -0.5) node[label=right:$\bar{Q}$] {};
	\end{circuitikz}
\end{center}
Anche questo approccio presenta alcune criticità, la più importante è che dobbiamo sempre mantenere
un $D$ significativo, poiché, se per una qualche ragione il valore di $D$ cambia inaspettatamente,
al ciclo di clock successivo verrà scritto il nuovo valore e perderemmo quindi il valore di $D$
memorizzato in precedenza.

Vogliamo quindi avere un qualcosa che ci permetta di decidere quando scrivere sia per evitare di
riscrivere continuamente lo stesso valore anche quando questo non cambia, sia per evitare di avere
costantemente un $D$ significativo in ingresso.

\subsection{Flip Flop}
Prima di ottenere il circuito sperato introduiciamo brevemente il circuito \textbf{Flip Flop}, il
quale, tramite due D latch concatenati (un \emph{master} e uno \emph{slave}), evita transizioni di
stato indesiderate salvando il dato significativo nel D latch slave.
\begin{center}
	\begin{tikzpicture}
		\draw[thick] (-2, 0) rectangle (-0.5, 1.5);
		\draw[thick] (1, 0) rectangle (2.5, 1.5);
		\node (master) at (-1.25, 0.75) {Master};
		\node (slave) at (1.75, 0.75) {Slave};

		\draw[dashed] (-3, 2) node[label=left:clock] {} -- (-1.25, 2) to[short, -o] (-1.25, 1.5);
		\draw[dashed] (-1.25, 2) to[short, *-] (1.75, 2) -- (1.75, 1.5);

		\draw (-3, 0.75) node[label=left:$D$] {} -- (-2, 0.75);

		\draw (-0.5, 1) -- (1, 1) node[label=above left:$Q$] {};
		\draw (-0.5, 0.5) to[short, -*] (0.25, 0.5) node[label=below left:$\bar{Q}$] {};

		\draw (2.5, 1) -- (3.5, 1) node[label=above left:$Q$] {};
		\draw (2.5, 0.5) -- (3.5, 0.5) node[label=below left:$\bar{Q}$] {};
	\end{tikzpicture}
\end{center}

\subsection{Registro da 1 bit}
Possiamo finalmente implementare, sulla base di quest'ultimo componente il nostro \textbf{registro}
da 1 bit introducendo un segnale di \textbf{enable} che regola le tempistiche di scrittura quando
necessario.

L'implementazione prevede semplicemente un \verb|AND| tra il segnale di clock in ingresso nel flip
flop e il segnale di \verb|ENABLE|. Quando tale segnale è a 1 la scrittura è abilitata e dunque al
primo ciclo di clock utile si scriverà il valore $D$ nel registro, quando invece il segnale va a 0
si disabilita la scrittura.
\begin{center}
	\begin{circuitikz}
		\draw[thick] (0, 0) rectangle (1.75, 1.5);
		\node (flip flop) at (0.875, 0.75) {Flip Flop};
		\node[and port] (and) at (-1.5, 1.25) {};

		\draw[dashed] (-4.5, 52 |- and.in 1) node[label=left:clock] {} -- (and.in 1);
		\draw (-4.5, 52 |- and.in 2) node[label=left:EN] {} -- (and.in 2);

		\draw (and.out) -- (0, 1.25);
		\draw (-1.5, 0.25) node[label=left:$D$] {} -- (0, 0.25);
		\draw (1.75, 0.75) --++ (1.5, 0) node[label=right:$Q$] {};
	\end{circuitikz}
\end{center}
Per effettuare un registro da 2 (o più) bit è sufficiente accostare due (o più) registri di questo
tipo dando in ingresso ad ognuno di essi il segnale il bit da salvare e condividendo i segnali di
clock ed \verb|ENABLE|.
\begin{center}
	\begin{tikzpicture}
		\draw[thick] (-2, 0) rectangle (-1, 1);
		\node (r1) at (-1.5, 0.5) {R};
		\draw[thick] (1, 0) rectangle (2, 1);
		\node (r2) at (1.5, 0.5) {R};

		\draw (-1.5, 1.5) -- (-1.5, 1);
		\draw (1.5, 1.5) -- (1.5, 1);

		\draw (-1.5, 0) -- (-1.5, -0.5) -- (0, -1) -- (0, -1.75) node[label=right:$Q$] {};
		\draw (1.5, 0) -- (1.5, -0.5) -- (0, -1);

		\draw[dashed] (-0.5, 1.5) node[label=above:clock] {} -- (-0.5, 0.75) -- (-1, 0.75);
		\draw[dashed] (-0.5, 0.75) to[short, *-] (1, 0.75);

		\draw (0.5, 1.5) node[label=above:EN] {} -- (0.5, 0.25) -- (1, 0.25);
		\draw (0.5, 0.25) to[short, *-] (-1, 0.25);
	\end{tikzpicture}
\end{center}
Per registri da 32 o 64 bit in realtà non si accostando 32 o 64 registri da un bit ma si procede in
un altro modo risparmiando molte componenti.


\section{Automi}
Questi registri ci permettono di implementare degli \textbf{automi} che ad esempio ci permettono
di riconoscere sottostringhe di un certo tipo. Se ad esempio avessimo un vocabolario $\{a, b, c\}$
e volessimo implementare un automa in grado di riconoscere la presenza di sottosequenze del tipo
$abb$ avremmo un qualcosa di questo tipo
\begin{center}
	\begin{tikzpicture}[
			->, >=Stealth,
			node distance=3cm,
			main node/.style={circle, draw, thick, font=\Large}
		]
		\node[state, main node] (init) {Init};
		\node[state, main node] (A) [below left of=init] {A};
		\node[state, main node] (AB) [below right of=init] {AB};

		\draw
		(init) edge node[->, below right, black] {a/0} (A)
		(init) edge[loop above] node{b,c/0} (init)
		(A) edge[above right] node{b/0} (AB)
		(A) edge[loop left] node{a/0} (A)
		(A) edge[bend left] node[above left, black] {c/0} (init)
		(AB) edge[bend right] node[above right, black] {b/1, c/0} (init)
		(AB) edge[bend left] node[below, black] {a/0} (A);
	\end{tikzpicture}
\end{center}
Quello che ora vorremmo essere in grado di fare è implementarlo su un circuito. Per farlo dobbiamo
soddisfare 4 requisiti fondamentali, i quali prevedono la definizione di
\begin{enumerate}
	\item Una rappresentazione delle stringhe in ingresso.
	\item Una rappresentazione per gli stati.
	\item Una funzione che, dato uno stato e un valore (un carattere nel nostro caso), restituisce
	      un valore d'uscita significativo.
	\item Una funzione che, dato uno stato e un valore, restituisce il prossimo stato in cui
	      passare.
\end{enumerate}
Iniziamo con il fornire una rappresentazione delle stringhe in ingresso
\begin{center}
	\begin{tabular}{c | c}
		a & 00 \\ \hline
		b & 01 \\ \hline
		c & 11
	\end{tabular}
\end{center}
Similmente la rappresentazione dei vari stati sarà
\begin{center}
	\begin{tabular}{c | c}
		Init & 00 \\ \hline
		A    & 01 \\ \hline
		AB   & 11
	\end{tabular}
\end{center}
Per definire la funzione in grado di dirci se abbiamo riconosciuto la sequenza oppure no abbiamo
bisogno di di due bit di stato $S_1$ ed $S_0$ e poi di due bit che rappresentano gli ingressi $I_1$
e $I_0$. Non stiamo a scrivere la tabella di verità deducibile dallo schema dell'automa stesso
ma alla fine ricaviamo che l'uscita $Z$ è definita da
\[ Z = S_1 \cdot S_0 \cdot \bar{I}_1 \cdot I_0 \]
In ultima battuta definiamo la funzione che ci permette di dedurre il prossimo stato in cui ci
troveremo basandosi sullo stato attuale e sugli ingressi.
\begin{align*}
	S_1' = & \bar{S}_1 S_0 	\bar{I}_1 I_0                                                    \\
	S_0' = & \bar{S}_1 		\bar{S}_0 \bar{I}_1 \bar{I}_0 + \bar{S}_1  S_0 \bar{I}_1 \bar{I}_0 +
	\bar{S}_1  S_0 \bar{I}_1 I_0 + S_1 S_0 \bar{I}_1 \bar{I}_0
\end{align*}
Non ci rimane che implementare il nostro automa tramite un circuito composto da due moduli che
calcolano rispettivamente $Z$ ed $S'$ e da un registro in grado di memorizzare 2 bit di stato.
\begin{center}
	\begin{tikzpicture}[->, >=Stealth]
		\draw[thick] (-2.5, 0) rectangle (-1, 1) (-1.75, 0.5) node{$S'$};
		\draw[thick] (0.5, 0) rectangle (1.5, 1) (1, 0.5) node{R2};
		\draw[thick] (3, 0) rectangle (4.5, 1) (3.75, 0.5) node{$Z$};

		\draw (-4, 0.5) node[label=left:$I$] {} -- (-2.5, 0.5);
		\draw (-3.25, 0.5) to[short, *-] (-3.25, 1.5) -- (3.75, 1.5) -- (3.75, 1);

		\draw (-1, 0.5) -- (0.5, 0.5);
		\draw (1.5, 0.5) -- (3, 0.5);
		\draw (2.25, 0.5) to[short, *-] (2.25, -0.5) -- (-1.75, -0.5) -- (-1.75, 0);
		\draw (4.5, 0.5) -- (5.5, 0.5);
		\draw (1.25, -1) node[label=right:EN] {} -- (1.25, 0);
		\draw[dashed] (0.75, -1) node[label=left:clock] {} -- (0.75, 0);
	\end{tikzpicture}
\end{center}
Questa che vediamo è detta \textbf{rete sequenziale di Mealy} in quanto implementa un automa di
Mealy. Se non avessimo la diramazione dell'ingresso $I$ che va in $Z$ avremmo parlato di
\textbf{rete sequenziale di Moore}, la quale implementa un automa di Moore che è definito in
maniera leggermente diversa.

\subsection{Sincronizzazione}
Un altro tipo di rete non prevede l'utilizzo del registro R2 ma così fancendo si rende instabile
tutto il circuito per motivi non di nostro interesse. In realtà anche la versione sopra descritta
dovrebbe essere resa stabile tramite un \textbf{sincronizzatore} che può essere sia un componente
specializzato sia un registro come quelli che abbiamo già descritto.
\begin{center}
	\begin{tikzpicture}[->, >=Stealth]
		\draw[thick] (-5.5, 0) rectangle (-4, 1) (-4.75, 0.5) node{Synch};
		\draw[thick] (-2.5, 0) rectangle (-1, 1) (-1.75, 0.5) node{$S'$};
		\draw[thick] (0.5, 0) rectangle (1.5, 1) (1, 0.5) node{R2};
		\draw[thick] (3, 0) rectangle (4.5, 1) (3.75, 0.5) node{$Z$};

		\draw (-6.5, 0.5) node[label=left:$I_1$] {} -- (-5.5, 0.5);
		\draw (-4, 0.5) -- (-2.5, 0.5);
		\node[label=below:$I_2$] at (-3.25, 0.5) {};
		\draw (-3.25, 0.5) to[short, *-] (-3.25, 1.5) -- (3.75, 1.5) -- (3.75, 1);

		\draw (-1, 0.5) -- (0.5, 0.5);
		\draw (1.5, 0.5) -- (3, 0.5);
		\draw (2.25, 0.5) to[short, *-] (2.25, -0.5) -- (-1.75, -0.5) -- (-1.75, 0);
		\draw (4.5, 0.5) -- (5.5, 0.5);
		\draw (1.25, -1) node[label=right:EN] {} -- (1.25, 0);
		\draw[dashed] (-4.75, -1) to[short, *-] (-4.75, 0);
		\draw[dashed] (-6, -1) node[label=left:clock] {} -- (0.75, -1) -- (0.75, 0);
	\end{tikzpicture}
\end{center}
Quello che vogliamo è che il valore degli ingressi sia stabile sul \emph{fronte di salita} del
ciclo di clock. Il blocco Synch serve proprio a stabilizzare e sincronizzare gli ingressi variandoli solo
quando il clock va alto.


\chapter{Componenti standard}
Andiamo ora a definire alcune delle componenti, con relativa implementazione, che andremo ad
utilizzare molto spesso d'ora in poi e che considereremo \textbf{componenti standard}.

Possiamo dividere tali componenti in \textbf{componenti di reti combinatorie} e
\textbf{componenti di reti sequenziali}.

\section{Componenti di reti combinatorie}
Le componenti che andremo a vedere sono le componenti principali per lo svolgimento dei calcoli
aritmetici e per la valutazione di espressioni booleane.

\subsection{Addizionatore}
La prima componente che andiamo a vedere è l'addizionatore di due sequenza da $n$ bit ciascuna che
restituisce una sequenza da $n$ bit e un riporto. Il componente lo indicheremo d'ora in poi in
questo modo
\begin{center}
	\begin{tikzpicture}[scale=0.75]
		\draw[thick] (-1, 0) -- (-1.5, 1.5) -- (-0.2, 1.5) --
		(0, 1.25) -- (0.2, 1.5) -- (1.5, 1.5) -- (1, 0) -- cycle;
		\node (add) at (0, 0.75) {+};

		\draw (-1, 2) node[label=above:$x$] {} to[short, o-] (-1, 1.5);
		\draw (1, 2) node[label=above:$y$] {} to[short, o-] (1, 1.5);
		\draw (2, 0.75) node[label=right:$r_\text{in}$] {} to[short, o-] (1.25, 0.75);
		\draw[->, >=Stealth] (-1.25, 0.75) -- (-2, 0.75) --
		(-2, -0.75) node[label=below:$r_\text{out}$] {};
		\draw[->, >=Stealth] (0, 0) -- (0, -0.75) node[label=below:$s$] {};
	\end{tikzpicture}
\end{center}
Per implementare un addizionatore di due sequenze da 2 bit ciascuna, possiamo concatenare due
addizionatori da 1 bit. Questo però implica che ogni addizionatore inizia il calcolo quando quello
che ha calcolato la cifra meno significativa ha finito il suo calcolo e generato un prodotto.

Dato che ogni addizionatore da 1 bit deve attraversare due livelli di porte logiche, se
concantenassimo $n$ addizionatori da 1 bit dovrebbe attendere $2 \cdot n \cdot \Delta t$ per
effettuare una somma a $n$ bit.

Si prova quindi a calcolare il riporto in anticipo con una tecnica chiamata
\textbf{Carry look ahead}. Questa tecnica tiene di conto che una somma tra due bit $x$ e $y$
\begin{itemize}
	\item \textbf{Genera} un riporto se e solo $x = y = 1$.
	\item \textbf{Propaga} un riporto se e solo se $x = 1$ o $y = 1$.
\end{itemize}
Vale quindi che, data una certa colonna della somma, abbiamo un riporto se generiamo o propaghiamo
un riporto. Da questa considerazione ricaviamo la formula ricorsiva
\[ C_{i+1} = G_i + P_i \cdot C_i \]
dove $G_i = x_i \cdot y_i$ e $P_i = x_i + y_i$, che ci permette di calcolare il riporto $C$
generato da una somma di due sequenze di $i+1$ bit.

Così facendo abbiamo un circuito con $k$ porte in cascata per ogni sottosequenza di bit e a seconda
di quanto è lunga la sequenza principale e di quante sottosequenze abbiamo generato si dovrebbe
ottenere una leggera ottimizzazione in termini di livelli di porte da attraversare.

In questo modo il calcolo del riporto è indipendente dal calcolo della somma e dunque il calcolo
della somma successiva può iniziare in anticipo.

\subsection{Comparatore}
Componente utile per capire se due sequenza di bit $x$ e $y$ sono uguali oppure no. Per la
comparazione di due bit singoli abbiamo già la porta \verb|XNOR| che ritorna 1 se e solo se i due
bit hanno lo stesso valore. Un circuito in grado di comparare due sequenze di $n$ bit sarebbe
\begin{center}
	\begin{circuitikz}
		\node[xnor port] (xnor1) at (-0.5, 1.25) {};
		\node[xnor port] (xnor2) at (-0.5, -1.25) {};
		\node[and port] (and) at (2, 0) {};

		\draw (-3, 52 |- xnor1.in 1) node[label=left:$x_0$] {} -- (xnor1.in 1);
		\draw (-3, 52 |- xnor1.in 2) node[label=left:$y_0$] {} -- (xnor1.in 2);

		\node at (-1.5, 0) {$\vdots$};
		\node at (-3.5, 0) {$\vdots$};

		\draw (-3, 52 |- xnor2.in 1) node[label=left:$x_{n-1}$] {} -- (xnor2.in 1);
		\draw (-3, 52 |- xnor2.in 2) node[label=left:$y_{n-1}$] {} -- (xnor2.in 2);

		\draw (xnor1.out) -- (and.in 1);
		\draw (xnor2.out) -- (and.in 2);
		\draw (and.out) --++ (0.75, 0) node[label=right:$z$] {};
	\end{circuitikz}
\end{center}
D'ora in poi sarà rappresentato in questo modo
\begin{center}
	\begin{tikzpicture}
		\draw[thick] (-0.75, -0.5) rectangle(0.75, 0.5);
		\node (eq) at (0, 0) {=};

		\draw (-0.5, 1) node[label=above:$x$] {} to[short, o-] (-0.5, 0.5);
		\draw (0.5,  1) node[label=above:$y$] {} to[short, o-] (0.5, 0.5);
		\draw[->, >=Stealth] (0, -0.5) -- (0, -1) node[label=below:$z$] {};
	\end{tikzpicture}
\end{center}
Dove $x$ e $y$ sono entrambe due sequenze da $n$ bit e $z$ vale 0 se $x \neq y$ e vale 1 se $x = y$.

\subsection{ALU}
Altro componente fondamentale è la \textbf{ALU}, la quale ci permette di effettuare ben quattro
operazioni tra due sequenze di $n$ bit $x$ e $y$: \verb|AND|, \verb|OR|, somma e sottrazione.

Le prime tre operazioni hanno già un circuito che le implementa e che conosciamo, per la
sottrazione facciamo prima una considerazione.

Come abbiamo già anticipato, se abbiamo un numero positivo rappresentato in complemento a 2, per
farlo diventare negativo dobbiamo negarlo e sommarci 1. Di conseguenza vale che
\[ x - y = x + (\bar{y} + 1) \]
Siamo quindi in grado di utilizzare un addizionatore per svolgere anche le sottrazioni e dunque non
dobbiamo implementare un circuito apposito.
\begin{center}
	\begin{circuitikz}
		\draw[thick] (2, -3.25) -- (2, -1.25) -- (2.75, -1.75) -- (2.75, -2.75) -- cycle; % MUX1

		% ADD
		\draw[thick] (4, -2) -- (4, -1.25) -- (4.25, -1) -- (4, -0.75) --
		(4, 0) -- (5, -0.5) -- (5, -1.5) -- cycle;
		\node at (4.625, -1) {+};

		\node[or port] (or) at (5, 2.5) {}; % OR
		\node[and port] (and) at (5, 1) {}; % AND
		\draw[thick] (7, -1.5) -- (7, 1.5) -- (8, 1) -- (8, -1) -- cycle; % MUX2

		% OP
		\draw[->, >=Stealth]  (0, -3.75)  node[label=left:OP] {} to[short, -*] (2.375, -3.75) --
		(2.375, -3);
		\draw[->, >=Stealth] (2.375, -3.75) to[short, -*] (4.625, -3.75) -- (4.625, -1.75);
		\draw[->, >=Stealth] (4.625, -3.75) -- (7.5, -3.75) -- (7.5, -1.25);

		% X
		\draw[->, >=Stealth] (0, 1) node[label=left:$x$] {} to[short, -*] (2, 1) --
		(2, -0.375) -- (4, -0.375);
		\draw (2, 1) -- (2, 52 |- or.in 1) -- (or.in 1);
		\draw (2, 52 |- and.in 1) to[short, *-] (and.in 1);

		% Y
		\draw[->, >=Stealth] (0, -1) node[label=left:$y$] {} to[short, -*]
		(1.25, -1) -- (1.25, -2.75) to[short, -o] (1.625, -2.75) -- (2, -2.75);
		\draw[->, >=Stealth] (1.25, -1.75) to[short, *-] (2, -1.75);
		\draw (1.25, -1) -- (1.25, 52 |- and.in 2) -- (and.in 2);
		\draw (1.25, 52 |- and.in 2) to[short, *-] (1.25, 52 |- or.in 2) -- (or.in 2);

		% MUX1 to ADD
		\draw[->, >=Stealth] (2.75, -2.25) -- (3.25, -2.25) -- (3.25, -1.625) -- (4, -1.625);

		% ADD exits
		\draw[->, >=Stealth] (5, -1) -- (7, -1);
		\draw[->, >=Stealth] (6, -1) to[short, *-] (6, -0.5) -- (7, -0.5);
		\draw[->, >=Stealth] (4.5, -0.25) -- (4.5, 0) -- (5, 0);


		\draw[->, >=Stealth] (and.out) -- (5.5, 52 |- and.out) -- (5.5, 0.5) -- (7, 0.5);
		\draw[->, >=Stealth] (or.out) -- (6, 52 |- or.out) -- (6, 1) -- (7, 1);
		\draw[->, >=Stealth] (8, 0) -- (8.75, 0);
	\end{circuitikz}
\end{center}
L'entrata OP è un ingresso da 2 bit che ha come funzione principale la selezione dell'operazione
tramite il multiplexer a destra. \`E inoltre usato, nel caso si voglia effettuare una sottrazione
per rendere $y$ negativo tramite il multiplexer iniziale e tramite il riporto immesso
nell'addizionatore.

Supponiamo infatti di avere come configurazione per la scelta delle operazioni della ALU quella
indicata dalla seguente tabella
\begin{center}
	\begin{tabular}{c | c}
		OP input & operazione \\ \hline
		00       & +          \\
		01       & -          \\
		10       & \verb|AND| \\
		11       & \verb|OR|
	\end{tabular}
\end{center}
e inviamo in ingresso $x = 1$, $y = 1$ e $OP = 01$. Dato che $OP = 01$, il primo multiplexer
sceglierà come uscita $\bar{y}$ e invierà all'addizionatore un riporto di 1. Come risultato avremo
che l'addizionatore sommerà $x$ e $\bar{y} + 1$ ottenendo 0. Dato che l'ultimo multiplexer è
impostato per dare come uscita la sottrazione quando $OP = 01$ avremo il risultato richiesto.

\subsection{Shift}
In precedenza abbiamo parlato di \textbf{shift logico} e \textbf{shift aritmetico}. Vediamo ora
come è possibile implementare un circuito che ci permette di shiftare a destra una sequenza di bit
di un certo numero di posizioni con una sequenza di esempio di 4 bit.
\begin{center}
	\begin{tikzpicture}
		\draw[thick] (4.25, 3.625) -- (4.25, 1.625) -- (5, 2.125) -- (5, 3.125) -- cycle;

		\draw (0, 5) node[label=above:$x_3$] {} to[short, -*] (0, 4.5) -- (2.5, 4.5);
		\draw (0.5, 5) node[label=above:$x_2$] {} to[short, -*] (0.5, 4.25) -- (2.5, 4.25);
		\draw (1, 5) node[label=above:$x_1$] {} to[short, -*] (1, 4)   -- (2.5, 4);
		\draw (1.5, 5) node[label=above:$x_0$] {} to[short, -*] (1.5, 3.75) -- (2.5, 3.75);


		\draw (2.25, 3.5) node[label=left:\footnotesize0] {} -- (2.5, 3.5);
		\draw (0, 4.5) to[short, -*] (0, 3.25) -- (2.5, 3.25);
		\draw (0.5, 4.25) to[short, -*] (0.5, 3) -- (2.5, 3);
		\draw (1, 4) to[short, -*] (1, 2.75) -- (2.5, 2.75);

		\draw (2.25, 2.5) node[label=left:\footnotesize0] {} -- (2.5, 2.5);
		\draw (2.25, 2.25) node[label=left:\footnotesize0] {} -- (2.5, 2.25);
		\draw (0, 3.25) to[short, -*] (0, 2) -- (2.5, 2);
		\draw (0.5, 3) -- (0.5, 1.75) -- (2.5, 1.75);

		\draw (2.25, 1.5) node[label=left:\footnotesize0] {} -- (2.5, 1.5);
		\draw (2.25, 1.25) node[label=left:\footnotesize0] {} -- (2.5, 1.25);
		\draw (2.25, 1) node[label=left:\footnotesize0] {} -- (2.5, 1);
		\draw (0, 2) -- (0, 0.75) -- (2.5, 0.75);

		\draw[very thick] (2.5, 4.5) -- (2.5, 3.75);
		\draw[very thick] (2.5, 3.5) -- (2.5, 2.75);
		\draw[very thick] (2.5, 2.5) -- (2.5, 1.75);
		\draw[very thick] (2.5, 1.5) -- (2.5, 0.75);

		\draw[->, >=Stealth] (2.5, 4.125) -- (3.5, 4.125) -- (3.5, 3.375) -- (4.25, 3.375);
		\draw[->, >=Stealth] (2.5, 3.125) -- (3, 3.125) -- (3, 2.875) -- (4.25, 2.875);
		\draw[->, >=Stealth] (2.5, 2.125) -- (3, 2.125) -- (3, 2.375) -- (4.25, 2.375);
		\draw[->, >=Stealth] (2.5, 1.125) -- (3.5, 1.125) -- (3.5, 1.875) -- (4.25, 1.875);
		\draw[->, >=Stealth] (4.625, 1) -- (4.625, 1.875);
		\draw[->, >=Stealth] (5, 2.625) --++ (1, 0) node[label=right:$z$] {};
	\end{tikzpicture}
\end{center}
Come è possibile notare quando si shifta a destra si aggiungono tanti 0 a sinistra quanto
necessario e si ridirezionano i bit più significativi. Per uno shift aritmetico a destra avremmo
dovuto collegare l'ingresso del bit più significativo al posto degli 0.

Se avessimo voluto effettuare degli shift a sinistra lo schema sarebbe stato uguale tranne per il
fatto che sarebbero stati i bit più significativi i primi ad andare a 0.

\subsection{Demultiplexer}
Altro componente considerato standard è il \textbf{demultiplexer} il quale ha un ingresso principale
che viene rediretto su una delle $2^k$ uscite. L'uscita su cui viene rediretto l'input è scelta
tramite un ingresso di controllo a $k$ bit, i quali codificano il numero dell'uscita in binario.
\begin{center}
	\begin{tikzpicture}
		\draw[thick] (-0.8, 1) -- (0.8, 1) -- (1.5, 0) -- (-1.5, 0) -- cycle;
		\node (mux) at (0, 0.5) {DEMUX};

		\draw (0, 1.5) node[label=above:$x$] {} to[short, o-] (0, 1);
		\draw (-2, 0.5) node[label=left:$c$] {} to[short, o-] (-1.15, 0.5);

		\draw (-1.25, 0) -- (-1.25, -0.5) node[label=below:$z_1$] {};
		\draw (-0.5, 0) -- (-0.5, -0.5) node[label=below:$z_2$] {};
		\draw (0.5, 0) -- (0.5, -0.5) node[label=below:$z_3$] {};
		\draw (1.25, 0) -- (1.25, -0.5) node[label=below:$z_4$] {};
	\end{tikzpicture}
\end{center}
e la tabella di verità corrispondente sarebbe la solita
\begin{center}
	\begin{tabular}{c c | c c c c}
		$x$ & $c$ & $z_1$ & $z_2$ & $z_3$ & $z_4$ \\ \hline
		1   & 00  & 1     & 0     & 0     & 0     \\
		1   & 01  & 0     & 1     & 0     & 0     \\
		1   & 10  & 0     & 0     & 1     & 0     \\
		1   & 11  & 0     & 0     & 0     & 1
	\end{tabular}
\end{center}
Le formule risultanti da questa tabella sono
\begin{align*}
	z_1 & = x \cdot \overline{c_1} \cdot \overline{c_2} \\
	z_2 & = x \cdot \overline{c_1} \cdot c_2            \\
	z_3 & = x \cdot c_1 \cdot \overline{c_2}            \\
	z_4 & = x \cdot c_1 \cdot c_2
\end{align*}
Il circuito risultante sarebbe
\begin{center}
	\begin{circuitikz}
		\node[and port, number inputs=3] (and1) at (0, 2.25) {};
		\node[and port, number inputs=3] (and2) at (0, 0.75) {};
		\node[and port, number inputs=3] (and3) at (0, -0.75) {};
		\node[and port, number inputs=3] (and4) at (0, -2.25) {};

		\draw (-4, 52 |- and2.in 1) node[label=left:$x$] {} to[short, o-*] (-2, 52 |- and2.in 1) --
		(-2, 52 |- and1.in 1) -- (and1.in 1)
		(-2, 52 |- and2.in 1) -- (and2.in 1)
		(-2, 52 |- and2.in 1) to[short, -*] (-2, 52 |- and3.in 1) -- (and3.in 1)
		(-2, 52 |- and3.in 1) -- (-2, 52 |- and4.in 1) -- (and4.in 1);

		\draw (-4, 52 |- and2.in 2) node[label=left:$c_1$] {} to[short, o-*] (-2.25, 52 |- and2.in 2) --
		(-2.25, 52 |- and1.in 2) to[short, -o] (and1.in 2)
		(-2.25, 52 |- and2.in 2) to[short, -o] (and2.in 2)
		(-2.25, 52 |- and2.in 2) to[short, -*] (-2.25, 52 |- and3.in 2) -- (and3.in 2)
		(-2.25, 52 |- and3.in 2) -- (-2.25, 52 |- and4.in 2) -- (and4.in 2);

		\draw (-4, 52 |- and2.in 3) node[label=left:$c_2$] {} to[short, o-*] (-2.5, 52 |- and2.in 3) --
		(-2.5, 52 |- and1.in 3) to[short, -o] (and1.in 3)
		(-2.5, 52 |- and2.in 3) -- (and2.in 3)
		(-2.5, 52 |- and2.in 3) to[short, -*] (-2.5, 52 |- and3.in 3) to[short, -o] (and3.in 3)
		(-2.5, 52 |- and3.in 3) -- (-2.5, 52 |- and4.in 3) -- (and4.in 3);

		\draw (and1.out) --++ (0.5, 0) node[label=right:$z_1$] {};
		\draw (and2.out) --++ (0.5, 0) node[label=right:$z_2$] {};
		\draw (and3.out) --++ (0.5, 0) node[label=right:$z_3$] {};
		\draw (and4.out) --++ (0.5, 0) node[label=right:$z_4$] {};
	\end{circuitikz}
\end{center}

\section{Componenti di reti sequenziali}
Le componenti standard implementate sotto forma di rete sequenziale sono di fatto registri e
memorie. In particolare andremo a parlare di
\begin{itemize}
	\item Registri e banchi di registri.
	\item Memorie: statiche, dinamiche e ROM.
\end{itemize}
Dei registri da 1 bit abbiamo già parlato in precedenza, proseguiamo con le altre componenti anche
se riprenderemo il discorso sulle memorie più avanti in modo più approfondito.

\subsection{Banco di registri}
I banchi di registri sono una prima forma di \textbf{memoria} in grado di memorizzare più
\emph{parole} le quali sono accessibili tramite un \textbf{indirizzo}. Di seguito un banco da 2
registri.
\begin{center}
	\begin{tikzpicture}
		\draw[->, >=Stealth] (0, 0) node[label=left:EN] {} -- (1, 0); % EN
		\draw[thick] (1, -0.5) -- (1, 0.5) -- (1.5, 0.75) -- (1.5, -0.75) -- cycle; % DEMUX
		\draw[thick] (2.5, 0.75) rectangle (3.5, 0.25); % R1
		\draw[thick] (2.5, -0.75) rectangle (3.5, -0.25); % R2
		\draw[thick] (4.5, -0.75) -- (4.5, 0.75) -- (5, 0.5) -- (5, -0.5) -- cycle; % MUX
		\draw[->, >=Stealth] (5, 0) -- (6, 0) node[label=right:out] {}; % OUT
		\draw[->, >=Stealth] (1.5, 0.375) -- (2.5, 0.375); % demux -> R1
		\draw[->, >=Stealth] (1.5, -0.625) -- (2.5, -0.625); % demux -> R2

		% clock
		\draw[->, >=Stealth, dashed] (1.875, 0.625) -- (2.5, 0.625);
		\draw[->, >=Stealth, dashed] (1.875, -0.375) -- (2.5, -0.375);
		\draw[dashed] (1.875, 1.5) node[label=above:clock] {} -- (1.875, -0.375);

		\draw[->, >=Stealth] (3.5, 0.5) -- (4.5, 0.5); % R1 -> MUX
		\draw[->, >=Stealth] (3.5, -0.5) -- (4.5, -0.5); % R2 -> MUX

		% INPUT
		\draw[->, >=Stealth] (3, 1.5) node[label=above:IN] {} -- (3, 0.75);
		\draw[->, >=Stealth] (3, 1.125) to[short, *-] (4, 1.125) -- (4, 0) -- (3, 0) -- (3, -0.25);

		% INDIRIZZAMENTO
		\draw[->, >=Stealth] (0, -1.25) node[label=left:IND] {} to[short, -*] (1.25, -1.25) --
		(1.25, -0.625);
		\draw[->, >=Stealth] (1.25, -1.25) -- (4.75, -1.25) -- (4.75, -0.625);
	\end{tikzpicture}
\end{center}
Come possiamo vedere, tramite il segnale di IND riusciamo a scrivere IN su uno specifico registro
scelto dal demultiplexer, lo stesso IND viene inviato al multiplexer alla fine per riuscire a
leggere la locazione di memoria indirizzata.

Questo metodo fa uso di registri così come li abbiamo definiti in precedenza e consuma un numero di
transistor molto alto.

\subsection{Memorie dinamiche}
Le \textbf{memorie dinamiche} sono implementate in modo tale da permetterci di risparmiare più
transistor ma hanno il continuo bisogno di \emph{"rinfrescare"} il loro contenuto per evitare che
questo vada perso.

Similmente ai banchi di memoria, abbiamo un demultiplexer a cui diamo in ingresso una parola da
scrivere e l'indirizzo a cui scriverla. L'output del demultiplexer è collegato ad una griglia
con un numero di righe pari al numero massimo di parole memorizzabili nella memoria e un numero
di colonne pari alla lunghezza in bit di ognuna delle parole.
\begin{center}
	\begin{tikzpicture}
		\draw[thick] (1, -0.5) -- (1, 0.5) -- (1.5, 0.75) -- (1.5, -0.75) -- cycle; % DEMUX

		% GRID
		\foreach \i in {-0.5, -0.25, 0, 0.25, 0.5}
		\draw[thick] (1.5, \i) -- (3, \i);

		\foreach \i in {1.75, 2, 2.25, 2.5, 2.75}
		\draw[thick] (\i, 0.75) -- (\i, -0.75);

		\draw[->, >=Stealth] (0, 0) node[label=left:IN] {} -- (1, 0);
		\draw[->, >=Stealth] (0, -1) node[label=left:IND] {} -- (1.25, -1) -- (1.25, -0.625);
		\draw[->, >=Stealth] (2.25, -1) -- (2.25, -1.875) node[label=right:OUT] {};
	\end{tikzpicture}
\end{center}
Se volessimo vedere più da vicino la griglia, nello specifico i punti di incrocio, saremmo in grado
di osservare che è presente un transistor il quale si comporta come un condensatore (mantiene una
carica elettrica) e come un interruttore poiché, quando è chiuso, mette in collegamento la riga con
la colonna, viceversa, quando è aperto non c'è alcun contatto.
\begin{center}
	\begin{tikzpicture}
		\draw[thick] (0, 0) -- (1.5, 0);
		\draw[thick] (1.25, 0.25) -- (1.25, -1.25);
		\draw[thick, red] (0.75, 0) -- (1.25, -0.5);
		\draw[->, >=Stealth] (0, -1) node[label=left:transistor] {} -- (0.875, -0.375);
	\end{tikzpicture}
\end{center}
Le linee orizzontali e verticali non si toccano a meno che il transistor non sia chiuso. Quando
facciamo passare un segnale 1 su una certa riga, questo incontra tutti i transistor, per tutti
quelli trovati chiusi, l'1 viene trasmesso sulla rispettiva colonna e inviato in output. Per ogni
transistor aperto, l'1 non si propagherà e dunque avremmo uno 0 in output. In questo modo è
possibile \textbf{leggere} la parola all'indirizzo IND.

Se invece volessimo \textbf{scrivere} una parola all'indirizzo IND dovremmo manipolare i bit sulle
colonne in modo tale che formino la parola desiderata. Come prima inviamo un segnale 1 all'indirizzo
scelto e i transistor chiusi tratterrano una carica elettrica in grado di memorizzare tutti gli 1
della parola. Sui bit messi a 0 i transistor avranno una configurazione aperta e quindi daranno in
output uno 0.

\subsection{Memorie statiche}
Le memorie statiche hanno una struttura a griglia come quelle dinamiche ma hanno una struttura più
complessa per quanto riguarda il collegamento tra righe e colonne. Si usano infatti due porte
\verb|NOT| e due transistor (T1 e T2) per la lettura e la scrittura dei bit. Come conseguenze
abbiamo un maggior consumo di spazio sul silicio ma anche una maggiore velocità nell'accesso alla
memoria.
\begin{center}
	\begin{circuitikz}
		\node[not port] (not1) at (0, 1) {};
		\node[not port, rotate=180] (not2) at (0, -1) {};

		\draw (-2, 0) node[label=left:T1] {} -- (-1, 0) -- (-1, 1) -- (not1.in);
		\draw (not1.out) -- (1, 1) -- (1, 0);
		\draw (2, 0) node[label=right:T2] {} -- (1, 0) -- (1, -1) -- (not2.in);
		\draw (not2.out) -- (-1, -1) -- (-1, 0);
	\end{circuitikz}
\end{center}
Inotre non è più necessario \emph{rinfrescare} il contenuto della memoria in continuazione poiché
questo tipo di memoria è in grado di trattenere l'informazione in modo autonomo.

\subsubsection{ROM}
Le \textbf{ROM} (Read Only Memory) sono molto simili, come struttura, alle RAM viste fino ad ora
ma non abbiamo alcun meccanismo di controllo sui transistor che collegano le righe alle colonne
della griglia. Abbiamo un collegamento diretto (nel caso volessimo trasmettere un 1) oppure
l'assenza di quest'ultimo (nel caso volessimo trasmettere uno 0).

\subsubsection{Considerazioni sulle RAM}
Queste appena descritte sono dette memorie \textbf{RAM} (Random Access Memory) poiché non è
necessario leggere tutte le locazioni in sequenza ma possiamo accedere direttamente alla locazione
desiderate. Comparando i tre tipi di RAM che abbiamo visto fino ad ora possiamo dire che
\begin{itemize}
	\item I registri sono i più veloci per quanto riguarda lettura e scrittura ma sono quelli
	      occupano più spazio sul silicio a parità di capienza.
	\item Le memorie dinamiche sono quelle più lente ma che occupano meno spazio sul silicio.
	\item Le memorie statiche si collocano a metà tra queste due per quanto riguarda velocità e
	      spazio occupato.
\end{itemize}
A seconda dell'uso che se ne fa e dei contesti sarà necessario scegliere un tipo di RAM piuttosto
che l'altro.

\subsection{Memorie modulari}
Da questo momento considereremo i registri come una sorta di vettore di registri, il quale prende
ad esempio in input tre indirizzi e ha 2 uscite. Questo perché se, come vedremo più avanti, in
assembler volessimo eseguire un'operazione di \verb|ADD| tra un registro \verb|R2| e un registro
\verb|R3| e volessimo salvare il risultato in un registro \verb|R1|, questo schema ci sarà molto
utile
\begin{center}
	\begin{tikzpicture}
		\draw[thick] (-0.75, -0.75) rectangle (0.75, 0.75);
		\node at (0, 0) {REG};
		\draw[->, >=Stealth] (-1.75, 0.5) node[label=left:\scriptsize{IND1}]	{} -- (-0.75, 0.5);
		\draw[->, >=Stealth] (-1.75, 0) node[label=left:\scriptsize{IND2}] {} -- (-0.75, 0);
		\draw[->, >=Stealth] (-1.75, -0.5) node[label=left:\scriptsize{IND3}]{} -- (-0.75, -0.5);

		\draw[->, >=Stealth] (-1.5, 1) node[label=left:\scriptsize{IN}] {} -- (0, 1) -- (0, 0.75);

		\draw[->, >=Stealth] (0.75, 0.25) -- (1.75, 0.25) node[label=right:\scriptsize{OUT1}] {};
		\draw[->, >=Stealth] (0.75, -0.25) -- (1.75, -0.25) node[label=right:\scriptsize{OUT2}] {};
	\end{tikzpicture}
\end{center}
Possiamo quindi leggere o scrivere più indirizzi di memoria in un singolo ciclo di clock. Per le
RAM invece dobbiamo, in generale, eseguire più cicli di clock. Possiamo anche vedere le memorie
come moduli di parole da $n$ bit, concatenabili e a cui è possibile inviare in uno stesso ciclo di
clock lo stesso indirizzo e il segnale di \verb|ENABLE|. In questo modo otteniamo una parola che
è più lunga del singolo modulo.

Se ad esempio avessimo a disposizione moduli da 1M parole di 32 bit e volessimo una memoria da 2M
parole, sarebbe possibile concatenare due di questi moduli.

Prima di tutto consideriamo che per indirizzare 1M servono $\log_2 (1 \times 10^6) = 20$ bit,
quindi per indirizzarne 2 servono 21 bit. Avremmo quindi 1 bit che sceglie quale dei due moduli
andare a leggere o scrivere e i restanti bit che andranno ad accedere la locazione di memoria
corretta.
\begin{center}
	\begin{tikzpicture}
		\draw[thick] (-2, 0) rectangle (-0.5, 1.5);
		\draw[thick] (0.5, 0) rectangle (2, 1.5);

		\draw[thick] (-1, -1) -- (1, -1) -- (0.75, -1.5) -- (-0.75, -1.5) -- cycle;
		\draw[thick] (-1, 2) rectangle (1, 2.25)
		(-0.75, 2.25) -- (-0.75, 2);

		\draw[->, >=Stealth] (-1, 2.125) -- (-2.5, 2.125) -- (-2.5, -1.25) -- (-0.875, -1.25);
		\draw[->, >=Stealth] (-1.25, 0) -- (-1.25, -0.5) -- (-0.5, -0.5) -- (-0.5, -1);
		\draw[->, >=Stealth] (1.25, 0) -- (1.25, -0.5) -- (0.5, -0.5) -- (0.5, -1);
		\draw[->, >=Stealth] (0, 2) to[short, -*] (0, 0.75) -- (-0.5, 0.75);
		\draw[->, >=Stealth] (0, 0.75) -- (0.5, 0.75);
		\draw[->, >=Stealth] (0, -1.5) -- (0, -2);

		\node at (-1.75, 1.25) {0};
		\node at (0.75, 1.25) {1};
	\end{tikzpicture}
\end{center}
Se il bit che determina quale modulo andare a leggere o scrivere è il più significativo parleremo
di memoria modulare \textbf{sequenziale}, il cui schema logico è il seguente
\begin{center}
	\begin{tikzpicture}
		\draw[thick] (0, 0) rectangle (1, 2);
		\draw[thick] (2, 0) rectangle (3, 2);

		\foreach \i in {1.5, 1, 0.5}{
				\draw[thick] (0, \i) -- (1, \i);
				\draw[thick] (2, \i) -- (3, \i);
			}

		\node at (0.5, 1.75) {0};
		\node at (0.5, 1.25) {1};
		\node at (0.5, 0.75) {2};
		\node at (0.5, 0.25) {3};

		\node at (2.5, 1.75) {4};
		\node at (2.5, 1.25) {5};
		\node at (2.5, 0.75) {6};
		\node at (2.5, 0.25) {7};
	\end{tikzpicture}
\end{center}
Se invece il bit in questione fosse il meno significativo dell'indirizzo quella che otteniamo è una
memoria modulare \textbf{interlacciata}, il cui schema logico è il seguente
\begin{center}
	\begin{tikzpicture}
		\draw[thick] (0, 0) rectangle (1, 2);
		\draw[thick] (2, 0) rectangle (3, 2);

		\foreach \i in {1.5, 1, 0.5}{
				\draw[thick] (0, \i) -- (1, \i);
				\draw[thick] (2, \i) -- (3, \i);
			}

		\node at (0.5, 1.75) {0};
		\node at (0.5, 1.25) {2};
		\node at (0.5, 0.75) {4};
		\node at (0.5, 0.25) {6};

		\node at (2.5, 1.75) {1};
		\node at (2.5, 1.25) {3};
		\node at (2.5, 0.75) {5};
		\node at (2.5, 0.25) {7};
	\end{tikzpicture}
\end{center}
Come possiamo vedere, in questo caso abbiamo che le locazioni di memoria sono enumerate saltando
costantemente da un modulo all'altro. Questo significa che se avessimo bisogno di scrivere in
memoria una parola da 64 bit potremmo farlo andando a scrivere prima nella locazione $m_i$ e poi in
quella $m_{i+1}$ che si troverà sull'altro modulo ma allo stesso livello.

Se adessimo eliminassimo il multiplexer alla fine del circuito mostrato in precedenza otterremmo
due uscite da 32 bit, ognuna con metà della parola da 64 bit. \`E quindi possibile leggere due
parole in un singolo ciclo di clock in questo modo.

Con il modello sequenziale avremmo avuto bisogno di due cicli di clock in quanto la locazione
$m_{i+1}$ si sarebbe trovata all'interno dello stesso modulo.

\subsection{Da rete combinatoria a sequenziale}
Ogni rete combinatoria si può implementare tramite una memoria. In generale, se abbiamo una rete
combinatoria con un certo numero di ingressi e un certo numero di uscite, sarà possibile salvare
ognuna delle possibili usciti in memoria utilizzando come indirizzo gli ingressi che generano tali
uscite.

Implementare tutte le componenti in questo modo è ovviamente svantaggioso. Il vantaggio si ha se
teniamo di conto che la memoria può essere riscritta e quindi, a patto di rispettare il numero di
bit in ingresso, in uscita e quelli memorizzabili sulla memoria sarebbe possibile implementare più
moduli su una singola memoria, riscrivendola quando necessario. Per riuscire nell'intento possiamo
\begin{enumerate}
	\item Derivare l'automa.
	\item Derivare la funzione che calcola le uscite.
	\item Derivare la funzione che calcola lo stato successivo.
	\item Implementare la rete sequenziale (di Mealy o Moore).
\end{enumerate}
In alternativa è possibile realizzare direttamente una rete combinatoria cercando di capire come
effettuare una conversione a rete sequenziale.


\chapter{Forme di parallelismo}\label{ch: parallelismo}
Al fine di ottimizzare le performance del processore si è passati ad un paradigma che introduce il
\textbf{parallelismo} nella gestione delle istruzioni da eseguire. Le forme di parallelismo
principali sono sostanzialmente due: \textbf{spaziale} e \textbf{temporale}.

\section{Parallelismo spaziale}
L'idea in questo caso è quella di suddividere il lavoro in più parti (\textbf{task}), ciascuna
calcolata da un \textbf{worker}. In generale possiamo dire che se dobbiamo eseguire $m$ task e
abbiamo a disposizione $n$ worker, per completare tutti i task ci metteremo un tempo pari a
\[ T_\text{par} (n) = \frac{m}{n} \cdot t \]
dove $t$ è il tempo di completamento di un singolo task.

Il parallelismo spaziale può essere suddiviso in due ulteriori tipologie: \textbf{map} e
\textbf{farm}. Nel caso in cui i task siano parte di un problema più grande e quindi coesistono
tutti allo stesso tempo, allora parliamo di map.

Nel caso in cui i task esistano in tempo diversi, per esempio arrivando uno dopo l'altro e che
quindi possono non essere in alcun modo correlati tra di loro, allora parliamo di farm. In questo
caso possiamo ad esempio affidarci ad uno \textbf{scheduler} in grado di assegnare, man mano che
arrivano, i vari task ai worker liberi.

Una grandezza in gioco per valutare l'aumento delle performance di un sistema parallelo è lo
\textbf{speed-up}, il quale è calcolato come segue
\[ \text{speed-up} (n) = \frac{T_\text{seq}}{T_\text{par} (n)} \]
dove $T_\text{seq}$ è il tempo che impiegheremmo con un calcolo sequenziale. Nel caso di
parallelismo spaziale avremmo uno speed-up pari a
\[ \text{speed-up} (n) = \frac{m \cdot t}{\frac{m}{n} \cdot t} = n \]
Altre grandezze in gioco sono la \textbf{latenza}, definita come il tempo assoluto di lavoro per il
calcolo di un risultato e il \textbf{throughput}, definito come il numero di task terminati
nell'unità di tempo.

Quest'ultimo è definito in funzione del \textbf{tempo di servizio}, ossia il tempo che intercorre
tra il termine di un task e il termine di un task successivo (in questo caso equivale a $t$).
\[ \text{Throughput} = \frac{1}{T_s} \]
Analogamente al throughput c'è il \textbf{tempo di completamento}, definito come il tempo di
completamento di tutti i lavori. Se definiamo quindi $t_0$ come il tempo di inizio del primo task
e $t_1$ come il tempo di fine dell'ultimo task, il tempo di completamento equivale a
\[ T_c = t_1 - t_0 \]
Nel caso di farm, avremmo un tempo di completamento pari a
\[ T_c = T_\text{sched} \cdot m + \frac{m}{n} \cdot t \approx \frac{m}{n} \cdot t \]
dove $T_\text{sched}$ è il tempo che lo schedulatore impiega ad assegnare i task. In genere questo
tempo è trascurabile e dunque l'approssimazione fatta sopra è lecita.

Nel caso invece di map dobbiamo considerare che c'è un momento in cui si suddivide il problema in
parti più piccole $T_\text{split}$ e un momento in cui si uniscono i risultati prodotti dai singoli
worker per generare il risultato finale $T_\text{merge}$. Avremo quindi un tempo di completamento
pari a
\[ T_c = T_\text{split} + \frac{m}{n} \cdot t + T_\text{merge} \]
ma ancora una volta consideriamo $T_\text{split}$ e $T_\text{merge}$ come trascurabili e dunque
avremmo un tempo di completamento pari a
\[ T_c \approx \frac{m}{n} \cdot t = \frac{T_\text{seq}}{n} \]
Il parallelismo spaziale è quello implementato dai processori superscalari.

\section{Parallelismo temporale}
Questa forma di parallelismo si addice di più ad un tipo di calcolo \emph{annidato}. Supponiamo
infatti che il completamento di una computazione dipenda da due elaborazioni $f$ e $g$ dei dati,
tali che il risultato è ottenuto calcolando
\[ y = f(g(x)) \]
dove $x$ è l'input. In questo caso possiamo assegnare ad un worker il calcolo di $f$ e ad un worker
il calcolo di $g$. Quando il worker che calcola $g$ termina passa il risultato a quello in grado di
calcolare $f$. Scalando al caso in cui abbiamo $m$ lavori, quello che succede è che il primo worker
calcola $g(x_i)$ e passa il risultato al secondo worker che calcola $f(g(x_i))$. Mentre quest'ultimo
lavora il primo worker sta già calcolando $g(x_{i+1})$.
\begin{center}
	\begin{tikzpicture}
		\node at(-0.5, 0.5) {$g$};
		\node at(-0.5, -0.5) {$f$};
		\draw[thick, red] (0, 0.5)  -- (1, 0.5);
		\draw[thick, red] (1, -0.5) -- (2, -0.5);
		\draw[thick, blue] (1, 0.5) -- (2, 0.5);
		\draw[thick, blue] (2, -0.5) -- (3, -0.5);
		\draw[thick, green] (2, 0.5) -- (3, 0.5);
		\draw[thick, green] (3, -0.5) -- (4, -0.5);
	\end{tikzpicture}
\end{center}
Come possiamo vedere dalla figura, mentre il worker $f$ lavora, il worker $g$ sta già elaborando
la parte di un altro task. Il momento in cui tutti i worker eseguono un task in contemporanea si
chiama \textbf{steady state}, o se vogliamo possiamo dire che siamo a \textbf{regime}.

Con questo tipo di parallelismo, se abbiamo $m$ task, ognuno di questi suddivisibile in $n$
\textbf{stadi}, impiegheremmo un tempo pari a
\[ m \cdot n \cdot t \]
per il completamento, dove $t$ è il tempo richiesto per completare uno stadio della computazione.
Se però assegnamo ogni stadio ad un worker impiegheremmo un tempo di
\[ t \cdot n + t \cdot (m - 1) = t \cdot (n + m - 1) \]
In questo caso, se, come in genere succede, $m \gg n$, otteniamo uno speed-up di
\[
	\text{speed-up} (n) = \frac{m \cdot n \cdot t}{t \cdot (n + m - 1)}
	= \frac{m \cdot n}{n + m - 1} \approx n
\]
Il parallelismo temporale è in genere implementato come un
meccanismo di \textbf{pipeline} il cui tempo di completamento equivale a
\[ T_c = m \cdot T_s + (n - 1) \cdot T_s \approx m \cdot T_s \]
Dato che ogni worker calcola uno stadio differente della computazione, se i task ad ogni stadio
necessitano dello stesso tempo di calcolo la formula sopra è valida. Se però i tempi di
elaborazione ad ogni stadio sono differenti consideriamo questo tempo di completamento
\[ T_c = m \cdot \max \{ t_i \} + \left( \sum_{i=1}^n t_i - \max \{ t_i \} \right) \]
In questo caso il tempo di servizio sarà pari a
\[ T_s = \max \{ t_i \} \]
poiché lo stadio di elaborazione che richiede più tempo rallenta anche tutti gli altri. Possiamo
quindi riscrivere la formula del tempo di completamento in questo modo
\[ T_c = m \cdot T_s + \left( \sum_{i=1}^n t_i - T_s \right) \]
dove $T_s$ equivale al tempo che il worker più lento impiega per terminare tutti i suoi task e dove
la sommatoria rappresenta il tempo precedente al momento in cui il worker più lento ha iniziato a
lavorare e quello successivo al momento in cui il worker più lento ha terminato il suo ultimo task.

I processori pipeline implementano un parallelismo temporale per eseguire il ciclo
fetch-decode-execute accennato all'inizio.


% PARTE 2: Architettura
\part{Architettura}

\chapter{Assembler}
In questa parte tratteremo l'\textbf{architettura} saltando per il momento $\mu$-architettura che
tratteremo più avanti. In particolare tratteremo i costrutti del linguaggio \textbf{assembly}
versione \textbf{ARMv7} nel quale saranno disponibili 16 registri e memorie da 4GB con indirizzi da
32 bit.

L'assembler è un linguaggio a basso livello con il quale possiamo di fatto programmare il nostro
processore eseguendo \textbf{istruzioni}
\begin{itemize}
	\item Di \textbf{calcolo}: come somme, sottrazioni, shift, operazioni booleane e così via. Tali
	      istruzioni operano sui registri interni al processore.
	\item Di \textbf{accesso alla memoria} come la \verb|load| e la \verb|store|, per manipolare
	      l'informazione dai registri alla memoria e viceversa.
	\item Di \textbf{controllo} o \textbf{salto}: costrutti equivalenti a \verb|IF-THEN-ELSE| e
	      chiamate di funzione.
\end{itemize}
Questo set di istruzioni è di tipo \textbf{RISC} (Reduced Instruction Set Computer), ossia un set
di istruzioni ridotto. I motivi per cui lo adottiamo sono
\begin{itemize}
	\item La regolarità favorisce la semplicità.
	\item Il caso più comune deve essere implementato in modo più veloce.
	\item Implementare un calcolatore più piccolo e semplice porta dei vantaggi.
	\item Un buon progetto richiede buoni compromessi.
\end{itemize}
Tramite il linguaggio assembler andremo di fatto ad utilizzare le componenti descritte in
precedenza tramite reti logiche.

Come vedremo il linguaggio è composto di sole istruzioni elementari e quindi ci darà molto più
controllo su ciò che stiamo facendo, pagando però il prezzo di scrivere molte più linee di codice
rispetto a linguaggi di più alto livello come C o Java.


\section{Strumenti per la programmazione}
Per programmare in assembler ARM sotto Linux possiamo installare il cross-compiler per ARM e
relativi tool di debugging tramite la seguente riga di codice
\begin{minted}{bash}
$ sudo apt install gcc-arm-linux-gnueabihf qemu-user gdb-multiarch
\end{minted}
Per la compilazione e l'esecuzione del debugger le istruzioni sono le seguenti
\begin{minted}{bash}
$ arm-linux-gnueabihf-gcc -ggdb3 -static <sorgenti> -o <eseguibile>
$ qemu-arm -g <porta> <eseguibile> &
$ gdb-multiarch -q --nh -ex "target remote localhost:<porta>" <eseguibile>
\end{minted}
Una volta dentro il debugger possiamo impostare break points e usare le istruzioni
\begin{itemize}
	\item \verb|continue|: per eseguire il codice fino al prossimo break point.
	\item \verb|next|: per eseguire le istruzioni una riga alla volta.
	\item \verb|tui reg general|: per visualizzare un'interfaccia in grado di farci vedere il
	      valore dei vari registri man mano che eseguiamo le istruzioni.
\end{itemize}
In alternativa è possibile usare lo strumento online \href{https://cpulator.01xz.net/}{CPUlator}.

\section{Istruzioni}
L'assembler è un linguaggio a basso livello con il quale possiamo programmare il nostro processore
eseguendo \textbf{istruzioni}
\begin{itemize}
	\item Di \textbf{operative}: come somme, sottrazioni, shift, operazioni booleane e così via.
	      Tali istruzioni operano sui registri interni al processore.
	\item Di \textbf{accesso alla memoria} per manipolare i dati dai registri alla memoria e
	      viceversa.
	\item Di \textbf{controllo} o \textbf{salto}: costrutti equivalenti a \verb|IF-THEN-ELSE| e
	      chiamate di funzione.
\end{itemize}

\subsection{Registri}
I registri, in ARMv7, sono 16 (da 0 a 15), quelli da 0 a 12 hanno delle convenzioni ma sono sotto
il completo controllo dell'utente, mentre gli ultimi tre sono registri \emph{speciali}:
\begin{itemize}
	\item \textbf{PC}: è il \textbf{program counter}, il quale contiene l'indirizzo dell'istruzione
	      da eseguire.
	\item \textbf{SP}: è lo \textbf{stack pointer}, utile per memorizzare l'indirizzo su cui si
	      trova l'ultimo elemento inserito nello \emph{stack}.
	\item \textbf{LR}: è il \textbf{link register} che viene utilizzato per le chiamate di funzione
	      per mantenere l'\emph{indirizzo di ritorno}.
\end{itemize}
I restanti registri sono di più libero uso ma l'assembler ARM definisce delle \textbf{convenzioni}
su di essi
\begin{itemize}
	\item I registri da 0 a 3 sono riservati per valori temporanei o per parametri di chiamate di
	      funzione. Solitamente il registro 0 viene usato per il valore di ritorno di una funzione,
	      mentre da 1 a 3 sono utilizzati come parametri attuali in una chiamata di funzione.
	\item I registri da 4 a 11 servono a memorizzare le variabili. Dopo il loro utilizzo dobbiamo
	      garantire che questi tornino ad avere il valore che avevano prima che venissero
	      utilizzati salvandoli sullo stack.
\end{itemize}
Attenersi a tali convenzioni aiuta a garantire l'interoperabilità.

\subsection{Istruzioni operative}
Le istruzioni \textbf{operative} prendono come argomenti tre registri, il primo sarà il registro di
\textbf{destinazione} e gli altri due saranno i registri \textbf{sorgente} da cui prenderemo i
valori necessari a svolgere l'operazione desiderata. Tra le istruzioni più importanti abbiamo le
operazioni di \textbf{calcolo}
\begin{itemize}
	\item \verb|ADD| e \verb|ADC|: addizione e addizione con riporto.
	\item \verb|SUB| e \verb|RSB|: sottrazione e sottrazione inversa.
\end{itemize}
Tra le istruzioni \textbf{logiche} abbiamo invece
\begin{itemize}
	\item \verb|AND|: \verb|AND| logico
	\item \verb|ORR|: \verb|OR| logico
	\item \verb|EOR|: \verb|XOR|, sta per Exclusive OR.
	\item \verb|BIC|: operazione di \emph{bit clear} la quale cancella dal secondo parametro i bit
	      a 1 del terzo parametro.
\end{itemize}
Le istruzioni di \textbf{shift} sono sia di tipo logico che aritmetico
\begin{itemize}
	\item \verb|LSL| e \verb|LSR|: rispettivamente shift logico a sinistra e a destra.
	\item \verb|ASL| e \verb|ASR|: rispettivamente shift aritmetico a sinistra e a destra.
	\item \verb|ROR|: shift ciclico in cui i bit più a destra rientrano da sinistra.
\end{itemize}
Le istruzioni di \textbf{moltiplicazione} sono molteplici:
\begin{itemize}
	\item \verb|MUL|: come le altre istruzioni accetta tre parametri ma dato che il risultato
	      sarebbe un numero a 64 bit considera solo i 32 bit meno significativi.
	\item \verb|SMULL| e \verb|UMULL|: che eseguono rispettivamente la moltiplicazione per numeri
	      con e senza segno ma stavolta mettono il risultato in due registri (infatti necessitano
	      di quattro parametri di cui i primi due sono le destinazioni) restituendo così un valore
	      a 64 bit.
\end{itemize}
Per quanto riguarda \textbf{la divisione non è supportata} nativamente dobbiamo quindi
implementarla in qualche modo.

Come ultima istruzione vediamo l'\textbf{assegnamento} (\verb|MOV|) che prende solo due parametri,
la destinazione e la sorgente, andando a scrivere il valore della sorgente nella destinazione.

\subsection{Istruzioni di accesso alla memoria}
Per l'accesso alla memoria abbiamo due istruzioni principali (\verb|LOAD| e \verb|STORE|), le quali
servono a muovere i dati dalla memoria ai registri e viceversa:
\begin{itemize}
	\item \verb|LDR|: da memoria a registro.
	\item \verb|STR|: da registro a memoria.
\end{itemize}
L'istruzione di \verb|LOAD| prende come primo parametro un registro che sarà la destinazione,
mentre la \verb|STORE| prenderà come primo parametro un registro che sarà la sorgente di cui
leggeremo il contenuto.

I parametri dopo il primo, sia per l'una che per l'altra istruzione possono variare a seconda dei
casi, e servono ad definire l'indirizzo da cui leggere o scrivere dati.
\begin{minted}{gas}
ldr r0, [r1]
\end{minted}
va a leggere il contenuto della memoria all'indirizzo scritto in \verb|r1| e lo scrive in \verb|r0|.
Supponiamo però di avere in \verb|r1| l'indirizzo in cui è presente un array, se volessimo prendere
un valore all'interno dell'array senza perdere l'indirizzo in \verb|r1| possiamo fare in questo
modo
\begin{minted}{gas}
ldr r0, [r1, r2]
\end{minted}
andando a specificare con \verb|r2| un offset e andando di fatto a leggere la memoria all'indirzzo
\verb|r1 + r2| lasciando questi ultimi invariati. Al posto di \verb|r2| possiamo usare anche una
costante numerica.

La memoria in ARMv7 è \textbf{indirizzata al byte} e dato che consideriamo locazioni di memoria da
32 bit, ossia 4 byte, l'operazione
\begin{minted}{gas}
ldr r0, [r1, #4]
\end{minted}
non legge il quarto elemento dell'array ma il secondo (abbiamo cioè un offset di una posizione).
Per gestire gli indici in maniera più ragionevole esiste anche un altro modo per l'accesso alla
memoria
\begin{minted}{gas}
ldr r0, [r1, r2, LSL #2]
\end{minted}
che va a leggere la memoria all'indirizzo $r1 + r2 \times 4$. Supponiamo ora di voler scrivere in
ARM un codice che in C scriveremmo in questo modo
\begin{minted}{c}
int x[16];
sum = x[0] + x[1];	
\end{minted}
Supponiamo ora che la base di \verb|x| sia in \verb|r1| e vogliamo usare \verb|r2| come indice,
quello che scriveremmo in ARM sarebbe
\begin{minted}{gas}
mov r2, #0
ldr r3, [r1, r2]
add r2, r2, #4
ldr r4, [r1, r2]
\end{minted}
In termini più generali possiamo scrivere
\begin{minted}{gas}
ldr r3, [r1, r2]
add r2, r2, #1
ldr r3, [r1, r2, lsl #2]
\end{minted}
Dato che scorrere la memoria in modo sequenziale accade molto spesso abbiamo a disposizione dei
costrutti che compattano il codice incrementando l'indice in automatico di una locazione di memoria
\begin{minted}{gas}
ldr r0, [r1, r2]!
\end{minted}
Questo di sopra è un meccanismo detto \textbf{pre-index} che va a mettere in \verb|r0| il contenuto
della memoria all'indirizzo $r1 + r2$ e implicitamente mettiamo in \verb|r1| il valore di $r1 + r2$.
In alternativa abbiamo il meccanismo di \textbf{post-index}
\begin{minted}{gas}
ldr r0, [r1], r2
\end{minted}
che, a differenza del pre-index, mette in \verb|r0| il valore in memoria all'indirizzo \verb|r1| e
solo dopo aggiorna \verb|r1| con $r1 + r2$.

Altre due implementazioni delle istruzioni di \verb|load| e \verb|store| sono \verb|ldrb| e
\verb|strb| le quali caricano il primo byte indirizzato nella parte meno significativa del
registro. Per esempio
\begin{minted}{gas}
ldrb r1, [r2]
\end{minted}
prende il primo byte del valore in memoria all'indirizzo \verb|r2| e lo scrive nella parte meno
significativa del registro \verb|r1|.

Esistono anche delle istruzioni per effettuare \verb|load| e \verb|store| di più registri che sono
rispettivamente \verb|ldmxx| e \verb|stmxx| dove i due caratteri \verb|xx| devono essere sostituiti
da una combinazione delle seguenti alternative
\begin{itemize}
	\item Al primo posto possiamo specificare \verb|f| (full), se vogliamo che il registro punti
	      all'ultima posizione piena o \verb|e| (empty) se vogliamo che il registro punti alla
	      prima posizione vuota.
	\item Al secondo posto possiamo specificare \verb|d| (descending) per caricare o scaricare i
	      dati in ordine decrescente o \verb|a| (ascending) per caricare o scaricare i dati in
	      ordine crescente.
\end{itemize}
e dove di seguito abbiamo il primo argomento che è un registro e il secondo argomento è una lista
di registri tra parentesi graffe. Per esempio
\begin{minted}{gas}
ldmfd sp!, {r1, r2, r3}
stmfd sp!, {r1, r2, r3}
\end{minted}
carica e scarica nello stack pointer i registri tra parentesi graffe in ordine decrescente e
incrementa lo stack pointer tramite il punto esclamativo. Dato che usare lo stack pointer in questo
modo è molto frequente e utile possiamo usare due istruzioni equivalenti
\begin{minted}{gas}
pop {r1, r2, r3}
push {r1, r2, r3}
\end{minted}
che scrivono e leggono lo stack pointer.

\subsection{Esecuzione condizionale}
Abbiamo a disposizione anche delle istruzioni condizionali che in sostanza vengono eseguite solo se
una certa condizione è vera. Per esempio
\begin{minted}{gas}
addeq r0, r1, r2
\end{minted}
somma \verb|r1| ed \verb|r2| se e solo se i \textbf{flag di condizione} indicano che questi sono
uguali. Se \verb|r1| ed \verb|r2| non sono uguali ma prima non si è fatto alcuna operazione per
settare i flag di condizione se effettuerà la somma in ogni caso. Per ottenere il risultato atteso
dovremmo fare in questo modo
\begin{minted}{gas}
cmp r1, r2
addeq r0, r1, r2
\end{minted}
I flag di condizione sono quattro e sono contenuti nel registro chiamato \text{cpsw} (Current
Program Status Word). I quattro flag sono \textbf{Negative}, \textbf{Zero}, \textbf{Carry} e
\textbf{Overflow} e indicano se una specifica condizione si è verificata dopo un'operazione.

Non solo l'operazione \verb|cmp| è in grado di settare i flag ma anche le operazioni aritmetiche
seguite da una "\verb|s|", per esempio
\begin{minted}{gas}
adds r0, r1, r2
\end{minted}
setta i flag in base al risultato di $r1 + r2$. Supponiamo di voler implementare un programma
equivalente al seguente programma C
\begin{minted}{c}
if (x + y == 0)
	y++;
\end{minted}
Con le istruzioni ARM di base scriveremmo un codice di questo tipo
\begin{minted}{gas}
add r0, r1, r2
cmp r0, #0
bne fine // etichetta d'esempio
add r2, r2, #1
\end{minted}
risparmiamo due righe di codice scrivendo
\begin{minted}{gas}
adds r0, r1, r2
addeq r2, r2, #1
\end{minted}
Similmente al suffisso \verb|eq| abbiamo anche \verb|ne|, \verb|gt|, \verb|lt|, \verb|ge| e
\verb|le| che rispettivamente fanno controlli su gli operandi tramite i seguenti operatori:
$\neq$, $>$, $<$, $\geq$ e $\leq$.

\subsection{Istruzioni di salto}
Le istruzioni di salto le dividiamo in istruzioni di salto \textbf{condizionato} e
\textbf{non condizionato}. Per quanto riguarda il salto incondizionato possiamo usare l'istruzione
\verb|b| affiancata dall'etichetta a cui vogliamo andare
\begin{minted}{gas}
main:
	b somma
	sub r0, r1, r2

somma:
	add r0, r1, r2
\end{minted}
In questo esempio saltiamo l'operazione di sottrazione ed eseguiamo direttamente all'operazione di
somma in modo incondizionato. Nel caso invece volessimo effettuare un salto condizionato potremmo
fare in questo modo
\begin{minted}{gas}
main:
	add r0, r1, r2
	cmp r0, #0
	bne azzera

azzera:
	mov r0, #0
\end{minted}
così che se il risultato della somma è diverso da 0 si salta all'etichetta che azzera il registro.
Le lettere che possiamo inserire dopo "\verb|b|" sono le stesse che possiamo mettere dopo le
operazioni aritmetiche condizionali.

Se c'è la possibilità di implementare il nostro programma tramite istruzioni condizionali, queste
sarebbero preferibili alle istruzioni di salto.

Aggiungiamo inoltre che nel caso di processori pipeline e superscalari le operazioni di salto vanno
a minare tutta l'ottimizzazione che questi due modelli offrono che invece hanno bisogno di eseguire
le istruzioni in un certo ordine.

Un'altra operazione di salto è la \textbf{branch and link} (\verb|bl|) che fa la stessa cosa del
comando \verb|b| ma setta il link register scrivendoci dentro il valore del program counter
avanzato di una posizione.
\begin{minted}{gas}
main:
	mov r0, #0
	mov r1, #1
	mov r2, #2
	
	bl somma
	cmp r0, #0
	bne azzera
	mov r0, r0

somma:
	add r0, r1, r2
	mov pc, lr	// return

azzera:
	mov r0, #0
\end{minted}
Questo ci è molto utile per implementare chiamata e ritorno di funzione. L'ultima istruzione di
salto che vediamo è la \verb|bx| seguita da un registro, che assegna al program counter il
contenuto del registro.

\section{Funzioni}
Vediamo ora come sia possibile realizzare \textbf{funzioni} e \textbf{procedure} in assembler
andando a capire come implementare una \textbf{chiamata} di funzione, il \textbf{ritorno} dalla
funzione e come effettuare il \textbf{passaggio di parametri}.

Vogliamo anche capire come riuscire a richiamare del codice già compilato e come generare codice
che permette di essere eseguito da altri.

Se ad esempio volessimo scrivere una funzione in C che prende un parametro e ritorna il parametro
stesso incrementato di 1, il codice sarebbe
\begin{minted}{c}
int f(int x) { return x + 1; }
\end{minted}
In questo caso \verb|x| è un parametro formale ma nel momento in cui andremo a passare un valore
al suo interno diventerà un parametro attuale e questo processo è un qualcosa che vogliamo riuscire
ad implementare in assembler.

L'istruzione \verb|return| restituisce un valore che dal nostro punto di vista può essere visto
anch'esso come un problema di passaggio di parametri.

\end{document}
