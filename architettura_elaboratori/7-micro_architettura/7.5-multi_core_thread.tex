\section{Multicore e multithreading}
Una nuova classe di processori è quella dei processori \textbf{multithread} e \textbf{multicore}
(o multiprocess), i quali amplificano ancora di più il grado di parallelismo spaziale ma in un modo
differente dai superscalari.

Entrambi i modelli gestiscono flussi di controllo \textbf{indipendenti} ma differiscono per lo
spazio di indirizzamento che vanno ad accedere.
\begin{itemize}
	\item I processori multithread eseguono istruzioni appartenenti allo stesso spazio di
	      indirizzamento \textbf{condiviso}.
	\item I processori multicore eseguono istruzioni appartenenti a spazi di indirizzamento
	      \textbf{differenti}.
\end{itemize}
A differenza del superscalare che provava a eseguire più istruzioni contemporaneamente ma comunque
era legato al flusso di esecuzione del nostro programma, questi modelli si propongono di eseguire
flussi di esecuzioni differenti in modo indipendente.

\subsection{Multithreading}
Il modello multithread cerca di eseguire più flussi in contemporanea che però condividono lo stesso
spazio di indirizzamento. Nella pratica andiamo a replicare lo stato architetturale $n$ volte. Se
per esempio avessimo un livello di multithreading pari a 2 avremmo grosso modo un modello in cui
parti come \verb|PC| e \verb|REG| sono doppie ma in cui la memoria rimane unica.
\begin{center}
	\includesvg[inkscapelatex=false, scale=0.7] {circuiti/multithread.svg}
\end{center}
Quello che viene fatto è alternare, secondo politiche scelte in fase di progettazione, l'esecuzione
di uno o l'altro flusso di esecuzione utilizzando unità di controllo che determinano il
comportamento dei multiplexer.

Si potrebbe pensare che uno stile architetturale del genere possa essere ancora più pesante dal
punto di vista delle dipendenze ma in realtà può addirittura alleggerire il carico sulle unità di
controllo dedicate alla gestione delle dipendenze.

