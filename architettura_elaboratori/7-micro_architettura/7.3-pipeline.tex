\section{Pipeline}
Il processore \textbf{pipeline} si organizza (al livello logico) più moduli in grado eseguire le
fasi di fetch, decode ed execute sfruttando il parallelismo di tipo temporale.

Come già anticipato nel capitolo \ref{ch: parallelismo}, una volta che riusciamo a far lavorare
tutti i moduli contemporaneamente andiamo \emph{a regime} e questo ci permette di ottenere un
risultato di un'istruzione ogni ciclo di clock.

Chiariamo che anche il processore single cycle esegue termina un'istruzione per ogni ciclo di clock
ma ha un ciclo di clock più lungo. Semplificando, dividiamo un processore pipeline in tre moduli,
uno per il fetch, uno per il decode e uno per l'execute e supponiamo che tutti e tre impieghino lo
stesso tempo per terminare.

Così facendo possiamo immaginarci di ridurre il ciclo di clock ad $1 / 3$ del ciclo di clock di un
processore single cycle.

Questo tipo di architettura porta con se delle complicazioni legati principalmente alla separazione
dei moduli e all'introduzione del parallelismo.
\begin{itemize}
	\item Ci sono più stadi che nello specifico sono, fetch, decode, execute, passaggio dalla
	      memoria e fase di \emph{write back} in cui andiamo a scrivere i risultati dell'esecuzione
	      nei registri.
	\item Possono esserci \textbf{dipendenze} tra i dati. Per esempio se andiamo ad eseguire se
	      abbiamo un'istruzione \verb|CMP| seguita da una \verb|BEQ|, dobbiamo andare a settare
	      i flags di condizione prima che avvenga la decodifica dell'istruzione \verb|BEQ|.
\end{itemize}
In un processore pipeline abbiamo di nuovo la memoria istruzioni e la memoria dati separate ma,
similmente al processore multi-cycle, abbiamo diverse zone del circuito separate da registri.
\begin{center}
	\includesvg[inkscapelatex=false, scale=0.65] {circuiti/pipeline.svg}
\end{center}
Come possiamo vedere dall'immagine, i registri colorati servono proprio separare le varie fasi di
elaborazione.

