\section{Superscalare}
Il processore \textbf{superscalare} aggiunge una forma di parallelismo spaziale al modello pipeline
cercando di parallelizzare $n$ processori pipeline effettuando il fetch di $n$ istruzioni in
parallelo e cercando quindi di eseguire $n$ istruzioni per ciclo di clock. Un processore
superscalare a due livelli avrebbe più o meno questa forma
\begin{center}
	\includesvg[inkscapelatex=false, scale=0.8] {circuiti/superscalare.svg}
\end{center}
Aumentare il parallelismo aumenta anche il peso delle dipendenze tra le istruzioni in due modi. Se
ad esempio abbiamo due istruzioni di questo tipo
\begin{minted}{gas}
add r1, r2, r3
sub r4, r1, r5
\end{minted}
Un processore superscalare a due livelli cercherà di eseguirle in parallelo ma la prima scrive il
registro \verb|r1| mentre la seconda lo legge. Abbiamo quindi compresso la dipendenza su due passi
contemporanei. Un modo per ovviare a questo problema effettuare del forwarding aggiungendo una
diramazione dalle uscite delle ALU, ridirezionandola come ingresso dell'altra ALU.

Il secondo tipo di problematica di un superscalare è la dipendenza tra $n$-uple di istruzioni
eseguite in cicli di clock differenti. Supponiamo di voler eseguire
\begin{minted}{gas}
ldr r1, [r2, r3]
add r4, r5, r6

sub r1, r1, #1
add r4, r4, #2
\end{minted}
con un processore superscalare a due livelli. La prima coppia di per se non ha un problema di
dipendenze, così come la seconda coppia ma se consideriamo che il processore superscalare a due
livelli, questo cerca di eseguire coppie di istruzioni in una volta sola.

Se consideriamo ora la prima istruzione di ogni coppia notiamo che la \verb|ldr| scrive \verb|r1|
e \verb|sub| legge e scrive \verb|r1| (discorso analogo per \verb|r4| nella seconda istruzione di
ogni coppia). Si ha quindi una dipendenza RAW tra le due coppie.

Questo complica ovviamente la logica di controllo come unità di controllo, hazard unit e così via
ma, se la situazione viene opportunamente gestita, si ha un incremento delle prestazioni notevole
soprattutto se andiamo ad aggiungere livelli al nostro processore superscalare.