\chapter{Microarchitettura}
Vogliamo ora mettere insieme ciò che abbiamo visto nella parte di reti logiche e in quella di
assembler e cercare di progettare ed implementare un piccolo processore in grado di eseguire
il famoso ciclo \emph{fetch-decode-execute} di cui abbiamo parlato all'inizio.

Partiremo dalla modellazione di un processore di tipo \textbf{single cycle} e andando avanti
andremo ad modellare anche processori di tipo \textbf{multi-cycle}, \textbf{pipeline} e
\textbf{superscalari}.

\section{Single cycle}
Proviamo ora a capire come modellare e dunque definire la $\mu$-architettura di un processore
\textbf{single cycle} in grado di eseguire alcune istruzioni di base tramite le reti combinatorie e
sequenziali viste in precedenza.

Al fine di ottenere il prima possibile una rappresentazione più generale della cosa ometteremo
alcune cose che però andranno ad arricchire più avanti il nostro modello.

Come abbiamo visto in assembler, abbiamo bisogno di lavorare con i registri del processore e, a
seconda dei casi, dobbiamo interagire con la memoria per aggiornare o salvare tali valori. Il
processore ha quindi bisogno di un modulo \verb|REG| per mantenere lo stato interno dei registri.
\begin{center}
	\includesvg[inkscapelatex=false, scale=0.8]{circuiti/reg.svg}
\end{center}
Abbiamo quindi un modulo per leggere e scrivere i registri del processore e che sarà indispensabile
per compiere le operazioni aritmetico-logiche.

Come sappiamo il registro \verb|r15| è il program counter ma quello che si trova all'interno del
modulo \verb|REG| è una copia di un registro indipendente \verb|PC|.

Per quanto riguarda la memoria, possiamo dividerla in due moduli principali per il momento: memoria
\textbf{dati} (\verb|MD|) e memoria \textbf{istruzioni} (\verb|MI|). La memoria dati contiene tutte
le variabili globali del nostro programma mentre la memoria istruzioni mantiene una sorta di
tabella in cui troviamo una corrispondenza tra l'istruzione assembler e la rispettiva
rappresentazione in linguaggio macchina.

\subsection{Fetch-Decode-Execute di un'istruzione}
Vogliamo ora modellare una $\mu$-architettura che riesca ad eseguire il fetch-decode-execute per la
lettura della memoria tramite l'istruzione
\begin{minted}{gas}
ldr r1, [r2, #4]
\end{minted}
Tramite i moduli precedentemente definiti siamo in grado di produrre un modello in primo luogo
dando come ingresso a \verb|MI| il contenuto di \verb|PC|
\begin{center}
	\includesvg[inkscapelatex=false]{circuiti/PC_to_MI.svg}
\end{center}
A questo punto l'uscita di \verb|MI| viene inoltrata in un modulo che chiameremo \verb|ISTR| che
effettua la traduzione da codice assembler a linguaggio macchina, generando l'istruzione nel
\hyperref[sz: formato]{formato} binario interpretabile dal processore.

Una volta effettuata la conversione avremo un'istruzione in linguaggio macchina contenente (in
questo caso) l'indirizzo di destinazione (\verb|r1|), il primo indirizzo sorgente (\verb|r2|) e
l'offset (\verb|#4|). I primi due diventeranno l'input del modulo \verb|REG| mentre l'offset,
rappresentato a 12 bit nel campo \verb|SRC2|, viene esteso a 32 bit tramite un modulo \verb|EXT|
apposito.
\begin{center}
	\includesvg[inkscapelatex=false, scale=0.7]{circuiti/arch0.svg}
\end{center}
Fino ad ora abbiamo solo scritto i valori nei registri e esteso il valore dell'offset in formato a
32 bit. Dato che vogliamo andare a scrivere nel registro \verb|r1| l'indirizzo di memoria in $r2+4$
ci serve una ALU in grado di effettuare tale somma.

Una volta effettuata la somma abbiamo a disposizione l'indirizzo $r2 + 4$ con il quale possiamo
accedere il modulo \verb|MD|, ricavare l'informazione desiderata e andarla a scrivere nel registro
\verb|r1|.
\begin{center}
	\includesvg[inkscapelatex=false, scale=0.7]{circuiti/arch1.svg}
\end{center}
Come possiamo vedere dal disegno l'output di \verb|MD| diventa l'input di \verb|REG| che, avendo
come tra gli ingressi l'indirizzo \verb|r1| come registro destinazione, va a scrivere il valore
ricavato da \verb|MD| in \verb|r1|.

Questa prima bozza rappresenta ovviamente un modello parziale che potremmo arricchire andando per
esempio ad aggiungere la possibilità di eseguire un'istruzione di lettura dalla memoria di questo
tipo
\begin{minted}{gas}
ldr r1, [r2, r0]
\end{minted}
In questo caso il campo \verb|SRC2| avrà negli ultimi 4 bit l'indirizzo del registro \verb|r0| da
cui sarà possibile andare a leggere il valore dell'offset.

Quello che dobbiamo fare ora è aggiungere al circuito la seconda uscita di \verb|REG| e mandarla in
un multiplexer in grado di decidere se usare il valore di quell'uscita o il valore \emph{immediato}
in uscita da \verb|EXT|.
\begin{center}
	\includesvg[inkscapelatex=false, scale=0.7]{circuiti/arch2.svg}
\end{center}
Il bit che serve al multiplexer per effettuare la selezione è il campo \verb|I| dell'istruzione
assembler convertita in binario.

\subsubsection{Incrementare il program counter}
Siamo quindi in grado di eseguire un'istruzione di lettura della memoria ma non di eseguirne
tante in sequenza. Dobbiamo infatti riuscire ad implementare un ciclo infinito dentro il quale
effettuiamo le operazioni di fetch-decode-execute. Per farlo è necessario incrementare il \verb|PC|
di 4 (poiché le istruzioni di 32 bit sono indirizzate al byte) tramite una ALU.
\begin{center}
	\includesvg[inkscapelatex=false, scale=0.7]{circuiti/cycle.svg}
\end{center}
Ora dobbiamo immaginarci che all'interno di un ciclo di clock abbiamo tutte le fasi descritte da
questo circuito e solo quando le abbiamo completate, il clock torna alto, incrementiamo \verb|PC| e
appena il clock torna stabile possiamo eseguire una nuova istruzione.

\subsection{Scrittura della memoria}
Quello che abbiamo visto fino ad ora permette solamente di effettuare un'operazione di \verb|ldr|
che, come sappiamo, prende come primo parametro il registro di destinazione su cui scrivere. Se
invece volessimo effettuare un'operazione di \verb|str|, il primo parametro sarebbe il registro
sorgente.

Con un'operazione di \verb|str| cambia anche l'interpretazione dell'istruzione in linguaggio
macchina e dunque anche il modo di interazione con il modulo \verb|REG|.

Abbiamo infatti che, nel caso di una \verb|ldr| mandiamo nell'ingresso \verb|ind B| di \verb|REG|
l'indirizzo nel campo $R_m$ (sorgente), nel caso di una \verb|str| mandiamo nell'ingresso
\verb|ind B| l'indirizzo nel campo $R_d$ (destinazione). Aggiungiamo quindi un multiplexer in grado
di effettuare tale scelta.
\begin{center}
	\includesvg[inkscapelatex=false, scale=0.7]{circuiti/store.svg}
\end{center}
Ancora una volta il bit di scelta sarà ricavabile dall'istruzione in linguaggio macchina.

\subsection{Istruzioni operative}
Supponiamo ora di eseguire un'istruzione operativa, come una somma di questo tipo
\begin{minted}{gas}
add r1, r2, r3
\end{minted}
Abbiamo un registro destinazione \verb|r1|, dunque dovremo scrivere nel modulo \verb|REG| e abbiamo
due registri sorgente di cui vogliamo sommare i valori.

Usando lo schema costruito fino ad ora siamo in grado di effettuare tale somma tramite la ALU posta
alle uscite di \verb|REG| e \verb|EXT|. Il problema è che il risultato viene mandato all'interno
della memoria dati e dunque non scriveremmo in \verb|r1| il valore \verb|r2 + r3| ma andremmo a
prendere il valore che in memoria si trova all'indirizzo \verb|r2 + r3|.

Per riuscire a distinguere dunque istruzioni operative da quelle di accesso alla memoria è
sufficiente aggiungere un multiplexer che prende come ingressi sia l'uscita della memoria dati sia
l'uscita della ALU.
\begin{center}
	\includesvg[inkscapelatex=false, scale=0.7]{circuiti/operative.svg}
\end{center}
Così facendo, tramite un bit di controllo è possibile selezionare uno o l'altro risultato e andarlo
a scrivere nel modulo \verb|REG|.

Se avessimo volutto effettuare un'operazione di \verb|sub| il circuito rimarrebbe il solito ma
dobbiamo aggiungere un bit di controllo sulla ALU per decidere se effettuare una somma o una
sottrazione.

\subsection{Istruzioni di salto}
Se volessimo eseguire istruzioni di salto dovremmo sostanzialmente andare a cambiare il valore del
program counter. Questo però non può essere fatto prima di essere passati dalla memoria istruzioni
e aver tradotto l'istruzione in binario in modo da reperire l'immediato che definisce come
effettuare il salto.

Ricordiamoci inoltre che dentro il modulo \verb|REG| abbiamo il registro \verb|r15| che è una copia
del valore del program counter. Andando quindi a reperire quel valore e sommandolo all'immediato
definito negli ultimi 24 bit dell'istruzione otteniamo il nuovo valore di \verb|PC|.

Per effettuare operazioni di salto dobbiamo sostanzialmente aggiungere due multiplexer: il primo in
grado di capire se leggere o meno il registro \verb|r15| e il secondo in grado di distinguere il
caso in cui il program counter viene incrementato normalmente o tramite un'istruzione di salto.
\begin{center}
	\includesvg[inkscapelatex=false, scale=0.7]{circuiti/salto.svg}
\end{center}
Come possiamo vedere il multiplexer che regola il primo ingresso di \verb|REG| può avere come
uscita la costante 15 in quanto, nel caso di un'istruzione di salto, non andiamo a specificare in
modo esplicito che vogliamo variare il \verb|PC|.

L'altro multiplexer prende come ingressi le uscite delle due ALU, quella che effettua l'incremento
standard e quella che somma al \verb|PC| l'offset specificato nell'istruzione.

\subsection{Unità di controllo}
Come possiamo vedere, tra multiplexer, ALU e memorie abbiamo un bel po' di moduli che hanno bisogno
di un segnale di controllo in grado di definire il loro comportamento.

Tali segnali sono generati e gestiti da una rete combinatoria chiamata \textbf{unità do controllo}
che tramite l'istruzione la sequenza di bit rappresentante l'istruzione assembler riesce a inviare
bit di controllo a multiplexer e ALU e segnali di \verb|WE| alle memorie esattamente quando serve.

La ALU che effettua istruzioni operative produce inoltre i flag di condizione (memorizzati nel
registro \verb|cpsr|) che invia all'unità di controllo in modo che questa abbia tutte le
informazioni necessarie anche per operazioni condizionali.
\begin{center}
	\includesvg[inkscapelatex=false, scale=0.6]{circuiti/ctrl_unit.svg}
\end{center}
All'interno dell'unità di controllo abbiamo due componenti principali: un \textbf{decoder} e una
rete combinatoria che implementa la \textbf{logica condizionale}.

Sia l'uno che l'altro hanno diversi ingressi provenienti dalla memoria istruzioni o dalla ALU che
potrebbe settare i flag per le istruzioni condizionali.

Affinché un'istruzione venga eseguita correttamente abbiamo bisogno che tutte le operazioni
attuabili da questo modello (nel caso peggiore) stiano all'interno di un ciclo di clock. Se il
ciclo di clock è ben definito, una volta che il segnale torna alto possiamo scrivere o leggere
registri o memoria ed eseguire operazioni aritmetico-logiche.

Supponendo di avere un ciclo di clock della durata di 1ns, possiamo dire di avere un frequenza di
clock di 1GHz. Questo significa che se abbiamo un programma di $n$ istruzioni, il tempo di
completamento sarà $n$ volte il ciclo di clock.

Solitamente non consideriamo quanto tempo il singolo programma impiega a terminare ma consideriamo
il \textbf{CPI} (Clock per Instruction) che, nel caso single cycle, equivale a 1.

\subsection{Shift}
Per le operazioni di \textbf{shift} in genere si usa un modulo chiamato \textbf{barrel shifter}
posizionato tra \verb|REG| e la \verb|ALU|. Non lo si mette all'interno della ALU perché, come
abbiamo visto, esistono delle istruzioni di accesso alla memoria che effettuano degli shift sul
valore dei registri per trovare l'indirizzo giusto.

Dobbiamo quindi effettuare prima gli eventuali shift logici o aritmetici e poi effettuare eventuali
somme o sottrazioni.
