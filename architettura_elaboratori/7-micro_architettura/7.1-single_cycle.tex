\chapter{Microarchitettura}
Vogliamo ora mettere insieme ciò che abbiamo visto nella parte di reti logiche e in quella di
assembler e cercare di progettare ed implementare un piccolo processore in grado di eseguire
il famoso ciclo \emph{fetch-decode-execute} di cui abbiamo parlato all'inizio.

Partiremo dalla modellazione di un processore di tipo \textbf{single cycle} e andando avanti
andremo ad modellare anche processori di tipo \textbf{multi-cycle}, \textbf{pipeline} e
\textbf{superscalari}.

\section{Processore Single cycle}
Proviamo ora a capire come modellare e dunque definire la $\mu$-architettura di un processore
single cycle in grado di eseguire alcune istruzioni di base tramite le reti combinatorie e
sequenziali viste in precedenza.

Al fine di ottenere il prima possibile una rappresentazione più generale della cosa ometteremo
alcune cose che però andranno ad arricchire più avanti il nostro modello.

Come abbiamo visto in assembler, abbiamo bisogno di