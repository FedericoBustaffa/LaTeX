\chapter{Reti sequenziali}
Le \textbf{reti sequenziali} ci servono ad implementare \textbf{macchine con stato} o
\textbf{automi}, i quali hanno bisogno di una componente di \textbf{memoria} che ci permetta di
salvare per l'appunto un certo \textbf{stato}.

Per riuscire ad implementare un automa abbiamo prima bisogno di un componente in grado di
memorizzare lo stato e questo è detto \textbf{registro}.

\section{Registri}
Per implementare un registro in grado di salvare lo stato di uno o più bit abbiamo prima bisogno
di implementare altre componenti.

\subsection{Latch SR}
Il primo oggetto di cui andiamo a parlare è chiamato \textbf{latch SR} dove SR sta per \emph{Set} e
\emph{Reset} ed è implementato in questo modo
\begin{center}
	\begin{circuitikz}
		\node[nor port] (or1) at (0, 1) {};
		\node[nor port] (or2) at (0, -1) {};

		\draw (-2.5, 52 |- or1.in 1) node[label=left:$R$] {} -- (or1.in 1);
		\draw (-2.5, 52 |- or2.in 2) node[label=left:$S$] {} -- (or2.in 2);

		\draw (or2.in 1) --++ (0, 0.5) -- (0.5, 0.5) |- (or1.out);
		\draw (or1.in 2) --++ (0, -0.5) -- (0.5, -0.5) |- (or2.out);

		\draw (0.5, 52 |- or1.out) to[short, *-] (1.5, 52 |- or1.out) node[label=right:$Q$]{};
		\draw (0.5, 52 |- or2.out) to[short, *-] (1.5, 52 |- or2.out) node[label=right:$\bar{Q}$]{};
	\end{circuitikz}
\end{center}
Quest'implementazione è pensata per avere l'uscita $Q$ a 1 quando l'ingresso $S$ è messo a 1 e,
finché non si imposta l'ingresso $R$ a 1 l'uscita $Q$ dovrebbe rimanere a 1. In questo modo, una
volta che impostiamo $S = 1$, l'uscita $Q$ rimane 1 anche se cambiamo l'ingresso $S$ a 0. Questo
metodo di memorizzazione presenta due problemi:
\begin{enumerate}
	\item Necessità di effettuare "manualmente" un reset ogni volta che vogliamo cancellare il
	      contenuto del registro.
	\item Ambiguità nel caso $S$ ed $R$ vadano nello stesso momento a 1: in questo caso si vuole
	      sia settare il bit a 1 sia resettarlo a 0. Come risultato otterremo entrambe le uscite a
	      0 in quanto entrambi gli ingressi a 1 fanno uscire 0 dalle due porte \verb|NOR|.
\end{enumerate}
Dato che $Q$ e $\bar{Q}$ dovrebbero essere l'uno l'opposto dell'altro e che si crea questa
situazione di ambiguità si è passati al D latch.

\subsection{D latch}
Il \textbf{D latch} introduce un \textbf{segnale di clock} costante e riduce il numero di ingressi
significativi ad uno, ossia $D$.

L'idea è avere un qualcosa che memorizza il valore di $D$ ogni volta che questo cambia. Quello che
succede è che, ad ogni ciclo di clock, all'interno del latch SR si effettua un reset e si salva il
valore di $S$ nel circuito.
\begin{center}
	\begin{circuitikz}
		\node[and port] (and1) at (0, 1) {};
		\node[and port] (and2) at (0, -1) {};
		\draw[thick] (1.5, -0.75) rectangle (3, 0.75);
		\node (latchSR) at (2.25, 0) {latch SR};

		\draw[dashed] (-3, 52 |- and1.in 1) node[label=left:clock] {} -- (and1.in 1);
		\draw[dashed] (-2.5, 52 |- and1.in 1) to[short, *-] (-2.5, 52 |- and2.in 1) -- (and2.in 1);

		\draw (-3, 52 |- and2.in 2) node[label=left:$D$] {} -- (and2.in 2);
		\draw (-2, 52 |- and2.in 2) to[short, *-] (-2, 52 |- and1.in 2) to[short, -o] (and1.in 2);

		\draw (and1.out) -- (1.5, 0.5) node[label=above left:$R$]{};
		\draw (and2.out) -- (1.5, -0.5) node[label=below left:$S$]{};

		\draw (3, 0.5) -- (4, 0.5) node[label=right:$Q$] {};
		\draw (3, -0.5) -- (4, -0.5) node[label=right:$\bar{Q}$] {};
	\end{circuitikz}
\end{center}
Anche questo approccio presenta alcune criticità, la più importante è che dobbiamo sempre mantenere
un $D$ significativo, poiché, se per una qualche ragione il valore di $D$ cambia inaspettatamente,
al ciclo di clock successivo verrà scritto il nuovo valore e perderemmo quindi il valore di $D$
memorizzato in precedenza.

Vogliamo quindi avere un qualcosa che ci permetta di decidere quando scrivere sia per evitare di
riscrivere continuamente lo stesso valore anche quando questo non cambia, sia per evitare di avere
costantemente un $D$ significativo in ingresso.

\subsection{Flip Flop}
Prima di ottenere il circuito sperato introduiciamo brevemente il circuito \textbf{Flip Flop}, il
quale, tramite due D latch concatenati (un \emph{master} e uno \emph{slave}), evita transizioni di
stato indesiderate salvando il dato significativo nel D latch slave.
\begin{center}
	\begin{tikzpicture}
		\draw[thick] (-2, 0) rectangle (-0.5, 1.5);
		\draw[thick] (1, 0) rectangle (2.5, 1.5);
		\node (master) at (-1.25, 0.75) {Master};
		\node (slave) at (1.75, 0.75) {Slave};

		\draw[dashed] (-3, 2) node[label=left:clock] {} -- (-1.25, 2) to[short, -o] (-1.25, 1.5);
		\draw[dashed] (-1.25, 2) to[short, *-] (1.75, 2) -- (1.75, 1.5);

		\draw (-3, 0.75) node[label=left:$D$] {} -- (-2, 0.75);

		\draw (-0.5, 1) -- (1, 1) node[label=above left:$Q$] {};
		\draw (-0.5, 0.5) to[short, -*] (0.25, 0.5) node[label=below left:$\bar{Q}$] {};

		\draw (2.5, 1) -- (3.5, 1) node[label=above left:$Q$] {};
		\draw (2.5, 0.5) -- (3.5, 0.5) node[label=below left:$\bar{Q}$] {};
	\end{tikzpicture}
\end{center}

\subsection{Registro da 1 bit}
Possiamo finalmente implementare, sulla base di quest'ultimo componente il nostro \textbf{registro}
da 1 bit introducendo un segnale di \textbf{enable} che regola le tempistiche di scrittura quando
necessario.

L'implementazione prevede semplicemente un \verb|AND| tra il segnale di clock in ingresso nel flip
flop e il segnale di \verb|ENABLE|. Quando tale segnale è a 1 la scrittura è abilitata e dunque al
primo ciclo di clock utile si scriverà il valore $D$ nel registro, quando invece il segnale va a 0
si disabilita la scrittura.
\begin{center}
	\begin{circuitikz}
		\draw[thick] (0, 0) rectangle (1.75, 1.5);
		\node (flip flop) at (0.875, 0.75) {Flip Flop};
		\node[and port] (and) at (-1.5, 1.25) {};

		\draw[dashed] (-4.5, 52 |- and.in 1) node[label=left:clock] {} -- (and.in 1);
		\draw (-4.5, 52 |- and.in 2) node[label=left:EN] {} -- (and.in 2);

		\draw (and.out) -- (0, 1.25);
		\draw (-1.5, 0.25) node[label=left:$D$] {} -- (0, 0.25);
		\draw (1.75, 0.75) --++ (1.5, 0) node[label=right:$Q$] {};
	\end{circuitikz}
\end{center}
Per effettuare un registro da 2 (o più) bit è sufficiente accostare due (o più) registri di questo
tipo dando in ingresso ad ognuno di essi il segnale il bit da salvare e condividendo i segnali di
clock ed \verb|ENABLE|.
\begin{center}
	\begin{tikzpicture}
		\draw[thick] (-2, 0) rectangle (-1, 1);
		\node (r1) at (-1.5, 0.5) {R};
		\draw[thick] (1, 0) rectangle (2, 1);
		\node (r2) at (1.5, 0.5) {R};

		\draw (-1.5, 1.5) -- (-1.5, 1);
		\draw (1.5, 1.5) -- (1.5, 1);

		\draw (-1.5, 0) -- (-1.5, -0.5) -- (0, -1) -- (0, -1.75) node[label=right:$Q$] {};
		\draw (1.5, 0) -- (1.5, -0.5) -- (0, -1);

		\draw[dashed] (-0.5, 1.5) node[label=above:clock] {} -- (-0.5, 0.75) -- (-1, 0.75);
		\draw[dashed] (-0.5, 0.75) to[short, *-] (1, 0.75);

		\draw (0.5, 1.5) node[label=above:EN] {} -- (0.5, 0.25) -- (1, 0.25);
		\draw (0.5, 0.25) to[short, *-] (-1, 0.25);
	\end{tikzpicture}
\end{center}
Per registri da 32 o 64 bit in realtà non si accostando 32 o 64 registri da un bit ma si procede in
un altro modo risparmiando molte componenti.

