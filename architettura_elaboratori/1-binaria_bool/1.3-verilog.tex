\section{Verilog}
Introduciamo brevemente il linguaggio \textbf{Verilog} per la simulazione ed implementazione di
piccole reti logiche. Nello specifico andremo a vedere com'è possibile implementare i circuiti
visti fino ad ora, senza specificare come funziona il linguaggio. Iniziamo con il rappresentare un
circuito qualsiasi tramite la sua tabella di verità.
\begin{minted}{verilog}
primitive rete(output out, input a, input b);
	table
		0 0 : 1;
		0 1 : 1;
		1 0 : 1;
		1 1 : 0;
	endtable
endprimitive
\end{minted}
In verilog, una \verb|primitive| può avere più ingressi ma una sola uscita, quindi nel caso
avessimo bisogno di una tabella di verità a più uscite dovremmo riscrivere una \verb|primitive| per
ogni uscita.

Dalla tabella di verità scritta sopra è possibile estrapolare un'espressione dell'algebra booleana
nel modo descritto nei paragrafi precedenti, ottenendo
\[ z = \bar{x} \bar{y} + \bar{x} y + x \bar{y} \]
In verilog possiamo usare una sintassi che ci permette di scrivere espressioni booleane in questo
modo
\begin{minted}{verilog}
module rete(output out, input a, input b);
	assign out = ~a & ~b | ~a & b | a & ~b;
endmodule
\end{minted}
A questo punto non ci rimane che testare la nostra rete tramite un programma "\verb|main|" che
possiamo definire in questo modo
\begin{minted}{verilog}
module test();
	reg a = 0, b = 0;
	wire out;
	rete r(out, a, b); // istanziazione della rete
	integer i = 0;
	
	initial begin // main
		$dumpfile("test.vcd"); // file con i risultati delle simulazione
		$dumpvars;

		for(i = 0; i < 4; i = i + 1) begin
            {a, b} = i; #3;
        end
    	$finish;
	end
endmodule
\end{minted}
Come abbiamo già fatto possiamo però usare le regole dell'algebra booleana per semplificare
l'espressione
\[ z = \bar{x} \bar{y} + \bar{x} y + x \bar{y} \]
e farla diventare
\[ z = \bar{x} + \bar{y} \]
e dunque possiamo creare un altro modulo di questo tipo
\begin{minted}{verilog}
module rete(output out, input a, input b);
	assign z = ~a | ~b;
endmodule
\end{minted}
che calcola la stessa cosa di prima.

\subsection{Multiplexer}
Vogliamo ora implementare un multiplexer come quello visto in precedenza, in cui si hanno tre
ingressi, $x$, $y$ e $c$ e una sola uscita $z$ che dipende da $c$. Come in precendenza, possiamo
scrivere una forma contratta della tabella di verità tramite il simbolo \verb|?| per indicare i
valori \textbf{non specificati}.
\begin{minted}{verilog}
primitive multiplexer(output out, input c, input a, input b);
	table
		0 0 ? : 0;
		0 1 ? : 1;
		1 ? 0 : 0;
		1 ? 1 : 1;
	endtable
endprimitive
\end{minted}
Se volessimo riscrivere la cosa in forma di algebra booleana abbiamo
\begin{minted}{verilog}
module multiplexer(output out, input c, input a, input b);
	assign out = ~c & a | c & b;
endmodule
\end{minted}
In maniera molto simile a prima testiamo il nostro circuito con un modulo che non fa altro che
provare tutte le possibili combinazioni dei valori di \verb|a|, \verb|b| e \verb|c|.

\subsection{Addizionatore}
Nel caso volessimo implementare un \textbf{addizionatore} di due sequenze, entrambe da 2 bit, le
cose si complicano leggermente. Un primo metodo per riuscire ad implementare tale oggetto consiste
nell'implementazione di un modulo addizionatore
\begin{minted}{verilog}
module adder(output s, output r_out, input a, input b, input r_in);
	adder_somma somma(s, a, b, r_in);
	adder_riporto riporto(r_out, a, b, r_in);
endmodule
\end{minted}
Dove \verb|adder_somma| è definito in questo modo
\begin{minted}{verilog}
primitive adder_somma(output s, input a, input b, input r_in);
	table
		0 0 0 : 0;
		0 0 1 : 1;
		0 1 0 : 1;
		0 1 1 : 0;
		1 0 0 : 1;
		1 0 1 : 0;
		1 1 0 : 0;
		1 1 1 : 1;
	endtable
endprimitive
\end{minted}
e \verb|adder_riporto| in questo modo
\begin{minted}{verilog}
primitive adder_riporto(output r_out, input a, input b, input r_in);
	table
		0 0 0 : 0;
		0 0 1 : 0;
		0 1 0 : 0;
		0 1 1 : 1;
		1 0 0 : 0;
		1 0 1 : 1;
		1 1 0 : 1;
		1 1 1 : 1;
	endtable
endprimitive
\end{minted}
A questo punto abbiamo un modulo che somma 2 bit con un 1 bit di riporto e ritorna 2 bit (somma e
riporto). Per effettuare la somma di due sequenze da 2 bit possiamo usare due di questi moduli
sommando i due bit meno significativi di ognuna delle due sequenze e dirottare il riporto
nell'ingresso del riporto del secondo modulo.
\begin{minted}{verilog}
module adder2(s, r_out, r_in, a, b);
	output [1:0]s;
	output r_out;
	input r_in;
	input [1:0]a;
	input [1:0]b;

	wire rip;
	adder a1(s[0], rip, r_in, a[0], b[0]);
	adder a2(s[1], r_out, rip, a[1], b[1]);
endmodule
\end{minted}
Ovviamente è possibile implementare \verb|adder_somma| e \verb|adder_riporto| sotto forma di
espressioni booleane e il risultato sarebbe il medesimo.
