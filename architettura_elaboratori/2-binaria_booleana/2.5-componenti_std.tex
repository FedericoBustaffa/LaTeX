\section{Componenti standard}
Andiamo ora a definire alcune delle componenti, con relativa implementazione, che andremo ad
utilizzare molto spesso d'ora in poi e che considereremo \textbf{componenti standard}.

Possiamo dividere tali componenti in \textbf{componenti di reti combinatorie} e
\textbf{componenti di reti sequenziali}.

\subsection{Componenti di reti combinatorie}
Le componenti che andremo a vedere sono le componenti principali per lo svolgimento dei calcoli
aritmetici e per la valutazione di espressioni booleane.

\subsubsection{Addizionatore}
La prima componente che andiamo a vedere è l'addizionatore di due sequenza da $n$ bit ciascuna che
restituisce una sequenza da $n$ bit e un riporto. Il componente lo indicheremo d'ora in poi in
questo modo
\begin{center}
	\begin{tikzpicture}[scale=0.75]
		\draw[thick] (-1, 0) -- (-1.5, 1.5) -- (-0.2, 1.5) -- (0, 1.25) -- (0.2, 1.5) -- (1.5, 1.5) -- (1, 0) -- cycle;
		\node (add) at (0, 0.75) {+};

		\draw (-1, 2) node[label=above:$x$] {} to[short, o-] (-1, 1.5);
		\draw (1, 2) node[label=above:$y$] {} to[short, o-] (1, 1.5);
		\draw (2, 0.75) node[label=right:$r_\text{in}$] {} to[short, o-] (1.25, 0.75);
		\draw[->, >=Stealth] (-1.25, 0.75) -- (-2, 0.75) -- (-2, -0.75) node[label=below:$r_\text{out}$] {};
		\draw[->, >=Stealth] (0, 0) -- (0, -0.75) node[label=below:$s$] {};
	\end{tikzpicture}
\end{center}
Come abbiamo visto in Verilog, è possibile implementare un addizionatore di due sequenze da 2 bit
ciascuna, concatenando due addizionatori da 1 bit. Questo però implica che ogni addizionatore
inizia il calcolo quando quello che ha calcolato la cifra meno significativa ha finito il suo
calcolo e generato un prodotto.

Dato che ogni addizionatore da 1 bit deve attraversare due livelli di porte logiche, se
concantenassimo $n$ addizionatori da 1 bit dovrebbe attendere $2 n \Delta t$ per effettuare una
somma a $n$ bit.

Si prova quindi a calcolare il riporto in anticipo con una tecnica chiamata \textbf{Carry look ahead}.
Questa tecnica tiene di conto che una somma tra due bit $x$ e $y$
\begin{itemize}
	\item \textbf{Genera} un riporto se e solo $x = y = 1$.
	\item \textbf{Propoaga} un riporto se e solo se $x = 1$ o $y = 1$.
\end{itemize}
Vale quindi che, data una certa colonna della somma, abbiamo un riporto se generiamo o propaghiamo
un riporto. Da questa considerazione ricaviamo la formula ricorsiva
\[ C_{i+1} = G_i + P_i \cdot C_i \]
dove $G_i = x_i \cdot y_i$ e $P_i = x_i + y_i$, che ci permette di calcolare il riporto $C$
generato da una somma di due sequenze di $i+1$ bit.

Così facendo abbiamo un circuito con $k$ porte in cascata per ogni sottosequenza di bit e a seconda
di quanto è lunga la sequenza principale e di quante sottosequenze abbiamo generato si dovrebbe
ottenere una leggera ottimizzazione in termini di livelli di porte da attraversare.

In questo modo il calcolo del riporto è indipendente dal calcolo della somma e dunque il calcolo
della somma successiva può iniziare in anticipo.

\subsubsection{Comparatore}
Componente utile per capire se due sequenza di bit $x$ e $y$ sono uguali oppure no. Per la
comparazione di due bit singoli abbiamo già la porta \verb|XNOR| che ritorna 1 se e solo se i due
bit hanno lo stesso valore. Un circuito in grado di comparare due sequenze di $n$ bit sarebbe
\begin{center}
	\begin{circuitikz}
		\node[xnor port] (xnor1) at (-0.5, 1.25) {};
		\node[xnor port] (xnor2) at (-0.5, -1.25) {};
		\node[and port] (and) at (2, 0) {};

		\draw (-3, 52 |- xnor1.in 1) node[label=left:$x_0$] {} -- (xnor1.in 1);
		\draw (-3, 52 |- xnor1.in 2) node[label=left:$y_0$] {} -- (xnor1.in 2);

		\node at (-1.5, 0) {$\vdots$};
		\node at (-3.5, 0) {$\vdots$};

		\draw (-3, 52 |- xnor2.in 1) node[label=left:$x_{n-1}$] {} -- (xnor2.in 1);
		\draw (-3, 52 |- xnor2.in 2) node[label=left:$y_{n-1}$] {} -- (xnor2.in 2);

		\draw (xnor1.out) -- (and.in 1);
		\draw (xnor2.out) -- (and.in 2);
		\draw (and.out) --++ (0.75, 0) node[label=right:$z$] {};
	\end{circuitikz}
\end{center}
D'ora in poi sarà rappresentato in questo modo
\begin{center}
	\begin{tikzpicture}
		\draw[thick] (-0.75, -0.5) rectangle(0.75, 0.5);
		\node (eq) at (0, 0) {=};

		\draw (-0.5, 1) node[label=above:$x$] {} to[short, o-] (-0.5, 0.5);
		\draw (0.5,  1) node[label=above:$y$] {} to[short, o-] (0.5, 0.5);
		\draw[->, >=Stealth] (0, -0.5) -- (0, -1) node[label=below:$z$] {};
	\end{tikzpicture}
\end{center}
Dove $x$ e $y$ sono entrambe due sequenze da $n$ bit e $z$ vale 0 se $x \neq y$ e vale 1 se $x = y$.

\subsubsection{ALU}
Altro componente fondamentale è la \textbf{ALU}, la quale ci permette di effettuare ben quattro
operazioni tra due sequenze di $n$ bit $x$ e $y$: \verb|AND|, \verb|OR|, somma e sottrazione.

Le prime tre operazioni hanno già un circuito che le implementa e che conosciamo, per la
sottrazione facciamo prima una considerazione.

Come abbiamo già anticipato, se abbiamo un numero positivo rappresentato in complemento a 2, per
farlo diventare negativo dobbiamo negarlo e sommarci 1. Di conseguenza vale che
\[ x - y = x + (\bar{y} + 1) \]
Siamo quindi in grado di utilizzare un addizionatore per svolgere anche le sottrazioni e dunque non
dobbiamo implementare un circuito apposito.
\begin{center}
	\begin{circuitikz}
		\draw[thick] (2, -4.25) -- (2, -2.25) -- (2.75, -2.75) -- (2.75, -3.75) -- cycle;
		\draw[thick] (4, -2.25) -- (4, -1.50) -- (4.25, -1.25) -- (4, -1) --
		(4, -0.25) -- (5, -0.75) -- (5, -1.75) -- cycle;
		\node at (4.625, -1.25) {+};
		\node[or port] (or) at (5, 4) {};
		\node[and port] (and) at (5, 2) {};
		\draw[thick] (7, -1.5) -- (7, 1.5) -- (8, 1) -- (8, -1) -- cycle;

		\draw[->, >=Stealth] (4.625, -5.5) node[label=below:OP] {} to[short, -*] (4.625, -4.75) --
		(2.375, -4.75) -- (2.375, -4);
		\draw[->, >=Stealth] (4.625, -4.75) -- (4.625, -2);
		\draw[->, >=Stealth] (4.625, -4.75) -- (7.5, -4.75) -- (7.5, -1.25);

		\draw[->, >=Stealth] (0, 1) node[label=left:$x$] {} to[short, -*]
		(2.5, 1) -- (2.5, -0.625) -- (4, -0.625);
		\draw (2.5, 1) -- (2.5, 52 |- or.in 1) -- (or.in 1);
		\draw (2.5, 52 |- and.in 1) to[short, *-] (and.in 1);

		\draw[->, >=Stealth] (0, -1) node[label=left:$y$] {} to[short, -*]
		(1.25, -1) -- (1.25, -3.75) to[short, -o] (1.625, -3.75) -- (2, -3.75);
		\draw[->, >=Stealth] (1.25, -2.75) to[short, *-] (2, -2.75);
		\draw (1.25, -1) -- (1.25, 52 |- and.in 2) -- (and.in 2);
		\draw (1.25, 52 |- and.in 2) to[short, *-] (1.25, 52 |- or.in 2) -- (or.in 2);

		\draw[->, >=Stealth] (2.75, -3.25) -- (3.25, -3.25) -- (3.25, -1.875) -- (4, -1.875);
		\draw[->, >=Stealth] (5, -1.25) -- (7, -1.25);
		\draw[->, >=Stealth] (6, -1.25) to[short, *-] (6, -0.5) -- (7, -0.5);
		\draw[->, >=Stealth] (4.5, -0.5) -- (4.5, 0) -- (5, 0);

		\draw[->, >=Stealth] (and.out) -- (5.125, 0.5) -- (7, 0.5);
		\draw[->, >=Stealth] (or.out) -- (6, 52 |- or.out) -- (6, 1) -- (7, 1);
		\draw[->, >=Stealth] (8, 0) -- (8.75, 0);
	\end{circuitikz}
\end{center}
L'entrata OP è un ingresso da 2 bit che ha come funzione principale la selezione dell'operazione
tramite il multiplexer a destra. \`E inoltre usato, nel caso si voglia effettuare una sottrazione
per rendere $y$ negativo tramite il multiplexer iniziale e tramite il riporto immesso
nell'addizionatore.

Supponiamo infatti di avere come configurazione per la scelta delle operazioni della ALU quella
indicata dalla seguente tabella
\begin{center}
	\begin{tabular}{c | c}
		OP input & operazione \\ \hline
		00       & +          \\
		01       & -          \\
		10       & \verb|AND| \\
		11       & \verb|OR|
	\end{tabular}
\end{center}
e inviamo in ingresso $x = 1$, $y = 1$ e $OP = 01$. Dato che $OP = 01$, il primo multiplexer
sceglierà come uscita $\bar{y}$ e invierà all'addizionatore un riporto di 1. Come risultato avremo
che l'addizionatore sommerà $x$ e $\bar{y} + 1$ ottenendo 0. Dato che l'ultimo multiplexer è
impostato per dare come uscita la sottrazione quando $OP = 01$ avremo il risultato richiesto.

\subsubsection{Shift}
In precedenza abbiamo parlato di \textbf{shift logico} e \textbf{shift aritmetico}. Vediamo ora
come è possibile implementare un circuito che ci permette di shiftare a destra una sequenza di bit
di un certo numero di posizioni con una sequenza di esempio di 4 bit.
\begin{center}
	\begin{tikzpicture}
		\draw[thick] (4.25, 3.625) -- (4.25, 1.625) -- (5, 2.125) -- (5, 3.125) -- cycle;

		\draw (0, 5) node[label=above:$x_3$] {} to[short, -*] (0, 4.5) -- (2.5, 4.5);
		\draw (0.5, 5) node[label=above:$x_2$] {} to[short, -*] (0.5, 4.25) -- (2.5, 4.25);
		\draw (1, 5) node[label=above:$x_1$] {} to[short, -*] (1, 4)   -- (2.5, 4);
		\draw (1.5, 5) node[label=above:$x_0$] {} to[short, -*] (1.5, 3.75) -- (2.5, 3.75);


		\draw (2.25, 3.5) node[label=left:\footnotesize0] {} -- (2.5, 3.5);
		\draw (0, 4.5) to[short, -*] (0, 3.25) -- (2.5, 3.25);
		\draw (0.5, 4.25) to[short, -*] (0.5, 3) -- (2.5, 3);
		\draw (1, 4) to[short, -*] (1, 2.75) -- (2.5, 2.75);

		\draw (2.25, 2.5) node[label=left:\footnotesize0] {} -- (2.5, 2.5);
		\draw (2.25, 2.25) node[label=left:\footnotesize0] {} -- (2.5, 2.25);
		\draw (0, 3.25) to[short, -*] (0, 2) -- (2.5, 2);
		\draw (0.5, 3) -- (0.5, 1.75) -- (2.5, 1.75);

		\draw (2.25, 1.5) node[label=left:\footnotesize0] {} -- (2.5, 1.5);
		\draw (2.25, 1.25) node[label=left:\footnotesize0] {} -- (2.5, 1.25);
		\draw (2.25, 1) node[label=left:\footnotesize0] {} -- (2.5, 1);
		\draw (0, 2) -- (0, 0.75) -- (2.5, 0.75);

		\draw[very thick] (2.5, 4.5) -- (2.5, 3.75);
		\draw[very thick] (2.5, 3.5) -- (2.5, 2.75);
		\draw[very thick] (2.5, 2.5) -- (2.5, 1.75);
		\draw[very thick] (2.5, 1.5) -- (2.5, 0.75);

		\draw[->, >=Stealth] (2.5, 4.125) -- (3.5, 4.125) -- (3.5, 3.375) -- (4.25, 3.375);
		\draw[->, >=Stealth] (2.5, 3.125) -- (3, 3.125) -- (3, 2.875) -- (4.25, 2.875);
		\draw[->, >=Stealth] (2.5, 2.125) -- (3, 2.125) -- (3, 2.375) -- (4.25, 2.375);
		\draw[->, >=Stealth] (2.5, 1.125) -- (3.5, 1.125) -- (3.5, 1.875) -- (4.25, 1.875);
		\draw[->, >=Stealth] (4.625, 1) -- (4.625, 1.875);
		\draw[->, >=Stealth] (5, 2.625) --++ (1, 0) node[label=right:$z$] {};
	\end{tikzpicture}
\end{center}
Come è possibile notare quando si shifta a destra si aggiungono tanti 0 a sinistra quanto
necessario e si ridirezionano i bit più significativi. Per uno shift aritmetico a destra avremmo
dovuto collegare l'ingresso del bit più significativo al posto degli 0.

Se avessimo voluto effettuare degli shift a sinistra lo schema sarebbe stato uguale tranne per il
fatto che sarebbero stati i bit più significativi i primi ad andare a 0.

\subsection{Componenti di reti sequenziali}
Le componenti standard implementate sotto forma di rete sequenziale sono di fatto registri e
memorie. In particolare andremo a parlare di
\begin{itemize}
	\item Registri e banchi di registri.
	\item Memorie: statiche dinamiche e ROM.
\end{itemize}
Dei registri da 1 bit abbiamo già parlato in precedenza, proseguiamo con le altre componenti anche
se riprenderemo il discorso sulle memorie più avanti in modo più approfondito.

\subsubsection{Banco di registri}
I banchi di registri sono una prima forma di \textbf{memoria} in grado di memorizzare più
\emph{parole} le quali sono accessibili tramite un \textbf{indirizzo}. Di seguito un banco da 2
registri.
\begin{center}
	\begin{tikzpicture}
		\draw[->, >=Stealth] (0, 0) node[label=left:EN] {} -- (1, 0); % EN
		\draw[thick] (1, -0.5) -- (1, 0.5) -- (1.5, 0.75) -- (1.5, -0.75) -- cycle; % DEMUX
		\draw[thick] (2.5, 0.75) rectangle (3.5, 0.25); % R1
		\draw[thick] (2.5, -0.75) rectangle (3.5, -0.25); % R2
		\draw[thick] (4.5, -0.75) -- (4.5, 0.75) -- (5, 0.5) -- (5, -0.5) -- cycle; % MUX
		\draw[->, >=Stealth] (5, 0) -- (6, 0) node[label=right:out] {}; % OUT
		\draw[->, >=Stealth] (1.5, 0.375) -- (2.5, 0.375); % demux -> R1
		\draw[->, >=Stealth] (1.5, -0.625) -- (2.5, -0.625); % demux -> R2

		% clock
		\draw[->, >=Stealth, dashed] (1.875, 0.625) -- (2.5, 0.625);
		\draw[->, >=Stealth, dashed] (1.875, -0.375) -- (2.5, -0.375);
		\draw[dashed] (1.875, 1.5) node[label=above:clock] {} -- (1.875, -0.375);

		\draw[->, >=Stealth] (3.5, 0.5) -- (4.5, 0.5); % R1 -> MUX
		\draw[->, >=Stealth] (3.5, -0.5) -- (4.5, -0.5); % R2 -> MUX

		% INPUT
		\draw[->, >=Stealth] (3, 1.5) node[label=above:IN] {} -- (3, 0.75);
		\draw[->, >=Stealth] (3, 1.125) to[short, *-] (4, 1.125) -- (4, 0) -- (3, 0) -- (3, -0.25);

		% INDIRIZZAMENTO
		\draw[->, >=Stealth] (0, -1.25) node[label=left:IND] {} to[short, -*] (1.25, -1.25) --
		(1.25, -0.625);
		\draw[->, >=Stealth] (1.25, -1.25) -- (4.75, -1.25) -- (4.75, -0.625);
	\end{tikzpicture}
\end{center}
Come possiamo vedere, tramite il segnale di IND riusciamo a scrivere IN su uno specifico registro
scelto dal demultiplexer, lo stesso IND viene inviato al multiplexer alla fine per riuscire a
leggere la locazione di memoria indirizzata.

Questo metodo fa uso di registri così come li abbiamo definiti in precedenza e consuma un numero di
transistor molto alto.

\subsubsection{Memorie dinamiche}
Le \textbf{memorie dinamiche} sono implementate in modo tale da permetterci di risparmiare più
transistor ma hanno il continuo bisogno di essere \emph{"rinfrescate"} andando a scrivere su di
esse altrimenti si rischia di perdere il valore.

Similmente ai banchi di memoria, abbiamo un demultiplexer a cui diamo in ingresso una parola da
scrivere e l'indirizzo a cui scriverla. L'output del demultiplexer è collegato ad una griglia
con un numero di righe pari al numero massimo di parole memorizzabili nella memoria e un numero
di colonne pari alla lunghezza in bit di ognuna delle parole.
\begin{center}
	\begin{tikzpicture}
		\draw[thick] (1, -0.5) -- (1, 0.5) -- (1.5, 0.75) -- (1.5, -0.75) -- cycle; % DEMUX

		% GRID
		\foreach \i in {-0.5, -0.25, 0, 0.25, 0.5}
		\draw[thick] (1.5, \i) -- (3, \i);

		\foreach \i in {1.75, 2, 2.25, 2.5, 2.75}
		\draw[thick] (\i, 0.75) -- (\i, -0.75);

		\draw[->, >=Stealth] (0, 0) node[label=left:IN] {} -- (1, 0);
		\draw[->, >=Stealth] (0, -1) node[label=left:IND] {} -- (1.25, -1) -- (1.25, -0.625);
		\draw[->, >=Stealth] (2.25, -1) -- (2.25, -1.875) node[label=right:OUT] {};
	\end{tikzpicture}
\end{center}
Tramite l'indirizzo scegliamo la riga, e la si attraversa per intero, se sulla colonna è presente
un 1 allora questo viene trasmesso in output.

\subsubsection{ROM}
Le \textbf{ROM} (Read Only Memory) sono molto simili, come struttura, alle memorie dinamiche. La
differenza sostanziale è che sulle colonne dove sono presenti gli 0 non c'è il collegamento con le
righe (NON CI HO CAPITO UNA SEGA).

\subsubsection{Memorie statiche}
Delle memorie statiche diciamo solo che non hanno il problema di dover essere costantemente
rinfrescate per mantenere i bit e sono più veloci delle memorie dinamiche ma occupano più spazio
sul silicio.
\begin{center}
	\begin{circuitikz}
		\node[not port] (not1) at (0, 1) {};
		\node[not port, rotate=180] (not2) at (0, -1) {};

		\draw (-2, 0) node[label=left:T1] {} -- (-1, 0) -- (-1, 1) -- (not1.in);
		\draw (not1.out) -- (1, 1) -- (1, 0);
		\draw (2, 0) node[label=right:T2] {} -- (1, 0) -- (1, -1) -- (not2.in);
		\draw (not2.out) -- (-1, -1) -- (-1, 0);
	\end{circuitikz}
\end{center}

Se volessimo comparare i tre tipi di memoria che abbiamo visto fino ad ora possiamo dire che
\begin{itemize}
	\item I registri sono i più veloci per quanto riguarda lettura e scrittura ma sono quelli
	      occupano più spazio sul silicio a parità di capienza.
	\item Le memorie dinamiche sono quelle più lente ma che occupano meno spazio sul silicio.
	\item Le memorie statiche si collocano a metà tra queste due per quanto riguarda velocità e
	      spazio occupato.
\end{itemize}
