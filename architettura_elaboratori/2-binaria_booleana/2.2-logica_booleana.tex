\section{Logica booleana}
Nel corso non andremo a trattare l'ultimo livello di astrazione, ossia quello più basso, ma andremo
a trattare lo strato soprastante, che tramite delle \textbf{porte logiche} e delle operazioni
aritmetiche binarie riesce a rappresentare quello che succede. Le tre operazioni implmentate dalle
porte logiche sono
\begin{itemize}
	\item \verb|AND(x, y)|: 1 se $x = y = 1$, 0 altrimenti.
	\item \verb|OR(x,y)|: 0 se $x = y = 0$, 1 altrimenti.
	\item \verb|NOT(x)|: 1 se $x = 0$, 0 se $x = 1$.
\end{itemize}
Altro strumento utile per capire meglio come funzionano tali porte e per vedere come funzionano
altre porte che risultano essere una combinazione di esse, sono le \textbf{tabelle di verità}.
Nelle tabelle di verità immettiamo tutti i possibili valori di input e calcoliamo i relativi output.
Per esempio, la tabella di verità di una porta logica \verb|AND| è la seguente
\begin{center}
	\begin{tabular}{c c | c}
		x & y & z \\ \hline
		0 & 0 & 0 \\
		0 & 1 & 0 \\
		1 & 0 & 0 \\
		1 & 1 & 1
	\end{tabular}
\end{center}
