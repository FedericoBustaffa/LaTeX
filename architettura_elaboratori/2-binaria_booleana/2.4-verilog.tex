\section{Verilog}
Introduciamo brevemente il linguaggio \textbf{Verilog} per la simulazione ed implementazione di
piccole reti logiche. Nello specifico andremo a vedere com'è possibile implementare i circuiti
visti fino ad ora, senza specificare come funziona il linguaggio. Iniziamo con il rappresentare un
circuito qualsiasi tramite la sua tabella di verità.
\begin{minted}{verilog}
primitive rete(output z, input x, input y);
	table
		0 0 : 1;
		0 1 : 1;
		1 0 : 1;
		1 1 : 0;
	endtable
endprimitive
\end{minted}
In verilog, una \verb|primitive| può avere più ingressi ma una sola uscita, quindi nel caso
avessimo bisogno di una tabella di verità a più uscite dovremmo riscrivere una \verb|primitive| per
ogni uscita.

Dalla tabella di verità scritta sopra è possibile estrapolare un'espressione dell'algebra booleana
nel modo descritto nei paragrafi precedenti, ottenendo
\[ z = \bar{x} \bar{y} + \bar{x} y + x \bar{y} \]
In verilog possiamo usare una sintassi che ci permette di scrivere espressioni booleane in questo
modo
\begin{minted}{verilog}
module rete(output z, input x, input y);
	assign z = (~x & ~y) | (~x & y) | (x & ~y);
endmodule
\end{minted}
A questo punto non ci rimane che testare la nostra rete tramite un programma "\verb|main|" che
possiamo definire in questo modo
\begin{minted}{verilog}
module test();
	reg a, b;
	wire c;
	rete r(c, a, b); // istanziazione della rete 
	
	initial begin // main
		$dumpfile("test.vcd"); // file con i risultati delle simulazione
		$dumpvars;

		a = 0; b = 0; #3 // attendi 3 unità di tempo
		b = 1; #3
		a = 1; b = 0; #3
		b = 1; #5
		$finish; // fine simulazione
	end
endmodule
\end{minted}