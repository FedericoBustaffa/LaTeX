\section{Algebra di Boole}
Per introdurre l'\textbf{algebra di Boole} introduciamo i seguenti \textbf{assiomi} che determinano
il comportamento dell'alfabeto $\{0, 1\}$ in relazione a delle operazioni di base che possiamo
fare con i suoi elementi.
\begin{gather*}
	a = 0 \implies a \neq 1 \quad \land \quad a = 1 \implies a \neq 0 \\
	a = 0 \implies \bar{a} = 1 \quad \land \quad a = 1 \implies \bar{a} = 0 \\
	0 \cdot 1 = 0 \quad \land \quad 1 \cdot 0 = 0 \\
	0 \cdot 0 = 0 \quad \land \quad 1 \cdot 1 = 1 \\
	0 + 1 = 1 \quad \land \quad 1 + 0 = 1 \\
	0 + 0 = 0 \quad \land \quad 1 + 1 = 1
\end{gather*}
Da questi deduciamo anche che
\begin{gather*}
	A \cdot 1 = A \quad \land \quad A + 0 = A \\
	A \cdot 0 = 0 \quad \land \quad A + 1 = 1 \\
	A \cdot A = A \quad \land \quad A + A = A \\
	A \cdot \bar{A} = 0 \quad \land \quad A + \bar{A} = 1 \\
	\bar{\bar{A}} = A
\end{gather*}
Le altre proprietà fondamentali per le operazioni di \verb|AND| e \verb|OR| nell'algebra booleana
sono
\begin{itemize}
	\item \textbf{Commutatività} per l'\verb|AND|: $A \cdot B = B \cdot A$
	\item \textbf{Commutatività} per l'\verb|OR|: $A + B = B + A$
	\item \textbf{Distributività}: $A \cdot (B + C) = A \cdot B + A \cdot C$ e la formula duale
	      $A + (B \cdot C) = (A \cdot B) + (A \cdot C)$
	\item \textbf{De Morgan}: $\overline{A \cdot B} = \bar{A} + \bar{B}$ e la formula duale
	      $\overline{A + B} = \bar{A} \cdot \bar{B}$
\end{itemize}
Con queste proprietà è possibile semplificare alcune delle formule generate da alcune tabelle di
verità come abbiamo fatto nel caso del multiplexer. Supponiamo che per un qualche motivo otteniamo
una funzione di $a$, $b$ e $c$ tale che
\[ f(a,b,c) = \bar{a} \bar{b} \bar{c} + a \bar{b} \bar{c} + a \bar{b} c \]
Se volessimo implementare questa formula tramite un circuito avremmo bisogno di tre porte
\verb|AND3| e di 1 porta \verb|OR3|. Usando le proprietà possiamo ottenere
\[
	\bar{a} \bar{b} \bar{c} + a \bar{b} \bar{c} + a \bar{b} c
	= \bar{b} \bar{c} (\bar{a} + a) + a \bar{b} c
	= \bar{b} \bar{c} + a \bar{b} c
\]
Passando così ad una formula che ci permette di implementare un circuito tramite due porte
\verb|AND3| e una porta \verb|OR3|. Proviamo un altro modo di procedere
\begin{align*}
	\bar{a} \bar{b} \bar{c} + a \bar{b} \bar{c} + a \bar{b} c
	 & = \bar{a} \bar{b} \bar{c} + a \bar{b} \bar{c} + a \bar{b} \bar{c} + a \bar{b} c         \\
	 & = \bar{b} \bar{c} (\bar{a} + a) + a \bar{b} (c + \bar{c}) = \bar{b} \bar{c} + a \bar{b}
\end{align*}
ottenendo così la possibilità di implementare un circuito tramite due porte \verb|AND2| e una porta
\verb|OR2|. Come possiamo vedere, a seconda di come usiamo queste proprietà, è possibile diminuire
notevolmente la dimensione dei circuiti e quindi la complessità di ciò che stiamo calcolando.
\begin{center}
	\begin{circuitikz}
		% gates
		\node[and port] (and1) at (3.5, 1) {};
		\node[and port] (and2) at (3.5, -1) {};
		\node[or port] (or) at (5.5, 0) {};

		\draw (0, 2) node[label=above:$a$] {} to[short, -*] (0, 52 |- and2.in 1) -- (and2.in 1);
		\draw (0.5, 2) node[label=above:$b$] {} to[short, -*] (0.5, 52 |- and1.in 1) to[short, -o] (and1.in 1);
		\draw (0.5, 52 |- and1.in 1) to[short, -*] (0.5, 52 |- and2.in 2) to[short, -o] (and2.in 2);
		\draw (1, 2) node[label=above:$c$] {} to[short, -*] (1, 52 |- and1.in 2) to[short, -o] (and1.in 2);

		\draw (and1.out) |- (or.in 1);
		\draw (and2.out) |- (or.in 2);
	\end{circuitikz}
\end{center}
A questo punto sarebbe possibile semplificare ulteriormente la formula raccogliendo $\bar{b}$ e
implementando il circuito descritto da
\[ \bar{b} \cdot (\bar{c} + a) \]
ma questo introduce un problema in quanto il circuito generato è asimmetrico, ossia i segnali in
ingresso non attraversano tutti lo stesso numero di porte come possiamo vedere in figura
\begin{center}
	\begin{circuitikz}
		\node[or port] (or) at (3.5, 1) {};
		\node[and port] (and) at (5.5, 0) {};

		\draw (0, 2) node[label=above:$a$] {} to[short, -*] (0, 52 |- or.in 1) -- (or.in 1);
		\draw (0.5, 2) node[label=above:$b$] {} to[short, -*] (0.5, 52 |- and.in 2) to[short, -o] (and.in 2);
		\draw (1, 2) node[label=above:$c$] {} to[short, -*] (1, 52 |- or.in 2) to[short, -o] (or.in 2);

		\draw (or.out) |- (and.in 1);
	\end{circuitikz}
\end{center}
Questo si traduce in un intervallo di tempo in cui la porta \verb|AND| riceve, da una parte il
vecchio segnale trasmesso dalla porta \verb|OR| prodotto al calcolo precedente, dall'altra l'ultimo
segnale prodotto dall'ingresso $b$.

Fino a che la porta \verb|OR| non finisce di elaborare i segnali in arrivo da $a$ e $c$ la porta
\verb|AND| potrebbe produrre risultati errati, dovuti a quello che viene chiamato \textbf{glitch}.

\subsection{Mappe di Karnaugh}
Come abbiamo appena visto, non sempre ridurre la complessità della nostra formula in modo
\emph{monotòno} ci porta alla migliore ottimizzazione. A volte conviene aumentare la complessità
per poi giungere ad un modello migliore.

Le \textbf{mappe di Karnaugh} forniscono un metodo grafico per riuscire a semplificare le formule
booleane senza però garantire la miglior minimizzazione di quest'ultime. Nell'esempio di prima
abbiamo una funzione booleana con la seguente tabella di verità
\begin{center}
	\begin{tabular}{c c c | c}
		$a$ & $b$ & $c$ & $f(a,b,c)$ \\ \hline
		0   & 0   & 0   & 1          \\
		0   & 0   & 1   & 0          \\
		0   & 1   & 0   & 0          \\
		0   & 1   & 1   & 0          \\
		1   & 0   & 0   & 1          \\
		1   & 0   & 1   & 1          \\
		1   & 1   & 0   & 0          \\
		1   & 1   & 1   & 0
	\end{tabular}
\end{center}
Da questa tabella possiamo ricavare una mappa di Karnaugh prendendo tutti i possibili valori di $a$
e mettendoli nella prima colonna e poi prendendo tutti i possibili valori della coppia $bc$ e
mettendoli sulla prima riga, disponendoli in modo che ogni valore differisca dal precedente al più
di un bit.

Il nostro obbiettivo è quello di individuare i quadrati o rettangoli contenenti un numero di 1 pari
ad una potenza di 2 e raggrupparli. Per tale raggruppamento è possibile
\begin{itemize}
	\item Uscire dalla tabella e rientrare dall'altra parte se ho degli 1 agli estremi.
	\item Includere degli 1 già raccolti in un precedente raggruppamento.
\end{itemize}
Nel nostro caso abbiamo due rettangoli da due 1: il primo verticale che prende la prima colonna per
intero e il secondo orizzontale che prende la prima metà della seconda riga.
\begin{center}
\begin{karnaugh-map}[4][2][1][$c$][$b$][$a$]
\maxterms{1, 2, 3, 6, 7}
\minterms{0, 4, 5}
\implicant{0}{4}
\implicant{4}{5}
\end{karnaugh-map}
\end{center}
A questo punto siamo
in grado di semplificare la formula di partenza
\begin{enumerate}
	\item Mettendo in \verb|AND| le variabili facenti parte dello stesso raggruppamento che
	      rimangono costanti e negando quelle con valore 0.
	\item Sommando tra di loro i raggruppamenti.
\end{enumerate}
Otteniamo così la formula ottenuta in precedenza con le proprietà dell'algebra booleana
\[ \bar{b} \bar{c} + a \bar{b} \]
in modo meccanico. Il primo termine della somma è ottenuto prendendo in considerazione il
raggruppamento verticale di 1 e considerando che $b$ e $c$ non variano ed essendo a 0 vengono
negati. Il secondo termini si ottiene similmente notando che $a$ e $b$ sono la parte costante del
raggruppamento ed inoltre $b$ è a 0 e dunque deve essere negato.

Prendiamo ora come esempio un \textbf{sommatore} di 2 bit con riporto, il cui funzionamento dipende
da tre parametri di ingresso: $x_1$ e $x_2$ i bit che vogliamo sommare e $r_0$ il possibile riporto
da aggiungere. Abbiamo inoltre due uscite: il risultato $s$ della somma e il possibile riporto $r_1$
generato da essa.
\begin{center}
	\begin{tikzpicture}
		\draw[thick] (0, 0) rectangle (2, 1.5);
		\node (label) at (1, 0.75) {ADD};

		\draw (0.5, 2) node[label=above:$x_1$] {} to[short, o-] (0.5, 1.5);
		\draw (1.5, 2) node[label=above:$x_2$] {} to[short, o-] (1.5, 1.5);
		\draw (2.5, 0.75) node[label=right:$r_0$] {} to[short, o-] (2, 0.75);
		\draw (0, 0.75) -- (-0.5, 0.75) -- (-0.5, -0.75) node[label=below:$r_1$] {};
		\draw (1, 0) -- (1, -0.75) node[label=below:$s$] {};
	\end{tikzpicture}
\end{center}
In questo caso la tabella di verità di tale oggetto è
\begin{center}
	\begin{tabular}{c c c | c | c }
		$x_1$ & $x_2$ & $r_0$ & $s$ & $r_1$ \\ \hline
		0     & 0     & 0     & 0   & 0     \\
		0     & 0     & 1     & 1   & 0     \\
		0     & 1     & 0     & 1   & 0     \\
		0     & 1     & 1     & 0   & 1     \\
		1     & 0     & 0     & 1   & 0     \\
		1     & 0     & 1     & 0   & 1     \\
		1     & 1     & 0     & 0   & 1     \\
		1     & 1     & 1     & 1   & 1
	\end{tabular}
\end{center}
Le mappe di Karnaugh per $s$ ed $r_1$ risultano le seguenti
\begin{figure}[h!]
\centering
\begin{subfigure}[b]{0.4\textwidth}
\centering
\begin{karnaugh-map}[4][2][1][$c$][$b$][$a$]
\minterms{1,2,4,7}
\maxterms{0,3,5,6}
\implicant{1}{1}
\implicant{2}{2}
\implicant{4}{4}
\implicant{7}{7}
\end{karnaugh-map}
\end{subfigure}
\begin{subfigure}[b]{0.4\textwidth}
\begin{karnaugh-map}[4][2][1][$c$][$b$][$a$]
\minterms{3,5,7,6}
\maxterms{0,1,2,4}
\implicant{3}{7}
\implicant{7}{6}
\implicant{5}{6}
\end{karnaugh-map}
\end{subfigure}
\end{figure}

Da tali mappe di Karnaugh ricaviamo le seguenti formule per $s$ ed $r_1$
\begin{align*}
	s   & = r_0 \bar{x_1} \bar{x_2} + \bar{r_0} \bar{x_1} x_2 + r_0 x_1 x_2 + \bar{r_0} x_1 \bar{x_2} \\
	r_1 & = x_1 x_2 + r_0 x_2 + r_0 x_1
\end{align*}
Di seguito raffiguriamo il circuito ricavato dalla formula per $r_1$.
\begin{center}
	\begin{circuitikz}
		\node[and port] (and1) at (3.5, 1.5) {};
		\node[and port] (and2) at (3.5, 0) {};
		\node[and port] (and3) at (3.5, -1.5) {};
		\node[or port, number inputs=3] (or) at (5.5, 0) {};

		\draw (0, 2) node[label=above:$x_1$] {} to[short, -*] (0, 52 |- and1.in 1) -- (and1.in 1);
		\draw (0, 52 |- and1.in 1) to[short, -*] (0, 52 |- and3.in 2) -- (and3.in 2);

		\draw (0.5, 2) node[label=above:$x_2$] {} to[short, -*] (0.5, 52 |- and1.in 2) -- (and1.in 2);
		\draw (0.5, 52 |- and1.in 2) to[short, -*] (0.5, 52 |- and2.in 2) -- (and2.in 2);

		\draw (1, 2) node[label=above:$r_0$] {} to[short, -*] (1, 52 |- and2.in 1) -- (and2.in 1);
		\draw (1, 52 |- and2.in 1) to[short, -*] (1, 52 |- and3.in 1) -- (and3.in 1);

		\draw (and1.out) -- (or.in 1);
		\draw (and2.out) -- (or.in 2);
		\draw (and3.out) -- (or.in 3);
		\draw (or.out) --++ (0.5, 0) node[label=right:$r_1$] {};
	\end{circuitikz}
\end{center}
Per riassumere possiamo usare sia le regole e gli assiomi dell'algebra booleana per semplificare le
formule ma questo potrebbe portarci sia alla minima forma possibile sia ad un'espressione più
complessa. Con le mappe di Karnaugh non abbiamo la certezza di ottenere la miglior minimizzazione
ma ci offre un modo meccanico per ridurre la complessità.

Vogliamo ora implementare un \textbf{moltiplicatore} che moltiplica due sequenze da 2 bit dando
come risultato una sequenza da 4 bit. Per capire come calcolare tale sequenza possiamo procedere
tramite una tabella di verità che però, avendo 4 bit di ingresso risulta avere 16 righe. Cerchiamo
quindi di rappresentare solo le righe significative.
\begin{center}
	\begin{tabular}{c c c c | c c c c}
		$x_1$ & $x_2$ & $y_1$ & $y_2$ & $z_1$ & $z_2$ & $z_3$ & $z_4$ \\ \hline
		0     & 0     & -     & -     & 0     & 0     & 0     & 0     \\
		-     & -     & 0     & 0     & 0     & 0     & 0     & 0     \\ \hline
		0     & 1     & 0     & 1     & 0     & 0     & 0     & 1     \\
		      &       & 1     & 0     & 0     & 0     & 1     & 0     \\
		      &       & 1     & 1     & 0     & 0     & 1     & 1     \\ \hline
		1     & 0     & 0     & 1     & 0     & 0     & 1     & 0     \\
		      &       & 1     & 0     & 0     & 1     & 0     & 0     \\
		      &       & 1     & 1     & 0     & 1     & 1     & 0     \\ \hline
		1     & 1     & 0     & 1     & 0     & 0     & 1     & 1     \\
		      &       & 1     & 0     & 0     & 1     & 1     & 0     \\
		      &       & 1     & 1     & 1     & 0     & 0     & 1     \\
	\end{tabular}
\end{center}
A questo punto possiamo disegnare una mappa di Karnaugh per ogni uscita $z_i$ che abbiamo,
limitiamoci per il momento a disegnare solo quelle di $z_1$ e $z_3$.
\begin{figure}[h!]
\centering
\begin{subfigure}[b]{0.4\textwidth}
\centering
\begin{karnaugh-map}[4][4][1][$y_2$][$y_1$][$x_2$][$x_1$]
\minterms{15}
\maxterms{0,1, 3, 2, 4, 5, 7, 6, 12, 13, 14, 8, 9, 11, 10}
\implicant{15}{15}
\end{karnaugh-map}
\end{subfigure}
\begin{subfigure}[b]{0.4\textwidth}
\begin{karnaugh-map}[4][4][1][$y_2$][$y_1$][$x_2$][$x_1$]
\minterms{6, 7, 15, 13, 9, 10}
\maxterms{0, 1, 3, 2, 4, 5, 12, 14, 8, 11}
\implicant{7}{15}
\implicant{7}{6}
\implicant{13}{9}
\implicant{10}{10}
\end{karnaugh-map}
\end{subfigure}
\end{figure}

Come possiamo vedere anche dalle formule che seguono, per il calcolo di $z_1$ è sufficiente una
porta \verb|AND| mentre per il calcolo di $z_3$ sono necessarie 4 porte \verb|AND| e 1 porta
\verb|OR|. C'è quindi un ritardo tra il calcolo di $z_1$ e $z_3$.
\begin{align*}
	z_1 & = x_1 x_2 y_1 y_2                                                                   \\
	z_3 & = x_1 \bar{y_1} y_2 + x_2 y_1 y_2 + \bar{x_1} x_2 y_1 + x_1 \bar{x_2} y_1 \bar{y_2}
\end{align*}