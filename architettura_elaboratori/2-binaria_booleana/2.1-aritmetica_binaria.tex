\chapter{Aritmetica binaria e reti logiche}
Passiamo ora ad un breve richiamo sull'aritmetica binaria e sulla logica booleana per capire meglio
cosa verrà utilizzato dal calcolatore per svolgere i calcoli.

\section{Aritmetica binaria}
Il sistema di numerazione binario, così come quelli decimale ed esadecimale, è detto
\textbf{posizionale}. Si assegna cioè un \textbf{peso} alle possibili posizioni in cui può cadere
una delle cifre a cui poi verrà moltiplicata la cifra effettiva.

Il numero $123_{10}$ si ottiene dalla somma delle sue singole cifre moltiplicate per il peso della
posizione in cui si trovano. Nello specifico, se la base di numerazione è $b$, partendo dalla
posizione più a destra, il peso della prima posizione sarà $b^0$ mentre il peso dell'$n$-esima
posizione sarà $b^{n-1}$.
\[ 123_{10} = 1 \cdot 10^2 + 2 \cdot 10^1 + 3 \cdot 10^0 = 100 + 20 + 3 \]
Se ad esempio volessimo fare lo stesso per il numero $1010_2$ otterremmo
\[ 1010_2 = 1 \cdot 2^3 + 0 \cdot 2^2 + 1 \cdot 2^1 + 0 \cdot 2^0 = 8 + 2 = 10_{10} \]
Come possiamo vedere abbiamo anche trovato un modo semplice per la conversione da binario a
decimale. In genere per la rappresentazione in macchina viene usata la notazione
\textbf{esadecimale} poiché, avendo 16 simboli a disposizione, fornisce una rappresentazione più
compatta di numeri binari altrimenti molto lunghi e difficilmente leggibili.
\[ 1F0_{16} = 1 \cdot 16^2 + 15 \cdot 16^1 + 0 \cdot 16^0 = 256 + 240 = 496_{10} \]
L'operazioni più semplice che possiamo fare tra numeri binari è ovviamente la \textbf{somma} che
funziona in modo analogo
\begin{center}
	\begin{tabular}{c | c c}
		$2_{10}$ & $0010$ & + \\
		$3_{10}$ & $0011$ & = \\ \hline
		$5_{10}$ & $0101$
	\end{tabular}
\end{center}

Per la \textbf{conversione} da decimale a binario possiamo
\begin{enumerate}
	\item Prendere la più grande potenza di 2 minore o uguale del numero che vogliamo convertire.
	\item Sottraiamo al numero la potenza di 2 trovata.
	\item Ripetiamo il procedimento con la differenza ottenuta finché non si ottiene 0.
	\item Quando si ottiene 0 si va a mettere un 1 in ognuna delle posizioni relative alle potenze
	      trovate.
\end{enumerate}
Se ad esempio volessimo rappresentare $17_{10}$ troveremmo che $2^4 = 16$ è la più grande potenza
di 2 minore o uguale di 17. Quindi eseguiamo $17 - 16 = 1$ a questo punto abbiamo che $2^0 = 1$ è
la più grande potenza minore o uguale di 1. Eseguiamo $1-1 = 0$ e concludiamo. Il risultato sarà
\[ 17_{10} = 10001_2 \]
dove gli 1 si trovano rispettivamente in posizione 1 e 5 ossia le posizioni di peso $2^0$ e $2^4$.

\section{Rappresentazione in macchina}
Se volessimo rappresentare numeri in macchina avendo a disposizione registri di $n$ bit potremmo
rappresentare $2^n$ possibili numeri (in particolare in numeri da 0 a $2^n - 1$).

Si ha però un problema nel momento in cui si vogliono rappresentare numeri negativi. Supponiamo di
voler rappresentare i numeri da $-x$ a $+x$, il primo metodo di rappresentazione dei numeri
negativi è il cosiddetto \textbf{modulo e segno}.

Questo metodo utilizza un bit per il segno e i restanti vengono usati per la rappresentazione come
abbiamo visto fin ad ora.

Per quanto riguarda i numeri positivi il bit di segno sarà a 0 e per i bit negativi sarà a 1. La
restante rappresentazione del numero rimane invariata. Notiamo che adesso, con registri da $n$ bit
ne possiamo usare $n-1$ per la rappresentazione e dunque avremo a disposizione i numeri da
$-2^{n-1}$ a $+2^{n-1}$.

Questa rappresentazione si porta dietro due problemi principali: il primo è che rappresenta due
volte lo 0 (-0 e +0) e poi introduce la necessità di una componente in grado di fare le sottrazioni
(necessaria quando si sottrae un numero più grande ad un numero più piccolo).

\subsection{Rappresentazione in complemento a 2}
Si è quindi passati alla rappresentazione in \textbf{complemento a 2}, la quale prevede che, per i
numeri positivi si utilizzino i primi $n-1$ bit per la rappresentazione e l'$n$-esimo bit messo a 0.
Per i numeri negativi si deve invece rappresentare prima di tutto il modulo del numero, dopodiché
si negano tutti i bit e infine si somma 1.

Se ad esempio volessimo rappresentare $-3$ in complemento a 2 su 8 bit il risultato sarebbe il
seguente:
\begin{enumerate}
	\item Rappresentazione 3 in base 2: $00000011_2$.
	\item Negazione di tutti i bit $00000011_2 \to 11111100_2$.
	\item Sommiamo 1 al risultato $11111100 + 1 = 11111101$.
\end{enumerate}
Come possiamo vedere otteniamo una rappresentazione in cui il numero più a sinistra è 1, e quindi
capiamo che il numero è negativo. Se volessimo sapere di quale numero si tratta ci basta fare il
procedimento inverso negando tutti i bit
\[ 11111101 \to 00000010 \]
e sommandoci 1
\begin{center}
	\begin{tabular}{r c}
		00000010 & + \\
		1        & = \\ \hline
		00000011
	\end{tabular}
\end{center}
che infatti è 3. Questo metodo ci permette di effettuare sottrazioni semplicemente facendo somme
con numero negativi. Se per esempio volessimo effettuare $2 - 3$ otterremmo
\begin{center}
	\begin{tabular}{r c}
		00000010 & + \\
		11111101 & = \\ \hline
		11111111
	\end{tabular}
\end{center}
Che se convertiamo in modulo diventa $11111111 \to 00000000 + 1 = 00000001$ ovvero 1 e quindi
deduciamo facilmente che il risultato della somma appena fatta è $-1$.

Altro fatto importante è che l'unica rappresentazione dello 0 è quella con tutti i bit messi a 0,
mentre la rappresentazione
\[ 10000000 \]
non è valida o comunque non viene mai generata da possibili calcoli in aritmetica binaria.

\subsection{Shift dei bit}
Quando si parla di moltiplicazioni e divisioni, un caso particolare è quello in cui si moltiplica
o si divide un numero per la base di numerazione o per una sua potenza. Se ad esempio volessimo
svolgere $123 \times 10$ è immediato riconoscere che il risultato è $1230$ poiché come sappiamo
basta aggiungere uno zero a destra.

Questo però implica uno \textbf{shift} delle cifre dalla lora posizione ad una posizione a peso
maggiore (a sinistra). Allo stesso modo, vale $123 / 10 = 12.3$, ottenuto tramite un shift a destra
delle cifre.

In modo del tutto analogo questo è anche possibile in aritmetica binaria ma invece di moltiplicare
o dividere per 10, moltiplichiamo o dividiamo per 2. Per esempio se dividiamo $10_{10} = 1010_2$
per 2 abbiamo, in decimale che il risultato è 5 e in binario, dato che dividiamo per la base ci
basterà shiftare tutti i bit a destra di una posizione ottenendo
\[ 0101 = 2^2 + 2^0 = 4 + 1 = 5_{10} \]
Se invece volessimo moltiplicare o dividere per una potenza di 2 dovremmo effettuare tanti shift
(a destra o a sinistra) quanto vale l'esponente. Per esempio $10 \times 2^2$ equivale ad effettuare
un doppio shift verso sinistra del numero $1010$, ottenendo così
\[ 101000 = 2^5 + 2^3 = 32 + 8 = 40_{10} \]
Questo ci sarà molto utile nell'indirizzamento della memoria che vedremo più avanti in quanto la
memoria del computer è partizionata in un numero di blocchi pari ad una potenza di due. I blocchi
sono a loro volta suddivisi in un numero di pagine sempre pari ad una potenza di 2. Questi ci
permette di muoverci tra i blocchi o le pagine della memoria in modo molto più agevole e veloce
tramite le operazioni di shift.

Per quanto riguarda i numeri negativi rappresentati in complemento a 2 dobbiamo avere una piccola
accortezza. Prendiamo per esempio l'operazione
\[ -8 / 2^2 = -2 \]
che in binario equivale a shiftare a destra di due posizioni il numero $11111000$ (-8 in
complemento a 2). Dato che però, tramite un normale shift a destra otterremmo $00111110$, cioè un
numero positivo, è chiaro che non è il risultato corretto. Semplicemente negli shift a destra di
numeri negativi le posizioni a sinistra vengono rimpiazzate da 1 invece che da 0, otteniamo quindi
$11111110$ che, se effettuiamo la conversione in complemento a 2, equivale a $00000010_2 = 2_{10}$.

Per quanto riguarda invece gli shift a sinistra di numeri negativi in complemento a 2 si effettua
un normale shift aggiungendo tanti 0 a destra quanto vale l'esponente della potenza per cui si
vuole moltiplicare. Per esempio
\[ -8 \times 2^2 = -32 \]
per ottenere il calcolo in binario dobbiamo shiftare a sinistra il numero $11111000$. Otteniamo
quindi $11100000$ che se convertito in modulo diventa $00100000_2 = 2^5 = 32_{10}$

\subsection{Rappresentazione in virgola mobile}
La rappresentazione in virgola mobile ha subito diverse variazioni negli anni fino ad arrivare ad
uno standard che ad oggi prende il nome di IEEE 754, il quale prevede l'utilizzo di
\begin{itemize}
	\item Un bit per il \textbf{segno}.
	\item Un certo numero di bit per l'\textbf{esponente}.
	\item I restanti bit per la \textbf{mantissa}.
\end{itemize}
Il numero di bit per esponente e mantissa dipende dalla lunghezza dei registri che utilizziamo (se
a 32 o a 64 bit).
\begin{itemize}
	\item Se il bit del segno è 0 il numero è positivo, se invece è 1 allora il numero è negativo.
	\item L'esponente indica il numero a cui elevare la base.
	\item La mantissa rappresenta il numero con una virgola in posizione \emph{"fissa"}.
\end{itemize}
Il numero $x$ cercato è rappresentato come
\[ x = \text{sign}(x) \cdot b^e \cdot m \]
L'esponente è rappresentato in \textbf{eccesso a k}, ovvero se l'esponente è $e$, il calcolatore
rappresenta $e+k$ in modo che numeri più grandi di $k$ rappresentano numeri positivi mentre numeri
più piccoli di $k$ rappresentano numeri negativi. Per lo standard IEEE 754, nel caso di numeri in
virgola mobile a
\begin{itemize}
	\item 32 bit, l'esponente è rappresentato con 8 bit in eccesso a 127 e la mantissa ha a
	      disposizione 23 bit.
	\item 64 bit, l'esponente è rappresentato con 11 bit in eccesso a 1023 e la mantissa ha a
	      disposizione 52 bit.
\end{itemize}
Aggiungiamo inoltre che lo standard vuole che la prima cifra per una rappresentazione in virgola
mobile sia non nulla e tale che alla sua sinistra vi siano solo cifre non significative
\[ 123.45 = 1.2345 \times 10^2 \]
Questo implica che le operazioni tra numeri rappresentati in virgola mobile non sono così immediate
nel caso in cui questi abbiano ordini di grandezza diversi.

Quel che viene fatto è passarli ad un \textbf{incolonnatore} che, tramite alcune operazioni riesce
a spostare la virgola e cambiare l'esponente dei due operandi in modo che il calcolo sia eseguibile
nel modo classico. Una volta effettuato il calcolo, il risultato passa da un \textbf{normalizzatore}
che riporta il risultato ad una rappresentazione che soddisfa lo standard.

Se vogliamo implementare la possibilità di eseguire operazioni di addizione tra numeri interi e tra
numeri in virgola mobile, abbiamo bisogno di una componente detta \textbf{ALU}. Per le operazioni
tra interi si parla di \textbf{ALU-I}, mentre per le operazioni tra numeri in virgola mobile si
parla di \textbf{ALU-FP}.

\subsection{Rappresentazione di caratteri}
In ultima battuta parliamo della rappresentazione di documenti testuali, i quali non sono altro
che \textbf{sequenze di caratteri}. Tali caratteri seguono la rappresentazione \textbf{ASCII} che
fa utilizzo di 8 o 16 bit.

Per sapere cosa rappresenta la sequenza di bit si fa utilizzo di una \textbf{tabella ASCII} che
mette in corrispondenza il numero rappresentato dalla sequenza di bit ad un carattere nella tabella.

In questo ambito è possibile effettuare piccole operazioni come \textbf{ordinamenti lessicografici}
poiché, per esempio, le lettere da "A" a "Z" hanno un codice che le mette in ordine una dopo
l'altra.