\part{Architettura}

\chapter{Assembler}
In questa parte tratteremo l'\textbf{architettura} saltando per il momento $\mu$-architettura che
tratteremo più avanti. In particolare tratteremo i costrutti del linguaggio \textbf{assembly}
versione \textbf{ARMv7} nel quale saranno disponibili 16 registri e memorie da 4GB con indirizzi
da 32 bit.

L'insieme di istruzioni è di tipo \textbf{RISC} (Reduced Instruction Set Computer), ossia un
insieme di istruzioni ridotto. I motivi per cui lo adottiamo sono
\begin{itemize}
	\item La regolarità favorisce la semplicità.
	\item Il caso più comune deve essere implementato in modo più veloce.
	\item Implementare un calcolatore più piccolo e semplice porta dei vantaggi.
	\item Un buon progetto richiede buoni compromessi.
\end{itemize}
Tramite il linguaggio assembler andremo di fatto ad utilizzare le componenti descritte in
precedenza tramite reti logiche.

Come vedremo il linguaggio è composto di sole istruzioni elementari e quindi ci darà molto più
controllo su ciò che stiamo facendo, pagando però il prezzo di scrivere molte più linee di codice
rispetto a linguaggi di più alto livello come C o Java.

\section{Strumenti per la programmazione}
Per programmare in assembler ARM sotto Linux possiamo installare il cross-compiler per ARM e
relativi tool di debugging tramite la seguente riga di codice
\begin{minted}{bash}
$ sudo apt install gcc-arm-linux-gnueabihf qemu-user gdb-multiarch
\end{minted}
Per la compilazione e l'esecuzione del debugger le istruzioni sono le seguenti
\begin{minted}{bash}
$ arm-linux-gnueabihf-gcc -ggdb3 -static <sorgenti> -o <eseguibile>
$ qemu-arm -g <porta> <eseguibile> &
$ gdb-multiarch -q --nh -ex "target remote localhost:<porta>" <eseguibile>
\end{minted}
Una volta dentro il debugger possiamo impostare break points e usare le istruzioni
\begin{itemize}
	\item \verb|continue|: per eseguire il codice fino al prossimo break point.
	\item \verb|next|: per eseguire le istruzioni una riga alla volta.
	\item \verb|tui reg general|: per visualizzare un'interfaccia in grado di farci vedere il
	      valore dei vari registri man mano che eseguiamo le istruzioni.
\end{itemize}
In alternativa è possibile usare lo strumento online \href{https://cpulator.01xz.net/}{CPUlator}.
