\section{Costrutti di programmazione strutturata}
In assembler non esistono costrutti di programmazione a cui siamo abituati come \verb|if-else|,
\verb|while| e \verb|for|, è comunque possibile implementarli seguendo alcune regole

Per implementare un \verb|if-then| sostanzialmente eseguiamo una \verb|cmp| seguita da un comando
di \emph{branching} e dal body
\begin{minted}{gas}
cmp cond
b!cond endif
// ...
// then body
// ...
endif:
\end{minted}
per implementare un \verb|if-then-else|, il mancato verificarsi della condizione fa saltare al ramo
\emph{else}, viceversa il verificarsi della condizione permette l'esecuzione del ramo \emph{then}
seguito da un salto incondizionato alla fine dell'\emph{if}.
\begin{minted}{gas}
cmp cond
b!cond else
// ...
// then body
// ...
b endif
else:
// ...
// else body
// ...
\end{minted}
Non andiamo a trattare costrutti come lo \verb|switch| perché sono analoghi. Per quanto riguarda
invece i costrutti \textbf{iterativi} vediamo come implementare il \verb|while| e il \verb|for|.
In assembler in realtà non si fa differenza tra un \verb|while| e un \verb|for| ma si parla più in
generale di \textbf{loop} i quali si implementano tramite un'etichetta \verb|loop| dove andiamo a
testare la condizione e, in caso questa non sia verificata si salta ad un'etichetta che rappresenta
la fine del \verb|loop|.
\begin{minted}{gas}
loop:
	cmp cond
	b!cond endloop
	// ...
	// body
	// ...
	b loop
endloop:
\end{minted}
ovviamente prima del salto che permette di tornare all'etichetta \verb|loop| dobbiamo eseguire una
qualche operazione che permette di cambiare la condizione su cui stiamo iterando altrimenti
rimarremmo bloccati in un ciclo infinito.

A volte può essere utile definire un'etichetta \verb|next| (in genere alla fine del loop) dove
andiamo a svolgere delle operazioni che definiscono l'iterazione e come questa varia
\begin{minted}{gas}
loop:
	cmp cond
	b!cond endloop
	// ...
	// body
	// ...
next:
	// per esempio i++
	b loop
endloop:
\end{minted}
Questo risulta utile quando nel corpo del loop abbiamo un si verifica una condizione tale per cui
vogliamo saltare direttamente all'iterazione successiva senza eseguire il resto delle istruzioni.

Per implementare loop che eseguono almeno un'iterazione (\verb|do-while|) basta spostare il
controllo della condizione d'uscita alla fine del loop.