\section{Introduzione all'assembly}
Le istruzioni di calcolo possiamo immaginarcele divise in quattro campi: \textbf{codice operativo},
che ci dice sostanzialmente cosa fare e tre \textbf{operandi}, il primo è quello di
\emph{destinazione}, il secondo e il terzo sono detti operandi \emph{sorgente}. Tipicamente gli
operandi nelle istruzioni di calcolo sono dei registri indicati con \verb|r0|, \verb|r1| e così
via fino a \verb|r15|. Per esempio l'istruzione
\begin{minted}{gas}
add r0, r0, r1
\end{minted}
prende il contenuto dei registri \verb|r0| ed \verb|r1|, li somma e mette il risultato in \verb|r0|.
Se invece volessimo effettuare la comparazione di due registri, la sintassi sarebbe la seguente
\begin{minted}{gas}
cmp r0, r1
\end{minted}
la quale prende il contenuto dei due registri e scrive il risultato nella \textbf{parola di stato},
che, come vedremo, è un registro interno al processore fatto per memorizzare proprio questo tipo
di risultati e che sarà utile per le istruzioni di controllo, la cui sintassi sono del tipo
\begin{minted}{c}
bxx <etichetta>
\end{minted}
dove \verb|b| sta per \textbf{branch} e al posto di \verb|xx| possiamo mettere due caratteri che
possono significare \textbf{uguaglianza} (\verb|EQ|), \textbf{disuguaglianza} (\verb|NE|),
\textbf{minoranza a sinistra} (\verb|LT|) e così via. L'\textbf{etichetta}, seguita dai due punti,
indica dove muovere il cursore della prossima istruzione.

Proviamo ora a realizzare un programma che divide il numero che sta in un registro per il numero
che sta in un registro \verb|r1| per il numero che sta in un altro registro \verb|r2|.

Dato che non disponiamo dell'operazione di divisione in ARMv7, possiamo provare ad implementare un
algoritmo servendoci solo di somme e sottrazioni. Possiamo sottrarre il divisore al dividendo,
facendo diventare il risultato il nuovo dividendo, un numero di volte tale che quest'ultimo sia
inferiore al divisore. Una volta arrivati a tal punto il risultato sarà dato dal numero di volte
che abbiamo sottratto e il resto sarà dato dal risultato dell'ultima sottrazione.
\begin{minted}{gas}
.text
.global main

main:
	mov r1, #7
	mov r2, #3
	mov r0, #0

divisione:
	cmp r1, r2
	blt fine
	sub r1, r1, r2
	add r0, r0, #1
	b divisione

fine:
	mov r0, r0
\end{minted}
Dove effettuiamo l'incremento del contatore in \verb|r0| è presente un'istruzione \verb|#1| che
indica la \textbf{costante} 1.