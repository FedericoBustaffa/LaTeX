\section{Funzioni}
Vediamo ora come sia possibile realizzare \textbf{funzioni} e \textbf{procedure} in assembler
andando a capire come implementare una \textbf{chiamata} di funzione, il \textbf{ritorno} dalla
funzione e come effettuare il \textbf{passaggio di parametri}.

Vogliamo anche capire come riuscire a richiamare del codice già compilato e come generare codice
che permette di essere eseguito da altri.

\subsection{Passaggio di parametri}
Prima di entrare nello specifico definiamo alcune convenzioni sul passaggio dei parametri. I
parametri di una funzione sono gestiti dai registri assegnado ad \verb|r0| il primo parametro,
ad \verb|r1| il secondo parametro e così via.

In realtà questa regola vale fine a \verb|r3| e se abbiamo più parametri questi vanno a finire
nello stack. L'ordinamento è una convenzione rispettata dalle funzioni della \verb|glibc|. Ne è un
esempio la \verb|printf|, che come sappiamo, prende una stringa che indica il formato e una serie
di parametri di cui vogliamo stampare il valore.
\begin{minted}{gas}
.data
fmt: .string "valore: %d\n"
.text
main:
	ldr r0, =fmt
	mov r1, #1
	bl printf
\end{minted}
Questo codice chiama la funzione \verb|printf| del C passandogli come parametri \verb|r0| ed
\verb|r1|. Se eseguito, tale programma stamperà la stringa "\verb|valore: 1|"

Quando i parametri sono pochi possiamo usare i registri e, nello specifico se il
\textbf{passaggio per valore} avviene scrivendo semplicemente in un registro il valore desiderato
con una \verb|mov|. Nel caso in cui si voglia effettuare un \textbf{passaggio per indirizzo}
un'operazione di \verb|load| sarebbe più indicata.

L'istruzione \verb|return| restituisce un valore che dal nostro punto di vista può essere visto
anch'esso come un problema di passaggio di parametri.

Siamo ora interessati a scrivere una funzione in assembler a cui poter passare dei parametri. Nello
specifico vogliamo scrivere del codice che sia equivalente ad una funzione scritta in C che prende
un parametro e lo incrementa di 1.
\begin{minted}{c}
int incrementa(int x) { return x + 1; }
\end{minted}
In questo caso \verb|x| è un parametro formale ma nel momento in cui andremo a passare un valore
al suo interno diventerà un parametro attuale.

La convenzione impone che, nel caso ci sia un valore di ritorno, questo venga scritto sul registro
\verb|r0|. Possiamo quindi definire la nostra funzione in questo modo
\begin{minted}{gas}
incrementa:
	add r0, r0, #1
	mov pc, lr
\end{minted}
con il registro \verb|r0| che dev'essere inizializzato correttamente in modo da contenere il valore
da incrementare.

Nel caso avessimo più di quattro parametri, come anticipato in precedenza, dobbiamo recuperare i
parametri che sono andati a finire nello stack. Se ad esempio avessimo una funzione che accetta
cinque parametri, tramite una \verb|pop| otterremmo il quinto parametro.

\subsection{Funzioni ricorsive}
Per l'implementazione di funzioni ricorsive dobbiamo andare a lavorare con lo stack e con il
\verb|link register| per riuscire a salvare tutti i punti di ritorno. Prendiamo ad esempio il
seguente codice che implementa il calcolo del fattoriale
\begin{minted}{gas}
fact:
	cmp r0, #0 // caso base n = 0
	bne else
	mov r0, #1
	mov pc, lr
else:
	push {lr}
	push {r0} // salvataggio indirizzo di ritorno e parametro n
	sub r0, r0, #1 // n-1
	bl fact // fact(n-1)
	pop {r1} // recupero n e indirizzo di ritorno
	mul r0, r0, r1 // n * fact(n-1)
	pop {pc}
\end{minted}
la cui parte importante è il salvataggio sullo stack dell'indirizzo di ritorno e di eventuali
parametri che non devono andare persi.
