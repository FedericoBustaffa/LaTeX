\section{Funzioni}
Vediamo ora come sia possibile realizzare \textbf{funzioni} e \textbf{procedure} in assembler
andando a capire come implementare una \textbf{chiamata} di funzione, il \textbf{ritorno} dalla
funzione e come effettuare il \textbf{passaggio di parametri}.

Vogliamo anche capire come riuscire a richiamare del codice già compilato e come generare codice
che permette di essere eseguito da altri.

\subsection{Passaggio di parametri}
Prima di entrare nello specifico definiamo alcune convenzioni sul passaggio dei parametri. I
parametri di una funzione sono gestiti dai registri assegnado ad \verb|r0| il primo parametro,
ad \verb|r1| il secondo parametro e così via.

In realtà questa regola vale fine a \verb|r3| e se abbiamo più parametri questi vanno a finire
nello stack. L'ordinamento è una convenzione rispettata dalle funzioni della \verb|glibc|. Ne è un
esempio la \verb|printf|, che come sappiamo, prende una stringa che indica il formato e una serie
di parametri di cui vogliamo stampare il valore.
\begin{minted}{gas}
.data
fmt: .string "valore: %d\n"
.text
main:
	ldr r0, =fmt
	mov r1, #1
	bl printf
\end{minted}
Questo codice chiama la funzione \verb|printf| del C passandogli come parametri \verb|r0| ed
\verb|r1|. Se eseguito, tale programma stamperà la stringa "\verb|valore: 1|"

Quando i parametri sono pochi possiamo usare i registri e, nello specifico se il
\textbf{passaggio per valore} avviene scrivendo semplicemente in un registro il valore desiderato
con una \verb|mov|. Nel caso in cui si voglia effettuare un passaggio per indirizzo un'operazione
di \verb|load| sarebbe più indicata.

L'istruzione \verb|return| restituisce un valore che dal nostro punto di vista può essere visto
anch'esso come un problema di passaggio di parametri.

Siamo ora interessati a scrivere del codice che sia equivalente ad una funzione scritta in C che
prende un parametro e lo incrementa di 1.
\begin{minted}{c}
int f(int x) { return x + 1; }
\end{minted}
In questo caso \verb|x| è un parametro formale ma nel momento in cui andremo a passare un valore
al suo interno diventerà un parametro attuale e questo processo è un qualcosa che vogliamo riuscire
ad implementare in assembler.
