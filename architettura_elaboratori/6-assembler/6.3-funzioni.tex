\section{Funzioni}
Vediamo ora come sia possibile realizzare \textbf{funzioni} e \textbf{procedure} in assembler
andando a capire come implementare una \textbf{chiamata} di funzione, il \textbf{ritorno} dalla
funzione e come effettuare il \textbf{passaggio di parametri}.

Vogliamo anche capire come riuscire a richiamare del codice già compilato e come generare codice
che permette di essere eseguito da altri.

Se ad esempio volessimo scrivere una funzione in C che prende un parametro e ritorna il parametro
stesso incrementato di 1, il codice sarebbe
\begin{minted}{c}
int f(int x) { return x + 1; }
\end{minted}
In questo caso \verb|x| è un parametro formale ma nel momento in cui andremo a passare un valore
al suo interno diventerà un parametro attuale e questo processo è un qualcosa che vogliamo riuscire
ad implementare in assembler.

L'istruzione \verb|return| restituisce un valore che dal nostro punto di vista può essere visto
anch'esso come un problema di passaggio di parametri.