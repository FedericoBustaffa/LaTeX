\chapter{Aritmetica di macchina}
I punti d'interesse del corso sono quindi
\begin{itemize}
	\item Capire come i numeri sono rappresentati in macchina.
	\item Capire come viene implementata l'aritmetica in macchina.
	\item Capire come il calcolatore approssima i calcoli eseguiti.
	\item Capire se tale approssimazione è accettabile per il problema che stiamo cercando di risolvere.
\end{itemize}
In questo capitolo andremo a trattare la \emph{rappresentazione} dei numeri in macchina, l'\emph{implementazione}
dell'aritmetica e quali \emph{errori} si possono generare.

\section{Rappresentazione dei numeri}
Cerchiamo come prima cosa di rappresentare i numeri in modo \textbf{univoco}. Come prima cosa abbandoniamo la
forma frazionaria di rappresentazione. Come sappiamo le quantità
\[ 0.1 \quad \frac{1}{10} \quad \frac{2}{20} \]
hanno tutte lo stesso valore ma, come possiamo facilmente intuire, la forma frazionaria fornisce infinite
rappresentazioni, tutte equivalenti, per la stessa quantità $0.1$.

Considerando anche che la riduzione ai minimi termini di una frazione è un'operazione in generale costosa ma
necessaria al fine di avere espressioni più facili da calcolare, non ci possiamo permettere un tipo di
rappresentazione di questo tipo.

Ecco perché, per rappresentare numeri reali, utilizzeremo la rappresentazione tramite sequenza di cifre decimali.

\begin{observation}
	Talvolta sarà necessario passare alla rappresentazione di numeri complessi ma, come sappiamo, possiamo vedere
	un numero complesso come una coppia di reali, dunque non ci saranno grosse differenze nella rappresentazione
	di questi ultimi.
\end{observation}

\begin{observation}
	La rappresentazione in cifre decimali potrebbe essere \textbf{infinita} (per esempio
	$\frac{1}{3} = 0.\overline{33}$), ma chiaramente non è possibile utilizzare una rappresentazione del genere
	su un calcolatore.
\end{observation}

Quest'ultima osservazione ci dice che un grado di approssimazione sia necessario e approssimare significa
generare un certo errore nella rappresentazione.

Per una questione di \emph{minimizzazione} dello spazio utilizzato per la rappresentazione dei numeri in
macchina si farà anche ricorso ad esponenziali per rapprentare cifre non significative in forma \emph{compatta}.

\begin{example}
	Per esempio, per rappresentare 0.0000001 conviene usare la seguente notazione
	\[ 0.0000001 = 0.1 \cdot 10^{-6} \]
	in quanto ci permette di rappresentare quei 6 zeri dopo la virgola rappresentando solo il 6 all'esponente.
\end{example}

\subsection{Rappresentazione normalizzata in virgola mobile}
\begin{theorem}[Rappresentazione]\label{th: rappr}
	Dato $x \in \R$ tale che $x \neq 0$, e sia $B \in \N$ tale che $B > 1$, la \textbf{base} di numerazione,
	esistono e sono univocamente determinati:
	\begin{itemize}
		\item Un intero $p \in \Z$, detto \textbf{esponente} della rappresentazione.
		\item Una serie di numeri naturali $\{ d_i \}_{i \geq 1}$ con
		      \begin{itemize}
			      \item $d_1 \neq 0$
			      \item $0 \leq d_i \leq B - 1$
			      \item Tutti i $d_i$ non definitivamente uguali a $B - 1$
		      \end{itemize}
		      dette \textbf{cifre} di rappresentazione.
	\end{itemize}
	tali che
	\begin{equation}\label{rappr}
		x = \sign(x) \cdot B^p \cdot \sum_{i=1}^{+\infty} d_i B^{-i}
	\end{equation}
\end{theorem}

\begin{definition}
	La serie $d_1 B^{-1} + d_2 B^{-2} + \dots$ è detta \textbf{mantissa}.
\end{definition}

La rappresentazione \ref{rappr} è detta \textbf{rappresentazione normalizzata in virgola mobile}
(\emph{floating point}) in quanto l'esponente $p$ è determinato in modo da avere parte intera nulla e prima
cifra dopo la virgola non nulla.

Andiamo ad analizzare meglio i vincoli imposti nel teorema \ref{th: rappr} sulla rappresentazione.
\begin{itemize}
	\item Le condizioni $d_1 \neq 0$ (rappresentazione normalizzata) e $d_i$ non definitivamente uguale a
	      $B - 1$ ci garantiscono l'\emph{unicità} della rappresentazione
	\item Il numero $x = 0$ non ammette rappresentazione normalizzata e in macchina viene trattato e
	      memorizzato in modo speciale.
\end{itemize}

\begin{example}
	Rappresentiamo 0.1 tramite la rappresentazione \ref{rappr}:
	\[ 0.1 = + 10^0 \cdot (1 \cdot 10^{-1} + 0 \cdot 10^{-2} + \dots) \]
	Se però volessimo rappresentare lo stesso numero in modo diverso, lo potremmo fare in questo modo
	\[ 0.1 = + 10^1 \cdot (0 \cdot 10^{-1} + 1 \cdot 10^{-2} + 0 \cdot 10^{-3} + \dots) \]
	Cambiando ogni volta l'esponente $p$ e adattando i $d_i$ è possibile trovare infinte rappresentazioni
	per 0.1. Dunque $d_1$ deve essere non nullo.
\end{example}

\begin{example}
	Come sappiamo $0.\overline{9} = 1$. Se volessimo rappresentare 1 otterremmo
	\[ 1 = + 10^1 \cdot (1 \cdot 10^{-1} + 0 \cdot 10^{-2} + \dots) \]
	mentre se rappresentassimo $0.\overline{9}$ otterremmo
	\[ 0.\overline{9} = + 10^0 \cdot (9 \cdot 10^{-1} + 9 \cdot 10^{-2} + \dots) \]
	Si ottiene così una rappresentazione normalizzata ($d_1 = 9$) e costituita da soli 9 dopo la virgola, ma
	dato che i due numeri sono equivalenti la rappresentazione utilizzata è la prima.
\end{example}

\subsection{Rappresentazione in macchina}