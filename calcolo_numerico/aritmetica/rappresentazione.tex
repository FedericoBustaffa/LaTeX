\chapter{L'aritmetica del calcolatore}
I punti d'interesse del corso sono quindi
\begin{itemize}
	\item Capire come i numeri sono rappresentati in macchina.
	\item Capire come viene implementata l'aritmetica in macchina.
	\item Capire come il calcolatore approssima i calcoli eseguiti.
	\item Capire se tale approssimazione è accettabile per il problema che stiamo cercando di risolvere.
\end{itemize}
In questo capitolo andremo a trattare la \emph{rappresentazione} dei numeri in macchina, l'\emph{implementazione}
dell'aritmetica e quali \emph{errori} si possono generare.

\section{Rappresentazione dei numeri}
Cerchiamo come prima cosa di rappresentare i numeri in modo \textbf{univoco}. Come prima cosa abbandoniamo la
forma frazionaria di rappresentazione. Come sappiamo le quantità
\[ 0.1 \quad \frac{1}{10} \quad \frac{2}{20} \]
hanno tutte lo stesso valore ma, come possiamo facilmente intuire, la forma frazionaria fornisce infinite
rappresentazioni tutte equivalenti per la stessa quantità $0.1$.

Considerando anche che la riduzione ai minimi termini di una frazione è un'operazione in generale costosa ma
necessaria al fine di avere espressioni più facili da calcolare, non ci possiamo permettere un tipo di
rappresentazione di questo tipo.

Ecco perché, per rappresentare numeri reali, utilizzeremo la rappresentazione tramite \emph{sequenza di cifre
	decimali}.

\begin{observation}
	Talvolta sarà necessario passare alla rappresentazione di numeri complessi ma, come sappiamo, possiamo vedere
	un numero complesso come una coppia di reali, dunque non ci saranno grosse differenze nella rappresentazione
	di questi ultimi.
\end{observation}

\begin{observation}
	La rappresentazione in cifre decimali potrebbe essere \textbf{infinita} (per esempio
	$\frac{1}{3} = 0.\overline{33}$), ma chiaramente non è possibile utilizzare una rappresentazione del genere
	su un calcolatore.
\end{observation}

Quest'ultima osservazione ci dice che un grado di approssimazione sarà necessario andando così a generare un
certo errore sulla rappresentazione.

Per una questione di \emph{minimizzazione} dello spazio utilizzato per la rappresentazione dei numeri in
macchina si farà ricorso ad esponenziali per rapprentare cifre non significative in forma \emph{compatta}.

\begin{example}
	Per esempio, per rappresentare 0.0000001 si usa la seguente notazione
	\[ 0.0000001 = 0.1 \cdot 10^{-6} \]
	in quanto ci permette di rappresentare quei 6 zeri dopo la virgola rappresentando solo il 6 all'esponente.
\end{example}

\subsection{Rappresentazione normalizzata in virgola mobile}
\begin{theorem}[Rappresentazione]\label{th: rappr}
	Dato $x \in \R$ tale che $x \neq 0$, e sia $B \in \N$ tale che $B > 1$, la \textbf{base} di numerazione,
	esistono e sono univocamente determinati:
	\begin{itemize}
		\item Un intero $p \in \Z$, detto \textbf{esponente} della rappresentazione.
		\item Una serie di numeri naturali $\{ d_i \}_{i \geq 1}$ con
		      \begin{itemize}
			      \item $d_1 \neq 0$
			      \item $0 \leq d_i \leq B - 1$
			      \item Tutti i $d_i$ non definitivamente uguali a $B - 1$
		      \end{itemize}
		      dette \textbf{cifre} di rappresentazione.
	\end{itemize}
	tali che
	\[ x = \sign(x) \cdot B^p \cdot \sum_{i=1}^{+\infty} d_i B^{-i} \]
\end{theorem}

\begin{definition}
	La serie $d_1 B^{-1} + d_2 B^{-2} + \dots$ è detta \textbf{mantissa}.
\end{definition}

La rappresentazione \ref{th: rappr} è detta \textbf{rappresentazione normalizzata in virgola mobile}
(\emph{floating point}) in quanto l'esponente $p$ è determinato in modo che il numero da rappresentare abbia
parte intera nulla e prima cifra dopo la virgola non nulla.

\subsubsection{Vincoli per la rappresentazione}
Andiamo ad analizzare meglio i vincoli imposti nel teorema \ref{th: rappr} sulla rappresentazione.
\begin{itemize}
	\item Le condizioni $d_1 \neq 0$ (rappresentazione normalizzata) e $d_i$ non definitivamente uguale a
	      $B - 1$ ci garantiscono l'\emph{unicità} della rappresentazione
	\item Il numero $x = 0$ non ammette rappresentazione normalizzata e in macchina viene trattato e
	      memorizzato in modo speciale.
\end{itemize}

\begin{example}
	Rappresentiamo 0.1 tramite la rappresentazione \ref{th: rappr}:
	\[ 0.1 = + 10^0 \cdot (1 \cdot 10^{-1} + 0 \cdot 10^{-2} + \dots) \]
	Se però volessimo rappresentare lo stesso numero in modo diverso, lo potremmo fare in questo modo
	\[ 0.1 = + 10^1 \cdot (0 \cdot 10^{-1} + 1 \cdot 10^{-2} + 0 \cdot 10^{-3} + \dots) \]
	Cambiando ogni volta l'esponente $p$ e adattando i $d_i$ è possibile trovare infinte rappresentazioni
	per 0.1. Dunque $d_1$ deve essere non nullo.
\end{example}

\begin{example}
	Come sappiamo $0.\overline{9} = 1$. Se volessimo rappresentare 1 otterremmo
	\[ 1 = + 10^1 \cdot (1 \cdot 10^{-1} + 0 \cdot 10^{-2} + \dots) \]
	mentre se rappresentassimo $0.\overline{9}$ otterremmo
	\[ 0.\overline{9} = + 10^0 \cdot (9 \cdot 10^{-1} + 9 \cdot 10^{-2} + \dots) \]
	Si ottiene così una rappresentazione normalizzata ($d_1 = 9$) e costituita da soli 9 dopo la virgola,
	ma dato che i due numeri sono equivalenti la rappresentazione utilizzata è la prima.
\end{example}

\subsection{Rappresentazione in macchina}
Per ora abbiamo dato una definizione ideale della rappresentazione in virgola mobile di un numero reale.
In macchina si ha però a disposizione un registro di lunghezza finita per memorizzarla. Tale registro è
suddiviso in tre parti:
\begin{itemize}
	\item Segno di $x$
	\item Esponente $p$
	\item Cifre di rappresentazione $\{ d_i \}$
\end{itemize}
Abbiamo quindi un numero finito di rappresentazioni possibili in macchina.

\begin{definition}\label{set_num_macchina}
	Si definisce \textbf{insieme dei numeri di macchina} in rappresentazione \emph{floating point} con $t$
	cifre, base $B$ e range $(-m, M)$, l'insieme dei numeri reali
	\[
		\F (B, t, m, M) = \{ 0 \} \cup
		\{ x \in \R : x = \sign(x) \cdot B^p \cdot \sum_{i=1}^t d_i B^{-i} \}
	\]
	con $0 \leq d_i \leq B-1$, $d_1 \neq 0$ e $-m \leq p \leq M$.
\end{definition}

\begin{observation}
	Guardando l'\emph{insieme dei numeri di macchina} possiamo osservare che
	\begin{itemize}
		\item Ha \textbf{cardinalità finita}
		      \[ N = 2 B^{t-1} (B - 1) (M + m + 1) + 1 \]
		      Otteniamo questo risultato dato che
		      \begin{itemize}
			      \item Il segno di $x$ può avere solo 2 valori possibili.
			      \item Avendo $t$ cifre disponibili per la rappresentazione abbiamo $B^t$ possibili
			            configurazioni. Ma dato che $d_1 \neq 0$ allora passiamo a $B^{t-1}$ configurazioni
			            effettive.
			      \item Ogni $d_i$ ha $B - 1$ possibili valori.
			      \item L'esponente è compreso tra $-m$ ed $M$ dunque abbiamo $m + M$ possibili valori ai quali
			            dobbiamo aggiungere lo zero.
			      \item Aggiungiamo lo zero.
		      \end{itemize}
		\item Se $x \in \F(B, t, m, M)$ e $x \neq 0$ allora $\omega = B^{-m - 1}$ è il più piccolo numero di
		      macchina positivo e $\Omega = B^M (1 - B^{-t})$ è il più grande numero di macchina positivo e
		      vale
		      \[ \omega \leq |x| \leq \Omega \]
		      Ne segue che non è possibile rappresentare esattamente numeri non nulli di modulo minore a
		      $\omega$. Ecco perché, per tali numeri, è prevista una rappresentazione \emph{denormalizzata}.

		      Quando $p = -m$ la condizione $d_1 \neq 0$ può essere abbandonata potendo quindi rappresentare come
		      numero più piccolo positivo $B^{-m-t}$ e dunque riuscendo a rappresentare numeri positivi e negativi
		      compresi in modulo tra
		      \[ [ B^{-m-t} \;;\; B^{-m-1} - B^{-m-t} ] \]

		      Analogamente se $p = M$ si introducono rappresentazioni speciali per i simboli $\pm \infty$
		      e NaN.
		\item L'insieme dei numeri di macchina $\F (B, t, m, M)$ è simmetrico rispetto all'origine.
		\item Posto $x = (-1)^s \cdot B^p \cdot \alpha$ appartente a $\F(B, t, m, M)$ allora il
		      successivo numero di macchina risulta essere
		      \[ y = (-1)^s \cdot B^p \cdot (\alpha + B^{p-t}) \]
		      Questo ci dice che, in un sistema in virgola mobile, la distanza
		      \[ |y - x| = B^{p-t} \]
		      varia con $p$, ovvero con l'ordine di grandezza dei numeri considerati.
	\end{itemize}
\end{observation}

\subsubsection{Standard IEEE per la doppia precisione}
Lo standard proposto dal IEEE per la rappresentazione in virgola mobile (base 2) prevede registri lunghi 64 bit.
In prima battuta si era pensato ad una suddivisione del registro di questo tipo:
\begin{itemize}
	\item 1 bit per il segno.
	\item 11 bit per l'esponente.
	\item 52 bit per le cifre di rappresentazione.
\end{itemize}
Il problema di questo modello di rappresentazione era la doppia rappresentazione dello 0 (-0 e +0) ed è stato
quindi scartato.

L'attuale standard, per risolvere il problema, rappresenta l'esponente in \textbf{traslazione}. Chiariamo meglio
il concetto procedendo con ordine:
\begin{itemize}
	\item Dato che si hanno 11 bit per l'esponente si possono rappresentare numeri che vanno da 0 a 2047. Abbiamo
	      quindi 2048 configurazioni totali per l'esponente.
	\item Le due configurazioni di bit con tutti 0 e tutti 1 vengono tenute da parte e trattate in modo
	      \emph{speciale}. Ci rimangono dunque 2046 configurazioni.
	\item Lo standard definisce l'esponente nell'intervallo
	      \[ [ -1021, \; 1024 ] \]
	      Abbiamo dunque che $m = -1021$ e $M = 1024$.
\end{itemize}
