\section{Aritmetica di macchina}
Adesso rimane da capire come la macchina mappa i numeri reali in numeri di macchina. Il problema che ci
poniamo è il seguente: sia $x \in \R$, quale numero di macchina $\tilde{x} \in \F$ viene associato a $x$
per la rappresentazione?

Associare un numero di macchina $\tilde{x}$ a $x$ significa \textbf{approssimare} $x$ commettendo un
\textbf{errore relativo} di rappresentazione
\[ \epsilon_x = \frac{\tilde{x} - x}{x} = \frac{\eta_x}{x}, \quad x \neq 0 \]
quanto più piccolo possibile in valore assoluto. La quantità
\[ \eta_x = \tilde{x} - x \]
è detta \textbf{errore assoluto} di rappresentazione. Dato $x \in \R$ con $x \neq 0$, distinguiamo 2 casi:
\begin{itemize}
	\item $|x| < \omega$ (\textbf{underflow}) o $|x| > \Omega$ (\textbf{overflow})
	\item $\omega \leq |x| \leq \Omega$
\end{itemize}
Nel secondo caso le tecniche di approssimazione previste dallo standard IEEE 754 sono:
\begin{itemize}
	\item \textbf{Round to the nearest} (arrotondamento): il numero $x$ viene approssimato con il numero di
	      macchina $\tilde{x}$ più vicino.
	\item \textbf{Round toward zero} (troncamento): il numero $x$ viene approssimato con il più grande numero
	      di macchina $\tilde{x}$ il cui valore assoluto risulti minore o uguale al valore assoluto di $x$.
	\item \textbf{Round toward plus infinity}: il numero $x$ viene approssimato al più piccolo numero
	      rappresentabile maggiore del dato.
	\item \textbf{Round toward minus infinity}: il numero $x$ viene approssimato al più grande numero
	      rappresentabile minore del dato.
\end{itemize}
Assumiamo per semplicità di considerare una macchina che opera con troncamento sull'insieme $\F$. Per
convenzione indichiamo con $\trn(x) = \tilde{x}$ il risultato dell'approssimazione di $x$ con troncamento e
più generalmente $\fl(x)$ l'approssimazione in macchina del dato $x$ nel sistema \emph{floating point}
considerato. Il primo risultato fornisce una maggiorazione uniforme dell'errore di rappresentazione.

\begin{theorem}\label{th: prec}
	Sia $x \in \R$ con $\omega \leq |x| \leq \Omega$. Si ha
	\[ |\epsilon_x| = \left| \frac{\trn(x) - x}{x} \right| \leq u = B^{1-t} \]
	dove $u$ è detta \textbf{precisione di macchina}, la quale non dipende da $x$ e fornisce errori più grandi
	per numeri grandi ed errori piccoli per numeri piccoli.
	\begin{proof}
		Sia $x = (-1)^s \cdot B^p \alpha$. L'errore assoluto $|\trn(x) - x|$ risulta maggiorato dalla distanza
		tra i due numeri di macchina consecutivi e quindi
		\[ |\trn(x) - x| \leq B^{p-t} \]
		Inoltre $|x| \geq B^{p-1}$. Pertanto vale
		\[ |\epsilon_x| = \left| \frac{\trn(x) - x}{x} \right| \leq \frac{B^{p-t}}{B^{p-1}} = B^{1-t} = u \]
	\end{proof}
\end{theorem}

\begin{observation}
	Possiamo osservare che
	\begin{itemize}
		\item La \emph{precisione di macchina} è indipendente dalla grandezza del numero e caratteristica
		      dell'aritmetica \emph{floating point} implementata sulla macchina su cui stiamo operando. Se ad
		      esempio operiamo con arrotondamento, la distanza tra $x$ e la sua approssimazione di macchina, e
		      quindi la precisione di macchina, si dimezzano.
		\item Per valutare la precisione di macchina possiamo determinare il più piccolo numero di macchina
		      maggiore di 1. Sia infatti $x$ il numero che cerchiamo, abbiamo che $x - 1 = |x - 1| = B^{1-t}$,
		      essendo $1 = B^1 \cdot B^{-1}$ rappresentato con esponente $p = 1$.
		\item Dal teorema \ref{th: prec} si ricava che, dato $x \in \R$, in assenza di situazione di
		      \emph{underflow} o \emph{overflow}, per la sua rappresentazione in macchina vale che
		      \[ \fl(x) = x (1 + \epsilon_x), \quad |\epsilon_x| \leq u \]
		      Questa relazione esprime il modo in cui viene generalmente descritto il legame tra un numero reale
		      e la sua rappresentazione in macchina.
	\end{itemize}
\end{observation}

\subsection{Operazioni aritmetiche}
