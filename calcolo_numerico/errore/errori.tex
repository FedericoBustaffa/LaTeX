\chapter{Analisi degli errori}
In questo capitolo andiamo a trattare gli errori nel calcolo di funzioni razionali e non e andiamo a vedere
alcune tecniche per l'analisi tali errori.

\section{Errori nel calcolo di una funzione razionale}
Inizialmente andremo a considerare funzioni di questo tipo
\[ f : \R \to \R\]
anche se si tratta di una semplificazione in quanto, in casi più realistici si ha a che fare con funzioni del tipo
\[ f : \R^n \to \R \]
Come abbiamo già anticipato nel capitolo precedente, in generale non riusciamo a calcolare $f(x)$ ma, se abbiamo
fortuna, riusciamo a calcolare $f(\tx)$ dato che introduciamo un errore sul dato $x$ in ingresso.

\begin{definition}
	Definiamo \textbf{errore inerente} (o \textbf{inevitabile}) la quantità
	\[ \epsilon_{\text{in}} = \frac{f(\tx) - f(x)}{f(x)} \]
	La quale rappresenta l'errore che viene sempre commesso in quanto la macchina non riceve in ingresso
	il dato esatto ma quello approssimato.
\end{definition}

\begin{observation}
	Possiamo osservare che
	\begin{itemize}
		\item L'\emph{errore inerente} non dipende dall'algoritmo usato per la risoluzione del problema ma
		      misura la sensibilità della funzione $f$ rispetto alla perturbazione del dato iniziale.
		\item Se l'\emph{errore inerente} è qualitativamente elevato in valore assoluto diciamo che il problema
		      è \textbf{mal condizionato}. Al contrario, se è piccolo, diciamo che il problema è
		      \textbf{ben condizionato}.
	\end{itemize}
\end{observation}

\begin{definition}
	Definiamo \textbf{errore algoritmico} la quantità
	\[ \epsilon_{\text{alg}} = \frac{g(\tx) - f(\tx)}{f(\tx)} \]
	l'errore generato nel calcolo di $f(\tx) \neq 0$.
\end{definition}

\begin{observation}
	Possiamo osservare che
	\begin{itemize}
		\item La funzione $g$ dipende dall'algoritmo utilizzato per calcolare $f(x)$. In generale possiamo dire
		      che diversi algoritmi conducono a differenti errori algoritmici.
		\item Se l'\emph{errore algoritmico} è qualitativamente elevato in valore assoluto diciamo che l'algoritmo
		      è \textbf{numericamente instabile}. Al contrario, se è piccolo, diciamo che il problema è
		      \textbf{numericamente stabile}.
	\end{itemize}
\end{observation}

\begin{definition}
	Definiamo \textbf{errore totale} la quantità
	\[ \epsilon_{\text{tot}} = \frac{g(\tx) - f(x)}{f(x)} \]
	l'errore generato nel calcolo di $f(x) \neq 0$ mediante l'algoritmo specificato da $g$. L'\emph{errore totale}
	misura la differenza relativa tra l'output atteso e l'output effettivamente calcolato.
\end{definition}

\begin{theorem}
	In un'analisi al primo ordine vale
	\[ \epsilon_{\text{tot}} = \epsilon_{\text{in}} + \epsilon_{\text{alg}} \]
\end{theorem}

Il teorema esprime il fatto che nel calcolo di una funzione razionale in un'analisi al primo ordine le due fonti
di errore individuate precedentemente forniscono contributi separati che possono essere analizzati
indipendentemente. L'obbiettivo dell'analisi numerica è individuare algoritmi numericamente \emph{stabili} per
problemi \emph{ben condizionati}.