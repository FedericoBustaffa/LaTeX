\section{Risoluzione di sistemi lineari}
Vediamo ora alcuni metodi per la risoluzione di sistemi lineari e andiamo a studiare il loro condizionamento
in macchina. Il problema che vogliamo provare a risolvere ha in input la matrice $A \in \R^{n \times n}$ e il
vettore $b \in \R^n$ dai quali vogliamo ricavare il vettore $x \in \R^n$ tale che $A x = b$.

La prima cosa che vogliamo osservare è che per particolari classi della matrice $A$ il sistema è facilmente
risolubile. Prendiamo ad esempio la classe delle matrici \textbf{diagonali} ossia matrici con tutti gli elementi
fuori dalla diagonale principale nulli:
\[ i \neq j \quad \Rightarrow \quad a_{ij} = 0 \]
Se proviamo a calcolare il determinante della matrice $A$ possiamo vedere che questo equivale al prodotto degli
elementi sulla diagonale principale così come possiamo dimostrare che gli autovalori sono gli elementi sulla
diagonale principale.

Una matrice diagonale è invertibile se e solo se ha tutti gli elementi sulla diagonale principale diversi da 0.
Per risolvere un sistema diagonale possiamo scrivere ogni equazione nella forma
\[ a_{ii} x_i = b_i \]
e quindi possiamo facilmente ottenere $x_i$ tramite
\[ x_i = \frac{b_i}{a_{ii}} \]
che è ben definito se tutti gli $a_{ii}$ sono diversi da 0. Un semplice algoritmo che risolve un sistema diagonale
$n \times n$ è il seguente:
\begin{lstlisting}[language=pseudo, style=pseudo-style]
	for i = 1 to n
		x[i] = b[i] / a[i][i]
\end{lstlisting}
Dunque il costo della risoluzione di un sistema diagonale è $O(n)$.