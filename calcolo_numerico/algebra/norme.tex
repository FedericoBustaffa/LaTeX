\chapter{Problemi di algebra lineare}
In questo capitolo verranno trattati i problemi dell'algebra lineare e il loro condizionamento tra cui la
risoluzione di sistemi lineari e il calcolo di autovalori e autovettori di matrici.

Il primo problema che andiamo a trattare è quello della risoluzione di sistemi lineari. Dall'algebra lineare
noi sappiamo che un sistema lineare è rappresentabile come
\[ A x = b \]
con $A \in \R^{n \times m}$ e $b \in \R^n$. La soluzione del sistema è quindi il vettore $x \in \R^n$ tale che
\[ x = A^{-1} b \]
Come sappiamo è possibile scrivere ognuna delle componenti $x_i$ di $x$ come funzione $f_i$ di $A$ e $b$ in
questo modo
\[ x_i = f_i (a_{11}, \dots, a_{nm}, b_1, \dots, b_n) \quad i = 1 \dots n \]
Abbiamo quindi $n$ funzioni del tipo $f : \R^{n \times m} \times \R^n \to \R$. Questa rappresentazione ci
fornisce un modo per scrivere la soluzione del problema dal punto di vista matematico ma non ci dà alcuna
informazione sul condizionamento del problema.

In macchina non abbiamo $A$ e $b$ ma $\hat{A}$ e $\hat{b}$, rispettivamente \emph{vicini} ad $A$ e $b$. Il
concetto di \emph{vicinanza} tra due numeri $x$ e $y$ per esempio è esprimibile come $|x - y|$. Quanto più
questo valore sarà vicino a 0 tanto più i due numeri saranno per l'appunto \emph{vicini}.

Il valore assoluto possiede delle proprietà interessanti che ci serviranno per quello che andremo a trattare
tra poco. Prendiamo $f(x) = |x|$, vale che
\begin{itemize}
	\item La distanza è maggiore o uguale a 0 $\to f(x) \geq 0$ e $f(x) = 0 \Leftrightarrow x = 0$
	\item La distanza da $x$ a $y$ e la stessa che da $y$ a $x$ $\to f(\alpha x) = \alpha \cdot f(x)$
	\item $f(x + y) \leq f(x) + f(y)$
\end{itemize}

\section{Norme vettoriali e matriciali}
Dopo la breve premessa introduciamo il concetto di \textbf{norma} e chiariamo meglio la sua utilità in questo
ambito.

\begin{definition}
	Definiamo \textbf{norma vettoriale} su $\F^n$ una funzione $f : \F^n \to \R$ che soddisfa le seguenti
	proprietà:
	\begin{itemize}
		\item $\forall v \in \F^n$ vale che
		      \[ f(v) > 0, \quad f(v) = 0 \Leftrightarrow v = 0 \]
		\item $\forall v \in \F^n$, $\forall \alpha \in \F$ vale che
		      \[ f(\alpha v) = |\alpha| \cdot f(v) \]
		\item $\forall v, w \in \F^n$ vale che
		      \[ f(v + w) \leq f(v) + f(w) \]
	\end{itemize}
\end{definition}

Se $f$ è una norma vettoriale su $\F^n$, la indicheremo per comodità di notazione con $f(v) = \| v \|$ e possiamo
definire la \textbf{distanza} come la \textbf{norma} vettoriale della differenza tra due vettori
\[ \dist(v, w) = f(v - w) = \| v - w \| \]
Questa non è l'unica funzione per il calcolo della distanza tra due vettori ma per il momento ci limitiamo ad
analizzare il funzionamento di questa per introdurre un concetto generale e per poi muoverci sulla distanza
tra matrici.