\chapter{Equazioni non lineari}
In questo capitolo andremo a trattare la risoluzione di \textbf{equazioni non lineari}, tipicamente del tipo
\[ f : \R \to \R \quad \vee \quad f : [a, b] \to \R \]
Risolvere un'equazione non lineare significa trovare tutti i valori della variabile $x$ tali che
\[ f(x) = 0 \]
Chiameremo questi valori \textbf{radici} dell'equazione o \textbf{zeri} della funzione. In generale, possiamo
dire che una funzione non lineare è una qualsiasi funzione differente da
\[ f(x) = a x + b \]
ossia, dalla funzione che definisce una retta nel piano. Polinomi di grado superiore al primo nella variabile
$x$, logaritmi, esponenziali, radici, seno e coseno sono tutte funzioni non lineari.

L'obbiettivo di questa parte del corso è definire dei metodi che siano in grado di risolvere equazioni non
lineari. Mentre per la risoluzione di sistemi lineari avevamo informazioni sulla risolvibilità del sistema
come ad esempio l'\emph{invertibilità} della matrice stavolta non abbiamo garanzia che l'equazione abbia
delle soluzioni né quante siano queste soluzioni.

L'unico approccio che abbiamo per avere informazioni sulla funzione è di tipo grafico. Cerchiamo quindi di
costruire il grafico della funzione per definire dove sono localizzate le soluzioni, di modo da poter procedere
in modo più mirato all'approssimazione di tali soluzioni.

\begin{example}
	Consideriamo la seguente funzione $f$
	\[ f = x^3 - 2x + 3 = 0 \]
	Tale equazione ha 3 soluzioni su $\C$ ma noi siamo interessati ad un'analisi su $\R$ e dunque possiamo dire
	che se c'è una soluzione complessa allora anche la sua coniugata è soluzione dell'equazione e in quel caso
	avremmo una soluzione reale. Nel caso non ci siano soluzioni complesse allora ce ne sono 3 reali.

	Per capire se l'equazione ha 1 o 3 soluzioni reali proviamo a tracciare un grafico approssimativo della
	funzione. Per farlo possiamo, prima di tutto, calcolare i limiti agli estremi del campo di esistenza, che,
	in questo caso è tutto $\R$.
	\begin{align*}
		\lim_{x \to +\infty} x^3 - 2x + 3 = & +\infty \\
		\lim_{x \to -\infty} x^3 - 2x + 3 = & -\infty
	\end{align*}
	Altra cosa che possiamo fare è calcolare la derivata della funzione
	\[ f'(x) = 3x^2 - 2 \]
	Stavolta possiamo studiare facilmente il segno dato che ci troviamo davanti un'equazione di secondo grado
	\[ 3x^2 - 2 = 0 \quad \Leftrightarrow \quad x = \pm \sqrt{\frac{2}{3}} \]
	Ricaviamo quindi che la derivata è positiva per
	\[ x < -\sqrt{\frac{2}{3}} \quad \vee \quad x > \sqrt{\frac{2}{3}} \]
	mentre è negativa per
	\[ -\sqrt{\frac{2}{3}} < x < \sqrt{\frac{2}{3}} \]
	Ne segue che nell'intervallo di positività della derivata la funzione è crescente mentre in quello di
	negatività la funzione è decrescente. Abbiamo inoltre un massimo locale in $-\sqrt{\frac{2}{3}}$ e un
	minimo locale in $\sqrt{\frac{2}{3}}$.

	Possiamo ora provare a tracciare un grafico approssimativo. Per prima cosa possiamo valutare la funzione in
	$x = 0$ e vediamo che $f(0) = 3$. Dato che $-\sqrt{\frac{2}{3}} < 0 < \sqrt{\frac{2}{3}}$ possiamo dedurre
	la funzione valutata in $-\sqrt{\frac{2}{3}}$ avrà valore positivo. Valutiamo invece la funzione in
	$\sqrt{\frac{2}{3}}$ e otteniamo che
	\[ f \left( \sqrt{\frac{2}{3}} \right) = \frac{2}{3} \sqrt{\frac{2}{3}} - 2 \sqrt{\frac{2}{3}} + 3 > 0 \]
	Nel punto di minimo locale la funzione è positiva e quindi possiamo concludere che la funzione interseca
	l'asse delle ascisse solo in un punto minore di $-\sqrt{\frac{2}{3}}$. Concludiamo che l'equazione ha una
	sola soluzione $\alpha$ reale.

	A questo dobbiamo cercare di definire un intervallo nel quale sia compreso $\alpha$. Per farlo possiamo
	valutare la funzione in diversi punti finché non troviamo un $x$ tale che la funzione è negativa e uno
	tale che la funzione è positiva. La nostra soluzione $\alpha$ si troverà da qualche parte in mezzo a questo
	intervallo. Possiamo calcolare ad esempio
	\[ f(-1) = 4 \quad \quad f(-2) = -1 \]
	Possiamo quindi concludere che $\alpha \in [-2, -1]$.
\end{example}

\begin{definition}
	Una funzione $f$ si dice di \textbf{classe} $C^k$ in $[a, b]$
	\[ f \in C^k ([a, b]) \]
	se è continua e derivabile $k$ volte nell'intervallo $[a, b]$ con tutte le derivate continue.
\end{definition}

Una funzione continua appartiene sicuramente alla classe $C^0$, una funzione continua e derivabile infinite volte
con derivate continue appartiene alla classe $C^\infty$.

La continuità di una funzione è un requisito minimo per la costruzione di un buon algoritmo di approssimazione
degli zeri.

\begin{theorem}[Esistenza degli zeri]
	Sia $f : [a, b] \to \R$ e $f \in C^0 ([a, b])$. Se
	\[ f(a) \cdot f(b) < 0 \]
	allora esiste $\xi \in [a, b]$ tale che $f(\xi) = 0$.
\end{theorem}

Il teorema degli zeri non ci dice che esiste una sola soluzione, ma ci dice che ne esiste almeno una.

\begin{definition}
	Sia $f : [a, b] \to \R$ se esiste ed è unico $\xi \in [a, b]$ tale che $f(\xi) = 0$
	allora definiamo $[a, b]$ \textbf{intervallo di separazione}.
\end{definition}

Tutti i metodi che vedremo non saranno di tipo diretto ma \emph{iterativi}. Andremo quindi a generare una
successione che, sotto le opportune ipotesi, andrà a convergere nella soluzione. Definiremo anche un criterio
di arresto per terminare l'algoritmo una volta raggiunto un certo livello di accuratezza dell'approssimazione.

\section{Metodo di bisezione}
Vediamo ora il metodo più semplice possibile per la risoluzione di equazioni non lineari, ossia il metodo di
\textbf{bisezione}. Per poter applicare il metodo facciamo le seguenti assunzioni
\begin{gather*}
	f : [a,b] \to \R \\
	f \in \C^0([a,b]) \\
	f(a) \cdot f(b) < 0 \\
	!\exists \xi \in [a, b] : f(\xi) = 0
\end{gather*}
Senza queste assunzioni il metodo potrebbe non convergere o fornirci risultati errati. Il metodo di bisezione
si compone dei seguenti passi:
\begin{enumerate}
	\item Definire un intervallo di separazione che contiene una radice dell'equazione, quindi avente estremi
	      tali che la funzione calcolata sui due estremi abbia segno discorde. Definiamo quindi un intervallo
	      $[a, b]$ tale che
	      \[ f(a) \cdot f(b) < 0 \]
	\item Calcolare la funzione nel punto di mezzo $c = \frac{a + b}{2}$, ossia
	      \[ y = f(c) \]
	\item Se $y \cdot f(a)$ è positivo allora il nuovo intervallo diventa $[c, b]$, viceversa se è negativo
	      il nuovo intervallo diventa $[a, c]$.
	\item Si va avanti finché l'intervallo non è talmente piccolo che uno dei due estremi non approssima alla
	      soluzione.
\end{enumerate}

\subsection{Valutazione dell'errore}
Indichiamo con $a_k$ e con $b_k$ il valore dei due estremi alla $k$-esima iterazione. Osserviamo che
\begin{itemize}
	\item Se $f(a_k) \cdot f(b_k) < 0$ per ogni $k$ allora $\xi \in [a_k, b_k]$.
	\item Il valore $b_k - a_k$ rappresenta l'ampiezza dell'intervallo. Tale ampiezza equivale alla metà
	      dell'ampiezza dell'intervallo precedente
	      \[ b_k - a_k = \sum_{i=1}^{k} \frac{b_{k-i} - a_{k-i}}{2^i} \]
\end{itemize}
Per poter fare una valutazione dell'errore possiamo osservare che
\[ 0 \leq |a_k - \xi| \leq b_k - a_k \leq \frac{b_0 - a_0}{2^k} \]
Dato che
\[ \lim_{k \to +\infty} \frac{b_0 - a_0}{2^k} = 0 \]
allora $a_k \to \xi$. Vale la stessa relazione anche per $b_k$
\[ 0 \leq |a_k - \xi| \leq b_k - a_k \leq \frac{b_0 - a_0}{2^k} \]
e come prima $b_k \to \xi$. Per $c_k$ vale invece
\[ 0 \leq |a_k - \xi| \leq \frac{b_k - a_k}{2} \leq \frac{b_0 - a_0}{2^{k+1}} \]
e anche in questo caso $c_k \to \xi$. Abbiamo quindi generato tre successioni che progressivamente convergono
tutte a $\xi$.

\subsection{Criterio di arresto}
Per definire un criterio di arresto per il metodo di bisezione possiamo imporre la condizione
\[ |x - \xi| \leq \epsilon \]
che equivale a
\[ \frac{b_0 - a_0}{2^k} \leq \epsilon \]
Possiamo quindi determinare a priori il numero $k$ di iterazioni necessarie per ottenere una certa precisione
nell'approssimazione
\begin{align*}
	\frac{b_0 - a_0}{2^k} \leq                            & \epsilon     \\
	b_0 - a_0 \leq                                        & 2^k \epsilon \\
	\frac{b_0 - a_0}{\epsilon} \leq                       & 2^k          \\
	\log_2 \left( \frac{b_0 - a_0}{\epsilon} \right) \leq & k
\end{align*}
Per avere il numero di iterazioni basterà prendere
\[ k = \left\lceil \log_2 \left( \frac{b_0 - a_0}{\epsilon} \right) \right\rceil \]
Un altro criterio di arresto potrebbe essere
\[ f(c_k) \leq \epsilon \]
si va quindi a imporre che il valore della funzione calcolato nel punto di mezzo sia sufficientemente piccolo
e quindi quanto più vicino a 0.

Il metodo di bisezione ha il problema di essere generalmente lento in quanto deve compiere tante valutazioni
della funzione.
