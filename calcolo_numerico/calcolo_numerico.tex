\documentclass[11pt, a4paper]{report}

\pdfpagewidth\paperwidth
\pdfpageheight\paperheight

\usepackage[utf8]{inputenc}
\usepackage[T1]{fontenc}
\usepackage[italian]{babel}
\usepackage[hidelinks]{hyperref}
\usepackage{amsmath, amssymb, amsthm, amsfonts, mathtools}
\usepackage{tikz}
\usepackage{../packages/codestyle}

\theoremstyle{definition}
\newtheorem{theorem}{Teorema}[chapter]
\newtheorem{definition}{Definizione}[chapter]
\newtheorem{proposition}{Proposizione}[chapter]
\newtheorem{corollary}{Corollario}[chapter]
\newtheorem{lemma}{Lemma}[chapter]
\newtheorem{observation}{Osservazione}[chapter]
\newtheorem{example}{Esempio}[chapter]

\newcommand{\N}{\mathbb{N}}
\newcommand{\Z}{\mathbb{Z}}
\newcommand{\R}{\mathbb{R}}
\newcommand{\C}{\mathbb{C}}
\newcommand{\F}{\mathbb{F}}
\newcommand{\K}{\mathcal{K}}

\newcommand{\ein}{\epsilon_\text{in}}
\newcommand{\ealg}{\epsilon_\text{alg}}
\newcommand{\etot}{\epsilon_\text{tot}}
\newcommand{\tx}{\tilde{x}}

\DeclareMathOperator{\sign}{sign}
\DeclareMathOperator{\trn}{trn}
\DeclareMathOperator{\arr}{arr}
\DeclareMathOperator{\fl}{fl}
\DeclareMathOperator{\dist}{dist}
\DeclareMathOperator{\Ker}{Ker}

\title{Calcolo numerico}
\author{Federico Bustaffa}
\date{13/02/2023}

\begin{document}

\maketitle
\tableofcontents

\part{Calcolabilità}

\chapter{Introduzione alla calcolabilità}
Iniziamo con la \textbf{teoria della calcolabilità} la quale
si pone come obbiettivo quello di definire cosa siano problemi,
funzioni e algoritmi, cercando di dare una definizione formale
di questi ultimi. Una volta definiti questi concetti sarà di
nostro interesse capire quali sono i problemi
\textbf{calcolabili} e quali invece no.

In questa prima parte non è di nostro interesse tenere di conto
le limitazioni che hanno i calcolatori reali. Ragioneremo quindi
supponendo che non questi non abbiamo limiti in tempo o spazio
per effettuare il calcolo.

Cercheremo quindi di capire quali sono i problemi
\textbf{calcolabili} mediante una \textbf{procedura effettiva},
quali invece \textbf{non} sono calcolabili per capire se ce ne
sono di interessanti, se ne esistono di reali o se sono solo
artificiali e puramente teorici.

\chapter{L'aritmetica del calcolatore}
I punti d'interesse del corso sono quindi
\begin{itemize}
	\item Capire come i numeri sono rappresentati in macchina.
	\item Capire come viene implementata l'aritmetica in macchina.
	\item Capire come il calcolatore approssima i calcoli eseguiti.
	\item Capire se tale approssimazione è accettabile per il problema che stiamo cercando di risolvere.
\end{itemize}
In questo capitolo andremo a trattare la \emph{rappresentazione} dei numeri in macchina, l'\emph{implementazione}
dell'aritmetica e quali \emph{errori} si possono generare.

\section{Rappresentazione dei numeri}
Cerchiamo come prima cosa di rappresentare i numeri in modo \textbf{univoco}. Come prima cosa abbandoniamo la
forma frazionaria di rappresentazione. Come sappiamo le quantità
\[ 0.1 \quad \frac{1}{10} \quad \frac{2}{20} \]
hanno tutte lo stesso valore ma, come possiamo facilmente intuire, la forma frazionaria fornisce infinite
rappresentazioni tutte equivalenti per la stessa quantità $0.1$.

Considerando anche che la riduzione ai minimi termini di una frazione è un'operazione in generale costosa ma
necessaria al fine di avere espressioni più facili da calcolare, non ci possiamo permettere un tipo di
rappresentazione di questo tipo.

Ecco perché, per rappresentare numeri reali, utilizzeremo la rappresentazione tramite \emph{sequenza di cifre
	decimali}.

\begin{observation}
	Talvolta sarà necessario passare alla rappresentazione di numeri complessi ma, come sappiamo, possiamo vedere
	un numero complesso come una coppia di reali, dunque non ci saranno grosse differenze nella rappresentazione
	di questi ultimi.
\end{observation}

\begin{observation}
	La rappresentazione in cifre decimali potrebbe essere \textbf{infinita} (per esempio
	$\frac{1}{3} = 0.\overline{33}$), ma chiaramente non è possibile utilizzare una rappresentazione del genere
	su un calcolatore.
\end{observation}

Quest'ultima osservazione ci dice che un grado di approssimazione sarà necessario andando così a generare un
certo errore sulla rappresentazione.

Per una questione di \emph{minimizzazione} dello spazio utilizzato per la rappresentazione dei numeri in
macchina si farà ricorso ad esponenziali per rapprentare cifre non significative in forma \emph{compatta}.

\begin{example}
	Per esempio, per rappresentare 0.0000001 si usa la seguente notazione
	\[ 0.0000001 = 0.1 \cdot 10^{-6} \]
	in quanto ci permette di rappresentare quei 6 zeri dopo la virgola rappresentando solo il 6 all'esponente.
\end{example}

\subsection{Rappresentazione normalizzata in virgola mobile}
\begin{theorem}[Rappresentazione]\label{th: rappr}
	Dato $x \in \R$ tale che $x \neq 0$, e sia $B \in \N$ tale che $B > 1$, la \textbf{base} di numerazione,
	esistono e sono univocamente determinati:
	\begin{itemize}
		\item Un intero $p \in \Z$, detto \textbf{esponente} della rappresentazione.
		\item Una serie di numeri naturali $\{ d_i \}_{i \geq 1}$ con
		      \begin{itemize}
			      \item $d_1 \neq 0$
			      \item $0 \leq d_i \leq B - 1$
			      \item Tutti i $d_i$ non definitivamente uguali a $B - 1$
		      \end{itemize}
		      dette \textbf{cifre} di rappresentazione.
	\end{itemize}
	tali che
	\[ x = \sign(x) \cdot B^p \cdot \sum_{i=1}^{+\infty} d_i B^{-i} \]
\end{theorem}

\begin{definition}
	La serie $d_1 B^{-1} + d_2 B^{-2} + \dots$ è detta \textbf{mantissa}.
\end{definition}

La rappresentazione \ref{th: rappr} è detta \textbf{rappresentazione normalizzata in virgola mobile}
(\emph{floating point}) in quanto l'esponente $p$ è determinato in modo che il numero da rappresentare abbia
parte intera nulla e prima cifra dopo la virgola non nulla.

\subsubsection{Vincoli per la rappresentazione}
Andiamo ad analizzare meglio i vincoli imposti nel teorema \ref{th: rappr} sulla rappresentazione.
\begin{itemize}
	\item Le condizioni $d_1 \neq 0$ (rappresentazione normalizzata) e $d_i$ non definitivamente uguale a
	      $B - 1$ ci garantiscono l'\emph{unicità} della rappresentazione
	\item Il numero $x = 0$ non ammette rappresentazione normalizzata e in macchina viene trattato e
	      memorizzato in modo speciale.
\end{itemize}

\begin{example}
	Rappresentiamo 0.1 tramite la rappresentazione \ref{th: rappr}:
	\[ 0.1 = + 10^0 \cdot (1 \cdot 10^{-1} + 0 \cdot 10^{-2} + \dots) \]
	Se però volessimo rappresentare lo stesso numero in modo diverso, lo potremmo fare in questo modo
	\[ 0.1 = + 10^1 \cdot (0 \cdot 10^{-1} + 1 \cdot 10^{-2} + 0 \cdot 10^{-3} + \dots) \]
	Cambiando ogni volta l'esponente $p$ e adattando i $d_i$ è possibile trovare infinte rappresentazioni
	per 0.1. Dunque $d_1$ deve essere non nullo.
\end{example}

\begin{example}
	Come sappiamo $0.\overline{9} = 1$. Se volessimo rappresentare 1 otterremmo
	\[ 1 = + 10^1 \cdot (1 \cdot 10^{-1} + 0 \cdot 10^{-2} + \dots) \]
	mentre se rappresentassimo $0.\overline{9}$ otterremmo
	\[ 0.\overline{9} = + 10^0 \cdot (9 \cdot 10^{-1} + 9 \cdot 10^{-2} + \dots) \]
	Si ottiene così una rappresentazione normalizzata ($d_1 = 9$) e costituita da soli 9 dopo la virgola,
	ma dato che i due numeri sono equivalenti la rappresentazione utilizzata è la prima.
\end{example}

\subsection{Rappresentazione in macchina}
Per ora abbiamo dato una definizione ideale della rappresentazione in virgola mobile di un numero reale.
In macchina si ha però a disposizione un registro di lunghezza finita per memorizzarla. Tale registro è
suddiviso in tre parti:
\begin{itemize}
	\item Segno di $x$
	\item Esponente $p$
	\item Cifre di rappresentazione $\{ d_i \}$
\end{itemize}
Abbiamo quindi un numero finito di rappresentazioni possibili in macchina.

\begin{definition}\label{set_num_macchina}
	Si definisce \textbf{insieme dei numeri di macchina} in rappresentazione \emph{floating point} con $t$
	cifre, base $B$ e range $(-m, M)$, l'insieme dei numeri reali
	\[
		\F (B, t, m, M) = \{ 0 \} \cup
		\{ x \in \R : x = \sign(x) \cdot B^p \cdot \sum_{i=1}^t d_i B^{-i} \}
	\]
	con $0 \leq d_i \leq B-1$, $d_1 \neq 0$ e $-m \leq p \leq M$.
\end{definition}

\begin{observation}
	Guardando l'\emph{insieme dei numeri di macchina} possiamo osservare che
	\begin{itemize}
		\item Ha \textbf{cardinalità finita}
		      \[ N = 2 B^{t-1} (B - 1) (M + m + 1) + 1 \]
		      Otteniamo questo risultato dato che
		      \begin{itemize}
			      \item Il segno di $x$ può avere solo 2 valori possibili.
			      \item Avendo $t$ cifre disponibili per la rappresentazione abbiamo $B^t$ possibili
			            configurazioni. Ma dato che $d_1 \neq 0$ allora passiamo a $B^{t-1}$ configurazioni
			            effettive.
			      \item Ogni $d_i$ ha $B - 1$ possibili valori.
			      \item L'esponente è compreso tra $-m$ ed $M$ dunque abbiamo $m + M$ possibili valori ai quali
			            dobbiamo aggiungere lo zero.
			      \item Aggiungiamo lo zero.
		      \end{itemize}
		\item Se $x \in \F(B, t, m, M)$ e $x \neq 0$ allora $\omega = B^{-m - 1}$ è il più piccolo numero di
		      macchina positivo e $\Omega = B^M (1 - B^{-t})$ è il più grande numero di macchina positivo e
		      vale
		      \[ \omega \leq |x| \leq \Omega \]
		      Ne segue che non è possibile rappresentare esattamente numeri non nulli di modulo minore a
		      $\omega$. Ecco perché, per tali numeri, è prevista una rappresentazione \emph{denormalizzata}.

		      Quando $p = -m$ la condizione $d_1 \neq 0$ può essere abbandonata potendo quindi rappresentare
		      come numero più piccolo positivo $B^{-m-t}$ e dunque riuscendo a rappresentare numeri positivi
		      e negativi compresi in modulo tra
		      \[ [ B^{-m-t} \;;\; B^{-m-1} - B^{-m-t} ] \]

		      Analogamente se $p = M$ si introducono rappresentazioni speciali per i simboli $\pm \infty$
		      e NaN.
		\item L'insieme dei numeri di macchina $\F (B, t, m, M)$ è simmetrico rispetto all'origine.
		\item Posto $x = (-1)^s \cdot B^p \cdot \alpha$ appartente a $\F(B, t, m, M)$ allora il
		      successivo numero di macchina si ottiene sommando alle cifre di rappresentazione il più piccolo
		      numero di macchina rappresentabile in quell'ordine di grandezza, ossia
		      \[ B^p \cdot 1 \cdot B^{-t} = B^{p - t} \]
		      Otteniamo quindi che il numero di macchina successivo a $x$ è
		      \[ y = (-1)^s \cdot B^p \cdot (\alpha + B^{p-t}) \]
		      Questo ci dice che, in un sistema in virgola mobile, la distanza tra due numeri di macchina
		      \[ |y - x| = B^{p-t} \]
		      varia con $p$, ovvero con l'ordine di grandezza dei numeri considerati. In generale possiamo
		      immaginarci l'insieme dei numeri di macchina più denso vicino a $\omega$, al contrario andando
		      verso $\Omega$ la distanza tra i numeri di macchina sarà più grande.
	\end{itemize}
\end{observation}

\subsubsection{Standard IEEE per la doppia precisione}
Lo standard proposto dal IEEE per la rappresentazione in virgola mobile (base 2) prevede registri lunghi 64 bit.
In prima battuta si era pensato ad una suddivisione del registro di questo tipo:
\begin{itemize}
	\item 1 bit per il segno.
	\item 11 bit per l'esponente.
	\item 52 bit per le cifre di rappresentazione.
\end{itemize}
Il problema di questo modello di rappresentazione è la doppia rappresentazione dello 0 ($-0$ e $+0$) ed è stato
quindi scartato.

L'attuale standard rappresenta l'esponente in \textbf{traslazione}. Chiariamo meglio
il concetto procedendo con ordine:
\begin{enumerate}
	\item Dato che si hanno 11 bit per l'esponente si possono rappresentare numeri che vanno da 0 a 2047. Abbiamo
	      quindi 2048 configurazioni totali per l'esponente.
	\item Le due configurazioni di bit con tutti 0 e tutti 1 vengono tenute da parte e trattate in modo
	      \emph{speciale}. Ci rimangono dunque 2046 configurazioni.
	\item Lo standard definisce l'esponente nell'intervallo
	      \[ [ -1021, \; 1024 ] \]
	      Abbiamo dunque che $m = -1021$ e $M = 1024$.
\end{enumerate}
Notiamo inoltre che, in notazione normalizzata, $d_1$ deve essere $\neq 0$ ma dato che stiamo operando in
base 2 possiamo dedurre che $d_1$ deve essere sempre uguale a 1. Questo ci permette di non includere $d_1$
nelle cifre della rappresentazione dandoci la possibilità di usare, di fatto, 53 cifre di rappresentazione.

Come abbiamo detto poco fa, le due configurazioni dell'esponente con tutti 0 e tutti 1, sono configurazioni
\emph{speciali}. Nel caso in cui l'esponente sia rappresentato da tutti 1 distinguono due casi:
\begin{itemize}
	\item Nel caso in cui nella mantissa ci sia almeno una cifra diversa da 0 allora si rappresenta
	      $\pm \infty$
	\item Nel caso in cui tutte le cifre della mantissa siano uguali a 0 allora si rappresenta NaN.
\end{itemize}
Nel caso in cui l'esponente sia rappresentato da tutti 0 allora si usa la notazione \textbf{denormalizzata}
per riuscire a rappresentare anche un po' dei numeri che stanno tra 0 e $\omega$. Per riuscire a farlo
abbandoniamo il vincolo $d_1 \neq 0$. Nello specifico, si va rappresentare il numero in questo modo
\[ x = \sign(x) \cdot 2^{-1022} \cdot \sum_{i=1}^t d_i 2^{-i} \]
andando a rappresentare lo 0 nel caso in cui ci siano tutti 0 nella mantissa.

\section{Aritmetica di macchina}
Adesso rimane da capire come la macchina mappa i numeri reali in numeri di macchina. Il problema che ci
poniamo è il seguente: sia $x \in \R$, quale numero di macchina $\tilde{x} \in \F$ viene associato a $x$
per la rappresentazione?

Associare un numero di macchina $\tilde{x}$ a $x$ significa \textbf{approssimare} $x$ commettendo un
\textbf{errore relativo} di rappresentazione
\[ \epsilon_x = \frac{\tilde{x} - x}{x} = \frac{\eta_x}{x}, \quad x \neq 0 \]
quanto più piccolo possibile in valore assoluto. La quantità
\[ \eta_x = \tilde{x} - x \]
è detta \textbf{errore assoluto} di rappresentazione. Dato $x \in \R$ con $x \neq 0$, distinguiamo 2 casi:
\begin{itemize}
	\item $|x| < \omega$ (\textbf{underflow}) o $|x| > \Omega$ (\textbf{overflow})
	\item $\omega \leq |x| \leq \Omega$
\end{itemize}
Nel secondo caso le tecniche di approssimazione previste dallo standard IEEE 754 sono:
\begin{itemize}
	\item \textbf{Round to the nearest} (arrotondamento): il numero $x$ viene approssimato con il numero di
	      macchina $\tilde{x}$ più vicino.
	\item \textbf{Round toward zero} (troncamento): il numero $x$ viene approssimato con il più grande numero
	      di macchina $\tilde{x}$ il cui valore assoluto risulti minore o uguale al valore assoluto di $x$.
	\item \textbf{Round toward plus infinity}: il numero $x$ viene approssimato al più piccolo numero
	      rappresentabile maggiore del dato.
	\item \textbf{Round toward minus infinity}: il numero $x$ viene approssimato al più grande numero
	      rappresentabile minore del dato.
\end{itemize}
Assumiamo per semplicità di considerare una macchina che opera con troncamento sull'insieme $\F$. Per
convenzione indichiamo con $\trn(x) = \tilde{x}$ il risultato dell'approssimazione di $x$ con troncamento e
più generalmente $\fl(x)$ l'approssimazione in macchina del dato $x$ nel sistema \emph{floating point}
considerato. Il primo risultato fornisce una maggiorazione uniforme dell'errore di rappresentazione.

\begin{theorem}\label{th: prec}
	Sia $x \in \R$ con $\omega \leq |x| \leq \Omega$. Si ha
	\[ |\epsilon_x| = \left| \frac{\trn(x) - x}{x} \right| \leq u = B^{1-t} \]
	dove $u$ è detta \textbf{precisione di macchina}, la quale non dipende da $x$ e fornisce errori più grandi
	per numeri grandi ed errori piccoli per numeri piccoli.
	\begin{proof}
		Sia $x = (-1)^s \cdot B^p \alpha$. L'errore assoluto $|\trn(x) - x|$ risulta maggiorato dalla distanza
		tra i due numeri di macchina consecutivi e quindi
		\[ |\trn(x) - x| \leq B^{p-t} \]
		Inoltre $|x| \geq B^{p-1}$. Pertanto vale
		\[ |\epsilon_x| = \left| \frac{\trn(x) - x}{x} \right| \leq \frac{B^{p-t}}{B^{p-1}} = B^{1-t} = u \]
	\end{proof}
\end{theorem}

\begin{observation}
	Possiamo osservare che
	\begin{itemize}
		\item La \emph{precisione di macchina} è indipendente dalla grandezza del numero e caratteristica
		      dell'aritmetica \emph{floating point} implementata sulla macchina su cui stiamo operando. Se ad
		      esempio operiamo con arrotondamento, la distanza tra $x$ e la sua approssimazione di macchina, e
		      quindi la precisione di macchina, si dimezzano.
		\item Per valutare la precisione di macchina possiamo determinare il più piccolo numero di macchina
		      maggiore di 1. Sia infatti $x$ il numero che cerchiamo, abbiamo che $x - 1 = |x - 1| = B^{1-t}$,
		      essendo $1 = B^1 \cdot B^{-1}$ rappresentato con esponente $p = 1$.
		\item Dal teorema \ref{th: prec} si ricava che, dato $x \in \R$, in assenza di situazione di
		      \emph{underflow} o \emph{overflow}, per la sua rappresentazione in macchina vale che
		      \[ \fl(x) = x (1 + \epsilon_x), \quad |\epsilon_x| \leq u \]
		      Questa relazione esprime il modo in cui viene generalmente descritto il legame tra un numero reale
		      e la sua rappresentazione in macchina.
	\end{itemize}
\end{observation}

\subsection{Operazioni aritmetiche}
Per le operazioni aritmetiche eseguite in un sistema \emph{floating point} si pone un analogo problema di
approssimazione in quanto il risultato dell'operazione tra due numeri di macchina, in generale, non sarà
un numero di macchina.

Introduciamo dunque una variante delle quattro operazioni fondamentali che indichiamo con $\oplus$, $\ominus$,
$\otimes$ e $\oslash$. Per tali operazioni è ovviamente richiesto che siano interne all'insieme dei numeri
di macchina e che producano un'approssimazione quanto più accurata del risultato esatto.

Prendiamo ad esempio l'operazione di addizione in macchina. Un possibile implementazione è la seguente
\[ \tilde{x} \oplus \tilde{y} = \trn(\tilde{x} + \tilde{y}) \]
Qui usiamo il troncamento per semplicità ma in realtà l'arrotondamento è il metodo più usato in quanto fornisce
maggiore accuratezza per l'approssimazione. L'operazione scritta sopra produrrà quindi un errore di
approssimazione, il quale viene chiamato \textbf{errore di rappresentazione locale} dell'operazione.

\begin{example}
	Supponiamo di voler calcolare la quantità $f(x) = \frac{x - 1}{x}$. \`E immediato accorgersi che
	\[ \frac{x - 1}{x} = x - \frac{1}{x} \]
	E dunque già si identificano due possibile strade per il calcolo della quantità ma concentriamoci solo
	sulla prima per il momento.

	La prima cosa che fa la macchina è mappare $x$ in $\tilde{x}$. A questo dobbiamo stare molto attenti in
	quanto l'algoritmo girerà con
	\[ \tilde{x} = x (1 + \epsilon_x), \quad |\epsilon_x| \leq u \]
	e non con $x$, fornendo un risultato potenzialmente del tutto sbagliato.

	La cifra 1 la possiamo assumere come numero di macchina in quanto ogni sistema ragionevole include 1 tra
	i numeri di macchina.

	Se siamo molto fortunati calcolaremo esattamente $f(\tilde{x})$ ma la maggior parte delle volte non sarà
	così e ne calcolaremo un'approssimazione
	\[ g(\tilde{x}) = \frac{\tilde{x} \ominus 1}{\tilde{x}} \]
	Dunque calcoliamo una funzione di fatto differente dalla prima. Sviluppiando l'espressione otteniamo
	\begin{align*}
		g(\tilde{x})  = & \frac{\tilde{x} \ominus 1}{\tilde{x}}                                                \\
		=               & \frac{x (1 + \epsilon_x) \ominus 1}{x (1 + \epsilon_x)}                              \\
		=               & \frac{[x (1 + \epsilon_x) - 1](1 + \epsilon_1)}{x (1 + \epsilon_x)} (1 + \epsilon_2)
	\end{align*}
	Dove
	\[ |\epsilon_x| \leq u \quad |\epsilon_1| \leq u \quad |\epsilon_2| \leq u \]
	Noi siamo interessati ad un'analisi \emph{qualitativa} e per farla dobbiamo introdurre una semplificazione:
	Svolgiamo un'\textbf{analisi al primo ordine}, semplifichiamo tutti i termini (di errore) di ordine superiore
	al primo, in quanto considerabili come trascurabili.
	\begin{align*}
		       & \frac{[x (1 + \epsilon_x) - 1](1 + \epsilon_1)}{x (1 + \epsilon_x)} (1 + \epsilon_2) \\
		\doteq & \frac{x (1 + \epsilon_x) - 1}{x (1 + \epsilon_x)} (1 + \epsilon_1 + \epsilon_2)
	\end{align*}
	A questo punto dobbiamo portare $1 + \epsilon_x$ che sta al denominatore, a numeratore. Sapendo che, in
	un'analisi al primo ordine, vale che
	\[ \frac{1}{1 + \epsilon_x} \doteq 1 - \epsilon_x \]
	possiamo sviluppare il calcolo in questo modo
	\begin{align*}
		       & \frac{x (1 + \epsilon_x) - 1}{x (1 + \epsilon_x)} (1 + \epsilon_1 + \epsilon_2)        \\
		\doteq & \frac{x (1 + \epsilon_x) - 1}{x} (1 - \epsilon_x + \epsilon_1 + \epsilon_2)            \\
		=      & \frac{x - 1 + x \epsilon_x}{x} (1 - \epsilon_x + \epsilon_1 + \epsilon_2)              \\
		\doteq & \left( \frac{x - 1}{x} + \epsilon_x \right) (1 - \epsilon_x + \epsilon_1 + \epsilon_2) \\
		\doteq & \frac{x - 1}{x} (1 - \epsilon_x + \epsilon_1 + \epsilon_2) + \epsilon_x
	\end{align*}
	La funzione effettivamente calcolata in macchina è dunque
	\[ g(\tilde{x}) = \frac{x - 1}{x} (1 - \epsilon_x + \epsilon_1 + \epsilon_2) + \epsilon_x \]
	L'errore totale sul risultato sarà dunque
	\begin{align*}
		       & \frac{g(\tilde{x}) - f(x)}{f(x)}                                                            \\
		\doteq & \frac{\frac{x - 1}{x} + \frac{x-1}{x}(-\epsilon_x + \epsilon_1 + \epsilon_2) + \epsilon_x -
		\frac{x - 1}{x}}{\frac{x-1}{x}}                                                                      \\
		\doteq & -\epsilon_x + \epsilon_1 + \epsilon_2 + \epsilon_x \frac{x}{x - 1}                          \\
		=      & \epsilon_x \left( \frac{1}{x - 1} \right) + \epsilon_1 + \epsilon_2
	\end{align*}
	Questo ci dice che l'errore finale dipende dagli errori di rappresentazione sui dati e dagli errori locali
	delle operazioni.

	In questo specifico caso l'errore di rappresentazione di $x$ potrebbe essere amplificato molto quando $x$
	tende a 1. Se però $x$ è un numero di macchina (errore di rappresentazione nullo) allora $\epsilon_x = 0$
	e l'errore diventa
	\[ \epsilon_1 + \epsilon_2 \leq 2 u \]
	e il risultato è accurato.
\end{example}
\chapter{Analisi degli errori}
In questo capitolo andiamo a trattare gli errori nel calcolo di funzioni razionali e non e andiamo a vedere
alcune tecniche per l'analisi tali errori.

\section{Errori nel calcolo di una funzione razionale}
Inizialmente andremo a considerare funzioni di questo tipo
\[ f : \R \to \R\]
anche se si tratta di una semplificazione in quanto, in casi più realistici si ha a che fare con funzioni del tipo
\[ f : \R^n \to \R \]
Come abbiamo già anticipato nel capitolo precedente, quello che calcoliamo non è $f(x)$ ma, se abbiamo fortuna,
quello che andiamo a calcolare è $f(\tx)$.

\begin{definition}
	Definiamo \textbf{errore inerente} la quantità
	\[ \epsilon_{\text{in}} = \frac{f(\tx) - f(x)}{f(x)} \]
	La quale rappresenta l'errore che viene sempre commesso in quanto la macchina non riceve in ingresso
	il dato esatto ma quello approssimato.
\end{definition}

\begin{observation}
	Possiamo osservare che
	\begin{itemize}
		\item L'\emph{errore inerente} non dipende dall'algoritmo usato per la risoluzione del problema ma
		      misura la sensibilità della funzione $f$ rispetto alla perturbazione del dato iniziale.
		\item Se l'\emph{errore inerente} è qualitativamente elevato in valore assoluto diciamo che il problema
		      è \textbf{mal condizionato}. Al contrario, se è piccolo, diciamo che il problema è
		      \textbf{ben condizionato}.
	\end{itemize}
\end{observation}

\begin{definition}
	Definiamo \textbf{errore algoritmico} la quantità
	\[ \epsilon_{\text{alg}} = \frac{g(\tx) - f(\tx)}{f(\tx)} \]
	l'errore generato nel calcolo di $f(\tx) \neq 0$.
\end{definition}

\begin{observation}
	Possiamo osservare che
	\begin{itemize}
		\item La funzione $g$ dipende dall'algoritmo utilizzato per calcolare $f(x)$. In generale possiamo dire
		      che diversi algoritmi conducono a differenti errori algoritmici.
		\item Se l'\emph{errore algoritmico} è qualitativamente elevato in valore assoluto diciamo che l'algoritmo
		      è \textbf{numericamente instabile}. Al contrario, se è piccolo, diciamo che il problema è
		      \textbf{numericamente stabile}.
	\end{itemize}
\end{observation}

\begin{definition}
	Definiamo \textbf{errore totale} la quantità
	\[ \epsilon_{\text{tot}} = \frac{g(\tx) - f(x)}{f(x)} \]
	l'errore generato nel calcolo di $f(x) \neq 0$ mediante l'algoritmo specificato da $g$. L'\emph{errore totale}
	misura la differenza relativa tra l'output atteso e l'output effettivamente calcolato.
\end{definition}

\begin{theorem}
	In un'analisi al primo ordine vale
	\[ \epsilon_{\text{tot}} = \epsilon_{\text{in}} + \epsilon_{\text{alg}} \]
\end{theorem}

Il teorema esprime il fatto che nel calcolo di una funzione razionale in un'analisi al primo ordine le due fonti
di errore individuate precedentemente forniscono contributi separati che possono essere analizzati
indipendentemente. L'obbiettivo dell'analisi numerica è individuare algoritmi numericamente \emph{stabili} per
problemi \emph{ben condizionati}.
\chapter{Analisi e gestione del rischio}
L'approccio incondizionale comporta costi molto elevati e spesso inaccettabili, inoltre richiede una quantità di lavoro
enorme e spesso inutile.

Con l'approccio condizionale invece si cerca di capire quali componenti del sistema si possono difendere e soprattutto
quali componenti \emph{conviene} difendere.

Per capirlo è necessaria un'\textbf{analisi del rischio} con la quale si cerca di individuare la tipologia di attacco
più probabile in relazione al sistema che stiamo cercando di proteggere.

\section{Tipologie di analisi}
L'analisi del rischio si divide in diverse sottocategorie più specifiche che ci permettono di individuare in modo più
mirato eventuali problemi di diversa natura. In particolare parliamo di
\begin{itemize}
	\item Analisi delle risorse da proteggere (asset)
	\item Analisi delle minacce
	\item Analisi delle vulnerabilità
	\item Analisi degli attacchi
	\item Analisi degli impatti
	\item Individuazione del rischio, rischio accettabile ed introduzione di contromisure
\end{itemize}

\subsection{Analisi delle risorse}
In questa fase si cerca di individuare un insieme di oggetti ed alcune proprietà di sicurezza. Si definisce in seguito
una politica su questi oggetti in termini delle proprietà precedenti come diritti di lettura, scrittura, esecuzione
e così via.

\subsection{Analisi delle minacce}
In questa fase cerchiamo di capire chi è interessato ad attaccare il nostro sistema per rubare o modificare informazioni
o per impedire agli utenti di utilizzare il sistema.

Le possibili minacce possono arrivare sia da attaccanti che vogliono violare il sistema sia da eventi naturali che
potrebbero in qualche modo comprometterne l'integrità (in questo caso parliamo di \emph{safety}).

\subsubsection{Safety e Security}
Come anticipato, quando ci riferiamo alla capacità di un sistema di resistere ad eventi di origine non umana e casuale
come terremoti, crolli e così via, parliamo di \textbf{safety}.

Parliamo invece di \textbf{security} quando ci riferiamo alla capacità del sistema di resistere ad attacchi umani con
uno scopo malizioso per raggiungere un obbiettivo.

\subsection{Analisi delle vulnerabilità}
In questo caso cerchiamo di individuare quali sono le vulnerabilità che permettono ad un attaccante di ottenere, in un 
certo numero di passi, accesso alle risorse di suo interesse.
\chapter{Problemi di algebra lineare}
In questo capitolo verranno trattati i problemi dell'algebra lineare e il loro condizionamento tra cui la
risoluzione di sistemi lineari e il calcolo di autovalori e autovettori di matrici.

Il primo problema che andiamo a trattare è quello della risoluzione di sistemi lineari. Dall'algebra lineare
noi sappiamo che un sistema lineare è rappresentabile come
\[ A x = b \]
con $A \in \R^{n \times n}$ e $b \in \R^n$. La soluzione del sistema è quindi il vettore $x \in \R^n$ tale che
\[ x = A^{-1} b \]
Come sappiamo è possibile scrivere ognuna delle componenti $x_i$ di $x$ come funzione $f_i$ di $A$ e $b$ in
questo modo
\[ x_i = f_i (a_{11}, \dots, a_{nn}, b_1, \dots, b_n) \quad i = 1 \dots n \]
Abbiamo quindi $n$ funzioni del tipo $f : \R^{n \times n} \times \R^n \to \R$. Questa rappresentazione ci
fornisce un modo per scrivere la soluzione del problema dal punto di vista matematico ma non ci dà alcuna
informazione sul condizionamento del problema.

In macchina non abbiamo $A$ e $b$ ma $\hat{A}$ e $\hat{b}$, rispettivamente \emph{vicini} ad $A$ e $b$. Il
concetto di \emph{vicinanza} tra due numeri $x$ e $y$ per esempio è esprimibile come $|x - y|$. Quanto più
questo valore sarà vicino a 0 tanto più i due numeri saranno per l'appunto \emph{vicini}.

Il valore assoluto possiede delle proprietà interessanti che ci serviranno per quello che andremo a trattare
tra poco. Prendiamo $f(x) = |x|$, vale che
\begin{itemize}
	\item $f(x) \geq 0$ e $f(x) = 0 \Leftrightarrow x = 0$
	\item $f(\alpha x) = \alpha \cdot f(x)$
	\item $f(x + y) \leq f(x) + f(y)$
\end{itemize}

\section{Norme vettoriali e matriciali}
Dopo la breve premessa introduciamo il concetto di \textbf{norma} e chiariamo meglio la sua utilità in questo
ambito.

\subsection{Norma vettoriale}
\begin{definition}
	Definiamo \textbf{norma vettoriale} su $\R^n$ di un vettore $v$, una funzione del tipo $f : \R^n \to \R$ che
	soddisfa le seguenti proprietà:
	\begin{itemize}
		\item $\forall v \in \R^n$ vale che
		      \[ f(v) > 0, \quad f(v) = 0 \Leftrightarrow v = 0 \]
		\item $\forall v \in \R^n$ e $\forall \alpha \in \R$ vale che
		      \[ f(\alpha v) = |\alpha| \cdot f(v) \]
		\item $\forall v, w \in \R^n$ vale che
		      \[ f(v + w) \leq f(v) + f(w) \]
	\end{itemize}
\end{definition}

Per comodoità di notazione indicheremo la norma vettoriale $f$ per il vettore $v$ con $f(v) = \| v \|$. Definiamo
inoltre la \textbf{distanza} tra $v$ e $w$ come la norma vettoriale della loro differenza:
\[ \dist(v, w) = f(v - w) = \| v - w \| \]
Una funzione norma tra le più comuni nell'algebra lineare è quella per il calcolo della lunghezza di un vettore
\[ f(v) = \sqrt{v_1^2 + v_2^2 + \dots + v_n^2} = \sqrt{v^T v} = \| v \|_2 \]
Il 2 indica il fatto che stiamo facendo riferimento alla \textbf{norma 2} o \textbf{norma euclidea} del vettore $v$.
Tra le altre funzioni norma che utilizzeremo più spesso abbiamo la \textbf{norma 1} e la \textbf{norma infinito}:
\begin{itemize}
	\item $f(v) = |v_1| + |v_2| + \dots + |v_n| = \| v \|_1$
	\item $f(v) = \max(|v_i|) = \| v \|_\infty$ con $i = 1 \dots n$
\end{itemize}
In generale questi tre metodi di calcolo della norma di un vettore non forniscono lo stesso risultato. Quello che
conta è che qualsiasi scegliamo, un'analisi qualitativa fornisce gli stessi risultati.

\subsection{Norma matriciale}
Discorso analogo per il calcolo della distanza tra matrici.

\begin{definition}
	Definiamo \textbf{norma matriciale} su $\R^{n \times n}$ di una matrice $A$, una funzione del tipo
	$f : \R^{n \times n} \to \R$ che soddisfa le seguenti proprietà:
	\begin{itemize}
		\item $\forall A \in \R^{n \times n}$ vale che
		      \[ f(A) \geq 0, \quad f(A) = 0 \Leftrightarrow A = 0 \]
		\item $\forall A \in \R^{n \times n}$ e $\forall \alpha \in \R$ vale che
		      \[ f(\alpha \cdot A) = |\alpha| \cdot f(A) \]
		\item $\forall A, B \in \R^{n \times n}$ vale che
		      \[ f(A + B) \leq f(A) + f(B) \]
		\item $\forall A, B \in \R^{n \times n}$ vale che
		      \[ f(A \cdot B) \leq f(A) \cdot f(B) \]
	\end{itemize}
\end{definition}

Abbiamo aggiunto una proprietà in più ma il punto del discorso non cambia. Anche in questo caso se $f$ è una
norma matriciale su $\R^{n \times n}$ allora possiamo indicarla come $f(A) = \| A \|$ e possiamo definire la
distanza tra due matrici $A$ e $B$ come la norma della loro differenza:
\[ \dist(A, B) = f(A - B) = \| A - B \| \]
Per poter proseguire dobbiamo riuscire a definire delle norme matriciali legate in qualche modo alla norma
vettoriale.

\subsection{Norme matriciali indotte dalla norma vettoriale}
Introduciamo ora l'ultimo strumento necessario per poter risolvere e studiare problemi di algebra lineare in
macchina.
\begin{definition}
	Data una norma vettoriale su $\R^n$ si dice \textbf{norma matriciale indotta} la funzione
	$f : \R^{n \times n} \to \R$ tale che $\forall A \in \R^{n \times n}$ vale che
	\[ f(A) = \max \| A v \| \]
	sull'insieme $\{ v \in \R^n \mid \| v \| = 1 \}$, ossia dei vettori $v$ che hanno norma 1.
\end{definition}

Prendere una norma matriciale indotta ci permette di introdurre una proprietà fondamentale per l'analisi:
\[ \| A v \| \leq \| A \| \cdot \| v \| \]
Questo ci permette di maggiorare la lunghezza di $A v$ con la "\emph{lunghezza}" di $A$ per la lunghezza di $v$.

\subsubsection{Caratterizzazione}
L'insieme di vettori con norma 1 è infinito e per riuscire a trovare il massimo di $\| A v \|$ ricorriamo ai
cosiddetti \textbf{teoremi di caratterizzazione}. Questi teoremi ci indicano una qualche  proprietà che
\emph{caratterizza} la norma che stiamo considerando.

\begin{theorem}
	Sia $A \in \R^{n \times n}$, si ha che
	\[ \| A \|_\infty = \max \| A v \|_\infty \]
	sull'insieme dei vettori $v$ tali che $\| v \|_\infty = 1$. Si può dimostrare che questo equivale a
	\[ \max \sum_{j=1}^n |a_{i,j}| \quad i = 1 \dots n \]
	Si va quindi a calcolare la somma dei valori assoluti dei valori che compongono le righe della matrice e
	andando a prendere il massimo di tali somme.
\end{theorem}

\begin{theorem}
	Sia $A \in \R^{n \times n}$, si ha che
	\[ \| A \|_1 = \max \| A v \|_1 \]
	sull'insieme dei vettori $v$ tali che $\| v \|_1 = 1$. Si può dimostrare che questo equivale a
	\[ \max \sum_{i=1}^n |a_{i,j}| \quad j = 1 \dots n \]
	Si va quindi a calcolare la somma dei valori assoluti dei valori che compongono le colonne della matrice e
	andando a prendere il massimo di tali somme.
\end{theorem}

Andiamo ora a vedere perché la norma euclidea non sarà usata per il calcolo della norma matriciale.

\begin{theorem}
	Sia $A \in \R^{n \times n}$, si ha che
	\[ \| A \|_2 = \max \| A v \|_2 \]
	sull'insieme dei vettori $v$ tali che $\| v \|_2 = 1$. Si può dimostrare che questo equivale a
	\[ \sqrt{\rho (A^T A)} \]
	ossia alla radice quadrata del \emph{raggio spettrale} di $A^T A$. Il raggio spettrale di una matrice è il
	modulo dell'autovalore di modulo massimo.
\end{theorem}

Per quanto riguarda i primi due metodi abbiamo un complessità computazionale di $\theta (n^2)$ in quanto dobbiamo
fare la somma di tutti i valori di ogni riga/colonna e poi trovarne il massimo.

Nel caso della norma euclidea invece dobbiamo fare un prodotto tra matrici di costo $\theta (n^3)$ e in più
dobbiamo trovare gli autovalori della matrice. Quest'ultimo problema in particolare diventa molto complesso anche
per matrici di dimensioni contenute. Ecco perché, in generale, non utilizzeremo la norma euclidea per il calcolo
della norma di una matrice. L'unico caso in cui la complessità del problema ci rende possibile utilizzarla è
quello in cui abbiamo $A^T = A$ (matrice simmetrica).

\section{Condizionamento di sistemi lineari}
Ora che abbiamo chiarito il concetto di norma riprendiamo il discorso fatto all'inizio del capitolo e poniamoci
il problema di studiare il condizionamento di
\[ A x = b \]
Per semplicità supponiamo di perturbare solo il termine noto ottenendo
\[ A \hat{x} = b + \delta b \]
Come sempre consideriamo che, in generale, quando si va ad inserire una quantità in macchina questa non sarà un
numero di macchina. Se abbiamo quindi il vettore
\[ b = \begin{pmatrix} b_1 \\ \vdots \\ b_n \end{pmatrix} \]
in macchina avremo il vettore
\[
	\hat{b} = \begin{pmatrix} \hat{b}_1 \\ \vdots \\ \hat{b}_n \end{pmatrix} =
	\begin{pmatrix} b_1 (1 + \epsilon_1) \\ \vdots \\ b_n (1 + \epsilon_n)	\end{pmatrix} =
	\begin{pmatrix} b_1 \\ \vdots \\ b_n \end{pmatrix} +
	\begin{pmatrix} b_1 \epsilon_1 \\ \vdots \\ b_n \epsilon_n \end{pmatrix}
\]
Per riuscire a stimare quanto $x$ sia perturbato nel calcolo in macchina calcoliamo
\[
	\frac{\| \hat{x} - x \|}{\| x \|} =
	\frac{\| A^{-1} b + A^{-1} \delta b - A^{-1} b \|}{\| x \|} =
	\frac{\| A^{-1} \delta b \|}{\| x \|}
\]
Dato che siamo interessati ad una maggiorazione, utilizzando la norma matriciale indotta, possiamo scrivere
\[ \frac{\| A^{-1} \delta b \|}{\| x \|} \leq \frac{\| A^{-1} \| \cdot \| \delta b \|}{\| x \|} \]
Ci rimane da sostituire il denominatore. Per farlo teniamo di conto che $A x = b$ e quindi
\[ \| b \| \leq \| A \| \cdot \| x \| \Rightarrow \| x \| \geq \frac{\| b \|}{\| A \|} \]
Otteniamo quindi
\[
	\frac{\| A^{-1} \| \cdot \| \delta b \|}{\| x \|} \leq
	\frac{\| A^{-1} \| \cdot \| \delta b \|}{\frac{\| b \|}{\| A \|}} \leq
	\| A \| \cdot \| A^{-1} \| \cdot \frac{\| \delta b \|}{\| b \|}
\]
Questa relazione ci dice che l'errore relativo sul risultato è dato dall'errore relativo sui dati
($\| \delta b \| / \| b \|$) amplificato dalla quantità $\| A \| \cdot \| A^{-1} \|$ detta
\textbf{numero di condizionamento} del sistema lineare che indichiamo con
\[ \K (A) = \| A \| \cdot \| A^{-1} \| \]
Se questa quantità è grande il problema è \emph{mal condizionato}, in caso contrario il problema è
\emph{ben condizionato}.
\section{Autovalori e invertibilità}
L'obbiettivo che ci poniamo ora è quello di studiare gli autovalori di una matrice in modo da ricavare un metodo
semplice per riuscire a determinare se una matrice sia invertibile senza calcolare il determinante e senza avere
altre informazioni sull'invertibilità.

Dato che gli autovalori di una matrice $A$ a componenti reali possono essere numeri complessi, andremo fin da
subito a considerare $A \in \C^{n \times n}$.

Prima di andare più in dettaglio riprendiamo i concetti di algebra lineare relativi al calcolo degli autovalori.

\begin{definition}
	Definiamo lo scalare $\lambda$, \textbf{autovalore} di $A$ se $\exists x \neq 0 \in \C^n$ tale che
	\[ A x = \lambda x \]
	Chiamiamo $x$ che soddisfa tale relazione \textbf{autovettore} di $A$ relativo all'autovalore $\lambda$.
\end{definition}

\begin{observation}
	Osserviamo che
	\begin{align*}
		A x = \lambda x       & \Leftrightarrow \\
		A x - \lambda x = 0   & \Leftrightarrow \\
		A x - \lambda I x = 0 & \Leftrightarrow \\
		(A - \lambda I) x = 0
	\end{align*}
	Ne deduciamo che $\lambda$ è un autovalore di $A$ se e solo se
	\[ \det (A - \lambda I) = 0 \]
\end{observation}

\begin{definition}
	Il determinante di una matrice è un polinomio di grado $n$ negli elementi della matrice $A$ di dimensione
	$n \times n$, dunque possiamo scrivere
	\[ p(\lambda) = \det (A - \lambda I) \]
	Questo è detto \textbf{polinomio caratteristico} di $A$.
\end{definition}

Per il \emph{teorema fondamentale dell'algebra} sappiamo che un polinomio di grado $n$ ha $n$ radici complesse
e lo scalare $\lambda$ è autovalore di $A$ se e solo se $p(\lambda) = 0$. Da qui deduciamo che una matrice $A$
di dimensione $n \times n$ ha $n$ autovalori corrispondenti alle radici del suo polinomio caratteristico.

\begin{definition}
	Il numero di volte che un autovalore compare come radice del polinomio caratteristico è detto
	\textbf{molteplicità algebrica}.
\end{definition}

\begin{definition}
	Sia $\lambda$ un autovalore di $A$, definiamo la \textbf{molteplicità geometrica} di $\lambda$ come
	\[ \dim (\ker (A - b I)) \]
\end{definition}

In termini più pratici, la \emph{molteplicità geometrica} ci dice quanti autovettori linearmente indipendenti
troviamo nel nucleo rispetto all'autovalore.

Si può dimostrare che la \emph{molteplicità geometrica} è minore o uguale della \emph{molteplicità algebrica}.

\begin{definition}
	Una matrice $A$ si dice \textbf{diagonalizzabile} se esiste una matrice $S$ con $\det (S) \neq 0$ tale che
	\[ B = S \cdot A \cdot S^{-1} \]
	è una matrice diagonale.
\end{definition}

\begin{theorem}
	Una matrice con tutti gli autovalori distinti e quindi con molteplicità algebrica uguale alla molteplicità
	geometrica per ogni autovalore è diagonalizzabile.
\end{theorem}

\begin{theorem}
	La matrice $A$ è invertibile se e solo se $\lambda = 0$ non è un suo autovalore.
\end{theorem}

\subsection{Teoremi di localizzazione}
Questi teoremi non ci permettono di calcolare gli autovalori, bensì di definire dove questi si trovano sul piano
complesso.

L'idea è quella di identificare l'area all'interno della quale si trovano gli autovalori e se questa non comprende
lo 0 allora possiamo dedurre che la matrice sia invertibile.

Il contrario, in generale, non è vero. Non è detto che se lo 0 è compreso nell'area degli autovalori questo sia
autovalore. Non possiamo quindi dedurre che la matrice sia singolare.

\begin{theorem}[Gershgorin]
	Sia $A = (a_{ij}) \in \C^{n \times n}$. Definiamo i \textbf{cerchi di Gershgorin} $K_i$, con $1 \leq i \leq n$
	come
	\[ K_i = \{ z \in \C \mid |z - a_{ii}| \leq \sum_{j=1,\;j\neq i}^n |a_{ij}| \} \]
	Se $\lambda$ è un autovalore di $A$ allora questo appartiene all'unione dei $K_i$
	\[ \lambda \in \cup_{i=1}^n K_i \]
	\begin{proof}
		Sia $\lambda$ autovalore di $A$ con corrispondente autovettore destro $x$. La relazione $A x = \lambda x$
		implica che
		\[ \sum_{j=1}^n a_{ij} x_j = \lambda x_i \quad 1 \leq i \leq n \]
	\end{proof}
\end{theorem}

\section{Risoluzione di sistemi lineari}
Vediamo ora alcuni metodi per la risoluzione di sistemi lineari e andiamo a studiare il loro condizionamento
in macchina. Il problema che vogliamo provare a risolvere ha in input la matrice $A \in \R^{n \times n}$ e il
vettore $b \in \R^n$ dai quali vogliamo ricavare il vettore $x \in \R^n$ tale che $A x = b$.

\subsection{Matrici diagonali}
La prima cosa che vogliamo osservare è che per particolari classi della matrice $A$ il sistema è facilmente
risolubile. Prendiamo ad esempio la classe delle matrici \textbf{diagonali} ossia matrici con tutti gli elementi
fuori dalla diagonale principale nulli (per $i \neq j$ vale $a_{i,j} = 0$). Per esempio sono matrici diagonali
\[
	\begin{bmatrix}
		1 & 0 & 0 \\
		0 & 3 & 0 \\
		0 & 0 & 2
	\end{bmatrix} \quad
	\begin{bmatrix}
		2 & 0 & 0 \\
		0 & 5 & 0 \\
		0 & 0 & 0
	\end{bmatrix} \quad
	\begin{bmatrix}
		0 & 0 & 0 \\
		0 & 0 & 0 \\
		0 & 0 & 0
	\end{bmatrix}
\]
Se proviamo a calcolare il determinante della matrice $A$ possiamo vedere che questo equivale al prodotto degli
elementi sulla diagonale principale, così come possiamo dimostrare che gli autovalori sono esattamente gli
elementi sulla diagonale principale.

Segue dalle considerazioni appena fatte che una matrice diagonale è invertibile se e solo se ha tutti gli elementi
sulla diagonale principale diversi da 0. Per risolvere un sistema diagonale possiamo scrivere ogni equazione del
sistema nella forma
\[ a_{i,i} x_i = b_i \]
ottenendo facilmente che
\[ x_i = \frac{b_i}{a_{i,i}} \]
che è ben definito se tutti gli $a_{i,i}$ sono diversi da 0. Un semplice algoritmo che risolve un sistema diagonale
$n \times n$ è il seguente:
\begin{lstlisting}[language=pseudo, style=pseudo-style]
for i = 1 to n
	x[i] = b[i] / a[i][i]
\end{lstlisting}
Dunque il costo per risoluzione di un sistema diagonale è di $O(n)$ operazioni.

\subsection{Matrici triangolari}
Prendiamo ora in esame il caso delle matrici triangolari.

\begin{definition}
	Una matrice si dice \textbf{triangolare} se ha tutti elementi nulli sopra la diagonale principale e, in
	tal caso, si dirà \textbf{inferiore} (per $i < j$ vale $a_{i,j} = 0$). Se invece ha tutti elementi nulli
	sotto la diagonale principale si dirà \textbf{superiore} (per $i > j$ vale $a_{i,j} = 0$).
\end{definition}

Non facciamo nessuna assunzione sugli altri elementi: la matrice nulla è sia triangolare superiore che inferiore.
Esempi di matrici triangolari inferiori e superiori sono
\[
	\begin{bmatrix}
		1 & 0 & 0 \\
		2 & 4 & 0 \\
		3 & 1 & 9
	\end{bmatrix} \quad
	\begin{bmatrix}
		1 & 3 & 5 \\
		0 & 2 & 7 \\
		0 & 0 & 4
	\end{bmatrix}
\]
Il determinante di una matrice triangolare, se calcolato ad esempio con lo sviluppo di Laplace secondo la prima
colonna, risulta ancora una volta equivalente al prodotto degli elementi sulla diagonale principale.

Il discorso fatto per gli autovalori di matrici diagonali vale anche qui, dato che la matrice $A - \lambda I$ è
ancora una matrice triangolare, abbiamo anche in questo caso che gli autovalori di una matrice triangolare sono
esattamente gli elementi sulla diagonale principale.

Prendiamo un generico sistema di equazioni $A x = b$ con $A$ triangolare superiore. Otterremo qualcosa di questo
tipo
\[
	\begin{cases}
		a_{1,1} x_1 + a_{1,2} x_2 + \dots + a_{1,n} x_n = b_1 \\
		a_{2,2} x_2 + \dots + a_{2,n} x_n = b_2               \\
		\dots                                                 \\
		a_{n,n} x_n = b_n
	\end{cases}
\]
Per risolvere il sistema si fa ricorso al metodo di \textbf{sostituzione all'indietro}: si ricava $x_n$
dall'ultima equazione, lo si sostituisce nella penultima trovando $x_{n-1}$ e così via fino a ricavare tutti gli
$x_i$.

Supponiamo di aver già ricavato $x_{k+1}, \dots, x_n$ e di voler determinare $x_k$. Dalla $k$-esima equazione si
ottiene
\[
	a_{k,k} x_k = \sum_{j=k+1}^n a_{k,j} x_j = b_k \quad \to \quad
	x_k = \frac{b_k - \displaystyle\sum_{j=k+1}^n a_{k,j} x_j}{a_{k,k}}
\]
In MatLab la \emph{sostituzione all'indietro} sarebbe così implementata:
\begin{lstlisting}[language=matlab]
function[x] = solve_tri(a, b)
	n = length(b);
	x = zeros(n, 1);
	x(n) = b(n) / a(n, n);
	for k = n-1 : -1 : 1
		s = 0;
		for j = k+1 : n
			s = s+ a(k, j) * x(j);
		end
		x(k) = (b(k) - s) / a(k, k);
	end
end
\end{lstlisting}
Il costo computazionale di questo algoritmo è di $O(n^2)$ moltiplicazioni:
\[ \sum_{k=1}^n k = \frac{n (n + 1)}{2} \]
Se $A \in \R^{n \times n}$ si può pensare, per la risoluzione del sistema lineare $A x = b$, di \emph{ridurre}
progressivamente $A$ in forma triangolare mediante una sequenza di trasformazioni del tipo
\begin{align*}
	A_0 = & A, \quad A_k \to A_{k+1}, \quad 0 \leq k \leq n - 2; \\
	b_0 = & b, \quad b_k \to b_{k+1}, \quad 0 \leq k \leq n - 2
\end{align*}

\subsection{Fattorizzazione LU}
Introduciamo quindi il concetto di \textbf{fattorizzazione}, ossia scrivere una matrice come prodotto di più
matrici.

Vogliamo quindi rappresentare la matrice $A \in \R^{n \times n}$ come prodotto di due matrici $L$ ed $U$, la
prima triangolare inferiore con elementi uguali a 1 sulla diagonale pricipale e la seconda triangolare superiore.
\[
	A = L \cdot U = \begin{bmatrix}
		1       & 0      & \dots     & 0      \\
		l_{2,1} & \ddots & \ddots    & \vdots \\
		\vdots  & \ddots & \ddots    & 0      \\
		l_{n,1} & \dots  & l_{n,n-1} & 1
	\end{bmatrix} \cdot
	\begin{bmatrix}
		u_{1,1} & \dots  & \dots  & u_{1,n} \\
		0       & \ddots &        & \vdots  \\
		\vdots  & \ddots & \ddots & \vdots  \\
		0       & \dots  & 0      & u_{n,n}
	\end{bmatrix}
\]

Se siamo in grado di fare questo allora diciamo che $A$ è \textbf{fattorizzabile} nella forma $LU$
(fattorizzazione di Gauss).

\begin{observation}
	Se $A$ può essere scritta in questa forma allora, per il teorema di Binet, possiamo dire che
	\[ \det(A) = \det(L) \cdot \det(U) \]
	Dato che $L$ ha tutti 1 sulla diagonale principale $\det(L) = 1$ quindi vale che
	\[ \det(A) = \det(U) \neq 0 \]
	Ne deduciamo quindi che, se tale fattorizzazione esiste, allora $L$ e $U$ sono invertibili. Vale quindi che
	\[ A x = b \quad \Leftrightarrow \quad LU x = b \]
	Imponendo $U x = y$ possiamo scrivere
	\[
		\begin{cases}
			L y = b \\
			U x = y
		\end{cases}
	\]
	Il sistema lineare è dunque risolvibile tramite la risoluzione di due sistemi triangolari.
\end{observation}

\begin{observation}
	In questo modo abbiamo anche un metodo veloce per il calcolo del determinante di $A$ in quanto questo equivale
	al determinante di $U$. E dato che $U$ è triangolare il suo determinante è equivalente al prodotto degli
	elementi sulla diagonale principale.
\end{observation}

Il problema è che, in generale, una decomposizione di questo tipo \textbf{non esiste} e se esiste
\textbf{non è univocamente} determinata.

\begin{example}
	Prendiamo la matrice
	\[
		A  = \begin{bmatrix}
			0 & 1 \\
			1 & 0
		\end{bmatrix}
	\]
	e proviamo a cercare una fattorizzazione per tale matrice in questo modo
	\[
		\begin{bmatrix}
			0 & 1 \\
			1 & 0
		\end{bmatrix} =
		\begin{bmatrix}
			1 & 0 \\
			x & 1
		\end{bmatrix} \cdot
		\begin{bmatrix}
			u_1 & u_2 \\
			0   & u_3
		\end{bmatrix}
	\]
	Vogliamo provare a trovare $x$, $u_1$, $u_2$ e $u_3$ che soddisfano questa relazione. Se facciamo prima riga
	per prima colonna notiamo subito che $u_1$ deve essere uguale a 0, ma a questo punto, se proviamo a fare
	seconda riga per prima colonna otteniamo come risultato 0 per qualunque valore di $x$, quando invece dovremmo
	ottenere come risultato 1.
\end{example}

In generale, possiamo dire che una fattorizzazione di quel tipo non esiste neanche se la matrice è invertibile.

\begin{example}
	Proviamo ora a trovare una fattorizzazione per la matrice nulla
	\[
		\begin{bmatrix}
			0 & 0 \\
			0 & 0
		\end{bmatrix} =
		\begin{bmatrix}
			1 & 0 \\
			x & 1
		\end{bmatrix} \cdot
		\begin{bmatrix}
			0 & 0 \\
			0 & 0
		\end{bmatrix}
	\]
	Come possiamo vedere, la fattorizzazione in questo caso esiste ma non è univocamente determinata. Questo
	perché, per qualsiasi valore di $x$, otteniamo una fattorizzazione $LU$.
\end{example}

Prima di proseguire introduciamo una notazione. Sia $A \in \R^{n \times n}$ la matrice ad elementi $(a_{i,j})$
chiamiamo $A_k = (a_{i,j}) \in \R^{k \times k}$ \textbf{sottomatrici principali di testa} di $A$. Si tratta delle
matrici formate dalle prime $k$ righe e prime $k$ colonne di $A$.

\begin{theorem}
	Sia $A \in \R^{n \times n}$, se $\det(A_k) \neq 0$ per $1 \leq k \leq n-1$ allora esiste ed è unica la
	fattorizzazione $LU$ di $A$.
	\begin{proof}
		Dimostriamo per induzione su $n$. Per $n = 1$ allora $n-1 = 0$ dunque il teorema ci dice che, presa
		una matrice $1 \times 1$ esiste ed è unica la sua fattorizzazione $LU$
		\[ (a) = (1) \cdot (a) \]
		Assumiamo che il teorema sia vero per una qualsiasi matrice di ordine fino a $n-1$ e consideriamo una
		matrice $A$ di ordine $n$ che scriviamo in questo modo
		\[
			A = \begin{bmatrix}
				A_{n-1} & z       \\
				w^T     & a_{n,n}
			\end{bmatrix} = L \cdot U =
			\begin{bmatrix}
				L_{n-1} & 0 \\
				x^T     & 1
			\end{bmatrix} \cdot
			\begin{bmatrix}
				U_{n-1} & y       \\
				0       & u_{n,n}
			\end{bmatrix}
		\]
		Vogliamo quindi trovare due matrici $L$ e $U$ tali che $L \cdot U = A$. Per riuscire a farlo cerchiamo
		gli elementi dei vari partizionamenti con i quali abbiamo organizzato le matrici tali che $L \cdot U = A$.

		Tenendo conto delle dimensioni è possibile fare il prodotto righe per colonne anche per interi blocchi di
		matrice. Otteniamo quindi
		\[
			\begin{cases}
				L_{n-1} \cdot U_{n-1} = A_{n-1} \\
				L_{n-1} \cdot y = z             \\
				x^T \cdot U_{n-1} = w^T         \\
				x^T \cdot y + u_{n,n} = a_{n,n}
			\end{cases}
		\]
		La prima equazione ci dice che $L_{n-1} \cdot U_{n-1}$ è una fattorizzazione della matrice $A_{n-1}$.
		Dato che $A_{n-1}$ è una matrice di ordine $n-1$ e le sue sottomatrici principali di testa fino all'ordine
		$n-2$ sono invertibili per l'ipotesi del teorema e per ipotesi induttiva il teorema è vero per tutte le
		matrici fino all'ordine $n-1$. Quindi tale fattorizzazione esiste ed è unica.

		Per trovare $y$ dobbiamo risolvere un sistema lineare e vale
		\[ y = L_{n-1}^{-1} \cdot z \]
		e sappiamo che $L_{n-1}$ è invertibile perché si tratta di una matrice triangolare con tutti 1 sulla
		diagonale principale.

		Anche la terza equazione è un sistema lineare e, dato che $U_{n-1}$ è invertibile per ipotesi del teorema
		la quale ci garantisce l'invertibilità fino all'ordine $n-2$ ma anche $A_{n-1}$ è invertibile e quindi,
		per il teorema di Binet vale che
		\[ \det(A_{n-1}) = \det(U_{n-1}) \]
		Possiamo quindi scrivere
		\[ x^T = w^T \cdot U_{n-1} \]
		Una volta ricavati $x^T$ e $y$ troviamo $u_{n,n}$ in modo univoco calcolando
		\[ u_{n,n} = a_{n,n} - x^T \cdot y \]
		Il sistema è quindi univocamente risolubile e quindi la fattorizzazione esiste ed è unica.
	\end{proof}
	Quando le ipotesi del teorema sono violate la fattorizzazione non esiste o non è univocamente determinata.
\end{theorem}

La dimostrazione del teorema appena enunciato ci dice che se siamo interessati a calcolare la fattorizzazione
$LU$ di una matrice di ordine $n$, è possibile farlo calcolando la fattorizzazione $LU$ della sua sottomatrice
di ordine $n-1$ per poi aggiornarla con tale costruzione.

Supponendo di avere la fattorizzazione di ordine $n-1$, il costo per trovare al fattorizzazione di ordine $n$
è dato dal costo per trovare la fattorizzazione di ordine $n-1$ sommato al costo dell'\emph{aggiornamento}.

Il costo dell'aggiornamento è pari al costo della risoluzione di due sistemi triangolari di ordine $n-1$ ossia
\[ c (n - 1)^2 \]
Il costo complessivo dell'algoritmo per la fattorizzazione è dunque
\[ F(n) = F(n-1) + c (n-1)^2 = c \cdot \sum_{i=1}^{n} (n - i)^2 \]
Il costo che ne ricaviamo è di $O(n^3)$ operazioni. Il problema di questo metodo è che, implementato in macchina,
riporta problemi di stabilità.

L'algoritmo può essere comunque interessanti in alcuni casi, ad esempio per alcune classi di matrici particolari.

\begin{example}
	Prendiamo la matrice $A \in \R^{n \times n}$ così definita
	\[
		A = \begin{bmatrix}
			1      & 0      & \dots  & 0      & x_1    \\
			0      & \ddots & \ddots & \vdots & \vdots \\
			\vdots & \ddots & \ddots & 0      & \vdots \\
			0      & \dots  & 0      & 1      & \vdots \\
			x_1    & \dots  & \dots  & \dots  & x_n
		\end{bmatrix}
	\]
	detta \textbf{matrice a freccia}. Il problema che ci poniamo è trovare per quali $x_i$ con $1 \leq i \leq n$
	la matrice ammette fattorizzazione $LU$ e in caso affermativo, trovare la fattorizzazione $LU$ per tali
	valori.

	\`E immediato notare che le sottomatrici principali di testa di $A$ sono tutte matrici identità e dunque vale
	\[ A_k = I_k \]
	quindi sono tutte invertibili. Per il teorema esiste ed è unica la sua fattorizzazione.

	Per trovare tale fattorizzazione dobbiamo trovare la fattorizzazione $LU$ della matrice di ordine $n-1$. Per
	farlo scriviamo
	\[
		A = \begin{bmatrix}
			L_{n-1} & 0 \\
			x^T     & 1
		\end{bmatrix} \cdot
		\begin{bmatrix}
			U_{n-1} & y       \\
			0       & u_{n,n}
		\end{bmatrix}
	\]
	Come abbiamo detto, abbiamo che
	\[ A_{n-1} = I_{n-1} \]
	L'identità di ordine $n-1$ si fattorizza come l'identità per se stessa. Possiamo quindi scrivere
	\[
		A = \begin{bmatrix}
			I_{n-1} & 0 \\
			x^T     & 1
		\end{bmatrix} \cdot
		\begin{bmatrix}
			I_{n-1} & y       \\
			0       & u_{n,n}
		\end{bmatrix}
	\]
	A questo punto ricaviamo che
	\begin{align*}
		y =       & \begin{bmatrix} x_1 \\ \vdots \\ x_{n-1} \end{bmatrix} \\
		x^T =     & \begin{bmatrix} x_1 & \dots & x_{n-1} \end{bmatrix}    \\
		u_{n,n} = & x_n - \sum_{i=1}^{n-1} x_i^2
	\end{align*}
	Quando $u_{n,n} \neq 0$ allora la matrice è invertibile, ossia quando
	\[ x_n - \sum_{i=1}^{n-1} x_i^2 \neq 0 \]
	Risolvere i sistemi lineari per $L$ e $U$ appena trovate con l'eliminazione gaussiana ha un costo di $O(n)$
	operazioni.
\end{example}

\begin{definition}
	La matrice $A$ si dice \textbf{tridiagonale} se, per $|i - j| > 1$, l'elemento $a_{i,j}$ è uguale a 0.
\end{definition}

\begin{definition}
	Sia $A$ una matrice di dimensione $n \times n$, $A$ si dice \textbf{predominante diagonale} se il valore
	assoluto di ogni elemento diagonale è maggiore alla somma dei valori assoluti degli elementi sulla stessa
	riga escluso l'elemento diagonale.
	\[ |a_{i,i}| > \sum_{j=1, j \neq i}^{n} |a_{i,j}| \quad i = 1 \dots n \]
	In questo caso la matrice si dice \emph{predominante diagonale per righe}. Se
	l'operazione viene fatta effettuando la somma degli elementi di ciascuna colonna si parlerà di
	\emph{predominanza diagonale per colonne}.
\end{definition}

\begin{proposition}
	Se una matrice è predominante diagonale allora è invertibile.
\end{proposition}

\begin{proposition}
	Se una matrice è predominante diagonale allora anche le sue sottomatrici principali di testa sono predominanti
	diagonali.
\end{proposition}

\begin{proposition}
	Se una matrice è predominante diagonale esiste ed è unica la sua fattorizzazione $LU$ in quanto le sue
	sottomatrici principali di testa sono anch'esse predominanti diagonali e quindi invertibili.
\end{proposition}

\begin{example}
	Prendiamo la seguente matrice \emph{tridiagonale} $A \in \R^{n \times n}$
	\[
		A = \begin{bmatrix}
			3      & -1     & 0      & \dots  & 0      \\
			-1     & \ddots & \ddots & \ddots & \vdots \\
			0      & \ddots & \ddots & \ddots & 0      \\
			\vdots & \ddots & \ddots & \ddots & -1     \\
			0      & \dots  & 0      & -1     & 3
		\end{bmatrix}
	\]
	Se volessimo applicare il teorema dobbiamo andare a verificare l'invertibilità di tutte le sottomatrici fino
	all'ordine $n-1$. Le sottomatrici sono tutte \emph{predominanti diagonali} in quanto, se $A$ è predominante
	diagonale anche le sue sottomatrici sono predominanti diagonali e dunque invertibili.
\end{example}

\subsection{Eliminazione gaussiana}
Parliamo ora di come sia possibile calcolare una fattorizzazione $LU$ per la risoluzione di sistemi lineari
tramite eliminazione gaussiana.

\subsubsection{Matrici elementari di Gauss}
Fino ad ora non abbiamo parlato di fattorizzazione $LU$ ma non abbiamo un vero e proprio algoritmo per riuscire a
calcolarla. A tal proposito dobbiamo introdurre le \textbf{matrici elementari di Gauss}, ossia matrici elementari
espresse come
\[ A = I + u \cdot v^T \]
Le matrici elementari di Gauss sono particolari matrici elementari nelle quali il vettore $v$ è un vettore della
base canonica e il vettore $u$ ha le prime $j$ componenti nulle.
\[ A = I + u \cdot e_j^T \]
con $u_1 = u_2 = \dots = u_j = 0$. Le matrici elementari di Gauss sono triangolari inferiori con tutti 1 sulla
diagonale principale. Per esempio per $j=1$ abbiamo
\[
	I + u \cdot e_1^T =
	\begin{bmatrix}
		1   & 0 & 0 \\
		u_2 & 1 & 0 \\
		u_3 & 0 & 1
	\end{bmatrix}
\]
\begin{theorem}
	L'inversa di una matrice elementare di Gauss è equivalente alla stessa matrice dove si cambia segno agli
	elementi del vettore $u$.
	\[ A^{-1} = I - u \cdot e_j^T \]
	\begin{proof}
		Per dimostrarlo è sufficiente verificare la seguente uguaglianza
		\[ (I + u \cdot e_j^T) \cdot (I - u \cdot e_j^T) = I \]
		Sviluppando il prodotto otteniamo
		\[
			(I + u \cdot e_j^T) \cdot (I - u \cdot e_j^T) =
			I + u \cdot e_j^T - u \cdot e_j^T - u \cdot e_j^T \cdot u \cdot e_j^T
		\]
		Dato che il prodotto riga per colonna tra matrici è associativo è possibile svolgere per primo il prodotto
		riga per colonna
		\[ e_j^T \cdot u = 0 \]
		A questo punto l'ultimo termine dell'equazione si annulla e otteniamo l'identità.
	\end{proof}
\end{theorem}

Le matrici elementari di Gauss godono di una proprietà importante per riuscire a calcolare la fattorizzazione
$LU$ di una matrice $A$. In particolare questa proprietà è necessaria per riuscire a mettere degli zeri dove
necessario in modo da ottenere una matrice triangolare.

\begin{theorem}\label{th: eliminazione_gauss}
	Sia $x \in \R^n$ con $x_k \neq 0$ allora esiste una matrice elementare di Gauss $E$ tale che
	\[
		E \cdot x = \begin{bmatrix}
			x_1 \\ \vdots \\ x_k \\ 0 \\ \vdots \\ 0
		\end{bmatrix}
	\]
	\begin{proof}
		Prendiamo
		\[ E = I + u e_k^T \]
		con $u_1 = \cdots = u_k = 0$. Calcoliamo quindi $E \cdot x$
		\[
			(I + u e_k^T) x = x + u x_k =
			\begin{bmatrix}
				x_1 \\ \vdots \\ x_k \\ x_{k+1} \\ \vdots \\ x_n
			\end{bmatrix} + x_k
			\begin{bmatrix}
				0 \\ \vdots \\ 0 \\ u_{k+1} \\ \vdots \\ u_n
			\end{bmatrix} =
			\begin{bmatrix}
				x_1 \\ \vdots \\ x_k \\ x_{k+1} + x_k u_{k+1} \\ \vdots \\ x_n + x_k u_n
			\end{bmatrix}
		\]
		Dato che vogliamo le componenti dalla $(k+1)$-esima alla $n$-esima uguali a 0 basta imporre
		\begin{align*}
			u_{k+1} = & -\frac{x_{k+1}}{x_k} \\
			\vdots    &                      \\
			u_n =     & -\frac{x_n}{x_k}
		\end{align*}
		Le quali sono quantità ben definite dato che $x_k \neq 0$.
	\end{proof}
\end{theorem}

\subsubsection{Algoritmo per l'eliminazione gaussiana}
Passiamo ora al calcolo della fattorizzazione $LU$ tramite matrici elementari di Gauss. Supponiamo di avere una
generica matrice $A$ così definita
\[
	A = A_0 = \begin{bmatrix}
		a_{1,1}^{(0)} & \dots & a_{1,n}^{(0)} \\
		\vdots        &       & \vdots        \\
		a_{n,1}^{(0)} & \dots & a_{n,n}^{(0)}
	\end{bmatrix}
\]
Per questo metodo supponiamo $a_{1,1}^{(0)} \neq 0$. Allora possiamo determinare $E_1$ tale che
\[
	E_1 \cdot \begin{bmatrix}
		a_{1,1}^{(0)} \\ \vdots \\ a_{n,1}^{(0)}
	\end{bmatrix} =
	\begin{bmatrix}
		a_{1,1}^{(0)} \\ 0 \\ \vdots \\ 0
	\end{bmatrix}
\]
Ricaviamo quindi che
\[ E_1 = I + u e_1^T \]
con
\[
	u = \begin{bmatrix}
		0 \\ -\frac{a_{2,1}^{(0)}}{a_{1,1}^{(0)}} \\ \vdots \\ -\frac{a_{n,1}^{(0)}}{a_{1,1}^{(0)}}
	\end{bmatrix}
\]
A questo punto, possiamo generare la matrice $A_1$ come segue
\[ A_1 = E_1 \cdot A_0 = E_1 \cdot A \]
La matrice $A_1$ ha la seguente forma
\[
	A_1 = \begin{bmatrix}
		a_{1,1}^{(1)} & \dots         & \dots & a_{1,n}^{(1)} \\
		0             & a_{2,2}^{(1)} & \dots & a_{2,n}^{(1)} \\
		\vdots        & \vdots        &       & \vdots        \\
		0             & a_{n,2}^{(1)} & \dots & a_{n,n}^{(1)}
	\end{bmatrix}
\]
dove
\[ a_{1,i}^{(1)} = a_{1,i}^{(0)} \]
e con
\[ a_{i,j}^{(1)} = a_{i,j}^{(0)} + \left( -\frac{a_{i,1}^{(0)}}{a_{1,1}^{(0)}} \right) \cdot a_{1,j}^{(0)} \]
Questo per ogni elemento $a_{i,j}^{(1)}$ con $2 \leq i,j \leq n$.

Gli elementi sulla prima colonna di $E_1$ eccetto l'1 sulla diagonale sono detti \textbf{moltiplicatori} di
Gauss. Per riuscire a calcolarli dobbiamo fare $n-1$ divisioni. Una volta ottenuti dobbiamo fare un prodotto
e un'addizione per $(n-1)^2$ elementi per aggiornare la matrice. Ne deduciamo che il passaggio da $A_0$ ad $A_1$
è costato $(n-1)^2$ operazioni moltiplicative.

Ora passiamo alla seconda colonna e supponiamo $a_{2,2}^{(1)} \neq 0$ allora possiamo determinare $E_2$ tale che
\[
	E_2 \cdot \begin{bmatrix}
		a_{1,2}^{(1)} \\ \vdots \\ a_{n,2}^{(1)}
	\end{bmatrix} =
	\begin{bmatrix}
		a_{1,2}^{(1)} \\ a_{2,2}^{(1)} \\ 0 \\ \vdots \\ 0
	\end{bmatrix}
\]
Per il teorema \ref{th: eliminazione_gauss} abbiamo che
\[ E_2 = I + u e_2^T \]
dove
\[
	u = \begin{bmatrix}
		0 \\ 0 \\ -\frac{a_{3,2}^{(1)}}{a_{2,2}^{(1)}} \\ \vdots \\ -\frac{a_{n,2}^{(1)}}{a_{2,2}^{(1)}}
	\end{bmatrix}
\]
Una volta ottenuto $u$ possiamo calcolare $A_2$ in questo modo
\[ A_2 = E_2 \cdot A_1 = E_2 \cdot E_1 \cdot A_0 \]
ottenendo una matrice con la seguente forma
\[
	A_2 = \begin{bmatrix}
		a_{1,1}^{(2)} & \dots         & \dots         & \dots & a_{1,n}^{(2)} \\
		0             & a_{2,2}^{(2)} & \dots         & \dots & a_{2,n}^{(2)} \\
		\vdots        & 0             & a_{3,3}^{(2)} & \dots & a_{3,n}^{(2)} \\
		\vdots        & \vdots        & \vdots        &       & \vdots        \\
		0             & 0             & a_{n,3}^{(2)} & \dots & a_{n,n}^{(2)}
	\end{bmatrix}
\]
A questo punto se $a_{3,3}^{(2)} \neq 0$ possiamo andare avanti fino a $a_{n,n}^{(n-1)}$ definendo così tutte
le matrici elementari di Gauss $E_i$ con $1 \leq i \leq n-1$ e ottenendo la matrice triangolare superiore $U$
in questo modo
\[ E_{n-1} \dots E_1 A_0 = U \]
Non ci rimane che calcolare $L$. Per farlo scriviamo
\[ A = A_0 = E_1^{-1} \dots E_{n-1}^{-1} U \]
da cui possiamo dedurre che
\[ L = E_1^{-1} \dots E_{n-1}^{-1} \]
Valutiamo ora il costo computazionale per il calcolo di una fattorizzazione $LU$ utilizzando il metodo appena
descritto. In generale, il passaggio da $A_i$ ad $A_j$ ha costo $O((n-j)^2)$ operazioni. Il costo complessivo
della computazione è quindi $O(n^3)$.

\begin{example}
	Calcoliamo la fattorizzazione $LU$ della matrice $A$ definita come segue
	\[ A = \begin{bmatrix}
			1 & 1 & 1 \\
			1 & 2 & 2 \\
			1 & 2 & 3
		\end{bmatrix}
	\]
	Prima di tutto vogliamo calcolarci $E_1$ tale che
	\[ E_1 \begin{bmatrix}
			1 \\ 1 \\ 1
		\end{bmatrix} = \begin{bmatrix}
			1 \\ 0 \\ 0
		\end{bmatrix}
	\]
	Per il teorema \ref{th: eliminazione_gauss} possiamo prendere $E_1$ tale che
	\[ E_1 = I + u e_1^T \]
	con
	\[ u = \begin{bmatrix} 0 \\ -1 \\ -1 \end{bmatrix} \]
	Ricaviamo quindi
	\[
		E_1 = \begin{bmatrix}
			1  & 0 & 0 \\
			-1 & 1 & 0 \\
			-1 & 0 & 1
		\end{bmatrix}
	\]
	Possiamo ora calcolare $A_1$
	\[
		A_1 = E_1 A = \begin{bmatrix}
			1 & 1 & 1 \\
			0 & 1 & 1 \\
			0 & 1 & 2
		\end{bmatrix}
	\]
	Dato che $a_{2,2}^{(1)} \neq 0$ allora procediamo con il calcolo di $E_2$ tale che
	\[
		E_2 = I + u e_2^T = \begin{bmatrix}
			1 & 0  & 0 \\
			0 & 1  & 0 \\
			0 & -1 & 1 \\
		\end{bmatrix}
	\]
	A questo punto possiamo calcolare $A_2$
	\[
		A_2 = E_2 A_1 = \begin{bmatrix}
			1 & 1 & 1 \\
			0 & 1 & 1 \\
			0 & 0 & 1
		\end{bmatrix} = U
	\]
	Non ci rimane che trovare $L$ in questo modo
	\[
		E_1^{-1} E_2^{-1} = \begin{bmatrix}
			1 & 0 & 0 \\
			1 & 1 & 0 \\
			1 & 0 & 1
		\end{bmatrix} \begin{bmatrix}
			1 & 0 & 0 \\
			0 & 1 & 0 \\
			0 & 1 & 1
		\end{bmatrix} = \begin{bmatrix}
			1 & 0 & 0 \\
			1 & 1 & 0 \\
			1 & 1 & 1
		\end{bmatrix} = L
	\]
	Facciamo il prodotto $L \cdot U$ per verificare che la fattorizzazione sia corretta.
\end{example}

\subsubsection{Eliminazione gaussiana con pivoting parziale}
Supponiamo di avere ora una matrice $A$ definta come segue
\[
	A = \begin{bmatrix}
		1 & 1 & 1 \\
		1 & 1 & 2 \\
		1 & 2 & 3
	\end{bmatrix}
\]
In questo caso se provassimo ad applicare l'algoritmo appena visto otteniamo dopo il primo passaggio
\[
	A_1 = \begin{bmatrix}
		1 & 1 & 1 \\
		0 & 0 & 1 \\
		0 & 1 & 2
	\end{bmatrix}
\]
La quale non ci permette di compiere un ulteriore passo. Ma come possiamo notare, scambiando le ultime due righe
si giunge ad una forma triangolare. Questa \textbf{non} è una fattorizzazione $LU$ di $A$, vale infatti
\[ P_2 E_1 A = U \]
dove $P_2$ è detta \textbf{matrice di permutazione}, la quale scambia le ultime due righe ed è definita come
segue
\[
	P_2 = \begin{bmatrix}
		1 & 0 & 0 \\
		0 & 0 & 1 \\
		0 & 1 & 0
	\end{bmatrix}
\]
A questo punto per ricaviamo $A$ in questo modo
\[ A = E_1^{-1} P_2^{-1} U \]
L'inversa di una matrice di permutazione coincide con la trasposta.
\[ P^{-1} = P^T \]

Supponiamo ora che una generica matrice
\[
	\begin{bmatrix}

	\end{bmatrix}
\]
sia invertibile.

\end{document}
