\section{Convergenza}
Cerchiamo ora di capire come scegliere $M$ ed $N$, come capire quando un metodo è convergente e che cosa
significa dire che un metodo è convergente. Prendiamo ad esempio la matrice quadrata $A$ definita come segue
\[
	A = \begin{bmatrix}
		0 & 1 \\
		1 & 0
	\end{bmatrix}
\]
In questo caso non possiamo usare il metodo di Jacobi e quindi nemmeno quello di Gauss-Seidel dato che la
matrice $A$ ha tutti 0 sulla diagonale principale. Ma la condizione necessaria affinché la successione
converga è che $M$ sia invertibile. Possiamo quindi definire $M$ ed $N$ come segue
\[
	M = \begin{bmatrix}
		0 & 1 \\
		1 & 0
	\end{bmatrix} \quad
	N = \begin{bmatrix}
		0 & 0 \\
		0 & 0
	\end{bmatrix}
\]
In questo caso la matrice $P$ è la matrice nulla
\[
	P = M^{-1} N = \begin{bmatrix}
		0 & 0 \\
		0 & 0
	\end{bmatrix}
\]
Abbiamo quindi che il vettore $x$ al tempo $k+1$ si ricava calcolando
\[ x^{(k+1)} P x^{(k)} + q = P x^{(k)} + M^{-1} b \]
Questa successione, partendo da un vettore $x^{(0)} \in \R^n$, è costante in quanto
\[ x^{(k+1)} M^{-1} b \]
e quindi converge a $M^{-1} b$ per qualsiasi $x^{(0)} \in \R^n$ che è dunque la soluzione del sistema lineare.

Consideriamo ora la stessa matrice $A$ ma scegliamo come $M$ l'identità. Ne segue che
\[
	N = M - A = \begin{bmatrix}
		1  & -1 \\
		-1 & 1
	\end{bmatrix}
\]
Il vettore all'iterazione $k+1$ sarà quindi calcolato come segue
\[
	x^{(k+1)} = \begin{bmatrix}
		1  & -1 \\
		-1 & 1
	\end{bmatrix} x^{(k)} + b
\]
Supponiamo inoltre che $b$ sia il vettore nullo, è facile dedurre che l'unica soluzione del sistema lineare è
anch'essa un vettore nullo.

Gli autovalori di $N$ sono $\lambda_1 = 0$ e $\lambda_2 = 2$. Gli autovettori relativi a tali autovalori sono
\[ \begin{bmatrix} 1 \\ 1 \end{bmatrix}_{\lambda_1} \quad \begin{bmatrix} 1 \\ -1 \end{bmatrix}_{\lambda_2} \]
Se ora scegliamo $x^{(0)}$ come l'autovettore relativo a $\lambda_1$ ricaviamo
\[
	x^{(1)} = \begin{bmatrix}
		1  & -1 \\
		-1 & 1
	\end{bmatrix}
	\begin{bmatrix} 1 \\ 1 \end{bmatrix} +
	\begin{bmatrix} 0 \\ 0 \end{bmatrix} =
	\begin{bmatrix} 0 \\ 0 \end{bmatrix}
\]
e tutti i vettori generati ai passi successivi sono vettori nulli. Ecco che la successione converge al vettore
nullo, ossia la soluzione che stavamo cercando.

Se invece scegliamo come vettore di partenza l'autovettore relativo a $\lambda_2$ allora otteniamo
\[
	x^{(1)} = \begin{bmatrix}
		1  & -1 \\
		-1 & 1
	\end{bmatrix}
	\begin{bmatrix} 1 \\ -1 \end{bmatrix} +
	\begin{bmatrix} 0 \\ 0 \end{bmatrix} =
	\begin{bmatrix} 2 \\ -2 \end{bmatrix}
\]
Andando avanti si nota che la successione diverge dal vettore nullo. Questo ci dice che la successione può
convergere a divergere a seconda del vettore iniziale.

\begin{definition}
	Dato che, in generale, non si hanno informazione utili per scegliere il vettore iniziale, diciamo che il
	metodo iterativo è \textbf{convergente} se le successione generate convergono per ogni scelta del vettore
	iniziale.
\end{definition}

\begin{theorem}
	Se esiste una norma matriciale indotta da una norma vettoriale $\norm{\cdot}$ tale che
	\[ \norm{P} < 1 \]
	allora il metodo è convergente.
	\begin{proof}
		Come sappiamo
		\[ x^{(k+1)} = P x^{(k)} + q \]
		e la soluzione del sistema lineare soddisfa
		\[ x = P x + q \]
		Se adesso sottraiamo la prima equazione alla seconda otteniamo
		\[ x^{(k+1)} - x = P x^{(k)} - P x = P (x^{(k)} - x) \]
		Se noi chiamiamo $e^{(k)}$ la differenza tra l'approssimazione alla $k$-esima iterazione e il vettore
		soluzione
		\[ e^{(k)} = x^{(k)} - x \]
		otteniamo la relazione
		\[ e^{(k+1)} = P e^{(k)} \]
		per ogni $k \geq 0$. Questo ci dice inoltre che
		\[ e^{(k+1)} = P e^{(k)} = P \cdot P e^{(k-1)} = \dots = P^{k+1} e^{(0)} \]
		Passiamo ora alle norme vettoriali e otteniamo
		\[
			\norm{e^{(k+1)}} = \norm{P^{k+1} e^{(0)}} \leq
			\norm{P^{k+1}} \norm{e^{(0)}} \leq \norm{P^{k+1}} \norm{e^{(0)}}
		\]
		per la quarta proprietà delle norme. Quindi per ogni norma vettoriale, considerando la sua norma
		matriciale indotta vale
		\[ 0 \leq \norm{e^{(k+1)}} \leq \norm{P}^{k+1} \cdot \norm{e^{(0)}} \]
		Se considero una norma tale che $\norm{P} < 1$ allora
		\[ \lim_{k \to +\infty} \norm{P}^{k+1} \norm{e^{(0)}} = 0 \]
		E quindi, per il teorema del confronto
		\[ \lim_{k \to +\infty} \norm{e^{(k+1)}} = 0 \]
		qualunque sia $e^{(0)}$.
	\end{proof}
\end{theorem}

\begin{example}
	Sia $A$ la matrice definita come segue
	\[
		A = \begin{bmatrix}
			3  & -1     &        &    \\
			-1 & \ddots & \ddots &    \\
			   & \ddots & \ddots & -1 \\
			   &        & -1     & 3
		\end{bmatrix}
	\]
	Il metodo di Jacobi, ad ogni passo costa $O(\nnz(A))$ ma per una matrice triadiagonale $\nnz(A) = n$ quindi
	il costo complessivo di ogni passo del metodo è $O(n)$.

	Vediamo ora se il metodo converge per tale matrice. Per il metodo la matrice $M$ equivale a $3I$, quindi
	\[
		N = M - A = \begin{bmatrix}
			0 & 1      &        &   \\
			1 & \ddots & \ddots &   \\
			  & \ddots & \ddots & 1 \\
			  &        & 1      & 0
		\end{bmatrix}
	\]
	e la matrice $P$ si ottiene calcolando
	\[
		P = M^{-1} N = \begin{bmatrix}
			0           & \frac{1}{3} &             &             \\
			\frac{1}{3} & \ddots      & \ddots      &             \\
			            & \ddots      & \ddots      & \frac{1}{3} \\
			            &             & \frac{1}{3} & 0
		\end{bmatrix}
	\]
	A questo punto proviamo a calcolare la norma infinto di $P$ (la norma 1 è uguale)
	\[ \norm{P}_\infty = \norm{P}_1 = \frac{2}{3} < 1 \]
	Quindi il metodo converge.
\end{example}

\begin{theorem}\label{th: convergenza_spettrale}
	Sia $P$ la matrice di iterazione della matrice $A$ associata al sistema lineare $Ax = b$. Allora il metodo
	iterativo è convergente se e solo se
	\[ \rho (P) < 1 \]
\end{theorem}

\begin{example}
	Sia $A$ la matrice definita come segue
	\[
		A = \begin{bmatrix}
			1  &        &        & -\alpha \\
			-1 & \ddots &        &         \\
			   & \ddots & \ddots &         \\
			   &        & -1     & 1
		\end{bmatrix}
	\]
	Per tale matrice otteniamo una matrice di iterazione
	\[
		P = \begin{bmatrix}
			  &        &   & \alpha \\
			1 &        &   &        \\
			  & \ddots &   &        \\
			  &        & 1 &
		\end{bmatrix}
	\]
	Se proviamo a calcolare la norma vediamo che essa è sempre maggiore di 1 e quindi non sappiamo nulla sulla
	convergenza o meno del metodo. Proviamo quindi ad utilizzare il teorema \ref{th: convergenza_spettrale}
	calcolando $\rho (P)$.

	Per calcolare il raggio spettrale di $P$ dobbiamo calcolare gli autovalori e quindi il polinomio
	caratteristico, che risulta uguale a
	\[ \det(\lambda I - P) = \lambda^n - \alpha \]
	Dato che a noi interessa il modulo degli autovalori possiamo dire che
	\[ |\lambda| = \sqrt[n]{|\alpha|} \]
	e dunque il metodo converge quando
	\[
		\sqrt[n]{|\alpha|} < 1 \quad
		\Leftrightarrow \quad |\alpha| < 1 \quad
		\Leftrightarrow \quad -1 < \alpha < 1
	\]
	Proviamo ora a vedere qual'è la condizione di convergenza per il metodo di Gauss-Seidel. Abbiamo le matrici
	$M$ ed $N$ così definite
	\[
		M = \begin{bmatrix}
			1  &        &            \\
			-1 & \ddots &            \\
			   & \ddots & \ddots     \\
			   &        & -1     & 1
		\end{bmatrix} \quad
		N = \begin{bmatrix}
			 &  & \alpha \\
			 &  &
		\end{bmatrix}
	\]
	La matrice di iterazione $P$ sarà una matrice tutta nulla tranne che per l'ultima colonna, calcolabile
	tramite la risoluzione del sistema lineare
	\[ A x = N_n \]
	dove $N_n$ è l'ultima colonna di $N$. Ricaviamo quindi
	\[
		P = \begin{bmatrix}
			 &  &  & \alpha \\
			 &  &  & \vdots \\
			 &  &  & \alpha
		\end{bmatrix}
	\]
	Gli autovalori di questa matrice sono 0 e $\alpha$ e quindi
	\[ \rho(P) = |\alpha| \]
	La condizione di convergenza è quindi la stessa del metodo di Jacobi in questo caso
	\[ -1 < \alpha < 1 \]
	Osserviamo però che i raggi spettrali delle due matrici di iterazione per i due metodi sono differenti
	\[ \rho(G) = |\alpha| \quad \rho(J) = \sqrt[n]{|\alpha|} \]
	Supponendo di essere in condizioni di convergenza ($|\alpha| < 1$) abbiamo che
	\[ \rho(G) \ll \rho(J) \]
	Questo si traduce in una convergenza più rapida per il metodo di Gauss-Seidel (in questo caso).
\end{example}

\begin{example}
	Consideriamo ora la matrice
	\[
		A = \begin{bmatrix}
			n      & -1     & \dots  & -1     \\
			-1     & \ddots & \ddots & \vdots \\
			\vdots & \ddots & \ddots & -1     \\
			-1     & \dots  & -1     & n
		\end{bmatrix}
	\]
	Proviamo a capire se il metodo di Jacobi applicato a questa matrice è convergente. Ricaviamo le matrici $M$
	ed $N$
	\[
		M = \begin{bmatrix}
			n &        &   \\
			  & \ddots &   \\
			  &        & n
		\end{bmatrix} \quad
		N = M - A = \begin{bmatrix}
			       & 1      & \dots  & 1      \\
			1      &        & \ddots & \vdots \\
			\vdots & \ddots &        & 1      \\
			1      & \dots  & 1      &
		\end{bmatrix}
	\]
	La matrice di iterazione è quindi
	\[
		P = M^{-1} N = \begin{bmatrix}
			            & \frac{1}{n} & \dots       & \frac{1}{n} \\
			\frac{1}{n} &             & \ddots      & \vdots      \\
			\vdots      & \ddots      &             & \frac{1}{n} \\
			\frac{1}{n} & \dots       & \frac{1}{n} &
		\end{bmatrix}
	\]
	Non ci rimane che calcolare la norma di questa matrice per vedere se il metodo converge
	\[ \norm{P}_\infty = \norm{P}_1 = \frac{n-1}{n} < 1 \]
	dunque il metodo converge.

	Vogliamo ora determinare un numero $k$ di iterazioni tale che l'errore al passo $k$-esimo fratto l'errore
	al passo iniziale per il metodo di Jacobi sia minore o uguale di $10^{-12}$, ossia $k$ tale che
	\[ \frac{\norm{e^{(k)}}_\infty}{\norm{e^{(0)}}_\infty} \leq 10^{-12} \]
	Ricordiamoci che l'errore al passo $k$-esimo equivale a
	\[ e^{(k)} = P^k e^{(0)} \]
	Quindi possiamo dire che
	\begin{align*}
		\norm{e^{(k)}}_\infty                               & \leq \norm{P}_\infty^k \cdot \norm{e^{(0)}}_\infty \\
		                                                    & \Downarrow                                         \\
		\frac{\norm{e^{(k)}}_\infty}{\norm{e^{(0)}}_\infty} & \leq \norm{P}_\infty^k
	\end{align*}
	Nel metodo di Jacobi la norma infinito della matrice di iterazione vale
	\begin{align*}
		\norm{P}                                            & = \frac{n-1}{n}                     \\
		                                                    & \Downarrow                          \\
		\norm{P}_\infty^k                                   & = \left( \frac{n-1}{n} \right)^k    \\
		                                                    & \Downarrow                          \\
		\frac{\norm{e^{(k)}}_\infty}{\norm{e^{(0)}}_\infty} & \leq \left( \frac{n-1}{n} \right)^k
	\end{align*}
	Non ci rimane che determinare $k$ tale che
	\[ \left( \frac{n-1}{n} \right)^k \leq 10^{-12} \]
	Per farlo proseguiamo in questo modo
	\begin{align*}
		\left( \frac{n-1}{n} \right)^k                          & \leq 10^{-12}             \\
		\log_{10} \left( \left( \frac{n-1}{n} \right)^k \right) & \leq \log_{10} (10^{-12}) \\
		k \cdot \log_{10} \left( \frac{n-1}{n} \right)          & \leq -12                  \\
		k                                                       & \geq \frac{-12}{\log_{10}
			\left( \frac{n-1}{n} \right)}
	\end{align*}
	Ecco che ci basta prendere come $k$
	\[ k = \left\lceil \frac{-12}{\left( \frac{n-1}{n} \right)} \right\rceil \]
	ossia la parte intera superiore del membro a destra della disequazione.
\end{example}

\subsection{Convergenza per matrici predominanti diagonali}
Vogliamo ora dimostrare che se una matrice $A$ è predominante diagonale allora i metodi di Jacobi e di
Gauss-Seidel sono convergenti. Questo ci permette di evitare qualsiasi valutazione sulla matrice di iterazione.

Se una matrice è predominante diagonale allora gli elementi diagonali hanno modulo strettamente maggiore di 0
e quindi diverso da 0. I metodi di Jacobi e di Gauss-Seidel sono dunque applicabili.

\begin{theorem}
	Se una matrice $A$ è predominante diagonale allora i metodi di Jacobi e Gauss-Seidel sono convergenti.
	\begin{proof}
		Ricordiamo che la condizione necessaria e sufficiente per la convergenza di un metodo è
		\[ \rho(P) < 1 \]
		Analizziamo quindi gli autovalori della matrice di iterazione $P = M^{-1} N$. Per farlo dobbiamo
		calcolare il polinomio caratteristico, ossia
		\begin{align*}
			p(\lambda) = & \det(M^{-1} N - \lambda I)             \\
			=            & \det(M^{-1} N - \lambda M^{-1} M)      \\
			=            & \det(M^{-1} (N - \lambda M))           \\
			=            & \det(M^{-1}) \cdot \det(N - \lambda M)
		\end{align*}
		Ricordiamo che $\lambda$ è un autovalore se e solo se la matrice $\lambda M - N$ è singolare. L'idea
		per dimostrare il teorema sta nel dimostrare che se abbiamo un $|\lambda| \geq 1$ allora $\lambda M - N$
		è predominante diagonale, dunque invertibile. Ecco che quindi gli autovalori devono avere tutti modulo
		minore di 1.

		Per il metodo di Jacobi la matrice $M$ è la diagonale principale di $A$
		\[
			M = \begin{bmatrix}
				a_{1,1} &        &         \\
				        & \ddots &         \\
				        &        & a_{n,n}
			\end{bmatrix}
		\]
		mentre $N = M - A$
		\[
			N = \begin{bmatrix}
				         & -a_{1,2} & \dots      & -a_{1,n}   \\
				-a_{2,1} &          & \ddots     & \vdots     \\
				\vdots   & \ddots   &            & -a_{n-1,n} \\
				-a_{n,1} & \dots    & -a_{n,n-1} &
			\end{bmatrix}
		\]
		Ne segue che $\lambda M - N$ è definita in questo modo
		\[
			\lambda M - N = \begin{bmatrix}
				\lambda a_{1,1} & a_{1,2} & \dots     & a_{1,n}         \\
				a_{2,1}         & \ddots  & \ddots    & \vdots          \\
				\vdots          & \ddots  & \ddots    & a_{n-1,n}       \\
				a_{n,1}         & \dots   & a_{n,n-1} & \lambda a_{n,n}
			\end{bmatrix}
		\]
		Dobbiamo quindi dimostrare che se $|\lambda| \geq 1$ allora la matrice è predominante diagonale. Che
		equivale a dire
		\[ |\lambda a_{i,i}| > \sum_{j=1, j \neq i}^{n} |a_{i,j}| \quad i=1, \dots, n \]
		Osserviamo che
		\[ |\lambda a_{i,i}| = |\lambda| \cdot |a_{i,i}| \]
		Ma se $|\lambda| \geq 1$ allora
		\[ |\lambda| \cdot |a_{i,i}| \geq |a_{i,i}| \]
		Per la predominanza diagonale di $A$ abbiamo che
		\[ |a_{i,i}| > \sum_{j=1, j \neq i}^n |a_{i,j}| \quad i=1, \dots, n \]
		Quindi, se $A$ è predominante diagonale, anche $\lambda M - N$ lo è per $|\lambda| \geq 1$ quindi tutti
		gli autovalori devono avere modulo inferiore a 1.

		La matrice $M$ per Gauss-Seidel è la parte triangolare inferiore di $A$ ovvero
		\[
			M = \begin{bmatrix}
				a_{1,1} &        &         \\
				\vdots  & \ddots &         \\
				a_{n,1} & \dots  & a_{n,n}
			\end{bmatrix}
		\]
		La matrice $N = M - A$ è quindi definita in questo modo
		\[
			N = \begin{bmatrix}
				 & -a_{1,2} & \dots  & -a_{1,n}    \\
				 &          & \ddots & \vdots      \\
				 &          &        & -a_{n-1, n} \\
				 &          &        &
			\end{bmatrix}
		\]
		Come prima ricaviamo la matrice $\lambda M - N$
		\[
			\lambda M - N = \begin{bmatrix}
				\lambda	a_{1,1}  & a_{1,2} & \dots  & a_{1,n}          \\
				\vdots          & \ddots  & \ddots & \vdots           \\
				\vdots          &         & \ddots & a_{n-1,n}        \\
				\lambda a_{n,1} & \dots   & \dots  & \lambda a_{n, n}
			\end{bmatrix}
		\]
		Come prima vogliamo dimostrare la predominanza diagonale di tale matrice per $|\lambda| \geq 1$, per
		farlo scriviamo
		\[ |\lambda a_{i,i}| > \sum_{j=1}^{i-1} |\lambda a_{i,j}| + \sum_{j=i+1}^n |a_{i,j}| \]
		Dato che la matrice $A$ è predominante diagonale allora
		\[ |a_{i,i}| > \sum_{j=1}^{i-1} |a_{i,j}| + \sum_{j=i+1}^n |a_{i,j}| \]
		Se moltiplichiamo ambo i membri per $|\lambda|$ otteniamo
		\begin{align*}
			|\lambda| |a_{i,i}| > & |\lambda| \sum_{j=1}^{i-1} |a_{i,j}| + |\lambda| \sum_{j=i+1}^n |a_{i,j}| \\
			|\lambda a_{i,i}| >   & \sum_{j=1}^{i-1} |\lambda a_{i,j}| + \sum_{j=i+1}^n |\lambda a_{i,j}|     \\
			\geq                  & \sum_{j=1}^{i-1} |\lambda a_{i,j}| + \sum_{j=i+1}^n |a_{i,j}|
		\end{align*}
		Concludiamo quindi che entrambi i metodi convergono se la matrice $A$ è predominante diagonale.
	\end{proof}
\end{theorem}

\begin{example}
	Consideriamo la matrice $A$ definita in questo modo
	\[
		A = \begin{bmatrix}
			1      & x_2    & \dots  & x_n \\
			x_2    & \ddots &        &     \\
			\vdots &        & \ddots &     \\
			x_n    &        &        & 1
		\end{bmatrix}
	\]
	La matrice è predominante diagonale se
	\[ \sum_{i=2}^n |x_i| < 1 \]
	Vogliamo ora determinare un numero $k$ di iterazioni per il metodo di Jacobi sufficienti a garantire che
	\[ \frac{\norm{e^{(k)}}_\infty}{\norm{e^{(0)}}_\infty} \leq 2^{-52} \]
	Dove $e^{(k)}$ equivale alla quantità
	\[ P^k e^{(0)} = x^{(k)} - x \]
	Vale quindi
	\begin{align*}
		\norm{e^{(k)}}  \leq                       & \norm{P^k} \cdot \norm{e^{(0)}} \\
		\leq                                       & \norm{P}^k \cdot \norm{e^{(0)}} \\
		\Updownarrow                               &                                 \\
		\frac{\norm{e^{(k)}}}{\norm{e^{(0)}}} \leq & \norm{P}^k
	\end{align*}
	Ci basta utilizzare la norma infinito per ottenere
	\[ \frac{\norm{e^{(k)}}_\infty}{\norm{e^{(0)}}_\infty} \leq \norm{P}_\infty^k \]
	Per la nostra matrice $A$, la matrice $P$ del metodo di Jacobi è la seguente
	\[
		P = \begin{bmatrix}
			       & x_2 & \dots & x_n \\
			x_2    &     &       &     \\
			\vdots &     &       &     \\
			x_n    &     &       &
		\end{bmatrix}
	\]
	La norma infinto è quindi molto semplice da calcolare ed equivale a
	\[ \norm{P}_\infty = \sum_{i=2}^{n} |x_i| < 1 \]
	Se ora riuscissimo a trovare un $k$ tale che $\norm{P}^k \leq 2^{-52}$, avremmo anche trovato un $k$ che
	soddisfa la richiesta. Proseguiamo quindi in questo modo
	\begin{align*}
		\norm{P}_\infty^k \leq                & 2^{-52}                              \\
		k \cdot \log_2 (\norm{P}_\infty) \leq & -52                                  \\
		k \geq                                & -\frac{52}{\log_2 (\norm{P}_\infty)}
	\end{align*}
	Possiamo quindi prendere come numero di iterazioni necessario a soddisfare la richiesta
	\[ k = \left\lceil -\frac{52}{\log_2 (\norm{P}_\infty)} \right\rceil \]
	Vogliamo ora determinare per quali valori $x_2, \dots, x_n$ il metodo di Jacobi converge ($\rho (P) < 1$).
	Osserviamo che la matrice ha rango 2 dunque abbiamo 2 autovalori diversi da 0. Possiamo provare a trovare un
	autovettore imponendo la seguente relazione
	\[
		P \begin{bmatrix} v_1 \\ \vdots \\ v_n \end{bmatrix} =
		\lambda \begin{bmatrix} v_1 \\ \vdots \\ v_n \end{bmatrix}
	\]
	Ricaviamo quindi un sistema in questa forma
	\[
		\begin{cases}
			- (x_2 v_2 + \dots + x_n v_n) = & \lambda v_1 \\
			-x_2 v_1 =                      & \lambda v_2 \\
			\vdots                          & \vdots      \\
			-x_n v_1 =                      & \lambda v_n
		\end{cases}
	\]
	Se dividiamo per $\lambda$ tutte le equazioni tranne la prima otteniamo
	\[
		\begin{cases}
			- (x_2 v_2 + \dots + x_n v_n) = & \lambda v_1              \\
			v_2 =                           & -\frac{x_2 v_1}{\lambda} \\
			\vdots                          & \vdots                   \\
			v_n =                           & -\frac{x_n v_1}{\lambda}
		\end{cases}
	\]
	Se ora sostituiamo $v_2, \dots, v_n$ nella prima equazione otteniamo
	\begin{align*}
		\frac{x_2^2 v_1}{\lambda} + \dots + \frac{x_n^2 v_1}{\lambda} = & \lambda v_1                      \\
		x_2^2 v_1 + \dots + x_n^2 v_1 =                                 & \lambda^2 v_1                    \\
		\lambda^2 =                                                     & x_2^2 + \dots + x_n^2            \\
		\lambda =                                                       & \pm \sqrt{x_2^2 + \dots + x_n^2}
	\end{align*}
	Possiamo quindi dire che
	\[ \rho(P) = \sqrt{x_2^2 + \dots + x_n^2} \]
	Il metodo di Jacobi quindi converge se e solo se
	\[ \norm{v}_2 < 1 \]
\end{example}