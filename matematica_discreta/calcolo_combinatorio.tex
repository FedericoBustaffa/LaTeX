\chapter{Calcolo combinatorio}
\section{Principio fondamentale del calcolo combinatorio}
Introduciamo questo capitolo sul calcolo combinatorio parlando del
\textbf{principio fondamentale del calcolo combinatorio}.

\begin{theorem}[Principio fondamentale del calcolo combinatorio]
	Se una scelta \`e univocamente determinata da una sequenza di $n$ passi successivi, con
	$k_1$ possibilit\`a per il primo passo, $k_2$ possibilit\`a per il secondo, ..., $k_n$
	possibilit\`a per l'$n$-esimo passo, il numero di scelte possibili \`e il prodotto:
	\begin{equation*}
		k_1 \cdot k_2 \cdot ... \cdot k_n
	\end{equation*}
\end{theorem}

Ora per\`o dobbiamo fare distinzione sui possibili modi in cui scelgo $k_i$ in base al tipo di
problema che sto trattando. Sostanzialmente devo fare due scelte nella scelta del passo $k_i$.
Devo decidere se
\begin{itemize}
	\item Conta l'ordine
	\item Sono ammesse ripetizioni
\end{itemize}
Le due scelte sono totalemente indipendenti l'una dall'altra e daranno vita a tipologie di problemi
diverse che tratteremo nei prossimi capitoli.

\section{Combinazioni semplici}
\begin{defn}
	Se non conta l'ordine e non ammetto ripetizioni si tratta di una
	\textbf{combinazione semplice}.
\end{defn}

In pratica abbiamo un insieme di partenza di $n$ elementi da cui vogliamo sceglierne $k$ senza
per\`o avere ripetizioni.
Avremmo dunque che le possibili combinazioni sono date dal coefficiente binomiale:
\begin{equation*}
	C_{n, k} =
	\begin{pmatrix}
		n \\ k
	\end{pmatrix} =
	\frac{n!}{k! (n - k)!}
\end{equation*}

\begin{example}
	In quanti modi posso scegliere 3 persone da un gruppo di 90 persone ?

	Le possibili combinazioni sono:
	\begin{equation*}
		C_{90, 3} =
		\begin{pmatrix}
			90 \\ 3
		\end{pmatrix} =
		\frac{90!}{3! \cdot 87!} = 117480
	\end{equation*}
\end{example}

\begin{theorem}
	Sia $X$ un insieme e sia
	\begin{equation*}
		\mathcal{P}(X) = \{ Y \mid Y \subseteq X \}
	\end{equation*}
	l'insieme di tutti i sottoinsiemi di $X$ allora se
	\begin{equation*}
		\begin{array}{ccc}
			|X| = n & \Rightarrow & |\mathcal{P}(X)| = 2^n
		\end{array}
	\end{equation*}
\end{theorem}

\section{Combinazioni con ripetizione}
\begin{defn}
	Se non conta l'ordine e ammetto ripetizioni si tratta di una
	\textbf{combinazione con ripetizione}.
\end{defn}

Qui abbiamo sempre un insieme di $n$ elementi dal quale vogliamo sceglierne $k$ ma in questo caso
ammetto ripetizioni quindi posso prendere lo stesso elemento pi\`u volte.
In questo caso abbiao che le possibili combinazioni sono date dal coefficiente binomiale:
\begin{equation*}
	C_{n, k}^r =
	\begin{pmatrix}
		n + k - 1 \\ k
	\end{pmatrix} =
	\frac{(n + k - 1)!}{k! (n - 1)!}
\end{equation*}

\begin{example}
	In quanti modi posso estrarre 4 biglie numerate con reimmissione da un'urna con con
	10 biglie ?

	Le possibili combinazioni sono:
	\begin{equation*}
		C_{10, 4}^r =
		\begin{pmatrix}
			10 + 4 - 1 \\ 4
		\end{pmatrix} =
		\frac{13!}{4! \cdot 9!} = 715
	\end{equation*}
\end{example}

\section{Disposizioni semplici}
\begin{defn}
	Se conta l'ordine e non ammetto ripetizioni si tratta di una \textbf{disposizione semplice}.
\end{defn}

In questo caso vogliamo estrarre $k$ elementi da un insieme di $n$ elementi tenendo conto
dell'ordine e senza ammettere ripetizioni.
Il numero di disposizioni possibili \`e dato dalla formula:
\begin{equation*}
	D_{n, k} = \frac{n!}{(n - k)!}
\end{equation*}

\begin{example}
	In quanti modi \`e possibile formare il podio alla fine di un campionato da 20 squadre ?

	Il numero di possibili disposizioni \`e dunque
	\begin{equation*}
		D_{20, 3} = \frac{20!}{17!} = 6840
	\end{equation*}
\end{example}

\section{Disposizioni con ripetizione}
\begin{defn}
	Se conta l'ordine e ammetto ripetizioni si tratta di una \textbf{disposizione con ripetizione}
\end{defn}

Anche qui vogliamo estrarre $k$ elementi da un insieme di $n$ elementi tenendo conto dell'ordine
per\`o stavolta ammettiamo ripetizioni.
Il numero di disposizioni possibili \`e dato dalla formula:
\begin{equation*}
	D_{n, k}^r = n^k
\end{equation*}

\begin{example}
	Quante password numeriche di 6 cifre esistono ?

	Il numero di possibili disposizioni \`e:
	\begin{equation*}
		D_{10, 6} = 10^6 = 1000000
	\end{equation*}
\end{example}

Esiste anche un metodo per visualizzare meglio il problema delle disposizioni tramite gli insiemi.
\begin{theorem}
	Siano $A$ e $B$ due insiemi finiti e sia $A \times B$ il loro prodotto cartesiano definito
	come segue
	\begin{equation*}
		A \times B = \{ (a, b) | a \in A, b \in B \}
	\end{equation*}
	allora
	\begin{equation*}
		|A \times B| = |A| \cdot |B|
	\end{equation*}
\end{theorem}

\begin{example}
	Se considero il lancio di due dadi a 6 facce, avrei 36 possibili coppie. Questo perch\'e
	dobbiamo immaginarci i possibili valori delle facce di un dado contenuti in un insieme
	\begin{equation*}
		F = \{ 1, 2, 3, 4, 5, 6 \}
	\end{equation*}
	Dato che i dadi lanciati sono due, le possibili coppie di valori sono elementi dell'insieme
	\begin{equation*}
		F \times F = \{ (1, 1), (1, 2), \dots (6, 6) \}
	\end{equation*}
	e come abbiamo visto prima la cardinalit\`a di un prodotto cartesiano \`e data dal prodotto
	delle cardinalit\`a degli insiemi che lo compongono, quindi:
	\begin{equation*}
		\begin{array}{lll}
			|F|          & = 6                \\
			             & \Downarrow         \\
			|F \times F| & = 6 \cdot 6 & = 36
		\end{array}
	\end{equation*}

	Lo stesso problema si risolve con la formula delle disposizioni con ripetizione. Abbiamo
	$n = 6$ dato che i valori possibili sono 6 e $k = 2$ dato che i dadi sono 2. Dunque abbiamo
	\begin{equation*}
		D_{6, 2}^r = 6^2 = 36
	\end{equation*}
	possibili coppie.
\end{example}

\begin{observation}
	Possiamo usare la formula insiemistica poich\'e la cardinalit\`a di un prodotto cartesiano
	tiene conto dell'ordine in cui sono gli elementi della tupla estratta e perch\'e conta
	anche le volte in cui ho ripetizioni.

	In sintesi:
	\begin{itemize}
		\item $(a, b) \neq (b, a)$ quindi conto due elementi.
		\item posso estrarre pi\`u volte lo stesso elemento quindi \`e contata anche la
		      coppia $(a, a)$.
	\end{itemize}

\end{observation}


\section{Permutazioni semplici}
\begin{defn}
	Se conta l'ordine, non ammetto ripetizioni e $n = k$ si tratta di una
	\textbf{permutazione semplice}.
\end{defn}

Dato che $n = k$ si tratta di ridistribuire tutti gli $n$ elementi dell'insieme di partenza per
ottenere nuove configurazioni. In questo caso per\`o non ammetteremo ripetizioni, ovvero non
potremo utilizzare pi\`u di una volta un elemento dell'insieme di partenza.
Il numero di permutazioni possibili \`e dato dalla formula:
\begin{equation*}
	P_n = n!
\end{equation*}

\begin{example}
	Quanti sono gli anagrammi della parola "gelato" ?

	Il numero di possibili configurazioni \`e:
	\begin{equation*}
		P_{6} = 6! = 720
	\end{equation*}
\end{example}

\section{Permutazioni con ripetizione}
\begin{defn}
	Se conta l'ordine, ammetto ripetizioni e $n = k$ si tratta di una
	\textbf{permutazione con ripetizione}.
\end{defn}

In questo caso ho a che fare con un insieme in cui lo stesso elemento, se ripetuto incrementa
la cardinalit\`a $n$ dell'insieme di partenza. Bisogna tenere di conto anche quante volte
questo elemento viene ripetuto.
Il numero di possibili configurazioni \`e dato dalla formula:
\begin{equation*}
	P_n^r = \frac{n!}{r_1! \cdot r_2! \cdot ... \cdot r_s!}
\end{equation*}
dove $r_i$ \`e il numero di volte che l'elemento \`e stato ripetuto.

\begin{example}
	Quanti anagrammi ha la parola "mamma" ?

	Stavolta abbiamo la lettera "m" che si ripete 3 volte e la lettera "a" che si ripete 2 volte.
	Quindi abbiamo che:
	\begin{equation*}
		\begin{array}{c}
			r_m = 3 \\
			r_a = 2
		\end{array}
	\end{equation*}
	Abbiamo dunque
	\begin{equation*}
		P_5^r = \frac{5!}{3! \cdot 2!} = 10
	\end{equation*}
	anagrammi possibili.
\end{example}

\section{Esempi}

\begin{example}
	Vogliamo contare quante sono le soluzioni intere non negative di
	\begin{equation*}
		x + y + z = 4
	\end{equation*}
	Prendiamo ad esempio la soluzione $(2, 2, 0)$.
	Pensiamo di rappresentarla con una stringa binaria (attenzione: non dobbiamo fare nessuna
	conversione in base 2) rappresentando $x, y, z$ come somma di 1 e mettendo degli 0 come
	separatori. La soluzione sopra sar\`a dunque rappresentata come segue:
	\begin{equation*}
		\begin{array}{cccc}
			\underbrace{11}_{\text{$x$ uni}}   &
			\underbrace{0}_{\text{separatore}} &
			\underbrace{11}_{\text{$y$ uni}}   &
			\underbrace{0}_{\text{separatore}}
		\end{array}
	\end{equation*}
	In questo caso abbiamo $z = 0$ quindi $z$ non pu\`o essere rappresentato poich\'e abbiamo
	0 uni con cui rappresentarlo.
	Il problema \`e diventato estrarre 2 posizioni dalle 6 possibili quindi parliamo di una
	combinazione semplice con $n = 6$ e $k = 2$.
	\begin{equation*}
		\begin{pmatrix}
			6 \\ 2
		\end{pmatrix} =
		\frac{6!}{2! \cdot 4!} = 15
	\end{equation*}
\end{example}

Il problema appena risolto ci da una formula generale per trovare il numero di soluzioni
intere non negative di una qualsiasi equazione del tipo
\begin{equation*}
	x_1 + x_2 + \cdots + x_k = n
\end{equation*}
Per trovare il numero di soluzioni ci baster\`a sviluppare il coefficiente binomiale
\begin{equation*}
	\begin{pmatrix}
		n + k - 1 \\ k - 1
	\end{pmatrix} =
	\frac{(n + k - 1)!}{(k - 1)! \cdot n!}
\end{equation*}