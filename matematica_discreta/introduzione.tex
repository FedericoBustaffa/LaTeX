\section{Introduzione}
\subsection{Insiemi}
Iniziamo col definire gli i seguenti insiemi
\begin{equation*}
	\begin{array}{lll}
		\mathbb{N} & = \{ 0, 1, 2, \dots \}
		           & \text{insieme dei numeri naturali}                              \\
		\mathbb{Z} & = \{ -2, -1, 0, 1, 2 \}
		           & \text{insieme dei numeri interi}                                \\
		\mathbb{Q} & = \{ \frac{a}{b} \mid a, b \in \mathbb{Z} \wedge b \neq 0 \}
		           & \text{insieme dei numeri razionali}                             \\
		\mathbb{R} & = \{ \dots, \sqrt{2}, \pi, \dots \}
		           & \text{Insieme dei numeri reali}                                 \\
		\mathbb{C} & = \{ a + ib \mid a, b \in \mathbb{R}, i = \sqrt{-1}, i^2 = -1\}
		           & \text{insieme dei numeri complessi}
	\end{array}
\end{equation*}

Le operazioni di addizione e moltiplicazione sono permesse ma possono essere definite
in maniera diversa a seconda dell'insieme su cui si sta lavorando.

\subsection{Propriet\`a}
\begin{itemize}
	\item Transitiva per il $<$
	      \begin{equation*}
		      x < y \wedge y < z \Rightarrow x < z
	      \end{equation*}

	\item Totalit\`a per il $<$
	      \begin{equation*}
		      x < y \vee x = y \vee y < x
	      \end{equation*}

	\item Transitiva per $\leq$
	      \begin{equation*}
		      x \leq y \wedge y \leq z \Rightarrow x \leq z
	      \end{equation*}

	\item Totalit\`a per il $\leq$
	      \begin{equation*}
		      x \leq y \vee y \leq x
	      \end{equation*}
\end{itemize}
Ovviamente queste propriet\`a sono analoghe per il $>$ e per il $\geq$.

\begin{observation}
	\begin{equation*}
		\begin{array}{lll}
			x < y    & \Leftrightarrow & x \leq y \wedge x \neq y \\
			x \leq y & \Leftrightarrow & x < y \vee x = y
		\end{array}
	\end{equation*}
\end{observation}

Tutte queste propriet\`a valgono per tutti gli insiemi definiti prima tranne che per
$\mathbb{C}$. In $\mathbb{N}$ vale un'altra proprit\`a, ovvero quella del
\emph{minimo}, definita in questo modo: Ogni insieme non vuoto di numeri naturali
ha un minimo.

\begin{example}
	\begin{equation*}
		A = \{x \in \mathbb{N} \mid \text{$x$ \`e primo}\}
	\end{equation*}
	In questo caso il $\min{A} = 2$.

\end{example}
Pi\`u in generale $n = \min{A}$ dove ha \`e un insieme non nullo di naturali
se e solo se
\begin{itemize}
	\item $n \in A$
	\item $\forall x \in A \; n \leq x$
\end{itemize}