\section{Massimo comun divisore e minimo comune multiplo}
Parliamo ora di massimo comune divisore, minimo comune multiplo e numeri primi.

\subsection{Massimo comune divisore}

\begin{defn}
	Chiamo \textbf{massimo comune divisore} tra $a$ e $b$ il numero $d$ tale che $d \mid a$ e
	$d \mid b$ ($d$ \`e un divisore comune) e preso un qualsiasi divisore comune $d'$ tra $a$
	e $b$, $d' \leq d$ ($d$ \`e il massimo). D'ora in poi indicheremo il massimo comune divisore
	tra due numeri $a, b$ con $(a, b)$.
\end{defn}

\begin{theorem}
	Dati $a, b \in \mathbb{Z}$, vale
	\begin{equation*}
		\begin{array}{llll}
			(a, b) & = (-a , b) & = (-a, -b) & = \dots
		\end{array}
	\end{equation*}
\end{theorem}

\begin{theorem}
	Dati $a, b \in \mathbb{Z}$ e sia $(a, b)$ il massimo comun divisore tra loro, vale
	\begin{equation*}
		\begin{array}{lll}
			(a, b) & = (a - b, b) & = (a, b - a)
		\end{array}
	\end{equation*}
\end{theorem}

\begin{proposition}
	Siano $a, b, n \in \mathbb{Z}$, vale
	\begin{equation*}
		(an, bn) = n(a, b)
	\end{equation*}
\end{proposition}

\begin{theorem}
	Siano $c, x, y \in \mathbb{Z}$ tre interi qualsiasi. Se
	\begin{equation*}
		\begin{array}{rcl}
			(c, x) = 1 & \Rightarrow & (c, xy) = (c, y)
		\end{array}
	\end{equation*}
\end{theorem}

\subsection{Minimo comune multiplo}

\begin{defn}
	Chiamo \textbf{minimo comune multiplo} tra $a$ e $b$ il numero $t$ tale che $a \mid t$
	e $b \mid t$ ($t$ \`e un multiplo comune) e preso un qualsiasi multiplo comune $t'$ fra
	$a$ e $b$, $t \leq t'$ ($t$ \`e il minimo). D'ora in poi verr\`a utilizzata la notazione
	$[a, b]$ per indicare il minimo comune multiplo tra $a$ e $b$.
\end{defn}

\begin{proposition}
	Siano $a, b \in \mathbb{Z}$ due interi tali che
	\begin{equation*}
		\begin{array}{ll}
			d & = (a, b) \\
			t & = [a, b]
		\end{array}
	\end{equation*}
	allora
	\begin{equation*}
		d \cdot t = a \cdot b
	\end{equation*}
\end{proposition}

\begin{proposition}
	Dati $a, b, c \in \mathbb{Z}$ vale che
	\begin{equation*}
		[a, b] \mid c \Leftrightarrow a \mid c \wedge b \mid c
	\end{equation*}
\end{proposition}

\subsection{Scomposizione in numeri primi}
Prima di addentrarci nell'argomento \`e necessario fornire una definizione di numero primo

\begin{defn}
	Se numero $p > 1$ possiede la propriet\`a:
	\begin{equation*}
		\begin{array}{rcl}
			p \mid ab & \Rightarrow & p \mid a \text{ oppure } p \mid b
		\end{array}
	\end{equation*}
	allora \`e un numero \textbf{primo}.
\end{defn}

\begin{theorem}[Teorema fondamentale dell'aritmetica]
	Ogni numero naturale $n \geq 1$ si scrive in maniera univoca come prodotto di numeri primi.

	Questo teorema si compone di due punti fondamentali: esistenza e unicit\`a.
	\begin{proof}[Esistenza]
		Dimostriamo per induzione che, per un qualsiasi numero naturale $n$, esistono $k$ numeri
		primi che lo fattorizzano.
		Consideriamo il numero $n$ e consideriamo due casi:
		\begin{enumerate}
			\item Se $n$ \`e primo ottengo direttamente il risultato poich\'e $n$ \`e la
			      fattorizzazione di se stesso in numeri primi.
			\item Se $n$ non \`e primo allora esiste almeno un altro suo divisore $a$ diverso da
			      1 e da $n$. Posso quindi affermare che
			      \begin{equation*}
				      n = ab
			      \end{equation*}
			      A questo punto posso applicare l'ipotesi anche ad $a$ e $b$ in questo modo:
			      \begin{equation*}
				      \begin{array}{ll}
					      a & = p_1 p_2 \dots p_k \\
					      b & = q_1 q_2 \dots q_r
				      \end{array}
			      \end{equation*}
			      Quindi posso scrivere $n$ in questo modo
			      \begin{equation*}
				      n = (p_1 p_2 \dots p_k)(q_1 q_2 \dots q_r)
			      \end{equation*}
			      Sono quindi riuscito a scrivere $n$ come prodotto di numeri primi.
		\end{enumerate}
	\end{proof}

	\begin{proof}[Unicit\`a]
		Dimostriamo per induzione, che, la rappresentazione trovata nella dimostrazione
		precedente \`e unica, qualsiasi sia il numero naturale $n$.
		\begin{enumerate}
			\item Nel caso iniziale $n = 1$ non abbiamo altra scelta che fare il prodotto vuoto
			      di numeri primi.
			\item Nel caso $n > 1$ consideriamo due possibili rappresentazioni di $n$:
			      \begin{equation*}
				      \begin{array}{ll}
					      n & = p_1 p_2 \dots p_k \\
					      n & = q_1 q_2 \dots q_r
				      \end{array}
			      \end{equation*}
			      con tutti i $p_i$ e $q_i$ numeri primi.
			      Dato che $p_1$ \`e primo e divide $n$ allora
			      \begin{equation*}
				      p \mid p_2 \dots p_k
			      \end{equation*}
			      e
			      \begin{equation*}
				      p \mid q_1 q_2 \dots q_r
			      \end{equation*}
			      Ma come sappiamo, per le propriet\`a dei numeri primi questo significa che
			      $p_1$ divide uno dei $q_i$. Senza perdere di generalit\`a possiamo supporre
			      $q_1$. Se per\`o $p_1 \mid q_1$ e sono entrambi primi allora $p_1 = q_1$ quindi:
			      \begin{equation*}
				      \frac{n}{p_1} = p_2 \dots p_k
			      \end{equation*}
			      e
			      \begin{equation*}
				      \frac{n}{q_1} = q_2 \dots q_r
			      \end{equation*}
			      Ma se $p_1 = q_1$
			      \begin{equation*}
				      \frac{n}{p_1} = q_2 \dots q_r
			      \end{equation*}
			      Seguendo questo schema possiamo affermare che per ogni $p_i$ c'\`e un $q_j$
			      tale che $p_i = q_j$ e dunque la rappresentazione \`e la stessa ed \`e dunque
			      unica.
		\end{enumerate}
	\end{proof}
\end{theorem}

\subsection{Principio di inclusione esclusione}
Il principio di inclusione esclusione ci permette di determinare la cardinalit\`a dell'unione
di insiemi finiti in funzione delle cardinalit\`a delle intersezioni.

A noi, in particolare, serve a contare multipli o divisori per certi insiemi. Questo ci sar\`a
utile pi\`u tardi per risolvere sistemi di congruenze con parametro.

\begin{theorem}[Principio di inclusione esclusione per 2 insiemi]
	Siano $A$ e $B$ due insiemi, la cardinalit\`a della loro unione \`e data da
	\begin{equation*}
		|A \cup B| = |A| + |B| - |A \cap B|
	\end{equation*}
\end{theorem}

\begin{theorem}[Principio di inclusione esclusione per $n$ insiemi]
	Consideriamo $n$ insiemi $X_1, \dots, X_n$. La cardinalit\`a della loro unione \`e
	data da
	\begin{equation*}
		|\cup_{i = 1}^n X_i| = \sum_{k=1}^n (-1)^{k-1}
		\sum_{1 \leq i_1 < \dots < i_k \leq n}
		|A_{i_1} \cap \cdots \cap A_{i_k}|
	\end{equation*}
\end{theorem}

\begin{example}
	Quanti $m$ tali che $m$ non \`e n\'e un multiplo di 8 n\'e un multiplo di 3 sono compresi
	tra 1 e 100 ?

	Contiamo i multipli di 8 tra 1 e 100
	\begin{equation*}
		\frac{100}{8} = 12
	\end{equation*}
	Contiamo ora i multipli di 3 tra 1 e 100:
	\begin{equation*}
		\frac{100}{3} = 33
	\end{equation*}
	Ora contiamo quanti $m$ sono sia multipli di 8 sia multipli di 3:
	\begin{equation*}
		\frac{100}{8 \cdot 3} = 4
	\end{equation*}
	(considero sempre la parte intera della divisione).

	A questo punto applico il principio di inclusione esclusione e ottengo
	\begin{equation*}
		12 + 33 - 4 = 41
	\end{equation*}
	Ho contato cos\`i tutti i multipli di 3 e tutti i multipli di 8 compresi fra 1 e 100.
	Dato che a noi servivano i numeri che non fossero multipli n\'e di 3 n\'e di 8 basta
	fare la differenza
	\begin{equation*}
		100 - 41 = 59
	\end{equation*}
	per ottenere il risultato desiderato.
\end{example}

Usiamo questo principio per vedere altre propriet\`a del MCD e del mcm

\begin{example}
	Consideriamo due numeri $n$ e $m$:
	\begin{equation*}
		\begin{array}{ll}
			n & = 2^3 \cdot 5^2 \cdot 7 \cdot 11^4 \\
			m & = 2^2 \cdot 3 \cdot 5^4 \cdot 11^3
		\end{array}
	\end{equation*}
	Siano $I_n$ e $I_m$ gli insiemi delle potenza dei numeri primi che dividono $n$ e $m$:
	\begin{equation*}
		\begin{array}{ll}
			I_n & = \{ 1, 2, 2^2, 2^3, 5, 5^2, 7, 11, 11^2, 11^3, 11^4 \} \\
			I_m & = \{ 1, 2, 2^2, 3, 5, 5^2, 5^3, 5^4, 11, 11^2, 11^3 \}
		\end{array}
	\end{equation*}
	Consideriamo ora il MCD e il mcm fra $n$ e $m$:
	\begin{equation*}
		\begin{array}{ll}
			(n, m)              & = 2^2 \cdot 5^2 \cdot 11^3         \\
			\left[ n, m \right] & = 2^3 \cdot 3 \cdot 5^4 \cdot 11^4
		\end{array}
	\end{equation*}
	L'intersezione dei due insiemi precedenti \`e l'insieme delle potenze dei numeri primi
	che dividono $(n, m)$, invece, l'unione dei due insiemi \`e l'insieme delle potenze dei
	numeri primi che dividono $[n, m]$:
	\begin{equation*}
		\begin{array}{ll}
			I_n \cap I_m & = \{ 1, 2, 2^2, 5, 5^2, 11, 11^2, 11^3 \}                         \\
			I_n \cup I_m & = \{ 1, 2, 2^2, 2^3, 3, 5, 5^2, 5^3, 5^4, 11, 11^2, 11^3, 11^4 \}
		\end{array}
	\end{equation*}
\end{example}

\begin{theorem}
	Consideriamo $a, x, y \in \mathbb{Z}$ tre numeri interi qualsiasi:
	\begin{equation*}
		(a, [x, y]) = [(a, x), (a, y)]
	\end{equation*}
	Quindi
	\begin{equation*}
		I_a \cap (I_x \cup I_y) = (I_a \cap I_x) \cup (I_a \cap I_y)
	\end{equation*}
	ne ricaviamo una sorta di propriet\`a distributiva.
\end{theorem}

\begin{theorem}
	Se $(x, y) = 1$ allora
	\begin{equation*}
		[x, y] = \pm xy
	\end{equation*}
	infatti
	\begin{equation*}
		I_x \cap I_y = \varnothing
	\end{equation*}
\end{theorem}

\begin{theorem}
	Siano $a, x, y \in \mathbb{Z}$
	\begin{equation*}
		\begin{array}{rcl}
			(a, x) = 1 & \Rightarrow & (a, xy) = (a, y)
		\end{array}
	\end{equation*}
\end{theorem}

\begin{observation}
	Siano $a, b, e \in \mathbb{Z}$ tre interi qualsiasi e $p$ un numero primo vale:
	\begin{equation*}
		a = \pm b \Leftrightarrow \forall p, e \; p^e \mid a \Leftrightarrow p^e \mid b
	\end{equation*}
\end{observation}