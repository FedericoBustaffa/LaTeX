\section{Fattorizzazione di polinomi}
\subsection{Strutture algebriche}
\begin{defn}
	Chiamo \textbf{struttura algebrica}, un insieme su cui sono definite delle operazioni
	fra i suoi elementi. Tali operazioni possono essere unarie, binarie, ternarie ecc.
\end{defn}

\begin{defn}
	Sia $S$ l'insieme considerato le \textbf{operazioni} $*$ sono funzioni che associano ad
	uno o pi\`u elementi dell'insieme $S$ un elemento dell'insieme $S$. Possono essere di
	vario tipo:
	\begin{equation*}
		\begin{array}{l}
			\text{unaria: } * : S \rightarrow S           \\
			\text{binaria: } * : S \times S \rightarrow S \\
			\dots
		\end{array}
	\end{equation*}
\end{defn}

\begin{defn}
	Chiamo \textbf{gruppo} una struttura algebrica caratterizzata da un insieme $G$ e da
	un'operazione binaria $*$ che gode delle seguenti propriet\`a:
	\begin{itemize}
		\item Deve essere \emph{associativa}: $(a * b) * c = a * (b * c)$.
		\item Deve esistere l'\emph{elemento neutro}: $a * E = a$.
		\item Ogni elemento di $G$ ha un \emph{inverso}: $a * inv(a) = E$
	\end{itemize}
	Aggiungiamo che nel caso in cui la struttura in questione possieda anche la propriet\`a
	commutativa per $*$
	\begin{equation*}
		a * b = b * a
	\end{equation*}
	allora si tratta di \textbf{gruppo abeliano}.
\end{defn}

\begin{example}
	L'insieme dei numeri reali $\mathbb{R}$ con l'operazione di somma \`e un gruppo abeliano.
	Se invece considero $\mathbb{R}$ con l'operazione di prodotto non ottengo pi\`u un gruppo
	dato che la terza propriet\`a non \`e soddisfatta per tutti i valori in $\mathbb{R}$.
\end{example}

\begin{defn}
	Chiamo $H$ un \textbf{sottogruppo} di $G$ se $H$ \`e esso stesso un gruppo rispetto
	all'operazione $*$. In tal caso si user\`a la notazione:
	\begin{equation*}
		H < G
	\end{equation*}
\end{defn}

\begin{theorem}
	Siano $G$ un gruppo, allora $H \subset G$ \`e un sottogruppo di $G$ se e solo se:
	\begin{itemize}
		\item Elemento neutro in $H$: $E \in H$.
		\item Chiusura rispetto a $*$: se $a, b \in H$ allora $a * b \in H$.
		\item Chiusura per l'inverso: se $a \in H$ allora $a^{-1} \in H$.
	\end{itemize}
\end{theorem}

\begin{defn}
	Sia $S$ un sottoinsieme chiamo \textbf{sottogruppo generato} da $S$ il pi\`u piccolo
	sottogruppo contenente $S$ e lo indico con $<S>$. Quando un gruppo \`e generato da un
	unico elemento si dir\`a che il gruppo \`e \textbf{ciclico}.
\end{defn}

\begin{defn}
	Chiamo \textbf{anello} una struttura algebrica formata da un insieme $A$ dotato di due
	operazioni (somma e prodotto).
\end{defn}

\begin{theorem}
	Dato un gruppo $A$ munito di somma e prodotto \`e un anello se e solo se:
	\begin{itemize}
		\item Il gruppo formato da $A$ munito della somma \`e abeliano.
		\item L'operazione di prodotto \`e associativa.
		\item Valgono le seguenti leggi distributive:
		      \begin{equation*}
			      \begin{array}{ll}
				      a \cdot (b + c) & = ab + ac \\
				      (a + b) \cdot c & = ac + bc
			      \end{array}
		      \end{equation*}
	\end{itemize}
	Ci sono ulteriori propriet\`a che il prodotto pu\`o avere per certi insiemi e questo
	pu\`o generare nuovi tipi di anello:
	\begin{itemize}
		\item Se il prodotto \`e commutativo, si ha un \textbf{anello commutativo}.
		\item Se il prodotto ha un elemento neutro, si ha un \textbf{anello con unit\`a}.
		\item Se l'insieme viene privato dello zero per la somma e si crea un gruppo rispetto
		      al prodotto si ha un \textbf{anello di divisione} o \textbf{corpo}.
	\end{itemize}
\end{theorem}

\begin{example}
	L'insieme $\mathbb{Z}$ rispetto a somma e prodotto \`e un anello commutativo.
\end{example}

\begin{example}
	L'insieme
	\begin{equation*}
		\mathbb{Z}/_6 = \{ [0], [1], [2], [3], [4], [5] \}
	\end{equation*}
	\`e un anello commutativo poich\'e
	\begin{equation*}
		[2] \cdot [3] = [0]
	\end{equation*}
\end{example}

\begin{defn}
	Chiamo \textbf{campo} un corpo commutativo.
\end{defn}

\begin{example}
	Tutti gli insiemi $\mathbb{R}, \mathbb{Q}, \mathbb{C}$ sono campi.
\end{example}

\begin{defn}
	Diremo che $B \subset A$ \`e un \textbf{sottoanello} di $A$ se $B$ \`e esso stesso un
	anello rispetto alle operazioni ereditate da $A$.
\end{defn}

\begin{defn}
	Sia $x \in A$. Se esiste $y \in A$, $y \neq 0$ tale che
	\begin{equation*}
		x \cdot y = 0
	\end{equation*}
	allora chiamo \textbf{divisore di zero sinistro} l'elemento $x$ di $A$.

	Analogamente se esiste $y \in A$, $y \neq 0$ tale che
	\begin{equation*}
		y \cdot x = 0
	\end{equation*}
	allora chiamo $x$ \textbf{divisore di zero destro}.
\end{defn}

\begin{defn}
	Un anello commutativo con unit\`a privo di divisori di zero si dice
	\textbf{dominio	di integrit\`a}.
\end{defn}

\begin{proposition}
	Ogni campo \`e un dominio di integrit\`a.
\end{proposition}

\begin{example}
	L'insieme $\mathbb{Z}$ \`e un dominio di integrit\`a.
\end{example}

\subsection{Polinomi}

\begin{defn}
	Chiamo
	\begin{equation*}
		p(x) = a_0 + a_1x + a_2x^2 + \cdots + a_n x^n \text{ con $a_n \neq 0$}
	\end{equation*}
	\textbf{polinomio di grado n} con $a_0, a_1, \dots, a_n \in \mathbb{R}$
	(oppure $\mathbb{C}$, $\mathbb{Z}/_m$, $\mathbb{Q}$, ecc.). Inoltre $p(x) \in \mathbb{R}[x]$
	(oppure $\mathbb{C}[x]$, $\mathbb{Z}/_m [x]$, $\mathbb{Q}[x]$ ecc.).
\end{defn}

\begin{defn}
	Sia $p(x)$ un generico polinomio di grado $n$, definisco \textbf{radice} del polinomio
	il numero $r$ tale che
	\begin{equation*}
		a_0 + a_1 r + a_2 r^2 + \cdots + a_n r^n = 0
	\end{equation*}
	In sintesi se $p(r) = 0$ allora $r$ \`e una radice del polinomio.
\end{defn}

\begin{observation}[Polinomi di primo grado]
	Possiamo dimostrare che i polinomi di primo grado con coefficienti sul campo $\mathbb{K}$
	hanno sempre una radice in $\mathbb{K}$. Consideriamo il generico polinomio di primo grado
	\begin{equation*}
		ax + b = 0
	\end{equation*}
	Abbiamo che la sua radice \`e
	\begin{equation*}
		x = -b a^{-1}
	\end{equation*}
	Sappiamo che $-b$ esiste perch\'e per ogni elemento in $\mathbb{K}$ esiste il suo inverso
	rispetto alla somma e sappiamo che esiste anche $a^{-1}$ perch\'e in un campo ogni elemento
	\`e invertibile rispetto al prodotto.

	Tuttavia c'\`e da precisare che i polinomi hanno sempre radice in un campo ma non \`e
	sempre detto che ne abbiano in un anello.
\end{observation}

\begin{observation}[polinomi di secondo grado]
	I polinomi di secondo grado potrebbe non avere radici nel campo $\mathbb{K}$ considerato
	ma potrebbero averne in altri campi.
	Per esempio
	\begin{equation*}
		x^2 = 2
	\end{equation*}
	non ha radici in $\mathbb{Q}$ poich\'e $\sqrt{2}$ non \`e un numero razionale ovvero non
	\`e esprimibile come $a / b$.
\end{observation}

\begin{theorem}[Regola di annullamento del prodotto]
	Se $ab = 0$ allora $a = 0$ oppure $b = 0$.
	\begin{proof}
		Se $ab = 0$ e $a \neq 0$ allora esiste $a^{-1}$. Quindi
		\begin{equation*}
			\begin{array}{ll}
				a^{-1}(ab) & = (a^{-1}a)b      \\
				1b         & \Rightarrow b = 0
			\end{array}
		\end{equation*}
	\end{proof}
	Va per\`o chiarito che funziona per i campi ma non per gli anelli.
\end{theorem}

\subsection{Divisione con resto tra polinomi}

\begin{theorem}
	Dati due polinomi $p(x), g(x) \in \mathbb{K}[x]$ con campo $\mathbb{K}$, esiste un
	quoziente $q(x)$ e un resto $r(x)$ in $\mathbb{K}[x]$ tali che
	\begin{equation*}
		p(x) = g(x) q(x) + r(x)
	\end{equation*}
	con
	\begin{equation*}
		deg(r(x)) < deg(g(x))
	\end{equation*}
\end{theorem}

Si procede come per le divisioni euclidee tra numeri interi ma l'obbiettivo ad ogni passaggio
\`e annullare il grado maggiore del polinomio.

\begin{example}
	Consideriamo
	\begin{equation*}
		\frac{x^3 - 2x^2 - x -6}{x^2 + x + 2} = x - 3 \text{ (resto 0)}
	\end{equation*}
	Infatti
	\begin{equation*}
		(x - 3)(x^2 + x + 2) = (x^3 - 2x^2 - x - 6)
	\end{equation*}
\end{example}

\begin{defn}
	Chiamo $p(x)$ \textbf{polinomio monico} se $a_n = 1$ ovvero se il coefficiente del monomio
	di grado massimo \`e proprio 1.
\end{defn}

\subsection{Teorema di Ruffini}
\begin{defn}
	Dico che $p(x) \mid g(x)$ (divide) se $p(x), g(x) \in \mathbb{K}[x]$ sono tali che
	la divisione $g(x) / p(x)$ da resto zero.
\end{defn}

\begin{theorem}[Teorema di Ruffini]
	Se $p(x) \in \mathbb{K}[x]$ e $a \in \mathbb{K}$ allora
	\begin{equation*}
		\begin{array}{ccc}
			p(a) = 0 & \Leftrightarrow & (x - a) \mid p(x)
		\end{array}
	\end{equation*}
	\begin{proof}
		Si dimostra facilmente scrivendo
		\begin{equation*}
			\begin{array}{ll}
				p(x) & = (x - a) q(x) + r \\
				p(a) & = (a - a) q(a) + r \\
				p(a) & = r
			\end{array}
		\end{equation*}
		Dunque $p(a) = 0$ se e solo se $r = 0$ ovvero se $(x - a) \mid p(x)$.
	\end{proof}
\end{theorem}

\subsection{Polinomi irriducibili}
Il concetto \`e simile ai numeri primi in $\mathbb{Z}$.

\begin{defn}
	Chiamo $p(x) \in \mathbb{K}[x]$ \textbf{polinomio irriducibile} se non si fattorizza nella
	forma $p(x) = a(x)b(x)$ con $deg(a(x)), deg(b(x)) < deg(p(x))$.
\end{defn}

\begin{observation}
	Siano $a(x), b(x) \in \mathbb{K}[x]$ allora
	\begin{equation*}
		deg(a(x) \cdot b(x)) = deg(a(x)) + deg(b(x))
	\end{equation*}
\end{observation}

\begin{observation}
	Non si deve confondere l'essere fattorizzabile con l'avere radici. Un polinomio ha
	radici se e solo se ha un fattore di primo grado.
\end{observation}

\begin{observation}
	In un campo vale che se $p(x) = a(x)b(x)$ allora le radici di $p(x)$ sono
	l'unione delle radici di $a(x)$ con le radici di $b(x)$.
\end{observation}

\begin{observation}[Polinomi di terzo grado]
	Se $p(x)$ \`e di terzo grado e si fattorizza allora ha soluzione perch\'e un fattore sar\`a
	di primo grado e l'altro di secondo grado.
\end{observation}

\begin{proposition}
	Ogni polinomio si fattorizza in $\mathbb{R}[x]$ in un prodotto di fattori di primo e secondo
	grado.
\end{proposition}

\begin{lemma}[Lemma di Gauss]
	Se un polinomio \`e irriducibile in $\mathbb{Z}[x]$, allora lo \`e anche in
	$\mathbb{Q}[x]$.

	Messo in altri termini: Se un polinomio in $\mathbb{Z}[x]$ \`e fattorizzabile in
	$\mathbb{Q}[x]$ allora \`e fattorizzabile anche in $\mathbb{Z}[x]$.
\end{lemma}

\begin{theorem}
	Dato $p(x)$ esistono $p_1(x), \dots, p_k(x)$ tali che
	\begin{equation*}
		p(x) = p_1(x) \cdot ... \cdot p_k(x) \text{ con $p_i(x)$ irriducibili}
	\end{equation*}
	e questa scomposizione \`e unica a meno di fattori numerici e ordine dei fattori.
\end{theorem}

\subsection{Fattorizzazione}
Di seguito tratteremo tecniche per la fattorizzazione e studieremo le condizioni in cui
un polinomio \`e riducibile o meno.

Partiamo con l'elencare i seguenti prodotti notevoli che ci saranno utili pi\`u avanti.
\begin{itemize}
	\item \emph{Differenza di quadrati}: $a^2 - b^2 = (a + b)(a - b)$
	\item \emph{Somma di cubi}: $a^3 + b^3 = (a + b)(a^2 - ab + b^2)$
	\item \emph{Differenza di cubi}: $a^3 - b^3 = (a - b)(a^2 + ab + b^2)$
\end{itemize}

\begin{example}
	Proviamo a fattorizzare
	\begin{equation*}
		x^6 - 1
	\end{equation*}
	Possiamo usare il prodotto notevole per la differenza di quadrati.
	\begin{equation*}
		x^6 - 1 = (x^3 + 1)(x^3 - 1)
	\end{equation*}
	A questo punto posso usare la somma di cubi per il primo fattore e la differenza di cubi per
	il secondo:
	\begin{equation*}
		(x^3 + 1)(x^3 - 1) = (x + 1)(x^2 - x + 1)(x - 1)(x^2 + x + 1)
	\end{equation*}
	Svolgendo i calcoli ottengo che le radici sono $x = -1$ e $x = 1$.
\end{example}

\begin{theorem}
	Sia
	\begin{equation*}
		p(x) = a_0 + a_1 x + a_2 x^2 + \dots + a_n x^n
	\end{equation*}
	un generico polinomio in $\mathbb{Z}[x]$ e sia $\frac{c}{d}  \in \mathbb{Q}$ con
	$c, d \in \mathbb{Z}$ ridotti ai minimi termini.

	Se
	\begin{equation*}
		p \left( \frac{c}{d} \right) = 0
	\end{equation*}
	allora $c \mid a_0$ e $d \mid a_n$
	\begin{proof}
		Scriviamo
		\begin{equation*}
			p \left( \frac{c}{d} \right) =
			a_0 + a_1 \frac{c}{d} + a_2 \frac{c^2}{d^2} + \dots + a_n \frac{c^n}{d^n} = 0
		\end{equation*}
		Questo \`e vero se e solo se
		\begin{equation*}
			a_0 d^n + a_1 c d^{n - 1} + a_2 c^2 d^{n-2} + \cdots + a_n c^n = 0
		\end{equation*}
		Da quest'ultima equazione deduco che $a_n c^n$ \`e un multiplo di $d$ quindi $d \mid a_n$.
		Deduco anche che $a_0 d^n$ \`e un multiplo di $c$ quindi $c \mid a_0$.
	\end{proof}
\end{theorem}

\begin{theorem}
	Se $\mathbb{K}$ \`e un campo, $p(x) \in \mathbb{K}[x]$, $p(x) = g(x)h(x)$ sempre in
	$\mathbb{K}[x]$ allora
	\begin{equation*}
		\{ \text{radici } p(x) \} =
		\{ \text{radici } g(x) \} \cup \{ \text{radici } h(x) \}
	\end{equation*}
	ovvero
	\begin{equation*}
		\begin{array}{ccc}
			p(a) = 0 & \Leftrightarrow & g(a) = 0 \vee h(a) = 0
		\end{array}
	\end{equation*}
	Segno del fatto che nei campi vale la regola di annullamento del prodotto.
\end{theorem}

\begin{theorem}
	Se $\mathbb{K}$ \`e solo un anello allora
	\begin{equation*}
		\{ \text{radici } p(x) \} \supseteq
		\{ \text{radici } g(x) \} \cup \{ \text{radici } h(x) \}
	\end{equation*}
\end{theorem}

\begin{theorem}
	Se $\mathbb{K}$ \`e un campo, $p(x) = a_0 + a_1 x + \cdots + a_n x^n$ di grado $n$ allora
	$p(x)$ ha al massimo $n$ radici in $\mathbb{K}$.
	\begin{proof}
		Se $n = 1$, quindi il polinomio \`e $p(x) = a_0 + a_1 x = 0$ l'unica radice \`e
		\begin{equation*}
			x = \frac{-a_0}{a_1}
		\end{equation*}

		Se $n > 1$ allora vedo se ha radici. Se non ne ha ho finito.

		Se $\alpha$ \`e una radice allora
		\begin{equation*}
			p(x) = (x - \alpha) q(x) \text{ con } deg(n - 1)
		\end{equation*}
		Si pu\`o dimostrare per induzione che $q(x)$ ha $n - 1$ radici.

		Le radici di $p(x)$ sono date da
		\begin{equation*}
			\{ \text{radici } q(x) \} \cup \{ \alpha \}
		\end{equation*}
		quindi $p(x)$ ha al massimo $n$ radici.
	\end{proof}
\end{theorem}

\begin{observation}
	Se $\mathbb{K}$ \`e un anello, un polinomio di grado $n$ in $\mathbb{K}[x]$ potrebbe
	avere pi\`u di $n$ radici.
\end{observation}