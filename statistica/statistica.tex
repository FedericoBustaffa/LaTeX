\documentclass[11pt, a4paper]{report}

\pdfpagewidth\paperwidth
\pdfpageheight\paperheight

\usepackage[utf8]{inputenc}
\usepackage[T1]{fontenc}
\usepackage[italian]{babel}
\usepackage[hidelinks]{hyperref}
\usepackage{amsmath, amssymb, amsthm, amsfonts, mathtools}
\usepackage{tikz, pgfplots, pgf-pie}
\usepackage{caption, subcaption}

\pgfplotsset{compat=newest}

\theoremstyle{definition}
\newtheorem{theorem}{Teorema}[chapter]
\newtheorem{definition}{Definizione}[chapter]
\newtheorem{proposition}{Proposizione}[chapter]
\newtheorem{corollary}{Corollario}[chapter]
\newtheorem{lemma}{Lemma}[chapter]
\newtheorem{observation}{Osservazione}[chapter]
\newtheorem{example}{Esempio}[chapter]

\newcommand{\N}{\mathbb{N}}
\newcommand{\Z}{\mathbb{Z}}
\newcommand{\R}{\mathbb{R}}
\newcommand{\C}{\mathbb{C}}

\newcommand{\F}{\mathcal{F}}

\DeclareMathOperator{\Var}{Var}
\DeclareMathOperator{\Cov}{Cov}

\title{Statistica}
\author{Federico Bustaffa}
\date{02/02/2023}

\begin{document}

\maketitle
\tableofcontents

\part{Calcolabilità}

\chapter{Introduzione alla calcolabilità}
Iniziamo con la \textbf{teoria della calcolabilità} la quale
si pone come obbiettivo quello di definire cosa siano problemi,
funzioni e algoritmi, cercando di dare una definizione formale
di questi ultimi. Una volta definiti questi concetti sarà di
nostro interesse capire quali sono i problemi
\textbf{calcolabili} e quali invece no.

In questa prima parte non è di nostro interesse tenere di conto
le limitazioni che hanno i calcolatori reali. Ragioneremo quindi
supponendo che non questi non abbiamo limiti in tempo o spazio
per effettuare il calcolo.

Cercheremo quindi di capire quali sono i problemi
\textbf{calcolabili} mediante una \textbf{procedura effettiva},
quali invece \textbf{non} sono calcolabili per capire se ce ne
sono di interessanti, se ne esistono di reali o se sono solo
artificiali e puramente teorici.

\chapter{Analisi e gestione del rischio}
L'approccio incondizionale comporta costi molto elevati e spesso inaccettabili, inoltre richiede una quantità di lavoro
enorme e spesso inutile.

Con l'approccio condizionale invece si cerca di capire quali componenti del sistema si possono difendere e soprattutto
quali componenti \emph{conviene} difendere.

Per capirlo è necessaria un'\textbf{analisi del rischio} con la quale si cerca di individuare la tipologia di attacco
più probabile in relazione al sistema che stiamo cercando di proteggere.

\section{Tipologie di analisi}
L'analisi del rischio si divide in diverse sottocategorie più specifiche che ci permettono di individuare in modo più
mirato eventuali problemi di diversa natura. In particolare parliamo di
\begin{itemize}
	\item Analisi delle risorse da proteggere (asset)
	\item Analisi delle minacce
	\item Analisi delle vulnerabilità
	\item Analisi degli attacchi
	\item Analisi degli impatti
	\item Individuazione del rischio, rischio accettabile ed introduzione di contromisure
\end{itemize}

\subsection{Analisi delle risorse}
In questa fase si cerca di individuare un insieme di oggetti ed alcune proprietà di sicurezza. Si definisce in seguito
una politica su questi oggetti in termini delle proprietà precedenti come diritti di lettura, scrittura, esecuzione
e così via.

\subsection{Analisi delle minacce}
In questa fase cerchiamo di capire chi è interessato ad attaccare il nostro sistema per rubare o modificare informazioni
o per impedire agli utenti di utilizzare il sistema.

Le possibili minacce possono arrivare sia da attaccanti che vogliono violare il sistema sia da eventi naturali che
potrebbero in qualche modo comprometterne l'integrità (in questo caso parliamo di \emph{safety}).

\subsubsection{Safety e Security}
Come anticipato, quando ci riferiamo alla capacità di un sistema di resistere ad eventi di origine non umana e casuale
come terremoti, crolli e così via, parliamo di \textbf{safety}.

Parliamo invece di \textbf{security} quando ci riferiamo alla capacità del sistema di resistere ad attacchi umani con
uno scopo malizioso per raggiungere un obbiettivo.

\subsection{Analisi delle vulnerabilità}
In questo caso cerchiamo di individuare quali sono le vulnerabilità che permettono ad un attaccante di ottenere, in un 
certo numero di passi, accesso alle risorse di suo interesse.

\part{Probabilità}

\chapter{Probabilità e indipendenza}
Il calcolo delle \textbf{probabilità} ci serve per la costruzione di un modello
\textbf{statico inferenziale}, il quale unisce statistica descrittiva e probabilità per descrivere
la realtà.

Prima di addentrarci nei formalismi matematici introduciamo qualche concetto \emph{primitivo}.
\begin{itemize}
	\item Quello che vogliamo calcolare è la probabilità che un certo \textbf{evento} accada. Il
	      concetto
	      di evento non si può formalizzare, ma possiamo fare qualche esempio: se lanciamo un dado,
	      il fatto che esca 6 è un evento.
	\item Il secondo concetto è quello di \textbf{casuale} o \textbf{aleatorio}
	      (\textbf{stocastico}). Queste tre parole hanno quasi lo stesso significato ma possiamo
	      generalizzare dicendo che significano \emph{"dovuto al caso"}. Più avanti introdurremo
	      altri termini come \emph{variabili aleatorie} e \emph{indipendenza stocastica}.
\end{itemize}
Due dei problemi che andremo a trattare sono:
\begin{itemize}
	\item La \textbf{rappresentazione} degli eventi.
	\item Quali \textbf{proprietà} devono soddisfare le probabilità.
\end{itemize}
Per riuscire a capire meglio tutti questi concetti sarà necessario anche conoscere qualche nozione
di base sulle serie.

\section{Spazi di probabilità}
Possiamo vedere tutti gli esiti possibili di un esperimento come un insieme, detto
\textbf{spazio campionario}, che indicheremo con $\Omega$.

Quando facciamo delle invece \emph{affermazioni} su tali eventi stiamo (in linea generale)
restringendo lo spazio di campionario e ne stiamo considerando quindi dei sottoinsiemi, detti
\textbf{eventi}.

Se invece facciamo riferimento a singoli eventi appartenenti ad un insieme si parla di
\textbf{eventi elementari} e sono sia gli elementi $\omega$ in $\Omega$ sia gli insiemi singoletto
$\{\omega\}$.

\begin{example}
	Consideriamo ad esempio 2 lanci di moneta, e assegnamo il valore 1 a testa e 0 a croce. Abbiamo
	che
	\[ \Omega = \{ (0,0), (0,1), (1,0), (1,1) \} \]
	rappresenta tutti possibili esiti di 2 lanci di moneta. E un evento elementare è per esempio
	$(0,1)$. Con l'affermazione "\emph{è uscita almeno una testa}" stiamo considerando un
	sottoinsieme $A$ di $\Omega$ definito come segue
	\[ A = \{ (0,1), (1,0), (1,1) \} \]
	definendo quindi un sottoinsieme di eventi.
\end{example}

Dato che possiamo vedere gli spazi campionari come insiemi di eventi, è facile trovare una
correlazione tra le operazioni insiemistiche e i connettivi logici.
\begin{itemize}
	\item $A \cup B =$ \verb|A OR B|
	\item $A \cap B =$ \verb|A AND B|
	\item $A^c = $ \verb|NOT A|
	\item $A \backslash B =$ \verb|A AND NOT B|
	\item $A \subseteq B =$ \verb|A implica B|
\end{itemize}
Aggiungiamo inoltre che se $A \cap B = \emptyset$ allora si dice che $A$ e $B$ sono
\textbf{disgiunti}.

\begin{example}
	Se lanciamo un dado a 6 facce abbiamo
	\[ \Omega = \{ 1, 2, 3, 4, 5, 6 \} \]
	L'insieme degli eventi tali che il numero uscito sia
	\begin{itemize}
		\item Pari è $A = \{ 2, 4, 6 \}$
		\item Pari e maggiore di 3 è $A \cap B = \{ 4, 6 \}$
		\item Pari ma non maggiore di 3 è $A \backslash B = A \cap B^c = \{2\}$
		\item Sia pari che dispari è $A \cap A^c = \emptyset$
	\end{itemize}
\end{example}

Ora possiamo gestire gli eventi legati da connettivi logici tramite operazioni insiemistiche
equivalenti.

Vogliamo quindi definire una classe di eventi che sia chiusa per le operazioni insiemistiche
appena menzionate, ovvero vogliamo che le operazioni insiemistiche tra eventi producano ancora
degli eventi. Questo è banale se consideriamo tutti i sottoinsiemi di $\Omega$, ossia le
\textbf{parti} di $\Omega$, indicati con $\mathcal{P}(\Omega)$, ma in generale non è sempre
possibile.

\begin{definition}
	Sia $\Omega \neq \emptyset$ e $\F \subseteq \mathcal{P}(\Omega)$ una famiglia di sottoinsiemi
	di $\Omega$. Diciamo che $\F$ è un'\textbf{algebra di parti} se
	\begin{itemize}
		\item $\Omega, \emptyset \in \F$
		\item $\F$ è stabile per la complementazione: se $A \in \F$ allora $A^c \in \F$
		\item $\F$ è stabile per unione finita, ovvero se $A_1, A_2, ..., A_n \in \F$ allora
		      anche la loro unione $\cup_{i=1}^n A_i$ appartiene a $\F$.
	\end{itemize}
	Se inoltre $\Omega$ e una sua algebra di parti $\F$ hanno infiniti elementi, diciamo che $\F$
	è una $\sigma$\textbf{-algebra} se soddisfa l'ipotesi addizionale:
	\begin{itemize}
		\item $\F$ è stabile per unione numerabile, ovvero se esiste una successione di
		      sottoinsiemi $A_1, A_2, ...$ appartiene a $\F$ anche la loro unione
		      $\cup_{i=1}^\infty A_i$ appartiene a $\F$.
	\end{itemize}
\end{definition}

La $\sigma$-algebra $\F$ è quindi un insieme chiuso per tutte le operazioni insiemistiche viste
sopra ed è la classe degli eventi \emph{ammissibili}.

\subsection{Funzione di probabilità}
La \textbf{probabilità} $P$ di un evento $A \in \F$ è il grado di fiducia che $A$ si realizzi ed è
compreso tra 0 e 1.

\`E intuitivo che, se due eventi $A$ e $B$ sono disgiunti, allora la probabilità che si realizzi
$A \cup B$ è $P(A) + P(B)$. Questo significa che la probabilità è una funzione d'insieme
\textbf{finitamente additiva}.

Queste prime nozioni servono per dare una definizione più generale e più rigorosa della probabilità
considerando anche i casi in cui dobbiamo trattare infiniti eventi.

\begin{definition}
	Dato $\Omega$ un insieme e $\F$ una $\sigma$-algebra di $\Omega$, la \textbf{probabilità} è
	una funzione
	\[ P : \F \to [0, 1] \]
	tale che
	\begin{itemize}
		\item L'evento certo ha probabilità unitaria: $P(\Omega) = 1$.
		\item Se la successione degli $A_i$ è una successione di elementi di $\F$ a due a due
		      disgiunti, si ha, nel caso in cui la successione sia finita, che
		      \[ P \left( \bigcup_{i=1}^n A_i \right) = \sum_{i=1}^n P(A_i) \]
		      Se invece la successione è infinita allora vale
		      \[
			      P \left( \bigcup_{i=1}^{+\infty} A_i \right) = \sum_{i=1}^{+\infty} P(A_i) =
			      \lim_{n \to +\infty} \sum_{i=1}^n P(A_i)
		      \]
		      Questa proprietà è detta $\sigma$\textbf{-additività}.
	\end{itemize}
\end{definition}

\begin{definition}
	La terna $(\Omega, \F, P)$ formata da uno spazio campionario, una $\sigma$-algebra ed una
	probabilità $P$ definita su $\F$ viene chiamata \textbf{spazio di probabilità}.
\end{definition}

\paragraph{Proprietà} La probabilità, così come l'abbiamo definita, comporta che
\begin{itemize}
	\item $P(\emptyset) = 0$
	\item $P(A^c) = 1 - P(A)$
	\item $P(A \backslash B) = P(A) - P(B)$
	\item $P(A \cup B) = P(A) + P(B) - P(A \cap B)$
	\item Caso non banale è quello di $P(A \cup B \cup C)$ in cui abbiamo che
	      \begin{multline*}
		      P(A \cup B \cup C) = P(A) + P(B) + P(C) - \\
		      P(A \cap B) - P(A \cap C) - P(B \cap C) + P(A \cap B \cap C)
	      \end{multline*}
\end{itemize}

\begin{proposition}
	Sia $A \in \F$ una successione infinita, allora si dice che
	\begin{itemize}
		\item $A$ è una successione \textbf{crescente}, cioè $A_i \subseteq A_{i+1}$, $\forall i$
		      se vale
		      \[ A = \cup_{i=1}^\infty A_i \]
		\item $A$ è una successione \textbf{decrescente}, cioè $A_{i+1} \subseteq A_i$,
		      $\forall i$ se vale
		      \[ A = \cap_{i=1}^\infty A_i \]
	\end{itemize}
	In entrambi i casi vale
	\[ P(A) = \lim_{i \to \infty} P(A_i) \]
\end{proposition}

\begin{definition}
	Se $P(A) = 0$ si dice che $A$ è un evento \textbf{trascurabile}.
\end{definition}

\begin{definition}
	Se $P(A) = 1$ si dice che $A$ è un evento \textbf{quasi certo}.
\end{definition}

Facciamo ora qualche considerazione sulla $\sigma$-algebra per capire meglio il perché la abbiamo
definita e perché in precedenza abbiamo detto che non sempre possiamo considerare l'insieme di
tutti gli esiti possibili. Consideriamo $\Omega$ finito o numerabile
\[ \Omega = \{ \omega_1, \omega_2, ..., \omega_i, ... \} \]
allora possiamo prendere $\F = \mathcal{P}(\Omega)$ e la probabilità è univocamente determinata
da $p_i = P(\{ \omega_i \})$. Vale infatti che
\[ P(A) = \sum_{\omega_i \in A} p_i \]
Quindi, nel caso $A$ abbia un numero finito di elementi, $P(A)$ equivale alla somma dei $p_i$, se
invece $A$ ha un numero infinito di elementi, $P(A)$ equivale alla serie dei $p_i$.

Proviamo ora a definire un modello di probabilità per la scelta casuale di un punto sull'intervallo
$[a,b]$ in modo che la scelta sia uniforme. In questo caso abbiamo che $\Omega$ è non
numerabile in quanto esistono infiniti punti tra $a$ e $b$. Ne segue che la probabilità di
scegliere un punto in questo intervallo con una funzione di probabilità definita come in
precedenza non può che essere 0 poiché abbiamo infiniti casi possibili e un solo caso favorevole.

Potrebbe venirci in mente di definire la probabilità di scegliere un punto $x$ all'interno di
$[a,b]$ come la lunghezza dell'intervallo ($b-a$) ma ci si può facilmente accorgere che non
si riesce a definire in modo coerente la lunghezza di ogni sottoinsieme di $[a,b]$.

Per questo si introduce una $\sigma$-algebra più \emph{piccola} di $\mathcal{P}(\Omega)$,
detta degli \textbf{insiemi misurabili} che tratteremo più avanti.

\subsubsection{Serie}
Andiamo ora a capire trattare alcuni concetti di base, relativi alle serie, necessarie ad
inquadrare meglio la situazione.

\begin{definition}
	Sia $a_1, a_2, \dots$ una successione di termini, e sia
	\[ s_n = a_1 + \dots + a_n \]
	la somma parziale dei primi $n$ termini. Definiamo quest'ultima come \textbf{serie} se esiste
	il limite delle somme parziali
	\[ \sum_{n=1}^\infty a_n = \lim_{n \to \infty} \sum_{k=1}^n a_k = \lim_{n \to \infty} s_n \]
	Se esiste finito questo limite si dice che la serie \textbf{converge}.
\end{definition}

\begin{observation}
	Da qui è facile osservare che
	\[ a_n = s_n - s_{n-1} \]
	ed è chiaro che se esiste il limite
	\[ \lim_{n \to \infty} s_n \]
	il valore $a_n$ tende a zero per $n \to \infty$ ma non è vero il viceversa.
\end{observation}

Supponiamo ora il caso in cui $a_n \geq 0$ per ogni $n \in \N$ allora le somme parziali crescono
\[ s_n \leq s_{n+1} \]
In questo caso ha sempre senso parlare di limite poiché, per una successione del genere, vale
\[ \sum_{n=1}^\infty = \lim_{n \to \infty} \sum_{k=1}^n a_k \in [0, +\infty]  \]
Il fatto che esista sempre il limite non significa che la serie converga. Come detto poco fa, la
serie converge se il limite tende ad un numero reale finito.

\begin{definition}
	Se il limite
	\[ \lim_{n \to \infty} \sum_{n=1}^\infty |a_n| \]
	tende ad un numero reale finito si dice che la serie converge \emph{assolutamente}.
\end{definition}

\begin{theorem}
	Se una serie converge assolutamente allora converge.
\end{theorem}

Per le serie di termini positivi che convergono assolutamente vale anche una sorta di proprietà
\textbf{associativa} in cui si partizionano gli indici della serie in sottoinsiemi di $\N$
($A_1, A_2, \dots$) e vale
\[ \sum_{n=1}^\infty a_n = \sum_{n=1}^\infty \left( \sum_{k \in A_n} a_k \right) \]

Consideriamo ora due serie fondamentali di cui non facciamo la dimostrazione ma che ci saranno
molto utili in futuro:
\begin{itemize}
	\item \textbf{Serie geometrica}: se $|a| < 1$ allora vale
	      \[ \sum_{n=0}^\infty a^n = 1 + a + a^2 + \dots = \frac{1}{1 - a} \]
	\item \textbf{Sviluppo in serie dell'esponenziale}
	      \[ e^x = \sum_{n=0}^\infty \frac{x^n}{n!} \]
\end{itemize}

\section{Spazi di probabilità}
Possiamo vedere tutti gli eventi possibili come un insieme, detto \textbf{spazio campionario} o
\textbf{spazio di probabilità}, che indicheremo con $\Omega$.
\begin{itemize}
	\item $\Omega$ rappresenta lo \textbf{spazio campionario} di tutti i possibili eventi.
	\item I sottoinsiemi di $\Omega$ sono detti \textbf{eventi}.
	\item I singoli esiti appartenenti ad un insieme sono detti \textbf{eventi elementari}
\end{itemize}

\begin{example}
	Se lanciamo un dado a 6 facce abbiamo
	\[ \Omega = \{ 1, 2, 3, 4, 5, 6 \} \]
	L'insieme degli eventi tali che il numero uscito sia
	\begin{itemize}
		\item Pari è $A = \{ 2, 4, 6 \}$
		\item Maggiore di 3 è $B = \{ 4, 5, 6 \}$
		\item Non pari è $A^c = \{ 1, 3, 5 \}$
		\item Pari o maggiore di 3 è $A \cup B = \{ 2, 4, 5, 6 \}$
		\item Pari e maggiore di 3 è $A \cap B = \{ 4, 6 \}$
	\end{itemize}
\end{example}

\begin{definition}
	Quando c'è un insieme di eventi $\A$ con queste proprietà:
	\begin{itemize}
		\item $\emptyset, \Omega \in \A$
		\item Se $A \in \A$ allora $A^c \in \A$
		\item Se $A, B \in \A$ allora $(A \cup B) \in \A$ e $(A \cap B) \in \A$
	\end{itemize}
	questo è chiamato \textbf{algebra di parti}.
\end{definition}

\section{Funzione di probabilità}
\begin{definition}[Provvisoria]
	Definiamo la \textbf{probabilità} come la funzione
	\[ P : \A \to [0, 1] \]
	tale che
	\[
		P (\Omega) = 1 \quad \land \quad
		A \cap B = \emptyset \Rightarrow P(A \cup B) = P(A) + P(B)
	\]
	Una funzione con queste proprietà è detta \textbf{finitamente additiva}.
\end{definition}

La probabilità, così come l'abbiamo definita, comporta che
\begin{itemize}
	\item $P(\emptyset) = 0$
	\item $P(A^c) = 1 - P(A)$
	\item $P(A \backslash B) = P(A) - P(B)$
	\item $P(A \cup B) = P(A) + P(B) - P(A \cap B)$
\end{itemize}

\subsection{Andare al limite}
Introduciamo ora due definizioni che possono sembrare controintuitive ma che saranno più chiare andando
avanti.

\begin{definition}
	Se $P(A) = 0$ si dice che $A$ è un evento \textbf{trascurabile}.
\end{definition}

\begin{definition}
	Se $P(A) = 1$ si dice che $A$ è un evento \textbf{quasi certo}.
\end{definition}

Per riuscire a capire perché a questi eventi viene dato questo nome dobbiamo introdurre il concetto di
\textbf{additività numerabile} o $\sigma$-\textbf{additività}, necessario per \emph{andare al limite},
senza il quale non si potrebbe fare calcolo statistico.

\begin{theorem}
	Consideriamo una successione di eventi a due a due disgiunti del tipo
	\[ A_1, \; A_2, \; A_3, \; \dots \]
	con $A_i \cap A_j = \emptyset$ se $i \neq j$ allora vale
	\[
		P(\cup_{n=1}^\infty A_n) = \sum_{n=1}^\infty P(A_n) =
		\lim_{n \to \infty} \sum_{k=1}^n P(A_k) = 	\lim_{n \to \infty} P(A_1) + \dots + P(A_n)
	\]
	Questa è detta \textbf{somma infinita}.
\end{theorem}

Siamo ora in grado di dare una nuova definizione di probabilità, la quale ci permette, come detto poco fa,
di \emph{andare al limite}.
\begin{definition}
	Definiamo la \textbf{probabilità} come la funzione $P$ definita sugli eventi
	\[ P : eventi \to [0, 1] \]
	tale che
	\[
		P(\Omega) = 1 \quad \land \quad
		P(\cup_{n=1}^\infty A_n) = \sum_{n=1}^\infty P(A_n)
	\]
\end{definition}

Consideriamo la successione di eventi
\[ A_1 \subseteq A_2 \subseteq A_3 \subseteq \dots \subseteq A_\infty \]
abbiamo che l'evento $A$, definito come l'unione infinita della successione, equivale a
\[ A = \cup_{n=1}^\infty A_n = \lim_{n \to \infty} A_n \]
La probabilità che $A$ si verifichi equivale quindi a
\[ P(A) = \lim_{n \to \infty} P(A_n) \]

\subsubsection{Spazio degli eventi finito}
Il punto di partenza è il caso in cui $\Omega$ è finito
\[ \Omega = \{ \omega_1, \dots, \omega_n \} \]
e dunque possiamo prendere come probabilità la funzione finitamente additiva definita su tutto lo spazio
degli eventi tale che $P(\Omega) = 1$. In particolare, quando $\Omega$ è finito allora la probabilità si
può sempre definire su tutti i suoi sottoinsiemi, come la somma della probabilità dei punti che lo
costituiscono
\[ P(A) = \sum_{\omega_i \in A} P(\omega_i) = \sum_{\omega_i \in A} p_i \]

\subsubsection{Spazio degli eventi infinito}
Se invece $\Omega$ è infinito ma numerabile, per esempio $\Omega = \N$. Quando $\Omega$ è numerabile allora
la probabilità si può definire su tutti i sottoinsiemi di $\Omega$ e basta conoscere la probabilità dei
singoli punti $p_i$
\[ p_i = P(\omega_i) \]
e deve valere
\[ \sum_{i=1}^\infty p_i = 1 \]
mentre se consideriamo un sottoinsieme $A$ di $\Omega$ allora vale
\[ P(A) = \sum_{\omega_i \in A} p_i \]

\subsection{Serie}
Andiamo ora a capire trattare alcuni concetti di base, relativi alle serie, necessarie ad inquadrare meglio
la situazione.

\begin{definition}
	Sia $a_1, a_2, \dots$ una successione di termini, e sia
	\[ s_n = a_1 + \dots + a_n \]
	la somma parziale dei primi $n$ termini. Definiamo quest'ultima come \textbf{serie} se esiste il limite
	delle somme parziali
	\[ \sum_{n=1}^\infty a_n = \lim_{n \to \infty} \sum_{k=1}^n a_k = \lim_{n \to \infty} s_n \]
	Se esiste finito questo limite si dice che la serie \textbf{converge}.
\end{definition}

\begin{observation}
	Da qui è facile osservare che
	\[ a_n = s_n - s_{n-1} \]
	ed è chiaro che se esiste il limite
	\[ \lim_{n \to \infty} s_n \]
	il valore $a_n$ tende a zero per $n \to \infty$ ma non è vero il viceversa.
\end{observation}

Supponiamo ora il caso in cui $a_n \geq 0$ per ogni $n \in \N$ allora le somme parziali crescono
\[ s_n \leq s_{n+1} \]
In questo caso ha sempre senso parlare di limite poiché, per una successione del genere, vale
\[ \sum_{n=1}^\infty = \lim_{n \to \infty} \sum_{k=1}^n a_k \in [0, +\infty]  \]
Il fatto che esista sempre il limite non significa che la serie converga. Come detto poco fa, la serie
converge se il limite tende ad un numero reale finito.

\subsubsection{Serie assolutamente convergenti}
\begin{definition}
	Consideriamo la serie
	\[ \sum_{n=1}^\infty |a_n| \]
	Se il limite per $n \to \infty$ tende ad un numero reale finito si dice che la serie converge
	\emph{assolutamente}.
\end{definition}

\begin{theorem}
	Se una serie converge assolutamente allora converge.
\end{theorem}

Per le serie di termini positivi che convergono assolutamente vale anche una sorta di proprietà
\textbf{associativa} in cui si partizionano gli indici della serie in sottoinsiemi di $\N$
($A_1, A_2, \dots$) e vale
\[ \sum_{n=1}^\infty a_n = \sum_{n=1}^\infty \left( \sum_{k \in A_n} a_k \right) \]

\subsubsection{Serie fondamentali}
Consideriamo ora due serie fondamentali di cui non facciamo la dimostrazione ma che ci saranno molto
utili in futuro:
\begin{itemize}
	\item \textbf{Serie geometrica}: se $|a| < 1$ allora vale
	      \[ \sum_{n=0}^\infty a^n = 1 + a + a^2 + \dots = \frac{1}{1 - a} \]
	\item \textbf{Sviluppo in serie dell'esponenziale}
	      \[ e^x = \sum_{n=0}^\infty \frac{x^n}{n!} \]
\end{itemize}

\chapter{Calcolo combinatorio}
Il calcolo combinatorio è un elemento fondamentale alla base del calcolo delle probabilità. Non ci
soffermiamo più di tanto su questo aspetto ma cerchiamo di fare nostro qualche concetto utile.

\section{Distribuzione uniforme}
Se $\Omega$ è finito e tutti i suoi punti sembrano ragionevolmente equiprobabili, allora vale
\[ P(\omega_i) = \frac{1}{n} \]
Questa è chiamata \textbf{distribuzione uniforme} di probabilità e vale
\[ P(A) = \frac{\# A}{\# \Omega} = \frac{\text{casi favorevoli}}{\text{casi possibili}} \]
Se $\Omega$ è infinito non possiamo parlare di distribuzione uniforme.

\section{Calcolo combinatorio}
A questo punto è necessario introdurre alcuni concetti fondamentali necessari a comprendere meglio
ciò che andremo a fare più avanti. Il calcolo combinatorio si occupa di proporre dei metodi per
raggruppare gruppi di elementi e, per ogni metodo, vuole contare il numero di possibili
raggruppamenti.

I metodi che andiamo a trattare possono tenere conto dell'ordine con il quale gli elementi sono
raggruppati oppure no. Ogni metodo ha una sua versione in cui si tiene conto delle possibili 
ripetizioni degli elementi.

\subsection{Combinazioni}
Le \textbf{combinazioni} contano il numero di sottoinsiemi di $k$ elementi estratti da un insieme di
$n$ elementi senza tener conto dell'ordine.

\begin{definition}
	Definiamo la \textbf{combinazione senza ripetizione} come il numero di sottoinsiemi di $k$ elementi,
	ognuno estratto una sola volta, da un insieme di $n$ elementi.
	\[ \binom{n}{k} = \frac{n!}{k! (n-k)!} = \frac{n (n-1) \dots (n-k+1)}{k!} \]
	La notazione $\binom{n}{k}$ indica il \textbf{coefficiente binomiale}.
\end{definition}

\begin{definition}
	Definiamo la \textbf{combinazione con ripetizione} come il numero di sottoinsiemi di $k$ elementi,
	dove ogni elemento può essere estratto $k$ volte da un insieme di $n$ elementi.
	\[ \binom{n+k-1}{k} = \frac{(n + k - 1)!}{k! (n-1)!} \]
\end{definition}

\subsubsection{Binomio di Newton}
Uno strumento interessante di cui non facciamo la dimostrazione ma che ci sarà utile più avanti è il
\textbf{binomio di Newton}.

\begin{definition}
	Definiamo il \textbf{binomio di Newton} come
	\[ (a + b)^n = \sum_{k=0}^n \binom{n}{k} \cdot a^k b^{n-k} \]
\end{definition}

\subsection{Disposizioni}
Le \textbf{disposizioni} contano il numero di possibili sequenze ordinate di $k$ elementi estratte da 
un insieme di $n$ elementi.

\begin{definition}
	Definiamo la \textbf{disposizione senza ripetizione} come il numero di sequenze ordinate di $k$
	elementi, estratta da un insieme di $n$ elementi. In questo caso i $k$ elementi sono tutti diversi
	ma vengono contati tutti i possibili modi di riordinare la stessa sequenza.
	\[ \frac{n!}{(n - k)!} \]
\end{definition}

\begin{definition}
	Definiamo la \textbf{disposizione con ripetizione} come il numero di sequenze ordinate di $k$
	elementi, estratta da un insieme di $n$ elementi. In questo caso i $k$ elementi possono ripetersi
	fino a $k$ volte
	\[ n^k \]
\end{definition}

\subsection{Permutazioni}
Le \textbf{permutazioni} contano in quanti possibili modi si può riordinare un insieme di $n$ elementi.

\begin{definition}
	Definiamo la \textbf{permutazione senza ripetizione} come il numero di modi in cui si possono ordinare
	$n$ elementi tutti diversi.
	\[ n! \]
\end{definition}

\begin{definition}
	Definiamo la \textbf{permutazione con ripetizione} come il numero di modi in cui è possibile riordinare
	$n$ elementi nel caso in cui $k$ di essi siano ripetuti.
	\[ \frac{n!}{n_1! \cdot n_2! \cdot ... \cdot n_k!} \]
	In questa formula $n_i$ è il numero di volte che l'elemento $i$ è ripetuto all'interno della sequenza.
\end{definition}


\section{Probabilità condizionata}
La \textbf{probabilità condizionata} indica la probabilità che si verfichi un evento sapendo che
se ne verifica un altro.

\begin{example}
	Supponiamo di lanciare un dado equilibrato e supponiamo di sapere che l'esito è un numero
	pari. Sapendo ciò, vogliamo sapere quale sia la probabilità che l'esito sia maggiore o uguale
	di 4. Consideriamo i due insiemi $A$ e $B$ così definiti.
	\[ A = \{ 4, 5, 6 \} \quad B = \{ 2, 4, 6 \} \]
	Vogliamo sapere qual'è la probabilità di estrarre dall'insieme $B$ un numero maggiore o uguale
	di 4. \`E immediato che la risposta debba essere $2/3$ dato che
	\[
		\frac{2}{3} = \frac{\# (A \cap B)}{\# B} =
		\frac{\# (A \cap B) / \# \Omega}{\# B / \# \Omega} =
		\frac{P(A \cap B)}{P(B)}
	\]
\end{example}

\begin{definition}
	Dato uno spazio di probabilità $(\Omega, \F, P)$ e un evento non trascurabile $B$, definiamo
	\textbf{probabilità condizionata} di $A$ dato $B$, come
	\[ P(A | B) = \frac{P(A \cap B)}{P(B)} \]
	Essa indica la probabilità che accada $A$ sapendo che accade $B$.
\end{definition}

Dalla formula segue che se $A$ e $B$ sono eventi e $B$ è non trascurabile, allora
\[ P(A \cap B) = P(A | B) \cdot P(B) \]
e vale una formula più generale data dalla seguente proposizione.

\begin{proposition}
	Siano $A_1, A_2, \dots, A_n$ eventi la cui intersezione è non trascurabile, allora
	\[
		P(A_1 \cap  \dots \cap A_n) = P(A_1) P(A_2 | A_1) P(A_3 | A_1 \cap A_2)
		\dots P(A_n | A_1 \cap \dots \cap A_{n-1})
	\]
\end{proposition}

\begin{definition}
	Siano $B_1, \dots, B_n$ eventi non trascurabili che formano una partizione dello spazio
	campionario $\Omega$, se
	\[ B_i \cap B_j = \emptyset \quad \land \quad B_1 \cup \dots \cup B_n = \Omega \]
	per ogni $i,j$ con $i \neq j$, allora sono detti \textbf{sistema di alternative}.
\end{definition}

\begin{theorem}[Fattorizzazione]\label{th: fattorizzazione}
	Sia $B_1, \dots, B_n$ un sistema di alternative, allora vale
	\[ P(A) = \sum_{i=1}^n P(A | B_i) \cdot P(B_i) \]
	Per ogni evento $A \in \F$.
	\begin{proof}
		Se partizioniamo $\Omega$ in $n$ alternative e poi definiamo un insieme
		$A \subseteq \Omega$, possiamo vedere $A$ in questo modo
		\[ A = \cup_{i=1}^n (A \cap B_i) \]
		ossia come l'unione disgiunta dei pezzi di $A$ che stanno nei vari $B_i$. Quindi vale che
		\[ P(A) = \sum_{i=1}^n P(A \cap B_i) = \sum_{i=1}^n P(A | B_i) P(B_i) \]
	\end{proof}
\end{theorem}

Generalmente questa formula si applica in casi in cui $P$ non è nota a priori ma sono note le
probabilità condizionate a un sistema di alternative.

\begin{example}
	Ci sono 2 urne, la prima con 5 biglie rosse e 5 blu, la seconda con 8 biglie rosse e 2 blu.
	Scegliamo casualmente una delle due urne e da questa estraiamo una biglia e ne osserviamo il
	colore. Vogliamo calcolare la probabilità che esca una biglia rossa. In questo esempio lo
	spazio campionario è il seguente
	\[ \Omega = \{ (\text{urna}, \text{colore biglia}) \} \]
	dove
	\[ \text{urna} \in \{ 1, 2 \} \quad \text{colore biglia} \in \{ r, b \} \]
	A questo punto noi sappiamo che la probabilità di scegliere una delle due urne è
	\[ P(1) = P(2) = \frac{1}{2} \]
	La seconda informazione di cui siamo a conoscienza è che se scegliamo l'urna 1 abbiamo 5
	biglie rosse e 5 blu e quindi
	\[ P(r | 1) = \frac{5}{10} = \frac{1}{2} \]
	Se invece scegliamo l'urna 2 abbiamo 8 biglie rosse e quindi
	\[ P(r | 2) = \frac{8}{10} = \frac{4}{5} \]
	A questo punto siamo in grado di calcolare la probabilità complessiva di estrarre una biglia
	rossa ricavandola con il teorema di fattorizzazione \ref{th: fattorizzazione} in questo modo
	\[ P(r) = P(r | 1) \cdot P(1) + P(r | 2) \cdot P(2) = \frac{13}{20} \]
\end{example}

\begin{example}
	Un test diagnostico per una data malattia ha un indice di sensibilità di $0.99$ per una
	persona malata, ossia, da esito positivo con probabilità $0.99$. E ha indice di specificità
	$0.97$, ossia, da esito negativo con probabilità $0.97$ per una persona sana.

	Supponendo che l'1\% della popolazione soffra di tale malattia, vogliamo calcolare la
	probabilità che il test, su un individuo a caso, dia esito positivo. Si applica la formula di
	fattorizzazione al sistema di alternative $B_1 = \{ \text{sano} \}$ e
	$B_2 = \{ \text{malato} \}$ e all'evento $A = \{ \text{test positivo} \}$. Svolgendo i calcoli
	in modo analogo a prima otteniamo che
	\[ P(A) = 0.0396 \]
\end{example}

\begin{theorem}[Bayes]\label{th: bayes}
	Siano $A$ e $B$ eventi non trascurabili. Allora vale
	\[ P(B | A) = \frac{P(A | B) \cdot P(B)}{P(A)} \]
	In particolare se $B_1, B_2, ..., B_n$ allora vale
	\[
		P(B_i | A) = \frac{P(A | B_i) \cdot P(B_i)}{\displaystyle\sum_{j=1}^n P(A | B_j)
			\cdot P(B_j)}
	\]
	per ogni $i = 1, \dots, n$.
	\begin{proof}
		La dimostrazione è molto semplice basta infatti notare che
		\[
			P(B | A) = \frac{P(B \cap A)}{P(A)} =
			\frac{P(A \cap B)}{P(A)} = \frac{P(A | B) \cdot P(B)}{P(A)}
		\]
		La seconda formula segue dalla prima dove $B=B_i$ e dalla formula di fattorizzazione
		per $P(A)$.
	\end{proof}
\end{theorem}

La formula di Bayes viene usata per \emph{invertire} il condizionamento tipicamente in contesti
in cui accade un evento $A$ riferito a un'\emph{osservabile} e vogliamo dedurre la probabilità di
un'alternativa $B_i$ di eventi \emph{causa}.

\begin{example}
	Facendo riferimento all'esempio di prima delle biglie, supponiamo di estrarre una biglia
	rossa e vogliamo calcolare la probabilità che questa sia stata estratta dalla prima urna.
	\[
		P(1 | r) = \frac{P(r | 1) P(1)}{P(r)} =
		\frac{\frac{5}{10} \cdot \frac{1}{2}}{\frac{13}{20}} = 0.385
	\]
\end{example}

\begin{example}
	Facendo riferimento al test diagnostico, supponiamo che il test dia esito positivo e vogliamo
	calcolare la probabilità che la persona sia effettivamente malata.
	\[
		P(\text{malato} | \text{positivo}) =
		\frac{P(\text{positivo} | \text{malato}) P(\text{malato})}{P(\text{positivo})} = 0.25
	\]
\end{example}

\section{Indipendenza}
Il concetto di \textbf{indipendenza} nasce dalla necessità di esprimere in modo rigoroso che la
probabilità di un evento $A$ non cambia sapendo che accade $B$ e viceversa. Per $A$ e $B$ non
trascurabili vale quindi
\[ P(A) = P(A | B) \quad \Leftrightarrow \quad P(A) P(B) = P(A \cap B) \]
Lo stesso vale per $P(B)$.

\begin{definition}
	Due eventi $A$ e $B$ sono \textbf{indipendenti} se
	\[ P(A \cap B) = P(A) P(B) \]
\end{definition}

Si può dimostrare per esercizio che, se $A$ e $B$ sono indipendenti allora lo sono anche
\begin{itemize}
	\item $A^c$ e $B$
	\item $A$ e $B^c$
	\item $A^c$ e $B^c$
\end{itemize}
L'indipendenza è dunque stabile per la complementazione. Si può inoltre dimostrare che
\begin{itemize}
	\item Se $P(A) \in \{ 0, 1 \}$ allora $A$ è indipendente da ogni altro evento.
	\item Se $A \cap B = \emptyset$ allora $A$ e $B$ non sono indipendente a meno che $P(A)$ o
	      $P(B)$ siano 0.
\end{itemize}

\begin{example}
	Si vuole estrarre una carta da un mazzo di 40 carte napoletane
	\[ \Omega = \{ \text{carte} \} \]
	Ci chiediamo se i due eventi
	\[ A = \{ \text{asso} \} \quad B = \{ \text{denari} \} \]
	sono indipendenti. La probabilità di estrarre un asso equivale a
	\[ P(A) = \frac{4}{40} = \frac{1}{10} \]
	La probabilità di estrarre una carta di denari è invece
	\[ P(B) = \frac{10}{40} = \frac{1}{4} \]
	La probabilità di estrarre l'asso di denari equivale a
	\[ P(A \cap B) = \frac{1}{40} = \frac{1}{10} \cdot \frac{1}{4} = P(A) P(B) \]
	Quindi gli eventi sono indipendenti.
\end{example}

\subsection{Indipendenza per 3 o più eventi}
Siano $A$, $B$ e $C$ degli eventi. Per riuscire a dire che sono indipendenti c'è bisogno che
\begin{itemize}
	\item Siano a due a due indipendenti
	      \begin{align*}
		      P(A \cap B) = & P(A) P(B) \\
		      P(A \cap C) = & P(A) P(C) \\
		      P(B \cap C) = & P(B) P(C)
	      \end{align*}
	\item La probabilità dell'intersezione di tutti e tre si spezzi come il prodotto delle
	      probabilità dei tre eventi.
	      \[ P(A \cap B \cap C) = P(A) P(B) P(C) \]
\end{itemize}

\begin{example}
	Consideriamo
	\begin{align*}
		\Omega = & \{ 1, 2, 3, 4 \} \\
		A =      & \{ 1, 2 \}       \\
		B =      & \{ 1, 3 \}       \\
		C =      & \{ 2, 3 \}
	\end{align*}
	con $P$ uniforme su $\Omega$. La probabilità che i singoli eventi si verifichino è di
	\[ P(A) = P(B) = P(C) = \frac{2}{4} = \frac{1}{2} \]
	La probabilità dell'intersezione degli eventi presi a due a due è di
	\[ P(A \cap B) = P(A \cap C) = P(B \cap C) = \frac{1}{4} \]
	Dunque la prima condizione per l'indipendenza è soddisfatta. Andiamo a calcolare ora quanto
	vale l'intersezione dei tre eventi. Consideriamo l'intersezione in questo modo
	\[ A \cap (B \cap C) \]
	Dunque possiamo calcolare la probabilità dell'intersezione in questo modo
	\[ P(A \cap (B \cap C)) = P(A | (B \cap C)) = 0 \neq P(A) \]
	Dunque sapere che accade $B \cap C$ cambia la probabilità che accada $A$.
\end{example}

\begin{definition}
	Dati $A_1, A_2, ..., A_n$ eventi, questi si dicono indipendenti se $\forall k$ intero con
	$1 \leq k \leq n$ e $\forall \; 1 \leq i_1 < ... < i_k \leq n$ vale
	\[ P(A_{i_1} \cap ... \cap A_{i_k}) = P(A_{i_1}) \cdot ... \cdot P(A_{i_k}) \]
	In altre parole la probabilità dell'intersezione degli eventi si deve poter spezzare come
	prodotto delle probabilità dei singoli eventi.
\end{definition}

\begin{observation}
	Se $A_1, A_2, ..., A_n$ sono indipendenti allora sono indipendenti a due a due.
\end{observation}

\subsubsection{Prove ripetute}
Un caso importante di eventi indipendenti è il caso delle \textbf{prove ripetute} in cui si ripete
un esperimento $n$ volte nelle medesime condizioni. Possiamo dire che per gli eventi riferiti a
ripetizioni distinte o a gruppi disgiunti di ripetizione sono indipendenti.

Un sottocaso importante delle prove ripetute è lo schema di Bernoulli per $n$ prove ripetute in
cui ciascuna prova ha esito \emph{successo} o \emph{insuccesso}. In questo specifico caso è
possibile scrivere lo spazio campionario
\[ \Omega = \{ a_1, a_2, ..., a_n | \forall i \; a_i \in \{ 0, 1 \} \} = \{ 0, 1 \}^n \]
In cui associamo tipicamente a 1 il successo e a 0 l'insuccesso. La probabilità associata ad una
sequenza equivale a
\[ P(\{ a_1, ... a_n \}) = p^{\# \{i | a_i = 1\}} \cdot (1 - p)^{\# \{i | a_i = 0\}}  \]
dove $p$ è la probabilità di successo della singola prova.

\begin{observation}
	Il concetto di indipendenza non è legato al concetto di causalità.
\end{observation}

\section{Probabilità sulla retta reale}
In questo paragrafo vedremo due classi principali di esempi di probabilità su $\Omega = \R$.
Queste due classi non esauriscono i possibili esempi di probabilità su $\R$ ma saranno l'oggetto
delle applicazioni che vedremo.

\subsection{Probabilità discrete}
Una \textbf{probabilità discreta} su $\Omega = \R$ è una probabilità su $\Omega = \R$, prendendo
come $\sigma$-algebra le parti di $\R$ ($\F (P(\R))$), che sia concentrata su una successione
finita o numerabile $x_1, x_2, \dots$ di punti.

Ad esempio la probabilità associata al numero di volte che esce testa in due lanci di moneta è
una probabilità discreta, concentrata su $\{0,1,2\}$. Chiamiamo $p_i = p(x_i) = P(x_i)$, allora
vale che
\begin{equation}\label{eq: 3.1} P(A) = \sum_{x_i \in A} p(x_i) \end{equation}

\begin{definition}
	La funzione
	\[ p : \{ x_1, x_2, \dots \} \to \R \]
	con $p(x_i) = P(\{x_i\})$ si dice \textbf{funzione di massa} o \textbf{densità discreta} di $P$.
\end{definition}

\begin{observation}
	Facciamo ora alcune osservazioni:
	\begin{itemize}
		\item $p$ soddisfa
		      \[ p(x_i) \geq 0 \quad \land \quad \sum_i p(x_i) = P(\R) = 1\]
		\item Viceversa, data una successione $x_1, x_2, \dots$ finita o numerabile e una funzione
		      $p$ con le proprietà descritte sopra, allora esiste un'unica probabilità $P$ su $\R$
		      avente $p$ come funzione di massa.
		\item Dalla formula \ref{eq: 3.1}, è possibile calcolare $P(A)$ partendo dalla successione
		      degli $x_i$ e dalla funzione di massa.
		\item Si definisce $p$ su tutto $\R$, ponendo $p(x) = 0$ per $x \neq x_i$.
	\end{itemize}
\end{observation}

\subsection{Probabilità con densità}
Prima di parlare di questa classe di probabilità dobbiamo enunciare il seguente teorema:
\begin{theorem}
	Esiste una $\sigma$-algebra $\F \subseteq P(\R) \neq P(\R)$ tale che
	\[ (a, b) \in \F \]
	tali che $-\infty \leq a < b \leq +\infty$ ed esiste $\lambda : \F \to [0, +\infty]$ che sia
	$\sigma$-additiva e tale che, per ogni $-\infty \leq a < b \leq +\infty$,
	\[ \lambda((a,b)) = \lambda((a,b]) = \lambda ([a,b)) = \lambda([a,b]) = b - a \]
\end{theorem}

Chiameremo gli elementi di $\F$ \textbf{insiemi misurabili}. Intersecando tutti gli elementi di
$\F$ con l'intervallo $[0,1]$ otteniamo i sottoinsiemi misurabili di quest'ultimo, che sono una
$\sigma$-algebra sulla quale possiamo definire la probabilità
\[ P (A) = \lambda (A) \]
per ogni $A \subset [0,1]$ misurabile, in modo che
\[ P((a, b)) = |b-a|, \quad \forall a,b \in [0,1] \]

\begin{observation}
	Osserviamo che $\lambda$ assegna lunghezza nulla ai singoli punti, ovvero
	\[ \lambda(x) = \lambda([x,x]) = x - x = 0 \]
	e lo stesso vale per ogni sottoinsieme $A \subset \R$ con al più numerabili elementi.
\end{observation}

\begin{definition}
	Si chiama \textbf{densità di probabilità} sulla retta reale, una funzione non negativa
	$f : \R \to [0, +\infty)$, integrabile e tale che
	\[ \int_{-\infty}^{+\infty} f(x) dx = 1 \]
	Ad ogni densità di probabilità si associa un'unica probabilità sulla $\sigma$-algebra degli
	insiemi misurabili di $\R$ definita da:
	\[ \forall A \in \F, \quad P(A) = \int_A f(x) dx \]
\end{definition}

Perché questa sia una buona definizione si deve controllare che $P$, così definita, soddisfi
\[ P(\R) = \int_\R f(x) dx = 1 \]
per ipotesi, ed inoltre se $A \cap B = \emptyset$ si ha che
\[ P(A \cup B) = \int_{A \cup B} f(x) dx = \int_A f(x) dx + \int_B f(x) dx = P(A) + P(B) \]
Se $X$ ha densità $f$, allora vale
\[ P(X = t) = P_X (t) = \int_{\{t\}} f(x) dx = 0 \]
per ogni $t \in \R$.

\chapter{Variabili aleatorie}
Le variabili aleatorie sono, di fatto, \textbf{caratteristiche quantitative} dell'esperimento
preso in esame. Non si tratta di altro che di una notazione per descrivere gli eventi ma sarà
chiaro più avanti il loro significato e la loro utilità.

\begin{example}
	Si effettuano $n$ lanci di una moneta equilibrata ed essere interessati solo alle volte che
	esce testa. Per riuscire a farlo potremmo assegnare a testa il valore 1 e a croce il valore 0.
	Di conseguenza otteniamo che il numero di volte che esce croce è
	\[ X(a_1, a_2, \dots, a_n) = \sum_{i=1}^n a_i \]
	dove gli $a_i$ sono gli esisti di ogni lancio.
\end{example}

\section{Legge di una variabile aleatoria}
Iniziamo ora a dare qualche definizione per formalizzare meglio il concetto di variabile aleatoria
e perché sono state introdotte.

\begin{definition}
	Dato uno spazio di probabilità $(\Omega, \F, P)$, si dice \textbf{variabile aleatoria} una
	funzione
	\[ X : \Omega \to \R \]
	tale che, per ogni $A \subseteq \R$ misurabile, allora vale
	\[ X^{-1}(A) \in \F \]
\end{definition}

\begin{example}
	Si effettuano $n$ lanci di una moneta equilibrata e otteniamo:
	\[ A = (a_1, \dots, a_n) \]
	Vogliamo ora considerare
	\begin{itemize}
		\item numero di teste: $X(A) = \sum_{i=1}^n a_i$
		\item numero di croci: $Y(A) = n - X(A)$
		\item coppie consecutive testa: $Z(A) = \sum_{i=1}^{n-1} a_i \cdot a_{i+1}$
	\end{itemize}
\end{example}

\begin{definition}
	Per ogni $A \subseteq \R$ misurabile, definiamo l'insieme
	\[ \{ X \in A \} = X^{-1}(A) = \{\omega \in \Omega | X(\omega) \in A \} \]
	come la \textbf{controimmagine} di $A$ tramite $X$.
\end{definition}

In altre parole possiamo dire che $P_X$ esprime come il carattere $X$ è distribuito
nell'esperimento.

\begin{example}
	Si effettuano $n$ lanci di moneta e consideriamo $X$ come il numero di volte che esce testa.
	\[ X(A) = a_1 + \dots + a_n \]
	L'insieme $\{ X \in \{ 0, 1 \} \}$ è l'insieme di casi in cui, per $n$ lanci di moneta,
	otteniamo al massimo una testa. La probabilità di avere al massimo una testa
	la esprimiamo come
	\[ P_X (\{0,1\}) = P (X \in \{0,1\}) \]
\end{example}

\begin{proposition}
	Dalle considerazioni fatte segue che $P_X : \F \to \R$ è una probabilità su $(\R, \F)$, dove
	$\F$ è l'insieme dei misurabili e verifica quindi
	\begin{itemize}
		\item $P_X (\R) = 1$
		\item $P_X(\cup_{i=1} A_i) = \sum_{i=1} P_X(A_i)$ con $A_1, A_2, ... \subseteq \R$
		      misurabili a due a due disgiunti.
	\end{itemize}
\end{proposition}

\begin{definition}
	Definiamo \textbf{legge di probabilità} o \textbf{distribuzione} di $X$, come
	\[ P_X (A) = P(X \in A) \]
	con $A \subseteq \R$ misurabile.
\end{definition}

\begin{example}
	Si effettuano 2 lanci di una moneta equilibrata ($\Omega = \{ 0, 1 \}^2$) e definiamo $X$
	come il numero di volte che esce testa. Quanto vale $P_X$? Come possiamo notare, $P_X$ è
	concentrata su $\{ 0, 1, 2 \}$.
	\[
		P_X (0) = \frac{1}{4} \quad \quad
		P_X (1) = \frac{1}{2} \quad \quad
		P_X (2) = \frac{1}{4}
	\]
	Possiamo quindi dire che $P_X$ è la distribuzione del carattere $X$ all'interno
	dell'esperimento.
\end{example}

\begin{definition}
	Diciamo che $X$ e $Y$ sono \textbf{equidistribuite} se hanno la stessa legge, ovvero se
	\[ P_X = P_Y \]
\end{definition}

Le variabili aleatorie altro non sono che una notazione, che diventa conveniente nel momento in cui
vogliamo considerare più caratteri di un esperimento (e quindi più variabili aleatorie).

Sono molto utili anche nel momento in cui è necessario svolgere operazioni matematiche tra
variabili aleatorie (ad esempio sommandole).

\begin{observation}
	Data $\tilde{P}$, probabilità su $(\R, \F)$, allora esiste uno spazio $\Omega$ e una variabile
	aleatoria $X : \Omega \to \R$ avente $\tilde{P}$ come legge ($P_X = \tilde{P})$.
\end{observation}

\begin{observation}
	Se c'è un'unica variabile aleatoria $X$, è equivalente considerare $X$ oppure $P_X$.
\end{observation}

\subsection{Tipi di variabili aleatorie}
Possiamo distinguere due classi di variabili aleatorie a seconda della loro legge.

\begin{definition}
	Una variabile aleatoria $X$ è detta \textbf{discreta} se $X(\Omega)$ è finita o numerabile, o
	in modo equivalente, se la sua legge $P_X$ è discreta.
\end{definition}

Per quanto visto sulle probabilità discrete, $P_X$ è determinata dalla sua funzione di massa,
ovvero
\[ P_X (x_i) = P(X = x_i) \]
vale infatti che
\[ P(\{X \in A\}) = P_X (A) = \sum_{x_i \in A} P_X (x_i) \]

\begin{definition}
	Una variabile aleatoria $X$ è detta \textbf{con densità} o \textbf{assolutamente continua} se
	$P_X$ ammette densità di probabilità $f$, ossia
	\[ P(\{X \in A\}) = P_X (A) = \int_A f(x) dx \]
\end{definition}

Teniamo a mente che, in questo caso, $P_X$ non è determinata dalla sua densità ma dall'integrale
della densità. In generale la funzione di densità ci fornisce solo un'idea di come il carattere sia
distribuito nell'esperimento e non ci deve stupire se assume valori anche maggiori di 1 in quanto
la densità \textbf{non} è una probabilità.

\begin{example}
	Consideriamo il caso dell'estrazione di un individuo $\omega$ da una popolazione. Supponiamo
	quindi che $\Omega$ rappresenti la popolazione italiana e che
	\begin{itemize}
		\item $X(\omega)$ equivale al numero di figli di $\omega$ (carattere discreto).
		\item $Y(\omega)$ equivale all'altezza di $\omega$ (carattere continuo).
	\end{itemize}
	Abbiamo che
	\[ P_X(x_i) = P(X = x_i) = \frac{\# \{ w | X(w) = x_i\} }{\# \Omega} \]
	ossia la frequenza relativa di $X = x_i$ su tutta la popolazione. $P_X(2)$ è quindi la
	frequenza relativa, sulla popolazione italiana, delle persone con 2 figli.

	Per quanto riguarda $Y$ invece, l'area sottesa da $f$ nell'intervallo $[a,b]$ equivale a
	\[
		\int_a^b f(x) dx = P (a \leq Y \leq b)
		= \frac{\# \{ \omega | Y(\omega) \in [a,b] \}}{\# \Omega}
	\]
	ossia la frequenza relativa di $Y \in [a,b]$ su tutta la popolazione. Nel nostro esempio, se
	consideriamo l'intervallo $[1.70, 1.80]$, abbiamo che
	\[ \int_{1.70}^{1.80} f(x) dx = P (1.70 \leq Y \leq 1.80) \]
	ossia la frequenza relativa, nella popolazione italiana, delle persone con altezza compresa tra
	1.70 e 1.80.

	Poiché anche nell'istogramma la frequenza relativa di $a \leq Y \leq b$ è l'area dei rettangoli
	tra $a$ e $b$, la curva $f$ rappresenta un'ottima approssimazione di tale istogramma.
\end{example}
\section{Funzione di ripartizione e quantili}
Siano $X : \Omega \to \R$ una variabile aleatoria e $P_X$ la sua legge. Per studiare $P_X$, si
introduce una funzione che codifica tutte informazioni ad essa associate, ossia la funzione di
ripartizione.

Introdurremo anche un valore utile per fare determinati calcoli, detto quantile, direttamente
collegato alla funzione di ripartizione e al suo significato.

\subsection{Funzione di ripartizione}
\begin{definition}
	Si chiama \textbf{funzione di ripartizione} di $X$ la funzione $F_X : \R \to [0,1]$ definita
	come segue
	\[ F_X(x) = P (X \leq x) = P_X ((-\infty, x]) \]
	In altre parole, la funzione di ripartizione di $X$ indica la probabilità che tale variabile
	aleatoria assuma valori inferiori ad un un certo valore $x$.
\end{definition}

La funzione di ripartizione $F_X$ dipende solo dalla legge $P_X$ di $X$. Questo significa che,
avere una variabile aleatoria con la stessa legge di $X$ implica avere anche la stessa funzione
di ripartizione. Vediamo ora alcune proprietà della funzione di ripartizione:
\begin{itemize}
	\item $F_X$ è non decrescente, ossia se $x \leq y$ allora
	      \[ F_X(x) = P(X \leq x) \leq P(X \leq y) = F_X(y) \]
	\item Esistono i limiti
	      \[ \lim_{x \to -\infty} F_X(x) = 0 \quad \quad \lim_{x \to +\infty} F_X(x) = 1 \]
	\item $F$ è continua a destra, ossia se $x_n$ è una successione che converge ad $x$ con
	      $x_n \geq x$, allora $F_X(x_n)$ converge ad $F(x)$.
\end{itemize}

\begin{proposition}
	Data una funzione $F : \R \to [0,1]$ con le proprietà sopra elencate, esiste ed è unica la
	probabilità $Q$ tale che $F$ sia la funzione di ripartizione di $Q$, cioè di una variabile
	aleatoria $X$ di legge $P_X = Q$. Questo equivale a dire che
	\[ F(x) = Q((-\infty, x]) = P(X \leq x) \]
	per ogni $x \in \R$. Di conseguenza due variabili aleatorie $X$ e $Y$ che hanno la stessa
	funzione di ripartizione hanno anche la stessa legge.
\end{proposition}

\begin{theorem}\label{th: diff_cdf}
	Data la variabile aleatoria $X$, la sua legge $P_X$ e la sua funzione di ripartizione $F_X$,
	vale che
	\[ P(a < X \leq b) = F_X (b) - F_X(a) \]
	Questa formula è utile per calcolare $P(a < X \leq b)$ quando si conoscono i valori di $F_X$
	almeno in modo approssimato.
	\begin{proof}
		La dimostrazione è molto semplice, basta notare che
		\begin{align*}
			P(a < X \leq b) = & P(X \leq b) - P(X \leq a) \\
			=                 & F(b) - F(a)
		\end{align*}
		per come abbiamo definito la funzione di ripartizione.
	\end{proof}
\end{theorem}

\begin{observation}
	Dal teorema appena enunciato segue che
	\begin{itemize}
		\item Per $a = -\infty$ otteniamo $P(X \leq b) = F(b)$
		\item Per $b = +\infty$ otteniamo $P(X > a) = 1 - F(a)$
	\end{itemize}
\end{observation}

La funzione di ripartizione di una variabile aleatoria discreta (che assume valori
$x_1, x_2, \dots$) è una funzione costante a tratti, ossia è costante tra due punti $x_i$ e in ogni
punto $x_i$ esibisce un salto di ampiezza $P_X(x_i) = P(X = x_i)$. Una funzione di ripartizione
si scrive in questo modo
\begin{align*}
	F_X(x) & = P(X \leq x)                              \\
	       & = P_X ((-\infty, x])                       \\
	       & = \sum_{i, x_i \in (-\infty, x]} P_X (x_i) \\
	       & = \sum_{i, x_i \leq x} P_X (x_i)
\end{align*}
Graficamente una generica funzione di ripartizione appare in questo modo
\begin{center}
	\begin{tikzpicture}
		\begin{axis}[
				axis lines = center,
				width = 8cm,
				height = 5cm,
				font = \footnotesize,
				ymin=0, ymax=1,
				xmin=-1, xmax=2,
				xtick={-1, 1, 2},
				ytick={0.5, 1},
				enlargelimits
			]
			\addplot [thick, red, domain={-2 : 0}] {0};
			\addplot [thick, red, domain={0 : 1}] {0.5};
			\addplot [thick, red, domain={1 : 3}] {1};
		\end{axis}
	\end{tikzpicture}
\end{center}
Per quanto riguarda invece le variabili aleatorie con densità $f$, abbiamo che la loro funzione di
ripartizione soddisfa
\[ F_X(x) = P(X \leq x) = \int_{-\infty}^x f(y) dy \]
In particolare $F_X$ è continua (senza salti).

\begin{center}
	\begin{tikzpicture}
		\begin{axis}[
				axis lines = center,
				width = 8cm,
				height = 4cm,
				font = \footnotesize,
				ymin=0, ymax=1,
				xmin=-1, xmax=2,
				xtick={-1, 1, 2},
				ytick={0.5, 1},
				enlargelimits
			]
			\addplot [thick, red, domain={-2 : 0}] {0};
			\addplot [thick, red, domain={0 : 1}] {x};
			\addplot [thick, red, domain={1 : 3}] {1};
		\end{axis}
	\end{tikzpicture}
\end{center}

\begin{observation}
	Notiamo anche che se $f$ è continua a tratti, cioè
	se $F_X$ è di classe $C^1$ a tratti, allora $f$ si ottiene derivando $F_X$
	\[ f(x) = \frac{d}{dx} F_X (x) \]
	per ogni $x$ in cui $F_X$ è derivabile.
\end{observation}

\begin{observation}
	Esistono variabili aleatorie con funzione di ripartizione continua ma che comunque non
	ammettono densità.
\end{observation}

\subsection{Quantili}
Intuitivamente, dato $\beta \in (0,1)$, un $\beta$-quantile è un numero $r_\beta \in \R$ tale che
la probabilità che la variabile aleatoria $X$ che stiamo considerando sia minore di $r_\beta$ è
proprio $\beta$. Vale quindi
\[ F_X (r_\beta) = P(X \leq r_\beta) = \beta \]
Tuttavia può non esistere un tal $\beta$, oppure se esiste, può non essere unico. Dobbiamo quindi
trovare una definizione diversa.

\begin{definition}\label{def: quantile}
	Data una variabile aleatoria $X$ e un $\beta \in (0,1)$, si chiama $\beta$\textbf{-quantile},
	un numero $r_\beta \in \R$ tale che
	\[ P(X \leq r_\beta) \geq \beta \quad \land \quad P(X \geq r_\beta) \geq 1 - \beta \]
	tale definizione dipende solo dalla legge $P_X$.
\end{definition}

Per calcolare il $\beta$-quantile nel caso di $X$ discreta dobbiamo prima ordinare gli $x_i$ e da
qui distinguiamo 2 casi:
\begin{itemize}
	\item Non esite $x_i$ tale che $F_X(x_i) = \beta$. Abbiamo quindi che $r_\beta$ equivale al più
	      piccolo degli $x_i$ tale che $F_X(x_i) \geq \beta$.
	\item Esiste $x_i$ tale che $F_X(x_i) = \beta$. Abbiamo quindi che ogni $r \in [x_i, x_{i+1}]$
	      è un $\beta$ quantile. Per convenzione si prende come quantile il punto medio
	      dell'intervallo.
\end{itemize}
Per calcolare il $\beta$-quantile nel caso di $X$ con densità distinguiamo ancora 2 casi:
\begin{itemize}
	\item $F_\beta$ è strettamente crescente allora esiste ed è unico $r_\beta$ tale che
	      \[ F_X(r_\beta) = P(X \leq r_\beta) = \beta \]
	      e $r_\beta$ è l'unico $\beta$-quantile.
	\item $F_X^{-1} (\beta)$ è un intervallo, ossia $F_\beta$ è costante su un intervallo, allora
	      anche in questo caso, per convenzione si prende come $\beta$-quantile l'estremo sinistro
	      dell'intervallo, ossia
	      \[ r_\beta = \inf \{ r \in \R | F_X(r) \geq \beta \} \]
\end{itemize}
In generale, data una densità $f$, l'area sottesa fino al punto $r_\beta$ vale $\beta$.

\begin{observation}
	Nel contesto dell'esempio dell'estrazione da una popolazione, la funzione di ripartizione è la
	funzione di ripartizione in cui il campione è sostituito da tutta la popolazione.
\end{observation}
\section{Variabili aleatorie notevoli}
In questa sezione trattiamo le distribuzione di probabilità su $\R$ indispensabili per trattare
ogni applicazione. Si tratta sempre di variabili discrete o definite tramite densità.

\subsection{Variabili binomiali}
Consideriamo come primo caso le \textbf{variabili binomiali} prendendo $n$ prove ripetute, con
esito (per ciascuna prova) successo o insuccesso (schema di Bernoulli) e sia $p$ la probabilità di
successo (nella singola prova).
\[ P(a_1, \dots, a_n) = p^{\# \{i | a_i=1\}} \cdot (1-p)^{\# \{ i | a_i=0 \}} \]
Sia $X$ la variabile aleatoria che conta il numero di successi, ossia
\[ X(a_1, \dots, a_n) = \sum_{i=1}^n a_i \]
Come possiamo notare $X$ è discreta a valori in $\{0,1,2,\dots,n\}$ e ha funzione di massa
\[ P_X(h) = P(X = h) = \binom{n}{h} \cdot p^h \cdot (1-p)^{n-h} \]
con $h \in \{ 0, 1, \dots, n \}$. Possiamo tradurre tutto questo nella probabilità che abbiamo di
avere $h$ successi.

Una \textbf{variabile aleatoria binomiale} di parametri $n \in \N^+$ e $p \in (0,1)$, è tale se
possiede la funzione di massa appena descritta ed è indicata con $B(n, p)$.

\begin{observation}
	Per $n=1$ si parla di variabile aleatoria di Bernoulli e si indica con $B(p)$.
\end{observation}

\begin{example}
	Su 5 lanci di un dado equilibrato, qual è la probabilità che il 6 appaia almeno 2 volte? Per
	prima cosa definiamo $X$ come il numero di volte che esce 6 nelle 5 prove. Vogliamo quindi
	calcolare la probabilità che il 6 esca almeno 2 volte in 5 lanci. Per farlo ci conviene passare
	al problema complementare:
	\[ P(X \geq 2) = 1 - (P(X = 0) + P(X = 1)) \]
	Calcoliamo quindi $P(X=0)$ e $P(X=1)$:
	\begin{gather*}
		P(X = 0) = \binom{5}{0} \cdot \left(\frac{1}{6}\right)^0 \cdot
		\left(1 - \frac{1}{6}\right)^5 = \left(\frac{5}{6}\right)^5 \\
		P(X = 1) = \binom{5}{1} \cdot \left(\frac{1}{6}\right)^1 \cdot
		\left(1 - \frac{1}{6}\right)^4 = \left(\frac{5}{6}\right)^4
	\end{gather*}
	Siamo ora in grado di calcolare $P(X \geq 2)$ come
	\[ P(X \geq 2) = 1 - \left( \left(\frac{5}{6}\right)^5 + \left(\frac{5}{6}\right)^4 \right) \]
\end{example}

\subsection{Variabili geometriche}
Rimaniamo nel contesto delle prove ripetute indipendenti con esito successo o insuccesso. Sia $X$
la variabile aleatoria che rappresenta l'istante del primo successo (l'istante è il numero della
prova).

Come possiamo notare, $X$ è discreta a valori in $\N^+ = \{ 1, 2, \dots \}$ e la sua funzione di
massa vale
\[ P(X = h) = (1-p)^{h-1} \cdot p \]
Questo ci dice la probabilità che abbiamo di ottenere il primo successo dopo $h$ tentativi.

Una \textbf{variabile aleatoria geometrica} di parametro $p \in (0,1)$, è tale se possiede la
funzione di massa appena descritta ed è indicata con $G(p)$.

\begin{proposition}[Assenza di memoria]
	Data una variabile geometrica di parametro $p$, per ogni $n,h \in \N^+$, vale
	\[ P(X = n + h | X > n) = P(X = h) \]
	La probabilità di successo dopo $h$ prove non cambia sapendo che il successo non si è
	verificato nelle prime $n$ prove.
\end{proposition}

\subsection{Variabili di Poisson}
Una variabile aleatoria $X$ si dice \textbf{variabile aleatoria di Poisson} di parametro
$\lambda > 0$ se è una variabile aleatoria discreta a valori in $\N$, con funzione di massa
\[ P(X = h) = \frac{\lambda^h}{h!} \cdot e^{-\lambda} \]
con $h \in \N$ e si indica con $P(\lambda)$.

La variabile di Poisson conta il numero di \emph{eventi rari}, infatti, tale variabile aleatoria,
approssima una binomiale di parametri $n$ e $p$ quando $n$ è grande, $p$ è piccola e
$\lambda \simeq np$. Si può dimostrare infatti che, detta
\[ p^{(n)} (h) = \binom{n}{h} \cdot p_n^h \cdot \left( 1-p_n \right)^{n-h} \]
la funzione di massa di $B(n, p_n)$, con
\[ p_n = \frac{\lambda}{n} \]
allora
\[ \lim_{n \to +\infty} p^{(n)} (h) = P_\lambda (h) = \frac{\lambda^h}{h!} \cdot e^{-\lambda} \]
per ogni $h \in \N$.

\begin{example}
	Il numero di clienti ad un dato sportello è descritto da una variabile di Poisson di parametro
	$\lambda = 2.3$. Qual è la probabilita di avere al massimo 2 clienti?
	\begin{align*}
		P(X \leq 2) = & P(X = 0) + P(X = 1) + P(X = 2)                          \\
		=             & \frac{2.3^0}{0!} e^{-2.3} + \frac{2.3^1}{1!} e^{-2.3} +
		\frac{2.3^2}{2!} e^{-2.3}                                               \\
		=             & e^{-2.3} \left( 1 + 2.3 + \frac{2.3^2}{2} \right)
	\end{align*}
\end{example}

\subsection{Variabili uniformi su un intervallo}
Una variabile aleatoria $X$ è detta \textbf{uniforme} su un intervallo $(a, b)$ dato, se ha densità
\[
	f(x) = \begin{cases}
		\dfrac{1}{b - a} & x \in (a, b)    \\[2ex]
		0                & x \notin (a, b)
	\end{cases}
\]
Una variabile aleatoria uniforme su $(a,b)$ rappresenta la posizione di un punto scelto a caso
senza preferenze sull'intervallo $(a, b)$.

\subsection{Variabili aleatorie esponenziali}
Una variabile aleatoria $X$ è detta \textbf{esponenziale} di parametro $\lambda > 0$ se ha densità
\[
	f(x) = \begin{cases}
		\lambda \cdot e^{-\lambda x} & x > 0    \\[2ex]
		0                            & x \leq 0
	\end{cases}
\]
e si indica con $E(\lambda)$. Possiamo verificare che $f$ sia effettivamente una densità notando
in primo luogo che $f \geq 0$ e poi che
\[
	\int_{-\infty}^{+\infty} f(x) dx =
	\int_0^{+\infty} \lambda \cdot e^{-\lambda x} dx =
	-e^{\lambda x}
\]
da valutarsi tra 0 e $+\infty$ che da come risultato 1. La variabile $X$ assume valori strettamente
positivi con probabilità 1, poiché $f(x) = 0$ per ogni $x \leq 0$ e descrive il tempo di attesa tra
due eventi aleatori.

\begin{proposition}[Assenza di memoria]
	Se $X$ è una variabile esponenziale di parametro $\lambda > 0$, allora $\forall s,t > 0$ vale
	\[ P(X \leq s + t | X > s) = P(X \leq t) \]
	In altre parole, la probabilità che accada l'evento, non cambia sapendo che abbiamo atteso $s$.
\end{proposition}

\begin{example}
	Il tempo di vita (in giorni) di un macchinario è descritto da una variabile esponenziale di
	parametro $\lambda = \frac{1}{8}$. Ci chiediamo qual è la probabilità che il primo guasto si
	verifichi dopo 6 giorni.
	\[ P(X > 6) = \int_{6}^{+\infty} \frac{1}{8} e^{-x/8} dx = -e^{-x/8} \]
	da valutare tra 6 e $+\infty$, il che equivale a
	\[ 0 + e^{-6/8} = e^{-3/4} \]
	Se il macchinario non si è rotto nei primi 2 giorni, qual è la probabilità che duri almeno 8
	giorni?
	\[ P(X \geq 8 | X > 2) = P(X \geq 6) = e^{-3/4} \]
	Per l'assenza di memoria, è irrilevante il fatto che non si sia rotto nei primi 2 giorni e
	dunque la probabilità è la stessa calcolata in precedenza.
\end{example}

\subsection{Trasformazioni di variabli aleatorie con densità}
Sia $X : \Omega \to \R$ una variabile aleatoria con densità $f_X$, e sia $h : \R \to \R$. Ci
chiediamo se la variabile aleatoria
\[ Y : \Omega \to \R, \quad Y = h \circ X \]
abbia densità e se sì, come calcolarla. In generale $Y$ può non avere densità, ad esempio, se $h$
è costante, allora $Y$ è costante, in particolare è discreta (e quindi non ammette densità).

Anche quando $Y$ ammette densità non esiste una formula generale (cioè valida per ogni $h$) per
calcolarla. Però si può sempre provare a calcolare la funzione di ripartizione di $Y$
\[ F_Y (y) = P(Y \leq y) = P(h(X) \leq y) \]
in particolare, se $F_Y$ è $C^1$ a tratti, allora $Y$ ammette densità
\[ f_Y = \frac{d}{dy} F_Y \]
Se $h$ è regolare e biunivoca, si ha una formula per la densità.

\begin{proposition}[Cambio di variabile]
	Supponiamo che $X$ sia una variabile aleatoria con densità $f_X$ supportata su un intervallo
	aperto $A$ (cioè $f_X(x) = 0$ $\forall x \notin A$). Sia $h : A \to B$ con $B$ intervallo
	aperto, con $h \in C^1$, biunivoca e con inversa $h^{-1} \in C^1$. Allora la variabile
	aleatoria $Y = h \circ X$ ha densità $f_Y$ data dalla formula
	\[
		f_Y(y) = \begin{cases}
			f_X (h^{-1} (y)) \left| \frac{d h^{-1}}{dy} (y) \right| & y \in B    \\[1ex]
			0                                                       & y \notin B
		\end{cases}
	\]
	Per ricordare meglio la formula possiamo tenere a mente che
	\[ y = h(x) \quad \Rightarrow \quad x = h^{-1}(y) \]
	e quindi
	\[ dx = \frac{d h^{-1}}{dy} (y) dy \]
\end{proposition}

\begin{example}
	Sia $X \sim U([-1, 2])$ e sia $Z = X^2$. Vogliamo sapere se $Z$ ha densità e se sì come
	calcolarla. Possiamo pensare di usare la formula del cambio di variabile, ma non possiamo
	poiché $h(x) = x^2$ non è biunivoca nell'intervallo considerato. Cerchiamo quindi di calcolare
	la funzione di ripartizione di $Z$. Per prima cosa ci conviene capire dove prende valori la
	variabile $Z$. Poiché $X$ sta in $(-1, 2)$, possiamo dire che $Z$ sta in $(0,4)$. Questo ci
	dice che
	\begin{gather*}
		F_Z(z) = P(Z \leq z) = 0 \quad z \leq 0 \\
		F_Z(z) = P(Z \leq z) = 1 \quad z \geq 4
	\end{gather*}
	Ci interessa quindi calcolare la funzione di ripartizione per valori $0 < z < 4$.
	\[ F_Z(z) = P(X^2 \leq z) = P(|X| \leq \sqrt{z}) \]
	Distinguiamo due ulteriori casi, per valori $0 < z < 1$ abbiamo che
	\[
		P(|X| \leq \sqrt{z}) = P(-\sqrt{z} \leq X \leq \sqrt{z}) =
		\frac{|-\sqrt{z} - \sqrt{z}|}{|-1 - 2|} = \frac{2 \sqrt{z}}{3}
	\]
	Per valori $1 \leq z < 4$ abbiamo che
	\[
		P(|X| \leq \sqrt{z}) = P(-1 \leq X \leq \sqrt{z}) =
		\frac{|-1 - \sqrt{z}|}{|-1 - 2|} = \frac{\sqrt{z} + 1}{3}
	\]
	Si può verificare che $F_Z$ è continua su $\R$ e $C^1$ a tratti quindi $Z$ ammette densità data
	dalla formula
	\[
		f_Z (z) = \frac{d}{dz} F_Z(z) = \begin{cases}
			0                     & z \notin (0,4) \\[1ex]
			\dfrac{1}{3 \sqrt{z}} & z \in (0,1)    \\[2ex]
			\dfrac{1}{6 \sqrt{z}} & z \in (1, 4)
		\end{cases}
	\]
\end{example}

Per l'esercizio appena visto tenere a mente che nel caso uniforme, la probabilità che $X$ sia
compreso tra $c$ e $d$ è data dalla lunghezza di $(c,d) \cap (a,b)$, diviso la lunghezza di $(a,b)$.

\begin{example}
	Sia $Y \simeq E(2)$ e sia $W = Y^2$, ci chiediamo se $W$ abbia densità e se sì come calcolarla.
	In questo caso si può applicare la formula del cambio di variabili dato che la funzione
	\[ h : (0, +\infty) \to (0, +\infty) \quad h(x) = x^2 \]
	è biunivoca, $C^1$ con inversa $C^1$.
\end{example}

\subsection{Variabili Gaussiane}
La funzione $f(x) = e^{-x^2 / 2}$ è regolarissima (infinitamente derivabile), tende a 0 molto
velocemente per $|x| \to \infty$ e quindi è integrabile, ma non è possibile scrivere la sua
primitiva in termini di funzioni elementari. In sintesi significa che l'unico modo di rappresentare
il valore di
\[ \int_0^t e^{-x^2 / 2} dx \]
per un generico $t$ è ricorrere ad approssimazioni numeriche. Tuttavia, per alcuni particolari
valori di $t$ tale integrale assume valori semplici, ad esempio
\[ \int_{-\infty}^{+\infty} e^{-x^2 / 2} dx = \sqrt{2 \pi} \]
Di conseguenza, dividendo la funzione $e^{-x^2/2}$ per $\sqrt{2 \pi}$ si ottiene una densità di
probabilità espressa da
\[ \varphi (x) = \frac{1}{\sqrt{2 \pi}} \cdot e^{-x^2 / 2} \]
che prende il nome di \textbf{densità Gaussiana} (o \textbf{normale}) \textbf{standard} e che viene
indicata con $N(0,1)$. La sua funzione di ripartizione (di cui non possiamo dare una
rappresentazione diversa) è
\[ \Phi(x) = \frac{1}{\sqrt{2 \pi}} \int_{-\infty}^x e^{-t^2 / 2} dt \]
Un altro valore importante è detto $\alpha$-quantile della variabile $N(0,1)$ ed indicato con la
notazione $q_\alpha$. La Gaussiana standard ha anche alcune proprietà utili:
\begin{itemize}
	\item La sua densità $\varphi$ è pari, cioè $\varphi(x) = \varphi(-x)$, quindi la funzione è
	      simmetrica rispetto all'asse $y$.
	\item La funzione di ripartizione $\Phi$ gode della seguente proprietà
	      \[ \Phi(-x) = 1 - \Phi(x) \]
	      infatti vale
	      \begin{align*}
		      \Phi(-x) = & \int_{-\infty}^{-x} \varphi(y) dy           \\
		      =          & \int_x^{+\infty} \varphi(-y') dy'           \\
		      =          & \int_x^{+\infty} \varphi(y') dy'            \\
		      =          & \int_{-\infty}^{+\infty} \varphi (y') dy' -
		      \int_{-\infty}^x \varphi (y') dy'                        \\
		      =          & 1 - \Phi(x)
	      \end{align*}
	      dove $y' = -y$ e vale in generale per densità $\varphi$ pari.
	\item Data $X$ Gaussiana standard vale che
	      \[ P(-t \leq X \leq t) = \Phi (t) - \Phi (-t) = 2 \Phi (t) - 1 \]
	\item $P(X \leq 0) = \Phi (0) = 1 / 2$
\end{itemize}
Vogliamo ora calcolare
\[ P(a < X < b) = \int_b^a \varphi (y) dy \]
ma come abbiamo detto non esiste una formula esplicita. Si fa quindi ricorso alla funzione di
ripartizione
\begin{multline*}
	P(a < X < b) = P(a \leq X < b) \\
	= P(a < X \leq b) = P(a \leq X \leq b) = \Phi(b) - \Phi(a)
\end{multline*}
Della funzione di ripartizione esistono tavole contenenti approssimazioni numeriche per
$x \in (0,4)$. Per $x \geq 4$ abbiamo che $\Phi(x) \simeq 1$, per $x < 0$ abbiamo invece
$\Phi(x) = 1 - \Phi(-x)$. Altre delle approssimazioni più importanti sono
\begin{gather*}
	P(-1 \leq X \leq 1) \simeq 0.68 \\
	P(-2 \leq X \leq 2) \simeq 0.94 \\
	P(-3 \leq X \leq 3) \simeq 0.997
\end{gather*}
Leggendo al contrario le tavole, si ricavano gli $\alpha$-quantili
\begin{itemize}
	\item Per $\alpha \in (1/2, 1)$ abbiamo che $q_\alpha$ è tale che $\Phi(q_\alpha) = \alpha$.
	\item Per $\alpha \in (0, 1/2)$ abbiamo che $q_{1-\alpha} = -q_\alpha$
\end{itemize}
Passiamo ora al caso generale di Gaussiana: data $X$ Gaussiana standard, siano $\sigma > 0$ e
$\mu \in \R$ e sia $Y = \sigma X + \mu$ una variabile aleatoria. La funzione di ripartizione di
$Y$ è data da
\[
	F_Y(y) = P(Y \leq y) = P(\sigma X + \mu \leq y)
	= P\left(X \leq \frac{y - m}{\sigma}\right)
	= \Phi \left(\frac{y - m}{\sigma}\right)
\]
e la sua densità (ottenuta derivando $F_Y$ o applicando la formula di cambio di variabile) è
\[
	f_Y(y) = \frac{1}{\sigma} \cdot \varphi \left(\frac{y - m}{\sigma}\right) =
	\frac{1}{\sqrt{2 \pi} \sigma} \cdot e^{-\frac{(y - m)^2}{2 \sigma^2}}
\]
Quest'ultima funzione è detta \textbf{densità Gaussiana} (o \textbf{normale}) $N(\mu, \sigma^2)$,
ovvero diciamo che $Y$ è Gaussiana $N(\mu, \sigma^2)$.

In generale è sempre possibile ricondurre tutti i calcoli relativi alla funzione di ripartizione
di una variabile Gaussiana generica alla funzione di ripartizione $\Phi$: è sufficiente sostituire
a $Y$ Gaussiana $N(\mu, \sigma^2)$ la rappresentazione $\sigma X + \mu$, con $X$ Gaussiana
standard. Ad esempio, in particolare
\[ P(a < Y < b) = P \left( \frac{a-m}{\sigma} < X < \frac{b-m}{\sigma} \right) \]

\begin{theorem}[Riproducibilità]
	Sia $Y$ una Gaussiana $N(\mu, \sigma^2)$, sia $V = \alpha Y + \beta$, con
	$\alpha, \beta \in \R$ e $\alpha \neq 0$. Allora $V$ è una Gaussiana
	$N(\alpha \mu + \beta, \alpha^2 \sigma^2)$.
\end{theorem}

\end{document}

