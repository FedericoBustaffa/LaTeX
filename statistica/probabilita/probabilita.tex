\part{Probabilità}

\chapter{Probabilità e indipendenza}
Il calcolo delle \textbf{probabilità} ci serve per la costruzione di un modello
\textbf{statico inferenziale}, il quale unisce statistica descrittiva e probabilità per descrivere la
realtà.

Prima di addentrarci nei formalismi matematici introduciamo qualche concetto \emph{primitivo}.
\begin{itemize}
	\item Quello che vogliamo calcolare è la probabilità che un certo \textbf{evento} accada. Il concetto
	      di evento non si può formalizzare, ma possiamo fare qualche esempio: se lanciamo un dado, il
	      fatto che esca 6 è un evento.
	\item Il secondo concetto è quello di \textbf{casuale} o \textbf{aleatorio} (\textbf{stocastico}).
	      Queste tre parole hanno quasi lo stesso significato ma possiamo generalizzare dicendo che
	      significano \emph{"dovuto al caso"}. Più avanti introdurremo altri termini come
	      \emph{variabili aleatorie} e \emph{indipendenza stocastica}.
\end{itemize}
Due dei problemi che andremo a trattare sono:
\begin{itemize}
	\item La \textbf{rappresentazione} degli eventi.
	\item Quali \textbf{proprietà} devono soddisfare le probabilità.
\end{itemize}
Per riuscire a capire meglio tutti questi concetti sarà necessario anche conoscere qualche nozione di base
sulle serie.