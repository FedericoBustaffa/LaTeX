\section{Indipendenza}
Il concetto di \textbf{indipendenza} nasce dalla necessità di esprimere in modo rigoroso che la
probabilità di un evento $A$ non cambia sapendo che accade $B$ e viceversa. Per $A$ e $B$ non
trascurabili vale quindi 
\[ P(A) = P(A | B) \quad \Leftrightarrow \quad P(A) P(B) = P(A \cap B) \]
Lo stesso vale per $P(B)$.

\begin{definition}
	Due eventi $A$ e $B$ sono \textbf{indipendenti} se
	\[ P(A \cap B) = P(A) P(B) \]
\end{definition}

Si può dimostrare per esercizio che, se $A$ e $B$ sono indipendenti allora lo sono anche
\begin{itemize}
	\item $A^c$ e $B$
	\item $A$ e $B^c$
	\item $A^c$ e $B^c$
\end{itemize}
L'indipendenza è dunque stabile per la complementazione. Si può inoltre dimostrare che 
\begin{itemize}
	\item Se $P(A) \in \{ 0, 1 \}$ allora $A$ è indipendente da ogni altro evento.
	\item Se $A \cap B = \emptyset$ allora $A$ e $B$ non sono indipendente a meno che $P(A)$ o
		$P(B)$ siano 0.
\end{itemize}

\begin{example}
	Si vuole estrarre una carta da un mazzo di 40 carte napoletane
\end{example}
