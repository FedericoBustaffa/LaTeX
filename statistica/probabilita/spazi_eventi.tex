\section{Spazio degli eventi}
Possiamo vedere tutti gli eventi possibili come un insieme, detto \textbf{spazio degli eventi} o
\textbf{universo}, che indicheremo con $\Omega$.
\begin{itemize}
	\item $\emptyset$, $\Omega$ sono spazi di eventi.
	\item Se $A$ e $B$ sono spazi di eventi, lo sono anche $A \cup B$ (rappresenta un \verb|or| logico di
	      eventi), $A \cap B$ (rappresenta un \verb|and| logico di eventi) e $A^c$ (rappresenta un
	      \verb|not| logico di eventi). Questi possono essere sottoinsiemi di $\Omega$ o $\Omega$ stesso.
\end{itemize}

\begin{example}
	Lanciamo un dado
	\[ \Omega = \{ 1, 2, 3, 4, 5, 6 \} \]
	L'insieme degli eventi tali che il numero uscito sia
	\begin{itemize}
		\item Pari $A = \{ 2, 4, 6 \}$
		\item Maggiore di 3 $B = \{ 4, 5, 6 \}$
		\item Non pari $A^c = \{ 1, 3, 5 \}$
		\item Pari o maggiore di 3 $A \cup B = \{ 2, 4, 5, 6 \}$
		\item Pari e maggiore di 3 $A \cap B = \{ 4, 6 \}$
	\end{itemize}
\end{example}

\begin{definition}
	Quando c'è un insieme di eventi $\A$ con queste proprietà:
	\begin{itemize}
		\item $\emptyset, \Omega \in \A$
		\item Se $A \in \A$ allora $A^c \in \A$
		\item Se $A, B \in \A$ allora $(A \cup B) \in \A$ e $(A \cap B) \in \A$
	\end{itemize}
	questo è chiamato \textbf{algebra di parti}.
\end{definition}

\section{Funzione di probabilità}
\begin{definition}[Provvisoria]
	Definiamo la \textbf{probabilità} come la funzione
	\[ P : \A \to [0, 1] \]
	tale che
	\[
		P (\Omega) = 1 \quad \land \quad
		A \cap B = \emptyset \Rightarrow P(A \cup B) = P(A) + P(B)
	\]
	Una funzione con queste proprietà è detta \textbf{finitamente additiva}.
\end{definition}

La probabilità, così come l'abbiamo definita, comporta che
\begin{itemize}
	\item $P(\emptyset) = 0$
	\item $P(A^c) = 1 - P(A)$
	\item $P(A \backslash B) = P(A) - P(B)$
	\item $P(A \cup B) = P(A) + P(B) - P(A \cap B)$
\end{itemize}

\subsection{Andare al limite}
Introduciamo ora due definizioni che possono sembrare controintuitive ma che saranno più chiare andando
avanti.

\begin{definition}
	Se $P(A) = 0$ si dice che $A$ è un evento \textbf{trascurabile}.
\end{definition}

\begin{definition}
	Se $P(A) = 1$ si dice che $A$ è un evento \textbf{quasi certo}.
\end{definition}

Per riuscire a capire perché a questi eventi viene dato questo nome dobbiamo introdurre il concetto di
\textbf{additività numerabile} o $\sigma$-\textbf{additività}, necessario per \emph{andare al limite},
senza il quale non si potrebbe fare calcolo statistico.

\begin{theorem}
	Consideriamo una successione di eventi a due a due disgiunti del tipo
	\[ A_1, \; A_2, \; A_3, \; \dots \]
	con $A_i \cap A_j = \emptyset$ se $i \neq j$ allora vale
	\[
		P(\cup_{n=1}^\infty A_n) = \sum_{n=1}^\infty P(A_n) =
		\lim_{n \to \infty} \sum_{k=1}^n P(A_k) = 	\lim_{n \to \infty} P(A_1) + \dots + P(A_n)
	\]
	Questa è detta \textbf{somma infinita}.
\end{theorem}

Siamo ora in grado di dare una nuova definizione di probabilità, la quale ci permette, come detto poco fa,
di \emph{andare al limite}.
\begin{definition}
	Definiamo la \textbf{probabilità} come la funzione $P$ definita sugli eventi
	\[ P : eventi \to [0, 1] \]
	tale che
	\[
		P(\Omega) = 1 \quad \land \quad
		P(\cup_{n=1}^\infty A_n) = \sum_{n=1}^\infty P(A_n)
	\]
\end{definition}

Consideriamo la successione di eventi
\[ A_1 \subseteq A_2 \subseteq A_3 \subseteq \dots \subseteq A_\infty \]
abbiamo che l'evento $A$, definito come l'unione infinita della successione, equivale a
\[ A = \cup_{n=1}^\infty A_n = \lim_{n \to \infty} A_n \]
La probabilità che $A$ si verifichi equivale quindi a
\[ P(A) = \lim_{n \to \infty} P(A_n) \]

\subsubsection{Spazio degli eventi finito}
Il punto di partenza è il caso in cui $\Omega$ è finito
\[ \Omega = \{ \omega_1, \dots, \omega_n \} \]
e dunque possiamo prendere come probabilità la funzione finitamente additiva definita su tutto lo spazio
degli eventi tale che $P(\Omega) = 1$. In particolare, quando $\Omega$ è finito allora la probabilità si
può sempre definire su tutti i suoi sottoinsiemi, come la somma della probabilità dei punti che lo
costituiscono
\[ P(A) = \sum_{\omega_i \in A} P(\omega_i) = \sum_{\omega_i \in A} p_i \]

\subsubsection{Spazio degli eventi infinito}
Se invece $\Omega$ è infinito ma numerabile, per esempio $\Omega = \N$. Quando $\Omega$ è numerabile allora
la probabilità si può definire su tutti i sottoinsiemi di $\Omega$ e basta conoscere la probabilità dei
singoli punti $p_i$
\[ p_i = P(\omega_i) \]
e deve valere
\[ \sum_{i=1}^\infty p_i = 1 \]
mentre se consideriamo un sottoinsieme $A$ di $\Omega$ allora vale
\[ P(A) = \sum_{\omega_i \in A} p_i \]

\subsection{Serie}
Andiamo ora a capire trattare alcuni concetti di base, relativi alle serie, necessarie ad inquadrare meglio
la situazione.

\begin{definition}
	Sia $a_1, a_2, \dots$ una successione di termini, e sia
	\[ s_n = a_1 + \dots + a_n \]
	la somma parziale dei primi $n$ termini. Definiamo quest'ultima come \textbf{serie} se esiste il limite
	delle somme parziali
	\[ \sum_{n=1}^\infty a_n = \lim_{n \to \infty} \sum_{k=1}^n a_k = \lim_{n \to \infty} s_n \]
	Se esiste finito questo limite si dice che la serie \textbf{converge}.
\end{definition}

\begin{observation}
	Da qui è facile osservare che
	\[ a_n = s_n - s_{n-1} \]
	ed è chiaro che se esiste il limite
	\[ \lim_{n \to \infty} s_n \]
	il valore $a_n$ tende a zero per $n \to \infty$ ma non è vero il viceversa.
\end{observation}

Supponiamo ora il caso in cui $a_n \geq 0$ per ogni $n \in \N$ allora le somme parziali crescono
\[ s_n \leq s_{n+1} \]
In questo caso ha sempre senso parlare di limite poiché, per una successione del genere, vale
\[ \sum_{n=1}^\infty = \lim_{n \to \infty} \sum_{k=1}^n a_k \in [0, +\infty]  \]
Il fatto che esista sempre il limite non significa che la serie converga. Come detto poco fa, la serie
converge se il limite tende ad un numero reale finito.

\subsubsection{Serie assolutamente convergenti}
\begin{definition}
	Consideriamo la serie
	\[ \sum_{n=1}^\infty |a_n| \]
	Se il limite per $n \to \infty$ tende ad un numero reale finito si dice che la serie converge
	\emph{assolutamente}.
\end{definition}

\begin{theorem}
	Se una serie converge assolutamente allora converge.
\end{theorem}

Per le serie di termini positivi che convergono assolutamente vale anche una sorta di proprietà
\textbf{associativa} in cui si partizionano gli indici della serie in sottoinsiemi di $\N$
($A_1, A_2, \dots$) e vale
\[ \sum_{n=1}^\infty a_n = \sum_{n=1}^\infty \left( \sum_{k \in A_n} a_k \right) \]

\subsubsection{Serie fondamentali}
Consideriamo ora due serie fondamentali di cui non facciamo la dimostrazione ma che ci saranno molto
utili in futuro:
\begin{itemize}
	\item \textbf{Serie geometrica}: se $|a| < 1$ allora vale
	      \[ \sum_{n=0}^\infty a^n = 1 + a + a^2 + \dots = \frac{1}{1 - a} \]
	\item \textbf{Sviluppo in serie dell'esponenziale}
	      \[ e^x = \sum_{n=0}^\infty \frac{x^n}{n!} \]
\end{itemize}
