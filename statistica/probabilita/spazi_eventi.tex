\section{Spazi di probabilità}
Possiamo vedere tutti gli eventi possibili come un insieme, detto \textbf{spazio campionario}, che
indicheremo con $\Omega$.
\begin{itemize}
	\item $\Omega$ rappresenta lo \textbf{spazio campionario} di tutti i possibili eventi.
	\item I sottoinsiemi di $\Omega$ sono detti \textbf{eventi}.
	\item I singoli esiti appartenenti ad un insieme sono detti \textbf{eventi elementari}
\end{itemize}
Dato che possiamo vedere gli spazi campionari come insiemi di eventi, è facile trovare una 
correlazione tra le operazioni insiemistiche e i connettivi logici.
\begin{itemize}
	\item All'unione ($A \cup B$) corrisponde un \verb|or| logico.
	\item All'intersezione ($A \cap B$) corrisponde un \verb|and| logico.
	\item Alla complementazione ($A^c$) corrisponde un \verb|not| logico.
	\item Alla differenza ($A \backslash B$) corrisponde $A \land \lnot B$
	\item La relazione insiemistica di sottoinsieme ($A \subseteq B$) corrisponde 
		all'implicazione logica.
	\item Se $A \cap B = \emptyset$ allora si dice che gli eventi sono \textbf{disgiunti}.
\end{itemize}
Ora possiamo gestire gli eventi legati da connettivi logici tramite operazioni insiemistiche
equivalenti.

\begin{example}
	Se lanciamo un dado a 6 facce abbiamo
	\[ \Omega = \{ 1, 2, 3, 4, 5, 6 \} \]
	L'insieme degli eventi tali che il numero uscito sia
	\begin{itemize}
		\item Pari è $A = \{ 2, 4, 6 \}$
		\item Maggiore di 3 è $B = \{ 4, 5, 6 \}$
		\item Non pari è $A^c = \{ 1, 3, 5 \}$
		\item Pari o maggiore di 3 è $A \cup B = \{ 2, 4, 5, 6 \}$
		\item Pari e maggiore di 3 è $A \cap B = \{ 4, 6 \}$
		\item Pari ma non maggiore di 3 è $A \backslash B = A \cap B^c = \{2\}$
		\item Sia pari che dispari è $A \cap A^c = \emptyset$
	\end{itemize}
\end{example}

\section{Funzione di probabilità}
Vogliamo ora definire la probabilità su una classe \emph{ammissibile} di eventi. Questo non è 
sempre possibile. Si vuole quindi definire una classe ammissibile che sia chiusa per le operazioni
insiemistiche viste in precedenza.

\begin{definition}
	Sia $\Omega \neq \emptyset$ e si consideri $\F \subseteq \mathcal{P}(\Omega)$ dove 
	$\mathcal{P}(\Omega)$ è
	un'insieme di sottoinsiemi di $\Omega$, allora $\F$ è detta $\sigma$\textbf{-algebra} se
	valgono le seguenti proprietà:
	\begin{itemize}
		\item $\Omega, \emptyset \in \F$
		\item $\F$ è stabile per la complementazione: se $A \in \F$ allora $A^c \in \F$
		\item $\F$ è stabile per unione finita o numerabile: se $A_1, A_2, ..., A_n, ...$ è una 
			successione finita o numerabile, allora l'unione di tutti gli $A_i$ appartiene a $\F$.
	\end{itemize}
\end{definition}

La $\sigma$-algebra è quindi un'operazione chiusa per tutte le operazioni insiemistiche viste
sopra ed è la classe degli eventi \emph{ammissibili}.

La \textbf{probabilità} di un evento $A \in \F$ è il grado di fiducia che $A$ si realizzi ed è 
compreso tra 0 e 1.

\`E intuitivo supporre che, se due eventi $A$ e $B$ sono disgiunti, allora la probabilità che si
realizzi $A \cup B$ è $P(A) + P(B)$. Questo equivale a dire che la probabilità è una funzione 
d'insieme \textbf{finitamente additiva}.

\begin{definition}
	Dato $\Omega$ un insieme e $\F$ una $\sigma$-algebra di $\Omega$, la \textbf{probabilità} è
	una funzione
	\[ P : \F \to [0, 1] \]
	tale che
	\begin{itemize}
		\item L'evento certo ha probabilità unitaria: $P(\Omega) = 1$.
		\item Se la successione degli $A_i$ è una successione di elementi di $\F$ a due a due
			disgiunti, si ha, nel caso in cui la successione sia finita, che
			\[ P (\cup_{n=i}^n A_i) = \sum_{i=1}^n P(A_i) \]
			Se invece la successione è infinita allora vale
			\[
				P (\cup_{i=1}^\infty A_i) = \sum_{i=1}^\infty P(A_i) =
				\lim_{n \to \infty} \sum_{i=1}^n P(A_i)
			\]
	\end{itemize}
\end{definition}

\begin{proposition}
	Supponiamo di avere una successione infinita $A \in \F$	allora
	\begin{itemize}
		\item $A$ è una successione \textbf{crescente}, cioè $A_i \subseteq A_{i+1}$, $\forall i$
			se vale
			\[ A = \cup_{i=1}^\infty A_i \]
		\item $A$ è una successione \textbf{decrescente}, cioè $A_{i+1} \subseteq A_i$,
			$\forall i$ se vale 
			\[ A = \cap_{i=1}^\infty A_i \]
	\end{itemize}
\end{proposition}

\begin{definition}
	Se $P(A) = 0$ si dice che $A$ è un evento \textbf{trascurabile}.
\end{definition}

\begin{definition}
	Se $P(A) = 1$ si dice che $A$ è un evento \textbf{quasi certo}.
\end{definition}

\begin{definition}
	La terna $(\Omega, \F, P)$ formata da uno spazio campionario, una $\sigma$-algebra ed una
	probabilità $P$ definita su $\F$ viene chiamata \textbf{spazio di probabilità}.
\end{definition}

\paragraph{Proprietà} La probabilità, così come l'abbiamo definita, comporta che
\begin{itemize}
	\item $P(\emptyset) = 0$
	\item $P(A^c) = 1 - P(A)$
	\item $P(A \backslash B) = P(A) - P(B)$
	\item $P(A \cup B) = P(A) + P(B) - P(A \cap B)$
\end{itemize}

\subsection{Andare al limite}
Introduciamo ora due definizioni che possono sembrare controintuitive ma che saranno più chiare
andando avanti.



Per riuscire a capire perché a questi eventi viene dato questo nome dobbiamo introdurre il
concetto di \textbf{additività numerabile} o $\sigma$-\textbf{additività}, necessario per
\emph{andare al limite}, senza il quale non si potrebbe fare calcolo statistico.

\begin{theorem}
	Consideriamo una successione di eventi a due a due disgiunti del tipo
	\[ A_1, \; A_2, \; A_3, \; \dots \]
	con $A_i \cap A_j = \emptyset$ se $i \neq j$ allora vale
	\[
		P(\cup_{n=1}^\infty A_n) = \sum_{n=1}^\infty P(A_n) =
		\lim_{n \to \infty} \sum_{k=1}^n P(A_k) = 	\lim_{n \to \infty} P(A_1) + \dots + P(A_n)
	\]
	Questa è detta \textbf{somma infinita}.
\end{theorem}

Siamo ora in grado di dare una nuova definizione di probabilità, la quale ci permette, come detto
poco fa, di \emph{andare al limite}.
\begin{definition}
	Definiamo la \textbf{probabilità} come la funzione $P$ definita sugli eventi
	\[ P : eventi \to [0, 1] \]
	tale che
	\[
		P(\Omega) = 1 \quad \land \quad
		P(\cup_{n=1}^\infty A_n) = \sum_{n=1}^\infty P(A_n)
	\]
\end{definition}

Consideriamo la successione di eventi
\[ A_1 \subseteq A_2 \subseteq A_3 \subseteq \dots \subseteq A_\infty \]
abbiamo che l'evento $A$, definito come l'unione infinita della successione, equivale a
\[ A = \cup_{n=1}^\infty A_n = \lim_{n \to \infty} A_n \]
La probabilità che $A$ si verifichi equivale quindi a
\[ P(A) = \lim_{n \to \infty} P(A_n) \]

\subsubsection{Spazio degli eventi finito}
Il punto di partenza è il caso in cui $\Omega$ è finito
\[ \Omega = \{ \omega_1, \dots, \omega_n \} \]
e dunque possiamo prendere come probabilità la funzione finitamente additiva definita su tutto lo
spazio degli eventi tale che $P(\Omega) = 1$. In particolare, quando $\Omega$ è finito allora la
probabilità si può sempre definire su tutti i suoi sottoinsiemi, come la somma della probabilità
dei punti che lo costituiscono
\[ P(A) = \sum_{\omega_i \in A} P(\omega_i) = \sum_{\omega_i \in A} p_i \]

\subsubsection{Spazio degli eventi infinito}
Se invece $\Omega$ è infinito ma numerabile, per esempio $\Omega = \N$. Quando $\Omega$ è
numerabile allora la probabilità si può definire su tutti i sottoinsiemi di $\Omega$ e basta
conoscere la probabilità dei singoli punti $p_i$
\[ p_i = P(\omega_i) \]
e deve valere
\[ \sum_{i=1}^\infty p_i = 1 \]
mentre se consideriamo un sottoinsieme $A$ di $\Omega$ allora vale
\[ P(A) = \sum_{\omega_i \in A} p_i \]

\subsection{Serie}
Andiamo ora a capire trattare alcuni concetti di base, relativi alle serie, necessarie ad
inquadrare meglio la situazione.

\begin{definition}
	Sia $a_1, a_2, \dots$ una successione di termini, e sia
	\[ s_n = a_1 + \dots + a_n \]
	la somma parziale dei primi $n$ termini. Definiamo quest'ultima come \textbf{serie} se esiste
	il limite delle somme parziali
	\[ \sum_{n=1}^\infty a_n = \lim_{n \to \infty} \sum_{k=1}^n a_k = \lim_{n \to \infty} s_n \]
	Se esiste finito questo limite si dice che la serie \textbf{converge}.
\end{definition}

\begin{observation}
	Da qui è facile osservare che
	\[ a_n = s_n - s_{n-1} \]
	ed è chiaro che se esiste il limite
	\[ \lim_{n \to \infty} s_n \]
	il valore $a_n$ tende a zero per $n \to \infty$ ma non è vero il viceversa.
\end{observation}

Supponiamo ora il caso in cui $a_n \geq 0$ per ogni $n \in \N$ allora le somme parziali crescono
\[ s_n \leq s_{n+1} \]
In questo caso ha sempre senso parlare di limite poiché, per una successione del genere, vale
\[ \sum_{n=1}^\infty = \lim_{n \to \infty} \sum_{k=1}^n a_k \in [0, +\infty]  \]
Il fatto che esista sempre il limite non significa che la serie converga. Come detto poco fa, la
serie converge se il limite tende ad un numero reale finito.

\subsubsection{Serie assolutamente convergenti}
\begin{definition}
	Consideriamo la serie
	\[ \sum_{n=1}^\infty |a_n| \]
	Se il limite per $n \to \infty$ tende ad un numero reale finito si dice che la serie converge
	\emph{assolutamente}.
\end{definition}

\begin{theorem}
	Se una serie converge assolutamente allora converge.
\end{theorem}

Per le serie di termini positivi che convergono assolutamente vale anche una sorta di proprietà
\textbf{associativa} in cui si partizionano gli indici della serie in sottoinsiemi di $\N$
($A_1, A_2, \dots$) e vale
\[ \sum_{n=1}^\infty a_n = \sum_{n=1}^\infty \left( \sum_{k \in A_n} a_k \right) \]

\subsubsection{Serie fondamentali}
Consideriamo ora due serie fondamentali di cui non facciamo la dimostrazione ma che ci saranno
molto utili in futuro:
\begin{itemize}
	\item \textbf{Serie geometrica}: se $|a| < 1$ allora vale
	      \[ \sum_{n=0}^\infty a^n = 1 + a + a^2 + \dots = \frac{1}{1 - a} \]
	\item \textbf{Sviluppo in serie dell'esponenziale}
	      \[ e^x = \sum_{n=0}^\infty \frac{x^n}{n!} \]
\end{itemize}
