\section{Spazi di probabilità}
Possiamo vedere tutti gli eventi possibili come un insieme, detto \textbf{spazio campionario}, che
indicheremo con $\Omega$.
\begin{itemize}
	\item $\Omega$ rappresenta lo \textbf{spazio campionario} di tutti i possibili eventi.
	\item I sottoinsiemi di $\Omega$ sono detti \textbf{eventi}.
	\item I singoli esiti appartenenti ad un insieme sono detti \textbf{eventi elementari}
\end{itemize}
Dato che possiamo vedere gli spazi campionari come insiemi di eventi, è facile trovare una 
correlazione tra le operazioni insiemistiche e i connettivi logici.
\begin{itemize}
	\item All'unione ($A \cup B$) corrisponde un \verb|or| logico.
	\item All'intersezione ($A \cap B$) corrisponde un \verb|and| logico.
	\item Alla complementazione ($A^c$) corrisponde un \verb|not| logico.
	\item Alla differenza ($A \backslash B$) corrisponde $A \land \lnot B$
	\item La relazione insiemistica di sottoinsieme ($A \subseteq B$) corrisponde 
		all'implicazione logica.
	\item Se $A \cap B = \emptyset$ allora si dice che gli eventi sono \textbf{disgiunti}.
\end{itemize}
Ora possiamo gestire gli eventi legati da connettivi logici tramite operazioni insiemistiche
equivalenti.

\begin{example}
	Se lanciamo un dado a 6 facce abbiamo
	\[ \Omega = \{ 1, 2, 3, 4, 5, 6 \} \]
	L'insieme degli eventi tali che il numero uscito sia
	\begin{itemize}
		\item Pari è $A = \{ 2, 4, 6 \}$
		\item Maggiore di 3 è $B = \{ 4, 5, 6 \}$
		\item Non pari è $A^c = \{ 1, 3, 5 \}$
		\item Pari o maggiore di 3 è $A \cup B = \{ 2, 4, 5, 6 \}$
		\item Pari e maggiore di 3 è $A \cap B = \{ 4, 6 \}$
		\item Pari ma non maggiore di 3 è $A \backslash B = A \cap B^c = \{2\}$
		\item Sia pari che dispari è $A \cap A^c = \emptyset$
	\end{itemize}
\end{example}

\subsection{Funzione di probabilità}
Vogliamo ora definire la probabilità su una classe \emph{ammissibile} di eventi. Questo non è 
sempre possibile. Si vuole quindi definire una classe ammissibile che sia chiusa per le operazioni
insiemistiche viste in precedenza.

\begin{definition}
	Sia $\Omega \neq \emptyset$ e si consideri $\F \subseteq \mathcal{P}(\Omega)$ dove 
	$\mathcal{P}(\Omega)$ è
	un'insieme di sottoinsiemi di $\Omega$, allora $\F$ è detta $\sigma$\textbf{-algebra} se
	valgono le seguenti proprietà:
	\begin{itemize}
		\item $\Omega, \emptyset \in \F$
		\item $\F$ è stabile per la complementazione: se $A \in \F$ allora $A^c \in \F$
		\item $\F$ è stabile per unione finita o numerabile: se $A_1, A_2, ..., A_n, ...$ è una 
			successione finita o numerabile, allora l'unione di tutti gli $A_i$ appartiene a $\F$.
	\end{itemize}
\end{definition}

La $\sigma$-algebra è quindi un'operazione chiusa per tutte le operazioni insiemistiche viste
sopra ed è la classe degli eventi \emph{ammissibili}.

La \textbf{probabilità} di un evento $A \in \F$ è il grado di fiducia che $A$ si realizzi ed è 
compreso tra 0 e 1.

\`E intuitivo supporre che, se due eventi $A$ e $B$ sono disgiunti, allora la probabilità che si
realizzi $A \cup B$ è $P(A) + P(B)$. Questo equivale a dire che la probabilità è una funzione 
d'insieme \textbf{finitamente additiva}.

\begin{definition}
	Dato $\Omega$ un insieme e $\F$ una $\sigma$-algebra di $\Omega$, la \textbf{probabilità} è
	una funzione
	\[ P : \F \to [0, 1] \]
	tale che
	\begin{itemize}
		\item L'evento certo ha probabilità unitaria: $P(\Omega) = 1$.
		\item Se la successione degli $A_i$ è una successione di elementi di $\F$ a due a due
			disgiunti, si ha, nel caso in cui la successione sia finita, che
			\[ P (\cup_{n=i}^n A_i) = \sum_{i=1}^n P(A_i) \]
			Se invece la successione è infinita allora vale
			\[
				P (\cup_{i=1}^\infty A_i) = \sum_{i=1}^\infty P(A_i) =
				\lim_{n \to \infty} \sum_{i=1}^n P(A_i)
			\]
	\end{itemize}
\end{definition}

\begin{definition}
	La terna $(\Omega, \F, P)$ formata da uno spazio campionario, una $\sigma$-algebra ed una
	probabilità $P$ definita su $\F$ viene chiamata \textbf{spazio di probabilità}.
\end{definition}

\paragraph{Proprietà} La probabilità, così come l'abbiamo definita, comporta che
\begin{itemize}
	\item $P(\emptyset) = 0$
	\item $P(A^c) = 1 - P(A)$
	\item $P(A \backslash B) = P(A) - P(B)$
	\item $P(A \cup B) = P(A) + P(B) - P(A \cap B)$
\end{itemize}

\begin{proposition}
	Supponiamo di avere una successione infinita $A \in \F$	allora
	\begin{itemize}
		\item $A$ è una successione \textbf{crescente}, cioè $A_i \subseteq A_{i+1}$, $\forall i$
			se vale
			\[ A = \cup_{i=1}^\infty A_i \]
		\item $A$ è una successione \textbf{decrescente}, cioè $A_{i+1} \subseteq A_i$,
			$\forall i$ se vale 
			\[ A = \cap_{i=1}^\infty A_i \]
	\end{itemize}
	In entrambi i casi vale 
	\[ P(A) = \lim_{i \to \infty} P(A_i) \]
\end{proposition}

\begin{definition}
	Se $P(A) = 0$ si dice che $A$ è un evento \textbf{trascurabile}.
\end{definition}

\begin{definition}
	Se $P(A) = 1$ si dice che $A$ è un evento \textbf{quasi certo}.
\end{definition}

Facciamo ora qualche considerazione sulla $\sigma$-algebra. Consideriamo $\Omega$ finito o
numerabile e quindi così formato
\[ \Omega = \{ \omega_1, \omega_2, ..., \omega_i, ... \} \]
allora possiamo prendere $\F = \mathcal{P}(\Omega)$ e la probabilità è univocamente determinata
da $p_i = P(\{ \omega_i \})$. Vale infatti che 
\[ P(A) = \sum_{\omega_i \in A} p_i \]
Quindi, nel caso $A$ abbia un numero 
\begin{itemize}
	\item finito di elementi, $P(A)$ equivale alla somma dei $p_i$.
	\item infinito di elementi, $P(A)$ equivale alla serie dei $p_i$.
\end{itemize}

\begin{example}
	Vogliamo definire un modello di probabilità per la scelta casuale di un punto sull'intervallo
	$[a,b]$ in modo che la scelta sia uniforme. In questo caso abbiamo che $\Omega$ è non
	numerabile in quanto esistono infiniti punti tra $a$ e $b$. Ne segue che la probabilità di 
	scegliere un punto in questo intervallo con una funzione di probabilità definita come in 
	precedenza non può che essere 0.

	Potrebbe venirci in mente di definire la probabilità di scegliere un punto $x$ all'interno di 
	$[a,b]$ come la lunghezza dell'intervallo ($b-a$) ma ci si può facilmente accorgere che non
	si riesce a definire in modo coerente la lunghezza di ogni sottoinsieme di $[a,b]$.

	Per questo si introduce una $\sigma$-algebra più \emph{piccola} di $\mathcal{P}(\Omega)$, 
	detta degli \textbf{insiemi misurabili}.
\end{example}

\subsubsection{Serie}
Andiamo ora a capire trattare alcuni concetti di base, relativi alle serie, necessarie ad
inquadrare meglio la situazione.

\begin{definition}
	Sia $a_1, a_2, \dots$ una successione di termini, e sia
	\[ s_n = a_1 + \dots + a_n \]
	la somma parziale dei primi $n$ termini. Definiamo quest'ultima come \textbf{serie} se esiste
	il limite delle somme parziali
	\[ \sum_{n=1}^\infty a_n = \lim_{n \to \infty} \sum_{k=1}^n a_k = \lim_{n \to \infty} s_n \]
	Se esiste finito questo limite si dice che la serie \textbf{converge}.
\end{definition}

\begin{observation}
	Da qui è facile osservare che
	\[ a_n = s_n - s_{n-1} \]
	ed è chiaro che se esiste il limite
	\[ \lim_{n \to \infty} s_n \]
	il valore $a_n$ tende a zero per $n \to \infty$ ma non è vero il viceversa.
\end{observation}

Supponiamo ora il caso in cui $a_n \geq 0$ per ogni $n \in \N$ allora le somme parziali crescono
\[ s_n \leq s_{n+1} \]
In questo caso ha sempre senso parlare di limite poiché, per una successione del genere, vale
\[ \sum_{n=1}^\infty = \lim_{n \to \infty} \sum_{k=1}^n a_k \in [0, +\infty]  \]
Il fatto che esista sempre il limite non significa che la serie converga. Come detto poco fa, la
serie converge se il limite tende ad un numero reale finito.

\begin{definition}
	Consideriamo la serie
	\[ \sum_{n=1}^\infty |a_n| \]
	Se il limite per $n \to \infty$ tende ad un numero reale finito si dice che la serie converge
	\emph{assolutamente}.
\end{definition}

\begin{theorem}
	Se una serie converge assolutamente allora converge.
\end{theorem}

Per le serie di termini positivi che convergono assolutamente vale anche una sorta di proprietà
\textbf{associativa} in cui si partizionano gli indici della serie in sottoinsiemi di $\N$
($A_1, A_2, \dots$) e vale
\[ \sum_{n=1}^\infty a_n = \sum_{n=1}^\infty \left( \sum_{k \in A_n} a_k \right) \]

Consideriamo ora due serie fondamentali di cui non facciamo la dimostrazione ma che ci saranno
molto utili in futuro:
\begin{itemize}
	\item \textbf{Serie geometrica}: se $|a| < 1$ allora vale
	      \[ \sum_{n=0}^\infty a^n = 1 + a + a^2 + \dots = \frac{1}{1 - a} \]
	\item \textbf{Sviluppo in serie dell'esponenziale}
	      \[ e^x = \sum_{n=0}^\infty \frac{x^n}{n!} \]
\end{itemize}
