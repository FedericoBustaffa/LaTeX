\chapter{Intervalli di fiducia}
Gli \textbf{intervalli di fiducia} sono un esempio particolare di test statistico in cui si va a
definire un intervallo entro il quale potrebbe ricadere la probabilita effettiva. Tale intervallo
si rimpicciolisce sempre di più man mano che che il campione cresce in numerosità.

Gli estremi di questi intervalli sono determinati a partire dagli esiti della misurazione del
campione, dunque dal punto di vista matematico dobbiamo definirli come variabili aleatorie
ottenute da funzioni del campione statistico.

\section{Definizione di intervallo di fiducia}
Per comprendere meglio il concetto consideriamo un campione \iid $X_1, \dots, X_n$ di $X$, con
legge $(P_X)_\theta$ dipendente da un parametro incognito $\theta$. Vorremmo determinare un
intervallo, a partire dal campione, tale che $\theta$ appartenga a questo intervallo con alta
probabilità.

\begin{definition}
	Dati un campione $X_1, \dots, X_n$ \iid di $X$ con legge $(P_X)_\theta$ e due parametri
	$\theta \in \R$ e $\alpha \in (0,1)$, un intervallo aleatorio
	\[ I = [a(X_1, \dots, X_n), b(X_1, \dots, X_n)] \]
	si dice \textbf{intervallo di fiducia} (\IF) per $\theta$ di livello $1-\alpha$ se, per ogni
	$\theta \in \Theta$, vale
	\[ P_\theta (\theta \in I) \geq 1 - \alpha \]
	Tipicamente se $\alpha$ è piccolo (ad esempio $\alpha = 0.05$), allora $\theta \in I$ con
	probabilità alta.
\end{definition}

\begin{observation}
	Non è detto che $\theta \in I$ in ogni caso. Non dobbiamo inoltre considerare il parametro
	$\theta$ come aleatorio solo perché non lo conosciamo. Ciò che è aleatorio è l'intervallo di
	fiducia che stiamo cercando.
\end{observation}

Idealmente vorremmo che $\alpha$ sia piccolo e che l'ampiezza dell'intervallo $I$ sia piccola.
Questo non è semplice poiché, di solito, un $\alpha$ piccolo implica generalmente una grande
ampiezza dell'intervallo e viceversa.