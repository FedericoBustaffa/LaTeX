\section{Variabili aleatorie notevoli}
In questa sezione trattiamo le distribuzione di probabilità su $\R$ indispensabili per trattare
ogni applicazione. Si tratta sempre di variabili discrete o definite tramite densità.

\subsection{Variabili binomiali}
Consideriamo come primo caso le \textbf{variabili binomiali} prendendo $n$ prove ripetute, con
esito (per ciascuna prova) successo o insuccesso (schema di Bernoulli) e sia $p$ la probabilità di
successo (nella singola prova).
\[ P(a_1, \dots, a_n) p^{\# \{i | a_i=1\}} \cdot (1-p)^{\# \{ i | a_i=0 \}} \]
Sia $X$ la variabile aleatoria che conta il numero di successi, ossia
\[ X(a_1, \dots, a_n) = \sum_{i=1}^n a_i \]
Come possiamo notare $X$ è discreta a valori in $\{0,1,2,\dots,n\}$ e ha funzione di massa
\[ P_X(h) = P(X = h) = \binom{n}{h} \cdot p^h \cdot (1-p)^{n-h} \]
con $h \in \{ 0, 1, \dots, n \}$. Possiamo tradurre tutto questo nella probabilità che abbiamo di
avere $h$ successi.

Una \textbf{variabile aleatoria binomiale} di parametri $n \in \N^+$ e $p \in (0,1)$, è tale se
possiede la funzione di massa appena descritta ed è indicata con $B(n, p)$.

\begin{observation}
	Per $n=1$ si parla di variabile aleatoria di Bernoulli e si indica con $B(p)$.
\end{observation}

\begin{example}
	Su 5 lanci di un dado equilibrato, qual è la probabilità che il 6 appaia almeno 2 volte?

	Per prima cosa definiamo $X$ come il numero di volte che esce 6 nelle 5 prove. Vogliamo
	quindi calcolare la probabilità che il 6 esca almeno 2 volte in 5 lanci. Per farlo ci conviene
	passare al problema complementare:
	\[ P(X \geq 2) = 1 - (P(X = 0) + P(X = 1)) \]
	Calcoliamo quindi $P(X=0)$ e $P(X=1)$:
	\begin{gather*}
		P(X = 0) = \binom{5}{0} \cdot \left(\frac{1}{6}\right)^0 \cdot
		\left(1 - \frac{1}{6}\right)^5 = \left(\frac{5}{6}\right)^5 \\
		P(X = 1) = \binom{5}{1} \cdot \left(\frac{1}{6}\right)^1 \cdot
		\left(1 - \frac{1}{6}\right)^4 = \left(\frac{5}{6}\right)^4
	\end{gather*}
	Siamo ora in grado di calcolare $P(X \geq 2)$ come
	\[ P(X \geq 2) = 1 - \left( \left(\frac{5}{6}\right)^5 + \left(\frac{5}{6}\right)^4 \right) \]
\end{example}

\subsection{Variabili geometriche}
Rimaniamo nel contesto delle prove ripetute indipendenti con esito successo o insuccesso. Sia $X$
la variabile aleatoria che rappresenta l'istante del primo successo (l'istante è il numero della
prova).

Come possiamo notare, $X$ è discreta a valori in $\N^+ = \{ 1, 2, \dots \}$ e la sua funzione di
massa vale
\[ P(X = h) = (1-p)^{h-1} \cdot p \]
Questo ci dice la probabilità che abbiamo di ottenere il primo successo dopo $h$ tentativi.

Una \textbf{variabile aleatoria geometrica} di parametro $p \in (0,1)$, è tale se possiede la
funzione di massa appena descritta ed è indicata con $G(p)$.

\begin{proposition}[Assenza di memoria]
	Data una variabile geometrica di parametro $p$, per ogni $n,h \in \N^+$, vale
	\[ P(X = n + h | X > n) = P(X = h) \]
	La probabilità di successo dopo $h$ prove non cambia sapendo che il successo non si è
	verificato nelle prime $n$ prove.
\end{proposition}

\subsection{Variabili di Poisson}
Una variabile aleatoria $X$ si dice \textbf{variabile aleatoria di Poisson} di parametro
$\lambda > 0$ se è una variabile aleatoria discreta a valori in $\N$, con funzione di massa
\[ P(X = h) = \frac{\lambda^h}{h!} \cdot e^{-\lambda} \]
con $h \in \N$ e si indica con $P(\lambda)$.

La variabile di Poisson conta il numero di \emph{eventi rari}, infatti, tale variabile aleatoria,
approssima una binomiale di parametri $n$ e $p$ quando $n$ è grande, $p$ è piccola e
$\lambda \simeq np$. Si può dimostrare infatti che, detta
\[ p^{(n)} (h) = \binom{n}{h} \cdot p_n^h \cdot \left( 1-p_n \right)^{n-h} \]
la funzione di massa di $B(n, p_n)$, con
\[ p_n = \frac{\lambda}{n} \]
allora
\[ \lim_{n \to +\infty} p^{(n)} (h) = P_\lambda (h) = \frac{\lambda^h}{h!} \cdot e^{-\lambda} \]
per ogni $h \in \N$.

\begin{example}
	Il numero di clienti ad un dato sportello è descritto da una variabile di Poisson di parametro
	$\lambda = 2.3$. Qual è la probabilita di avere al massimo 2 clienti?
	\begin{align*}
		P(X \leq 2) = & P(X = 0) + P(X = 1) + P(X = 2)                          \\
		=             & \frac{2.3^0}{0!} e^{-2.3} + \frac{2.3^1}{1!} e^{-2.3} +
		\frac{2.3^2}{2!} e^{-2.3}                                               \\
		=             & e^{-2.3} \left( 1 + 2.3 + \frac{2.3^2}{2} \right)
	\end{align*}
\end{example}

\subsection{Variabili uniformi su un intervallo}
Una variabile aleatoria $X$ è detta \textbf{uniforme} su un intervallo $(a, b)$ dato, se ha densità
\[
	f(x) = \begin{cases}
		\dfrac{1}{b - a} & x \in (a, b)    \\[2ex]
		0                & x \notin (a, b)
	\end{cases}
\]
Una variabile aleatoria uniforme su $(a,b)$ rappresenta la posizione di un punto scelto a caso
senza preferenze sull'intervallo $(a, b)$.

\subsection{Variabili aleatorie esponenziali}
Una variabile aleatoria $X$ è detta \textbf{esponenziale} di parametro $\lambda > 0$ se ha densità
\[
	f(x) = \begin{cases}
		\lambda \cdot e^{-\lambda x} & x > 0    \\[2ex]
		0                            & x \leq 0
	\end{cases}
\]
e si indica con $E(\lambda)$. Possiamo verificare che $f$ sia effettivamente una densità notando
in primo luogo che $f \geq 0$ e poi che
\[
	\int_{-\infty}^{+\infty} f(x) dx =
	\int_0^{+\infty} \lambda \cdot e^{-\lambda x} dx =
	-e^{\lambda x}
\]
da valutarsi tra 0 e $+\infty$ che da come risultato 1. La variabile $X$ assume valori strettamente
positivi con probabilità 1, poiché $f(x) = 0$ per ogni $x \leq 0$ e descrive il tempo di attesa tra
due eventi aleatori.

\begin{proposition}[Assenza di memoria]
	Se $X$ è una variabile esponenziale di parametro $\lambda > 0$, allora $\forall s,t > 0$ vale
	\[ P(X \leq s + t | X > s) = P(X \leq t) \]
	In altre parole, la probabilità che accada l'evento, non cambia sapendo che abbiamo atteso $s$.
\end{proposition}

\begin{example}
	Il tempo di vita (in giorni) di un macchinario è descritto da una variabile esponenziale di
	parametro $\lambda = \frac{1}{8}$. Ci chiediamo qual è la probabilità che il primo guasto si
	verifichi dopo 6 giorni.
	\[ P(X > 6) = \int_{6}^{+\infty} \frac{1}{8} e^{-x/8} dx = -e^{-x/8} \]
	da valutare tra 6 e $+\infty$, il che equivale a
	\[ 0 + e^{-6/8} = e^{-3/4} \]
	Se il macchinario non si è rotto nei primi 2 giorni, qual è la probabilità che duri almeno 8
	giorni?
	\[ P(X \geq 8 | X > 2) = P(X \geq 6) = e^{-3/4} \]
	Per l'assenza di memoria, è irrilevante il fatto che non si sia rotto nei primi 2 giorni e
	dunque la probabilità è la stessa calcolata in precedenza.
\end{example}

\subsection{Trasformazioni di variabli aleatorie con densità}
Sia $X : \Omega \to \R$ una variabile aleatoria con densità $f_X$, e sia $h : \R \to \R$. Ci
chiediamo se la variabile aleatoria
\[ Y : \Omega \to \R, \quad Y = h \circ X \]
abbia densità e se sì, come calcolarla. In generale $Y$ può non avere densità, ad esempio, se $h$
è costante, allora $Y$ è costante, in particolare è discreta (e quindi non ammette densità).

Anche nel caso in cui $Y$ ammette densità non esiste una formula generale (cioè valida per ogni
$h$). In generale, si può sempre provare a calcolare la funzione di ripartizione di $Y$
\[ F_Y (y) = P(Y \leq y) = P(h(X) \leq y) \]
in particolare, se $F_Y$ è $C^1$ a tratti, allora $Y$ ammette densità
\[ f_Y = \frac{d}{dy} F_Y \]
Se $h$ è regolare e biunivoca, si ha una formula per la densità.

\begin{proposition}[Cambio di variabile]
	Supponiamo che $X$ sia una variabile aleatoria con densità $f_X$ supportata su un intervallo
	aperto $A$ (cioè $f_X(x) = 0$ $\forall x \notin A$). Sia $h : A \to B$ con $B$ intervallo
	aperto, con $h \in C^1$, biunivoca e con inversa $h^{-1} \in C^1$. Allora la variabile
	aleatoria $Y = h \circ X$ ha densità $f_Y$ data dalla formula
	\[
		f_Y(y) = \begin{cases}
			f_X (h^{-1} (y)) \left| \frac{d h^{-1}}{dy} (y) \right| & y \in B    \\[2ex]
			0                                                       & y \notin B
		\end{cases}
	\]
	Per ricordare meglio la formula possiamo tenere a mente che
	\[ y = h(x) \quad \Rightarrow \quad x = h^{-1}(y) \]
	e quindi
	\[ dx = \frac{d h^{-1}}{dy} (y) dy \]
\end{proposition}

\begin{example}
	Sia $X = U([-1, 2])$ e sia $Z = X^2$. Vogliamo sapere se $Z$ ha densità e se sì come calcolarla.

	Possiamo pensare di usare la formula del cambio di variabile, ma non possiamo poiché
	$h(x) = x^2$ non è biunivoca nell'intervallo considerato. Cerchiamo quindi di calcolare la
	funzione di ripartizione di $Z$.

	Per prima cosa ci conviene capire dove prende valori la variabile $Z$. Poiché $X$ sta in
	$(-1, 2)$, possiamo dire che $Z$ sta in $(0,4)$. Questo ci dice che
	\begin{gather*}
		F_Z(z) = P(Z \leq z) = 0 \quad z \leq 0 \\
		F_Z(z) = P(Z \leq z) = 1 \quad z \geq 4
	\end{gather*}
	Ci interessa quindi calcolare la funzione di ripartizione per valori $0 < z < 4$.
	\[ F_Z(z) = P(X^2 \leq z) = P(|X| \leq \sqrt{z}) \]
	Distinguiamo due ulteriori casi, per valori $0 < z < 1$ abbiamo che
	\[
		P(|X| \leq \sqrt{z}) = P(-\sqrt{z} \leq X \leq \sqrt{z}) =
		\frac{|-\sqrt{z} - \sqrt{z}|}{|-1 - 2|} = \frac{2 \sqrt{z}}{3}
	\]
	Per valori $1 \leq z < 4$ abbiamo che
	\[
		P(|X| \leq \sqrt{z}) = P(-1 \leq X \leq \sqrt{z}) =
		\frac{|-1 - \sqrt{z}|}{|-1 - 2|} = \frac{\sqrt{z} + 1}{3}
	\]
	Si può verificare che $F_Z$ è continua su $\R$ e $C^1$ a tratti quindi $Z$ ammette densità data
	dalla formula
	\[
		f_Z (z) = \frac{d}{dz} F_Z(z) = \begin{cases}
			0                     & z \notin (0,4) \\[2ex]
			\dfrac{1}{3 \sqrt{z}} & z \in (0,1)    \\[2ex]
			\dfrac{1}{6 \sqrt{z}} & z \in (1, 4)
		\end{cases}
	\]
\end{example}

\begin{example}
	Sia $Y \simeq E(2)$ e sia $W = Y^2$, ci chiediamo se $W$ abbia densità e se sì come calcolarla.

	In questo caso si può applicare la formula del cambio di variabili dato che la funzione
	\[ h : (0, +\infty) \to (0, +\infty) \quad h(x) = x^2 \]
	è biunivoca, $C^1$ con inversa $C^1$.
\end{example}

\subsection{Variabili Gaussiane}
La funzione $f(x) = e^{-x^2 / 2}$ è regolarissima (infinitamente derivabile), tende a 0 molto
velocemente per $|x| \to \infty$ e quindi è integrabile, ma non è possibile scrivere la sua
primitiva in termini di funzioni elementari. In sintesi significa che l'unico modo di rappresentare
il valore di
\[ \int_0^t e^{-x^2 / 2} dx \]
per un generico $t$ è ricorrere ad approssimazioni numeriche. Tuttavia, per alcuni particolari
valori di $t$ tale integrale assume valori semplici, ad esempio
\[ \int_{-\infty}^{+\infty} e^{-x^2 / 2} dx = \sqrt{2 \pi} \]
Di conseguenza, dividendo la funzione $e^{-x^2/2}$ per $\sqrt{2 \pi}$ si ottiene una densità di
probabilità espressa da
\[ \varphi (x) = \frac{1}{\sqrt{2 \pi}} \cdot e^{-x^2 / 2} \]
che prende il nome di \textbf{densità Gaussiana} (o \textbf{normale}) \textbf{standard} e che viene
indicata con $N(0,1)$. La sua funzione di ripartizione (di cui non possiamo dare una
rappresentazione diversa) è
\[ \Phi(x) = \frac{1}{\sqrt{2 \pi}} \int_{-\infty}^x e^{-t^2 / 2} dt \]
Un altro valore importante è detto $\alpha$-quantile della variabile $N(0,1)$ ed indicato con la
notazione $q_\alpha$. La Gaussiana standard ha anche alcune proprietà utili:
\begin{itemize}
	\item La sua densità $\varphi$ è pari, cioè $\varphi(x) = \varphi(-x)$, quindi la funzione è
	      simmetrica rispetto all'asse $y$.
	\item La funzione di ripartizione $\Phi$ gode della seguente proprietà
	      \[ \Phi(-x) = 1 - \Phi(x) \]
	      infatti vale
	      \begin{align*}
		      \Phi(-x) = & \int_{-\infty}^{-x} \varphi(y) dy           \\
		      =          & \int_x^{+\infty} \varphi(-y') dy'           \\
		      =          & \int_x^{+\infty} \varphi(y') dy'            \\
		      =          & \int_{-\infty}^{+\infty} \varphi (y') dy' -
		      \int_{-\infty}^x \varphi (y') dy'                        \\
		      =          & 1 - \Phi(x)
	      \end{align*}
	      dove $y' = -y$ e vale in generale per densità $\varphi$ pari.
	\item Data $X$ Gaussiana standard vale che
	      \[ P(-t \leq X \leq t) = \Phi (t) - \Phi (-t) = 2 \Phi (t) - 1 \]
	\item $P(X \leq 0) = \Phi (0) = 1 / 2$
\end{itemize}
Vogliamo ora calcolare
\[ P(a < X < b) = \int_b^a \varphi (y) dy \]
ma come abbiamo detto non esiste una formula esplicita. Si fa quindi ricorso alla funzione di
ripartizione
\begin{multline*}
	P(a < X < b) = P(a \leq X < b) \\
	= P(a < X \leq b) = P(a \leq X \leq b) = \Phi(b) - \Phi(a)
\end{multline*}
Della funzione di ripartizione esistono tavole contenenti approssimazioni numeriche per
$x \in (0,4)$. Per $x \geq 4$ abbiamo che $\Phi(x) \simeq 1$, per $x < 0$ abbiamo invece
$\Phi(x) = 1 - \Phi(-x)$. Altre delle approssimazioni più importanti sono
\begin{gather*}
	P(-1 \leq X \leq 1) \simeq 0.68 \\
	P(-2 \leq X \leq 2) \simeq 0.94 \\
	P(-3 \leq X \leq 3) \simeq 0.997
\end{gather*}
Leggendo al contrario le tavole, si ricavano gli $\alpha$-quantili
\begin{itemize}
	\item Per $\alpha \in (1/2, 1)$ abbiamo che $q_\alpha$ è tale che $\Phi(q_\alpha) = \alpha$.
	\item Per $\alpha \in (0, 1/2)$ abbiamo che $q_{1-\alpha} = -q_\alpha$
\end{itemize}
Passiamo ora al caso generale di Gaussiana: data $X$ Gaussiana standard, siano $\sigma > 0$ e
$\mu \in \R$ e sia $Y = \sigma X + \mu$ una variabile aleatoria. La funzione di ripartizione di
$Y$ è data da
\[
	F_Y(y) = P(Y \leq y) = P(\sigma X + \mu \leq y)
	= P\left(X \leq \frac{y - m}{\sigma}\right)
	= \Phi \left(\frac{y - m}{\sigma}\right)
\]
e la sua densità (ottenuta derivando $F_Y$ o applicando la formula di cambio di variabile) è
\[
	f_Y(y) = \frac{1}{\sigma} \cdot \varphi \left(\frac{y - m}{\sigma}\right) =
	\frac{1}{\sqrt{2 \pi} \sigma} \cdot e^{-\frac{(y - m)^2}{2 \sigma^2}}
\]
Quest'ultima funzione è detta \textbf{densità Gaussiana} (o \textbf{normale}) $N(\mu, \sigma^2)$,
ovvero diciamo che $Y$ è Gaussiana $N(\mu, \sigma^2)$.

In generale è sempre possibile ricondurre tutti i calcoli relativi alla funzione di ripartizione
di una variabile Gaussiana generica alla funzione di ripartizione $\Phi$: è sufficiente sostituire
a $Y$ Gaussiana $N(\mu, \sigma^2)$ la rappresentazione $\sigma X + \mu$, con $X$ Gaussiana
standard. Ad esempio, in particolare
\[ P(a < Y < b) = P \left( \frac{a-m}{\sigma} < X < \frac{b-m}{\sigma} \right) \]

\begin{theorem}[Riproducibilità]
	Sia $Y$ una Gaussiana $N(\mu, \sigma^2)$, sia $V = \alpha Y + \beta$, con
	$\alpha, \beta \in \R$ e $\alpha \neq 0$. Allora $V$ è una Gaussiana
	$N(\alpha \mu + \beta, \alpha^2 \sigma^2)$.
\end{theorem}