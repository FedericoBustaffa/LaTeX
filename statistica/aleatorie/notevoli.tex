\section{Variabili aleatorie notevoli}
In questa sezione trattiamo le distribuzione di probabilità su $\R$ indispensabili per trattare
ogni applicazione. Si tratta sempre di variabili discrete o definite tramite densità.

\subsection{Variabili binomiali}
Consideriamo come primo caso le \textbf{variabili binomiali} prendendo $n$ prove ripetute, con
esito (per ciascuna prova) successo o insuccesso (schema di Bernoulli) e sia $p$ la probabilità di
successo (nella singola prova).
\[ P(a_1, \dots, a_n) p^{\# \{i | a_i=1\}} \cdot (1-p)^{\# \{ i | a_i=0 \}} \]
Sia $X$ la variabile aleatoria che conta il numero di successi, ossia
\[ X(a_1, \dots, a_n) = \sum_{i=1}^n a_i \]
Come possiamo notare $X$ è discreta a valori in $\{0,1,2,\dots,n\}$ e ha funzione di massa
\begin{equation}\label{eq: binom}
	P_X(h) = P(X = h) = \binom{n}{h} \cdot p^h \cdot (1-p)^{n-h}
\end{equation}
con $h \in \{ 0, 1, \dots, n \}$. Possiamo tradurre tutto questo nella probabilità che abbiamo di
avere $h$ successi.

Una variabile aleatoria avente come funzione di massa \ref{eq: binom} è detta
\textbf{variabile aleatoria binomiale} di parametri $n \in \N^+$ e $p \in (0,1)$ e la si indica
con $B(n, p)$.

\begin{observation}
	Per $n=1$ si parla di variabile aleatoria di Bernoulli e si indica con $B(p)$.
\end{observation}

\begin{example}
	Su 5 lanci di un dado equilibrato, qual è la probabilita che il 6 appaia almeno 2 volte?

	Per prima cosa definiamo $X$ come il numero di volte esce 6 nelle 5 prove. Vogliamo calcolare
	la probabilità che il 6 esca almeno 2 volte in 5 lanci:
	\begin{align*}
		P(X \geq 2) = & 1 - (P(X = 0) + P(X = 1))                               \\
		=             & 1 - \binom{5}{0} \cdot \left(\frac{1}{6}\right)^0 \cdot
		\left(1 - \frac{1}{6}\right)^{6} - \binom{5}{1} \left(\frac{1}{6}\right)^1 \cdot
		\left(1 - \frac{1}{6}\right)^5                                          \\
		=             & 1 - \left( \frac{5}{6} \right)^6 - 5 = \dots
	\end{align*}
\end{example}

\subsection{Variabili geometriche}
Rimaniamo nel contesto delle prove ripetute indipendenti con esito successo o insuccesso. Sia $X$
la variabile aleatoria che rappresenta l'istante del primo successo (l'istante è il numero della
prova) del primo successo.

Come possiamo notare, $X$ è discreta a valori in $\N^+ = \{ 1, 2, \dots \}$ e la sua funzione di
massa vale
\begin{equation}\label{eq: geom}
	P(X = h) = (1-p)^{h-1} \cdot p
\end{equation}
Questo ci dice la probabilità che abbiamo di ottenere il primo successo dopo $h$ tentativi.

Una variabile aleatoria avente \ref{eq: geom} come funzione di massa è detta
\textbf{variabile aleatoria geometrica} di parametro $p \in (0,1)$ e la si indica con $G(p)$.

\begin{proposition}[Assenza di memoria]
	Data una variabile geometrica di parametro $p$, per ogni $n,h \in \N^+$, vale
	\[ P(X = n + h | X > n) = P(X = h) \]
	La probabilità di successo dopo $h$ prove non cambia sapendo che il successo non si è
	verificato nelle prime $n$ prove.
\end{proposition}