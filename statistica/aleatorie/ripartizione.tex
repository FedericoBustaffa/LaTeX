\section{Funzione di ripartizione e quantili}
Siano $X : \Omega \to \R$ una variabile aleatoria e $P_X$ la sua legge. Per studiare $P_X$, si
introduce una funzione che codifica tutte informazioni ad essa associate, ossia la funzione di
ripartizione.

\subsection{Funzione di ripartizione}
\begin{definition}
	Si chiama \textbf{funzione di ripartizione} di $X$ la funzione $F_X : \R \to [0,1]$, e ha
	valore
	\[ F_X(x) = P (X \leq x) = P_X ((-\infty, x]) \]
	In altre parole, la funzione di ripartizione di $X$ indica la probabilità che tale variabile
	aleatoria assuma valori inferiori ad un un certo valore $x$.
\end{definition}

La funzione di ripartizione $F_X$ dipende solo dalla legge $P_X$ di $X$. Questo significa che,
avere una variabile aleatoria con la stessa legge di $X$ implica avere anche la stessa funzione
di ripartizione. Vediamo ora alcune proprietà della funzione di ripartizione:
\begin{itemize}
	\item $F_X$ è non decrescente, ossia se $x \leq y$ allora
	      \[ F_X(x) = P(X \leq x) \leq P(X \leq y) = F_X(y) \]
	\item Esistono i limiti
	      \begin{gather*}
		      \lim_{x \to -\infty} F_X(x) = 0 \\
		      \lim_{x \to +\infty} F_X(x) = 1
	      \end{gather*}
	\item $F$ è continua a destra, ossia se $x_n$ è una successione che converge ad $x$ con
	      $x_n \geq x$, allora $F_X(x_n)$ converge ad $F(x)$.
\end{itemize}

\begin{proposition}
	Data una funzione $F : \R \to [0,1]$ con le proprietà sopra elencate, esiste ed è unica la
	probabilità $Q$ tale che $F$ sia la funzione di ripartizione di $Q$, cioè di una variabile
	aleatoria $X$ di legge $P_X = Q$. Questo equivale a dire che
	\[ F(x) = Q((-\infty, x]) = P(X \leq x) \]
	per ogni $x \in \R$. Di conseguenza due variabili aleatorie $X$ e $Y$ che hanno la stessa
	funzione di ripartizione hanno anche la stessa legge.
\end{proposition}

\begin{theorem}
	Data la variabile aleatoria $X$, la sua legge $P_X$ e la sua funzione di ripartizione $F_X$, vale
	che
	\[ P(a < X \leq b) = F_X (b) - F_X(a) \]
	Questa formula è utile per calcolare $P(a < X \leq b)$ quando si conoscono i valori di $F_X$
	almeno in modo approssimato.
	\begin{proof}
		La dimostrazione è molto semplice, basta notare che
		\[ P(a < X \leq b) = P(X \leq b) - P(X \leq a) = F(b) - F(a) \]
		Osserviamo anche che
		\begin{itemize}
			\item Per $a = -\infty$ otteniamo $P(X \leq b) = F(b)$
			\item Per $b = +\infty$ otteniamo $P(X > a) = 1 - F(a)$
		\end{itemize}
	\end{proof}
\end{theorem}

La funzione di ripartizione di una variabile aleatoria discreta (che assume valori
$x_1, x_2, \dots$) è una funzione costante a tratti, ossia è costante tra due punti $x_i$ e in ogni
punto $x_i$ esibisce un salto di ampiezza $P_X(x_i) = P(X = x_i)$. Una funzione di ripartizione
si scrive in questo modo
\begin{align*}
	F_X(x) & = P(X \leq x)                              \\
	       & = P_X ((-\infty, x])                       \\
	       & = \sum_{i, x_i \in (-\infty, x]} P_X (x_i) \\
	       & = \sum_{i, x_i \leq x} P_X (x_i)
\end{align*}
Graficamente una generica funzione di ripartizione appare in questo modo
\begin{center}
	\begin{tikzpicture}
		\begin{axis}[
				axis lines = center,
				width = 10cm,
				height = 5cm,
				font = \small,
				ymin=-0.2, ymax=1.2,
				xmin=-1.3, xmax=4,
				ytick=\empty
			]
			\addplot [thick, red, domain={-2 : -0.7}] {0};
			\draw [thick, red] {(-0.7, 0) -- (-0.7, 0.2)};
			\addplot [thick, red, domain={-0.7 : 1.2}] {0.2};
			\draw [thick, red] {(1.2, 0.2) -- (1.2, 0.4)};
			\addplot [thick, red, domain={1.2 : 1.6}] {0.4};
			\draw [thick, red] {(1.6, 0.4) -- (1.6, 0.6)};
			\addplot [thick, red, domain={1.6 : 2.1}] {0.6};
			\draw [thick, red] {(2.1, 0.6) -- (2.1, 0.8)};
			\addplot [thick, red, domain={2.1 : 3.4}] {0.8};
			\draw [thick, red] {(3.4, 0.8) -- (3.4, 1)};
			\addplot [thick, red, domain={3.4 : 4}] {1};
		\end{axis}
	\end{tikzpicture}
\end{center}
Per quanto riguarda invece le variabili aleatorie con densità $f$, abbiamo che la loro funzione di
ripartizione soddisfa
\[ F_X(x) = P(X \leq x) = \int_{-\infty}^x f(y) dy \]
In particolare $F_X$ è continua (senza salti). Notiamo anche che se $f$ è continua a tratti, cioè
se $F_X$ è di classe $C^1$ a tratti, allora $f$ si ottiene derivando $F_X$
\[ f(x) = \frac{d}{dx} F_X (x) \]
per ogni $x$ in cui $F_X$ è derivabile.

\begin{observation}
	Esistono variabili aleatorie con funzione di ripartizione continua ma che comunque non
	ammettono densità.
\end{observation}

\subsection{Quantili}
Intuitivamente, dato $\beta \in (0,1)$, un $\beta$-quantile è un numero $r_\beta \in \R$ tale che
la probabilità che la variabile aleatoria $X$ che stiamo considerando sia minore di $r_\beta$ è
proprio $\beta$. Vale quindi
\[ F_X (r_\beta) = P(X \leq r_\beta) = \beta \]
Tuttavia può non esistere un tal $\beta$, oppure se esiste, può non essere unico. Dobbiamo quindi
trovare una definizione diversa.

\begin{definition}
	Data una variabile aleatoria $X$ e un $\beta \in (0,1)$, si chiama $\beta$-quantile, un numero
	$r \in \R$ tale che
	\[ P(X \leq r) \geq \beta \quad \land \quad P(X \geq r) \geq 1 - \beta \]
	tale definizione dipende solo dalla legge $P_X$.
\end{definition}

Per calcolare il $\beta$-quantile nel caso di $X$ discreta dobbiamo prima ordinare gli $x_i$ e da
qui distinguiamo 2 casi:
\begin{itemize}
	\item Non esite $x_i$ tale che $F_X(x_i) = \beta$. Abbiamo quindi che $r_\beta$ equivale al più
	      piccolo degli $x_i$ tale che $F_X(x_i) \geq \beta$.
	\item Esiste $x_i$ tale che $F_X(x_i) = \beta$. Abbiamo quindi che ogni $r \in [x_i, x_{i+1}]$
	      è un $\beta$ quantile. Per convenzione si prende come quantile il punto medio
	      dell'intervallo.
\end{itemize}
Per calcolare il $\beta$-quantile nel caso di $X$ con densità distinguiamo ancora 2 casi:
\begin{itemize}
	\item $F_\beta$ è strettamente crescente allora esiste ed è unico $r_\beta$ tale che
	      \[ F_X(r_\beta) = P(X \leq r_\beta) = \beta \]
	      e $r_\beta$ è l'unico $\beta$-quantile.
	\item $F_X^{-1} (\beta)$ è un intervallo, ossia $F_\beta$ è costante su un intervallo, allora
	      anche in questo caso, per convenzione si prende come $\beta$-quantile l'estremo sinistro
	      dell'intervallo, ossia
	      \[ r_\beta = \inf \{ r \in \R | F_X(r) \geq \beta \} \]
\end{itemize}
In generale, data una densità $f$, l'area sottesa fino al punto $r_\beta$ vale $\beta$.

\begin{observation}
	Nel contesto dell'esempio dell'estrazione da una popolazione, la funzione di ripartizione è la
	funzione di ripartizione in cui il campione è sostituito da tutta la popolazione.
\end{observation}