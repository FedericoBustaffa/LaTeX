\section{Valore atteso, varianza e momenti}
Abbiamo definito la media campionaria di un insieme di dati come la loro media aritmetica
\[ \tilde{x} = \sum_{i=1}^n \frac{x_i}{n} \]
per dare una misura del \emph{centro} della distribuzione dei dati. Abbiamo poi definito la
varianza campionaria come misura della \emph{dispersione} dei dati. Le stesse idee sono alla base
dei concetti di \textbf{valore atteso} e \textbf{varianza} di una variabile aleatoria, quest'ultimo
inteso come oggetto matematico che si propone di descrivere la \emph{casualità} dell'esito di un
esperimento.

