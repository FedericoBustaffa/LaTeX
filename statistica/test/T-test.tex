\section{T-test}
Vogliamo ora formulare un test per la media di una popolazione Gaussiana senza però conoscere la
varianza. Sia $X \sim N(\mu, \sigma^2)$ con $\sigma^2$ non noto e sia $X_1, \dots, X_n$ un campione
\iid di $X$.

Il ragionamento è analogo allo $z$-test ma considerando come statistica
\[ \sqrt{n} \cdot \frac{\overline{X} - \mu}{S} \]
che, come abbiamo visto in precedenza, si tratta di una distribuzione $t$ di Student a $n-1$ gradi
di libertà, con
\[ S^2 = \frac{1}{n-1} \cdot \sum_{i=1}^n \left( X_i - \overline{X} \right)^2 \]
varianza campionaria. Una volta definito questo ci basta usare i quantili della distribuzione $t$
di Student invece di quelli della Gaussiana.

\subsection{Test bilatero}
Poniamoci nell'ipotesi
\[ H_0: \mu = \mu_0 \quad H_1: \mu \neq \mu_0 \]

\subsubsection{Formulazione del test}
Cerchiamo una regione critica della forma
\[ C = \left\{ |\overline{X} - \mu_0| > d \right\} \]
Per farlo imponiamo quindi il livello $\alpha$
\begin{align*}
	P_{\mu_0} (C) =                          & \alpha \\
	P_{\mu_0} (|\overline{X} - \mu_0| > d) = &
	P_{\mu_0} \left( \frac{\overline{X} - \mu_0}{S} \cdot \sqrt{n} >
	\frac{\sqrt{n}}{S} \cdot d \right)                \\
	=                                        &
	P \left( |T_{n-1}| > \frac{\sqrt{n}}{S} \cdot d \right)
\end{align*}