\section{Confronto tra due campioni statistici}
Poniamoci per prima cosa nel caso in cui abbiamo due campioni che non sono indipendenti l'uno
dall'altro e cerchiamo quindi di inquadrare la situazione tramite un esempio.

\begin{example}
	Viene pubblicizzata una cura dimagrante che promette 7kg in meno in 7 giorni. Per verificare
	la plausibilità di tale affermazione vengono misurati 7 pazienti prima e dopo la cura ottenendo
	i seguenti valori
	\begin{center}
		\begin{tabular}{| c | c | c | c | c | c | c | c |}
			\hline
			Paziente   & 1  & 2  & 3  & 4  & 5  & 6  & 7  \\
			\hline
			Peso prima & 72 & 89 & 94 & 77 & 86 & 91 & 83 \\
			\hline
			Peso dopo  & 68 & 83 & 90 & 71 & 80 & 88 & 74 \\
			\hline
			Differenza & 4  & 6  & 4  & 6  & 6  & 3  & 9  \\
			\hline
		\end{tabular}
	\end{center}
	Vogliamo verificare se l'affermazione \emph{"la riduzione media è di almeno 7kg"} è plausibile
	oppure no. Se indichiamo con $X_i$ il peso dell'$i$-esimo paziente prima della cura e con $Y_i$
	il peso dell'$i$-esimo paziente dopo la cura possiamo valutare direttamente il campione dei
	$V_i = X_i - Y_i$, che in assenza di ulteriori indicazioni possiamo supporre Gaussiano con
	varianza sconociuta. Da notare che
	\[ \mu = \E[V_i] = \E[X_i] - \E[Y_i] \]
	è la diminuzione media del peso prima e dopo la cura. Poniamo ora nelle ipotesi
	\[ H_0 : \mu \geq 7 \qquad H_1 : \mu < 7 \]
	Siamo quindi nel caso di popolazione Gaussiana con varianza non nota ed è dunque naturale
	effettuare un $T$-test unilatero. Usiamo quindi la statistica
	\[ \sqrt{7} \cdot \frac{\bar{V} - 7}{S_V} \sim T_6 \]
	Dato che non viene fornito un livello $\alpha$ per il test calcoliamo direttamente il $p$-value
	\begin{align*}
		P \left( \sqrt{7} \cdot \frac{\bar{V} - 7}{S_V} \leq
		\sqrt{7} \cdot \frac{\bar{v} - 7}{s_V} \right) = &
		P \left( T \leq \sqrt{7} \cdot \frac{\bar{v} - 7}{s_V} \right)                   \\
		=                                                & P \left( T \leq -2.09 \right) \\
		=                                                & F_6 (-2.09) = 0.04
	\end{align*}
	Dato il $p$-value abbastanza basso possiamo dire che l'ipotesi $H_0$ è poco plausibile.
\end{example}

\subsection{Confronto tra due campioni indipendenti}
Se invece ci trovassimo davanti un caso in cui i due campioni sono indipendenti l'uno dall'altro ed
entrambi seguono delle distribuzioni Gaussiane di varianza non nota ma uguale si può ricorrere al
seguente teorema.

\begin{theorem}\label{th: confronto}
	Sia $X_1, \dots, X_n$ un campione i.i.d. di $X \sim N(\mu_1, \sigma^2)$ e $Y_1, \dots, Y_k$ un
	campione i.i.d. di $Y \sim N(\mu_2, \sigma^2)$, allora la variabile aleatoria
	\[
		T_{n,k} = \frac{(\bar{X} - \bar{Y}) - (\mu_1 - \mu_2)}
		{\sqrt{\sum_{i=1}^n (X_i - \bar{X})^2 + \sum_{j=1}^k (Y_j - \bar{Y})^2}} \cdot
		\frac{\sqrt{n+k-2}}{\sqrt{\frac{1}{n} + \frac{1}{k}}}
	\]
	ha densità $t$ di Student a $n+k-2$ gradi di libertà.
\end{theorem}

Si può quindi formulare un $T$-test su $\mu_1 - \mu_2$ tramite la statistica appena descritta dal
teorema.

\begin{example}
	Vengono misurate le lunghezze delle tibie di uomini adulti provenienti da reperti di tombe
	etrusche: dal sito di Cerveteri vengono effettuate 13 misurazioni ottenendo
	\[ \bar{X} = 47.2 \qquad S_X^2 = 7.98 \]
	mentre dal sito di Ladispoli vengono effettuate 8 misurazioni dalle quali si ottiene
	\[ \bar{Y} = 44.9 \qquad S_Y^2 = 8.85 \]
	Si può affermare che la differenza sia una semplice fluttuazione statistica oppure si deve
	concludere che gli abitanti di Cerveteri erano veramente più alti?

	Iniziamo con l'indicare tramite $X_i$ la lunghezza dell'$i$-esimo campione di Cerveteri e con
	$Y_j$ la lunghezza del $j$-esimo campione di Ladispoli. Supponiamo che i due campioni siano
	Gaussiani e che le loro varianze siano uguali.
	\[ \mu_1 = \E[X_i] \qquad \mu_2 = \E[Y_j] \]
	e dunque $\mu = \mu_1 - \mu_2$. Poniamoci quindi nelle ipotesi
	\[ H_0 : \mu \leq 0 \qquad H_1 : \mu > 0 \]
	Formuliamo quindi un $T$-test e per il teorema \ref{th: confronto} abbiamo che
	\begin{align*}
		T = & \frac{(\bar{X} - \bar{Y})}
		{\sqrt{(n-1) \cdot S_X^2 + (k-1) \cdot S_Y^2}} \cdot
		\frac{\sqrt{n+k-2}}{\sqrt{\frac{1}{n} + \frac{1}{k}}} \\
		=   & \frac{(\bar{X} - \bar{Y})}
		{\sqrt{12 \cdot S_X^2 + 7 \cdot S_Y^2}} \cdot
		\frac{\sqrt{19}}{\sqrt{\frac{1}{13} + \frac{1}{8}}}
	\end{align*}
	ha distribuzione $t$ di Student a $19$ gradi di libertà. La regione critica di livello $\alpha$
	è dunque
	\[ C = \left\{ T > \tau_{1 - \alpha, 19} \right\} \]
	Passiamo ora al calcolo del $p$-value
\end{example}