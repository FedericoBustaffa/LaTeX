\section{Test per la media di un campione Bernoulliano}
Dati $X \sim B(p)$ e $X_1, \dots, X_n$ un campione i.i.d. di $X$, vogliamo formulare un test per la
media $p$. Come per gli intervalli di fiducia usiamo il fatto che, per $n$ grande, la statistica
\[ \sqrt{n} \cdot \frac{\bar{X} - p}{\sqrt{p \cdot (1-p)}} \sim N(0,1) \]
per il \hyperref[th: tcl]{teorema centrale del limite}. Fatte tali considerazioni è naturale fare
uso di uno \hyperref[sec: z-test]{$Z$-test} per la statistica scritta sopra.

\subsection{Test bilatero}
Poniamoci nell'ipotesi
\[ H_0: p = p_0 \qquad H_1: p \neq p_0 \]

\subsubsection{Formulazione del test}
La regione critica approssimata sarà della forma
\[
	C = \left\{ \sqrt{n} \cdot \frac{|\bar{X} - p_0|}{\sqrt{p_0 \cdot (1-p_0)}} >
	q_{1 - \frac{\alpha}{2}}\right\}
\]

\subsubsection{Calcolo del p-value}
Come al solito calcoliamo la probabilità, sotto l'ipotesi nulla, che la statistica assuma un valore
più estremo del valore assunto dal campione.

\begin{align*}
	P_{p_0} \left( \sqrt{n} \cdot \frac{|\bar{X} - p_0|}{\sqrt{p_0 \cdot (1-p_0)}} >
	\sqrt{n} \cdot \frac{|\bar{x} - p_0|}{\sqrt{p_0 \cdot (1-p_0)}} \right) = &
	P \left( |Z| > \sqrt{n} \cdot \frac{|\bar{x} - p_0|}{\sqrt{p_0 \cdot (1-p_0)}} \right) \\
	=                                                                         &
	2 \cdot \left( 1 - \Phi \left( \sqrt{n} \cdot
		\frac{|\bar{x} - p_0|}{\sqrt{p_0 \cdot (1-p_0)}} \right) \right)
\end{align*}

\begin{example}
	Nel caso di un controllo qualità su un certo tipo di pezzi, vengono estratti e verificati 1000
	pezzi. Vogliamo formulare formulare un test di livello $\alpha = 0.05$ per decidere se
	l'ipotesi $H_0 : p \leq 0.02$ è plausibile o no, dove $p$ è la percentuale di pezzi difettosi,
	e vogliamo applicarlo nel caso vengano rilevati 25 pezzi difettosi su 1000.

	La regione critica è
	\begin{align*}
		\left\{ \sqrt{n} \cdot \frac{\bar{X} - p_0}{\sqrt{p_0 \cdot (1-p_0)}}
		> q_{1-\alpha} \right\} = &
		\left\{ \sqrt{1000} \cdot \frac{\bar{X} - 0.02}{\sqrt{0.02 \cdot 0.98}} > 1.64 \right\} \\
		=                         & \left\{ \bar{X} - 0.02 > 0.0073 \right\}
	\end{align*}
	Se il numero di pezzi difettosi è 25 su 1000 allora $\bar{x} = 25 / 1000 = 0.025$ e quindi
	\[ 0.025 - 0.02 = 0.05 \leq 0.0073 \]
	Di conseguenza $p \leq 0.02$ è plausibile a livello 0.05. Calcoliamo ora il $p$-value che è
	\begin{align*}
		P_{p_0} \left( \sqrt{n} \cdot \frac{\bar{X} - p_0}{\sqrt{p_0 \cdot (1-p_0)}} >
		\sqrt{n} \cdot \frac{\bar{x} - p_0}{\sqrt{p_0 \cdot (1-p_0)}} \right) = &
		P \left( Z > \sqrt{n} \cdot \frac{\bar{x} - p_0}{\sqrt{p_0 \cdot (1-p_0)}} \right)    \\
		=                                                                       &
		1 - \Phi \left( \sqrt{n} \cdot \frac{\bar{x} - p_0}{\sqrt{p_0 \cdot (1-p_0)}} \right) \\
		=                                                                       &
		1 - \Phi(1.13) = 0.129
	\end{align*}
	Ne deduciamo che non c'è grossa evidenza contro $H_0 : p \leq 0.02$.
\end{example}