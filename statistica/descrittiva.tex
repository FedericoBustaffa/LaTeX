\chapter{Statistica descrittiva}

Il corso tratterà tre principali argomenti: \textbf{statistica descrittiva}, \textbf{probabilità} e
\textbf{inferenza statistica}.

Questa prima parte di \emph{statistica descrittiva} tratterà l'analisi di dati senza la costruzione di un modello
d'interpretazione.

Due dei concetti preliminari e fondamentali sono quelli di \textbf{popolazione} e \textbf{campione}:
\begin{itemize}
	\item \textbf{Popolazione}: cardinalità dell'insieme che stiamo considerando.
	\item \textbf{Campione}: cardinalità di un sottoinsieme più piccolo dell'insieme che stiamo considerando.
\end{itemize}

\begin{example}
	Gli italiani che hanno partecipato alle ultime votazioni (45 milioni circa) sono una \emph{popolazione}. Quando si
	fa un sondaggio elettorale si prende in considerazione un \emph{campione} della popolazione (per esempio qualche
	migliaio di votanti).
\end{example}

Altri due concetti di base sono quelli di \textbf{frequenza assoluta} e \textbf{frequenza relativa}:
\begin{itemize}
	\item \textbf{Frequenza assoluta}:
	\item \textbf{Frequenza relativa}:
\end{itemize}