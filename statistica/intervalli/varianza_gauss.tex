\section{Varianza di un campione Gaussiano}
Sia $X_1, \dots, X_n$ un campione i.i.d. di $X \sim N(\mu, \sigma^2)$, cerchiamo un intervallo
per la varianza $\sigma^2$, occupandoci principalmente di intervalli unilateri. Come abbiamo visto,
lo stimatore della varianza $\sigma^2$ è la varianza campionaria
\[ S_n^2 = \frac{1}{n-1} \cdot \sum_{i=1}^n (X_i - \overline{X}_n)^2 \]

\begin{proposition}
	La variabile aleatoria
	\[ \frac{(n-1) \cdot S_n^2}{\sigma^2} \sim \chi_{n-1}^2 \]
	ha distribuzione Chi-Quadro a $n-1$ gradi di libertà, la quale si indica con $\chi_k^2$ e
	ha densità
	\[
		f_k(x) = \begin{cases}
			c_k \cdot x^{\frac{k}{2} - 1} \cdot e^{-x/2} & x > 0    \\
			0                                            & x \leq 0
		\end{cases}
	\]
\end{proposition}

La distribuzione $\chi_k^2$ è asimmetrica che, all'aumentare di $k$, si schiaccia verso valori più
grandi. Si può dimostrare che, prese $k$ variabili aleatorie Gaussiane standard indipendenti, vale
\[ \sum_{i=1}^k Z_i^2 \sim \chi_k^2 \]
e da questo fatto si può poi dimostrare che se $Q_k \sim \chi_k^2$, allora
\[ \frac{C_k}{k} \to \E[Z_i^2] = 1 \quad \text{ e } \quad \frac{C_k - k}{\sqrt{2k}} \to N(0, 1) \]

\begin{proposition}
	Dato $X_1, \dots, X_n$ un campione i.i.d. di $X \sim N(\mu, \sigma^2)$ e dato $\alpha \in (0,1)$,
	l'intervallo aleatorio
	\begin{equation*}
		\left( 0, \quad \frac{(n-1) \cdot S_n^2}{\chi_{\alpha, n-1}^2} \right] =
		\left( 0, \quad \frac{\sum_{i=1}^n (X_i - \overline{X}_n)^2}{\chi_{\alpha, n-1}^2} \right]
	\end{equation*}
	è un intervallo di fiducia per $\sigma^2$ con livello di fiducia $1-\alpha$, con
	$\chi_{\alpha, n-1}^2$ che è il quantile di ordine $\alpha$ di $\chi_{n-1}^2$. Analogamente
	\begin{equation*}
		\left[ \frac{(n-1) \cdot S_n^2}{\chi_{1-\alpha, n-1}^2}, \quad +\infty \right) =
		\left[ \frac{\sum_{i=1}^n (X_i - \overline{X}_n)^2}{\chi_{1-\alpha, n-1}^2}, \quad +\infty \right)
	\end{equation*}
	è un intervallo di fiducia per $\sigma^2$ con livello di fiducia $1-\alpha$.
\end{proposition}