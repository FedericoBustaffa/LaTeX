\section{Varianza di un campione Gaussiano}
Sia $X_1, \dots, X_n$ un campione i.i.d. di $X \sim N(\mu, \sigma^2)$ cerchiamo un intervallo per
la varianza $\sigma^2$, occupandoci principalmente di intervalli unilateri. Come abbiamo visto,
lo stimatore della varianza $\sigma^2$ è la varianza campionaria
\[ S_n^2 = \frac{1}{n-1} \cdot \sum_{i=1}^n (X_i - \overline{X_n})^2 \]

\begin{proposition}
	La variabile aleatoria
	\[ \frac{(n-1) \cdot S_n^2}{\sigma^2} \]
	ha distribuzione $\chi_{n-1}^2$ dove $\chi_k^2$ è la distribuzione avente densità
	\[
		f_k(x) = \begin{cases}
			c_k \cdot x^{\frac{k}{2} - 1} \cdot e^{-x/2} & x > 0    \\
			0                                            & x \leq 0
		\end{cases}
	\]
\end{proposition}

\begin{proposition}
	Dato $X_1, \dots, X_n$ un campione i.i.d. di $X \sim N(\mu, \sigma^2)$ e dato $\alpha \in (0,1)$,
	l'intervallo aleatorio
	\[
		\left( 0, \frac{(n-1) \cdot S_n^2}{\chi_{\alpha, n-1}^2} \right] =
		\left( 0, \frac{\sum_{i=1}^n (X_i - \overline{X_n})^2}{\chi_{\alpha, n-1}^2} \right]
		\]
		è un intervallo di fiducia per $\sigma^2$ con livello di fiducia $1-\alpha$, con
		$\chi_{\alpha, n-1}^2$ che è il quantile di ordine $\alpha$ di $\chi_{n-1}^2$. Analogamente
		\[ \left[ \frac{(n-1) \cdot S_n^2}{\chi_{1-\alpha, n-1}^2}, +\infty \right) \]
	è un intervallo di fiducia per $\sigma^2$ con livello di fiducia $1-\alpha$.
\end{proposition}