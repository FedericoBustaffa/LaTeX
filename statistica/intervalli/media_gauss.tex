\section{Media di un campione Gaussiano}
Negli esempi che vedremo di seguito supponiamo di avere un campione statistico e supponiamo di
sapere che la sua distribuzione sia Gaussiana di parametri $\mu$ e $\sigma^2$. Nel nostro caso ci
poniamo il problema di cercare degli intevalli di fiducia per $\mu$ che è non nota. Come sappiamo,
la media campionaria
\[ \overline{X}_n = \frac{X_1 + \dots + X_n}{n} \]
è uno stimatore di $\E[X] = \mu$, quindi è naturale cercare un intervallo di fiducia della forma
\[ I = [\overline{X}_n - d, \overline{X}_n + d] = [\overline{X}_n \pm d] \]
Dobbiamo quindi determinare $d$ tale che valga
\[ P(\mu \in I) \geq 1 - \alpha \]

\begin{definition}
	Se $[\overline{X}_n \pm d]$ è un intervallo di fiducia per la media $\mu$, allora $d$ è detta
	\textbf{precisione della stima} ($d$ piccola corrisponde ad un'alta precisione) e
	$\frac{d}{\overline{X}_n}$ è detta \textbf{precisione relativa della stima}.
\end{definition}

Per evitare di confonderci possiamo vedere $d$ come l'errore che \emph{possiamo} commettere, più
questo è piccolo, più vincoleremo la nostra stima ad essere precisa. Dato che però l'intervallo è
aleatorio, l'osservazione appena fatta si traduce in una maggiore possibilità di errore.

\begin{proposition}[Intervallo di fiducia per la media con varianza nota]
	Dato il parametro $\alpha \in (0,1)$ e assumendo $\sigma^2 > 0$ nota, l'intervallo aleatorio
	\[
		I = \left[ \overline{X}_n \pm \frac{\sigma}{\sqrt{n}}
			\cdot q_{1 - \frac{\alpha}{2}} \right]
	\]
	è un intervallo di fiducia per la media $\mu$ di livello $1 - \alpha$. Questo vale per il
	teorema di \hyperref[prop: riprod_gauss]{riproducibilità delle Gaussiane}. In particolare se
	$X_i \sim N(\mu, \sigma^2)$ sono indipendenti, allora
	\[
		\overline{X}_n = \frac{1}{n} \sum_{i=1}^n X_i \sim
		N \left( \mu, \frac{\sigma^2}{n} \right)
	\]
	\begin{proof}
		Dobbiamo dimostrare che
		\[
			1 - \alpha \leq P \left( \mu \in \left[ \overline{X}_n \pm
				\frac{\sigma}{\sqrt{n}} \cdot q_{1-\frac{\alpha}{2}} \right] \right)
		\]
		Partiamo con il calcolare la probabilità
		\[
			P \left( \mu \in \left[ \overline{X}_n \pm
				\frac{\sigma}{\sqrt{n}} \cdot q_{1-\frac{\alpha}{2}} \right] \right) =
			P \left( \left| \overline{X}_n - \mu \right| \leq
			\frac{\sigma}{\sqrt{n}} q_{1 - \frac{\alpha}{2}} \right)
		\]
		A questo punto, standardizziamo e otteniamo
		\begin{align*}
			P \left( |\overline{X}_n - \mu| \leq
			\frac{\sigma}{\sqrt{n}} q_{1 - \frac{\alpha}{2}} \right) = &
			P \left( \frac{|\overline{X}_n - \mu|}{\sigma / \sqrt{n}} \leq
			\frac{\frac{\sigma}{\sqrt{n}} \cdot q_{1-\frac{\alpha}{2}}}{\sigma/\sqrt{n}} \right) \\
			=                                                          &
			P \left( |Z| \leq q_{1 - \frac{\alpha}{2}} \right)                                   \\
			=                                                          &
			P \left( -q_{1 - \frac{\alpha}{2}} \leq Z \leq q_{1-\frac{\alpha}{2}} \right)        \\
			=                                                          &
			\Phi \left( q_{1-\frac{\alpha}{2}} \right) -
			\Phi \left(-q_{1-\frac{\alpha}{2}} \right)                                           \\
			=                                                          &
			\Phi \left( q_{1-\frac{\alpha}{2}} \right) -
			\Phi \left(q_{\frac{\alpha}{2}} \right)
		\end{align*}
		Infine per la definizione di \hyperref[def: quantile]{quantile} abbiamo che
		\begin{align*}
			\Phi \left( q_{1-\frac{\alpha}{2}} \right) -
			\Phi \left(q_{\frac{\alpha}{2}} \right) = & 1 - \frac{\alpha}{2} -
			\frac{\alpha}{2}                                                   \\
			=                                         & 1 - \alpha
		\end{align*}
	\end{proof}
\end{proposition}

\begin{observation}
	La precisione $d$ della stima
	\begin{itemize}
		\item Cresce al crescere del livello di fiducia $1-\alpha$.
		\item Cresce al crescere di $\sigma^2$.
		\item Decresce al crescere di $n$.
	\end{itemize}
	In particolare, dalla formula
	\[ d = \frac{\sigma}{\sqrt{n}} \cdot q_{1 - \frac{\alpha}{2}} \]
	è possibile, dati $\alpha$ e $\sigma$, determinare la numerosità $n$ tale che l'intervallo
	abbia una data precisione e un dato livello di fiducia.
\end{observation}

\begin{example}
	Il peso di una specie di salmoni segue una distribuzione Gaussiana di media $\mu$ incognita e
	deviazione standard $\sigma = 0.4$ kg. Vengono pesati 10 esemplari e viene ricavato un peso medio
	di 3.58 kg.
	\begin{enumerate}
		\item Vogliamo determinare un intervallo di fiducia per la media $\mu$ di livello
		      $1 - \alpha = 0.95$.
		\item Mantenendo il livello di fiducia al 95\%, quanto grande deve essere $n$ affinché la
		      precisione $d$ della stima sia al massimo 0.1?
	\end{enumerate}
	Il campione che consideriamo è di $n=10$ salmoni, la variabile aleatoria $X$ ci dice
	il peso di un esemplare scelto a caso e $X_i$ è il peso dell'$i$-esimo esemplare del campione.
	Abbiamo poi che $1-\alpha = 0.95$ quindi $\alpha=0.05$ e quindi $1 - \frac{\alpha}{2} = 0.975$.
	Dalle tavole ricaviamo infine che $q_{0.975} = 1.96$. L'intervallo di fiducia cercato è quindi
	\[
		I = \left[ 3.58 \pm \frac{0.4}{\sqrt{10}} \cdot 1.96 \right]
		= [ 3.58 \pm 0.248 ] = [ 3.332, 3.828 ]
	\]
	Risolviamo ora il secondo punto imponendo $d \leq 0.1$ dove sappiamo $d$ essere
	$\frac{\sigma}{\sqrt{n}} \cdot q_{1-\frac{\alpha}{2}}$ e quindi abbiamo che
	\[
		\sigma \cdot q_{1-\frac{\alpha}{2}} \leq 0.1 \cdot \sqrt{n} \iff
		n \geq \frac{\sigma^2 q_{1-\frac{\alpha}{2}}^2}{0.1^2} \approx 61.46
	\]
	Otteniamo quindi che il campione deve avere numerosità $n \geq 62$.
\end{example}

Prima di proseguire dobbiamo introdurre la distribuzione di \textbf{Student}, utile per capire
meglio il risultato che seguirà. Tale distribuzione, a $k$ \textbf{gradi di libertà}, ha densità
\[
	f_{k} (x) = c_k \cdot
	\left( 1 + \frac{t^2}{k} \right)^{-\frac{k}{2} - \frac{1}{2}}
\]
Su questa distribuzione possiamo dire che ha una densità simmetrica \emph{"a campana"}, ha code
più pesanti di $N(0,1)$, ovvero
\[ P(T > \gamma) \geq P(Z > \gamma) \]
per $\gamma$ grande. Il fatto di avere code più pesanti implica che il quantile della $t$ di
Student è più grande di quello della Gaussiana. Per il calcolo dei quantili $\tau$ della
distribuzione $t$ di Student si ricorre, come per le Gaussiane a tavole o software. Ultima
considerazione è che per $k \to \infty$ abbiamo che $t_k$ tende in legge a $N(0,1)$.

\begin{proposition}[Intervallo di fiducia per la media con varianza non nota]
	In questo caso non possiamo usare $\sigma$ nella determinazione dell'intervallo di fiducia.
	Questo problema si risolve considerando lo stimatore di $\sigma^2$:
	\[ S_n^2 = \frac{1}{n-1} \cdot \sum_{i=1}^n \left( X_i - \overline{X}_n \right)^2 \]
	Consideriamo quindi la variabile aleatoria
	\[ T = \frac{\overline{X}_n - \mu}{S_n / \sqrt{n}} \]
	la quale ha distribuzione nota (per $X_1, \dots, X_n$ i.i.d. $\sim N(\mu, \sigma^2)$), che è la
	distribuzione $t$ di Student a $n-1$ gradi di libertà. Dato quindi $\alpha \in (0,1)$ e
	assumendo $\sigma^2 > 0$ non nota, l'intervallo aleatorio
	\[
		\left[ \overline{X}_n \pm
			\frac{S_n}{\sqrt{n}} \cdot \tau_{1 - \frac{\alpha}{2}, n-1} \right]
	\]
	è un intervallo di fiducia per la media $\mu$ di livello $1-\alpha$. Nella formula $\tau$ è il
	quantile $1-\frac{\alpha}{2}$ della $t$ di Student a $n-1$ gradi di libertà.
\end{proposition}

\begin{example}
	Consideriamo come prima il peso di una specie di salmoni che segue una distribuzione di media
	$\mu$ e di varianza $\sigma^2$ incognita. Vengono pesati 10 esemplari e viene ricavato un peso
	medio di 3.58 kg e deviazione standard campionaria di 0.4 kg.
	\begin{enumerate}
		\item Vogliamo determinare un intervallo di fiducia per la media $\mu$ di livello
		      $1 - \alpha = 0.95$.
		\item Mantenendo il livello di fiducia al 95\%, quanto grande deve essere $n$ affinché la
		      precisione $d$ della stima sia al massimo 0.1?
	\end{enumerate}
	Il procedimento è analogo a prima ma dobbiamo lavorare valori differenti, in particolare
	l'intervallo cercato avrà forma
	\[
		I = \left[ 3.58 \pm \frac{0.4}{\sqrt{10}} \cdot 2.262 \right]
		= [3.58 \pm 0.286] = [3.294, 3.866]
	\]
	come possiamo notare $d$ è più grande, questo è dovuto al fatto che il quantile della t di
	Student è maggiore della Gaussiana.

	Per il secondo punto notiamo che $d$ dipende da $S_n$ e quindi dipende dai dati del campione,
	il che vuol dire che se variamo $n$ variamo anche $S_n$. Non si può quindi determinare $n$ a
	priori ma si può dare una \emph{stima ragionevole} di $S_n$, ad esempio supponendo che sia
	limitata da $2 \cdot 0.4 = 0.8$, quindi calcoliamo $n$ basandoci su tale stima, cioè
	\[
		d = \frac{S_n}{\sqrt{n}} \cdot \tau_{1 - \frac{\alpha}{2}, n-1} \leq
		\frac{0.8}{\sqrt{n}} \cdot \tau_{1 - \frac{\alpha}{2}, n-1} \leq 0.1
	\]
	e, dopo aver raccolto di nuovo i dati dal campione di taglia $n$, verifichiamo che la $S_n$
	verifichi la stima che avevamo imposto o verifichiamo che $d$ rispetti la stima richiesta.
\end{example}

Per il momento abbiamo sempre considerato \textbf{intervalli bilateri}, ossia intervalli i cui
estremi sono entrambi variabili aleatorie. Possiamo considerare anche \textbf{intervalli unilateri}
della forma
\[ (-\infty, b(X_1, \dots, X_n)] \]

\begin{proposition}
	Dato $\alpha \in (0,1)$, gli intervalli aleatori
	\[
		\left( -\infty, \overline{X}_n + \frac{\sigma}{\sqrt{n}} q_{1-\alpha} \right],
		\quad
		\left[ \overline{X}_n - \frac{\sigma}{\sqrt{n}} q_{1-\alpha}, +\infty \right)
	\]
	sono intervalli di fiducia per la media $\mu$ di livello $1-\alpha$. Nel caso $\sigma^2$
	incognita, le formule diventano
	\[
		\left( -\infty, \overline{X}_n + \frac{S_n}{\sqrt{n}} \tau_{1-\alpha, n-1} \right],
		\quad
		\left[ \overline{X}_n - \frac{S_n}{\sqrt{n}} \tau_{1-\alpha, n-1}, +\infty \right)
	\]
\end{proposition}