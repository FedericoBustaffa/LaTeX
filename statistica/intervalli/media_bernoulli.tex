\section{Media di un campione Bernoulliano}
Sia $X \sim B(p)$ che rappresenta il verificarsi o meno di un \emph{successo} dove $p$ è la
probabilità del successo. Consideriamo un campione $X_1, \dots, X_n$ i.i.d. di $X$, abbiamo che
\[ X_1 + \dots + X_n \]
è il numero successi nel campione e che
\[ \bar{X}_n = \frac{X_1 + \dots + X_n}{n} \]
è la frequenza relativa del successo nel campione. Ricordiamo inoltre che
\[ X_1 + \dots + X_n \sim B(n,p) \quad \Rightarrow \quad \E[X] = \E[X_i] = p \]
e in particolare $\bar{X}_n$ è uno stimatore di $p$.

A questo punto è ragionevole cercare un intervallo di fiducia per $p$ del tipo
$[\bar{X}_n \pm d]$. In questo caso, dovremo usare i quantili della distribuzione binomiale,
i quali dipendono da $n$. Questo è possibile ma essendo la formula complessa, offriamo
un'alternativa che è valida per $n$ grande (in genere $n \geq 80$), la quale fa uso del
\hyperref[th: tcl]{teorema centrale del limite} poiché
\[
	\frac{X_1 + \dots + X_n - np}{\sqrt{n \cdot p \cdot (1-p)}} =
	\sqrt{n} \cdot \frac{\bar{X}_n - p}{\sqrt{p \cdot (1-p)}} \sim N(0,1)
\]
Dato che la varianza $\sigma^2 = p \cdot (1-p)$ dipende da $p$ che è incognita ma possiamo stimare
$p$ con $\bar{X}_n$ e quindi $p \cdot (1-p)$ con
\[ \bar{X}_n \cdot (1 - \bar{X}_n) \]
Si può dimostrare, come conseguenza del teorema centrale del limite, che
\[
	\frac{X_1 + \dots + X_n - np}{\sqrt{n \cdot \bar{X}_n \cdot (1 - \bar{X}_n)}} =
	\sqrt{n} \cdot \frac{\bar{X}_n - p}{\sqrt{\bar{X}_n \cdot (1 - \bar{X}_n)}}
\]
converge in legge ad una Gaussiana standard per $n \to +\infty$.

\begin{proposition}
	Dato $\alpha \in (0,1)$, l'intervallo aleatorio
	\[
		\left[
			\bar{X}_n \pm \sqrt{\frac{\bar{X}_n \cdot (1 - \bar{X}_n)}{n}}
			\cdot q_{1 - \frac{\alpha}{2}}
			\right]
	\]
	è un intervallo di fiducia per $p$ con livello di fiducia approssimativamente $1-\alpha$, più
	precisamente, si ha
	\[
		\lim_{n \to +\infty} P \left( p \in \left[ \bar{X}_n \pm
			\sqrt{\frac{\bar{X}_n \cdot (1 - \bar{X}_n)}{n}} \cdot
			q_{1 - \frac{\alpha}{2}} \right] \right) = 1 - \alpha
	\]
\end{proposition}

\begin{example}
	Si vuole condurre un sondaggio per determinare la percentuale di gradimento nei confronti del
	governo. Qual è il numero minimo di telefonta da effettuare affinché la precisione $d$ della
	stima sia inferiore all'1\%, con livello di fiducia del 95\%? La precisione è
	\[
		d = \sqrt{\frac{\bar{X}_n \cdot (1 - \bar{X}_n)}{n}} \cdot
		q_{1 - \frac{\alpha}{2}} = \sqrt{\frac{\bar{X}_n \cdot (1 - \bar{X}_n)}{n}} \cdot
		1.96
	\]
	che dipende da $\bar{X}_n$, non noto a priori. Sappiamo però che la funzione
	$x \cdot (1-x)$ definita in $[0,1]$ ha un massimo in $x=1/2$ che vale $1/4$, quindi
	\[ \bar{X}_n \cdot (1 - \bar{X}_n) \leq \frac{1}{4} \]
	e di conseguenza
	\[
		d \leq \sqrt{\frac{1/4}{n}} \cdot 1.96 = \frac{1}{2 \sqrt{n}} \cdot 1.96
	\]
	A questo punto otteniamo
	\[
		\frac{1}{2 \sqrt{n}} \cdot 1.96 \leq 0.01 \iff
		n \geq \frac{1.96^2}{4 \cdot 0.01^2} = 9604
	\]
	Dato che $n$ è grande vale l'approssimazione Gaussiana.
\end{example}