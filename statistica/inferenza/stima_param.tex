\section{Stima parametrica}
La stima parametrica si occupa di fornirci dei metodi per scegliere un buon stimatore. In
particolare ne vedremo due: il metodo di verosimiglianza e il metodo dei momenti.

Iniziamo con il considerare un campione statistico la cui legge di probabilità dipende da un
parametro $\theta \in \Theta$, nel quale le variabili possono essere discrete con funzione di
massa $p_\theta(x)$, oppure con densità $f_\theta(x)$.

\begin{definition}
	Si chiama \textbf{funzione di verosimiglianza} la funzione $L(\theta; \dots)$ definita, nel
	caso di variabili discrete da
	\[ L(\theta; x_1, \ldots, x_n) = \prod_{i=1}^n p_\theta(x_i) \]
	e nel caso di variabili con densità
	\[ L(\theta; x_1, \ldots, x_n) = \prod_{i=1}^n f_\theta(x_i) \]
\end{definition}

Si noti che nel caso discreto la verosimiglianza equivale alla densità congiunta di
$(X_1, \dots, X_n)$. Analoga interpretazione nel caso di variabili con densità.

\begin{definition}
	Si chiama \textbf{stima di massima verosimiglianza}, se esiste, una statistica campionaria,
	usualmente indicata $\hat{\theta} = \hat{\theta} (x_1, \dots, X_n)$, tale che valga
	l'eguaglianza
	\[ L(\hat{\theta}; x_1, \dots, x_n) = \max_{\theta \in \Theta} (L(\theta, x_1, \dots, x_n)) \]
	per ogni $(x_1, \dots, x_n)$.
\end{definition}

Nel caso discreto, se $x_1, \dots, x_n$ sono gli esiti la stima di massima verosimiglianza sceglie
un parametro $\hat{\theta}$ che \emph{massimizza} la probabilità degli esiti $x_1, \dots, x_n$
effettivamente ottenuti.