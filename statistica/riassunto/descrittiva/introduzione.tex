\part{Statistica descrittiva}
\chapter{Introduzione}
Il corso tratterà tre principali argomenti: \textbf{statistica descrittiva}, \textbf{probabilità}
e \textbf{inferenza statistica}. Questa prima parte di \emph{statistica descrittiva} tratterà
l'analisi di dati evitando di andare a costruire un modello d'interpretazione.

\section{Concetti di base}
\textbf{campione}:
\begin{itemize}
	\item \textbf{Popolazione}: cardinalità dell'insieme che stiamo considerando.
	\item \textbf{Campione}: cardinalità di un sottoinsieme più piccolo dell'insieme che stiamo
	      considerando.
\end{itemize}

\begin{itemize}
	\item \textbf{Frequenza assoluta}: il valore assoluto con il quale occorre un certo valore.
	\item \textbf{Frequenza relativa}: la percentuale con la quale un certo valore compare
	      all'interno dell'insieme dell'insieme considerato.
\end{itemize}
