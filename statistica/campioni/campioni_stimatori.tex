\part{Statistica inferenziale}

\chapter{Campioni di variabili aleatorie}
Lo scopo dell’\textbf{inferenza statistica} è l'analisi di un campione per ricavare informazioni
su un carattere di una intera popolazione. Ad esempio, un sondaggio sulle intenzioni di voto
raccoglie le risposte di un certo numero di intervistati per ricostruire gli orientamenti
elettorali dell’intera popolazione, in questo caso l’intero corpo elettorale.

L’assunzione di base della statistica inferenziale è che si possa descrivere la misura del carattere
desiderato su un individuo scelto casualmente con una variabile aleatoria, di cui vogliamo
determiare la legge, o informazioni su di essa.

\section{Campioni statistici e stimatori}
Il carattere che vogliamo studiare viene rappresentato con una variabile aleatoria
$X$, la cui distribuzione $P_X$ è secondo i casi parzialmente o del tutto sconosciuta. In altre
parole ripetiamo l'esperimento aleatorio $n$ volte e, per ciascuna ripetizione, misuriamo il
carattere in esame. In altre parole assumiamo di avere a disposizione $n$ variabili aleatorie
$X_1, \dots, X_n$, indipendenti e aventi la stessa legge di X, che rappresentano il
\textbf{campione} estratto dalla popolazione.

\begin{definition}\label{def: campione statistico}
	Data una popolazione in cui un carattere $X$ segue una certa distribuzione $P_X$, una famiglia
	finita di variabili aleatorie i.i.d. $X_1, \dots, X_n$ che hanno la stessa legge $P_X$ di $X$
	si dice campione statistico di taglia $n$ di $X$.
\end{definition}

\begin{observation}
	Le $X_i$ sono variabili aleatorie che esistono solo nel modello matematico. Con la notazione
	$x_i$ indichiamo invece l'\textbf{esito} dell'$i$-esima misurazione.
	\[ x_i = X_i(\omega) \]
	per un certo $\omega$ estratto dalla popolazione.
\end{observation}

Sia $(X_1, \dots, X_n)$ un campione i.i.d di $X$ di legge non nota $P_X$. Supponiamo che la
distribuzione $P_X$ sia parzialmente specificata, appartiene cioè ad una famiglia di probabilità
dipendenti da un parametro $\theta$ o da più parametri $\theta_1, \dots, \theta_n$, non noti.

\begin{example}
	L'altezza di una popolazione ha una distribuzione gaussiana di cui non conosciamo né media né
	varianza (i parametri sono media e varianza).
\end{example}

\subsection{Stimatori}\label{def: stimatore}
L'obbiettivo della stima parametrica è ricostruire il parametro o i parametri incogniti, a partire
dalle osservazioni, cioè in funzione del campione. Una funzione $g(X_1, \dots, X_n)$ di un campione
statistico è chiamata \textbf{statistica campionaria} (per esempio la media campionaria).

Uno \textbf{stimatore} di un parametro $\theta$ della distribuzione è una statistica campionaria,
cioè una funzione $g(X_1, \dots, X_n)$ del campione, atta a stimare $\theta$.

\begin{example}
	La media campionaria $\bar{X}_n$ è lo stimatore abituale del valore atteso $\E[X]$ e la
	varianza campionaria $S_n^2$ è lo stimatore abituale della varianza $\Var(X)$.
\end{example}

Poiché uno stimatore è una funzione del campione, anch'esso è una variabile aleatoria. In
particolare, realizzazioni diverse del campione $X_1, \dots, X_n$, portano a valori diversi dello
stimatore. Si pone quindi il problema di come scegliere un buon stimatore e di come valutarne la
\emph{bontà}.

\begin{definition}\label{def: stimatore corretto}
	Dato $\theta$ parametro della distribuzione e dato $X_1, \dots, X_n$ un campione della
	distribuzione, una statistica $g(X_1, \dots, X_n)$ si dice \textbf{stimatore corretto} o
	\textbf{non distorto} del parametro $\theta$ se
	\[ \E[g(X_1, \dots, X_n)] = \theta \]
	cioè la media dello stimatore è proprio il parametro $\theta$.
\end{definition}

\begin{proposition}
	Dati un campione i.i.d. $X_1, \dots, X_n$ con momento secondo finito, siano
	\[ \mu = \E[X_i] \quad \quad \sigma^2 = \Var(X_i) \]
	e siano
	\[ \bar{X} = \frac{X_1 + \dots + X_n}{n} \quad \quad
		S^2 = \dfrac{\sum_{i=1}^n (X_i - \bar{X})^2}{n-1}
	\]
	allora valgono
	\[
		\E[\bar{X}] = \mu \quad \quad
		\Var(\bar{X}) = \sigma^2 / n \quad \quad
		\E \left[ S^2 \right] = \sigma^2
	\]
	\begin{proof}
		Le prime due uguaglianze sono immediate dalla proprietà di linearità del valore atteso e
		dalle ipotesi di indipendenza. Per quanto riguarda l'ultima osserviamo che
		\[ \sum_{i=1}^n (X_i - \bar{X})^2 = \sum_{i=1}^n X_i^2 - n \cdot \bar{X}^2 \]
		Inoltre, per ogni $X_i$ si ha che
		\[ \E[X_i^2] = \Var(X_i) + \E[X_i]^2 = \sigma^2 + \mu^2 \]
		per cui, ricordando le prime due formule enunciate si conclude che
		\begin{align*}
			\E \left[ \sum_{i=1}^n (X_i - \bar{X})^2 \right]
			= & \sum_{i=1}^n \E \left[ X_i^2 \right] - n \cdot \E \left[ \bar{X}^2 \right] \\
			= & n \cdot (\sigma^2 + \mu^2) - n \cdot (\sigma^2 / n + \mu^2)                \\
			= & (n-1) \cdot \sigma^2
		\end{align*}
		da cui la tesi.
	\end{proof}
\end{proposition}

In altre parole media campionaria e varianza campionaria sono stimatori corretti,
rispettivamente di valore atteso e varianza.

\begin{definition}\label{def: stimatore consistente}
	Dato un parametro $\theta$ della distribuzione e un campione $X_1, \dots, X_n, \dots$ di
	infinite variabili aleatorie i.i.d, la successione di statistiche $g_n(X_1, \dots, X_n)$ si
	dice uno \textbf{stimatore consistente} di $\theta$ se, per $n \to +\infty$, allora
	$g_n(X_1, \dots, X_n)$ tende a $\theta$ in probabilità, cioè
	\[ \lim_{n \to +\infty} P(|g_n(X_1, \dots, X_n) - \theta| > \varepsilon) = 0 \]
	per ogni $\varepsilon > 0$. In altre parole, quando la taglia del campione diventa molto
	grande, $g_n(X_1, \dots, X_n)$ si avvicina con alta probabilità al valore $\theta$.
\end{definition}

La media campionaria è uno stimatore consistente del valore atteso per la
\hyperref[th: lgn]{legge dei grandi numeri} e la varianza campionaria è uno stimatore consistente
della varianza per il corollario \ref{cor: lgn}.

\begin{definition}\label{def: stimatore efficiente}
	Dato un parametro $\theta$ e un campione $X_1, \dots, X_n$ e dati due stimatori corretti
	$g(X_1, \dots, X_n)$ e $h(X_1, \dots, X_n)$, diciamo che $g(X_1, \dots, X_n)$ è
	\textbf{più efficiente} di $h(X_1, \dots, X_n)$ se
	\[ \Var(g(X_1, \dots, X_n)) \leq \Var(h(X_1, \dots, X_n)) \]
\end{definition}

Vogliamo cioè che la dispersione di $g(X_1, \dots, X_n)$ attorno a
$\E[g(X_1, \dots, X_n)] = \theta$ sia minore della dispersione di $h(X_1, \dots, X_m)$ attorno a
$\E[h(X_1, \dots, X_m)] = \theta$.

Per $n \geq m$, la media campionaria $\bar{X}_n$ è più efficiente della media campionaria
$\bar{X}_m$, infatti
\[ \Var(\bar{X}_n) = \frac{\sigma^2}{n} \leq \frac{\sigma^2}{m} = \Var(\bar{X}_m) \]
Similmente si può dimostrare che, al crescere di $n$, $S_n^2$ è sempre più efficiente.