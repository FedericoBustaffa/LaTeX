\section{Variabili aleatorie indipendenti}
Passiamo ora a parlare di indipendenza di variabili aleatorie partendo dalla definizione di
indipendenza

\begin{definition}
	Due variabili aleatorie $X$ e $Y$ su uno spazio di probabilità $(\Omega, \F, P)$ sono dette
	\textbf{indipendenti} se $\forall A, B \subseteq \R$ (misurabili), gli eventi $X \in A$ e
	$Y \in B$ sono indipendenti. Cioè se
	\[ P(X \in A, Y \in B) = P(X \in A) \cdot P(Y \in B) \]
	Più in generale un insieme di variabili aleatorie $X_1, X_2, \dots, X_n$ su uno spazio di
	probabilità $(\Omega, \F, P)$ sono dette indipendenti se per ogni
	$A_1, A_2, \dots, A_n \subseteq \R$ (misurabili), allora vale
	\[ P(X_1 \in A_1, X_2 \in A_2, \dots, X_n \in A_n) = \prod_{i=1}^n P(X_i \in A_i) \]
\end{definition}

Dire che $X$ e $Y$ sono indipendenti, significa che ogni informazione legata ad $X$ non modifica
le probabilità relative ad $Y$.

\begin{example}
	Se lanciamo 2 volte una moneta, le variabile aleatorie $X$ e $Y$ definite come segue
	\begin{align*}
		X = & \begin{cases}
			      1 & \text{testa al primo lancio} \\
			      0 & \text{croce al primo lancio}
		      \end{cases}   \\[1ex]
		Y = & \begin{cases}
			      1 & \text{testa al secondo lancio} \\
			      0 & \text{croce al secondo lancio}
		      \end{cases}
	\end{align*}
	sono indipendenti.
\end{example}

\begin{example}
	Data $X$ variabile aleatoria non costante, allora $X$ e $-X$ non sono indipendenti.
\end{example}

Più rigorosamente se volessi dimostrare la correttezza di quest'ultimo esempio dovremmo considerare
$A \subseteq \R$ e $B = -A = \{ -x | x \in A \}$, allora
\[ P(X \in A, -X \in B) = P(X \in A, X \in A) = P(X \in A) \]
possiamo anche dimostrare che
\[ P(X \in A) \cdot P(X \in B) = P(X \in A) \cdot P(X \in A) = P(X \in A)^2 \]
Se avessimo indipendenza, potremmo scrivere
\[ P(X \in A)^2 = P(X \in A) \]
per ogni $A \in \R$, cioè $P(X \in A) = 0$ oppure $P(X \in A) = 1$ che equivale a dire che $X$ è
costante.