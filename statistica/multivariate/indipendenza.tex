\section{Variabili aleatorie indipendenti}
Passiamo ora a parlare dell'\textbf{indipendenza} di variabili aleatorie partendo dalla definizione
di indipendenza.

\begin{definition}
	Due variabili aleatorie $X$ e $Y$ su uno spazio di probabilità $(\Omega, \F, P)$ sono dette
	\textbf{indipendenti} se $\forall A, B \subseteq \R$ (misurabili), gli eventi $X \in A$ e
	$Y \in B$ sono indipendenti. Cioè se
	\[ P(X \in A, Y \in B) = P(X \in A) \cdot P(Y \in B) \]
	Più in generale un insieme di variabili aleatorie $X_1, X_2, \dots, X_n$ su uno spazio di
	probabilità $(\Omega, \F, P)$ sono dette indipendenti se per ogni
	$A_1, A_2, \dots, A_n \subseteq \R$ (misurabili), allora vale
	\[ P(X_1 \in A_1, X_2 \in A_2, \dots, X_n \in A_n) = \prod_{i=1}^n P(X_i \in A_i) \]
\end{definition}

Dire che $X$ e $Y$ sono indipendenti, significa che ogni informazione legata ad $X$ non modifica
le probabilità relative ad $Y$.

\begin{example}
	Se lanciamo 2 volte una moneta, le variabile aleatorie $X$ e $Y$ definite come segue
	\begin{align*}
		X = & \begin{cases}
			      1 & \text{testa al primo lancio} \\
			      0 & \text{croce al primo lancio}
		      \end{cases}   \\[1ex]
		Y = & \begin{cases}
			      1 & \text{testa al secondo lancio} \\
			      0 & \text{croce al secondo lancio}
		      \end{cases}
	\end{align*}
	sono indipendenti.
\end{example}

\begin{example}
	Data $X$ variabile aleatoria non costante, allora $X$ e $-X$ non sono indipendenti.
\end{example}

Più rigorosamente se volessi dimostrare la correttezza di quest'ultimo esempio dovremmo considerare
$A \subseteq \R$ e $B = -A = \{ -x | x \in A \}$, allora
\[ P(X \in A, -X \in B) = P(X \in A, X \in A) = P(X \in A) \]
possiamo anche dimostrare che
\[ P(X \in A) \cdot P(X \in B) = P(X \in A) \cdot P(X \in A) = P(X \in A)^2 \]
Se avessimo indipendenza, potremmo scrivere
\[ P(X \in A)^2 = P(X \in A) \]
per ogni $A \in \R$, cioè $P(X \in A) = 0$ oppure $P(X \in A) = 1$ che equivale a dire che $X$ è
costante.

\begin{observation}
	Se $X_1, X_2, \dots, X_n$ sono variabili aleatorie indipendenti, allora sono indipendenti a 2
	a 2 ($X_i, X_j$ sono indipendenti per ogni $i \neq j$). Il viceversa non vale.
\end{observation}

\begin{proposition}
	Siano $X$ e $Y$ due variabili aleatorie discrete, con immagine rispettivamente nei punti $x_i$
	e $y_j$, allora $X$ e $Y$ sono indipendenti se e solo se
	\[ p(x_i, y_j) = p_X (x_i) \cdot p_Y (y_j) \]
	per ogni $x_i, y_j$ e con $p(x,y)$ funzione di massa di $(X, Y)$ e $p_X$ e $p_Y$ funzioni di
	massa rispettivamente di $X$ e $Y$.
	\begin{proof}
		Per dimostrare questo basta notare che
		\begin{align*}
			p_{(X,Y)} (x_i, y_j) = & P(X = x_i, Y = y_j)         \\
			=                      & P(X = x_i) \cdot P(Y = y_j) \\
			=                      & p_X (x_i) \cdot p_Y (y_j)
		\end{align*}
	\end{proof}

	Data $(X, Y)$ con densità, $X$ e $Y$ sono indipendenti se e solo se
	\[ f_{(X,Y)} (x,y) = f_X (x) \cdot f_Y (y) \]
	per ogni $(x, y) \in \R^2$. Con $f_{(X,Y)}$, $f_X$, e $f_Y$ densità rispettivamente di
	$(X,Y)$, $X$ e $Y$.
	\begin{proof}
		In questo caso supponiamo di avere $A, B \subseteq \R$ e vogliamo dimostrare che
		\begin{align*}
			P(X \in A, Y \in B) = & P((X,Y) \in A \times B)                                    \\
			=                     & \sum_{(x_i, y_j) \in A \times B} p_{(x,y)} (x_i, y_j)      \\
			=                     & \sum_{x_i \in A, y_j \in B} p_X(x_i) \cdot p_Y(y_j)        \\
			=                     & \sum_{x_i \in A} \left( \sum_{y_j \in B} p_Y(y_j) \right)
			p_X(x_i)                                                                           \\
			=                     & \sum_{y_j \in B} p_Y(y_j) \cdot \sum_{x_i \in A} p_X (x_i) \\
			=                     & P(Y \in B) \cdot P(X \in A)
		\end{align*}
	\end{proof}
\end{proposition}

\begin{observation}
	Se $X$ e $Y$ sono indipendenti allora la funzione di massa/densità congiunta si ricava dalle
	funzioni di massa/densità marginali. L'indipendenza è una proprietà della legge congiunta
	perché
	\[ P_{(X,Y)} (A \times B) = P_X(A) \cdot P_Y(B) \]
	per ogni $A, B \subseteq \R$.
\end{observation}

\begin{example}
	Riprendiamo l'esempio fatto in precedenza in cui abbiamo 4 punti, ciascuno con probabilità
	$1/4$. In questo caso la densità congiunta di $i$ e $j$ è uguale a
	\[ p_{(X,Y)} (i,j) = \frac{1}{4} \]
	per ogni $i,j \in \{ -1, 1 \}^2$. Le densità marginali sono
	\begin{align*}
		p_X(i) = & \frac{1}{2} & \forall i = -1, 1 \\
		p_Y(j) = & \frac{1}{2} & \forall j = -1, 1
	\end{align*}
	A questo punto si può verificare che
	\[ p_{(X,Y)} (i,j) = \frac{1}{4} = p_X(i) \cdot p_Y(j) \]
	quindi $X$ e $Y$ sono indipendenti
\end{example}

\begin{example}
	Riprendendo anche il secondo esempio in cui avevamo la densità congiunta che valeva
	\[ P_{(x,y)} (i,j) = \frac{1}{2} \]
	se $(i,j) = (1,1)$ o se $(i,j) = (-1,-1)$ e abbiamo le densità marginali che valgono
	\begin{align*}
		p_X(i) = & \frac{1}{2} & \forall i = -1, 1 \\
		p_Y(j) = & \frac{1}{2} & \forall j = -1, 1 \\
	\end{align*}
	A questo punto possiamo constatare che se conosciamo $X$ allora conosciamo anche $Y$, per
	esempio
	\[ p_{(x,y)} (1,-1) = 0 \neq p_x(1) \cdot p_y(-1) \]
	dunque $X$ e $Y$ non sono indipendenti.
\end{example}

\begin{proposition}[Stabilità dell'indipendenza per composizione]
	Se $X$ e $Y$ sono indipendenti, date $h, k : \R \to \R$, allora $h(X)$ e $k(Y)$ sono
	indipendenti.

	Più in generale se $X_1, \dots, X_n$ e $Y_1, \dots, Y_m$ sono variabili indipendenti e
	$h : \R^n \to \R$ e $k : \R^m \to \R$, allora $h(X_1, \dots, X_n)$ e $k(Y_1, \dots, Y_m)$ sono
	indipendenti.
\end{proposition}

Come per gli eventi, un caso rilevante di variabili aleatorie indipendenti, è dato dalle prove
ripetute. In un esperimento ripetute più volte nelle medesime condizioni, variabili aleatorie
associate a ripetizioni distinte o anche a gruppi disgiunti di ripetizioni, sono indipendenti.

\begin{example}
	Consideriamo $n$ prove ripetute, ciascuna con probabilità di successo $p$ (e di insuccesso
	$1-p$). Abbiamo quindi $\Omega = \{0,1\}^n$ e la probabilità è definita come
	\[ P(a_1, \dots, a_n) = p^{\# \{i | a_i = 1\}} \cdot (1-p)^{\# \{i | a_i = 0\}} \]
	Consideriamo ora $X_i$ con $i = 1, \dots, n$ la variabile aleatoria che modellizza l'esito
	dell'$i$-esimo esperimento e definita in come segue
	\[ X_i (a_1, \dots, a_n) = a_i \]
	Possiamo dubito dire che le varie $X_i$ sono indipendenti, inoltre, ciascun $X_i$ è una
	variabile binomiale di parametro $p$ ($B(p)$). Se inoltre consideriamo la variabile aleatoria
	\[ X = X_i + \dots + X_n \]
	possiamo dire che $X$ conta il numero di successi nelle $n$ prove e sarà quindi una binomiale
	$B(n,p)$.
\end{example}

\begin{example}
	Sia $S$ una popolazione e sia $X : S \to \R$ una variabile aleatoria che rappresenta un
	carattere (ad esempio $X(\omega)$ è il numero di figli di $\omega$).

	Consideriamo ora l'estrazione di un campione di $n$ elementi da $S$ e sia $X_i$ l'esito della
	misurazione del carattere dell'$i$-esimo elemento del campione.

	Supponiamo che ogni elemento del campione sia scelto in modo casuale indipendentemente dalla
	scelta di altri elementi.

	In questo caso abbiamo $\Omega = S^n$, $P$ uniforme su $\Omega$ e il carattere dell'$i$-esimo
	elemento definito come $X_i(\omega_1, \dots, \omega_n) = X (\omega_i)$. In questo caso si può
	dimostrare che le $X_i$ sono indipendenti e hanno la stessa legge $P_X$ di $X$.
\end{example}

Data una coppia di variabili aleatorie $(X, Y)$ e $h : \R^2 \to \R$, vogliamo determinare la legge
di $h(X, Y) : \Omega \to \R$. Non esiste una formula generale, ma si possono dare alcuni fatti e
regole

\begin{proposition}[Riproducibilità delle binomiali]
	Siano $X \sim B(n, p)$, $Y \sim B(m, p)$, $X$ e $Y$ indipendenti, allora la variabile
	\[ Z := X + Y \sim B(n + m, p) \]
	\begin{proof}
		Dobbiamo considerare due casi:
		\begin{itemize}
			\item \textbf{Caso (m = 1)}: in questo caso abbiamo $Y \sim B(p)$ quindi
			      \[
				      Y = \begin{cases}
					      1 & \text{con probabilità } 1 \\
					      0 & \text{con probabilità } 0
				      \end{cases}
			      \]
			      e la probabilità che $Z = h$ è
			      \begin{align*}
				      P(Z = h) = & P(X = h, Y = 0) + P(X = h-1, Y = 1)                          \\
				      =          & P(X = h) \cdot P(Y = 0) + P(X = h-1) \cdot P(Y = 1)          \\
				      =          & \binom{n}{h} p^h (1-p)^{n-h} (1-p) +
				      \binom{n}{h-1} p^{h-1} (1-p)^{n-h+1} p                                    \\
				      =          & p^h (1-p)^{n-h+1} \left(\binom{n}{h} + \binom{n}{h-1}\right) \\
				      =          & \binom{n+1}{h} p^h (1-p)^{n-h+1}
			      \end{align*}
			      che infatti è la funzione di massa di una binomiale $B(n+1 = m, p)$.
			\item Il caso generale si ricava tramite induzione su $m$.
		\end{itemize}
	\end{proof}
\end{proposition}

\begin{proposition}
	Siano $X$ e $Y$ discrete e indipendenti e sia $Z = X + Y$, allora
	\[ P_Z(n) = P(Z = n) = \sum_{h=0}^n P_X(h) \cdot P_Y(n-h) \]
\end{proposition}

\begin{proposition}[Riproducibilità delle Poisson]
	Siano $X \sim P(\lambda)$ e $Y \sim P(\nu)$ due variabili aleatorie indipendenti, allora
	$X+Y \sim P(\lambda + \nu)$.
\end{proposition}

\begin{proposition}[Convoluzione]
	Siano $X$ e $Y$ due variabili aleatorie con densità rispettivamente $f_X$ e $f_Y$ e
	indipendenti, allora la variabile aleatoria $Z = X + Y$ ha densità
	\[
		f_Z (z) = \int_{-\infty}^{+\infty} f_X (x) f_Y (z - x) dx =
		\int_{-\infty}^{+\infty} f_Y (y) f_X (z - y) dy
	\]
\end{proposition}

\begin{proposition}[Riproducibilità delle Gaussiane]
	Siano $X \sim N(\mu, \sigma^2)$ e $Y \sim N(\alpha, \beta^2)$ indipendenti, allora
	\[ X + Y \sim N(\mu + \alpha, \sigma^2 + \beta^2) \]
\end{proposition}