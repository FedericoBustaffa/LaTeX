\chapter{Determinante}
\section{Definizione di determinante}
Il determinante \`e una funzione
\begin{equation*}
	Det : Mat_{n \times n}(\mathbb{K}) \to \mathbb{K}
\end{equation*}
Per allegerire la notazione talvolta indicheremo con $|a_{ij}|$ oppure con
$Det(a_{ij})$ o con $Det(A)$ il determinante di una matrice $A = (a_{ij})$.

Il determinante \`e definito ricorsivamente, al crescere di $n$, nel seguente
modo:
\begin{itemize}
	\item Il determinante di una matrice $1 \times 1$ \`e uguale all'unico
	      coefficiente della matrice:
	      \begin{equation*}
		      Det(a) = a
	      \end{equation*}
	\item Dato $n \geq 2$ il determinante di una matrice $A = (a_{ij})$ di
	      formato $n \times n$ pu\`o essere ottenuto come combinazione lineare
	      di una qualunque riga, diciamo la $i$-esima, tramite la seguente
	      formula:
	      \begin{equation*}
		      Det(A) = (-1)^{i + 1}a_{i1}Det(A_{i1}) + ... +
		      (-1)^{i + n}a_{in}Det(A_{in})
	      \end{equation*}
	      dove $A_{ij}$ indica la matrice quadrata di formato
	      $(n - 1) \times (n - 1)$ che si ottiene da $A$ cancellando la riga
	      $i$-esima e la colonna $j$-esima.
\end{itemize}

\begin{observation}
	Dalla definizione possiamo immediatamente ricavare la seguente formula
	per le matrici $2 \times 2$:
	\begin{equation*}
		Det \begin{pmatrix}
			a & b \\
			c & d
		\end{pmatrix} =
		ad - bc
	\end{equation*}
\end{observation}

\begin{observation}
	Il determinante si pu\`o mettere anche come combinazione lineare dei
	coefficienti di una qualsiasi colonna, diciamo la $j$-esima, tramite
	la seguente formula:
	\begin{equation*}
		Det(A) = (-1)^{1 + j}a_{1j}Det(A_{1j}) + ... +
		(-1)^{n + j}a_{nj}Det(A_{nj})
	\end{equation*}
\end{observation}

\begin{example}
	Data in $Mat_{3 \times 3}(\mathbb{K})$ la matrice
	\begin{equation*}
		\begin{pmatrix}
			3 & 2 & 5 \\
			2 & 0 & 1 \\
			4 & 2 & 6
		\end{pmatrix}
	\end{equation*}
	per calcolare il determinante si sceglie una riga (o una colonna) e poi si
	applica la formula. Per esempio, scegliamo la seconda riga:
	\begin{gather*}
		Det(A) = -2Det \begin{pmatrix}
			2 & 5 \\
			2 & 6
		\end{pmatrix} +
		0 Det \begin{pmatrix}
			3 & 5 \\
			4 & 6
		\end{pmatrix} -
		Det \begin{pmatrix}
			3 & 2 \\
			4 & 2
		\end{pmatrix} = \\
		= -2(12 - 10) - (6 - 8) = -4 + 2 = -2
	\end{gather*}
\end{example}

\begin{observation}
	Nel caso delle matrici $3 \times 3$ il determinante si pu\`o calcolare
	anche mediante la seguente \emph{regola di Sarrus}. Data
	\begin{equation*}
		B = \begin{pmatrix}
			a & b & c \\
			d & e & f \\
			g & h & i \\
		\end{pmatrix}
	\end{equation*}
	si forma la seguente matrice $3 \times 5$
	\begin{equation*}
		\begin{pmatrix}
			a & b & c & a & b \\
			d & e & f & d & e \\
			g & h & i & g & h
		\end{pmatrix}
	\end{equation*}
	dopodich\'e si sommano i tre prodotti dei coefficienti che si trovano sulle
	tre diagonali che scendono da sinistra a destra e si sottragono i tre
	prodotti dei coefficienti che si trovano sulle tre diagonali che salgono
	da sinistra a destra:
	\begin{equation*}
		Det(B) = aci + bfg + cdh - gec - hfa - idb
	\end{equation*}
	Verifichiamo che nel caso della matrice
	\begin{equation*}
		A = \begin{pmatrix}
			3 & 2 & 5 \\
			2 & 0 & 1 \\
			4 & 2 & 6
		\end{pmatrix}
	\end{equation*}
	la regola di Sarrus dia lo stesso risultato -2 che abbiamo gi\`a calcolato:
	\begin{equation*}
		3 \cdot 0 \cdot 6 + 2 \cdot 1 \cdot 4 + 5 \cdot 2 \cdot 2 -
		4 \cdot 0 \cdot 5 - 2 \cdot 1 \cdot 3 - 6 \cdot 2 \cdot 2 =
		8 + 20 - 6 - 24 = -2
	\end{equation*}
\end{observation}

\section{Determinante e calcolo del rango}
Data una matrice $A = (a_{ij}) \in Mat_{n \times m}(\mathbb{K})$ e dato un
numero intero positivo $k$ minore o uguale al minimo fra $m$ e $n$, possiamo
scegliere $k$ righe fra le $n$ righe (diciamo $i_1, i_2, \dots i_k$ e
$k$ colonne fra le $m$ colonne (diciamo $j_1, j_2, \dots, j_k$).

\begin{defn}
	Data la scelta di $k$ righe e $k$ colonne come sopra, chiamiamo
	\textbf{minore} di $A$ di formato $k \times k$, la matrice ottenuta da $A$
	cancellando tutti i coefficienti eccetto quelli che giacciono
	contemporaneamente su una delle $k$ righe e su una delle $k$ colonne scelte,
	in altre parole cancellando tutti i coefficienti eccetto gli $a_{ij}$ per
	cui $i \in \{i_1, i_2, \dots, i_k\}$ e $j \in \{j_1, j_2, \dots j_k\}$.
\end{defn}

\begin{observation}
	Da una matrice $n \times m$ \`e possibile ricavare $\begin{psmallmatrix}
			n \\ k \end{psmallmatrix} \begin{psmallmatrix}
			m \\ k \end{psmallmatrix}$ minori di formato $k \times k$.
\end{observation}

I determinanti dei minori possono essere usati per calcolare il rango di una
matrice $n \times m$ come risulta dal seguente teorema

\begin{theorem}
	Data una matrice $A = (a_{ij}) \in Mat_{n \times m}(\mathbb{K})$, supponiamo
	che esista un minore di formato $k \times k$ il cui determinante \`e diverso
	da 0. Allora il rango di $A$ \`e maggiore o uguale a $k$. Se $k = n$ oppure
	$k = m$ allora il rango di $A$ \`e uguale a $k$. Se $k < n$ e $k < m$ e tutti
	i determinanti dei minori di formato $(k + 1) \times (k + 1)$ sono uguali a 0
	allora il rango di $A$ \`e uguale a $k$.
\end{theorem}

\begin{observation}
	Se una matrice quadrata $n \times n$ ha determinante diverso da zero, allora
	per il teorema precedente ha rango $n$ e dunque \`e invertibile. Sempre per il
	teorema vale anche il viceversa: se una matrice $n \times n$ ha determinante
	uguale a 0, allora il suo rango \`e strettamente minore di $n$ e dunque la
	matrice non \`e invertibile.

	Nel caso $2 \times 2$, data
	\begin{equation*}
		A = \begin{pmatrix}
			a & b \\
			c & d
		\end{pmatrix}
	\end{equation*}
	con determinante diverso da 0, ossia $ad - bc \neq 0$, l'inversa si scrive
	esplicitamente come:
	\begin{equation*}
		A^{-1} = \frac{1}{ad - bc} \begin{pmatrix}
			d  & -b \\
			-c & a
		\end{pmatrix}
	\end{equation*}
\end{observation}

\begin{example}
	La matrice
	\begin{equation*}
		\begin{pmatrix}
			3 & 9          & 4          & 7 & 12         \\
			1 & \textbf{3} & \textbf{2} & 0 & \textbf{5} \\
			1 & \textbf{2} & \textbf{0} & 0 & \textbf{1} \\
			1 & \textbf{4} & \textbf{2} & 7 & \textbf{6}
		\end{pmatrix}
	\end{equation*}
	ha rango maggiore o uguale a 3 in quanto contiene un minore $3 \times 3$
	(quello individuato dalle righe seconda, terza e quarta e dalle colonne
	seconda terza e quinta) che ha determinante diverso da 0. Inoltre, poich\'e
	tutti i minori $4 \times 4$ hanno determinante uguale a zero, la matrice ha
	rango esattamente 3. \`E pi\`u rapido osservare che la prima riga \`e uguale
	alla somma delle altre delle altre tre righe, dunque il rango \`e minore o
	uguale a 3: poich\'e dal calcolo del determinante del minore $3 \times 3$
	sappiamo che il rango \`e maggiore o uguale a 3, allora \`e esattamente 3.
\end{example}

Il calcolo del rango attraverso il teorema precedente pu\`o richiedere molte
verifiche, ed \`e in generale meno conveniente della riduzione di Gauss. Il
seguente teorema pu\`o comunque ridurre i determinanti da calcolare:

\begin{theorem}[Teorema degli orlati]
	Data $A = (a_{ij}) \in Mat_{n \times m}(\mathbb{K})$, supponiamo
	che esista un minore $K$ di formato $k \times k$ il cui determinante \`e
	diverso da 0. Se sono uguali a 0 tutti i determinanti dei minori di formato
	$(k + 1) \times (k + 1)$ che si ottengono aggiungendo una riga e una colonna
	a quelle scelte per formare il minore $K$, allora il rango di $A$ \`e uguale
	a $k$, altrimenti \`e strettamente maggiore di $k$.
\end{theorem}

Dunque se abbiamo un minore di formato $k \times k$ con determinante diverso da 0
e vogliamo decidere se la matrice ha rango $k$ oppure ha rango strettamente
maggiore di $k$ basta controllare $(n - k)(m - k)$ minori di formato
$(k + 1) \times (k + 1)$, non tutti i
$\begin{psmallmatrix} n \\ k + 1 \end{psmallmatrix}$
$\begin{psmallmatrix} m \\ k + 1 \end{psmallmatrix}$ minori di formato
$(k + 1) \times (k + 1)$.

\section{Teorema di Binet}
Il determinante non \`e un'applicazione lineare. In generale
\[ Det(A + B) \neq Det(A) + Det(B) \]

\begin{theorem}[Teorema di Binet]
	Siano $A, B \in Mat_{n \times n}(\mathbb{K})$. Allora
	\begin{equation*}
		Det(AB) = Det(A)Det(B)
	\end{equation*}
\end{theorem}

\begin{corollary}
	Se $M \in Mat_{n \times n}(\mathbb{K})$ \`e una matrice invertibile, allora
	\begin{equation*}
		Det(M^{-1}) = \frac{1}{Det(M)}
	\end{equation*}
	\begin{proof}
		Calcoliamo $Det(M^{-1} M)$. Per il teorema di Binet vale
		\begin{gather*}
			Det(M^{-1} M) = Det(M^{-1})Det(M) \\
			M^{-1}M = I \\
			Det(I) = 1
		\end{gather*}
	\end{proof}
\end{corollary}

Grazie al teorema di Binet possiamo osservare che, dato un endomorfismo
$L \in End(V)$, il determinante assume lo stesso valore su tutte le matrici
che si associano a $V$ al variare delle basi dello spazio, ossia vale:
\begin{equation*}
	Det \left(
	[L]_{\substack{
			v_1, v_2, \dots, v_n \\
			v_1, v_2, \dots, v_n
		}}
	\right) =
	Det \left(
	[L]_{\substack{
			e_1, e_2, \dots, e_n \\
			e_1, e_2, \dots, e_n
		}}
	\right)
\end{equation*}
per ogni scelta di due basi $e_1, e_2, \dots, e_n$ e $v_1, v_2, \dots, v_n$
di $V$.

Possima scrivere dunque:
\begin{equation*}
	Det \left(
	[L]_{\substack{
			v_1, v_2, \dots, v_n \\
			v_1, v_2, \dots, v_n
		}}
	\right) =
	Det \left(
	M^{-1}[L]_{\substack{
			e_1, e_2, \dots, e_n \\
			e_1, e_2, \dots, e_n
		}} M
	\right)
\end{equation*}
A questo punto, per il teorema di Binet, possiamo concludere che
\begin{gather*}
	Det \left(
	[L]_{\substack{
			v_1, v_2, \dots, v_n \\
			v_1, v_2, \dots, v_n
		}}
	\right) = \\
	= Det(M^{-1}) Det \left(
	[L]_{\substack{
			e_1, e_2, \dots, e_n \\
			e_1, e_2, \dots, e_n
		}}
	\right) Det(M) = \\
	= Det \left(
	[L]_{\substack{
			e_1, e_2, \dots, e_n \\
			e_1, e_2, \dots, e_n
		}}
	\right)
\end{gather*}

\begin{example}
	Consideriamo l'endomorfismo $L : \mathbb{R}^2 \to \mathbb{R}^2$
	\[
		L \begin{pmatrix}
			x \\ y
		\end{pmatrix} =
		\begin{pmatrix}
			2x - y \\
			3y
		\end{pmatrix}
	\]
	La sua matrice associata rispetto alla base standard di $\mathbb{R}^2$ \`e
	\[
		[L] = \begin{pmatrix}
			2 & -1 \\
			0 & 3
		\end{pmatrix}
	\]
	il cui determinante \`e \[ Det([L]) = 2 \cdot 3 - (-1) \cdot 0 = 6 \]
	Consideriamo ora la base di $\mathbb{R}^2$
	\[
		\mathcal{B} =
		\left\{
		\begin{pmatrix} 1 \\ 1 \end{pmatrix}, \quad
		\begin{pmatrix} 1 \\ 0 \end{pmatrix}
		\right\}
	\]
	La matrice associata a $L$ rispetto a tale base \`e
	\[
		[L]_{\substack{\mathcal{B} \\ \mathcal{B}}} =
		\begin{pmatrix}
			3  & 0 \\
			-2 & 2
		\end{pmatrix}
	\]
	Calcoliamo il determinante
	\[
		Det \left( [L]_{\substack{\mathcal{B} \\ \mathcal{B}}} \right) =
		3 \cdot 2 - 0 \cdot (-2) = 6
	\]
\end{example}

\section{Propriet\`a del determinante rispetto alle mosse di riga e colonna}
Ricordiamo le operazioni elementari di colonna:
\begin{itemize}
	\item si somma alla colonna $i$ la colonna $j$ moltiplicata per uno
	      scalare $\lambda$.
	\item si moltiplica la colonna $s$ per uno scalare $k \neq 0$.
	\item si permutano tra loro due colonne, diciamo $i$ e $j$.
\end{itemize}

Vediamo ora come cambia il determinante se facciamo un'operazione elementare di colonna
su una matrice $A \in Mat_{n \times n}(\mathbb{K})$.
\begin{itemize}
	\item Un'operazione del primo tipo equivale a moltiplicare $A$ a destra per la
	      matrice $n \times n$ che ha tutti 1 sulla diagonale, e 0 in tutte le altre
	      caselle eccetto che nella casella identificata da "riga $j$, colonna $i$",
	      dove troviamo $\lambda$. La matrice $M_{ij}$ \`e triangolare e il suo
	      determinante \`e uguale a 1, dunque $Det(AM_{ij})$ \`e uguale a $Det(A)$ per
	      il teorema di Binet.
	\item Un'operazione di colonna del terzo tipo ha come
	      effetto quello di cambiare il segno del determinante.
	\item Quanto alle operazioni del secondo tipo, dalla definizione stessa di
	      determinante si ricava che, se si moltiplica una colonna per uno scalare
	      $k \neq 0$, anche il determinante della matrice risulter\`a moltiplicato
	      per $k$.
\end{itemize}

Considerazioni analoghe valgono, ovviamente, per le operazioni elementari di
riga.

Una conseguenza di queste osservazioni \`e che se vogliamo sapere se il
determinante di una certa matrice \`e uguale a 0 oppure no possiamo,
prima di calcolarlo, fare alcune operazioni di riga e/o di colonna.
Di solito questo risulta utile se con tali operazioni otteniamo una riga
(o una colonna) con molti coefficienti uguali a 0, facilitando il calcolo.