
\section{Determinante}
\subsection{Definizione di determinante}
Il determinante \`e una funzione
\begin{equation*}
	Det : Mat_{n \times n}(\mathbb{K}) \to \mathbb{K}
\end{equation*}
Per allegerire la notazione talvolta indicheremo con $|a_{ij}|$ oppure con
$Det(a_{ij})$ o con $Det(A)$ il determinante di una matrice $A = (a_{ij})$.

Il determinante \`e definito ricorsivamente, al crescere di $n$, nel seguente
modo:
\begin{itemize}
	\item Il determinante di una matrice $1 \times 1$ \`e uguale all'unico
	      coefficiente della matrice:
	      \begin{equation*}
		      Det(a) = a
	      \end{equation*}
	\item Dato $n \geq 2$ il determinante di una matrice $A = (a_{ij})$ di
	      formato $n \times n$ pu\`o essere ottenuto come combinazione lineare
	      di una qualunque riga, diciamo la $i$-esima, tramite la seguente
	      formula:
	      \begin{equation*}
		      Det(A) = (-1)^{i + 1}a_{i1}Det(A_{i1}) + ... + 
		      (-1)^{i + n}a_{in}Det(A_{in})
	      \end{equation*}
	      dove $A_{ij}$ indica la matrice quadrata di formato
	      $(n - 1) \times (n - 1)$ che si ottiene da $A$ cancellando la riga
	      $i$-esima e la colonna $j$-esima.
\end{itemize}

\begin{observation}
	Dalla definizione possiamo immediatamente ricavare la seguente formula
	per le matrici $2 \times 2$:
	\begin{equation*}
		Det \begin{pmatrix}
			a & b \\
			c & d
		\end{pmatrix} =
		ad - bc
	\end{equation*}
\end{observation}

\begin{observation}
	Il determinante si pu\`o mettere anche come combinazione lineare dei
	coefficienti di una qualsiasi colonna, diciamo la $j$-esima, tramite
	la seguente formula:
	\begin{equation*}
		Det(A) = (-1)^{1 + j}a_{1j}Det(A_{1j}) + ... +
		(-1)^{n + j}a_{nj}Det(A_{nj})
	\end{equation*}
\end{observation}

\begin{example}
	Data in $Mat_{3 \times 3}(\mathbb{K})$ la matrice
	\begin{equation*}
		\begin{pmatrix}
			3 & 2 & 5 \\
			2 & 0 & 1 \\
			4 & 2 & 6
		\end{pmatrix}
	\end{equation*}
	per calcolare il determinante si sceglie una riga (o una colonna) e poi si
	applica la formula. Per esempio, scegliamo la seconda riga:
	\begin{gather*}
		Det(A) = -2Det \begin{pmatrix}
			2 & 5 \\
			2 & 6
		\end{pmatrix} +
		0 Det \begin{pmatrix}
			3 & 5 \\
			4 & 6
		\end{pmatrix} -
		Det \begin{pmatrix}
			3 & 2 \\
			4 & 2
		\end{pmatrix} = \\
		= -2(12 - 10) - (6 - 8) = -4 + 2 = -2
	\end{gather*}
\end{example}

\begin{observation}
	Nel caso delle matrici $3 \times 3$ il determinante si pu\`o calcolare
	anche mediante la seguente \emph{regola di Sarrus}. Data
	\begin{equation*}
		B = \begin{pmatrix}
			a & b & c \\
			d & e & f \\
			g & h & i \\
		\end{pmatrix}
	\end{equation*}
	si forma la seguente matrice $3 \times 5$
	\begin{equation*}
		\begin{pmatrix}
			a & b & c & a & b \\
			d & e & f & d & e \\
			g & h & i & g & h
		\end{pmatrix}
	\end{equation*}
	dopodich\'e si sommano i tre prodotti dei coefficienti che si trovano sulle
	tre diagonali che scendono da sinistra a destra e si sottragono i tre
	prodotti dei coefficienti che si trovano sulle tre diagonali che salgono
	da sinistra a destra:
	\begin{equation*}
		Det(B) = aci + bfg + cdh - gec - hfa - idb
	\end{equation*}
	Verifichiamo che nel caso della matrice
	\begin{equation*}
		A = \begin{pmatrix}
			3 & 2 & 5 \\
			2 & 0 & 1 \\
			4 & 2 & 6
		\end{pmatrix}
	\end{equation*}
	la regola di Sarrus dia lo stesso risultato -2 che abbiamo gi\`a calcolato:
	\begin{equation*}
		3 \cdot 0 \cdot 6 + 2 \cdot 1 \cdot 4 + 5 \cdot 2 \cdot 2 -
		4 \cdot 0 \cdot 5 - 2 \cdot 1 \cdot 3 - 6 \cdot 2 \cdot 2 =
		8 + 20 - 6 - 24 = -2
	\end{equation*}
\end{observation}