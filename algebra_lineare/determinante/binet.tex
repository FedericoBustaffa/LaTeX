
\subsection{Teorema di Binet}
Il determinante non \`e un'applicazione lineare. In generale
\[ Det(A + B) \neq Det(A) + Det(B) \]

\begin{theorem}[Teorema di Binet]
	Siano $A, B \in Mat_{n \times n}(\mathbb{K})$. Allora
	\begin{equation*}
		Det(AB) = Det(A)Det(B)
	\end{equation*}
\end{theorem}

\begin{corollary}
	Se $M \in Mat_{n \times n}(\mathbb{K})$ \`e una matrice invertibile, allora
	\begin{equation*}
		Det(M^{-1}) = \frac{1}{Det(M)}
	\end{equation*}
	\begin{proof}
		Calcoliamo $Det(M^{-1} M)$. Per il teorema di Binet vale
		\begin{gather*}
			Det(M^{-1} M) = Det(M^{-1})Det(M) \\
			M^{-1}M = I \\
			Det(I) = 1
		\end{gather*}
	\end{proof}
\end{corollary}

Grazie al teorema di Binet possiamo osservare che, dato un endomorfismo
$L \in End(V)$, il determinante assume lo stesso valore su tutte le matrici
che si associano a $V$ al variare delle basi dello spazio, ossia vale:
\begin{equation*}
	Det \left(
	[L]_{\substack{
			v_1, v_2, \dots, v_n \\
			v_1, v_2, \dots, v_n
		}}
	\right) =
	Det \left(
	[L]_{\substack{
			e_1, e_2, \dots, e_n \\
			e_1, e_2, \dots, e_n
		}}
	\right)
\end{equation*}
per ogni scelta di due basi $e_1, e_2, \dots, e_n$ e $v_1, v_2, \dots, v_n$
di $V$.

Possima scrivere dunque:
\begin{equation*}
	Det \left(
	[L]_{\substack{
			v_1, v_2, \dots, v_n \\
			v_1, v_2, \dots, v_n
		}}
	\right) =
	Det \left(
	M^{-1}[L]_{\substack{
			e_1, e_2, \dots, e_n \\
			e_1, e_2, \dots, e_n
		}} M
	\right)
\end{equation*}
A questo punto, per il teorema di Binet, possiamo concludere che
\begin{gather*}
	Det \left(
	[L]_{\substack{
			v_1, v_2, \dots, v_n \\
			v_1, v_2, \dots, v_n
		}}
	\right) = \\
	= Det(M^{-1}) Det \left(
	[L]_{\substack{
			e_1, e_2, \dots, e_n \\
			e_1, e_2, \dots, e_n
		}}
	\right) Det(M) = \\
	= Det \left(
	[L]_{\substack{
			e_1, e_2, \dots, e_n \\
			e_1, e_2, \dots, e_n
		}}
	\right)
\end{gather*}

\begin{example}
	Consideriamo l'endomorfismo $L : \mathbb{R}^2 \to \mathbb{R}^2$
	\[
		L \begin{pmatrix}
			x \\ y
		\end{pmatrix} =
		\begin{pmatrix}
			2x - y \\
			3y
		\end{pmatrix}
	\]
	La sua matrice associata rispetto alla base standard di $\mathbb{R}^2$ \`e
	\[
		[L] = \begin{pmatrix}
			2 & -1 \\
			0 & 3
		\end{pmatrix}
	\]
	il cui determinante \`e \[ Det([L]) = 2 \cdot 3 - (-1) \cdot 0 = 6 \]
	Consideriamo ora la base di $\mathbb{R}^2$
	\[
		\mathcal{B} =
		\left\{
		\begin{pmatrix} 1 \\ 1 \end{pmatrix}, \quad
		\begin{pmatrix} 1 \\ 0 \end{pmatrix}
		\right\}
	\]
	La matrice associata a $L$ rispetto a tale base \`e
	\[
		[L]_{\substack{\mathcal{B} \\ \mathcal{B}}} =
		\begin{pmatrix}
			3  & 0 \\
			-2 & 2
		\end{pmatrix}
	\]
	Calcoliamo il determinante
	\[
		Det \left( [L]_{\substack{\mathcal{B} \\ \mathcal{B}}} \right) =
		3 \cdot 2 - 0 \cdot (-2) = 6
	\]
\end{example}