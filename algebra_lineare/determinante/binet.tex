\section{Teorema di Binet}
Il determinante non è un'applicazione lineare. In generale
\[ \det(A + B) \neq \det(A) + \det(B) \]

\begin{theorem}[Teorema di Binet]
	Siano $A, B \in \Mat_{n \times n}(\K)$. Allora
	\[
		\det(AB) = \det(A)\det(B)
	\]
\end{theorem}

\begin{corollary}
	Se $M \in \Mat_{n \times n}(\K)$ è una matrice invertibile, allora
	\[
		\det(M^{-1}) = \frac{1}{\det(M)}
	\]
	\begin{proof}
		Calcoliamo $\det(M^{-1} M)$. Per il teorema di Binet vale
		\begin{gather*}
			\det(M^{-1} M) = \det(M^{-1})\det(M) \\
			M^{-1}M = I \\
			\det(I) = 1
		\end{gather*}
	\end{proof}
\end{corollary}

Grazie al teorema di Binet possiamo osservare che, dato un endomorfismo
$L \in End(V)$, il determinante assume lo stesso valore su tutte le matrici
che si associano a $V$ al variare delle basi dello spazio, ossia vale:
\[
	\det \left(
	[L]_{\substack{
			v_1, v_2, \dots, v_n \\
			v_1, v_2, \dots, v_n
		}}
	\right) =
	\det \left(
	[L]_{\substack{
			e_1, e_2, \dots, e_n \\
			e_1, e_2, \dots, e_n
		}}
	\right)
\]
per ogni scelta di due basi $e_1, e_2, \dots, e_n$ e $v_1, v_2, \dots, v_n$
di $V$.

Possima scrivere dunque:
\[
	\det \left(
	[L]_{\substack{
			v_1, v_2, \dots, v_n \\
			v_1, v_2, \dots, v_n
		}}
	\right) =
	\det \left(
	M^{-1}[L]_{\substack{
			e_1, e_2, \dots, e_n \\
			e_1, e_2, \dots, e_n
		}} M
	\right)
\]
A questo punto, per il teorema di Binet, possiamo concludere che
\begin{gather*}
	\det \left(
	[L]_{\substack{
			v_1, v_2, \dots, v_n \\
			v_1, v_2, \dots, v_n
		}}
	\right) = \\
	= \det(M^{-1}) \det \left(
	[L]_{\substack{
			e_1, e_2, \dots, e_n \\
			e_1, e_2, \dots, e_n
		}}
	\right) \det(M) = \\
	= \det \left(
	[L]_{\substack{
			e_1, e_2, \dots, e_n \\
			e_1, e_2, \dots, e_n
		}}
	\right)
\end{gather*}

\begin{example}
	Consideriamo l'endomorfismo $L : \R^2 \to \R^2$
	\[
		L \begin{pmatrix}
			x \\ y
		\end{pmatrix} =
		\begin{pmatrix}
			2x - y \\
			3y
		\end{pmatrix}
	\]
	La sua matrice associata rispetto alla base standard di $\R^2$ è
	\[
		[L] = \begin{pmatrix}
			2 & -1 \\
			0 & 3
		\end{pmatrix}
	\]
	il cui determinante è \[ \det([L]) = 2 \cdot 3 - (-1) \cdot 0 = 6 \]
	Consideriamo ora la base di $\R^2$
	\[
		\B =
		\left\{
		\begin{pmatrix} 1 \\ 1 \end{pmatrix}, \quad
		\begin{pmatrix} 1 \\ 0 \end{pmatrix}
		\right\}
	\]
	La matrice associata a $L$ rispetto a tale base è
	\[
		[L]_{\substack{\B \\ \B}} =
		\begin{pmatrix}
			3  & 0 \\
			-2 & 2
		\end{pmatrix}
	\]
	Calcoliamo il determinante
	\[
		\det \left( [L]_{\substack{\B \\ \B}} \right) =
		3 \cdot 2 - 0 \cdot (-2) = 6
	\]
\end{example}
