\documentclass[11pt, a4paper]{report}

\pdfpagewidth\paperwidth
\pdfpageheight\paperheight

\usepackage[utf8]{inputenc}
\usepackage[T1]{fontenc}
\usepackage[italian]{babel}
\usepackage{mathtools, amsmath, amssymb, amsthm}
\usepackage[hidelinks]{hyperref}

\title{Algebra Lineare}
\author{Federico Bustaffa}
\date{13/02/2022}

\theoremstyle{definition}
\newtheorem{definition}{Definizione}[subsection]
\newtheorem{observation}[definition]{Osservazione}
\newtheorem{theorem}[definition]{Teorema}
\newtheorem{proposition}[definition]{Proposizione}
\newtheorem{lemma}[definition]{Lemma}
\newtheorem{example}[definition]{Esempio}
\newtheorem{corollary}[definition]{Corollario}

\newcommand{\K}{\mathbb{K}}
\newcommand{\Q}{\mathbb{Q}}
\newcommand{\R}{\mathbb{R}}
\newcommand{\Z}{\mathbb{Z}}
\newcommand{\B}{\mathcal{B}}

\DeclareMathOperator{\Imm}{Imm}
\DeclareMathOperator{\Mat}{Mat}
\DeclareMathOperator{\Ker}{Ker}

\begin{document}

\maketitle
\tableofcontents

\chapter{Spazi Vettoriali}

\section{Definizione di spazio vettoriale}
Per fornire la definizione di spazio vettoriale si ha bisogno di un insieme non vuoto $V$ e di un campo
$\K$, dove sia possibile definire le operazioni di \textbf{somma vettoriale} e
\textbf{prodotto per scalare}.

\begin{definition}
	Uno \textbf{spazio vettoriale su un campo} $\K$ è un insieme $V$ su cui sono definite la somma
	fra due elementi di $V$ (il cui risultato è ancora un elemento di V, si dice quindi che $V$ è chiuso
	per la somma), e il prodotto di un elemento di $\K$ per un elemento di $V$ (il cui risultato è
	sempre un elemento di $V$, si dice quindi che V è chiuso per il prodotto con elementi di $\K$)
	che verificano le seguenti \textbf{proprietà}:
	\begin{enumerate}
		\item \textbf{Associatività della somma}: $\forall u, v, w \in V$ vale
		      \[ (u + v) + w = u + (v + w) \]
		\item \textbf{Commutatività della somma}: $\forall v, w \in V$ vale
		      \[ v + w = w + v \]
		\item \textbf{Elemento neutro per la somma}: $\exists O \in V$ tale che $\forall v \in V$ vale
		      \[ v + O = v \]
		\item \textbf{Inverso per la somma}: $\forall v \in V$, $\exists w \in V$ tale che
		      \[ v + w = O \]
		\item \textbf{Distributività del prodotto per uno scalare}: $\forall \lambda, \mu \in \K$ e
		      $\forall v, w \in V$ vale
		      \[ \lambda(v + w) = \lambda v + \lambda w \]
		      e anche
		      \[ (\lambda + \mu)v = \lambda v + \mu v \]
		\item \textbf{Associatività del prodotto per uno scalare}: $\forall \lambda, \mu \in \K$ e
		      $\forall v \in V$ vale
		      \[ (\lambda \mu)v = \lambda(\mu v) \]
		\item \textbf{Invariante moltiplicativo}: $\forall v \in V$ vale
		      \[ 1v = v \]
	\end{enumerate}
\end{definition}

\begin{observation}
	L'elemento neutro della somma $O$ e lo $0$, elemento neutro di $\K$ sono due cose ben distinte, il primo è
	un vettore, il secondo è uno scalare.
\end{observation}

\begin{example}
	Ogni campo $\K$ è uno spazio vettoriale su $\K$ stesso con le operazioni di somma vettoriale e prodotto per
	scalare che sono definite identiche alle operazioni di somma e prodotto sul campo. In particolare $\R$ è uno
	spazio vettoriale su $\R$, così come $\Q$ è uno spazio vettoriale su $\Q$.
\end{example}

\begin{example}
	$\R^2 = \{(a, b) \mid a,b \in \R\}$ è uno spazio vettoriale su $\R$ con le operazioni di somma vettoriale e
	prodotto scalare definite come segue:
	\begin{align*}
		(a,b) + (c,d) =  & (a + c, b + d)         \\
		\lambda (a, b) = & (\lambda a, \lambda b)
	\end{align*}
\end{example}

\begin{example}
	Anche l'insieme dei polinomi $\K[x]$, con la somma tra polinomi e il prodotto tra polinomi e costanti di
	$\K$ definiti come segue:
	\begin{itemize}
		\item Il polinomio somma di $p(x)$ e $q(x)$ è quello il cui coefficiente di grado $n$ è la somma dei
		      coefficienti di grado $n$ dei polinomi $p(x)$ e $q(x)$.
		\item Il polinomio prodotto di $k \in \K$ e $p(x)$ è il polinomio che ha come coefficiente di
		      grado $n$ $k$ volte il coefficiente di grado $n$ di $p(x)$.
	\end{itemize}
	è uno spazio vettoriale su $\K$.
\end{example}

\section{Sottospazi vettoriali}

\begin{definition}
	Un \textbf{sottospazio vettoriale} $W$ di $V$ è un sottoinsieme di $V$ che (rispetto alle operazioni $+$
	e $\cdot$ che rendono $V$ uno spazio vettoriale su $\K$) è uno spazio vettoriale su $\K$.
\end{definition}

\begin{example}
	Dato uno spazio vettoriale $V$ su un campo $\K$, $V$ e l'insieme ${O}$ sono sempre sottospazi di $V$.
\end{example}

\begin{definition}
	Chiamiamo \textbf{sottospazio proprio} di $V$ un qualsiasi sottospazio vettoriale di $V$ che sia diverso da
	$V$ e dal sottospazio ${O}$.
\end{definition}

\begin{proposition}
	Dato uno spazio vettoriale $V$ su $\K$ e $W \subseteq V$, $W$ è sottospazio vettoriale di $V$
	(rispetto alle operazioni $+$ e $\cdot$ che rendono $V$ uno spazio vettoriale su $\K$) se e solo
	se:
	\begin{enumerate}
		\item Il vettore $O$ appartiene a $W$.
		\item $\forall u, v \in W$ vale $u + v \in W$.
		\item $\forall k \in \K$ e $\forall u \in W$ vale $ku \in W$.
	\end{enumerate}
\end{proposition}

\begin{example}
	Consideriamo lo spazio vettoriale $\R^2$ su $\R$ e proviamo vedere se l'insieme
	$X = \{\forall x,y \in \R \mid x^2 + y^2 = 1\}$ è un sottospazio vettoriale di $\R^2$.
	L'insieme in questione è l'insieme di punti di una circonferenza. Subito notiamo che il vettore $(0, 0)$
	non appartiene all'insieme dunque possiamo subito concludere che $X$ non è un sottospazio di
	$\R^2$.
\end{example}

\begin{observation}
	Tutte le rette passanti per l'origine sono gli unici sottospazi vettoriali di $\R^2$. Tutti gli altri
	sottoinsiemi non sono chiusi per somma e prodotto.
\end{observation}

\begin{example}
	Consideriamo il sottoinsieme $L$ di $\K[x]$ che contiene tutti e soli i polinomi che hanno radice
	$1$, ovvero:
	\[ L = \{p(x) \in \K[x] \mid p(1) = 0\} \]
	Verifichiamo che $L$ è sottospazio vettoriale di $\K[x]$.
	\begin{itemize}
		\item Il polinomio $0$, che è il vettore $O$ di $\K[x]$, appartiene a $L$, infatti ha $1$
		      come radice (addirittura ogni elemento di $\K$ è una radice di $0$).
		\item Se $p(x), q(x) \in L$ allora $(p + q)(x)$ appartiene a $L$, infatti:
		      \[ (p + q)(1) = p(1) + q(1) = 0 + 0 = 0 \]
		\item Se $p(x) \in L$ e $k \in \K$ allora $k \cdot p(x) \in L$, infatti:
		      \[ (k \cdot p)(1) = k \cdot p(1) = k \cdot 0 = 0 \]
	\end{itemize}
\end{example}

\section{Intersezione e somma di sottospazi vettoriali}
Dati due sottospazi vettoriali $U$ e $W$ di uno spazio vettoriale $V$, la somma è il più piccolo sottospazio
vettoriale di $V$ che contenga sia $U$ che $W$ mentre l'intersezione è il più grande sottospazio vettoriale
di $V$ contenuto sia in $U$ che in $W$.

\begin{proposition}
	Sia $V$ uno spazio vettoriale su un campo $\K$, $U$ e $W$ due sottospazi di $V$, allora $U \cap W$
	è un sottospazio vettoriale di $V$.
	\begin{proof}
		Ci interessa verificare che $U \cap W$ verifichi le proprietà della definizione:
		\begin{enumerate}
			\item $O \in U \cap W$, infatti essendo $U$ e $W$ due sottospazi, certamente $O \in U$ e $O \in W$.
			\item Siano $v_1, v_2 \in U \cap W$ allora:
			      \[
				      \begin{cases}
					      v_1 + v_2 \in U \\
					      v_1 + v_2 \in W
				      \end{cases}
				      \Rightarrow{v_1 + v_2 \in U \cap W}
			      \]
			\item Sia $v \in U \cap W$ allora $\forall \lambda \in \K$ si ha:
			      \[
				      \begin{cases}
					      \lambda v \in U \\
					      \lambda v \in W
				      \end{cases}
				      \Rightarrow{\lambda v \in U \cap W}
			      \]
		\end{enumerate}
	\end{proof}
\end{proposition}

Per cercare il più piccolo sottospazio contenente sia $U$ che $W$ verrebbe da pensare all'unione insiemistica,
tuttavia, in generale non è vero che $U \cup W$ è un sottospazio vettoriale di $V$.

Dunque il più piccolo sottospazio vettoriale di $V$ che contiene sia $U$ che $W$ deve necessariamente (per
essere chiuso per la somma) contenere tutti gli elementi della forma $u + v$ dove $u \in U$ e $w \in W$.

\begin{example}
	Provare che se $V = \R^2$ e $U$ e $W$ sono due rette distinte passanti per $O$, allora $U \cup W$
	non è un sottospazio di $V$.

	Basta mostrare che, presi $u \in U$ e $w \in W$, entrambi diversi dall'origine, $v + w$ non appartiene
	all'unione $U \cup W$.
\end{example}

\begin{definition}
	Dati due sottospazi vettoriali $U$ e $W$ di uno spazio vettoriale $V$ su $\K$, chiamo \textbf{somma}
	di $U$ e $W$ l'insieme
	\[ U + W = \{u + w \mid u \in U, w \in W\} \]
\end{definition}

\begin{proposition}
	Dati due sottospazi vettoriali $U$ e $W$ di uno spazio vettoriale $V$ su $\K$, $U + W$ è un
	sottospazio vettoriale di $V$ (ed è il più piccolo contenente $U$ e $W$).
	\begin{proof}
		Partiamo col dire che $O \in U + W$, infatti $O$ appartiene sia ad $U$ che a $W$. Ora, dati
		$a \in \K$ e $x, y \in U + W$, per definizione di $U + W$ esistono $u_1, u_2 \in U$ e
		$w_1, w_2 \in W$ tali che
		\begin{align*}
			x = & u_1 + w_1 \\ y = & u_2 + w_2
		\end{align*}
		Dunque
		\[ x + y = (u_1 + w_1) + (u_2 + w_2) = (u_1 + u_2) + (w_1 + w_2) \in U + W \]
		e
		\[ ax = a(u_1 + w_1) = au_1 + aw_1 \in U + W \]
	\end{proof}
\end{proposition}


% BASI

\subsection{Base di uno spazio vettoriale}
Sia $V$ uno spazio vettoriale su $\mathbb{K}$.
Per definizione di $V$, se $v_1, v_2, ..., v_n$ sono $n$ vettori
di $V$, allora per qualsiasi scelta di $n$ elementi
$k_1, k_2, ..., k_n$ di $\mathbb{K}$ il vettore:
\begin{equation*}
	v = k_1 v_1 + ... + k_n v_n = \sum_{i=1}^n k_i v_i
\end{equation*}
appartiene a $V$, in quanto $V$ \`e chiuso per somma vettoriale
e prodotto per scalare.

\begin{defn}
	Dato un insieme di vettori $\{v_1, v_2, ..., v_n\}$ di $V$,
	spazio vettoriale sul campo $\mathbb{K}$, il vettore:
	\begin{equation*}
		v = k_1 \cdot v_1 + ... + k_n \cdot v_n
	\end{equation*}
	con $\{k_1, k_2, ..., k_n\}$ scalari di $\mathbb{K}$,
	si dice una \textbf{combinazione lineare} dei vettori
	$\{v_1, v_2, ..., v_n\}$. I $k_i$ sono detti \textbf{coefficienti}
	della combinazione lineare.
\end{defn}

\begin{example}
	Consideriamo lo spazio vettoriale $\mathbb{R}^3$ su $\mathbb{R}$ e i seguenti
	due vettori:
	\begin{equation*}
		v_1 = \begin{pmatrix}
			3 \\ -1 \\ 3
		\end{pmatrix}
		\quad
		v_2 = \begin{pmatrix}
			1 \\ 0 \\ 2
		\end{pmatrix}
	\end{equation*}
	Allora il vettore $v_3$ seguente:
	\begin{equation*}
		v_3 = \begin{pmatrix}
			5 \\ -1 \\ 7
		\end{pmatrix}
		= 1 \cdot \begin{pmatrix}
			3 \\ -1 \\ 3
		\end{pmatrix}
		+ 2 \cdot \begin{pmatrix}
			1 \\ 0 \\ 2
		\end{pmatrix}
	\end{equation*}
	\`E una combinazione lineare dell'insieme dei vettori $\{v_1, v_2\}$ di
	coefficienti 1 e 2.
\end{example}

\begin{defn}
	Dati $\{v_1, v_2, ..., v_t\}$ vettori di $V$ su $\mathbb{K}$,
	si definisce \textbf{span} dei vettori $v_1, v_2, ..., v_t$
	(e si indica con $Span(v_1, v_2, ..., v_t)$) l'insieme di tutte
	le possibili combinazioni lineari dell'insieme di vettori
	$\{v_1, v_2, ..., v_t\}$.
\end{defn}

\begin{defn}
	Un insieme di vettori $\{v_1, v_2, ..., v_t\}$ di $V$ per cui
	$V = Span(v_1, v_2, ..., v_t)$, ovvero $\forall v \in V$, esistono
	degli scalari $a_1, a_2, ..., a_t$ tali che
	\begin{equation*}
		a_1 v_1 + a_2 v_2 + ... + a_t v_t = v
	\end{equation*}
	si dice un \textbf{insieme di generatori} di $V$. In tal caso si dice
	anche che i vettori $v_1, v_2, ..., v_t$ \textbf{generano} $V$.
\end{defn}

L'esistenza di un sistema finito di generatori per uno spazio vettoriale
$V$ su un campo $\mathbb{K}$ \`e un fatto molto importante, dato che
si riduce la descrizione di uno spazio vettoriale con cardinalit\`a
infinita, ad una lista di numero finito di vettori di $V$.


Dato un sistema di generatori $\{v_1, ..., v_t\}$ di $V$ sappiamo dunque
che ogni $v \in V$ si pu\`o scrivere, con opportuni coefficienti
$\{k_1, ..., k_t\}$, come:
\begin{equation*}
	v = \sum_{i=1}^t k_i v_i
\end{equation*}
In generale, tale scrittura non \`e unica, ovvero non ci permette di
identificare univocamente ogni vettore di $v \in V$.

\begin{example}
	Si verifica che i vettori
	\begin{equation*}
		\begin{pmatrix}
			1 \\ 2 \\ 3
		\end{pmatrix} \quad
		\begin{pmatrix}
			1 \\ 0 \\ 1
		\end{pmatrix} \quad
		\begin{pmatrix}
			0 \\ 0 \\ 1
		\end{pmatrix} \quad
		\begin{pmatrix}
			2 \\ 2 \\ 4
		\end{pmatrix}
	\end{equation*}

	generano $\mathbb{R}^3$. Si possono facilmente trovare due distinte
	combinazioni lineari di tali vettori che esprimono il vettore
	\begin{equation*}
		\begin{pmatrix}
			2 \\ 2 \\ 5
		\end{pmatrix}
	\end{equation*}

	Per esempio:
	\begin{equation*}
		\begin{pmatrix}
			1 \\ 2 \\ 3
		\end{pmatrix} +
		\begin{pmatrix}
			1 \\ 0 \\ 1
		\end{pmatrix} +
		\begin{pmatrix}
			0 \\ 0 \\ 1
		\end{pmatrix} =
		\begin{pmatrix}
			2 \\ 2 \\ 5
		\end{pmatrix} =
		\begin{pmatrix}
			2 \\ 2 \\ 4
		\end{pmatrix} +
		\begin{pmatrix}
			0 \\ 0 \\ 1
		\end{pmatrix}
	\end{equation*}
\end{example}

\begin{defn}
	Si dice che un insieme finito di vettori $\{v_1, v_2, ..., v_r\}$ \`e
	un \textbf{insieme di vettori linearmente indipendenti} se l'unico
	modo di scrivere il vettore $O$ come combinazione lineare di questi
	vettori \`e con tutti i coefficienti nulli, ossia se
	\begin{equation*}
		a_1 v_1 + a_2 v_2 + ... + a_r v_r = O \quad \Leftrightarrow \quad
		a_1 = a_2 = ... = a_r = 0
	\end{equation*}
	Si pu\`o dire anche che i vettori sono \textbf{linearmente indipendenti}.
	Se invece i vettori $v_1, v_2, ..., v_r$ non sono linearmente indipendenti
	si dice che sono \textbf{linearmente dipendenti}.
\end{defn}

\begin{proposition}
	Un insieme $A = \{v_1, ..., v_n\}$ di vettori di uno spazio
	vettoriale $V$ su $\mathbb{K}$ \`e un insieme di vettori linearmente
	indipendenti se e solo se nessun $v_i$, appartenente ad $A$, si pu\`o
	scrivere come combinazione lineare dell'insieme
	$B = A \backslash \{v_i\}$ (ovvero $v_i$ non appartiene a $Span(B)$).
\end{proposition}

\begin{defn}
	Sia $V$ uno spazio vettoriale su $\mathbb{K}$, un insieme di vettori
	$\{v_1, v_2, ..., v_n\} \in V$, che generano lo spazio $V$ e che sono
	linearmente indipendenti, si dice una \textbf{base} (finita) di $V$.
\end{defn}

\begin{observation}
	Nella definizione \`e specificato \emph{finita}. Non sempre uno
	spazio vettoriale ammette un numero finito di generatori, e di
	conseguenza nemmeno una base finita.
\end{observation}

Fissata la defizione di base siamo interessati a capire:
\begin{enumerate}
	\item
	      Se la scelta di una base garantisce l'unicit\`a di scrittura di un vettore
	      in termini di combinazione lineare degli elementi della base.
	\item
	      Quando uno spazio vettoriale ammette una base finita, ed in
	      particolare se il fatto che uno spazio vettoriale $V$ abbia un
	      insieme finito di generatori, garantisca che $V$ abbia una
	      base finita o meno.
\end{enumerate}

\begin{proposition}
	Ogni vettore $v \in V$ si scrive \emph{in modo unico} come
	combinazione lineare degli elementi della base.
\end{proposition}

\begin{theorem}
	Sia $V$ uno spazio vettoriale su $\mathbb{K}$ diverso da $\{O\}$
	e generato dall'insieme finito di vettori \emph{non nulli}
	$\{w_1, w_2, ..., w_s\}$. Allora \`e possibile estrarre da
	$\{w_1, w_2, ..., w_s\}$ un sottoinsieme
	$\{w_{i_1}, w_{i_2}, ..., w_{i_n}\}$ (con $n \leq s$) che \`e
	una base di $V$.
\end{theorem}

\begin{defn}
	Sia $V$  uno spazio vettoriale con basi di cardinalit\`a $n$.
	Tale cardinalit\`a $n$ \`e detta \textbf{dimensione} di $V$.
\end{defn}

\section{Applicazioni Lineari}
Le applicazioni lineari non sono altro che funzioni che mandano sottospazi in sottospazi.

\begin{example}
	Consideriamo la funzione $\textit{f} : \R^2 \to \R^2$ definita da
	\[
		f \begin{pmatrix}
			x \\ y
		\end{pmatrix}
		=
		\begin{pmatrix}
			x \\ x^2
		\end{pmatrix}
	\]
	La funzione $f$ manda il punto $(x, x)$, con la prima e seconda coordinata uguali, ovvero i punti della
	retta di equazione $x = y$, nella parabola di equazione $y = x^2$. Ma, come sappiamo, la retta $y = x$,
	passando dall'origine, è un sottospazio di $\R^2$, mentre la parabola non lo è. Si devono dunque
	considerare applicazioni con proprietà particolari.
\end{example}

\begin{definition}
	Siano $V$ e $W$ spazi vettoriali di dimensione finita sul campo $\K$. Un'applicazione $L$ da $V$
	in $W$ è detta \textbf{lineare} se soddisfa le seguenti proprietà:
	\begin{itemize}
		\item $\forall v_1, v_2 \in V$ vale
		      \[ L(v_1 + v_2) = L(v_1) + L(v_2) \]
		\item $\forall \lambda \in \K$ e $\forall v \in V$ vale
		      \[ L(\lambda v) = \lambda L(v) \]
	\end{itemize}
\end{definition}

\begin{observation}
	Soddisfare le due proprietà, da parte di un'applicazione lineare $L$, è equivalente a soddisfare la seguente
	proprietà: $\forall v_1, v_2 \in V$ e $\forall \lambda, \mu \in \K$ vale
	\[ L(\lambda v_1 + \mu v_2) = \lambda L(v_1) + \mu L(v_2) \]
\end{observation}

\begin{definition}
	Siano $V$ e $W$ spazi vettoriali su $\K$ e $L$ un'applicazione lineare da $V$ in $W$. Chiamo
	\textbf{immagine} di $L$, e la indico con $\Imm(L)$, il seguente sottoinsieme di $W$:
	\[ \Imm(L) = \{w \in W \mid \forall v \in V, \quad L(v) = w\} \]
\end{definition}

\begin{proposition}
	È dimostrabile che \[ \Imm(L) = Span(L(e_1), \dots, L(e_n)) \] dove $\{e_1, \dots, e_n\}$ è una base di $V$.
\end{proposition}

\begin{definition}
	Siano $V$ e $W$ spazi vettoriali su $\K$ e $L$ un'applicazione lineare da $V$ in $W$. Chiamo
	\textbf{nucleo} di $L$, e lo indico con $\Ker(L)$, il seguente sottoinsieme di $V$:
	\[ \Ker(L) = \{v \in V \mid L(v) = O\} \]
\end{definition}

Di seguito qualche proprietà utile:
\begin{enumerate}
	\item $\Ker(L)$ è un sottospazio vettoriale di $V$.
	\item $\Imm(L)$ è un sottospazio vettoriale di $W$.
	\item $L$ è iniettiva se e solo se $\Ker(L) = \{O\}$.
\end{enumerate}



% MATRICI E VETTORI

\subsection{Matrici e vettori}

\begin{defn}
	Dati due interi positivi $m, n$, una \textbf{matrice} $m \times n$
	a coefficienti in $\mathbb{K}$ \`e una griglia composta da $m$ righe
	e $n$ colonne in cui in ogni posizione c'\`e un elemento di
	$\mathbb{K}$:
	\begin{equation*}
		A = \begin{pmatrix}
			a_{11} & a_{12} & \dots & a_{1n} \\
			a_{21} & a_{22} & \dots & \dots  \\
			\dots  & \dots  & \dots & \dots  \\
			a_{m1} & \dots  & \dots & a_{mn}
		\end{pmatrix}
	\end{equation*}
	Per indicare l'elemento che si trova nella riga $i$-esima
	dall'alto e	nella colonna $j$-esima da sinistra viene indicato
	con $a_{ij}$. Spesso per indicare la matrice $A$ useremo la notazione
	$A = (a_{ij})$ e per ricordare le dimensioni della matrice scriveremo:
	\begin{equation*}
		A = (a_{ij})_{\substack{
					i = 1, 2, \dots, m \\
					j = 1, 2, \dots, n
				}}
	\end{equation*}
\end{defn}

\begin{defn}
	Dati due interi positivi m, n, chiamiamo
	$Mat_{m \times n} (\mathbb{K})$, l'insieme di tutte le matrici
	$m \times n$ a coefficienti in $\mathbb{K}$.
\end{defn}

\begin{defn}
	Sull'insieme $Mat_{m \times n} (\mathbb{K})$ possiamo definire la
	\textbf{somma} e il \textbf{prodotto per scalare}. Date due matrici
	$A, B \in Mat_{m \times n} (\mathbb{K})$ e dato uno scalare
	$k \in \mathbb{K}$, definiamo:
	\begin{itemize}
		\item
		      La \textbf{matrice somma} $A + B = C$, il cui generico coefficiente nella
		      $i$-esima riga e $j$-esima colonna si ottiene sommando i
		      coefficienti nella stessa posizione indicata da $(i, j)$ di $A$ e
		      di $B$. Ovvero per ogni $i \leq m$ e per ogni $j \leq n$
		      ho che $c_{ij} = a_{ij} + b_{ij}$.
		\item
		      La \textbf{matrice prodotto per scalare} $k \cdot A = D$, il cui
		      generico coefficiente nella $i$-esima riga e $j$-esima
		      colonna si ottiene moltiplicando lo scalare $k$ per il
		      coefficiente di $A$ in posizione $(i, j)$. Ovvero per ogni
		      $i \leq m$ e per ogni $j \leq n$ ho che
		      $d_{ij} = k \cdot a_{ij}$.
	\end{itemize}
\end{defn}

Esiste un'altra operazione, si tratta del
\emph{prodotto righe per colonne}. Per definire tale prodotto \`e
importante l'ordine in cui si considerano le due matrici (quindi non vale la
proprit\`a commutativa). Inoltre tale operazione \`e definita solo quando il
numero di colonne di $A$ \`e uguale al numero di righe di $B$.

\begin{defn}
	Data una matrice $A = (a_{ij}) \in Mat_{m \times n} (\mathbb{K})$ e una
	matrice $B = (b_{st}) \in Mat_{n \times k} (\mathbb{K})$, il
	\textbf{prodotto riga per colonna} $AB$, \`e la matrice
	$C = (c_{rh}) \in Mat_{m \times k} (\mathbb{K})$, i cui coefficienti,
	per ogni $r, h$, sono definiti come segue:
	\begin{equation*}
		c_{rh} = a_{r1} b_{1h} + a_{r2} b_{2h} + \cdots + a_{rn} b_{nh}
	\end{equation*}
\end{defn}

\begin{example}
	Consideriamo la matrice $A \in Mat_{2 \times 3}(\mathbb{K})$:
	\begin{equation*}
		A = \begin{pmatrix}
			1 & 2 & 4 \\
			0 & 6 & 3
		\end{pmatrix}
	\end{equation*}
	e la matrice $B \in Mat_{3 \times 3}(\mathbb{K})$:
	\begin{equation*}
		B = \begin{pmatrix}
			2 & 2 & 2  \\
			5 & 6 & -8 \\
			0 & 1 & 0
		\end{pmatrix}
	\end{equation*}
	La definizione ci dice che possiamo definire $C = AB$ e che $C$ \`e la matrice di
	$Mat_{2 \times 3}(\mathbb{K})$ i cui coefficienti sono ottenuti come segue:
	\begin{gather*}
		c_{11} = 1 \cdot 2 + 2 \cdot 5 + 4 \cdot 0 = 12 \\
		c_{12} = 1 \cdot 2 + 2 \cdot 6 + 4 \cdot 1 = 18 \\
		c_{13} = 1 \cdot 2 + 2 \cdot (-8) + 4 \cdot 0 = -14 \\
		c_{21} = 0 \cdot 2 + 6 \cdot 5 + 3 \cdot 0 = 30 \\
		c_{22} = 0 \cdot 2 + 6 \cdot 6 + 3 \cdot 1 = 39 \\
		c_{23} = 0 \cdot 2 + 6 \cdot (-8) + 3 \cdot 0 = -48
	\end{gather*}
	E dunque si ha:
	\begin{equation*}
		AB = C = \begin{pmatrix}
			12 & 18 & -14 \\
			30 & 39 & -48
		\end{pmatrix}
	\end{equation*}
\end{example}

\begin{defn}
	Data un'applicazione lineare $L$ da uno spazio vettoriale $V$ di
	dimensione $n$ ad uno spazio vettoriale $W$ di dimensione $m$, si dice
	\textbf{matrice associata} all'applicazione lineare $L$ nelle basi
	$\{e_1, e_2, \dots, e_n\}$ di $V$ e
	$\{\epsilon_1, \epsilon_2, \dots, \epsilon_m\}$ di $W$, la seguente
	matrice di $m$ righe ed $n$ colonne:
	\begin{equation*}
		[L]_{\substack{
				e_1, e_2, \dots, e_n\\
				\epsilon_1, \epsilon_2, \dots, \epsilon_m
			}} = \begin{pmatrix}
			a_{11} & a_{12} & \dots & a_{1n} \\
			a_{21} & a_{22} & \dots & \dots  \\
			\dots  & \dots  & \dots & \dots  \\
			a_{m1} & \dots  & \dots & a_{mn}
		\end{pmatrix}
	\end{equation*}
	dove $\{e_1, e_2, \dots, e_n\}$ \`e la base di partenza e
	$\{\epsilon_1, \epsilon_2, \dots, \epsilon_m\}$ \`e la base di arrivo.
\end{defn}

Sar\`a tutto pi\`u chiaro con un esempio che vedremo tra poco. In ogni caso, per
alleggerire la notazione, si possono omettere le basi, tuttavia si deve ricordare che
la matrice $[L]$ associata all'applicazione lineare $L$, non dipende solo da $L$ stessa,
ma anche dalle basi scelte per $V$ e $W$.

\begin{example}
	Consideriamo gli spazi vettoriali $\mathbb{R}^4$ con la sua base
	\begin{equation*}
		v_1 = \begin{pmatrix}
			1 \\ 1 \\ 0 \\ 0
		\end{pmatrix} \quad
		v_2 = \begin{pmatrix}
			0 \\ 1 \\ 1 \\ 0
		\end{pmatrix} \quad
		v_3 = \begin{pmatrix}
			0 \\ 0 \\ 1 \\ 1
		\end{pmatrix} \quad
		v_4 = \begin{pmatrix}
			0 \\ 0 \\ 0 \\ 1
		\end{pmatrix}
	\end{equation*}
	e $\mathbb{R}^3$ con la sua base
	\begin{equation*}
		w_1 = \begin{pmatrix}
			1 \\ 0 \\ 1
		\end{pmatrix} \quad
		w_2 = \begin{pmatrix}
			1 \\ 1 \\ 1
		\end{pmatrix} \quad
		w_3 = \begin{pmatrix}
			0 \\ 0 \\ 2
		\end{pmatrix}
	\end{equation*}
	Quel che vogliamo fare \`e scrivere la matrice associata all'applicazione lineare
	\begin{equation*}
		L \begin{pmatrix}
			x \\ y \\ z \\ w
		\end{pmatrix} = \begin{pmatrix}
			x + y + z \\
			y - z     \\
			x + w
		\end{pmatrix}
	\end{equation*}
	Procediamo calcolando l'immagine di ognuna delle componenti della base di
	$\mathbb{R}^4$
	\begin{gather*}
		L \begin{pmatrix}
			1 \\ 1 \\ 0 \\ 0
		\end{pmatrix} =
		\begin{pmatrix}
			2 \\ 1 \\ 1
		\end{pmatrix} \quad
		L \begin{pmatrix}
			0 \\ 1 \\ 1 \\ 0
		\end{pmatrix} =
		\begin{pmatrix}
			2 \\ 0 \\ 0
		\end{pmatrix} \\
		L \begin{pmatrix}
			0 \\ 0 \\ 1 \\ 1
		\end{pmatrix} =
		\begin{pmatrix}
			1 \\ -1 \\ 1
		\end{pmatrix} \quad
		L \begin{pmatrix}
			0 \\ 0 \\ 0 \\ 1
		\end{pmatrix} =
		\begin{pmatrix}
			0 \\ 0 \\ 1
		\end{pmatrix}
	\end{gather*}
	Ora dobbiamo esprimere i risultati trovati come combinazioni lineari della base di
	$\mathbb{R}^3$.
	\begin{gather*}
		\begin{pmatrix}
			2 \\ 1 \\ 1
		\end{pmatrix} =
		a \begin{pmatrix}
			1 \\ 0 \\ 1
		\end{pmatrix} +
		b \begin{pmatrix}
			1 \\ 1 \\ 1
		\end{pmatrix} +
		c \begin{pmatrix}
			0 \\ 0 \\ 2
		\end{pmatrix}\\
		\\
		\begin{pmatrix}
			2 \\ 0 \\ 0
		\end{pmatrix} =
		a \begin{pmatrix}
			1 \\ 0 \\ 1
		\end{pmatrix} +
		b \begin{pmatrix}
			1 \\ 1 \\ 1
		\end{pmatrix} +
		c \begin{pmatrix}
			0 \\ 0 \\ 2
		\end{pmatrix}\\
		\\
		\begin{pmatrix}
			1 \\ -1 \\ 1
		\end{pmatrix} =
		a \begin{pmatrix}
			1 \\ 0 \\ 1
		\end{pmatrix} +
		b \begin{pmatrix}
			1 \\ 1 \\ 1
		\end{pmatrix} +
		c \begin{pmatrix}
			0 \\ 0 \\ 2
		\end{pmatrix}\\
		\\
		\begin{pmatrix}
			0 \\ 0 \\ 1
		\end{pmatrix} =
		a \begin{pmatrix}
			1 \\ 0 \\ 1
		\end{pmatrix} +
		b \begin{pmatrix}
			1 \\ 1 \\ 1
		\end{pmatrix} +
		c \begin{pmatrix}
			0 \\ 0 \\ 2
		\end{pmatrix}
	\end{gather*}
	Ottengo dunque quattro sistemi.
	\begin{gather*}
		\begin{cases}
			a + b      & = 2 \\
			b          & = 1 \\
			a + b + 2c & = 1
		\end{cases}
		\quad
		\begin{cases}
			a + b      & = 2 \\
			b          & = 0 \\
			a + b + 2c & = 0
		\end{cases} \\
		\\
		\begin{cases}
			a + b      & = 1  \\
			b          & = -1 \\
			a + b + 2c & = 1
		\end{cases}
		\quad
		\begin{cases}
			a + b      & = 0 \\
			b          & = 0 \\
			a + b + 2c & = 2
		\end{cases}
	\end{gather*}
	Se li risolvo ottengo
	\begin{gather*}
		\begin{cases}
			a & = 1  \\
			b & = 1  \\
			c & = -1
		\end{cases}
		\quad
		\begin{cases}
			a & = 2  \\
			b & = 0  \\
			c & = -1
		\end{cases} \\
		\\
		\begin{cases}
			a & = 2  \\
			b & = -1 \\
			c & = 0
		\end{cases}
		\quad
		\begin{cases}
			a & = 0 \\
			b & = 0 \\
			c & = 1
		\end{cases}
	\end{gather*}
	Ora non devo fare altro che prendere i vettori
	\begin{equation*}
		\begin{pmatrix}
			1 \\ 1 \\ -1
		\end{pmatrix}
		\quad
		\begin{pmatrix}
			2 \\ 0 \\ -1
		\end{pmatrix}
		\quad
		\begin{pmatrix}
			2 \\ -1 \\ 0
		\end{pmatrix}
		\quad
		\begin{pmatrix}
			0 \\ 0 \\ -1
		\end{pmatrix}
	\end{equation*}
	e formare la matrice associata all'applicazione lineare $L$.
	\begin{equation*}
		[L]_{\substack{v_1, v_2, v_3, v_4 \\
				w_1, w_2, w_3}} =
		\begin{pmatrix}
			1  & 2  & 2  & 0  \\
			1  & 0  & -1 & 0  \\
			-1 & -1 & 0  & -1
		\end{pmatrix}
	\end{equation*}
\end{example}

\begin{observation}
	Dati due spazi vettoriali $V$ e $W$, esiste una sola applicazione
	lineare da $V$ a $W$ la cui matrice associata \`e indipendente dalle
	basi scelte. Questa \`e l'\emph{applicazione nulla}
	$\mathcal{O} : V \rightarrow W$ che manda ogni $v \in V$ in $O \in W$.
	Qualunque siano le basi scelte, la matrice associata avr\`a tutti
	i coefficienti uguali a $0$.
\end{observation}

\begin{observation}
	Consideriamo l'applicazione \emph{identit\`a} $I : V \rightarrow V$,
	che lascia fisso ogni elemento di $v: I(v) = v \forall v \in V$, e
	fissiamo la base $\mathcal{B}$ di $V$. Si verifica che la matrice
	$[I] = (a_{ij})$, associata ad $I$ rispetto a $\mathcal{B}$, sia in
	arrivo che in partenza, \`e la matrice quadrata di formato $n \times n$
	che ha tutti i coefficienti uguali a $0$ eccetto quelli sulla
	diagonale, che sono invece uguali a $1$.

	Tale matrice \`e l'elemento neutro rispetto alla moltiplicazione riga
	per colonna in $Mat_{n \times n}(\mathbb{K})$.

	In seguito useremo solo il simbolo $I$ per indicare sia la matrice identit\`a
	che	l'applicazione lineare $I$.
\end{observation}

\chapter{Riduzione a scalini}
\section{Operazioni elementari sulle colonne}
Consideriamo una generica matrice in $\Mat_{m \times n}(\K)$:
\[
	A = \begin{pmatrix}
		a_{11} & a_{12} & \dots & a_{1n} \\
		a_{21} & a_{22} & \dots & \dots  \\
		\dots  & \dots  & \dots & \dots  \\
		a_{m1} & \dots  & \dots & a_{mn}
	\end{pmatrix}
\]
e i tre seguenti tipi di mossa sulle colonne, detti anche
\textbf{operazioni elementari sulle colonne}:
\begin{enumerate}
	\item si somma alla colonna $i$ la colonna $j$ moltiplicata per uno scalare
	      $\lambda$.
	\item si moltiplica la colonna $i$ per uno scalare $\lambda$
	\item si scambiano fra di loro due colonne $i$ e $j$.
\end{enumerate}

\begin{definition}
	La \textbf{profondità} di una colonna è definita come la posizione
	occupata (contata dal basso) dal suo più alto coefficiente diverso da 0.
	Alla colonna nulla si assegna per convenzione profondità uguale a 0.
\end{definition}

\begin{example}
	Consideriamo la seguente matrice:
	\[
		A = \begin{pmatrix}
			0            & 4 - \sqrt{3} & 0  \\
			\sqrt{3} + 1 & 0            & 0  \\
			-2           & -2           & -2
		\end{pmatrix}
	\]
	In questo caso la prima colonna ha profodità uguale a 2, la
	seconda colonna uguale a 3 mentre la terza ha profondità uguale a 1.
\end{example}

\begin{definition}
	Una matrice $A \in \Mat_{m \times n}(\K)$, si dice
	\textbf{in forma a scalini per colonne} se rispetta le seguenti proprietà:
	\begin{itemize}
		\item leggendo la matrice da sinistra a destra, le colonne non nulle si
		      incontrano tutte prima delle colonne nulle.
		\item leggendo la matrice da sinistra a destra, le profondità
		      delle sue colonne non nulle risultano strettamente
		      decrescenti.
	\end{itemize}
\end{definition}

\begin{example}
	Questi sono esempi di matrici in forma \emph{a scalini}:
	\[
		A = \begin{pmatrix}
			1            & 0           & 0 & 0 \\
			\sqrt{3} + 1 & 1           & 0 & 0 \\
			-2           & \frac{5}{2} & 1 & 0
		\end{pmatrix}
	\]
	\[
		B = \begin{pmatrix}
			1  & 0           & 0 & 0 \\
			0  & 1           & 0 & 0 \\
			-2 & \frac{5}{2} & 0 & 0
		\end{pmatrix}
	\]
\end{example}

\begin{definition}
	In una matrice in forma a scalini per colonna, i coefficienti diversi
	da zero più alti di posizione di ogni colonna non nulla si
	chiamano \textbf{pivot}.
\end{definition}

\begin{theorem}
	Data una matrice $A \in \Mat_{m \times n}(\K)$ è sempre
	possibile, usando operazioni elementari sulle colonne, ridurre la
	matrice in forma a scalini per colonne.
\end{theorem}

\begin{observation}
	Quando si riduce una matrice in forma a scalini, la forma a scalini
	ottenuta non è unica.
\end{observation}

\begin{definition}
	Una matrice $A \in \Mat_{m \times n}(\K)$, si dice \textbf{in forma a
		scalini per colonne ridotta} se:
	\begin{itemize}
		\item $A$ è a scalini per colonne.
		\item Tutte le entrate nella stessa riga di un pivot, precedenti al pivot,
		      sono nulle
	\end{itemize}
\end{definition}

\begin{example}
	Esempio di matrice in forma a scalini ridotta:
	\[
		\begin{pmatrix}
			1            & 0           & 0 & 0 \\
			\sqrt{3} + 1 & 1           & 0 & 0 \\
			-2           & \frac{5}{2} & 1 & 0
		\end{pmatrix} \rightarrow
		\begin{pmatrix}
			1 & 0 & 0 & 0 \\
			0 & 1 & 0 & 0 \\
			0 & 0 & 1 & 0
		\end{pmatrix}
	\]
\end{example}

\begin{proposition}
	Data una matrice in forma a scalini per colonne, è sempre possibile, usando
	solo la prima delle operazioni elementari sulle colonne, portare $A$ in forma
	a scalini ridotta.
\end{proposition}

\begin{corollary}
	Ogni matrice $A$ può essere trasformata, tramite le operazioni elementari
	sulle colonne, in una matrice in forma a scalini per colonne ridotta.
\end{corollary}

\begin{proposition}
	Se operiamo attraverso le operazioni elementari sulle colonne, lo
	$Span$ dei vettori colonna rimane invariato. Ovvero se indichiamo con
	$v_1, \dots, v_n$ i vettori colonna di una matrice
	$A \in \Mat_{m \times n}(\K)$, per ogni matrice $A'$ ottenuta da $A$
	attraverso le operazioni elementari sulle colonne, si ha, indicando con
	$w_1, \dots, w_n$ i vettori colonna di $A'$, che:
	\[
		Span(v_1, \dots, v_n) = Span(w_1, \dots, w_n)
	\]
\end{proposition}

\section{Riduzione a scalini e studio delle basi}

\begin{theorem}
	Sia $V$ uno spazio vettoriale su $\K$ che ammette una base
	finita, allora tutte le basi di $V$ hanno la stessa cardinalità.
\end{theorem}

\begin{corollary}
	In uno spazio vettoriale $V$ di dimensione $n$, dati $n$ vettori
	linearmente indipendenti questi sono anche una base di $V$. Allo
	stesso modo, dati $n$ vettori che generano $V$, questi sono anche una
	base di $V$.
\end{corollary}

Considerzioni simili a quelle fatte fino ad ora ci consentono di descrivere
un criterio concreto per decidere se, dato uno spazio vettoriale $V$ di
dimensione $n$ ed una base $e_1, \dots, e_n$ di $V$, e dati $n$ vettori
$v_1, \dots, v_n$ di $V$, tali vettori costituiscono una base di $V$ o no.

Il criterio è espresso dai seguenti punti:
\begin{enumerate}
	\item Esprimiamo i vettori $v_1, \dots, v_n$ trovandone i coefficienti rispetto
	      alla base $e_1, \dots, e_n$.
	\item Poniamoli in colonna uno accanto all'altro. Così facendo otteniamo una
	      matrice $M$ che è $n \times n$.
	\item Riduciamo $M$ in forma a scalini ridotta ottenendo $M'$.
	\item A questo punto, se $M'$ è l'identità, allora $\{v_1, \dots, v_n\}$
	      è una base di $V$, altrimenti no.
\end{enumerate}

\begin{example}
	Verificare che i seguenti vettori siano una base di $\R^4$.
	\[
		v_1 = \begin{pmatrix}
			1 \\ 1 \\ 0 \\ 0
		\end{pmatrix} \quad
		v_2 = \begin{pmatrix}
			0 \\ 1 \\ 1 \\ 0
		\end{pmatrix} \quad
		v_3 = \begin{pmatrix}
			0 \\ 0 \\ 1 \\ 1
		\end{pmatrix} \quad
		v_4 = \begin{pmatrix}
			0 \\ 0 \\ 0 \\ 1
		\end{pmatrix}
	\]
	Verifichiamo col nuovo metodo che si tratta di una base.
	Scriviamo dunque la matrice
	\[
		\begin{pmatrix}
			1 & 0 & 0 & 0 \\
			1 & 1 & 0 & 0 \\
			0 & 1 & 1 & 0 \\
			0 & 0 & 1 & 1 \\
		\end{pmatrix}
	\]
	Portandola in forma a scalini ridotta diventa
	\[
		\begin{pmatrix}
			1 & 0 & 0 & 0 \\
			0 & 1 & 0 & 0 \\
			0 & 0 & 1 & 0 \\
			0 & 0 & 0 & 1
		\end{pmatrix}
	\]
	Dunque $\{v_1, v_2, v_3, v_4\}$ è una base di $\R^4$.
\end{example}

\begin{theorem}[Completamento]
	Dato uno spazio vettoriale $V$ di dimensione $n$, ogni
	sottoinsieme $B = \{v_1, \dots, v_k\} \subset V$ di vettori
	linearmente indipendenti di cardinalità $k$, con
	$1 \leq k \leq n$, può essere completato ad una base di $V$
	aggiungendo a $B$ $n - k$ vettori di $V \backslash Span(B)$.
	\begin{proof}
		La dimostrazione che segue ci fornisce un algoritmo per trovare
		vettori da aggiungere al sottoinsieme $B$ per riuscire a trovare
		una base di $V$.
		\begin{enumerate}
			\item Si scrivono i vettori $v_1, \dots, v_k$ come
			      vettori colonna rispetto a una base data, e si considera
			      la matrice $M$ che ha questi vettori come colonne.
			\item Si riduce $M$ in forma a scalini.
			\item Tutte le volte le volte che troviamo uno scalino lungo
			      (altezza $\geq 2$, dove l'altezza è la differenza di profondità
			      tra due colonne adiacenti che formano lo scalino lungo) possiamo
			      facilmente trovare $i - 1$ vettori $w_1, \dots, w_{i - 1}$ tali che
			      $\{v_1, \dots, v_k, w_1, \dots, w_{i - 1}\}$ è ancora un
			      insieme di vettori linearmente indipendenti.
			\item Supponiamo infatti che $M$ abbia uno scalino di lunghezza $i$,
			      ovvero in una certa colonna abbia il pivot alla riga $t$ e nella
			      colonna successiva alla riga $t+i$, e sia $i > 1$. Allora
			      per $j$ che varia tra 1 ed $i - 1$.
			\item Scegliere un vettore $w_j$ tale che abbia tutti
			      0 tranne un 1 in corrispondeza della riga $(t + j)$-esima.
			      È facile osservare che ogni $w_j$ non appartiene
			      allo $Span(v_1, \dots, v_k, w_1, \dots, w_{j-1})$. Dunque ad
			      ogni aggiunta di $w_j$, l'insieme
			      $\{v_1, \dots, v_k, w_1, \dots, w_j\}$ rimane un insieme di
			      vettori linearmente indipendenti di $V$.
			\item Ripetiamo questa aggiunta di vettori $w_k$ per ogni scalino
			      lungo che troviamo in $M$, trovando così alla fine $n$ vettori
			      linearmente indipendenti, dunque una base di $V$ come richiesto.
		\end{enumerate}
	\end{proof}
\end{theorem}

\begin{example}
	Illustriamo il metodo descritto con un esempio. Supponiamo che
	$V = \R^7$ e siano dati i 4 vettori linearmente
	indipendenti che, scritti rispetto alla base standard di
	$\R^7$, sono rappresentati come segue:
	\[
		v_1 = \begin{pmatrix}
			1 \\ 1 \\ 0 \\ 2 \\ 1 \\ 0 \\ 1
		\end{pmatrix} \quad
		v_2 = \begin{pmatrix}
			0 \\ 1 \\ 0 \\ 2 \\ 1 \\ 0 \\ 3
		\end{pmatrix} \quad
		v_3 = \begin{pmatrix}
			0 \\ 0 \\ 0 \\ 0 \\ 1 \\ 0 \\ 1
		\end{pmatrix} \quad
		v_4 = \begin{pmatrix}
			0 \\ 0 \\ 0 \\ 0 \\ 1 \\ 0 \\ 2
		\end{pmatrix}
	\]
	La matrice $M$ in questo caso è:
	\[
		M = \begin{pmatrix}
			1 & 0 & 0 & 0 \\
			1 & 1 & 0 & 0 \\
			0 & 0 & 0 & 0 \\
			2 & 2 & 0 & 0 \\
			1 & 1 & 1 & 1 \\
			0 & 0 & 0 & 0 \\
			1 & 3 & 1 & 2
		\end{pmatrix}
	\]
	Riduciamola a scalini per colonne.
	\[
		M' = \begin{pmatrix}
			1 & 0 & 0 & 0 \\
			0 & 1 & 0 & 0 \\
			0 & 0 & 0 & 0 \\
			0 & 2 & 0 & 0 \\
			1 & 0 & 1 & 0 \\
			0 & 0 & 0 & 0 \\
			0 & 1 & 2 & 1
		\end{pmatrix}
	\]
	Il primo scalino lungo ha altezza 3. Come osservato nella dimostrazione
	del teorema del completamento, i vettori
	\[
		w_1 = \begin{pmatrix}
			0 \\ 0 \\ 1 \\ 0 \\ 0 \\ 0 \\ 0
		\end{pmatrix} \quad
		w_2 = \begin{pmatrix}
			0 \\ 0 \\ 0 \\ 1 \\ 0 \\ 0 \\ 0
		\end{pmatrix}
	\]
	non appartengono al sottospazio generato dalle colonne di $M'$, e si
	osserva immediatamente che $\{v_1, v_2, v_3, v_4, w_1, w_2\}$ è un
	insieme di vettori linearmente indipendenti di $\R^7$.

	Similmente prendendo in considerazione il secondo scalino lungo, notiamo
	che il vettore
	\[
		w_3 = \begin{pmatrix}
			0 \\ 0 \\ 0 \\ 0 \\ 0 \\ 1 \\ 0
		\end{pmatrix}
	\]
	non appartiene al sottospazio generato da $v_1, v_2, v_3, v_4, w_1, w_2$.

	A questo punto abbiamo ottenuto l'insieme di vettori linearmente indipendenti
	$\{v_1, v_2, v_3, v_4, w_1, w_2, w_3\}$ di $\R^7$ e dato che
	$\R^7$ ha dimensione 7 abbiamo ottenuto una base di $\R^7$.
\end{example}


\section{Il teorema della dimensione del nucleo e dell'immagine di una applicazione lineare}
Il teorema del completamento, ha come importante corollario un risultato che stabilisce
una relazione tra la dimensione del nucleo e quella dell'immagine di una applicazione
lineare.

\begin{theorem}
	Considerata una applicazione lineare $L : V \to W$, dove $V$ e $W$ sono spazi
	vettoriali su $\K$, vale
	\[
		\dim(\Ker(L)) + \dim(\Imm(L)) = \dim(V)
	\]
\end{theorem}

\begin{definition}
	Una applicazione lineare bigettiva $L : V \to W$, tra due spazi vettoriali $V$ e $W$
	sul campo $\K$, si dice un \textbf{isomorfismo lineare}.

	Dal teorema precedente segue che:
	\begin{itemize}
		\item Se $L : V \to W$ è una applicazione lineare iniettiva allora
		      \[ \dim(\Imm(L)) = \dim(V) \]
		      Infatti sappiamo che $\Ker(L) = \{O\}$ dunque $\dim(\Ker(L)) = 0$.
		\item Se $L : V \to W$ è un isomorfismo lineare allora \[\dim(V) = \dim(W)\]
		      Infatti se $L$ è bigettiva, in particolare è iniettiva e surgettiva.
		\item Se $L : V \to W$ è una applicazione lineare iniettiva, allora $L$,
		      pensata come applicazione da $V$ ad $\Imm(L)$, è un isomorfismo lineare.
	\end{itemize}
\end{definition}

\section{Immagine di un'applicazione lineare}
La riduzione a scalini per colonna risulta molto utile anche per lo
studio di applicazioni lineari, ed in particolare per la
determinazione di dimensione e base dell'immagine di una applicazione
lineare.

Dati due spazi vettoriali $V$ e $W$ sul campo $\K$ di dimensione $n$
e $m$ rispettivamente, consideriamo una applicazione lineare:
\[
	L : V \to W
\]
Fissiamo una base $\{e_1, e_2, \dots, e_n\}$ di $V$ e una base
$\{\epsilon_1, \epsilon_2, \dots, \epsilon_m\}$ di $W$. Indichiamo con $[L]$
la matrice, di forma $m \times n$, associata a $L$ nelle basi scelte.

Per quanto abbiamo fin qui detto, possiamo, tramite un numero finito $k$ di
operazioni elementari sulle colonne di $[L]$, portarla in forma a scalini
ridotta. Ma c'è di più. Ogni operazione elementare sulle colonne corrisponde
a moltiplicare la matrice iniziale $[L] \in \Mat_{m \times n}(\K)$,
a destra, per una matrice $B$ di dimensioni $n \times n$ invertibile.

\begin{theorem}
	Siano $V, W, U$ spazi vettoriali su $\K$, fissiamo per ciascuno
	una base. Siano $T : V \to W$ e $S : W \to U$ applicazioni lineari. Allora,
	rispetto alle basi fissate, vale:
	\[
		[S \circ T] = [S][T]
	\]
	dove nel membro di destra stiamo considerando il prodotto righe per colonne
	fra matrici.
\end{theorem}

Tornando all'applicazione lineare $L$, sappiamo già che lo span delle colonne
di $[L]$ coincide con lo span delle colonne della matrice ottenuta portando $[L]$
in forma a scalini. Sappiamo cioè che $\Imm(L)$ coincide con l'immagine
dell'applicazione lineare associata alla matrice ottenuta portando i forma a
scalini $[L]$.

\begin{proposition}
	Siano $L$ ed $M$ due applicazioni lineari, vale
	\[ \Imm(L \circ M) = \Imm(L) \]
	ossia, scritto con un'altra notazione,
	\[ (L \circ M)(V) = L(V) \]
\end{proposition}

\begin{proposition}
	Sia $B : V \to V$ un'applicazione lineare invertibile, allora vale
	\[ \Imm(L) = \Imm(L \circ B) \]
	\begin{proof}
		Dato che $B$ è una funzione invertibile, è bigettiva, ossia
		\[ B(V) = V \]. Dunque
		\[
			\Imm(L \circ B) = L(B(V)) = L(V)
		\]
	\end{proof}
\end{proposition}

\begin{definition}
	Data una applicazione lineare $L : V \to W$, dove $V$ e $W$ sono due spazi
	vettoriali di dimensione finita sul campo $\K$ e sia $[L]$ la matrice
	associata ad $L$. Se riduco in forma a scalini $[L]$, il \textbf{rango} equivale
	al numero di pivot della matrice ottenuta.
\end{definition}

\begin{theorem}
	Data una applicazione lineare $L$ come sopra e fissate le basi, vale che
	il rango di $L$ è uguale al numero di colonne non nulle che si trovano
	quando si trasforma $[L]$ in forma a scalini.
\end{theorem}

\begin{observation}
	Osserviamo che il rango di una applicazione lineare $L$ è anche uguale al
	\textbf{massimo numero di colonne linearmente indipendenti} di $[L]$.
	Infatti sappiamo che $\Imm(L)$ è il sottospazio vettoriale di $W$ generato
	dai vettori colonna di $[L]$. Da questi vettori è possibile estrarre una
	base di $\Imm(L)$ e, inoltre, possiamo dire che \[\dim(\Imm(L)) = rango (L)\]
\end{observation}

\begin{observation}
	Possiamo ora definire un algoritmo in 3 passi che, data una
	applicazione lineare tra due spazi vettoriali $V$ e $W$, e fissata una base
	$\{e_1, e_2, \dots, e_n\}$ di $V$ e una base
	$\{\epsilon_1, \epsilon_2, \dots, \epsilon_m\}$ di $W$, ci permette di
	determinare la dimensione ed una base di $\Imm(L)$:
	\begin{enumerate}
		\item Scrivere la matrice $[L]$ associata ad $L$ rispetto alle basi
		      fissate
		\item Ridurre $[L]$ in forma a scalini
		\item Contare il numero di pivot per ottenere la dimensione di $\Imm(L)$.
	\end{enumerate}
\end{observation}

\section{Riduzione a scalini per righe}
Data una matrice in $\Mat_{m \times n}(\K)$ è possibile definire le
operazioni elementari di riga in modo analogo alle mosse di colonna in questo
modo:
\begin{enumerate}
	\item Sommare alla riga $i$ la riga $j$ moltiplicata per uno scalare $\lambda$
	\item Moltiplicare la riga $i$ per uno scalare $\lambda$
	\item Permutare fra loro due righe, diciamo $i$ e $j$
\end{enumerate}
A questo punto possiamo definire la forma a scalini per righe di una matrice.
In questo caso chiameremo \emph{profondità} di una riga la posizione occupata,
contata da destra, dal suo coefficiente diverso da zero che sta più a sinistra
nella riga. La riga nulla ha \emph{profondità} 0.

\begin{definition}
	Una matrice $A$ in $\Mat_{m \times n}(\K)$, si dice \textbf{in forma
		a scalini per righe} se:
	\begin{itemize}
		\item Leggendo la matrice dall'alto verso il basso, le righe non nulle si
		      incontrano tutte prima delle righe nulle.
		\item Leggendo la matrice dall'alto verso il basso, le profondità
		      delle sue righe non nulle risultano strettamente decrescenti.
	\end{itemize}
\end{definition}

\begin{theorem}
	Data una matrice $A$ in $\Mat_{m \times n}(\K)$ è sempre possibile,
	usando un numero finito di operazioni elementari sulle righe, ridurre la
	matrice in forma a scalini per righe.
\end{theorem}

In particolare, anche quando abbiamo una matrice in forma a scalini per righe, si
possono definire i \textbf{pivot} della matrice, come i coefficienti più a
sinistra delle righe non nulle.

Inoltre anche in questo caso è possibile definire una forma a scalini
\emph{particolare}: la forma a \textbf{scalini per righe ridotta}.

\begin{example}
	matrice in forma a scalini per righe
	\[
		\begin{pmatrix}
			1 & 0 & 0 & -20 \\
			0 & 1 & 0 & 0   \\
			0 & 0 & 1 & 5
		\end{pmatrix}
	\]
\end{example}

\begin{corollary}
	Ogni matrice $A$ può essere trasformata, attraverso le operazioni elementari
	sulle righe, in una matrice in forma a scalini per righe ridotta.
\end{corollary}

In generale valgono tutte le proprietà già elencate per le operazioni sulle
colonne.

\section{Riduzione a scalini e applicazioni lineari}

\begin{theorem}
	Data una applicazione lineare $L : V \to W$ tra due spazi vettoriali $V$ e
	$W$ sul campo $\K$, e data la matrice $[L]$ associata a $L$ rispetto
	a due basi fissate di $V$ e $W$:
	\begin{enumerate}
		\item Il massimo numero di righe linearmente indipendenti di $[L]$ è
		      uguale al massimo numero di colonne linearmente indipendenti di
		      $[L]$, ossia al rango di $L$.
		\item Se si riduce la matrice $[L]$ a scalini, sia che lo si faccia
		      per righe, sia che lo si faccia per colonne, il numero di scalini
		      sarà sempre uguale al rango di $L$.
	\end{enumerate}
\end{theorem}

\begin{observation}
	Se componiamo $L$ a destra o a sinistra per una	applicazione invertibile,
	il rango non cambia. Dunque, se moltiplichiamo $[L]$ a destra o a sinistra
	per matrici invertibili, anche il rango della matrice prodotto non cambia.
\end{observation}

\begin{proposition}
	Se $A$ è una matrice $m \times n$ a valori su un campo $\K$, ed
	indichiamo con $e_1, \dots, e_n$ le sue colonne, e $B$ è una riduzione
	a scalini per righe di $A$, allora le colonne di $A$ in corrispondenza alla
	posizione dei pivot di $B$ formano una base dello $Span$ delle colonne di $A$
	($Span(e_1, \dots, e_n)$).
\end{proposition}

\begin{observation}
	A partire dalla proposizione precedente possiamo ricavare un algoritmo per
	estrarre una base di uno spazio vettoriale da un insieme di generatori.

	Consideriamo $V$ uno spazio vettoriale di dimensione $n$ e del quale
	conosciamo una base $\{e_1, \dots, e_n\}$. Consideriamo $k$ vettori
	$v_1, \dots, v_k$ di $V$ e il sottospazio di $V$ generato da
	$Span(v_1, \dots, v_k)$. Se vogliamo estrarre una base di
	$Span(v_1, \dots, v_k)$ da $v_1, \dots, v_k$ seguiamo questi passaggi:
	\begin{enumerate}
		\item Scriviamo le coordinate dei $v_i$ rispetto alla base
		      $\{e_1, \dots, e_n\}$ e le	consideriamo come colonne di
		      una matrice $M$.
		\item Porto $M$ in forma a scalini per riga.
		\item I vari vettori $v_i$ con indice $i$ in corrispondenza dei pivot di $M'$
		      ($M$ in forma a scalini) formano una base di $Span(v_1, \dots, v_k)$,
		      estratta dall'insieme di generatori $\{v_1, \dots, v_k\}$.
	\end{enumerate}
\end{observation}

\chapter{Sistemi lineari}
\section{Risoluzione di sistemi tramite riduzione per righe}
Per risolvere sistemi lineari non omogenei tornerà molto utile la riduzione
a scalini per righe di una matrice. Quello che vedremo sarà il metodo noto come
\emph{metodo di	eliminazione di Gauss}.

\begin{example}
	Prendiamo il sistema
	\[
		\begin{cases}
			x + 2y + 2z + 2t = 1 \\
			x +5y + 6z -2t = 9   \\
			8x - y -2z -2t = 0   \\
			2y + 6z + 8t = 3
		\end{cases}
	\]

	Tutte le informazioni del sistema sono contenute nella seguente
	\emph{matrice completa associata al sistema}:
	\[
		M = \begin{pmatrix}
			1 & 2  & 2  & 2  & 1  \\
			1 & 5  & 6  & -2 & -5 \\
			8 & -1 & -2 & -2 & 0  \\
			0 & 2  & 6  & 8  & 3
		\end{pmatrix}
	\]
	Ogni riga contiene i coefficienti di una delle equazioni.
	Sia $S \subset \R^4$ l'insieme delle soluzioni del sistema, ovvero
	il sottoinsieme di $\R^4$ costituito dai vettori
	$\begin{psmallmatrix} a \\ b \\ c \\ d \end{psmallmatrix}$ tali che, se poniamo
	\begin{gather*}
		x = a \\
		y = b \\
		z = c \\
		t = d
	\end{gather*}
	tutte le equazioni del sistema diventano delle uguaglianze vere.
\end{example}


\begin{theorem}
	L'insieme delle soluzioni di un sistema di equazioni lineari associato alla
	matrice $M$ coincide con l'insieme delle soluzioni di un sistema associato
	alla matrice $M'$ ottenuta riducendo $M$ in forma a scalini per righe.
\end{theorem}

\begin{observation}
	In concreto questo significa che, nel risolvere il sistema, ogni scalino
	lungo lascerà "libere" alcune variabili, come vediamo nel seguente esempio.
	Supponiamo che un certo sistema omogeneo a coefficienti in $\R$
	conduca alla matrice a scalini:
	\[
		M' = \begin{pmatrix}
			1 & 0 & 2        & 2  & 0 \\
			0 & 1 & \sqrt{3} & 12 & 0 \\
			0 & 0 & 0        & 6  & 0 \\
			0 & 0 & 0        & 0  & 0 \\
			0 & 0 & 0        & 0  & 0
		\end{pmatrix}
	\]
	Allora il sistema finale associato è
	\[
		\begin{cases}
			x + 2z + 2t         & = 0 \\
			y + \sqrt{3}z + 12t & = 0 \\
			6t                  & = 0
		\end{cases}
	\]
	Se facciamo qualche sostituzione otteniamo:
	\[
		\begin{cases}
			x & = -2z         \\
			y & = -\sqrt{3} z \\
			t & = 0
		\end{cases}
	\]
	La variabile $z$ resta "libera" e l'insieme delle soluzioni è il seguente
	sottospazio di $\R^4$:
	\[
		S = \{ (-2z, -\sqrt{3}z, z, 0 ) \mid z \in \R \}
	\]
\end{observation}

Lo stesso procedimento vale per sistemi lineari non omogenei.
Quello che dobbiamo fare è sostanzialmente considerare la matrice $M$ associata
al sistema portarla in forma a scalini e risolvere il nuovo sistema associato.

\begin{example}
	Consideriamo il sistema
	\[
		\begin{cases}
			x + 2y + z & = 1 \\
			x - y      & = 3 \\
			y + z      & = 2
		\end{cases}
	\]
	Scriviamo la matrice associata.
	\[
		\begin{pmatrix}
			1 & 2  & 1 & 1 \\
			1 & -1 & 0 & 3 \\
			0 & 1  & 1 & 2
		\end{pmatrix}
	\]
	Se ridotta a scalini per righe otteniamo
	\[
		\begin{pmatrix}
			1 & 2  & 1  & 1 \\
			0 & -3 & -1 & 2 \\
			0 & 0  & 2  & 8
		\end{pmatrix}
	\]
	E dunque otteniamo il nuovo sistema
	\[
		\begin{cases}
			x + 2y + z & = 1 \\
			-3y - z    & = 2 \\
			2z         & = 8
		\end{cases}
		\quad \Rightarrow \quad
		\begin{cases}
			x & = 1  \\
			y & = -2 \\
			z & = 4
		\end{cases}
	\]
\end{example}


\chapter{La formula di Grassmann}
\section{La formula di Grassmann}

Dati due sottospazi vettoriali $A$ e $B$ in $\R^3$ di dimensione 2,
di che dimensione può essere la loro intersezione ?

Possono intersecarsi lungo una retta: in tal caso si nota che il sottospazio
generato dai vettori di $A \cup B$, ossia $A + B$, è tutto $\R^3$.

Oppure vale $A = B$: allora la loro intersezione è uguale ad $A$ (e a $B$) e
ha dimensione 2, e anche il sottospazio $A + B$ coincide con $A$.

In entrambi i casi, la somma delle dimensioni di $A \cap B$ e di $A + B$ è
sempre uguale a 4.

E se in $\R^4$ consideriamo un piano $C$ e un sottospazio $D$ di
dimensione 3?
Possono darsi tre casi per l'intersezione: $C \cap D = \{O\}$,
$\dim(C \cap D) = 1$, $C \cap D = C$.

Qualunque sia il caso si verifica sempre che
\[
	\dim(C \cap D) + \dim(C + D) = 5 = \dim(C) + \dim(D)
\]

In generale vale la formula
\[
	\dim(A \cap B) + \dim(A + B) = \dim(A) + \dim(B)
\]

Dati due spazi vettoriali $V$ e $W$ sul campo $\K$, sul loro prodotto
cartesiano $V \times W$, c'è una struttura naturale di spazio vettoriale, dove
la somma è definita da:
\[
	(v, w) + (v_1, w_1) = (v + v_1, w + w_1)
\]
e il prodotto per scalare da:
\[
	\lambda(v, w) = (\lambda v, \lambda w)
\]
Si verifica che, se $\{v_1, \dots, v_n\}$ è una base di $V$
e $\{w_1, \dots, w_m\}$ è una base di $W$, allora
$\{(v_1, O), \dots (v_n, O), (O, w_1), \dots, (O, w_m)\}$ è una base di
$V \times W$, che dunque ha dimensione $n + m = (\dim(V)) + (\dim(W))$.

\begin{theorem}[Grassmann]
	Dati due sottospazi $A, B$ di uno spazio vettoriale $V$ sul campo
	$\K$, vale
	\[
		\dim(A) + \dim(B) = \dim(A \cap B) + \dim(A + B)
	\]
	\begin{proof}
		Consideriamo l'applicazione
		\[
			\Phi : A \times B \to V
		\]
		definita da
		\[ \Phi((a, b)) = a - b \]
		Cosa sappiamo dire del nucleo di $\Phi$ ? Per definizione
		\[
			\Ker(\Phi) = \{(a, b) \in A \times B \mid a - b = O\}
		\]
		dunque
		\[
			\Ker(\Phi) = \{(a, b) \in A \times B \mid a = b\}
		\]
		che equivale a scrivere:
		\[
			\Ker(\Phi) = \{(z, z) \in A \times B \mid z \in A \cap B\}
		\]
		Si nota subito che la applicazione lineare
		\[ \theta : A \cap B \to \Ker(\Phi) \]
		è iniettiva e surgettiva, dunque è un isomorfismo. Allora il suo dominio e
		il suo codominio hanno la stessa dimensione, ovvero
		\[
			\dim(\Ker(\Phi)) = \dim(A \cap B)
		\]
		Cosa sappiamo dire invece dell'immagine di $\Phi$ ? Per definizione
		\[
			\Imm(\Phi) = \{a - b \mid a \in A, b \in B\}
		\]
		Visto che $B$, come ogni spazio vettoriale, se contiene un elemento
		$b$ contiene anche il suo opposto $-b$, possiamo scrivere la seguente
		uguaglianza fra insiemi:
		\[
			\{ a - b \mid a \in A, b \in B \} =
			\{ a + b \in V \mid a \in A, b \in B \} =
			A + B
		\]
		Dunque
		\[
			\Imm(\Phi) = A + B
		\]
		Sappiamo che:
		\[
			\dim(A \times B) = \dim(\Ker(\Phi)) + \dim(\Imm(\Phi))
		\]
		Questa formula, viste le osservazioni fatte fin qui, si traduce come:
		\[
			\dim(A) + \dim(B) = \dim(A \cap B) + \dim(A + B)
		\]
	\end{proof}
\end{theorem}

\section{Calcolo dell'intersezione di due sottospazi}
Consideriamo due sottospazi, $U$ e $W$, di $V$. Se entrambi sono presentati
come l'insieme delle soluzioni di un sistema è facile calcolare $U \cap W$:
basta calcolare le soluzioni del sistema 'doppio', ottenuto considerando tutte
le equazioni dei due sistemi.

Per esempio se $U$ e $W$ in $\R^4$ sono dati rispettivamente dalle
soluzioni dei sistemi $S_U$:
\[
	\begin{cases}
		3x + 2y + 4w = 0 \\
		2x + y + z + w = 0
	\end{cases}
\]
e $S_W$:
\[
	\begin{cases}
		x + 2y + z + w = 0 \\
		x + z + w = 0
	\end{cases}
\]
allora $U \cap W$ è dato dalle soluzioni del sistema:
\[
	\begin{cases}
		3x + 2y + 4w = 0   \\
		2x + y + z + w = 0 \\
		x + 2y + z + w = 0 \\
		x + z + w = 0
	\end{cases}
\]

\begin{observation}
	Visto che $U$ ha dimensione 2, un sistema le cui soluzioni coincidono con
	l'insieme $U$ deve avere almeno 3 equazioni.
\end{observation}

Come calcolare però $U \cap W$ se i due sottospazi sono presentati come span
di certi vettori ? Consideriamo per esempio $U$ e $W$ in $\R^5$
definiti così:
\begin{gather*}
	U = <\begin{pmatrix}
		1 \\ 2 \\ 3 \\ -1 \\ 2
	\end{pmatrix},
	\begin{pmatrix}
		2 \\ 4 \\ 7 \\ 2 \\ -1
	\end{pmatrix}> \\
	W = <\begin{pmatrix}
		1 \\ 2 \\ 0 \\ -2 \\ -1
	\end{pmatrix},
	\begin{pmatrix}
		0 \\ 1 \\ 1 \\ -1 \\ -1
	\end{pmatrix},
	\begin{pmatrix}
		0 \\ 1 \\ -3 \\ -6 \\ 1
	\end{pmatrix}>
\end{gather*}

Un metodo per calcolare $U \cap W$ è quello di esprimere $U$ e $W$ come
soluzioni di un sistema lineare. Cominciamo da $U$.

Per prima cosa si scrive la matrice:
\[
	\begin{pmatrix}
		1  & 2  & x_1 \\
		2  & 4  & x_2 \\
		3  & 7  & x_3 \\
		-1 & 2  & x_4 \\
		2  & -1 & x_5
	\end{pmatrix}
\]
Ora riduciamo la matrice (senza incognite) a scalini per righe
\[
	\begin{pmatrix}
		1 & 2 & x_1                 \\
		0 & 1 & x_3 - 3x_1          \\
		0 & 0 & 2x_1 - x_2          \\
		0 & 0 & 13x_1 - 4x_3 + x_4  \\
		0 & 0 & -17x_1 + 5x_3 + x_5
	\end{pmatrix}
\]
Tale matrice ha rango 2 se e solo se i coefficienti $x_1, x_2, x_3, x_4, x_5$ soddisfano
il sistema
\[
	\begin{cases}
		2x_1 - x_2          & = 0 \\
		13x_1 - 4x_3 + x_4  & = 0 \\
		-17x_1 + 5x_3 + x_5 & = 0
	\end{cases}
\]

Ora dobbiamo fare la stessa cosa con $W$. Scriviamo quindi la matrice
\[
	\begin{pmatrix}
		1  & 0  & 0  & x_1 \\
		2  & 1  & 1  & x_2 \\
		0  & 1  & -3 & x_3 \\
		-2 & -1 & -6 & x_4 \\
		-1 & -1 & 1  & x_5
	\end{pmatrix}
\]
e riduciamola a scalini per righe
\[
	\begin{pmatrix}
		1 & 0 & 0 & x_1                       \\
		0 & 1 & 1 & x_2 - 2x_1                \\
		0 & 0 & 2 & x_5 - x_1 + x_2           \\
		0 & 0 & 0 & x_3 + x_2 + 2x_5          \\
		0 & 0 & 0 & 2x_4 + 7x_2 + 5x_5 - 5x_1
	\end{pmatrix}
\]
La matrice ha rango 3 se e solo se i coefficienti $x_1, x_2, x_3, x_4, x_5$ soddisfano il
sistema
\[
	\begin{cases}
		x_3 + x_2 + 2x_5          & = 0 \\
		2x_4 + 7x_2 + 5x_5 - 5x_1 & = 0 \\
	\end{cases}
\]

Uniamo i due sistemi ottenuti e otteniamo
\[
	\begin{cases}
		2x_1 - x_2                & = 0 \\
		13x_1 - 4x_3 + x_4        & = 0 \\
		17x_1 - 5x_3 - x_5        & = 0 \\
		x_2 + x_3 + 2x_5          & = 0 \\
		5x_1 - 7x_2 - 2x_4 - 5x_5 & = 0
	\end{cases}
\]
Se risolviamo questo sistema otteniamo una base di $U \cap W$. Per verificare che i calcoli
siano corretti basta vedere se la dimensione risulta uguale a quella prevista dalla formula
di Grassmann.

\section{Somma diretta di sottospazi}
Si dice che due sottospazi $U$ e $W$ di uno spazio vettoriale $V$ sono in
\textbf{somma diretta} se vale che \[ U \cap W = \{O\} \] In questo caso,
come sappiamo dalla formula di Grassmann, la dimensione di $U + W$ è "la
massima possibile", ovvero \[ \dim(U) + \dim(W) \] Vale anche il
viceversa, ossia due sottospazi sono in somma diretta se e solo se
\[ \dim(U + W) = \dim(U) + \dim(W) \] Quando siamo sicuri che $U + W$ è la somma di due
sottospazi che sono in somma diretta, al posto di $U + W$ possiamo scrivere:
\[
	U \oplus W
\]
In particolare, per avere una base di $U \oplus W$ basta fare l'unione di una
base di $U$ con una base di $W$.

\begin{observation}
	Attenzione: un sottospazio vettoriale $U$ di $V$ che non è uguale a $V$
	possiede in generale molti complementari. Per esempio, in $\R^3$
	un piano passante per l'origine ha per complementare una qualunque retta
	passante per l'origine e che non giace sul piano.
\end{observation}

In generale dati $k$ sottospazi $U_1, \dots, U_k$ di uno spazio vettoriale $V$,
si dice che tali sottospazi sono in somma diretta se, per ogni
$i = 1, \dots, k$, vale che l'intersezione di $U_i$ con la somma di tutti
gli altri è uguale a $\{O\}$, ovvero
\[
	U_i \cap (U_1 + \cdots + \hat{U_i} + \cdots + U_k) = \{O\}
\]
dove il simbolo $\hat{U_i}$ indica che nella somma si è saltato il termine
$U_i$.

In tal caso per indicare $U_1 + \cdots + U_k$ si può usare la notazione:
\[
	U_1 \oplus \cdots \oplus U_k
\]
\section{Calcolo dell'intersezione di due sottospazi}
Consideriamo due sottospazi, $U$ e $W$, di $V$. Se entrambi sono presentati
come l'insieme delle soluzioni di un sistema è facile calcolare $U \cap W$:
basta calcolare le soluzioni del sistema 'doppio', ottenuto considerando tutte
le equazioni dei due sistemi.

Per esempio se $U$ e $W$ in $\R^4$ sono dati rispettivamente dalle
soluzioni dei sistemi $S_U$:
\[
	\begin{cases}
		3x + 2y + 4w = 0 \\
		2x + y + z + w = 0
	\end{cases}
\]
e $S_W$:
\[
	\begin{cases}
		x + 2y + z + w = 0 \\
		x + z + w = 0
	\end{cases}
\]
allora $U \cap W$ è dato dalle soluzioni del sistema:
\[
	\begin{cases}
		3x + 2y + 4w = 0   \\
		2x + y + z + w = 0 \\
		x + 2y + z + w = 0 \\
		x + z + w = 0
	\end{cases}
\]

\begin{observation}
	Visto che $U$ ha dimensione 2, un sistema le cui soluzioni coincidono con
	l'insieme $U$ deve avere almeno 3 equazioni.
\end{observation}

Come calcolare però $U \cap W$ se i due sottospazi sono presentati come span
di certi vettori ? Consideriamo per esempio $U$ e $W$ in $\R^5$
definiti così:
\begin{gather*}
	U = <\begin{pmatrix}
		1 \\ 2 \\ 3 \\ -1 \\ 2
	\end{pmatrix},
	\begin{pmatrix}
		2 \\ 4 \\ 7 \\ 2 \\ -1
	\end{pmatrix}> \\
	W = <\begin{pmatrix}
		1 \\ 2 \\ 0 \\ -2 \\ -1
	\end{pmatrix},
	\begin{pmatrix}
		0 \\ 1 \\ 1 \\ -1 \\ -1
	\end{pmatrix},
	\begin{pmatrix}
		0 \\ 1 \\ -3 \\ -6 \\ 1
	\end{pmatrix}>
\end{gather*}

Un metodo per calcolare $U \cap W$ è quello di esprimere $U$ e $W$ come
soluzioni di un sistema lineare. Cominciamo da $U$.

Per prima cosa si scrive la matrice:
\[
	\begin{pmatrix}
		1  & 2  & x_1 \\
		2  & 4  & x_2 \\
		3  & 7  & x_3 \\
		-1 & 2  & x_4 \\
		2  & -1 & x_5
	\end{pmatrix}
\]
Ora riduciamo la matrice (senza incognite) a scalini per righe
\[
	\begin{pmatrix}
		1 & 2 & x_1                 \\
		0 & 1 & x_3 - 3x_1          \\
		0 & 0 & 2x_1 - x_2          \\
		0 & 0 & 13x_1 - 4x_3 + x_4  \\
		0 & 0 & -17x_1 + 5x_3 + x_5
	\end{pmatrix}
\]
Tale matrice ha rango 2 se e solo se i coefficienti $x_1, x_2, x_3, x_4, x_5$ soddisfano
il sistema
\[
	\begin{cases}
		2x_1 - x_2          & = 0 \\
		13x_1 - 4x_3 + x_4  & = 0 \\
		-17x_1 + 5x_3 + x_5 & = 0
	\end{cases}
\]

Ora dobbiamo fare la stessa cosa con $W$. Scriviamo quindi la matrice
\[
	\begin{pmatrix}
		1  & 0  & 0  & x_1 \\
		2  & 1  & 1  & x_2 \\
		0  & 1  & -3 & x_3 \\
		-2 & -1 & -6 & x_4 \\
		-1 & -1 & 1  & x_5
	\end{pmatrix}
\]
e riduciamola a scalini per righe
\[
	\begin{pmatrix}
		1 & 0 & 0 & x_1                       \\
		0 & 1 & 1 & x_2 - 2x_1                \\
		0 & 0 & 2 & x_5 - x_1 + x_2           \\
		0 & 0 & 0 & x_3 + x_2 + 2x_5          \\
		0 & 0 & 0 & 2x_4 + 7x_2 + 5x_5 - 5x_1
	\end{pmatrix}
\]
La matrice ha rango 3 se e solo se i coefficienti $x_1, x_2, x_3, x_4, x_5$ soddisfano il
sistema
\[
	\begin{cases}
		x_3 + x_2 + 2x_5          & = 0 \\
		2x_4 + 7x_2 + 5x_5 - 5x_1 & = 0 \\
	\end{cases}
\]

Uniamo i due sistemi ottenuti e otteniamo
\[
	\begin{cases}
		2x_1 - x_2                & = 0 \\
		13x_1 - 4x_3 + x_4        & = 0 \\
		17x_1 - 5x_3 - x_5        & = 0 \\
		x_2 + x_3 + 2x_5          & = 0 \\
		5x_1 - 7x_2 - 2x_4 - 5x_5 & = 0
	\end{cases}
\]
Se risolviamo questo sistema otteniamo una base di $U \cap W$. Per verificare che i calcoli
siano corretti basta vedere se la dimensione risulta uguale a quella prevista dalla formula
di Grassmann.



\subsection{Somma diretta di sottospazi}
Si dice che due sottospazi $U$ e $W$ di uno spazio vettoriale $V$ sono in
\textbf{somma diretta} se vale che \[ U \cap W = \{O\} \] In questo caso,
come sappiamo dalla formula di Grassmann, la dimensione di $U + W$ \`e "la
massima possibile", ovvero \[ dim(U) + dim(W) \] Vale anche il
viceversa, ossia due sottospazi sono in somma diretta se e solo se
\[ dim(U + W) = dim(U) + dim(W) \] Quando siamo sicuri che $U + W$ \`e la somma di due
sottospazi che sono in somma diretta, al posto di $U + W$ possiamo scrivere:
\begin{equation*}
	U \oplus W
\end{equation*}
In particolare, per avere una base di $U \oplus W$ basta fare l'unione di una
base di $U$ con una base di $W$.

\begin{observation}
	Attenzione: un sottospazio vettoriale $U$ di $V$ che non \`e uguale a $V$
	possiede in generale molti complementari. Per esempio, in $\mathbb{R}^3$
	un piano passante per l'origine ha per complementare una qualunque retta
	passante per l'origine e che non giace sul piano.
\end{observation}

In generale dati $k$ sottospazi $U_1, \dots, U_k$ di uno spazio vettoriale $V$,
si dice che tali sottospazi sono in somma diretta se, per ogni
$i = 1, \dots, k$, vale che l'intersezione di $U_i$ con la somma di tutti
gli altri \`e uguale a $\{O\}$, ovvero
\begin{equation*}
	U_i \cap (U_1 + \cdots + \hat{U_i} + \cdots + U_k) = \{O\}
\end{equation*}
dove il simbolo $\hat{U_i}$ indica che nella somma si \`e saltato il termine
$U_i$.

In tal caso per indicare $U_1 + \cdots + U_k$ si pu\`o usare la notazione:
\begin{equation*}
	U_1 \oplus \cdots \oplus U_k
\end{equation*}

\chapter{Applicazioni lineari e matrici invertibili}
\section{Endomorfismi lineari invertibili}

\begin{defn}
	Consideriamo uno spazio vettoriale $V$ di dimensione $n$ sul campo $\mathbb{K}$ e
	una applicazione lineare $L : V \to V$. Una tale applicazione lineare si dice
	\textbf{endomorfismo lineare di $V$}. Indicheremo con $End(V)$ l'insieme di tutti gli
	endomorifsimi lineari di $V$.
\end{defn}

\begin{proposition}
	Un endomorfismo $L$ di $V$ \`e invertibile se e solo se ha rango $n$. La
	funzione inversa $L^{-1} : V \to V$ \`e anch'essa un'applicazione lineare.
\end{proposition}

\begin{observation}
	Dati due spazi $V$ e $W$ entrambi di dimensione $n$ e una applicazione lineare
	$L : V \to W$, l'applicazione lineare $L$ \`e invertibile se e solo se ha rango
	$n$; l'applicazione inversa $L^{-1}$ \`e anch'essa lineare.
	Abbiamo invece gi\`a osservato che se $V$ e $W$ hanno dimensioni diverse,
	rispettivamente $m$ e $n$, nessuna applicazione lineare $L$ da $V$ a $W$ pu\`o
	essere invertibile. Infatti, avendo in mente la relazione che lega la dimensione
	del nucleo di $L$, dell'immagine di $L$ e di $V$
	($dim(Imm(L)) + dim(Ker(L)) = dim(V)$), si ha che:
	\begin{itemize}
		\item se $m > n$ allora la dimensione di $Imm(L)$ \`e al massimo $n$.
		      Quindi la dimensione di $Ker(L)$ \`e almeno $m - n$, ovvero
		      maggiore di 0, e $L$ non \`e iniettiva.
		\item se $m < n$ allora la dimensione di $Imm(L)$ \`e al massimo $m$.
		      Quindi la dimensione di $Imm(L)$ \`e minore della dimensione di
		      $W$, e $L$ non \`e surgettiva.
	\end{itemize}
\end{observation}

Se fissiamo una base di $V$, ad ogni endomorfismo $L \in End(V)$ viene associata
una matrice $[L] \in Mat_{n \times n}(\mathbb{K})$. Se $L$ \`e invertibile,
consideriamo l'inversa $L^{-1}$ e la matrice ad essa associata $[L^{-1}]$.
Visto che $L \circ L^{-1} = L^{-1} \circ L = I$, vale
\begin{equation*}
	[L^{-1}][L] = [L][L^{-1}] = [I] = I
\end{equation*}

Dunque la matrice $[L]$ \`e invertibile e ha per inversa $[L^{-1}]$.
Possiamo affermare anche il viceversa: se la matrice $[L]$ associata ad un
endomorfismo lineare \`e invertibile allora anche $L$ \`e invertibile e la sua
inversa \`e l'applicazione associata alla matrice $[L^{-1}]$.

\begin{corollary}
	Una matrice $A \in Mat_{n \times n}(\mathbb{K})$ \`e invertibile se e solo se
	il suo rango \`e $n$.
\end{corollary}

\section{Metodo per trovare l'inversa di una matrice}
Come abbiamo visto nel paragrafo precedente, il problema di trovare un'inversa
di $L \in End(V)$ si pu\`o tradurre nel problema di trovare l'inversa in
$Mat_{n \times n}(\mathbb{K})$ di una matrice data. In questo paragrafo descriviamo
un metodo per trovare l'inversa di una matrice $A \in Mat_{n \times n}(\mathbb{K})$.

\begin{defn}
	Una matrice in forma a scalini per righe (o per colonne) ridotta, si dice
	\textbf{normalizzata} se ha tutti i pivot uguali a 1.
\end{defn}

\begin{observation}
	Una matrice pu\`o essere portata in forma a scalini per righe (o colonne)
	ridotta e normalizzata, attraverso un numero finito di operazioni elementari.
\end{observation}

\begin{example}
	Consideriamo la matrice
	\[
		A = \begin{pmatrix}
			3 & 2 & 1 \\
			0 & 1 & 1 \\
			1 & 1 & 0
		\end{pmatrix}
	\]
	che ha rango 3, dunque \`e invertibile, e calcoliamo la
	sua inversa. Per prima cosa formiamo la matrice
	\begin{equation*}
		(A I) = \begin{pmatrix}
			3 & 2 & 1 & 1 & 0 & 0 \\
			0 & 1 & 1 & 0 & 1 & 0 \\
			1 & 1 & 0 & 0 & 0 & 1
		\end{pmatrix}
	\end{equation*}
	Ora con delle operazioni elementari di riga portiamola in forma a scalini
	per righe ridotta, per esempio nel seguente modo: si sottrae alla prima riga
	la terza moltiplicata per 3
	\begin{equation*}
		\begin{pmatrix}
			3 & 2 & 1 & 1 & 0 & 0 \\
			0 & 1 & 1 & 0 & 1 & 0 \\
			1 & 1 & 0 & 0 & 0 & 1
		\end{pmatrix} \to
		\begin{pmatrix}
			0 & -1 & 1 & 1 & 0 & -3 \\
			0 & 1  & 1 & 0 & 1 & 0  \\
			1 & 1  & 0 & 0 & 0 & 1
		\end{pmatrix}
	\end{equation*}
	poi si somma alla prima riga la seconda
	\begin{equation*}
		\begin{pmatrix}
			0 & -1 & 1 & 1 & 0 & -3 \\
			0 & 1  & 1 & 0 & 1 & 0  \\
			1 & 1  & 0 & 0 & 0 & 1
		\end{pmatrix} \to
		\begin{pmatrix}
			0 & 0 & 2 & 1 & 1 & -3 \\
			0 & 1 & 1 & 0 & 1 & 0  \\
			1 & 1 & 0 & 0 & 0 & 1
		\end{pmatrix}
	\end{equation*}
	a questo punto si permutano le righe e si ottiene
	\begin{equation*}
		\begin{pmatrix}
			1 & 1 & 0 & 0 & 0 & 1  \\
			0 & 1 & 1 & 0 & 1 & 0  \\
			0 & 0 & 2 & 1 & 1 & -3
		\end{pmatrix}
	\end{equation*}
	Per ottenere la forma a scalini ridotta, moltiplichiamo l'ultima riga per
	$\frac{1}{2}$
	\begin{equation*}
		\begin{pmatrix}
			1 & 1 & 0 & 0           & 0           & 1            \\
			0 & 1 & 1 & 0           & 1           & 0            \\
			0 & 0 & 1 & \frac{1}{2} & \frac{1}{2} & -\frac{3}{2}
		\end{pmatrix}
	\end{equation*}
	sottraiamo alla seconda riga la terza riga
	\begin{equation*}
		\begin{pmatrix}
			1 & 1 & 0 & 0           & 0           & 1            \\
			0 & 1 & 0 & \frac{1}{2} & \frac{1}{2} & \frac{3}{2}  \\
			0 & 0 & 1 & \frac{1}{2} & \frac{1}{2} & -\frac{3}{2}
		\end{pmatrix}
	\end{equation*}
	infine sottriamo alla prima riga la seconda:
	\begin{equation*}
		\begin{pmatrix}
			1 & 0 & 0 & \frac{1}{2}  & -\frac{1}{2} & -\frac{1}{2} \\
			0 & 1 & 0 & -\frac{1}{2} & \frac{1}{2}  & \frac{3}{2}  \\
			0 & 0 & 1 & \frac{1}{2}  & \frac{1}{2}  & -\frac{3}{2}
		\end{pmatrix}
	\end{equation*}
	La matrice
	\begin{equation*}
		B = \begin{pmatrix}
			\frac{1}{2}  & -\frac{1}{2} & -\frac{1}{2} \\
			-\frac{1}{2} & \frac{1}{2}  & \frac{3}{2}  \\
			\frac{1}{2}  & \frac{1}{2}  & -\frac{3}{2}
		\end{pmatrix}
	\end{equation*}
	\`e l'inversa di $A$.
\end{example}

\textbf{Perch\`e il metodo funziona ?}

Consideriamo una matrice $A \in Mat_{n \times n}(\mathbb{K})$ e cerchiamo la sua
inversa; supponiamo che $A$ abbia rango $n$.

Per prima cosa creiamo una matrice $n \times 2n$ ponendo accanto le colonne di $A$
e quelle di $I$. Indicheremo tale matrice con $(A I)$.

Adesso possiamo agire con operazioni elementari di riga in modo da ridurre la
matrice in forma a scalini per righe ridotta. Poich\`e $A$ ha rango $n$, anche
$(A I)$ ha rango $n$. Un modo per rendersene conto \`e il seguente: il rango di
$(A I)$ \`e minore o uguale a $n$ visto che ha $n$ righe ed \`e maggiore o uguale
a $n$ visto che individuiamo facilmente $n$ colonne linearmente indipendenti.

Allora quando la matrice $(A I)$ viene ridotta in forma a scalini per righe ridotta,
deve avere esattamente $n$ scalini, dunque deve avere la forma $(I B)$.
Affermiamo che la matrice $B$ che si ricava dalla matrice precedente \`e proprio
l'inversa di $A$ che cercavamo.

Infatti agire con operazioni di riga equivale a moltiplicare a sinistra la matrice
$(A I)$ per una matrice invertibile $U$ di formato $n \times n$, dunque:
\begin{equation*}
	U (A I) = (I B)
\end{equation*}
Per come \`e definito il prodotto righe per colonne,
\begin{equation*}
	U (A I) = (U A U I)
\end{equation*}
Dalle uguaglianze precedenti ricaviamo
\begin{equation*}
	(U A U I) = (I B)
\end{equation*}
ossia le relazioni $U A = I$ e $U I = B$ che ci dicono che $U$ \`e l'inversa di
$A$ e che $U = B$ come avevamo annunciato.

\begin{observation}
	La relazione $U A = I$, ossia $BA = I$, ci dice solo che $B$ \`e l'inversa
	sinistra di $A$. In generale sappiamo che il prodotto tra matrici non \`e
	commutativo. Dunque, il fatto che $A$ sia invertibile, ci dice che esiste una
	matrice $B$ tale che $BA = I$, e che esiste una matrice $C$ tale che $AC = I$.
	Ma possiamo mostrare facilmente che $B$ coincide con l'inversa destra $C$ di
	$A$. Come detto $C$ deve soddisfare per definizione la condizione $AC = I$.
	Se moltiplichiamo per $C$ entrambi i membri della relazione $BA = I$ (a destra)
	\[ BAC = IC \] Usando la propriet\`a associativa del prodotto in
	$Mat_{n \times n}(\mathbb{K})$ otteniamo \[B = C\] visto che $AC = I$.
\end{observation}

\section{Cambiamento di base negli endomorfismi lineari}
Sia $V$ uno spazio vettoriale di dimensione $n$ sul campo $\mathbb{K}$
e sia $L \in End(V)$. Supponiamo di avere due basi di $V$, una data dai vettori
$v_1, v_2, \dots, v_n$ e l'altra dai vettori $e_1, e_2, \dots, e_n$. Quello che
vedremo sar\`a la relazione che lega le matrici associate a $L$ rispetto a tali
basi,
\begin{equation*}
	[L]_{\substack{
			v_1, v_2, \dots, v_n \\
			v_1, v_2, \dots, v_n
		}}
\end{equation*}
e
\begin{equation*}
	[L]_{\substack{
			e_1, e_2, \dots, e_n \\
			e_1, e_2, \dots, e_n
		}}
\end{equation*}
Per prima cosa scriviamo ogni vettore $v_i$ come combinazione lineare dei vettori
della base $e_1, e_2, \dots, e_n$:
\begin{gather*}
	v_1 = a_{11}e_1 + a_{21}e_2 + \cdots + a_{n1}e_n \\
	v_2 = a_{12}e_1 + a_{22}e_2 + \cdots + a_{n2}e_n \\
	\cdots                                           \\
	v_n = a_{1n}e_1 + a_{2n}e_2 + \cdots + a_{nn}e_n \\
\end{gather*}
Se proviamo ora a scrivere la matrice associata all'endomorfismo identit\`a
$I \in End(V)$ prendendo come base in partenza $v_1, v_2, \dots, v_n$ e come base
in arrivo $e_1, e_2, \dots, e_n$ \`e la seguente:
\begin{equation*}
	[I]_{\substack{
			v_1, v_2, \dots, v_n \\
			e_1, e_2, \dots, e_n
		}} = \begin{pmatrix}
		a_{11} & a_{12} & \dots & a_{1n} \\
		a_{21} & a_{22} & \dots & \dots  \\
		\dots  & \dots  & \dots & \dots  \\
		a_{n1} & \dots  & \dots & a_{nn}
	\end{pmatrix}
\end{equation*}
Infatti nella prima colonna abbiamo scritto i coefficienti di $I(v_1)$ rispetto
alla base $e_1, e_2, \dots, e_n$, nella seconda colonna i coefficienti di
$I(v_2) = v_2$ e cos\`i via.

La matrice appena trovata \`e una matrice di
\textbf{cambiamento di base} e la chiameremo $M$. Osserviamo subito che $M$ \`e
invertibile. Infatti pensiamo alla composizione di endomorfismi $I \circ I$
ovvero $V \to^{I} V \to^{I} V$ e consideriamo il primo spazio $V$ e l'ultimo muniti
della base $v_1, v_2, \dots, v_n$, mentre lo spazio $V$ al centro lo consideriamo
con la base $e_1, e_2, \dots, e_n$. A questo punto otteniamo:
\begin{equation*}
	[I \circ I]_{\substack{
		v_1, v_2, \dots, v_n \\
		v_1, v_2, \dots, v_n
	}} =
		[I]_{\substack{
				e_1, e_2, \dots, e_n\\
				v_1, v_2, \dots, v_n
			}}
		[I]_{\substack{
				v_1, v_2, \dots, v_n\\
				e_1, e_2, \dots, e_n
			}}
\end{equation*}
Visto che $I \circ I = I$ possiamo riscrivere
\begin{equation*}
	[I]_{\substack{
		v_1, v_2, \dots, v_n \\
		v_1, v_2, \dots, v_n
	}} =
		[I]_{\substack{
				e_1, e_2, \dots, e_n\\
				v_1, v_2, \dots, v_n
			}}
		[I]_{\substack{
				v_1, v_2, \dots, v_n\\
				e_1, e_2, \dots, e_n
			}}
\end{equation*}
Ora la matrice al membro di sinistra \`e la matrice identit\`a $I$, mentre quella
pi\`u a destra \`e $M$, dunque:
\begin{equation*}
	I = [I]_{\substack{
		e_1, e_2, \dots, e_n\\
		v_1, v_2, \dots, v_n
	}} M
\end{equation*}
Questo ci permette di concludere che $M$ \`e invertibile e che
\[
	M^{-1} = [I]_{\substack{
				e_1, e_2, \dots, e_n\\
				v_1, v_2, \dots, v_n
			}}
\]
A questo punto possiamo enunciare il teorema che descrive la relazione fra matrici
assocaite a $L$ rispetto alle due diverse basi:
\begin{theorem}
	Con le notazioni introdotte sopra, vale:
	\begin{equation*}
		[L]_{\substack{
				v_1, v_2, \dots, v_n \\
				v_1, v_2, \dots, v_n
			}} =
		M^{-1}[L]_{\substack{
					e_1, e_2, \dots, e_n\\
					e_1, e_2, \dots, e_n
				}}M
	\end{equation*}
\end{theorem}

Ricordiamo che il problema di trovare la matrice associata a $L$ rispetto ad una
base se si conosce la matrice associata rispetto ad un'altra base pu\`o essere
affrontato anche senza scrivere le matrici $M$ e $M^{-1}$ ma il teorema precedente
ha una grande importanza dal punto di vista teorico.

Per esempio, se definiamo l'applicazione traccia
\begin{equation*}
	\tau : Mat_{n \times n}(\mathbb{K}) \to \mathbb{K}
\end{equation*}
nel seguente modo:
\begin{equation*}
	\tau((a_{ij})) = a_{11} + a_{22} + \dots + a_{nn}
\end{equation*}
\`e naturale chiedersi se, dato un endomorfismo $L \in End(V)$, la funzione traccia
dia lo stesso valore su tutte le matrici che si possono associare a $V$, in altre
parole se vale:
\begin{equation*}
	\tau \left(
	[L]_{\substack{
			v_1, v_2, \dots, v_n \\
			v_1, v_2, \dots, v_n
		}}
	\right) =
	\tau \left(
	[L]_{\substack{
			e_1, e_2, \dots, e_n \\
			e_1, e_2, \dots, e_n
		}}
	\right)
\end{equation*}
per ogni scelta delle basi $e_1, e_2, \dots, e_n$ e $v_1, v_2, \dots, v_n$.

La risposta \`e si: la traccia non dipende dalla base scelta e dunque possiamo
anche considerarla come applicazione lineare da $End(V)$ a $\mathbb{K}$.
Per mostrarlo scriviamo:
\begin{equation*}
	\tau \left(
	[L]_{\substack{
			v_1, v_2, \dots, v_n \\
			v_1, v_2, \dots, v_n
		}}
	\right) =
	\tau \left(
	M^{-1} [L]_{\substack{
			e_1, e_2, \dots, e_n \\
			e_1, e_2, \dots, e_n
		}} M
	\right)
\end{equation*}
A questo punto ricordiamo che per ogni $A, B \in Mat_{n \times n}(\mathbb{K})$
vale $\tau(AB) = \tau(BA)$, dunque:
\begin{gather*}
	\tau \left(
	\left(
		M^{-1}[L]_{\substack{
				e_1, e_2, \dots, e_n\\
				e_1, e_2, \dots, e_n
			}}
		\right) M
	\right) =\\
	\tau \left(
	M \left(
		M^{-1}[L]_{\substack{
				e_1, e_2, \dots, e_n\\
				e_1, e_2, \dots, e_n
			}}
		\right)
	\right) =\\
	\tau \left(
	M M^{-1} [L]_{\substack{
			e_1, e_2, \dots, e_n\\
			e_1, e_2, \dots, e_n
		}}
	\right) =\\
	\tau \left(
	[L]_{\substack{
			e_1, e_2, \dots, e_n \\
			e_1, e_2, \dots, e_n
		}}
	\right)
\end{gather*}
Questa catena di uguaglianza conduce, come annunciato, a:
\begin{equation*}
	\tau \left(
	[L]_{\substack{
			v_1, v_2, \dots, v_n \\
			v_1, v_2, \dots, v_n
		}}
	\right) =
	\tau \left(
	[L]_{\substack{
			e_1, e_2, \dots, e_n \\
			e_1, e_2, \dots, e_n
		}}
	\right)
\end{equation*}

\textbf{Ricapitolando:} Sia $V$ uno spazio vettoriale, sia $L \in End(V)$ e siano
$\mathcal{B}_1$ e $\mathcal{B}_2$ due basi di $V$. Se conosciamo la matrice
$[L]_{\substack{\mathcal{B}_1 \\ \mathcal{B}_1}}$ e vogliamo scrivere la matrice
$[L]_{\substack{\mathcal{B}_2 \\ \mathcal{B}_2}}$ passando per la matrice di
cambiamento di base dobbiamo:
\begin{enumerate}
	\item Trovare la matrice di cambiamento di base. Per farlo scriviamo la
	      matrice $[I]_{\substack{\mathcal{B}_1 \\ \mathcal{B}_2}}$ ovvero la
	      matrice associata all'indentit\`a con $\mathcal{B}_1$ in partenza e
	      $\mathcal{B}_2$ in arrivo. Questa sar\`a la nostra $M$.
	\item Troviamo l'inversa di $M$ tramite il metodo specificato al capitolo
	      precedente.
	\item A questo punto risolviamo
	      \[
		      [L]_{\substack{\mathcal{B}_2 \\ \mathcal{B}_2}} =
		      M^{-1} [L]_{\substack{\mathcal{B}_1 \\ \mathcal{B}_1}} M
	      \]
\end{enumerate}

\begin{example}
	Consideriamo $L : \mathbb{R}^2 \to \mathbb{R}^2$
	\[
		L \begin{pmatrix}
			x \\ y
		\end{pmatrix} =
		\begin{pmatrix}
			x + y \\ x
		\end{pmatrix}
	\]
	e la sua matrice associata rispetto alla base standard di $\mathbb{R}^2$ in
	partenza e in arrivo
	\[
		[L] = \begin{pmatrix}
			1 & 1 \\
			1 & 0
		\end{pmatrix}
	\]
	Vogliamo scrivere ora la matrice associata a $L$ rispetto alla base
	\[
		\mathcal{B} = \left\{
		\begin{pmatrix}
			1 \\ 1
		\end{pmatrix}, \quad
		\begin{pmatrix}
			0 \\ 1
		\end{pmatrix}
		\right\}
	\]
	di $\mathbb{R}^2$ in partenza e in arrivo. Per farlo seguiamo i punti elencati
	sopra.
	\begin{enumerate}
		\item Troviamo la matrice $M$ di cambiamento di base scrivendo la matrice
		      associata all'identit\`a con la base standard di $\mathbb{R}^2$ in partenza
		      e la base $\mathcal{B}$ in arrivo.
		      \[
			      M = \begin{pmatrix}
				      1 & 0 \\
				      1 & 1
			      \end{pmatrix}
		      \]
		\item Troviamo l'inversa di $M$.
		      \[
			      M^{-1} = \begin{pmatrix}
				      1  & 0 \\
				      -1 & 1
			      \end{pmatrix}
		      \]
		\item Scriviamo infine la matrice cercata risolvendo l'equazione
		      \[
			      [L]_{\substack{\mathcal{B} \\ \mathcal{B}}} =
			      M^{-1} [L] M
		      \]
		      Avremo dunque che
		      \begin{gather*}
			      [L]_{\substack{\mathcal{B} \\ \mathcal{B}}} =
			      \begin{pmatrix}
				      1  & 0 \\
				      -1 & 1
			      \end{pmatrix}
			      \begin{pmatrix}
				      1 & 1 \\
				      1 & 0
			      \end{pmatrix}
			      \begin{pmatrix}
				      1 & 0 \\
				      1 & 1
			      \end{pmatrix} \\
			      \Downarrow \\
			      [L]_{\substack{\mathcal{B} \\ \mathcal{B}}} =
			      \begin{pmatrix}
				      1 & 1  \\
				      0 & -1
			      \end{pmatrix}
			      \begin{pmatrix}
				      1 & 0 \\
				      1 & 1
			      \end{pmatrix} \\
			      \Downarrow \\
			      [L]_{\substack{\mathcal{B} \\ \mathcal{B}}} =
			      \begin{pmatrix}
				      2  & 1  \\
				      -1 & -1
			      \end{pmatrix}
		      \end{gather*}
	\end{enumerate}
	Lo stesso identico risultato si ottiene scrivendo esprimendo
	l'immagine di ogni elemento della base $\mathcal{B}$ come combinazione lineare
	della stessa base $\mathcal{B}$ e mettendo poi i vettori colonna ottenuti
	uno di fianco all'altro (come abbiamo visto nei primi capitoli).
\end{example}

\chapter{Determinante}
\section{Definizione di determinante}
Il determinante è una funzione
\[ \det : \Mat_{n \times n}(\K) \to \K \]
Per allegerire la notazione talvolta indicheremo con $|a_{ij}|$ oppure con
$\det(a_{ij})$ o con $\det(A)$ il determinante di una matrice $A = (a_{ij})$.

Il determinante è definito ricorsivamente, al crescere di $n$, nel seguente
modo:
\begin{itemize}
	\item Il determinante di una matrice $1 \times 1$ è uguale all'unico
	      coefficiente della matrice:
	      \[ \det(a) = a \]
	\item Dato $n \geq 2$ il determinante di una matrice $A = (a_{ij})$ di
	      formato $n \times n$ può essere ottenuto come combinazione lineare
	      di una qualunque riga, diciamo la $i$-esima, tramite la seguente
	      formula:
	      \[ \det(A) = (-1)^{i + 1}a_{i1}\det(A_{i1}) + \dots + (-1)^{i + n}a_{in}\det(A_{in}) \]
	      dove $A_{ij}$ indica la matrice quadrata di formato
	      $(n - 1) \times (n - 1)$ che si ottiene da $A$ cancellando la riga
	      $i$-esima e la colonna $j$-esima.
\end{itemize}

\begin{observation}
	Dalla definizione possiamo immediatamente ricavare la seguente formula
	per le matrici $2 \times 2$:
	\[
		\det \begin{pmatrix}
			a & b \\
			c & d
		\end{pmatrix} =
		ad - bc
	\]
\end{observation}

\begin{observation}
	Il determinante si può mettere anche come combinazione lineare dei
	coefficienti di una qualsiasi colonna, diciamo la $j$-esima, tramite
	la seguente formula:
	\[ \det(A) = (-1)^{1 + j}a_{1j}\det(A_{1j}) + \dots + (-1)^{n + j}a_{nj}\det(A_{nj}) \]
\end{observation}

\begin{example}
	Data in $\Mat_{3 \times 3}(\K)$ la matrice
	\[
		\begin{pmatrix}
			3 & 2 & 5 \\
			2 & 0 & 1 \\
			4 & 2 & 6
		\end{pmatrix}
	\]
	per calcolare il determinante si sceglie una riga (o una colonna) e poi si
	applica la formula. Per esempio, scegliamo la seconda riga:
	\begin{gather*}
		\det(A) = -2Det \begin{pmatrix}
			2 & 5 \\
			2 & 6
		\end{pmatrix} +
		0 \det \begin{pmatrix}
			3 & 5 \\
			4 & 6
		\end{pmatrix} -
		\det \begin{pmatrix}
			3 & 2 \\
			4 & 2
		\end{pmatrix} = \\
		= -2(12 - 10) - (6 - 8) = -4 + 2 = -2
	\end{gather*}
\end{example}

\begin{observation}
	Nel caso delle matrici $3 \times 3$ il determinante si può calcolare
	anche mediante la seguente \emph{regola di Sarrus}. Data
	\[
		B = \begin{pmatrix}
			a & b & c \\
			d & e & f \\
			g & h & i \\
		\end{pmatrix}
	\]
	si forma la seguente matrice $3 \times 5$
	\[
		\begin{pmatrix}
			a & b & c & a & b \\
			d & e & f & d & e \\
			g & h & i & g & h
		\end{pmatrix}
	\]
	dopodiché si sommano i tre prodotti dei coefficienti che si trovano sulle
	tre diagonali che scendono da sinistra a destra e si sottragono i tre
	prodotti dei coefficienti che si trovano sulle tre diagonali che salgono
	da sinistra a destra:
	\[
		\det(B) = aci + bfg + cdh - gec - hfa - idb
	\]
	Verifichiamo che nel caso della matrice
	\[
		A = \begin{pmatrix}
			3 & 2 & 5 \\
			2 & 0 & 1 \\
			4 & 2 & 6
		\end{pmatrix}
	\]
	la regola di Sarrus dia lo stesso risultato -2 che abbiamo già calcolato:
	\[
		3 \cdot 0 \cdot 6 + 2 \cdot 1 \cdot 4 + 5 \cdot 2 \cdot 2 -
		4 \cdot 0 \cdot 5 - 2 \cdot 1 \cdot 3 - 6 \cdot 2 \cdot 2 =
		8 + 20 - 6 - 24 = -2
	\]
\end{observation}


\chapter{Diagonalizzazione di endomorfismi lineari}
\section{Autovalori e autovettori}
Sia $T : V \to V$ un endomorfismo lineare dello spazio $V$ sul campo $\K$.

\begin{definition}
	Un vettore $v \in V - \{O\}$ si dice un \textbf{autovettore} di $T$ se
	\[
		T(v) = \lambda v
	\]
	per un certo $\lambda \in \K$.
\end{definition}

In altre parole un autovettore di $T$ è un vettore diverso da $O$ dello spazio $V$
che ha la seguente proprietà: la $T$ lo manda in un multiplo di se stesso.

\begin{definition}
	Se $v \in V - \{O\}$ è un autovettore di $T$ tale che
	\[
		T(v) = \lambda v
	\]
	allora lo scalare $\lambda \in \K$ si dice \textbf{autovalore} di $T$
	relativo a $v$ (e viceversa si dice che $v$ è un autovettore relativo a
	$\lambda$).
\end{definition}

Si noti che l'autovalore può essere $0 \in \K$: se per esempio $T$ non
è iniettiva, ossia $\Ker(T) \supsetneq \{O\}$, tutti gli elementi
$w \in (\Ker(T)) - \{O\}$ soddisfano
\[
	T(w) = O = 0w
\]
ossia sono autovettori relativi all'autovalore 0.

\begin{definition}
	Dato $\lambda \in \K$ chiamiamo l'insieme
	\[
		V_\lambda = \{v \in V \mid T(v) = \lambda v\}
	\]
	\textbf{autospazio} relativo a $\lambda$.
\end{definition}

\begin{observation}
	Possiamo notare dalla definizione precedente che $V_0 = \Ker(T)$.
\end{observation}

Anche se abbiamo definito l'autospazio $V_\lambda$ per qualunque
$\lambda \in \K$, in realtà $V_\lambda$ è sempre uguale a $\{O\}$ a
meno che $\lambda$ non sia un autovalore. Questo è dunque il caso interessante:
se $\lambda$ è un autovalore di $T$ allora $V_\lambda$ è costituito da $O$ e
da tutti gli autovettori relativi a $\lambda$.

Ma perché sono importanti autovettori e autovalori ?
Supponiamo che $V$ abbia dimensione $n$ e pensiamo a cosa succederebbe se
riuscissimo a trovare una base di $V$, $\{v_1, v_2, \dots, v_n\}$, composta solo
da autovettori di $T$.

Avremmo, per ogni $i = 1, 2, \dots, n$,
\[
	T(v_i) = \lambda_i v_i
\]
per certi autovalori $\lambda_i$.

Come sarebbe fatta la matrice
\[
	[T]_{\substack{
				v_1, v_2, \dots, v_n \\
				v_1, v_2, \dots, v_n
			}}
\]
associata a $T$ rispetto a questa base ?

Ricordandoci come si costruiscono le matrici osserviamo che la prima colonna
conterrebbe il vettore $T(v_1)$ scritto in termini della base $\{v_1, \dots v_n\}$,
ossia
\[
	T(v_1) = \lambda_1 v_1 + 0 v_2 + 0 v_3 + \cdots + 0 v_n
\]
la seconda il vettore $T(v_2) = 0 v_1 + \lambda_2 v_2 + 0 v_3 + \cdots + 0 v_n$ e
così via. Otterremo quindi una matrice diagonale.
\[
	[T]_{\substack{
				v_1, v_2, \dots, v_n \\
				v_1, v_2, \dots, v_n
			}} = \begin{pmatrix}
		\lambda_1 & 0         & 0     & 0         \\
		0         & \lambda_2 & 0     & 0         \\
		0         & 0         & \dots & 0         \\
		0         & 0         & 0     & \lambda_n
	\end{pmatrix}
\]

Da questa matrice possiamo ricavare a colpo d'occhio informazioni come
\begin{itemize}
	\item Il rango di $T$.
	\item La dimensione del nucleo.
	\item Quali sono i vettori di $\Ker(T)$.
	\item Quali sono (se esistono) i sottospazi in cui $T$ si comporta come l'identità,
	      ossia i sottospazi costituiti dai vettori di $V$ che $T$ lascia fissi.
\end{itemize}

Dunque l'obbiettivo di studiare autovalori e autovettori di $T$ è quello di
trovare basi "buone" che ci permettano di conoscere bene il comportamento di $T$.
Tuttavia non esistono sempre queste basi buone. E si dice che, se per un certo
endomorfismo $T$ esiste una base buona, questo è \textbf{diagonalizzabile}.

\begin{example}
	Consideriamo l'endomorfismo $R_{\theta} : \R^2 \to \R^2$ dato
	da una \emph{rotazione} di angolo $\theta$ con centro l'origine. Si verifica
	immediatamente che, rispetto alla base standard di $\R^2$, questo
	endomorfismo è rappresentato dalla matrice
	\[
		[R_\theta] = \begin{pmatrix}
			\cos{\theta} & -\sin{\theta} \\
			\sin{\theta} & \cos{\theta}
		\end{pmatrix}
	\]
	Per esempio nel caso di una rotazione di $60^\circ$ (ovvero $\frac{\pi}{3}$),
	abbiamo:
	\[
		R_{\frac{\pi}{3}} = \begin{pmatrix}
			\frac{1}{2}        & -\frac{\sqrt{3}}{2} \\
			\frac{\sqrt{3}}{2} & \frac{1}{2}
		\end{pmatrix}
	\]
	Nel caso in cui $0 < \theta < \pi$, non ci sono vettori $v \neq O$ che vengono
	mandati in un multiplo di se stessi, visto che tutti i vettori vengono ruotati
	di un angolo che non è nullo e non è di $180^\circ$. Dunque non ci sono
	autovalori e autovettori.

	Nel caso $\theta = 0$ la rotazione è l'identità, dunque tutti i vettori
	$v \neq O$ sono autovettori relativi all'autovalore 1, e $V_1 = \R^2$.

	Nel caso $\theta = \pi$ la rotazione è uguale a $-I$, dunque tutti i vettori
	$v \neq O$ sono autovettori relativi all'autovalore $-1$, e
	$V_{-1} = \R^2$.
\end{example}


\subsection{Polinomio caratteristico}
Vogliamo trovare dei criteri semplici per stabilire se un endomorfismo
\`e diagonalizzabile o no. Prima di tutto troviamo un metodo che, dato un
endomorfismo $T : V \to V$ e posto $n = dim(V)$, ci permetta di decidere se
uno scalare $\lambda \in \mathbb{K}$ \`e o no un autovalore di $T$. Entrano
qui in gioco i polinomi e le loro radici.

Innanzitutto osserviamo che, perch\'e $\lambda \in \mathbb{K}$ sia un
autovalore, secondo la definizione bisogna che esista un $v \in V - \{O\}$
tale che
\begin{equation*}
	T(v) = \lambda v
\end{equation*}
Questo si pu\`o riscrivere anche come
\begin{equation*}
	T(v) - \lambda I(v) = O
\end{equation*}
dove $I : V \to V$ \`e l'identit\`a. Riscriviamo ancora:
\begin{equation*}
	(T - \lambda I)(v) = O
\end{equation*}
Abbiamo scoperto che, se $T$ possiede un autovalore $\lambda$, allora
l'endomorfismo $T - \lambda I$ non \`e iniettivo: infatti manda il vettore
$v$ in $O$. Dunque, se scegliamo una base qualunque per $V$ e costruiamo la
matrice $[T]$ associata a $T$, la matrice $[T - \lambda I] = [T] - \lambda I$
dovr\`a avere determinante uguale a 0:
\begin{equation*}
	det([T] - \lambda I) = 0 = det(\lambda I - [T])
\end{equation*}
dove come consuetudine abbiamo indicato con $I$ anche la matrice identit\`a.

\begin{defn}
	Dato un endomorfismo $T : V \to V$ con $n = dim(V)$, scegliamo una base
	per $V$ e costruiamo la matrice $[T]$ associata a $T$ rispetto a tale
	base. Il \textbf{polinomio caratteristico} $P_T(t) \in \mathbb{K}[t]$
	dell'endomorfismo $T$ \`e definito da:
	\begin{equation*}
		P_T(t) = det(t[I] - [T])
	\end{equation*}
\end{defn}

\begin{observation}
	Prima di procedere dobbiamo fare un paio di considerazioni:
	\begin{enumerate}
		\item Perch\'e la definizione precedente abbia senso si deve verificare
		      che \[det(t[I] - [T])\] sia veramente un polinomio. Questo si pu\`o
		      dimostrare facilmente per induzione sulla dimensione
		      $n$ di $V$.
		\item \`E fondamentale inoltre che la definizione appena data non dipenda
		      dalla base scelta di $V$: non sarebbe una definizione buona se con
		      la scelta di due basi diverse ottenessimo due polinomi
		      caratteristici diversi.
	\end{enumerate}
\end{observation}

Questo problema per fortuna non si verifica. Infatti se scegliamo due basi $b$ e
$b'$ di $V$, come sappiamo, le due matrici $[T]_{\substack{b \\ b}}$ e
$[T]_{\substack{b' \\ b'}}$ sono legate dalla seguente relazione: esiste una
matrice $[B]$ invertibile tale che
\begin{equation*}
	[T]_{\substack{b \\ b}} =
		[B]^{-1} [T]_{\substack{b' \\ b'}} [B]
\end{equation*}

Usando il teorema di Binet a questo punto verifichiamo che
\begin{gather*}
	det \left(tI - [T]_{\substack{b \\ b}}\right) = \\
	det \left(tI - [B]^{-1} [T]_{\substack{b' \\ b'}} [B]\right) = \\
	det \left([B]^{-1} \left(tI - [T]_{\substack{b' \\ b'}}\right) [B] \right) = \\
	det \left([B]^{-1}\right) det \left(tI - [T]_{\substack{b' \\ b'}}\right)
	det \left([B]\right) = \\
	det \left(tI - [T]_{\substack{b' \\ b'}}\right)
\end{gather*}

Abbiamo dunque mostrato che $P_T(t) = det(tI - [T])$ non dipende dalla scelta della
base.

\begin{theorem}
	Considerato $T$ come sopra, vale che uno scalare $\lambda \in \mathbb{K}$ \`e un
	autovalore di $T$ se e solo se $\lambda$ \`e una radice di $P_T(t)$, ossia se e
	solo se $P_T(\lambda) = 0$.
\end{theorem}

\begin{example}
	Consideriamo l'endomorfismo $T : \mathbb{C}^2 \to \mathbb{C}^2$ che, rispetto
	alla base standard di $\mathbb{C}^2$, \`e rappresentato dalla matrice
	\begin{equation*}
		[T] = \begin{pmatrix}
			\frac{1}{2}        & -\frac{\sqrt{3}}{2} \\
			\frac{\sqrt{3}}{2} & \frac{1}{2}
		\end{pmatrix}
	\end{equation*}
	Il suo polinomio caratteristico risulta $P_T(t) = t^2 - t + 1$. Questo polinomio
	ha due radici in $\mathbb{C}$, ovvero $\frac{1 - i\sqrt{3}}{2}$ e
	$\frac{1 + i\sqrt{3}}{2}$, che in effetti, come sappiamo, sono gli autovalori
	di $T$.
\end{example}
\section{Strategia per scoprire se un endomorfismo è diagonalizzabile}
Con il metodo descritto a breve si potrà scoprire se un endomorfismo
$T : V \to V$, dove $V$ è uno spazio vettoriale su $\K$ di dimensione $n$
è diagonalizzabile, e, in caso lo sia, si potrà trovare una base che lo diagonalizza,
ossia una base di $V$ fatta tutta di autovettori di $T$.

\begin{itemize}
	\item PASSO 1. Troviamo gli autovalori di un endomorfismo lineare $T$ calcolando il polinomio
	      caratteristico e le sue radici in $\K$.
	\item PASSO 2. Supponiamo dunque di aver trovato gli autovalori di $T$. A questo punto
	      vogliamo individuare gli autospazi relativi a tali autovalori.
	      Per questo basterà calcolare il $\Ker([T] - \lambda_i I)$. Prendiamo dunque la
	      matrice $[T] - \lambda_i I$ e risolviamo il sistema lineare omogeneo associato.
	\item PASSO 3. Per prima cosa enunciamo il seguente teorema
	      \begin{theorem}
		      Dato un endomorfismo $T : V \to V$, siano $\lambda_1, \dots, \lambda_k$
		      degli autovalori di $T$ distinti fra loro. Consideriamo ora degli autovettori
		      $v_1 \in V_{\lambda_1}, \dots, v_k \in V_{\lambda_k}$. Allora
		      $v_1, \dots, v_k$ è un insieme di vettori linearmente indipendenti.
	      \end{theorem}

	      Il seguente teorema è un rafforazamento del precedente.
	      \begin{theorem}
		      Dato un endomorfismo $T : V \to V$, siano $\lambda_1, \dots, \lambda_k$ degli
		      autovalori di $T$ distinti fra loro. Allora gli autospazi
		      $V_{\lambda_1}, \dots, V_{\lambda_k}$, sono in somma diretta.
	      \end{theorem}

	      Nelle ipotesi del teorema precedente sappiamo allora, che la dimensione della somma
	      degli autospazi è la massima possibile, ossia
	      \[
		      \dim(V_{\lambda_1} \oplus \cdots \oplus V_{\lambda_k}) = \dim(V_{\lambda_1}) +
		      \cdots + \dim(V_{\lambda_k})
	      \]

	      Osserviamo che abbiamo già un criterio per dire se $T$ è diagonalizzabile o no.
	      Ovvero, se
	      \[
		      \dim(V_{\lambda_1}) + \cdots + \dim(V_{\lambda_k}) = n = \dim(V)
	      \]
	      altrimenti se
	      \[
		      \dim(V_{\lambda_1}) + \cdots + \dim(V_{\lambda_k}) < n = \dim(V)
	      \]
	      $T$ non è diagonalizzabile. Infatti non è possibile trovare una base di
	      autovettori.
	\item PASSO 4. Se l'endomorfismo $T$ è diagonalizzabile, scegliamo allora una base di
	      autovettori nel modo descritto al Passo 3, e avremo una matrice associata $[T]$ che
	      risulterà diagonale. Mantenendo la notazione introdotta al Passo 3 troviamo sulla
	      diagonale $\dim(V_{\lambda_1})$ coefficienti uguali a
	      $\lambda_1$, ... e $\dim(V_{\lambda_k})$ coefficienti uguali a $\lambda_k$.
\end{itemize}
Il rango di $T$ sarà uguale al numero dei coefficienti non nulli che troviamo
sulla diagonale di $[T]$, la dimensione del nucleo sarà uguale al numero dei
coefficienti uguali a zero che troviamo sulla diagonale di $[T]$.

\begin{example}
	Consideriamo l'endomorfismo
	\[
		T \begin{pmatrix} x \\ y \end{pmatrix} =
		\begin{pmatrix}
			x + 2y \\
			-y
		\end{pmatrix}
	\]
	e la sua matrice associata rispetto alle basi standard
	\[
		[T] = \begin{pmatrix}
			1 & 2  \\
			0 & -1
		\end{pmatrix}
	\]
	Per prima cosa calcoliamo la matrice $[T] - tI$ che chiameremo $M$ per comodità
	\[
		M = \begin{pmatrix}
			t - 1 & -2    \\
			0     & t + 1
		\end{pmatrix}
	\]
	Troviamo il polinomio caratteristico calcolando il determinante di $M$ e otteniamo
	\[
		P_T(t) = (t - 1)(t + 1)
	\]
	Le radici di tale polinomio (e quindi gli autovalori di $T$) sono $t = 1$ e $t = -1$.
	Dobbiamo trovare quindi i relativi autospazi.
	\begin{itemize}
		\item Se $t = 1$ dobbiamo calcolare $\Ker([T] - 1I)$. Dobbiamo quindi risolvere il
		      sistema associato alla matrice
		      \[
			      \begin{pmatrix}
				      1 - 1 & -2    \\
				      0     & 1 + 1
			      \end{pmatrix} =
			      \begin{pmatrix}
				      0 & -2 \\
				      0 & 2
			      \end{pmatrix}
		      \]
		      ovvero
		      \[
			      \begin{cases}
				      -2y & = 0 \\
				      2y  & = 0
			      \end{cases} \quad \Rightarrow \quad
			      y = 0
		      \]
		      otteniamo dunque che l'autospazio $V_1$ è definito come segue
		      \[
			      V_1 = < \begin{pmatrix} 1 \\ 0 \end{pmatrix} > \quad \Rightarrow \quad
			      \dim(V_1) = 1
		      \]
		\item Se $t = -1$ procediamo in maniera analoga. Stavolta otteniamo il sistema
		      \[
			      \begin{cases}
				      -2x - 2y & = 0 \\
			      \end{cases} \quad \Rightarrow \quad
			      x = -y
		      \]
		      Ne deduciamo che l'autospazio $V_2$ sarà definito come segue
		      \[
			      V_2 = < \begin{pmatrix} -1 \\ 1 \end{pmatrix} > \quad \Rightarrow \quad
			      \dim(V_2) = 1
		      \]
	\end{itemize}
	Dato che $T$ è definta su $\R^2$ che ha dimensione 2 e dato che
	\[ \dim(V_1) + \dim(V_2) = 2 \] l'endomorfismo è diagonalizzabile.
\end{example}
\section{Criteri di molteplicità}


\section{Prodotti scalari e spazi euclidei}
\subsection{Prodotto scalare}
\begin{defn}
	Sia $V$ uno spazio vettoriale sul campo $\mathbb{K}$, dove
	$\mathbb{K} = \mathbb{R}^n$ o $\mathbb{K} = \mathbb{C}^n$. Il 
	\textbf{prodotto scalare} \`e una funzione che ad ogni coppia di vettori $u, v$
	appartenenti a $V$ associa lo scalare $\langle u, v \rangle \in \mathbb{K}$
	calcolato come segue:
	\[
		\langle u, v \rangle = u_1 v_1 + u_2 v_2 + ... + u_n v_n
	\]
	dove gli $u_i$ e $v_i$ sono le $i$-esime coordinate di $u$ e $v$.
	
	Il prodotto scalare gode inoltre delle seguenti propriet\`a:
	\begin{enumerate}
		\item $\langle au_1 + bu_2, v \rangle =
			      a \langle u_1, v \rangle + b \langle u_2, v \rangle$, per ogni
		      $u_1, u_2, v \in V$ e per ogni $a, b \in \mathbb{K}$.
		\item $\langle u, v \rangle = \overline{\langle v, u \rangle}$ per ogni
		      $u, v \in V$ (dove $\overline{\langle v, u \rangle}$ indica il numero
		      complesso coniugato di $\langle u, v \rangle$).
		\item per ogni $u \in V$ vale $\langle u, u \rangle \geq 0$ e
		      $\langle u, u \rangle = 0$ se e solo se $u = O$.
	\end{enumerate}
	Uno spazio vettoriale reale $V$ munito di un prodotto scalare si dice
	\textbf{spazio euclideo}.
\end{defn}

\begin{observation}
	Osserviamo che, per la seconda propriet\`a, il prodotto scalare
	$\langle u, u \rangle$ \`e sempre un numero reale, dunque ha senso la disuguaglianza
	che compare nella terza propriet\`a.
\end{observation}

\begin{defn}
	Per ogni vettore $u \in V$, scriveremo
	\begin{equation*}
		\| u \| = \sqrt{\langle u, u \rangle}
	\end{equation*}
	e diremo che $\| u \|$ \`e la \textbf{norma} di $u$.
\end{defn}

\begin{observation}
	Dalle propriet\`a (1) e (2) segue che
	\begin{equation*}
		\langle u, av_1 + bv_2 \rangle =
		\overline{a} \langle u, v_1 \rangle + \overline{b} \langle u, v_2 \rangle
	\end{equation*}
	Se il campo $\mathbb{K}$ \`e $\mathbb{R}$ tutti i simboli di coniugazione
	complessa possono essere ignorati.
\end{observation}

\begin{observation}
	Il prodotto scalare standard estende a tutti gli spazi $\mathbb{R}^n$ un concetto
	che in $\mathbb{R}^2$ e in $\mathbb{R}^3$ ci \`e gi\`a familiare. Si pu\`o
	facilmente verificare che in questi casi per esempio $\| u \|$ coincide con
	ci\`o che \`e chiamata lunghezza del vettore $u$ e che $\| u - v \|$ coincide
	con la distanza tra i vettori $u$ e $v$. Inoltre $\langle u, v \rangle = 0$ se
	e solo se $u$ e $v$ sono ortogonali fra loro. Potete anche gi\`a osservare in
	$\mathbb{R}^2$ che vale la seguente relazione:
	\begin{equation*}
		\langle u, v \rangle = \| u \| \| v \| \cos \theta
	\end{equation*}
	dove $\theta$ \`e l'angolo compreso fra i vettori $u$ e $v$.
\end{observation}

\begin{example}
	In $\mathbb{C}^n$, dati i vettori
	\begin{equation*}
		u = \begin{pmatrix}
			a_1 \\ \dots \\ a_n
		\end{pmatrix}, \quad
		v = \begin{pmatrix}
			b_1 \\ \dots \\ b_n
		\end{pmatrix}
	\end{equation*}
	scritti rispetto alla base standard, abbiamo il prodotto scalare
	\begin{equation*}
		\langle u, v \rangle = a_1 \overline{b_1} + \cdots + a_n \overline{b_n}
	\end{equation*}
\end{example}

\begin{example}
	In $\mathbb{R}^3$, dati i vettori 
	\[
		u = \begin{pmatrix}
			2 \\ 1 \\ 1
		\end{pmatrix}, \quad
		v = \begin{pmatrix}
			-1 \\ 0 \\ 3
		\end{pmatrix}
	\]
	Abbiamo che il prodotto scalare $\langle u, v \rangle$ \`e dato da 
	\[
		\langle u, v \rangle = 2 \cdot (-1) + 1 \cdot 0 + 1 \cdot 3 = 1
	\]
\end{example}

\chapter{Teorema spettrale}
\section{Introduzione al teorema spettrale}

\begin{theorem}
	Sia $T : V \to V$ un endomorfismo di uno spazio vettoriale $V$ di dimensione
	finita sul campo $\mathbb{K}$ (con $\mathbb{K} = \mathbb{R}$ o
	$\mathbb{K} = \mathbb{C}$) e munito di prodotto scalare. Esiste allora un
	endomorfismo $T^*$ univocamente determinato tale che
	\begin{equation*}
		\langle T(u), v \rangle = \langle u, T^*(v) \rangle
	\end{equation*}
	per ogni $u, v \in V$. Se si fissa in $V$ una base ortonormale vale che la
	matrice $[T^*]$ di $T$ rispetto a tale base \`e la trasposta della coniugata
	di $[T]$.
\end{theorem}

Sottolineiamo che per l'esistenza dell'endomorfismo $T^*$ \`e fondamentale che
lo spazio $V$ abbia dimensione finita.

\begin{definition}
	L'endomorfismo $T^*$ si dice \textbf{aggiunto} di $T$. Se vale che $T = T^*$
	allora $T$ si dice \textbf{autoaggiunto}.
\end{definition}

Osserviamo che nel caso in cui $\mathbb{K} = \mathbb{R}$, se fissiamo una base
ortonormale di $V$ e consideriamo un endomorfismo $T$ autoaggiunto, allora la
matrice $[T]$ rispetto a tale base \`e simmetrica, ossia, indicando con $[T]^t$
la trasposta di $[T]$, vale
\begin{equation*}
	[T] = [T]^t
\end{equation*}
Viceversa, se un endomorfismo \`e rappresentato, rispetto ad una base ortonormale,
da una matrice simmetrica, allora si mostra facilmente che \`e autoaggiunto.
Per questo motivo nel caso in cui $\mathbb{K} = \mathbb{R}$ un endomorfismo
autoaggiunto si dice anche \textbf{simmetrico}.

\begin{theorem}
	Sia $T : V \to V$ un endomorfismo autoaggiunto di uno spazio vettoriale $V$
	di dimensione finita sul campo $\mathbb{K}$. Sia $\lambda$ un autovalore di
	$T$. Allora $\lambda \in \mathbb{R}$.
\end{theorem}

\begin{theorem}
	Sia $T : V \to V$ un endomorfismo lineare simmetrico di uno spazio vettoriale
	$V$ di dimensione finita sul campo $\mathbb{K} = \mathbb{R}$. Allora sono vere
	le seguenti affermazioni:
	\begin{enumerate}
		\item Il polinomio caratteristico $P_T(t)$ ha tutte le radici reali e si
		      fattorizza come prodotto di fattori lineari su $\mathbb{R}$.
		\item L'endomorfismo $T$ ha almeno un autovettore non nullo.
		\item Siano $v_1, \dots, v_r$ autovettori di $T$, relativi ad autovalori
		      $\lambda_1, \dots, \lambda_r$. Allora l'insieme di vettori
		      $\{v_1, \dots, v_r\}$ \`e ortogonale.
	\end{enumerate}
\end{theorem}

\begin{lemma}
	Sia $V$ uno spazio vettoriale sul campo $\mathbb{K} = \mathbb{R}$ oppure
	$\mathbb{K} = \mathbb{C}$, di dimensione finita e munito di prodotto scalare.
	Sia $T : V \to V$ un endomorfismo, e sia $W$ un sottospazio di $V$ invariante
	per $T$, ossia tale che $T(W) \subset W$. Allora $W^\perp$ \`e invariante per
	$T^*$.
\end{lemma}

\begin{theorem}[Teorema spettrale, caso reale]
	Sia $T : V  \to V$ un endomorfismo simmetrico di uno spazio vettoriale $V$ su
	$\mathbb{K} = \mathbb{R}$ di dimensione finita maggiore di 0. Allora esiste una
	base ortonormale di $V$ i cui elementi sono autovettori di $T$.
\end{theorem}

\begin{example}
	Consideriamo l'endomorfismo $T : \mathbb{R}^2 \to \mathbb{R}^2$ la cui matrice,
	rispetto alla base standard \`e
	\begin{equation*}
		[T] = \begin{pmatrix}
			1 & 2 \\
			2 & 1
		\end{pmatrix}
	\end{equation*}
	Osserviamo che \`e un endomorfismo simmetrico, infatti la sua matrice rispetto
	alla base standard (che \`e una base ortonormale) \`e simmetrica. Dunque per
	il teorema spettrale sappiamo che $T$ \`e diagonalizzabile e ci aspettiamo di
	trovare una base ortonormale che lo diagonalizzi.

	Il polinomio caratteristico \`e
	\begin{equation*}
		P_T(t) = t^2 - 2t - 3 = (t - 3)(t + 1)
	\end{equation*}
	dunque gli autovalori sono 3 e -1, entrambi con molteplicit\`a algebrica 1.
	Questo gi\`a ci conferma che $T$ \`e diagonalizzabile. Troviamo l'autospazio
	$V_3$:
	\begin{equation*}
		V_3 = Ker(T - 3I) = Ker \begin{pmatrix}
			-2 & 2  \\
			2  & -2
		\end{pmatrix}
	\end{equation*}
	L'autospazio $V_3$ ha dunque dimensione 1 ed \`e generato dal vettore
	$v = \begin{pmatrix} 1 \\ 1	\end{pmatrix}$. Troviamo l'autospazio $V_{-1}$:
	\begin{equation*}
		V_{-1} = Ker(T + I) = Ker \begin{pmatrix}
			2 & 2 \\
			2 & 2
		\end{pmatrix}
	\end{equation*}
	L'autospazio $V_{-1}$ ha dimensione 1 ed \`e generato dal vettore
	$w = \begin{pmatrix} 1 \\ -1 \end{pmatrix}$. I due vettori $v$ e $w$ sono
	ortogonali. Concludiamo dunque che la base ortonormale data dai vettori
	\begin{equation*}
		\begin{pmatrix}
			\frac{1}{\sqrt{2}} \\ \frac{1}{\sqrt{2}}
		\end{pmatrix},
		\begin{pmatrix}
			\frac{1}{\sqrt{2}} \\ \frac{1}{-\sqrt{2}}
		\end{pmatrix}
	\end{equation*}
	\`e una base di autovettori per $T$, in accordo col teorema spettrale
\end{example}

\begin{theorem}[Teorema spettrale in $\mathbb{R}$ e $\mathbb{C}$]
	Sia $T : V  \to V$ un endomorfismo autoaggiunto di uno spazio vettoriale $V$ di
	dimensione finita su $\mathbb{K} = \mathbb{R}$ o $\mathbb{K} = \mathbb{C}$.
	Allora esiste una base ortonormale di $V$ i cui elementi sono autovettori di $T$.
\end{theorem}


\section{Endomorfismi simmetrici definiti positivi o negativi}
Come gi\`a visto in precedenza, un endomorfismo autoaggiunto \`e diagonalizzabile e
tutti i suoi autovalori sono reali.

\begin{definition}
	Sia $T : V \to V$ un endomorfismo autoaggiunto di uno spazio vettoriale $V$ di
	dimensione finita su $\mathbb{K}$ (con $\mathbb{K} = \mathbb{R}$ o
	$\mathbb{K} = \mathbb{C}$). Se tutti i suoi autovalori sono numeri reali $\geq 0$
	si dice che $T$ \`e \textbf{semidefinito positivo}. Se tutti i suoi autovalori
	sono numeri reali $\leq 0$ si dice che $T$ \`e \textbf{semidefinito negativo}.
	In particolare, se tutti i suoi autovalori sono numeri reali $> 0$ si dice che
	$T$ \`e \textbf{definito positivo}. Se invece tutti i suoi autovalori sono numeri
	reali $< 0$ si dice che $T$ \`e \textbf{definito negativo}.
\end{definition}

La definizione appena data si applica in particolare nel caso reale degli
endomorfismi simmetrici.

Consideriamo, per esempio, cosa accade nel caso in cui $V$ ha dimensione 2.
Ricordiamo innanzitutto che, dato un endomorfismo, sono ben definiti la sua traccia
e il suo determinante.

\begin{proposition}
	Sia $T : \mathbb{R}^2 \to \mathbb{R}^2$ un endomorfismo simmetrico. Vale che:
	\begin{enumerate}
		\item
		      $T$ \`e definito positivo se e solo se il suo determinante e la sua
		      traccia sono entrambi $> 0$.
		\item
		      $T$ \`e definito negativo se e solo se il suo determinante \`e $> 0$ e
		      la sua traccia \`e $< 0$.
	\end{enumerate}

	\begin{proof}
		Come sappiamo dal teorema spettrale, $T$ \`e diagonalizzabile. Siano
		$\lambda_1, \lambda_2$ i suoi autovalori. Fissata una base di autovettori,
		la matrice che rappresenta $T$ rispetto a tale base \`e:
		\begin{equation*}
			\begin{pmatrix}
				\lambda_1 & 0         \\
				0         & \lambda_2
			\end{pmatrix}
		\end{equation*}
		da cui si ricava subito che il determinante di $T$ \`e $\lambda_1 \lambda_2$
		e la traccia \`e $\lambda_1 + \lambda_2$. A questo punto si deve solo
		analizzare caso per caso i valori di $\lambda_1$ e $\lambda_2$ e valutare
		il segno di determinante e traccia.
	\end{proof}
\end{proposition}



\end{document}