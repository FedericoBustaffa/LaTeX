\chapter{Applicazioni lineari e matrici invertibili}
\section{Endomorfismi lineari invertibili}

\begin{definition}
	Consideriamo uno spazio vettoriale $V$ di dimensione $n$ sul campo $\K$ e
	una applicazione lineare $L : V \to V$. Una tale applicazione lineare si dice
	\textbf{endomorfismo lineare di $V$}. Indicheremo con $End(V)$ l'insieme di tutti gli
	endomorifsimi lineari di $V$.
\end{definition}

\begin{proposition}
	Un endomorfismo $L$ di $V$ è invertibile se e solo se ha rango $n$. La
	funzione inversa $L^{-1} : V \to V$ è anch'essa un'applicazione lineare.
\end{proposition}

\begin{observation}
	Dati due spazi $V$ e $W$ entrambi di dimensione $n$ e una applicazione lineare
	$L : V \to W$, l'applicazione lineare $L$ è invertibile se e solo se ha rango
	$n$; l'applicazione inversa $L^{-1}$ è anch'essa lineare.
	Abbiamo invece già osservato che se $V$ e $W$ hanno dimensioni diverse,
	rispettivamente $m$ e $n$, nessuna applicazione lineare $L$ da $V$ a $W$ può
	essere invertibile. Infatti, avendo in mente la relazione che lega la dimensione
	del nucleo di $L$, dell'immagine di $L$ e di $V$
	($\dim(\Imm(L)) + \dim(\Ker(L)) = \dim(V)$), si ha che:
	\begin{itemize}
		\item se $m > n$ allora la dimensione di $\Imm(L)$ è al massimo $n$.
		      Quindi la dimensione di $\Ker(L)$ è almeno $m - n$, ovvero
		      maggiore di 0, e $L$ non è iniettiva.
		\item se $m < n$ allora la dimensione di $\Imm(L)$ è al massimo $m$.
		      Quindi la dimensione di $\Imm(L)$ è minore della dimensione di
		      $W$, e $L$ non è surgettiva.
	\end{itemize}
\end{observation}

Se fissiamo una base di $V$, ad ogni endomorfismo $L \in End(V)$ viene associata
una matrice $[L] \in \Mat_{n \times n}(\K)$. Se $L$ è invertibile,
consideriamo l'inversa $L^{-1}$ e la matrice ad essa associata $[L^{-1}]$.
Visto che $L \circ L^{-1} = L^{-1} \circ L = I$, vale
\[ [L^{-1}][L] = [L][L^{-1}] = [I] = I \]
Dunque la matrice $[L]$ è invertibile e ha per inversa $[L^{-1}]$.
Possiamo affermare anche il viceversa: se la matrice $[L]$ associata ad un
endomorfismo lineare è invertibile allora anche $L$ è invertibile e la sua
inversa è l'applicazione associata alla matrice $[L^{-1}]$.

\begin{corollary}
	Una matrice $A \in \Mat_{n \times n}(\K)$ è invertibile se e solo se
	il suo rango è $n$.
\end{corollary}
