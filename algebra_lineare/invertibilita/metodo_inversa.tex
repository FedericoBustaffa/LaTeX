\section{Metodo per trovare l'inversa di una matrice}
Come abbiamo visto nel paragrafo precedente, il problema di trovare un'inversa
di $L \in End(V)$ si può tradurre nel problema di trovare l'inversa in
$\Mat_{n \times n}(\K)$ di una matrice data. In questo paragrafo descriviamo
un metodo per trovare l'inversa di una matrice $A \in \Mat_{n \times n}(\K)$.

\begin{definition}
	Una matrice in forma a scalini per righe (o per colonne) ridotta, si dice
	\textbf{normalizzata} se ha tutti i pivot uguali a 1.
\end{definition}

\begin{observation}
	Una matrice può essere portata in forma a scalini per righe (o colonne)
	ridotta e normalizzata, attraverso un numero finito di operazioni elementari.
\end{observation}

\begin{example}
	Consideriamo la matrice
	\[
		A = \begin{pmatrix}
			3 & 2 & 1 \\
			0 & 1 & 1 \\
			1 & 1 & 0
		\end{pmatrix}
	\]
	che ha rango 3, dunque è invertibile, e calcoliamo la
	sua inversa. Per prima cosa formiamo la matrice
	\[
		(A I) = \begin{pmatrix}
			3 & 2 & 1 & 1 & 0 & 0 \\
			0 & 1 & 1 & 0 & 1 & 0 \\
			1 & 1 & 0 & 0 & 0 & 1
		\end{pmatrix}
	\]
	Ora con delle operazioni elementari di riga portiamola in forma a scalini
	per righe ridotta, per esempio nel seguente modo: si sottrae alla prima riga
	la terza moltiplicata per 3
	\[
		\begin{pmatrix}
			3 & 2 & 1 & 1 & 0 & 0 \\
			0 & 1 & 1 & 0 & 1 & 0 \\
			1 & 1 & 0 & 0 & 0 & 1
		\end{pmatrix} \to
		\begin{pmatrix}
			0 & -1 & 1 & 1 & 0 & -3 \\
			0 & 1  & 1 & 0 & 1 & 0  \\
			1 & 1  & 0 & 0 & 0 & 1
		\end{pmatrix}
	\]
	poi si somma alla prima riga la seconda
	\[
		\begin{pmatrix}
			0 & -1 & 1 & 1 & 0 & -3 \\
			0 & 1  & 1 & 0 & 1 & 0  \\
			1 & 1  & 0 & 0 & 0 & 1
		\end{pmatrix} \to
		\begin{pmatrix}
			0 & 0 & 2 & 1 & 1 & -3 \\
			0 & 1 & 1 & 0 & 1 & 0  \\
			1 & 1 & 0 & 0 & 0 & 1
		\end{pmatrix}
	\]
	a questo punto si permutano le righe e si ottiene
	\[
		\begin{pmatrix}
			1 & 1 & 0 & 0 & 0 & 1  \\
			0 & 1 & 1 & 0 & 1 & 0  \\
			0 & 0 & 2 & 1 & 1 & -3
		\end{pmatrix}
	\]
	Per ottenere la forma a scalini ridotta, moltiplichiamo l'ultima riga per
	$\frac{1}{2}$
	\[
		\begin{pmatrix}
			1 & 1 & 0 & 0           & 0           & 1            \\
			0 & 1 & 1 & 0           & 1           & 0            \\
			0 & 0 & 1 & \frac{1}{2} & \frac{1}{2} & -\frac{3}{2}
		\end{pmatrix}
	\]
	sottraiamo alla seconda riga la terza riga
	\[
		\begin{pmatrix}
			1 & 1 & 0 & 0           & 0           & 1            \\
			0 & 1 & 0 & \frac{1}{2} & \frac{1}{2} & \frac{3}{2}  \\
			0 & 0 & 1 & \frac{1}{2} & \frac{1}{2} & -\frac{3}{2}
		\end{pmatrix}
	\]
	infine sottriamo alla prima riga la seconda:
	\[
		\begin{pmatrix}
			1 & 0 & 0 & \frac{1}{2}  & -\frac{1}{2} & -\frac{1}{2} \\
			0 & 1 & 0 & -\frac{1}{2} & \frac{1}{2}  & \frac{3}{2}  \\
			0 & 0 & 1 & \frac{1}{2}  & \frac{1}{2}  & -\frac{3}{2}
		\end{pmatrix}
	\]
	La matrice
	\[
		B = \begin{pmatrix}
			\frac{1}{2}  & -\frac{1}{2} & -\frac{1}{2} \\
			-\frac{1}{2} & \frac{1}{2}  & \frac{3}{2}  \\
			\frac{1}{2}  & \frac{1}{2}  & -\frac{3}{2}
		\end{pmatrix}
	\]
	è l'inversa di $A$.
\end{example}

\textbf{Perchè il metodo funziona ?}

Consideriamo una matrice $A \in \Mat_{n \times n}(\K)$ e cerchiamo la sua
inversa; supponiamo che $A$ abbia rango $n$.

Per prima cosa creiamo una matrice $n \times 2n$ ponendo accanto le colonne di $A$
e quelle di $I$. Indicheremo tale matrice con $(A I)$.

Adesso possiamo agire con operazioni elementari di riga in modo da ridurre la
matrice in forma a scalini per righe ridotta. Poichè $A$ ha rango $n$, anche
$(A I)$ ha rango $n$. Un modo per rendersene conto è il seguente: il rango di
$(A I)$ è minore o uguale a $n$ visto che ha $n$ righe ed è maggiore o uguale
a $n$ visto che individuiamo facilmente $n$ colonne linearmente indipendenti.

Allora quando la matrice $(A I)$ viene ridotta in forma a scalini per righe ridotta,
deve avere esattamente $n$ scalini, dunque deve avere la forma $(I B)$.
Affermiamo che la matrice $B$ che si ricava dalla matrice precedente è proprio
l'inversa di $A$ che cercavamo.

Infatti agire con operazioni di riga equivale a moltiplicare a sinistra la matrice
$(A I)$ per una matrice invertibile $U$ di formato $n \times n$, dunque:
\[
	U (A I) = (I B)
\]
Per come è definito il prodotto righe per colonne,
\[
	U (A I) = (U A U I)
\]
Dalle uguaglianze precedenti ricaviamo
\[
	(U A U I) = (I B)
\]
ossia le relazioni $U A = I$ e $U I = B$ che ci dicono che $U$ è l'inversa di
$A$ e che $U = B$ come avevamo annunciato.

\begin{observation}
	La relazione $U A = I$, ossia $BA = I$, ci dice solo che $B$ è l'inversa
	sinistra di $A$. In generale sappiamo che il prodotto tra matrici non è
	commutativo. Dunque, il fatto che $A$ sia invertibile, ci dice che esiste una
	matrice $B$ tale che $BA = I$, e che esiste una matrice $C$ tale che $AC = I$.
	Ma possiamo mostrare facilmente che $B$ coincide con l'inversa destra $C$ di
	$A$. Come detto $C$ deve soddisfare per definizione la condizione $AC = I$.
	Se moltiplichiamo per $C$ entrambi i membri della relazione $BA = I$ (a destra)
	\[ BAC = IC \] Usando la proprietà associativa del prodotto in
	$\Mat_{n \times n}(\K)$ otteniamo \[B = C\] visto che $AC = I$.
\end{observation}
