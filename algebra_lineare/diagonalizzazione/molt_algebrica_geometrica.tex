
\subsection{Criterio di molteplicit\`a algebrica e molteplicit\`a geometrica}
Nel paragrafo precedente abbiamo descritto un algoritmo che ci permette di decidere
se un endomorfismo \`e diagonalizzabile o no. In questo paragrafo faremo
un'osservazione che ci permetter\`a di riformulare questo algoritmo e di individuare
alcune scorciatoie.

Consideriamo come al solito un endomorfismo lineare $T : V \to V$, dove $V$ \`e uno
spazio vettoriale sul campo $\mathbb{K}$ con $n = dim(V)$.

Calcoliamo il suo polinomio caratteristico e fattorizziamolo in $\mathbb{K}[t]$.
Otterremo un'espressione del tipo:
\begin{equation*}
	P_T(t) = (t - \lambda_1)^{a_1} \cdots (t - \lambda_k)^{a_k} f(t)
\end{equation*}
dove $\lambda_1, \dots, \lambda_k$ sono gli autovalori di $T$ in $\mathbb{K}$ e sono
tutti distinti fra loro, e $f(t)$ o \`e 1 o \`e un polinomio irriducibile in
$\mathbb{K}[t]$ di grado $> 1$.

Se $T$ \`e diagonalizzabile, allora esiste una base $b$ di $V$ in cui la matrice
associata $[T]_{\substack{b \\ b}}$ ha forma diagonale e sulla diagonale compaiono,
per ogni $i = 1, 2, \dots, k$, $\lambda_i$ compare $V_{\lambda_i}$ compare
$dim(V_{\lambda_i})$ volte. Dunque in questo caso possiamo ricalcolare il polinomio
caratteristico $P_T$ usando $[T]_{\substack{b \\ b}}$:
\begin{equation*}
	P_T(t) = det(tI - [T]_{\substack{b \\ b}})
\end{equation*}
Si tratta di calcolare il determinante di una matrice diagonale e si osserva allora
che $P_T$ si spezza nel prodotto di fattori lineari:
\begin{equation*}
	P_T(t) = (t - \lambda_1)^{dim(V_{\lambda_1})} \cdots
	(t - \lambda_k)^{dim(V_{\lambda_k})}
\end{equation*}
il che dimostra che il fattore $f(t)$ \`e 1.

In sintesi:
\begin{proposition}
	Se l'endomorfismo $T$ \`e diagonalizzabile sul campo $\mathbb{K}$, allora il suo
	polinomio caratteristico $P_T(t)$ si fattorizza come prodotto di fattori lineari
	in $\mathbb{K}[t]$:
	\begin{equation*}
		P_T(t) = (t - \lambda_1)^{dim(V_{\lambda_1})} \cdots
		(t - \lambda_k)^{dim(V_{\lambda_k})}
	\end{equation*}
\end{proposition}

Dunque se nella fattorizzazione di $P_T$ rimane un fattore irriducibile di grado
maggiore di 1 possiamo dedurre che $T$ non \`e diagonalizzabile. In generale non
\`e vero il viceversa.

\begin{defn}
	Chiamo \textbf{molteplicit\`a algebrica} dell'autovalore $\lambda_i$ il numero di volte
	che $\lambda_i$ azzera il polinomio caratteristico.
\end{defn}

\begin{defn}
	Chiamo \textbf{molteplicit\`a geometrica} dell'autovalore $\lambda_i$ il valore di
	$dim(V_{\lambda_i})$.
\end{defn}

\begin{theorem}
	Per ogni autovalore $\lambda_i$, vale che la sua molteplicit\`a
	geometrica \`e minore o uguale alla sua molteplicit\`a algebrica:
	\begin{equation*}
		dim(V_{\lambda_i}) \leq a_i
	\end{equation*}
\end{theorem}

\begin{theorem}[Criterio delle molteplicit\`a algebrica e geometrica]
	Dato un endomorfismo $T : V \to V$ di uno spazio vettoriale $V$ su
	$\mathbb{K}$, siano $\lambda_1, \dots, \lambda_k$ gli autovalori di $T$ in
	$\mathbb{K}$. Allora $T$ \`e diagonalizzabile se e solo se $P_T$ si fattorizza
	come prodotto di fattori lineari e, per ogni autovalore $\lambda_i$, la sua
	molteplicit\`a algebrica e quella geometrica sono uguali.
\end{theorem}

\begin{example}
	Prendiamo per semplicit\`a l'endomorfismo che abbiamo considerato nel capitolo precedente
	\[
		T \begin{pmatrix} x \\ y \end{pmatrix} =
		\begin{pmatrix}
			x + 2y \\
			-y
		\end{pmatrix}
	\]
	Ripetendo i soliti passaggi ottengo due autovalori: $t = 1$ e $t = -1$.
	I due valori azzerano il polinomio caratteristico una sola volta dunque hanno
	molteplicit\`a algebrica uguale a 1.
	\[ m_a(1) = 1 \quad m_a(-1) = 1 \]
	Sempre nell'esempio precedente avevamo anche trovato gli autospazi relativi ai due
	autovalori.
	\[
		V_1 = < \begin{pmatrix} 1 \\ 0 \end{pmatrix} > \quad
		V_2 = < \begin{pmatrix} -1 \\ 1 \end{pmatrix} >
	\]
	Quindi sia $V_1$ che $V_2$ hanno dimensione 1. Notiamo ora che 
	\begin{gather*}
		m_a(1) = 1 \quad m_g(1) = 1 \\
		m_a(-1) = 1 \quad m_g(-1) = 1
	\end{gather*}
	Possiamo quindi concludere che l'endomorfismo \`e diagonalizzabile. Stavolta per\`o
	completiamo l'esercizio. La base "buona" di cui si parlava, in questo caso \`e
	\[
		\mathcal{B} = \left\{
		\begin{pmatrix} 1 \\ 0 \end{pmatrix}, \quad
		\begin{pmatrix} -1 \\ 1	\end{pmatrix}
		\right\}
	\]
	Scriviamo quindi la matrice
	\[
		[T]_{\substack{\mathcal{B} \\ \mathcal{B}}} = 
		\begin{pmatrix}
			1 & 0  \\
			0 & -1
		\end{pmatrix}
	\]
	Come previsto abbiamo una matrice diagonale con gli autovalori sulla diagonale principale
	ognuno ripetuto il numero di volte indicato dalla sua molteplicit\`a algebrica (in questo
	caso \`e 1 per entrambi quindi compaiono una volta sola).
\end{example}