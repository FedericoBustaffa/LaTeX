\chapter{Riduzione a scalini}
\section{Operazioni elementari sulle colonne}
Consideriamo una generica matrice in $Mat_{m \times n}(\mathbb{K})$:
\begin{equation*}
	A = \begin{pmatrix}
		a_{11} & a_{12} & \dots & a_{1n} \\
		a_{21} & a_{22} & \dots & \dots  \\
		\dots  & \dots  & \dots & \dots  \\
		a_{m1} & \dots  & \dots & a_{mn}
	\end{pmatrix}
\end{equation*}
e i tre seguenti tipi di mossa sulle colonne, detti anche
\textbf{operazioni elementari sulle colonne}:
\begin{enumerate}
	\item si somma alla colonna $i$ la colonna $j$ moltiplicata per uno scalare
	      $\lambda$.
	\item si moltiplica la colonna $i$ per uno scalare $\lambda$
	\item si scambiano fra di loro due colonne $i$ e $j$.
\end{enumerate}

\begin{defn}
	La \textbf{profondit\`a} di una colonna \`e definita come la posizione
	occupata (contata dal basso) dal suo pi\`u alto coefficiente diverso da 0.
	Alla colonna nulla si assegna per convenzione profondit\`a uguale a 0.
\end{defn}

\begin{example}
	Consideriamo la seguente matrice:
	\begin{equation*}
		A = \begin{pmatrix}
			0            & 4 - \sqrt{3} & 0  \\
			\sqrt{3} + 1 & 0            & 0  \\
			-2           & -2           & -2
		\end{pmatrix}
	\end{equation*}
	In questo caso la prima colonna ha profodit\`a uguale a 2, la
	seconda colonna uguale a 3 mentre la terza ha profondit\`a uguale a 1.
\end{example}

\begin{defn}
	Una matrice $A \in Mat_{m \times n}(\mathbb{K})$, si dice
	\textbf{in forma a scalini per colonne} se rispetta le seguenti propriet\`a:
	\begin{itemize}
		\item leggendo la matrice da sinistra a destra, le colonne non nulle si
		      incontrano tutte prima delle colonne nulle.
		\item leggendo la matrice da sinistra a destra, le profondit\`a
		      delle sue colonne non nulle risultano strettamente
		      decrescenti.
	\end{itemize}
\end{defn}

\begin{example}
	Questi sono esempi di matrici in forma \emph{a scalini}:
	\begin{equation*}
		A = \begin{pmatrix}
			1            & 0           & 0 & 0 \\
			\sqrt{3} + 1 & 1           & 0 & 0 \\
			-2           & \frac{5}{2} & 1 & 0
		\end{pmatrix}
	\end{equation*}
	\begin{equation*}
		B = \begin{pmatrix}
			1  & 0           & 0 & 0 \\
			0  & 1           & 0 & 0 \\
			-2 & \frac{5}{2} & 0 & 0
		\end{pmatrix}
	\end{equation*}
\end{example}

\begin{defn}
	In una matrice in forma a scalini per colonna, i coefficienti diversi
	da zero pi\`u alti di posizione di ogni colonna non nulla si
	chiamano \textbf{pivot}.
\end{defn}

\begin{theorem}
	Data una matrice $A \in Mat_{m \times n}(\mathbb{K})$ \`e sempre
	possibile, usando operazioni elementari sulle colonne, ridurre la
	matrice in forma a scalini per colonne.
\end{theorem}

\begin{observation}
	Quando si riduce una matrice in forma a scalini, la forma a scalini
	ottenuta non \`e unica.
\end{observation}

\begin{defn}
	Una matrice $A \in Mat_{m \times n}(\mathbb{K})$, si dice \textbf{in forma a
		scalini per colonne ridotta} se:
	\begin{itemize}
		\item $A$ \`e a scalini per colonne.
		\item Tutte le entrate nella stessa riga di un pivot, precedenti al pivot,
		      sono nulle
	\end{itemize}
\end{defn}

\begin{example}
	Esempio di matrice in forma a scalini ridotta:
	\begin{equation*}
		\begin{pmatrix}
			1            & 0           & 0 & 0 \\
			\sqrt{3} + 1 & 1           & 0 & 0 \\
			-2           & \frac{5}{2} & 1 & 0
		\end{pmatrix} \rightarrow
		\begin{pmatrix}
			1 & 0 & 0 & 0 \\
			0 & 1 & 0 & 0 \\
			0 & 0 & 1 & 0
		\end{pmatrix}
	\end{equation*}
\end{example}

\begin{proposition}
	Data una matrice in forma a scalini per colonne, \`e sempre possibile, usando
	solo la prima delle operazioni elementari sulle colonne, portare $A$ in forma
	a scalini ridotta.
\end{proposition}

\begin{corollary}
	Ogni matrice $A$ pu\`o essere trasformata, tramite le operazioni elementari
	sulle colonne, in una matrice in forma a scalini per colonne ridotta.
\end{corollary}

\begin{proposition}
	Se operiamo attraverso le operazioni elementari sulle colonne, lo
	$Span$ dei vettori colonna rimane invariato. Ovvero se indichiamo con
	$v_1, \dots, v_n$ i vettori colonna di una matrice
	$A \in Mat_{m \times n}(\mathbb{K})$, per ogni matrice $A'$ ottenuta da $A$
	attraverso le operazioni elementari sulle colonne, si ha, indicando con
	$w_1, \dots, w_n$ i vettori colonna di $A'$, che:
	\begin{equation*}
		Span(v_1, \dots, v_n) = Span(w_1, \dots, w_n)
	\end{equation*}
\end{proposition}

\section{Riduzione a scalini e studio delle basi}

\begin{theorem}
	Sia $V$ uno spazio vettoriale su $\mathbb{K}$ che ammette una base
	finita, allora tutte le basi di $V$ hanno la stessa cardinalit\`a.
\end{theorem}

\begin{corollary}
	In uno spazio vettoriale $V$ di dimensione $n$, dati $n$ vettori
	linearmente indipendenti questi sono anche una base di $V$. Allo
	stesso modo, dati $n$ vettori che generano $V$, questi sono anche una
	base di $V$.
\end{corollary}

Considerzioni simili a quelle fatte fino ad ora ci consentono di descrivere
un criterio concreto per decidere se, dato uno spazio vettoriale $V$ di
dimensione $n$ ed una base $e_1, \dots, e_n$ di $V$, e dati $n$ vettori
$v_1, \dots, v_n$ di $V$, tali vettori costituiscono una base di $V$ o no.

Il criterio \`e espresso dai seguenti punti:
\begin{enumerate}
	\item Esprimiamo i vettori $v_1, \dots, v_n$ trovandone i coefficienti rispetto
	      alla base $e_1, \dots, e_n$.
	\item Poniamoli in colonna uno accanto all'altro. Cos\`i facendo otteniamo una
	      matrice $M$ che \`e $n \times n$.
	\item Riduciamo $M$ in forma a scalini ridotta ottenendo $M'$.
	\item A questo punto, se $M'$ \`e l'identit\`a, allora $\{v_1, \dots, v_n\}$
	      \`e una base di $V$, altrimenti no.
\end{enumerate}

\begin{example}
	Verificare che i seguenti vettori siano una base di $\mathbb{R}^4$.
	\begin{equation*}
		v_1 = \begin{pmatrix}
			1 \\ 1 \\ 0 \\ 0
		\end{pmatrix} \quad
		v_2 = \begin{pmatrix}
			0 \\ 1 \\ 1 \\ 0
		\end{pmatrix} \quad
		v_3 = \begin{pmatrix}
			0 \\ 0 \\ 1 \\ 1
		\end{pmatrix} \quad
		v_4 = \begin{pmatrix}
			0 \\ 0 \\ 0 \\ 1
		\end{pmatrix}
	\end{equation*}
	Verifichiamo col nuovo metodo che si tratta di una base.
	Scriviamo dunque la matrice
	\begin{equation*}
		\begin{pmatrix}
			1 & 0 & 0 & 0 \\
			1 & 1 & 0 & 0 \\
			0 & 1 & 1 & 0 \\
			0 & 0 & 1 & 1 \\
		\end{pmatrix}
	\end{equation*}
	Portandola in forma a scalini ridotta diventa
	\begin{equation*}
		\begin{pmatrix}
			1 & 0 & 0 & 0 \\
			0 & 1 & 0 & 0 \\
			0 & 0 & 1 & 0 \\
			0 & 0 & 0 & 1
		\end{pmatrix}
	\end{equation*}
	Dunque $\{v_1, v_2, v_3, v_4\}$ \`e una base di $\mathbb{R}^4$.
\end{example}

\begin{theorem}[Completamento]
	Dato uno spazio vettoriale $V$ di dimensione $n$, ogni
	sottoinsieme $B = \{v_1, \dots, v_k\} \subset V$ di vettori
	linearmente indipendenti di cardinalit\`a $k$, con
	$1 \leq k \leq n$, pu\`o essere completato ad una base di $V$
	aggiungendo a $B$ $n - k$ vettori di $V \backslash Span(B)$.
	\begin{proof}
		La dimostrazione che segue ci fornisce un algoritmo per trovare
		vettori da aggiungere al sottoinsieme $B$ per riuscire a trovare
		una base di $V$.
		\begin{enumerate}
			\item Si scrivono i vettori $v_1, \dots, v_k$ come
			      vettori colonna rispetto a una base data, e si considera
			      la matrice $M$ che ha questi vettori come colonne.
			\item Si riduce $M$ in forma a scalini.
			\item Tutte le volte le volte che troviamo uno scalino lungo
			      (altezza $\geq 2$, dove l'altezza \`e la differenza di profondit\`a
			      tra due colonne adiacenti che formano lo scalino lungo) possiamo
			      facilmente trovare $i - 1$ vettori $w_1, \dots, w_{i - 1}$ tali che
			      $\{v_1, \dots, v_k, w_1, \dots, w_{i - 1}\}$ \`e ancora un
			      insieme di vettori linearmente indipendenti.
			\item Supponiamo infatti che $M$ abbia uno scalino di lunghezza $i$,
			      ovvero in una certa colonna abbia il pivot alla riga $t$ e nella
			      colonna successiva alla riga $t+i$, e sia $i > 1$. Allora
			      per $j$ che varia tra 1 ed $i - 1$.
			\item Scegliere un vettore $w_j$ tale che abbia tutti
			      0 tranne un 1 in corrispondeza della riga $(t + j)$-esima.
			      \`E facile osservare che ogni $w_j$ non appartiene
			      allo $Span(v_1, \dots, v_k, w_1, \dots, w_{j-1})$. Dunque ad
			      ogni aggiunta di $w_j$, l'insieme
			      $\{v_1, \dots, v_k, w_1, \dots, w_j\}$ rimane un insieme di
			      vettori linearmente indipendenti di $V$.
			\item Ripetiamo questa aggiunta di vettori $w_k$ per ogni scalino
			      lungo che troviamo in $M$, trovando cos\`i alla fine $n$ vettori
			      linearmente indipendenti, dunque una base di $V$ come richiesto.
		\end{enumerate}
	\end{proof}
\end{theorem}

\begin{example}
	Illustriamo il metodo descritto con un esempio. Supponiamo che
	$V = \mathbb{R}^7$ e siano dati i 4 vettori linearmente
	indipendenti che, scritti rispetto alla base standard di
	$\mathbb{R}^7$, sono rappresentati come segue:
	\begin{equation*}
		v_1 = \begin{pmatrix}
			1 \\ 1 \\ 0 \\ 2 \\ 1 \\ 0 \\ 1
		\end{pmatrix} \quad
		v_2 = \begin{pmatrix}
			0 \\ 1 \\ 0 \\ 2 \\ 1 \\ 0 \\ 3
		\end{pmatrix} \quad
		v_3 = \begin{pmatrix}
			0 \\ 0 \\ 0 \\ 0 \\ 1 \\ 0 \\ 1
		\end{pmatrix} \quad
		v_4 = \begin{pmatrix}
			0 \\ 0 \\ 0 \\ 0 \\ 1 \\ 0 \\ 2
		\end{pmatrix}
	\end{equation*}
	La matrice $M$ in questo caso \`e:
	\begin{equation*}
		M = \begin{pmatrix}
			1 & 0 & 0 & 0 \\
			1 & 1 & 0 & 0 \\
			0 & 0 & 0 & 0 \\
			2 & 2 & 0 & 0 \\
			1 & 1 & 1 & 1 \\
			0 & 0 & 0 & 0 \\
			1 & 3 & 1 & 2
		\end{pmatrix}
	\end{equation*}
	Riduciamola a scalini per colonne.
	\begin{equation*}
		M' = \begin{pmatrix}
			1 & 0 & 0 & 0 \\
			0 & 1 & 0 & 0 \\
			0 & 0 & 0 & 0 \\
			0 & 2 & 0 & 0 \\
			1 & 0 & 1 & 0 \\
			0 & 0 & 0 & 0 \\
			0 & 1 & 2 & 1
		\end{pmatrix}
	\end{equation*}
	Il primo scalino lungo ha altezza 3. Come osservato nella dimostrazione
	del teorema del completamento, i vettori
	\begin{equation*}
		w_1 = \begin{pmatrix}
			0 \\ 0 \\ 1 \\ 0 \\ 0 \\ 0 \\ 0
		\end{pmatrix} \quad
		w_2 = \begin{pmatrix}
			0 \\ 0 \\ 0 \\ 1 \\ 0 \\ 0 \\ 0
		\end{pmatrix}
	\end{equation*}
	non appartengono al sottospazio generato dalle colonne di $M'$, e si
	osserva immediatamente che $\{v_1, v_2, v_3, v_4, w_1, w_2\}$ \`e un
	insieme di vettori linearmente indipendenti di $\mathbb{R}^7$.

	Similmente prendendo in considerazione il secondo scalino lungo, notiamo
	che il vettore
	\begin{equation*}
		w_3 = \begin{pmatrix}
			0 \\ 0 \\ 0 \\ 0 \\ 0 \\ 1 \\ 0
		\end{pmatrix}
	\end{equation*}
	non appartiene al sottospazio generato da $v_1, v_2, v_3, v_4, w_1, w_2$.

	A questo punto abbiamo ottenuto l'insieme di vettori linearmente indipendenti
	$\{v_1, v_2, v_3, v_4, w_1, w_2, w_3\}$ di $\mathbb{R}^7$ e dato che
	$\mathbb{R}^7$ ha dimensione 7 abbiamo ottenuto una base di $\mathbb{R}^7$.
\end{example}


\section{Il teorema della dimensione del nucleo e dell'immagine di una applicazione lineare}
Il teorema del completamento, ha come importante corollario un risultato che stabilisce
una relazione tra la dimensione del nucleo e quella dell'immagine di una applicazione
lineare.

\begin{theorem}
	Considerata una applicazione lineare $L : V \to W$, dove $V$ e $W$ sono spazi
	vettoriali su $\mathbb{K}$, vale
	\begin{equation*}
		dim(Ker(L)) + dim(Imm(L)) = dim(V)
	\end{equation*}
\end{theorem}

\begin{defn}
	Una applicazione lineare bigettiva $L : V \to W$, tra due spazi vettoriali $V$ e $W$
	sul campo $\mathbb{K}$, si dice un \textbf{isomorfismo lineare}.

	Dal teorema precedente segue che:
	\begin{itemize}
		\item Se $L : V \to W$ \`e una applicazione lineare iniettiva allora
		      \[ dim(Imm(L)) = dim(V) \]
		      Infatti sappiamo che $Ker(L) = \{O\}$ dunque $dim(Ker(L)) = 0$.
		\item Se $L : V \to W$ \`e un isomorfismo lineare allora \[dim(V) = dim(W)\]
		      Infatti se $L$ \`e bigettiva, in particolare \`e iniettiva e surgettiva.
		\item Se $L : V \to W$ \`e una applicazione lineare iniettiva, allora $L$,
		      pensata come applicazione da $V$ ad $Imm(L)$, \`e un isomorfismo lineare.
	\end{itemize}
\end{defn}

\section{Immagine di un'applicazione lineare}
La riduzione a scalini per colonna risulta molto utile anche per lo
studio di applicazioni lineari, ed in particolare per la
determinazione di dimensione e base dell'immagine di una applicazione
lineare.

Dati due spazi vettoriali $V$ e $W$ sul campo $\mathbb{K}$ di dimensione $n$
e $m$ rispettivamente, consideriamo una applicazione lineare:
\begin{equation*}
	L : V \to W
\end{equation*}
Fissiamo una base $\{e_1, e_2, \dots, e_n\}$ di $V$ e una base
$\{\epsilon_1, \epsilon_2, \dots, \epsilon_m\}$ di $W$. Indichiamo con $[L]$
la matrice, di forma $m \times n$, associata a $L$ nelle basi scelte.

Per quanto abbiamo fin qui detto, possiamo, tramite un numero finito $k$ di
operazioni elementari sulle colonne di $[L]$, portarla in forma a scalini
ridotta. Ma c'\`e di pi\`u. Ogni operazione elementare sulle colonne corrisponde
a moltiplicare la matrice iniziale $[L] \in Mat_{m \times n}(\mathbb{K})$,
a destra, per una matrice $B$ di dimensioni $n \times n$ invertibile.

\begin{theorem}
	Siano $V, W, U$ spazi vettoriali su $\mathbb{K}$, fissiamo per ciascuno
	una base. Siano $T : V \to W$ e $S : W \to U$ applicazioni lineari. Allora,
	rispetto alle basi fissate, vale:
	\begin{equation*}
		[S \circ T] = [S][T]
	\end{equation*}
	dove nel membro di destra stiamo considerando il prodotto righe per colonne
	fra matrici.
\end{theorem}

Tornando all'applicazione lineare $L$, sappiamo già che lo span delle colonne
di $[L]$ coincide con lo span delle colonne della matrice ottenuta portando $[L]$
in forma a scalini. Sappiamo cio\`e che $Imm(L)$ coincide con l'immagine
dell'applicazione lineare associata alla matrice ottenuta portando i forma a
scalini $[L]$.

\begin{proposition}
	Siano $L$ ed $M$ due applicazioni lineari, vale
	\[ Imm(L \circ M) = Imm(L) \]
	ossia, scritto con un'altra notazione,
	\[ (L \circ M)(V) = L(V) \]
\end{proposition}

\begin{proposition}
	Sia $B : V \to V$ un'applicazione lineare invertibile, allora vale
	\[ Imm(L) = Imm(L \circ B) \]
	\begin{proof}
		Dato che $B$ \`e una funzione invertibile, \`e bigettiva, ossia
		\[ B(V) = V \]. Dunque
		\begin{equation*}
			Imm(L \circ B) = L(B(V)) = L(V)
		\end{equation*}
	\end{proof}
\end{proposition}

\begin{defn}
	Data una applicazione lineare $L : V \to W$, dove $V$ e $W$ sono due spazi
	vettoriali di dimensione finita sul campo $\mathbb{K}$ e sia $[L]$ la matrice
	associata ad $L$. Se riduco in forma a scalini $[L]$, il \textbf{rango} equivale
	al numero di pivot della matrice ottenuta.
\end{defn}

\begin{theorem}
	Data una applicazione lineare $L$ come sopra e fissate le basi, vale che
	il rango di $L$ \`e uguale al numero di colonne non nulle che si trovano
	quando si trasforma $[L]$ in forma a scalini.
\end{theorem}

\begin{observation}
	Osserviamo che il rango di una applicazione lineare $L$ \`e anche uguale al
	\textbf{massimo numero di colonne linearmente indipendenti} di $[L]$.
	Infatti sappiamo che $Imm(L)$ \`e il sottospazio vettoriale di $W$ generato
	dai vettori colonna di $[L]$. Da questi vettori \`e possibile estrarre una
	base di $Imm(L)$ e, inoltre, possiamo dire che \[dim(Imm(L)) = rango (L)\]
\end{observation}

\begin{observation}
	Possiamo ora definire un algoritmo in 3 passi che, data una
	applicazione lineare tra due spazi vettoriali $V$ e $W$, e fissata una base
	$\{e_1, e_2, \dots, e_n\}$ di $V$ e una base
	$\{\epsilon_1, \epsilon_2, \dots, \epsilon_m\}$ di $W$, ci permette di
	determinare la dimensione ed una base di $Imm(L)$:
	\begin{enumerate}
		\item Scrivere la matrice $[L]$ associata ad $L$ rispetto alle basi
		      fissate
		\item Ridurre $[L]$ in forma a scalini
		\item Contare il numero di pivot per ottenere la dimensione di $Imm(L)$.
	\end{enumerate}
\end{observation}

\section{Riduzione a scalini per righe}
Data una matrice in $Mat_{m \times n}(\mathbb{K})$ \`e possibile definire le
operazioni elementari di riga in modo analogo alle mosse di colonna in questo
modo:
\begin{enumerate}
	\item Sommare alla riga $i$ la riga $j$ moltiplicata per uno scalare $\lambda$
	\item Moltiplicare la riga $i$ per uno scalare $\lambda$
	\item Permutare fra loro due righe, diciamo $i$ e $j$
\end{enumerate}
A questo punto possiamo definire la forma a scalini per righe di una matrice.
In questo caso chiameremo \emph{profondit\`a} di una riga la posizione occupata,
contata da destra, dal suo coefficiente diverso da zero che sta pi\`u a sinistra
nella riga. La riga nulla ha \emph{profondit\`a} 0.

\begin{defn}
	Una matrice $A$ in $Mat_{m \times n}(\mathbb{K})$, si dice \textbf{in forma
		a scalini per righe} se:
	\begin{itemize}
		\item Leggendo la matrice dall'alto verso il basso, le righe non nulle si
		      incontrano tutte prima delle righe nulle.
		\item Leggendo la matrice dall'alto verso il basso, le profondit\`a
		      delle sue righe non nulle risultano strettamente decrescenti.
	\end{itemize}
\end{defn}

\begin{theorem}
	Data una matrice $A$ in $Mat_{m \times n}(\mathbb{K})$ \`e sempre possibile,
	usando un numero finito di operazioni elementari sulle righe, ridurre la
	matrice in forma a scalini per righe.
\end{theorem}

In particolare, anche quando abbiamo una matrice in forma a scalini per righe, si
possono definire i \textbf{pivot} della matrice, come i coefficienti pi\`u a
sinistra delle righe non nulle.

Inoltre anche in questo caso \`e possibile definire una forma a scalini
\emph{particolare}: la forma a \textbf{scalini per righe ridotta}.

\begin{example}
	Matrice in forma a scalini per righe
	\begin{equation*}
		\begin{pmatrix}
			1 & 0 & 0 & -20 \\
			0 & 1 & 0 & 0   \\
			0 & 0 & 1 & 5
		\end{pmatrix}
	\end{equation*}
\end{example}

\begin{corollary}
	Ogni matrice $A$ pu\`o essere trasformata, attraverso le operazioni elementari
	sulle righe, in una matrice in forma a scalini per righe ridotta.
\end{corollary}

In generale valgono tutte le propriet\`a gi\`a elencate per le operazioni sulle
colonne.

\section{Riduzione a scalini e applicazioni lineari}

\begin{theorem}
	Data una applicazione lineare $L : V \to W$ tra due spazi vettoriali $V$ e
	$W$ sul campo $\mathbb{K}$, e data la matrice $[L]$ associata a $L$ rispetto
	a due basi fissate di $V$ e $W$:
	\begin{enumerate}
		\item Il massimo numero di righe linearmente indipendenti di $[L]$ \`e
		      uguale al massimo numero di colonne linearmente indipendenti di
		      $[L]$, ossia al rango di $L$.
		\item Se si riduce la matrice $[L]$ a scalini, sia che lo si faccia
		      per righe, sia che lo si faccia per colonne, il numero di scalini
		      sar\`a sempre uguale al rango di $L$.
	\end{enumerate}
\end{theorem}

\begin{observation}
	Se componiamo $L$ a destra o a sinistra per una	applicazione invertibile,
	il rango non cambia. Dunque, se moltiplichiamo $[L]$ a destra o a sinistra
	per matrici invertibili, anche il rango della matrice prodotto non cambia.
\end{observation}

\begin{proposition}
	Se $A$ \`e una matrice $m \times n$ a valori su un campo $\mathbb{K}$, ed
	indichiamo con $e_1, \dots, e_n$ le sue colonne, e $B$ \`e una riduzione
	a scalini per righe di $A$, allora le colonne di $A$ in corrispondenza alla
	posizione dei pivot di $B$ formano una base dello $Span$ delle colonne di $A$
	($Span(e_1, \dots, e_n)$).
\end{proposition}

\begin{observation}
	A partire dalla proposizione precedente possiamo ricavare un algoritmo per
	estrarre una base di uno spazio vettoriale da un insieme di generatori.

	Consideriamo $V$ uno spazio vettoriale di dimensione $n$ e del quale
	conosciamo una base $\{e_1, \dots, e_n\}$. Consideriamo $k$ vettori
	$v_1, \dots, v_k$ di $V$ e il sottospazio di $V$ generato da
	$Span(v_1, \dots, v_k)$. Se vogliamo estrarre una base di
	$Span(v_1, \dots, v_k)$ da $v_1, \dots, v_k$ seguiamo questi passaggi:
	\begin{enumerate}
		\item Scriviamo le coordinate dei $v_i$ rispetto alla base
		      $\{e_1, \dots, e_n\}$ e le	consideriamo come colonne di
		      una matrice $M$.
		\item Porto $M$ in forma a scalini per riga.
		\item I vari vettori $v_i$ con indice $i$ in corrispondenza dei pivot di $M'$
		      ($M$ in forma a scalini) formano una base di $Span(v_1, \dots, v_k)$,
		      estratta dall'insieme di generatori $\{v_1, \dots, v_k\}$.
	\end{enumerate}
\end{observation}