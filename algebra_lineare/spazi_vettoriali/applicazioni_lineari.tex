
% APPLICAZIONI LINEARI

\subsection{Applicazioni Lineari}
Le applicazioni lineari non sono altro che funzioni che mandano
sottospazi in sottospazi

\begin{example}
	Consideriamo la funzione $\textit{f} : \mathbb{R}^2 \to \mathbb{R}^2$
	definita da
	\begin{equation*}
		\textit{f}\left(
		\begin{pmatrix}
				x \\ y
			\end{pmatrix}
		\right) =
		\begin{pmatrix}
			x \\ x^2
		\end{pmatrix}
	\end{equation*}

	La funzione $f$ manda i punti
	$\begin{psmallmatrix}
			x \\ x
		\end{psmallmatrix}$,
	con la prima e seconda coordinata uguali, ovvero i punti della retta
	di equazione $x = y$, nella parabola di equazione $y = x^2$.
	Ma, come sappiamo, la retta $y = x$, passando dall'origine, \`e un
	sottospazio di $\mathbb{R}^2$, mentre la parabola non lo \`e.
	Si devono dunque considerare applicazioni con propriet\`a particolari.
\end{example}

\begin{defn}
	Siano $V$ e $W$ spazi vettoriali di dimensione finita sul campo
	$\mathbb{K}$. Un'applicazione $L$ da $V$ in $W$ \`e detta
	\textbf{lineare} se soddisfa le seguenti propriet\`a:
	\begin{itemize}
		\item $\forall v_1, v_2 \in V$ vale \[ L(v_1 + v_2) = L(v_1) + L(v_2) \]
		\item
		      $\forall \lambda \in \mathbb{K}$ e $\forall v \in V$
		      vale \[ L(\lambda v) = \lambda L(v) \]
	\end{itemize}
\end{defn}

\begin{observation}
	Soddisfare le due propriet\`a, da parte di un'applicazione lineare $L$,
	\`e equivalente a soddisfare la seguente propriet\`a:

	$\forall v_1, v_2 \in V$ e $\forall \lambda, \mu \in \mathbb{K}$ vale
	\begin{equation*}
		L(\lambda v_1 + \mu v_2) = \lambda L(v_1) + \mu L(v_2)
	\end{equation*}
\end{observation}

\begin{defn}
	Siano $V$ e $W$ spazi vettoriali su $\mathbb{K}$ e $L$ un'applicazione
	lineare da $V$ in $W$. Chiamo \textbf{immagine} di $L$, e la indico con
	$Imm(L)$, il seguente sottoinsieme di $W$:
	\begin{equation*}
		Imm(L) = \{w \in W \mid \forall v \in V, \quad L(v) = w\}
	\end{equation*}
\end{defn}

\begin{proposition}
	\`E dimostrabile che \[ Imm(L) = Span(L(e_1), \dots, L(e_n)) \]
	dove $\{e_1, \dots, e_n\}$ \`e una base di $V$.
\end{proposition}

\begin{defn}
	Siano $V$ e $W$ spazi vettoriali su $\mathbb{K}$ e $L$ un'applicazione
	lineare da $V$ in $W$. Chiamo \textbf{nucleo} di $L$, e lo indico con
	$Ker(L)$, il seguente sottoinsieme di $V$:
	\begin{equation*}
		Ker(L) = \{v \in V \mid L(v) = O\}
	\end{equation*}
\end{defn}

Di seguito qualche propriet\`a utile:
\begin{enumerate}
	\item $Ker(L)$ \`e un sottospazio vettoriale di $V$.
	\item $Imm(L)$ \`e un sottospazio vettoriale di $W$.
	\item $L$ \`e iniettiva se e solo se $Ker(L) = \{O\}$.
\end{enumerate}