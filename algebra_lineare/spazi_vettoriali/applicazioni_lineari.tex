\section{Applicazioni Lineari}
Le applicazioni lineari non sono altro che funzioni che mandano sottospazi in sottospazi.

\begin{example}
	Consideriamo la funzione $\textit{f} : \R^2 \to \R^2$ definita da
	\[
		f \begin{pmatrix}
			x \\ y
		\end{pmatrix}
		=
		\begin{pmatrix}
			x \\ x^2
		\end{pmatrix}
	\]
	La funzione $f$ manda il punto $(x, x)$, con la prima e seconda coordinata uguali, ovvero i punti della
	retta di equazione $x = y$, nella parabola di equazione $y = x^2$. Ma, come sappiamo, la retta $y = x$,
	passando dall'origine, è un sottospazio di $\R^2$, mentre la parabola non lo è. Si devono dunque
	considerare applicazioni con proprietà particolari.
\end{example}

\begin{definition}
	Siano $V$ e $W$ spazi vettoriali di dimensione finita sul campo $\K$. Un'applicazione $L$ da $V$
	in $W$ è detta \textbf{lineare} se soddisfa le seguenti proprietà:
	\begin{itemize}
		\item $\forall v_1, v_2 \in V$ vale
		      \[ L(v_1 + v_2) = L(v_1) + L(v_2) \]
		\item $\forall \lambda \in \K$ e $\forall v \in V$ vale
		      \[ L(\lambda v) = \lambda L(v) \]
	\end{itemize}
\end{definition}

\begin{observation}
	Soddisfare le due proprietà, da parte di un'applicazione lineare $L$, è equivalente a soddisfare la seguente
	proprietà: $\forall v_1, v_2 \in V$ e $\forall \lambda, \mu \in \K$ vale
	\[ L(\lambda v_1 + \mu v_2) = \lambda L(v_1) + \mu L(v_2) \]
\end{observation}

\begin{definition}
	Siano $V$ e $W$ spazi vettoriali su $\K$ e $L$ un'applicazione lineare da $V$ in $W$. Chiamo
	\textbf{immagine} di $L$, e la indico con $\Imm(L)$, il seguente sottoinsieme di $W$:
	\[ \Imm(L) = \{w \in W \mid \forall v \in V, \quad L(v) = w\} \]
\end{definition}

\begin{proposition}
	È dimostrabile che \[ \Imm(L) = Span(L(e_1), \dots, L(e_n)) \] dove $\{e_1, \dots, e_n\}$ è una base di $V$.
\end{proposition}

\begin{definition}
	Siano $V$ e $W$ spazi vettoriali su $\K$ e $L$ un'applicazione lineare da $V$ in $W$. Chiamo
	\textbf{nucleo} di $L$, e lo indico con $\Ker(L)$, il seguente sottoinsieme di $V$:
	\[ \Ker(L) = \{v \in V \mid L(v) = O\} \]
\end{definition}

Di seguito qualche proprietà utile:
\begin{enumerate}
	\item $\Ker(L)$ è un sottospazio vettoriale di $V$.
	\item $\Imm(L)$ è un sottospazio vettoriale di $W$.
	\item $L$ è iniettiva se e solo se $\Ker(L) = \{O\}$.
\end{enumerate}

