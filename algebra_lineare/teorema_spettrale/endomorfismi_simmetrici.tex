
\subsection{Endomorfismi simmetrici definiti positivi o negativi}
Come gi\`a visto in precedenza, un endomorfismo autoaggiunto \`e diagonalizzabile e
tutti i suoi autovalori sono reali.

\begin{defn}
	Sia $T : V \to V$ un endomorfismo autoaggiunto di uno spazio vettoriale $V$ di
	dimensione finita su $\mathbb{K}$ (con $\mathbb{K} = \mathbb{R}$ o
	$\mathbb{K} = \mathbb{C}$). Se tutti i suoi autovalori sono numeri reali $\geq 0$
	si dice che $T$ \`e \textbf{semidefinito positivo}. Se tutti i suoi autovalori
	sono numeri reali $\leq 0$ si dice che $T$ \`e \textbf{semidefinito negativo}.
	In particolare, se tutti i suoi autovalori sono numeri reali $> 0$ si dice che
	$T$ \`e \textbf{definito positivo}. Se invece tutti i suoi autovalori sono numeri
	reali $< 0$ si dice che $T$ \`e \textbf{definito negativo}.
\end{defn}

La definizione appena data si applica in particolare nel caso reale degli
endomorfismi simmetrici.

Consideriamo, per esempio, cosa accade nel caso in cui $V$ ha dimensione 2.
Ricordiamo innanzitutto che, dato un endomorfismo, sono ben definiti la sua traccia
e il suo determinante.

\begin{proposition}
	Sia $T : \mathbb{R}^2 \to \mathbb{R}^2$ un endomorfismo simmetrico. Vale che:
	\begin{enumerate}
		\item
		      $T$ \`e definito positivo se e solo se il suo determinante e la sua
		      traccia sono entrambi $> 0$.
		\item
		      $T$ \`e definito negativo se e solo se il suo determinante \`e $> 0$ e
		      la sua traccia \`e $< 0$.
	\end{enumerate}

	\begin{proof}
		Come sappiamo dal teorema spettrale, $T$ \`e diagonalizzabile. Siano
		$\lambda_1, \lambda_2$ i suoi autovalori. Fissata una base di autovettori,
		la matrice che rappresenta $T$ rispetto a tale base \`e:
		\begin{equation*}
			\begin{pmatrix}
				\lambda_1 & 0         \\
				0         & \lambda_2
			\end{pmatrix}
		\end{equation*}
		da cui si ricava subito che il determinante di $T$ \`e $\lambda_1 \lambda_2$
		e la traccia \`e $\lambda_1 + \lambda_2$. A questo punto si deve solo
		analizzare caso per caso i valori di $\lambda_1$ e $\lambda_2$ e valutare
		il segno di determinante e traccia.
	\end{proof}
\end{proposition}