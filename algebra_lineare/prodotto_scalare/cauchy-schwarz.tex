\section{La disuguaglianza di Cauchy-Schwarz}
Dato un prodotto scalare $\langle , \rangle$ su $V$, abbiamo il seguente
teorema:

\begin{theorem}[Disuguaglianza di Cauchy-Schwarz]
	Per ogni coppia di vettori $u, v \in V$ vale
	\[
		|\langle u, v \rangle| \leq \| u \| \| v \|
	\]
\end{theorem}

\begin{example}
	Nel caso in cui $V$ sia $\R^n$ col prodotto scalare standard, la
	disuguaglianza di Cauchy-Schwarz si traduce così. Dati
	\[
		u = \begin{pmatrix}
			a_1 \\ \dots \\ a_n
		\end{pmatrix},
		v = \begin{pmatrix}
			b_1, \dots, b_n
		\end{pmatrix}
	\]
	vale
	\[
		|a_1 b_1 + \cdots + a_n b_n| \leq
		(a_1^2 + \cdots + a_n^2)(b_1^2 + \cdots + b_n^2)
	\]
	che può essere pensata anche, svincolandosi dai vettori e dagli spazi
	vettoriali, come una disuguaglianza che riguarda due qualsiasi $n$-uple
	$a_1, \dots, a_n$ e $b_1, \dots, b_n$ di numeri reali.
\end{example}

\begin{example}
	Nel caso in cui $V$ sia lo spazio delle funzioni continue reali definite
	sull'intervallo $[0, 1]$ la disuguaglianza di Cauchy-Schwarz si traduce
	nella seguente importante disuguaglianza fra integrali:
	\[
		\left( \int_0^1 f(t)g(t)dt \right)^2 \leq
		\int_0^1 f(t)^2 dt \int_0^1 g(t)^2 dt
	\]
\end{example}

