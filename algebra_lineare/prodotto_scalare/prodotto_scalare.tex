
\section{Prodotti scalari e spazi euclidei}
\subsection{Prodotto scalare}
\begin{defn}
	Sia $V$ uno spazio vettoriale sul campo $\mathbb{K}$, dove
	$\mathbb{K} = \mathbb{R}^n$ o $\mathbb{K} = \mathbb{C}^n$. Il 
	\textbf{prodotto scalare} \`e una funzione che ad ogni coppia di vettori $u, v$
	appartenenti a $V$ associa lo scalare $\langle u, v \rangle \in \mathbb{K}$
	calcolato come segue:
	\[
		\langle u, v \rangle = u_1 v_1 + u_2 v_2 + ... + u_n v_n
	\]
	dove gli $u_i$ e $v_i$ sono le $i$-esime coordinate di $u$ e $v$.
	
	Il prodotto scalare gode inoltre delle seguenti propriet\`a:
	\begin{enumerate}
		\item $\langle au_1 + bu_2, v \rangle =
			      a \langle u_1, v \rangle + b \langle u_2, v \rangle$, per ogni
		      $u_1, u_2, v \in V$ e per ogni $a, b \in \mathbb{K}$.
		\item $\langle u, v \rangle = \overline{\langle v, u \rangle}$ per ogni
		      $u, v \in V$ (dove $\overline{\langle v, u \rangle}$ indica il numero
		      complesso coniugato di $\langle u, v \rangle$).
		\item per ogni $u \in V$ vale $\langle u, u \rangle \geq 0$ e
		      $\langle u, u \rangle = 0$ se e solo se $u = O$.
	\end{enumerate}
	Uno spazio vettoriale reale $V$ munito di un prodotto scalare si dice
	\textbf{spazio euclideo}.
\end{defn}

\begin{observation}
	Osserviamo che, per la seconda propriet\`a, il prodotto scalare
	$\langle u, u \rangle$ \`e sempre un numero reale, dunque ha senso la disuguaglianza
	che compare nella terza propriet\`a.
\end{observation}

\begin{defn}
	Per ogni vettore $u \in V$, scriveremo
	\begin{equation*}
		\| u \| = \sqrt{\langle u, u \rangle}
	\end{equation*}
	e diremo che $\| u \|$ \`e la \textbf{norma} di $u$.
\end{defn}

\begin{observation}
	Dalle propriet\`a (1) e (2) segue che
	\begin{equation*}
		\langle u, av_1 + bv_2 \rangle =
		\overline{a} \langle u, v_1 \rangle + \overline{b} \langle u, v_2 \rangle
	\end{equation*}
	Se il campo $\mathbb{K}$ \`e $\mathbb{R}$ tutti i simboli di coniugazione
	complessa possono essere ignorati.
\end{observation}

\begin{observation}
	Il prodotto scalare standard estende a tutti gli spazi $\mathbb{R}^n$ un concetto
	che in $\mathbb{R}^2$ e in $\mathbb{R}^3$ ci \`e gi\`a familiare. Si pu\`o
	facilmente verificare che in questi casi per esempio $\| u \|$ coincide con
	ci\`o che \`e chiamata lunghezza del vettore $u$ e che $\| u - v \|$ coincide
	con la distanza tra i vettori $u$ e $v$. Inoltre $\langle u, v \rangle = 0$ se
	e solo se $u$ e $v$ sono ortogonali fra loro. Potete anche gi\`a osservare in
	$\mathbb{R}^2$ che vale la seguente relazione:
	\begin{equation*}
		\langle u, v \rangle = \| u \| \| v \| \cos \theta
	\end{equation*}
	dove $\theta$ \`e l'angolo compreso fra i vettori $u$ e $v$.
\end{observation}

\begin{example}
	In $\mathbb{C}^n$, dati i vettori
	\begin{equation*}
		u = \begin{pmatrix}
			a_1 \\ \dots \\ a_n
		\end{pmatrix}, \quad
		v = \begin{pmatrix}
			b_1 \\ \dots \\ b_n
		\end{pmatrix}
	\end{equation*}
	scritti rispetto alla base standard, abbiamo il prodotto scalare
	\begin{equation*}
		\langle u, v \rangle = a_1 \overline{b_1} + \cdots + a_n \overline{b_n}
	\end{equation*}
\end{example}

\begin{example}
	In $\mathbb{R}^3$, dati i vettori 
	\[
		u = \begin{pmatrix}
			2 \\ 1 \\ 1
		\end{pmatrix}, \quad
		v = \begin{pmatrix}
			-1 \\ 0 \\ 3
		\end{pmatrix}
	\]
	Abbiamo che il prodotto scalare $\langle u, v \rangle$ \`e dato da 
	\[
		\langle u, v \rangle = 2 \cdot (-1) + 1 \cdot 0 + 1 \cdot 3 = 1
	\]
\end{example}