
\subsection{Sottospazi ortogonali}
Sia $V$, uno spazio vettoriale munito di prodotto scalare.

\begin{proposition}
	Sia $U$ un sottospazio vettoriale di $V$. Allora esiste una base
	ortonormale di $U$ che \`e un sottoinsieme di una base ortonormale di $V$.
\end{proposition}

\begin{theorem}
	Sia $U$ un sottospazio vettoriale di $V$. Allora $V$ si decompone come
	somma diretta di $U$ e di $U^\perp$:
	\begin{equation*}
		V = U \oplus U^\perp
	\end{equation*}
	In particolare vale che $dim(U^\perp) = dim(V) - dim(U)$.
\end{theorem}

\begin{observation}
	Consideriamo $V = \mathbb{R}^n$ munito del prodotto scalare standard. Dato
	un sottospazio $W$, con base $w_1, \dots, w_r$, se pensiamo a come si
	scrive il prodotto scalare ci rendiamo conto che i vettori $W^\perp$ sono
	esattamente le soluzioni del sistema
	\begin{equation*}
		\begin{pmatrix}
			a_1   & \dots & a_n   \\
			b_1   & \dots & b_n   \\
			\dots & \dots & \dots \\
			\dots & \dots & \dots
		\end{pmatrix}
		\begin{pmatrix}
			x_1 \\ x_2 \\ \dots \\ \dots \\ x_n
		\end{pmatrix} =
		\begin{pmatrix}
			0 \\ 0 \\ \dots \\ \dots \\ 0
		\end{pmatrix}
	\end{equation*}
\end{observation}