
\section{La formula di Grassmann}
\subsection{La formula di Grassmann}

Dati due sottospazi vettoriali $A$ e $B$ in $\mathbb{R}^3$ di dimensione 2,
di che dimensione pu\`o essere la loro intersezione ?

Possono intersecarsi lungo una retta: in tal caso si nota che il sottospazio
generato dai vettori di $A \cup B$, ossia $A + B$, \`e tutto $\mathbb{R}^3$.

Oppure vale $A = B$: allora la loro intersezione \`e uguale ad $A$ (e a $B$) e
ha dimensione 2, e anche il sottospazio $A + B$ coincide con $A$.

In entrambi i casi, la somma delle dimensioni di $A \cap B$ e di $A + B$ \`e
sempre uguale a 4.

E se in $\mathbb{R}^4$ consideriamo un piano $C$ e un sottospazio $D$ di
dimensione 3?
Possono darsi tre casi per l'intersezione: $C \cap D = \{O\}$,
$dim(C \cap D) = 1$, $C \cap D = C$.

Qualunque sia il caso si verifica sempre che
\begin{equation*}
	dim(C \cap D) + dim(C + D) = 5 = dim(C) + dim(D)
\end{equation*}

In generale vale la formula
\begin{equation*}
	dim(A \cap B) + dim(A + B) = dim(A) + dim(B)
\end{equation*}

Dati due spazi vettoriali $V$ e $W$ sul campo $\mathbb{K}$, sul loro prodotto
cartesiano $V \times W$, c'\`e una struttura naturale di spazio vettoriale, dove
la somma \`e definita da:
\begin{equation*}
	(v, w) + (v_1, w_1) = (v + v_1, w + w_1)
\end{equation*}
e il prodotto per scalare da:
\begin{equation*}
	\lambda(v, w) = (\lambda v, \lambda w)
\end{equation*}
Si verifica che, se $\{v_1, \dots, v_n\}$ \`e una base di $V$
e $\{w_1, \dots, w_m\}$ \`e una base di $W$, allora
$\{(v_1, O), \dots (v_n, O), (O, w_1), \dots, (O, w_m)\}$ \`e una base di
$V \times W$, che dunque ha dimensione $n + m = (dim(V)) + (dim(W))$.

\begin{theorem}[Grassmann]
	Dati due sottospazi $A, B$ di uno spazio vettoriale $V$ sul campo
	$\mathbb{K}$, vale
	\begin{equation*}
		dim(A) + dim(B) = dim(A \cap B) + dim(A + B)
	\end{equation*}
	\begin{proof}
		Consideriamo l'applicazione
		\begin{equation*}
			\Phi : A \times B \to V
		\end{equation*}
		definita da
		\[ \Phi((a, b)) = a - b \]
		Cosa sappiamo dire del nucleo di $\Phi$ ? Per definizione
		\begin{equation*}
			Ker(\Phi) = \{(a, b) \in A \times B \mid a - b = O\}
		\end{equation*}
		dunque
		\begin{equation*}
			Ker(\Phi) = \{(a, b) \in A \times B \mid a = b\}
		\end{equation*}
		che equivale a scrivere:
		\begin{equation*}
			Ker(\Phi) = \{(z, z) \in A \times B \mid z \in A \cap B\}
		\end{equation*}
		Si nota subito che la applicazione lineare
		\[ \theta : A \cap B \to Ker(\Phi) \]
		\`e iniettiva e surgettiva, dunque \`e un isomorfismo. Allora il suo dominio e
		il suo codominio hanno la stessa dimensione, ovvero
		\begin{equation*}
			dim(Ker(\Phi)) = dim(A \cap B)
		\end{equation*}
		Cosa sappiamo dire invece dell'immagine di $\Phi$ ? Per definizione
		\begin{equation*}
			Imm(\Phi) = \{a - b \mid a \in A, b \in B\}
		\end{equation*}
		Visto che $B$, come ogni spazio vettoriale, se contiene un elemento
		$b$ contiene anche il suo opposto $-b$, possiamo scrivere la seguente
		uguaglianza fra insiemi:
		\begin{equation*}
			\{ a - b \mid a \in A, b \in B \} =
			\{ a + b \in V \mid a \in A, b \in B \} =
			A + B
		\end{equation*}
		Dunque
		\begin{equation*}
			Imm(\Phi) = A + B
		\end{equation*}
		Sappiamo che:
		\begin{equation*}
			dim(A \times B) = dim(Ker(\Phi)) + dim(Imm(\Phi))
		\end{equation*}
		Questa formula, viste le osservazioni fatte fin qui, si traduce come:
		\begin{equation*}
			dim(A) + dim(B) = dim(A \cap B) + dim(A + B)
		\end{equation*}
	\end{proof}
\end{theorem}