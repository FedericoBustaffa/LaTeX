
\subsection{Calcolo dell'intersezione di due sottospazi}
Consideriamo due sottospazi, $U$ e $W$, di $V$. Se entrambi sono presentati
come l'insieme delle soluzioni di un sistema \`e facile calcolare $U \cap W$:
basta calcolare le soluzioni del sistema 'doppio', ottenuto considerando tutte
le equazioni dei due sistemi.

Per esempio se $U$ e $W$ in $\mathbb{R}^4$ sono dati rispettivamente dalle
soluzioni dei sistemi $S_U$:
\begin{equation*}
	\begin{cases}
		3x + 2y + 4w = 0 \\
		2x + y + z + w = 0
	\end{cases}
\end{equation*}
e $S_W$:
\begin{equation*}
	\begin{cases}
		x + 2y + z + w = 0 \\
		x + z + w = 0
	\end{cases}
\end{equation*}
allora $U \cap W$ \`e dato dalle soluzioni del sistema:
\begin{equation*}
	\begin{cases}
		3x + 2y + 4w = 0   \\
		2x + y + z + w = 0 \\
		x + 2y + z + w = 0 \\
		x + z + w = 0
	\end{cases}
\end{equation*}

\begin{observation}
	Visto che $U$ ha dimensione 2, un sistema le cui soluzioni coincidono con
	l'insieme $U$ deve avere almeno 3 equazioni.
\end{observation}

Come calcolare per\`o $U \cap W$ se i due sottospazi sono presentati come span
di certi vettori ? Consideriamo per esempio $U$ e $W$ in $\mathbb{R}^5$
definiti cos\`i:
\begin{gather*}
	U = <\begin{pmatrix}
		1 \\ 2 \\ 3 \\ -1 \\ 2
	\end{pmatrix},
	\begin{pmatrix}
		2 \\ 4 \\ 7 \\ 2 \\ -1
	\end{pmatrix}> \\
	W = <\begin{pmatrix}
		1 \\ 2 \\ 0 \\ -2 \\ -1
	\end{pmatrix},
	\begin{pmatrix}
		0 \\ 1 \\ 1 \\ -1 \\ -1
	\end{pmatrix},
	\begin{pmatrix}
		0 \\ 1 \\ -3 \\ -6 \\ 1
	\end{pmatrix}>
\end{gather*}

Un metodo per calcolare $U \cap W$ \`e quello di esprimere $U$ e $W$ come
soluzioni di un sistema lineare. Cominciamo da $U$.

Per prima cosa si scrive la matrice:
\begin{equation*}
	\begin{pmatrix}
		1  & 2  & x_1 \\
		2  & 4  & x_2 \\
		3  & 7  & x_3 \\
		-1 & 2  & x_4 \\
		2  & -1 & x_5
	\end{pmatrix}
\end{equation*}
Ora riduciamo la matrice (senza incognite) a scalini per righe
\begin{equation*}
	\begin{pmatrix}
		1 & 2 & x_1                 \\
		0 & 1 & x_3 - 3x_1          \\
		0 & 0 & 2x_1 - x_2          \\
		0 & 0 & 13x_1 - 4x_3 + x_4  \\ 
		0 & 0 & -17x_1 + 5x_3 + x_5
	\end{pmatrix}
\end{equation*}
Tale matrice ha rango 2 se e solo se i coefficienti $x_1, x_2, x_3, x_4, x_5$ soddisfano 
il sistema 
\begin{equation*}
	\begin{cases}
		2x_1 - x_2          & = 0 \\ 
		13x_1 - 4x_3 + x_4  & = 0 \\ 
		-17x_1 + 5x_3 + x_5 & = 0
	\end{cases}
\end{equation*}

Ora dobbiamo fare la stessa cosa con $W$. Scriviamo quindi la matrice 
\begin{equation*}
	\begin{pmatrix}
		1  & 0  & 0  & x_1 \\ 
		2  & 1  & 1  & x_2 \\ 
		0  & 1  & -3 & x_3 \\ 
		-2 & -1 & -6 & x_4 \\ 
		-1 & -1 & 1  & x_5
	\end{pmatrix}
\end{equation*}
e riduciamola a scalini per righe 
\begin{equation*}
	\begin{pmatrix}
		1 & 0 & 0 & x_1                       \\ 
		0 & 1 & 1 & x_2 - 2x_1                \\ 
		0 & 0 & 2 & x_5 - x_1 + x_2           \\ 
		0 & 0 & 0 & x_3 + x_2 + 2x_5          \\ 
		0 & 0 & 0 & 2x_4 + 7x_2 + 5x_5 - 5x_1
	\end{pmatrix}
\end{equation*}
La matrice ha rango 3 se e solo se i coefficienti $x_1, x_2, x_3, x_4, x_5$ soddisfano il 
sistema 
\begin{equation*}
	\begin{cases}
		x_3 + x_2 + 2x_5          & = 0 \\ 
		2x_4 + 7x_2 + 5x_5 - 5x_1 & = 0 \\ 
	\end{cases}
\end{equation*}

Uniamo i due sistemi ottenuti e otteniamo 
\begin{equation*}
	\begin{cases}
		2x_1 - x_2                & = 0 \\ 
		13x_1 - 4x_3 + x_4        & = 0 \\ 
		17x_1 - 5x_3 - x_5        & = 0 \\
		x_2 + x_3 + 2x_5          & = 0 \\
		5x_1 - 7x_2 - 2x_4 - 5x_5 & = 0
	\end{cases}
\end{equation*}
Se risolviamo questo sistema otteniamo una base di $U \cap W$. Per verificare che i calcoli 
siano corretti basta vedere se la dimensione risulta uguale a quella prevista dalla formula 
di Grassmann.