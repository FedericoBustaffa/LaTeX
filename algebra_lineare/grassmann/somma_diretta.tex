
\subsection{Somma diretta di sottospazi}
Si dice che due sottospazi $U$ e $W$ di uno spazio vettoriale $V$ sono in
\textbf{somma diretta} se vale che \[ U \cap W = \{O\} \] In questo caso,
come sappiamo dalla formula di Grassmann, la dimensione di $U + W$ \`e "la
massima possibile", ovvero \[ dim(U) + dim(W) \] Vale anche il
viceversa, ossia due sottospazi sono in somma diretta se e solo se
\[ dim(U + W) = dim(U) + dim(W) \] Quando siamo sicuri che $U + W$ \`e la somma di due
sottospazi che sono in somma diretta, al posto di $U + W$ possiamo scrivere:
\begin{equation*}
	U \oplus W
\end{equation*}
In particolare, per avere una base di $U \oplus W$ basta fare l'unione di una
base di $U$ con una base di $W$.

\begin{observation}
	Attenzione: un sottospazio vettoriale $U$ di $V$ che non \`e uguale a $V$
	possiede in generale molti complementari. Per esempio, in $\mathbb{R}^3$
	un piano passante per l'origine ha per complementare una qualunque retta
	passante per l'origine e che non giace sul piano.
\end{observation}

In generale dati $k$ sottospazi $U_1, \dots, U_k$ di uno spazio vettoriale $V$,
si dice che tali sottospazi sono in somma diretta se, per ogni
$i = 1, \dots, k$, vale che l'intersezione di $U_i$ con la somma di tutti
gli altri \`e uguale a $\{O\}$, ovvero
\begin{equation*}
	U_i \cap (U_1 + \cdots + \hat{U_i} + \cdots + U_k) = \{O\}
\end{equation*}
dove il simbolo $\hat{U_i}$ indica che nella somma si \`e saltato il termine
$U_i$.

In tal caso per indicare $U_1 + \cdots + U_k$ si pu\`o usare la notazione:
\begin{equation*}
	U_1 \oplus \cdots \oplus U_k
\end{equation*}