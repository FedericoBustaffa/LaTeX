\section{Riduzione a scalini}

\subsection{Operazioni elementari sulle colonne}
Consideriamo una generica matrice in $Mat_{m \times n}(\mathbb{K})$:
\begin{equation*}
	A = \begin{pmatrix}
		a_{11} & a_{12} & \dots & a_{1n} \\
		a_{21} & a_{22} & \dots & \dots  \\
		\dots  & \dots  & \dots & \dots  \\
		a_{m1} & \dots  & \dots & a_{mn}
	\end{pmatrix}
\end{equation*}
e i tre seguenti tipi di mossa sulle colonne, detti anche
\textbf{operazioni elementari sulle colonne}:
\begin{enumerate}
	\item si somma alla colonna $i$ la colonna $j$ moltiplicata per uno scalare
	      $\lambda$.
	\item si moltiplica la colonna $i$ per uno scalare $\lambda$
	\item si scambiano fra di loro due colonne $i$ e $j$.
\end{enumerate}

\begin{defn}
	La \textbf{profondit\`a} di una colonna \`e definita come la posizione
	occupata (contata dal basso) dal suo pi\`u alto coefficiente diverso da 0.
	Alla colonna nulla si assegna per convenzione profondit\`a uguale a 0.
\end{defn}

\begin{example}
	Consideriamo la seguente matrice:
	\begin{equation*}
		A = \begin{pmatrix}
			0            & 4 - \sqrt{3} & 0  \\
			\sqrt{3} + 1 & 0            & 0  \\
			-2           & -2           & -2
		\end{pmatrix}
	\end{equation*}
	In questo caso la prima colonna ha profodit\`a uguale a 2, la
	seconda colonna uguale a 3 mentre la terza ha profondit\`a uguale a 1.
\end{example}

\begin{defn}
	Una matrice $A \in Mat_{m \times n}(\mathbb{K})$, si dice
	\textbf{in forma a scalini per colonne} se rispetta le seguenti propriet\`a:
	\begin{itemize}
		\item leggendo la matrice da sinistra a destra, le colonne non nulle si
		      incontrano tutte prima delle colonne nulle.
		\item leggendo la matrice da sinistra a destra, le profondit\`a
		      delle sue colonne non nulle risultano strettamente
		      decrescenti.
	\end{itemize}
\end{defn}

\begin{example}
	Questi sono esempi di matrici in forma \emph{a scalini}:
	\begin{equation*}
		A = \begin{pmatrix}
			1            & 0           & 0 & 0 \\
			\sqrt{3} + 1 & 1           & 0 & 0 \\
			-2           & \frac{5}{2} & 1 & 0
		\end{pmatrix}
	\end{equation*}
	\begin{equation*}
		B = \begin{pmatrix}
			1  & 0           & 0 & 0 \\
			0  & 1           & 0 & 0 \\
			-2 & \frac{5}{2} & 0 & 0
		\end{pmatrix}
	\end{equation*}
\end{example}

\begin{defn}
	In una matrice in forma a scalini per colonna, i coefficienti diversi
	da zero pi\`u alti di posizione di ogni colonna non nulla si
	chiamano \textbf{pivot}.
\end{defn}

\begin{theorem}
	Data una matrice $A \in Mat_{m \times n}(\mathbb{K})$ \`e sempre
	possibile, usando operazioni elementari sulle colonne, ridurre la
	matrice in forma a scalini per colonne.
\end{theorem}

\begin{observation}
	Quando si riduce una matrice in forma a scalini, la forma a scalini
	ottenuta non \`e unica.
\end{observation}

\begin{defn}
	Una matrice $A \in Mat_{m \times n}(\mathbb{K})$, si dice \textbf{in forma a
		scalini per colonne ridotta} se:
	\begin{itemize}
		\item $A$ \`e a scalini per colonne.
		\item Tutte le entrate nella stessa riga di un pivot, precedenti al pivot,
		      sono nulle
	\end{itemize}
\end{defn}

\begin{example}
	Esempio di matrice in forma a scalini ridotta:
	\begin{equation*}
		\begin{pmatrix}
			1            & 0           & 0 & 0 \\
			\sqrt{3} + 1 & 1           & 0 & 0 \\
			-2           & \frac{5}{2} & 1 & 0
		\end{pmatrix} \rightarrow
		\begin{pmatrix}
			1 & 0 & 0 & 0 \\
			0 & 1 & 0 & 0 \\
			0 & 0 & 1 & 0
		\end{pmatrix}
	\end{equation*}
\end{example}

\begin{proposition}
	Data una matrice in forma a scalini per colonne, \`e sempre possibile, usando
	solo la prima delle operazioni elementari sulle colonne, portare $A$ in forma
	a scalini ridotta.
\end{proposition}

\begin{corollary}
	Ogni matrice $A$ pu\`o essere trasformata, tramite le operazioni elementari
	sulle colonne, in una matrice in forma a scalini per colonne ridotta.
\end{corollary}

\begin{proposition}
	Se operiamo attraverso le operazioni elementari sulle colonne, lo
	$Span$ dei vettori colonna rimane invariato. Ovvero se indichiamo con
	$v_1, \dots, v_n$ i vettori colonna di una matrice
	$A \in Mat_{m \times n}(\mathbb{K})$, per ogni matrice $A'$ ottenuta da $A$
	attraverso le operazioni elementari sulle colonne, si ha, indicando con
	$w_1, \dots, w_n$ i vettori colonna di $A'$, che:
	\begin{equation*}
		Span(v_1, \dots, v_n) = Span(w_1, \dots, w_n)
	\end{equation*}
\end{proposition}