
\section{Sistemi lineari}

\subsection{Risoluzione di sistemi tramite riduzione per righe}
Per risolvere sistemi lineari non omogenei torner\`a molto utile la riduzione
a scalini per righe di una matrice. Quello che vedremo sar\`a il metodo noto come
\emph{metodo di	eliminazione di Gauss}.

\begin{example}
	Prendiamo il sistema
	\begin{equation*}
		\begin{cases}
			x + 2y + 2z + 2t = 1 \\
			x +5y + 6z -2t = 9   \\
			8x - y -2z -2t = 0   \\
			2y + 6z + 8t = 3
		\end{cases}
	\end{equation*}
	
	Tutte le informazioni del sistema sono contenute nella seguente
	\emph{matrice completa associata al sistema}:
	\begin{equation*}
		M = \begin{pmatrix}
			1 & 2  & 2  & 2  & 1  \\
			1 & 5  & 6  & -2 & -5 \\
			8 & -1 & -2 & -2 & 0  \\
			0 & 2  & 6  & 8  & 3
		\end{pmatrix}
	\end{equation*}
	Ogni riga contiene i coefficienti di una delle equazioni.
	Sia $S \subset \mathbb{R}^4$ l'insieme delle soluzioni del sistema, ovvero
	il sottoinsieme di $\mathbb{R}^4$ costituito dai vettori
	$\begin{psmallmatrix} a \\ b \\ c \\ d \end{psmallmatrix}$ tali che, se poniamo
	\begin{gather*}
		x = a \\
		y = b \\
		z = c \\
		t = d
	\end{gather*}
	tutte le equazioni del sistema diventano delle uguaglianze vere.
\end{example}


\begin{theorem}
	L'insieme delle soluzioni di un sistema di equazioni lineari associato alla
	matrice $M$ coincide con l'insieme delle soluzioni di un sistema associato
	alla matrice $M'$ ottenuta riducendo $M$ in forma a scalini per righe.
\end{theorem}

\begin{observation}
	In concreto questo significa che, nel risolvere il sistema, ogni scalino
	lungo lascerà "libere" alcune variabili, come vediamo nel seguente esempio.
	Supponiamo che un certo sistema omogeneo a coefficienti in $\mathbb{R}$
	conduca alla matrice a scalini:
	\begin{equation*}
		M' = \begin{pmatrix}
			1 & 0 & 2        & 2  & 0 \\
			0 & 1 & \sqrt{3} & 12 & 0 \\
			0 & 0 & 0        & 6  & 0 \\
			0 & 0 & 0        & 0  & 0 \\
			0 & 0 & 0        & 0  & 0
		\end{pmatrix}
	\end{equation*}
	Allora il sistema finale associato \`e
	\begin{equation*}
		\begin{cases}
			x + 2z + 2t         & = 0 \\
			y + \sqrt{3}z + 12t & = 0 \\
			6t                  & = 0
		\end{cases}
	\end{equation*}
	Se facciamo qualche sostituzione otteniamo:
	\begin{equation*}
		\begin{cases}
			x & = -2z         \\
			y & = -\sqrt{3} z \\
			t & = 0
		\end{cases}
	\end{equation*}
	La variabile $z$ resta "libera" e l'insieme delle soluzioni \`e il seguente
	sottospazio di $\mathbb{R}^4$:
	\begin{equation*}
		S = \{ (-2z, -\sqrt{3}z, z, 0 ) \mid z \in \mathbb{R} \}
	\end{equation*}
\end{observation}

Lo stesso procedimento vale per sistemi lineari non omogenei.
Quello che dobbiamo fare \`e sostanzialmente considerare la matrice $M$ associata 
al sistema portarla in forma a scalini e risolvere il nuovo sistema associato.

\begin{example}
	Consideriamo il sistema
	\[
		\begin{cases}
			x + 2y + z & = 1 \\
			x - y      & = 3 \\
			y + z      & = 2
		\end{cases}
	\]
	Scriviamo la matrice associata.
	\[
		\begin{pmatrix}
			1 & 2  & 1 & 1 \\
			1 & -1 & 0 & 3 \\
			0 & 1  & 1 & 2
		\end{pmatrix}
	\]
	Se ridotta a scalini per righe otteniamo
	\[
		\begin{pmatrix}
			1 & 2  & 1  & 1  \\ 
			0 & -3 & -1 & 2  \\ 
			0 & 0  & 2  & 8 
		\end{pmatrix}
	\]
	E dunque otteniamo il nuovo sistema 
	\[
		\begin{cases}
			x + 2y + z & = 1 \\
			-3y - z    & = 2 \\ 
			2z         & = 8
		\end{cases}
		\quad \Rightarrow \quad
		\begin{cases}
			x & = 1  \\ 
			y & = -2 \\ 
			z & = 4
		\end{cases}
	\]
\end{example}