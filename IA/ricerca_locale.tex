\chapter{Ricerca locale}
Nella ricerca classica, vista nei capitoli \ref{chapter: ricerca non informata}
e \ref{chapter: euristica}, l'agente risolutore assume che l'ambiente sia
completamente osservabile e deterministico e che si trovi nelle condizioni di
produrre un piano che pu\`o essere eseguito senza intoppi per raggiungere
l'obbiettivo.

\subsubsection{Ambienti pi\`u realistici}
La ricerca nell'intero spazio degli stati, che sia sistematica o con euristica,
\`e troppo costosa. Abbiamo bisogno di metodi di \textbf{ricerca locale} che siano
anche in grado di gestire ambienti parzialmente osservabili e azioni non
deterministiche.

\subsubsection{Algoritmi di ricerca locale}
Gli algoritmi visti fino ad ora esplorano gli spazi di ricerca alla ricerca di un
goal e restituiscono un \emph{cammino soluzione}. Ma a volte la soluzione del problema
coincide con lo stato goal. Gli algoritmi di ricerca locale sono adatti per problemi in
cui:
\begin{itemize}
	\item La sequenza di azioni non \`e importante: quello che conta \`e unicamente
	      lo stato goal.
	\item Tutti gli elementi della soluzione sono nello stato ma alcuni vincoli sono
	      violati.
\end{itemize}
Gli algoritmi di ricerca locale in generale:
\begin{itemize}
	\item Non sono sistematici.
	\item Tengono traccia solo del nodo corrente e si spostano sui nodi adiacenti.
	\item Non tengono traccia dei cammini, dunque
	      \begin{itemize}
		      \item Sono efficienti in memoria.
		      \item Possono trovare soluzioni ragionevoli anche in spazi molto grandi e
		            infiniti (come nel continuo).
	      \end{itemize}
	\item Sono utili per risolvere problemi di ottimizzazione, come trovare:
	      \begin{itemize}
		      \item Lo stato migliore secondo una funzione obbiettivo.
		      \item Lo stato di costo minore.
	      \end{itemize}
\end{itemize}