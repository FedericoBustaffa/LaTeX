\chapter{Ricerca locale}
Nella ricerca classica, vista nei capitoli \ref{chapter: ricerca non informata}
e \ref{chapter: euristica}, l'agente risolutore assume che l'ambiente sia
completamente osservabile e deterministico e che si trovi nelle condizioni di
produrre un piano che pu\`o essere eseguito senza intoppi per raggiungere
l'obbiettivo.

\subsubsection{Ambienti pi\`u realistici}
La ricerca nell'intero spazio degli stati, che sia sistematica o con euristica,
\`e troppo costosa. Abbiamo bisogno di metodi di \textbf{ricerca locale} che siano
anche in grado di gestire ambienti parzialmente osservabili e azioni non
deterministiche.

\subsubsection{Algoritmi di ricerca locale}
Gli algoritmi visti fino ad ora esplorano gli spazi di ricerca alla ricerca di un
goal e restituiscono un \emph{cammino soluzione}. Ma a volte la soluzione del problema
coincide con lo stato goal. Gli algoritmi di ricerca locale sono adatti per problemi in
cui:
\begin{itemize}
	\item La sequenza di azioni non \`e importante: quello che conta \`e unicamente
	      lo stato goal.
	\item Tutti gli elementi della soluzione sono nello stato ma alcuni vincoli sono
	      violati.
\end{itemize}
Gli algoritmi di ricerca locale in generale:
\begin{itemize}
	\item Non sono sistematici.
	\item Tengono traccia solo del nodo corrente e si spostano sui nodi adiacenti.
	\item Non tengono traccia dei cammini, dunque
	      \begin{itemize}
		      \item Sono efficienti in memoria.
		      \item Possono trovare soluzioni ragionevoli anche in spazi molto grandi e
		            infiniti (come nel continuo).
	      \end{itemize}
	\item Sono utili per risolvere problemi di ottimizzazione, come trovare:
	      \begin{itemize}
		      \item Lo stato migliore secondo una funzione obbiettivo.
		      \item Lo stato di costo minore.
	      \end{itemize}
\end{itemize}

\section{Hill climbing}
\`E una ricerca locale di tipo greedy. Vengono generati i successori e valutati. Il successore
viene scelto se migliora lo stato attuale. Ci sono tre sottocategorie di questo algoritmo che
dipendono dalla modalit\`a di scelta del successore:
\begin{itemize}
	\item Se viene scelto il successore migliore abbiamo un \textbf{Hill climbing a salita rapida}.
	\item Se ne viene preso uno a caso tra quelli che migliorano lo stato abbiamo un
	      \textbf{Hill climbing stocastico}.
	\item Se viene scelto il primo successore che migliore lo stato abbiamo un
	      \textbf{Hill climbing con prima scelta}.
\end{itemize}
Se nessuno dei successori \`e migliore l'algoritmo fallisce.

\begin{lstlisting}[style=pseudo-style]
Hill_climbing(problema)
	nodo = problema.statoIniziale;
	while true
		successore = nodo.migliorSuccessore()
		if successore.valore <= nodo.valore then
			return nodo.valore;
		nodo = successore;
\end{lstlisting}

\subsubsection{Massimi e minimi locali}
Questo algoritmo ha per\`o a che fare con il problema dei \textbf{massimi e minimi locali}, ovvero
si potrebbe arrivare ad un punto in cui nessun successore migliora lo stato facendo cos\`i terminare
l'algoritmo senza arrivare all'ottimo, ovvero il \textbf{massimo o minimo globale} che potrebbe
trovarsi pi\`u avanti.

\subsubsection{Miglioramenti per Hill climbing}
\begin{itemize}
	\item Consentire un numero limitato di mosse \textbf{laterali}, ovvero si continua anche
	      a parit\`a di $h$
	\item Hill climbing stocastico: si sceglie a caso tra le mosse in salita (se il problema
	      \`e trovare un minimo) per provare a superare il minimo locale e riuscire a proseguire.
	\item Hill climbing con prima scelta.
	\item Hill climbing con riavvio casuale: si riparte da un punto passato. Se la probabilit\`a
	      di successo \`e $p$, saranno necessarie in media $1 / p$ ripartenze per trovare la soluzione.
	      \`E tendenzialmente completo ma se ci sono molti minimi/massimi locali si blocca spesso.
\end{itemize}

\section{Tempra simulata}
Questo algoritmo \`e in analogia con il processo di tempra dei metalli in metallurgia. I metalli vengono
portati a temperature molto elevate e raffreddati gradualmente consentendo di cristallizzare in uno stato
a pi\`u bassa energia.

L'algoritmo di \textbf{tempra simulata} combina Hill climbing con una scelta stocastica (non del tutto
casuale).
\subsubsection{Parametri}
Per simulare al meglio il processo abbiamo bisogno di alcuni parametri:
\begin{itemize}
	\item $T$: \`e la temperatura. Decresce col progredire dell'algoritmo secondo un piano definito.
	\item $\Delta E$: corrisponde alla variazione dell'energia. Equivale a $f(n') - f(n)$.
	\item $p$: probabilit\`a del nodo $n'$ di essere scelto. Equivale a $p = e^{\Delta E / T}$ ed \`e
	      compresa tra 0 e 1.
\end{itemize}

\subsubsection{Procedimento}
Ad ogni passo si sceglie un successore $n'$ a caso:
\begin{itemize}
	\item Se migliora lo stato corrente viene espanso.
	\item Altrimenti (caso $\Delta E = f(n') - f(n) < 0$) quel nodo viene scelto con probabilit\`a: $p = e^{\Delta E / T}$.
	      Si genera un numero casuale compreso tra 0 e 1: se \`e $< p$ allora viene scelto come successore, altrimenti no.
\end{itemize}
Dunque $p$ \`e inversamente proporzionale al peggioramento. Infatti se la mossa peggiora molto, $\Delta E$ sar\`a un negativo
molto alto che abbasser\`a molto il valore di $p$.

Come gi\`a detto in precedenza $T$ decresce col progredire dell'algoritmo facendo cos\`i abbassare anche il valore di $p$ rendendo
Progredendo rende pi\`u improbabili mosse peggiorative. Il valore iniziale di $T$ \`e tale che, per valori medi di $\Delta E$, la
probabilit\`a $p$ sia $0.5$ circa.

\subsubsection{Analisi}
La probabilit\`a di una mossa in discesa diminuisce col passare del tempo e l'algoritmo si comporta sempre di pi\`u come Hill climbing.
Se $T$ viene decrementato abbastanza lentamente, con probabilit\`a tendente a 1, si raggiunge l'ottimo.

\section{Local Beam}
\`E una versione locale di Beam Search in cui si tengono in memoria $k$ stati, anzich\'e uno solo. Ad ogni passo si generano i successori
di tutti i $k$ stati: se si trova un goal ci si ferma, altrimenti si prosegue con i $k$ migliori tra questi.

\subsection{Locale Beam stocastica}
Si introduce un elemento di casualit\`a. Si scelgono $k$ successori ma con probabilit\`a maggiore per i migliori.

\section{Algoritmi genetici}
Sono varianti della Beam Search stocastica in cui gli stati successori sono ottenuti combinando due stati genitore.
Introduciamo una nuova terminologia:
\begin{itemize}
	\item \textbf{individui}: stati.
	\item \textbf{popolazione}: insieme di individui.
	\item \textbf{fitness}: funzione di valutazione degli individui.
	\item \textbf{accoppiamenti}: crossover tra due individui.
	\item \textbf{mutazione genetica}: elemento casuale inserito nello stato successore.
	\item \textbf{generazioni}: popolazione successiva ad una popolazione di stati genitore.
\end{itemize}

\subsubsection{Procedimento}
\begin{enumerate}
	\item Abbiamo la popolazione iniziale composta da $k$ individui generati casualmente. Ogni individuo \`e rappresentato come una stringa.
	\item Gli individui sono valutati da una funzione fitness.
	\item Si selezionano gli individui per gli accoppiamenti con una probabilit\`a proporzionale alla fitness.
	\item Le coppie danno vita alla generazione successiva:
	      \begin{itemize}
		      \item Combinando materiale genetico (crossover).
		      \item Mutazione genetica (casuale).
	      \end{itemize}
	\item La popolazione ottenuta dovrebbe essere migliore.
	\item Si ripete fino ad ottenere stati abbastanza buoni (stato obbiettivo).
\end{enumerate}

\subsubsection{Pro}
Gli algoritmi di questo tipo:
\begin{itemize}
	\item Hanno la tendenza a salire della Beam Search stocastica.
	\item Favoriscono l'interscambio di informazioni tra thread paralleli di ricerca.
	\item Funzionano meglio se il problema ha componenti significative rappresentate in sottostringhe.
\end{itemize}

\subsubsection{Contro}
Il punto critico di questi algoritmi \`e proprio la rappresentazione del problema in stringhe.

\section{Ricerca negli spazi continui}
Molti casi reali hanno spazi di ricerca continua che hanno quindi un fattore di ramificazione infinito. I tipi di ricerca visti fino
ad ora non sono quindi opzioni valide per questo tipo di problemi.

L'approccio che invece viene utilizzato ci porta a descrivere lo stato con \textbf{variabili continue} e \textbf{vettori}.

Per trovare il minimo o il massimo di una funzione \emph{continua} e \emph{differenziabile} viene utilizzato il \textbf{gradiente},
che restituisce la direzione di massima pendenza in un determinato punto.

\begin{definition}
	Data una funzione obbiettivo $f$, il gradiente \`e definito come il vettore che ha per componenti le derivate parziali rispetto alle
	variabili della funzione:
	\[ \nabla f(x_1, \dots, x_k) = ((f_{x_1}(x_1, \dots, x_k), \dots, f_{x_k}(x_1, \dots, x_k))) \]
	dove $f_{x_i}$ \`e la derivata parziale rispetto alla variabile $x_i$.
\end{definition}

In generale si procede con un Hill climbing iterativo in cui si sceglie il successore in questo modo
\[ x_{\text{new}} = x \pm \eta \nabla f (x) \]
Dove $\eta$ \`e la dimensione del passo da compiere. Se devo salire, dunque massimizzare il valore cercato svolgo un'operazione di somma,
se devo scendere faccio la differenza (ecco perch\'e ho messo $\pm$ nella formula).

Questo algoritmo quantifica lo spostamento senza cercarlo tra gli infiniti possibili successori. Mi dice quindi di quanto spostarmi.
Se vado oltre il massimo o il minimo il segno dell'operazione si inverte e torno indietro.

La dimensione del passo $\eta$ dev'essere scelta correttamente: troppo grande fa fare passi troppo ampi e si rischia di non arrivare mai all'ottimo,
troppo piccolo fa fare passi troppo piccoli e si ha una ricerca troppo lenta.