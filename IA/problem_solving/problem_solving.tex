\chapter{Problem solving}
\section{Formulazione di un problema}
\subsection{Agenti risolutori di problemi}
Gli \textbf{agenti risolutori di problemi} adottano il paradigma della risoluzione di problemi
come \textbf{ricerca} in uno \textbf{spazio di stati}.

Sono agenti basati su modello e con obbiettivo che adottano una rappresentazione
\textbf{atomica} dello stato e pianificano l'intera sequenza di mosse prima di agire.

\subsection{Processo di risoluzione}
Il processo di risoluzione di un problema si compone di quattro passi principali:
\begin{enumerate}
	\item Determinazione dell'obbiettivo.
	\item Formulazione del problema, che consiste nella rappresentazione di
	      \begin{itemize}
		      \item stati
		      \item azioni
	      \end{itemize}
	\item Determinazione della soluzione mediante \textbf{ricerca}
	\item Esecuzione del piano.
\end{enumerate}

\subsection{Assunzioni sull'ambiente}
Per adesso assumiamo che l'ambiente sia:
\begin{itemize}
	\item Statico
	\item Osservabile
	\item Discreto
	\item Deterministico
\end{itemize}

\subsection{Formulazione del problema}
Un problema pu\`o essere definito formalmente con queste cinque componenti:
\begin{description}
	\item[Stato iniziale:] Lo stato da cui si parte.
	\item[Azioni possibili:] Dipendono dallo stato corrente e portano in un nuovo stato.
	\item[Modello di transizione:] Descrizione dello spazio degli stati e come si effettua
	      la transizione dall'uno all'altro.
	\item[Test obbiettivo:] Ho un insieme di stati obbiettivo e devo verificare se gli stati
	      che ho raggiunto sono quelli nell'insieme obbiettivo oppure no.
	\item[Costo del cammino:] \`E la somma del costo delle azioni (transizioni di stato).
\end{description}

\section{Algoritmi di ricerca}
Gli \textbf{algoritmi di ricerca} prendono in input un problema e restituiscono un
\textbf{cammino soluzione}, ovvero un cammino che porta dallo stato iniziale ad uno stato
obbiettivo.

Vengono valutati in base al \textbf{costo di ricerca} di una soluzione e anche
all'\textbf{efficienza} della soluzione trovata.