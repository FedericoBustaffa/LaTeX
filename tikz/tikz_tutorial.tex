\documentclass[11pt, a4paper]{article}

\pdfpagewidth\paperwidth
\pdfpageheight\paperheight

\usepackage[utf8]{inputenc}
\usepackage[T1]{fontenc}
\usepackage[italian]{babel}
\usepackage[hidelinks]{hyperref}

% maths
\usepackage{mathtools}
\usepackage{amsmath}
\usepackage{amssymb}
\usepackage{amsfonts}

% tikz
\usepackage{tikz}
\usepackage{scalerel}
\usepackage{pict2e}
\usepackage{tkz-euclide}
\usepackage{pgfplots, pgfplotstable, pgf-pie}

% colori
\usepackage{xcolor}

\usetikzlibrary{calc}
\usetikzlibrary{patterns, arrows}
\usetikzlibrary{shadows}
\usetikzlibrary{external}

\pgfplotsset{compat=newest}

\usepgfplotslibrary{statistics, fillbetween}

% \pgfplotsset{
% 	standard/.style={
% 			axis line style = thick,
% 			trig format = rad,
% 			enlargelimits,
% 			axis x line = middle,
% 			axis y line = middle,
% 			enlarge x limits = 0.15,
% 			enlarge y limits = 0.15,
% 			every axis x label/.style={at{(current axis.right of origin)}, anchor=north west},
% 			every axis y label/.style={at{(current axis.above origin)}, anchor=south east},
% 			grid=both
% 		}
% }

\title{Tikz}
\author{Federico Bustaffa}
\date{10/08/2022}

\begin{document}

\maketitle
\tableofcontents

\section{Figure}

\begin{center}
	\begin{tikzpicture}
		\draw[lightgray] (-4.9,-2.9) grid (4.9, 2.9);

		\draw[{Circle[scale=2]}-{Stealth}] (-2, -2) -- (2, 2) node[rotate=45, midway, above] {testo a caso};
	\end{tikzpicture}
\end{center}

\begin{center}
	\begin{tikzpicture}
		\draw[lightgray] (-4.9,-1.9) grid (4.9, 1.9);

		\draw[ultra thick, blue] (-3, 0) circle[radius=1];
		\fill[red] (0, 0) circle[radius=1];
		\filldraw[ultra thick, green, fill=green!30] (3, 0) circle[radius=1];
	\end{tikzpicture}
\end{center}

\begin{center}
	\begin{tikzpicture}
		\draw[lightgray] (-4.9,-1.9) grid (4.9, 1.9);

		\filldraw[ultra thick, purple, fill=purple!30] (-1.5, 0) ellipse (1.5 and 1);

		\filldraw[ultra thick, orange, fill=orange!30] (3, -1) arc(0 : 90 : 2) -- (1, 1) -- (1, -1) -- (3, -1);
	\end{tikzpicture}
\end{center}
\section{Strutture}

\begin{center}
	\begin{tikzpicture}[level/.style={sibling distance=3cm/#1}]
		\node[circle, draw, text=black, draw=blue, thick, fill=blue!20] {Root}
		child[dotted] { node[circle, draw, thick, text=black, draw=green, fill=green!20] {A}
				child {node[circle, draw] {A1}}
				child {node[circle, draw] {A2}}
			}
		child[thick] { node[rectangle, draw, text=black, draw=red, fill=red!20] {B}
				child[red] { node[circle, draw, text=black, fill=red!20] {B1}}
				child[red] { node[rectangle, draw, text=black, fill=red!20] {B2}}
			}
		child[thick] { node[circle, draw, text=black] {C}
				child { node[circle, draw, text=black, fill=yellow!30] {C1}}
				child { node[circle, draw, text=black, fill=yellow!30] {C2}}
			};
	\end{tikzpicture}
\end{center}

\begin{center}
	\begin{tikzpicture}[node distance=2cm, main node/.style={circle, draw, thick, font=\Large}]
		\node[main node, fill=green!20, draw=green, thick] (1) {1};
		\node[main node] (3) [below of=1] {3};
		\node[main node] (2) [right of=3] {2};
		\node[main node] (4) [left of=3] {4};
		\node[main node, fill=blue!20, draw=blue, thick] (5) [below of=3] {5};

		\path
		(1) edge[ultra thick, red] node[left, black] {1} (4)
		(1) edge[thick] node[left, black] {3} (3)
		(1) edge[thick] node[right, black] {2} (2)
		(3) edge[ultra thick, red] node[right, black] {2} (5)
		(3) edge[thick] node[above, black] {3} (2)
		(2) edge[thick] node[right, black] {4} (5)
		(3) edge[ultra thick, red] node[below, black] {1} (4)
		(4) edge[bend right, -Stealth, thick] (5);
	\end{tikzpicture}
\end{center}
\section{Diagrammi}

\begin{center}
	\begin{tikzpicture}[scale=0.65]
		\pie[color = {red!70, yellow!70, green!70, cyan!70}] {
			20/Rosso,
			25/Giallo,
			30/Verde,
			25/Blu
		}
	\end{tikzpicture}
\end{center}

\begin{center}
	\begin{tikzpicture}
		\begin{axis}[
				x tick label style={/pgf/number format/1000 sep=},
				y label=LABEL,
				enlargelimits=0.05
			]

			\addplot coordinates {(2012, 408184)};

		\end{axis}
	\end{tikzpicture}
\end{center}
\section{Funzioni}

\begin{center}
	\begin{tikzpicture}[scale=1.5]
		\begin{axis}[
				title={Funzioni},
				font=\scriptsize,
				axis lines = center,
				xlabel = $x$,
				ylabel = $y$,
				xtick = {-4, -2, 2, 4},
				ytick = {-4, -2, 2, 4, 6, 8},
				ymin = -5, ymax = 10,
				xmin = -5, xmax = 5,
				grid = both,
				grid style = dashed,
				legend pos = north west,
				enlargelimits
			]

			\addplot [
				domain=0:5,
				samples=100,
				thick,
				color=red
			]
			{log2(x)};
			\addlegendentry{$\log_2{x}$};

			\addplot [
				domain=0:5,
				samples=100,
				thick,
				color=blue
			]
			{sqrt(x)};
			\addlegendentry{$\sqrt{x}$};

			\addplot [
				domain=-5:5,
				samples=100,
				thick,
				color=yellow,
			]
			{2^x};
			\addlegendentry{$2^x$};

			\addplot [
				domain=-5:5,
				samples=100,
				thick,
				color=green,
				mark=square
			]
			coordinates {(-4, -4) (-2, -2) (0, 0) (2, 2) (4, 4)};
			\addlegendentry{$x$};

		\end{axis}
	\end{tikzpicture}
\end{center}

\begin{center}
	\begin{tikzpicture}
		\begin{axis}[
				% standard,
				xtick={-2, -1, 1, 2},
				xticklabels={-2, -1, 1, 2},
				ytick={-2, -1, 1, 2},
				yticklabels={-2, -1, 1, 2},
				samples=1000,
				xmin=-2, xmax=2, ymin=-2, ymax=2
			]
			\node[anchor=center, label=south west: $O$] at (axis cs: 0, 0){};

			\addplot[name path=F, domain={-3:3}, thick, blue] {x^3 + x^2 - x - 0.5};

			\path[name path=xAxis] (axis cs:-1.45, 0) -- (axis cs:0.85, 0);
			\addplot[fill=blue, fill opacity=0.2] fill between [of=F and xAxis, soft clip={domain=-1.45:0.85}];
		\end{axis}
	\end{tikzpicture}
\end{center}

\end{document}