\chapter{Giochi con avversario}
Fino ad ora abbiamo avuto a che fare con ambienti completamente osservabili, deterministici, a utente singolo e
composti da stati atomici.

Per trattare problemi come i \textbf{giochi con avversario} dobbiamo cambiare il paradigma e andare verso ambienti
\emph{pi\`u realistici} e considerare ambienti parzialmente osservabili, non deterministici, multiagente e con una
rappresentazione degli stati pi\`u complessa.

In questo tipo di ambiente le \textbf{percezioni} sono fondamentali per restringere lo spazio degli stati e informano
l'algoritmo sull'effetto delle azioni. L'agente in questo caso elabora un \textbf{piano di contingenza}, ovvero una
strategia che tiene conto delle diverse eventualit\`a.

Il modello di transizione in questo caso restituisce un insieme di stati (l'agente non sa in quale si ritrover\`a).

\subsubsection{Alberi di ricerca AND-OR}
Un albero di ricerca AND-OR ci viene incontro in questa situazione ed \`e strutturato come segue:
\begin{itemize}
	\item I nodi \textbf{OR} sono le scelte dell'agente.
	\item I nodi \textbf{AND} sono le diverse contingenze da considerare. Sono cio\`e le scelte dell'ambiente, composte
	      da pi\`u stati possibili.
	\item Una soluzione ad un problema di ricerca AND-OR \`e un albero che:
	      \begin{itemize}
		      \item Ha un nodo obbiettivo in ogni foglia.
		      \item Specifica un'unica azione nei nodi OR.
		      \item Include tutti gli archi uscenti da nodi AND (tutte le contingenze).
	      \end{itemize}
\end{itemize}

\subsubsection{Panoramica giochi con avversario}
In questo capitolo parleremo dei \textbf{giochi con avversario}, in cui si deve tener conto della mossa dell'avversario,
che sar\`a ignota fino a che non sar\`a compiuta. Il compito dell'agente \`e cercare di prevederla analizzando tra le
mosse possibili, le pi\`u probabili e costruire di conseguenza una strategia per vincere.

\section{Paradigma per giochi con avversario}
Il paradigma che stiamo per introdurre serve a trattare giochi con avversario come ad esempio gli scacchi. Ed \`e formulato
come segue:
\begin{itemize}
	\item Si devono avere regole semplici e formalizzabili.
	\item L'ambiente dev'essere accessibile e deterministico.
	\item Due giocatori che giocano a turno.
	\item Giochi a somma zero (se uno vince l'altro perde).
	\item Ambiente multiagente competitivo e strategico.
	\item Complessit\`a e vincoli di tempo reale (si ha a disposizione un tempo limitato per pensare).
\end{itemize}
In questo tipo di giochi una delle strategie migliori \`e quella di eseguire una mossa, provare a identificare la
contromossa dell'avversario per elaborare una contromossa, vedere cosa fa l'avversario e rielaborare in base alla
mossa compiuta.

\subsubsection{Giochi come problemi di ricerca}
Il problema del gioco visto come un problema di ricerca apparir\`a in questo modo:
\begin{itemize}
	\item \textbf{Stati}: configurazioni o posizioni del gioco. Con la notazione \verb|Player(s)| indichiamo il giocatore
	      a cui tocca muovere nello stato \verb|s|.
	\item \textbf{Stato iniziale}: configurazione iniziale del gioco.
	\item \verb|Actions(s)|: mosse legali nello stato \verb|s|.
	\item \verb|Result(s, a)|: stato risultante da una mossa.
	\item \verb|Terminal_Test(s)|: vero se \verb|s| \`e uno stato terminale, falso altrimenti.
	\item \verb|Utility(s, p)|: valore numerico che valuta gli stati terminali del gioco per il giocatore \verb|p|.
\end{itemize}

\section{MinMax}
La \textbf{ricerca MinMax} consiste nella valutazione dello stato successivo facendo riferimento al suo
\textbf{valore MinMax}.

La miglior mossa da fare \`e quella che tiene di conto anche della contromossa dell'avversario. L'algoritmo MinMax
assume che l'avversario giochi sempre al massimo delle sue possibilit\`a.

Ci\`o che vogliamo fare \`e massimizzare l'efficacia della mossa del giocatore a cui tocca fare la mossa e minimizzare
l'efficacia della mossa dell'avversario. Chiameremo quindi \verb|MAX| il giocatore a cui tocca effettuare la mossa
e \verb|MIN| l'avversario.

\subsection{Calcolo del valore MinMax}
Nella pratica il valore MinMax di uno stato si calcola come segue:
\begin{enumerate}
	\item Se lo stato \`e uno stato terminale applico ritorno \verb|Utility(s, MAX)|.
	\item Se sto valutando la mossa che deve fare \verb|MAX| ritorno il massimo tra i valori MinMax di tutti i possibili
	      successori generabili a partire dallo stato \verb|s|.
	\item Se sto valutando la mossa che deve fare \verb|MIN| ritorno il minimo tra i valori MinMax di tutti i possibili
	      successori generabili a partire dallo stato \verb|s|.
\end{enumerate}

\begin{lstlisting}[style=pseudo-style]
MinMaxDecision(state)
	value = -infty;
	best_action = None;
	foreach action in Actions(state)
		next_state = Result(state, action);
		min_value = MinValue(next_state);
		if min_value > value then
			value = min_value;
			best_action = action;

	return best_action;
\end{lstlisting}

\begin{lstlisting}[style=pseudo-style]
MinValue(state)
	if Terminal_Test(state) then
		return Utility(state);
	value = infty;
	foreach action in Actions(state)
		next_state = Result(state, action);
		value = Min(value, MaxValue(next_state));
	
	return value;
\end{lstlisting}

\begin{lstlisting}[style=pseudo-style]
MaxValue(state)
	if Terminal_Test(state) then
		return Utility(state);
	value = -infty;
	foreach action in Actions(state)
		next_state = Result(state, action);
		value = Max(value, MinValue(next_state));
	
	return value;
\end{lstlisting}
Come possiamo vedere, l'algoritmo sceglie la mossa successiva tra quelle che mettono maggiormente in difficolt\`a
l'avversario. Ovvero tra quelle che minimizzano maggiormente l'efficacia della mossa successiva dell'avversario.

\textbf{NOTA}: In generale non conviene scegliere la mossa ottima se l'avversario gioca in maniera \emph{sub-ottima}.

\subsection{Analisi MinMax}
\begin{itemize}
	\item La complessit\`a in tempo \`e $O(b^m)$.
	\item La complessit\`a in spazio \`e $O(bm)$.
\end{itemize}

\section{MinMax euristico}
Per giochi molto complessi \`e impensabile un'esplorazione sistematica dello spazio degli stati. \`E necessario fare
uso di \textbf{euristiche} per stimare la bont\`a di uno stato del gioco.

Per farlo aggiungeremo al paradigma descritto in precedenza una funzionae di valutazione euristica, \verb|Eval(s)|.

La nuova strategia consiste nel guardare avanti di $d$ livelli.
\begin{itemize}
	\item Si espande l'albero di ricerca un certo numero di livelli $d$ compatibile con tempo e spazio a disposizione.
	\item Si valutano gli stati ottenuti e si propaga indietro il risultato con la regola del \verb|MAX| e \verb|MIN|.
\end{itemize}
Il risultato \`e come prima un algoritmo MinMax ma con \textbf{Cutoff}. Perci\`o nelle funzioni \verb|MinValue| e
\verb|MaxValue| viste prima:
\begin{lstlisting}[style=pseudo-style]
	if Terminal_Test(state) then
		return Utility(state);
\end{lstlisting}
diventa
\begin{lstlisting}[style=pseudo-style]
	if Cutoff(state, depth) then
		return Eval(state);
\end{lstlisting}
La funzione \verb|Cutoff| mi dice se ho raggiunto il livello $d$ che interpreto come stato terminale. Se l'ho raggiunto
allora applica \verb|Eval|. Ovviamente \verb|Cutoff| riconosce anche veri e propri stati terminali come faceva
\verb|Terminal_Test|.

\subsection{Calcolo del valore H-MinMax}
Il calcolo del valore H-MinMax non cambia molto rispetto a prima, semplicemente qui facciamo un numero di previsioni in
avanti limitato e ci affidiamo ad un'euristica per la valutazione dello stato terminale:
\begin{itemize}
	\item Se lo stato \`e uno stato terminale oppure ho raggiunto il limite di profondit\`a $d$ ritorno \verb|Eval(s)|.
	\item Se sto valutando la mossa che deve fare \verb|MAX| ritorno il massimo tra i valori H-MinMax di tutti i possibili
	      successori generabili a partire dallo stato \verb|s|.
	\item Se sto valutando la mossa che deve fare \verb|MIN| ritorno il minimo tra i valori H-MinMax di tutti i possibili
	      successori generabili a partire dallo stato \verb|s|.
\end{itemize}

\subsection{Funzione di valutazione euristica}
La funzione \verb|Eval| di valutazione \`e una \textbf{stima dell'utilit\`a attesa} a partire da una certa posizione.
La qualit\`a di questa funzione \`e determinante e deve:
\begin{itemize}
	\item Essere \textbf{consistente} con l'utilit\`a se applicata a stati terminali di gioco.
	\item Essere \textbf{efficiente} da calcolare.
	\item Riflettere le probabilit\`a effettive di vittoria.
	\item Avere un \textbf{valore atteso} che combina probabilit\`a con utilit\`a dello stato terminale; pu\`o essere
	      appreso con l'esperienza.
\end{itemize}

\subsubsection{Stati non quiescienti}
A volte l'esplorazione fino ad un certo livello pu\`o mostrare una situazione molto vantaggiosa ma dopo si finisce per
trovarsi in una situazione di svantaggio. La soluzione in questi casi \`e applicare la valutazione a
\textbf{stati quiescienti}, ossia stati in cui la funzione di valutazione non \`e soggetta a mutamenti repentini.

\subsubsection{Effetto orizzonte}
In alcune occasioni pu\`o succedere che vengano privilegiate mosse diversive che hanno il solo effetto di spingere il
problema oltre l'orizzonte. Quindi si deve riuscire a capire quando si stanno compiendo mosse inutili che stanno solo
rimandando il problema.

\section{Potatura Alfa-Beta}
Non \`e necessario esplorare ogni cammino, la \textbf{potatura Alfa-Beta} dimezza la ricerca mantenendo la valutazione
MinMax corretta.

Si va avanti in profondit\`a fino al livello desiderato e propagando indietro i valori si
decide se si pu\`o abbandonare l'esplorazione del sotto-albero.
\begin{enumerate}
	\item \verb|MaxValue| e \verb|MinValue| vengono invocate rispettivamente con due valori di riferimento $\alpha$
	      (inizialmente $-\infty$) e $\beta$ (inizialmente $+\infty$).
	\item Rappresentano rispettivamente la migliore alternativa per \verb|MAX| e per \verb|MIN| fino a quel momento.
	\item I tagli avvengono quando, nel propagare all'indietro:
	      \begin{itemize}
		      \item $v \geq \beta$ per i nodi \verb|MAX| (taglio $\beta$).
		      \item $v \leq \alpha$ per i nodi \verb|MIN| (taglio $\alpha$).
	      \end{itemize}
\end{enumerate}
Tutto rimane uguale tranne per il fatto che \verb|value| viene inizializzato con \verb|MaxValue| valutato
tra $-\infty$ e $+\infty$.
\begin{lstlisting}[style=pseudo-style]
AlphaBeta(state)
	value = MaxValue(state, -infty, +infty);
	best_action = None;
	foreach action in Actions(state)
		next_state = Result(state, action);
		min_value = MinValue(next_state, -infty, +infty);
		if min_value > value then
			value = min_value;
			best_action = action
	
			return best_action;
\end{lstlisting}

\begin{lstlisting}[style=pseudo-style]
MaxValue(state, alpha, beta)
	if Terminal_Test(state) then
		return Utility(state);
	value = -infty;
	foreach action in Actions(state)
		next_state = Result(state, action);
		value = Max(value, MinValue(next_state, alpha, beta));
		if value >= beta then
			return value;
		alpha = Max(alpha, value);
	return value;
\end{lstlisting}

\begin{lstlisting}[style=pseudo-style]
MinValue(state, alpha, beta)
	if Terminal_Test(state) then
		return Utility(state);
	value = +infty;
	foreach action in Actions(state)
		next_state = Result(state, action);
		value = Min(value, MaxValue(next_state, alpha, beta));
		if value <= alpha then
			return value;
		beta = Max(beta, value);
	return value;
\end{lstlisting}

\subsection{Ordinamento delle mosse}
La potatura ottimale si ottiene quando ad ogni livello sono generate prima le mosse migliori per chi gioca:
\begin{itemize}
	\item Per i nodi \verb|MAX| sono generate prima le mosse con valore pi\`u alto.
	\item Per i nodi \verb|MIN| sono generate prima le mosse con valore pi\`u basso (migliori per \verb|MIN|).
\end{itemize}

\subsection{Analisi Alfa-Beta}
\begin{itemize}
	\item La complessit\`a in tempo \`e $O(b^{m/2})$.
	\item La complessit\`a in spazio rimane $O(bm)$.
\end{itemize}

\subsection{Miglioramenti}
Dunque Alfa-Beta pu\`o arrivare a profondit\`a doppia rispetto a MinMax a parit\`a di tempo.

Per avvicinarsi all'ordinamento ottimale possiamo:
\begin{itemize}
	\item Usare un \textbf{approfondimento iterativo} cos\`i da scoprire,
	      ad una certa iterazione, informazioni utili per l'ordinamento delle mosse. Informazioni da usare
	      in una successiva iterazione.
	\item Tenere in memoria una \textbf{tabella delle trasposizioni} in cui memorizzare la valutazione di ogni stato.
	\item \textbf{Potatura in avanti}: si esplorano solo alcune mosse ritenute promettenti e si tagliano le altre.
	\item \textbf{Database di mosse di apertura e chiusura}: aprire e chiudere il gioco tramite schemi gi\`a conosciuti
	      e collaudati invece di stare ad esplorare stati, come quelli iniziali, in cui ci sono poche mosse sensate.
\end{itemize}