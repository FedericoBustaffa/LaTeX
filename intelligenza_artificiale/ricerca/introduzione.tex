\part{Problemi di ricerca}
\chapter{Introduzione}
L'intelligenza artificiale si occupa della comprensione e riproduzione del
comportamento \emph{intelligente}. Nello specifico, l'intelligenza pu\`o riguardare
le sole facolt\`a mentali, ovvero il ragionamento, oppure l'azione umana in senso
pi\`u ampio, ovvero il comportamento.

\section{I diversi approcci all'IA}
\subsection{Approccio psicologico cognitivo}
Da un lato vediamo l'IA come una scienza che si basa sulla sperimentazione e possiamo
definire questo approccio come \textbf{approccio della psicologia cognitiva}.

Qui  l'obbiettivo \`e quello di comprendere l'intelligenza umana e costruire dei
modelli computazionali che sono poi sottoposti ad una verifica sperimentale.

Il criterio di successo \`e capire i meccanismi che stanno alla base
dell'intelligenza umana per poi risolvere i problemi con gli stessi processi usati
dall'uomo.

\subsection{Approccio costruttivo}
Dall'altro possiamo vedere l'IA come una disciplina informatica in cui si cerca di
costruire entit\`a dotate di intelligenza senza perforza essere legati al modello di
intelligenza umana.

L'obbiettivo \`e quello codificare quelli che sono il pensiero e il comportamento
\emph{razionale} cos\`i come li conosciamo.

Il criterio di successo \`e risolvere problemi che richiedono intelligenza, non
importa come.

\subsection{Una definizione di IA}
Possiamo riassumere ci\`o che abbiamo detto prima con la seguente definizione di
intelligenza artificiale: Il settore dell'IA consiste nell'indagine tecnologica e
intellettuale che mira al raggiungimento dei seguente obbiettivi:
\begin{itemize}
	\item Costruzione di macchine intelligenti, sia che operino come l'uomo che
	      diversamente.
	\item Formalizzazione della conoscenza e meccanizzazione del ragionamento, in
	      tutti i settori di azione dell'uomo.
	\item Comprensione mediante modelli computazionali della psicologia e
	      comportamento di uomini, animali e agenti artificiali.
	\item Rendere il lavoro con il calcolatore altrettanto facile e utile del lavoro
	      con persone capaci, collaborative e possibilmente esperte.
\end{itemize}

\section{Capacit\`a dell'IA}
Ma che tipo di capacit\`a deve avere un agente intelligente ? La definizione di
intelligenza non \`e unica e nemmeno il tipo di intelligenza lo \`e. Ecco perch\'e,
in base al tipo di agente intelligente e al problema che deve risolvere si enfatizza
un tipo di intelligenza piuttosto che un altro. Vediamo alcune delle capacit\`a che
pu\`o avere uno di questi agenti:
\begin{itemize}
	\item Simulazione del comportamento umano.
	\item Ragionamento logico/matematico.
	\item Competenza in un certo ambito.
	\item Buon senso (senso di collettivit\`a).
	\item Efficienza nell'interazione con un certo ambiente.
	\item Socialit\`a e collaborazione.
	\item Emozioni.
	\item Apprendimento.
	\item Creativit\`a.
\end{itemize}

\section{Test di Turing}
Il test di Turing consiste nel far dialogare una persona (interrogatore)
contemporaneamente con un'altra persona e con una macchina (chat bot) per cinque
minuti. Se il 30\% degli interrogatori non riesce a distinguere la persona dalla
macchina allora la macchina ha superato il test ed \`e quindi considerata
"\emph{intelligente}".

Questo test non \`e pi\`u cos\`i utilizzato per definire "\emph{intelligente}" una
macchina perch\'e non sempre l'intelligenza che vogliamo misurare \`e coerente col
test e perch\'e non sempre l'intelligenza artificiale si basa sulla simulazione.

\section{Breve storia}
Nel 1943 Mc Culloch e Pitts pubblicano il primo lavoro sulle reti neurali, nel 1949 arriva
il lavoro di Hebb. Nel 1956, durante la conferenza di Darthmouth viene coniato il termine
"\emph{Intelligenza Artificiale}".

Si raggiunge una \textbf{capacit\`a di ragionamento simbolico}, ovvero l'intelligenza riesce a
risolvere problemi utilizzando simboli di un determinato linguaggio. Per esempio \`e possibile
tradurre frasi in linguaggio naturale in proposizioni logiche che poi vengono trattate.

Si riesce anche a far giocare a scacchi una macchina che riesce a battere il campione mondiale
di scacchi. Il computer usato riusciva a valutare 200 milioni di mosse al secondo e conosceva
600.000 aperture.

In seguito a questi traguardi le aspettative erano alte: si riescono a fare programmi in grado
di dimostrare teoremi, sistemi basati su conoscenza (fornisco in anticipo delle conoscenze al
programma), vengono risolti problemi su dei cosidetti "\emph{micromondi}".

In seguito ci si rende conto che la manipolazione simbolica non \`e pi\`u adeguata a molti tipi
di problema, che la complessit\`a dei problemi va incontro ad una
\textbf{intrattabilit\`a computazionale}, che ci sono limiti alla rappresentazione delle rete
neurali di tipo semplice (\textbf{percettroni}).

Con l'arrivo dei \textbf{sistemi esperti} si riesce a risolvere tanti dei problemi a cui si era
andati incontro. Questi sistemi sono specializzati sul risolvere problemi specifici e conoscono
molto bene il dominio di questi ultimi. Questi sistemi tuttavia fallivano su problemi molto
banali poich\'e sprovveduti di \textbf{senso comune}, che comunque era qualcosa di molto
vasto e difficile da codificare.

Nel 1986 c'\`e il ritorno delle \textbf{reti neurali}, con algoritmi basati sulla
\textbf{retropropagazione} che si sostituiscono ai modelli a ragionamento simbolico,
dominanti all'epoca. Con questo ritorno si adottano sempre di pi\`u sistemi basati sul
\textbf{ragionamento probabilistico} e sull'\textbf{apprendimento automatico}.
