\chapter{Problemi di soddisfacimento di vincoli}
I \textbf{problemi di soddisfacimento di vincoli} (\textbf{CSP}) hanno una rappresentazione dello stato
\textbf{fattorizzata} ed esistono euristiche generali che consentono la risoluzione efficiente di tali problemi.

\section{Paradigma per problemi CSP}
Il problema \`e descritto da tre componenti:
\begin{itemize}
	\item Un insieme di variabili $X$.
	\item Un insieme di domini $D$, dove $D_i$ \`e l'insieme di valori possibili per $X_i$.
	\item Un insieme di vincoli $C$, espressi come relazioni tra le variabili.
\end{itemize}
Se volessimo formulare il problema nel modo visto fin ora abbiamo:
\begin{itemize}
	\item \textbf{Stato}: assegnamento (\textbf{parziale} o \textbf{completo}) di valori a variabili.
	\item \textbf{Stato iniziale}: assegnamento vuoto.
	\item \textbf{Azione}: assegnamento di un valore ad una variabile.
	\item \textbf{Soluzione}: assegnamento \textbf{completo} (le variabili hanno tutte un valore) e
	      \textbf{consistente} (i vincoli sono tutti soddisfatti).
\end{itemize}
In questo caso non avremo un grafo degli stati ma uno stato dei vincoli, in cui i nodi sono le variabili
e gli archi sono i vincoli.

\section{Strategie per problemi CSP}
Le strategie che possiamo adottare ora sono:
\begin{itemize}
	\item Utilizzare delle euristiche specifiche per questa classe di problemi.
	\item Fare delle \textbf{inferenze} che ci portano a restringere i domini e quindi a limitare la ricerca:
	      \textbf{propagazione dei vincoli}.
	\item Fare \textbf{backtracking intelligente}.
\end{itemize}

\section{Ricerca in problemi CSP}
La ricerca in questo tipo di problemi si basa sull'assegnazione, ad ogni passo, di un valore ad una variabile.
La massima profondit\`a di ricerca \`e data dal numero $n$ di variabili.

Ingenuamente si potrebbe pensare che l'\textbf{ampiezza dello spazio di ricerca} sia il prodotto della cardinalit\`a
dei domini. Se cos\`i fosse avremmo un \textbf{fattore di diramazione} che \`e uguale a: $nd$ al primo passo,
$(n-1)d$ al secondo e cos\`i via. Otteremmo quindi un fattore di diramazione complessivo di $n! \cdot d^n$.

Fortunatamente si pu\`o \textbf{ridurre drasticamente} il fattore di diramazione tenendo di conto che l'ordine con
cui si assegnano le variabili non conta (il \emph{goal} \`e \textbf{commutativo}).

\subsection{Backtracking a profondit\`a limitata}
Questo tipo di ricerca ha due punti cardine:
\begin{itemize}
	\item \textbf{Controllo anticipato} della violazione dei vincoli: \`e inutile andare avanti se un vincolo \`e
	      stato violato.
	\item La ricerca \`e \textbf{limitata} naturalmente in profondit\`a dal numero di variabili quindi il metodo
	      \`e \textbf{completo}.
\end{itemize}
L'algoritmo si compone di 8 passi:
\begin{enumerate}
	\item Controllo completezza assegnamento.
	\item Scelgo una delle variabili non ancora assegnate.
	\item Prendo un valore del dominio della variabile appena scelta e ne controllo la consistenza.
	\item Se \`e consistente
	      \begin{enumerate}
		      \item Assegno il valore alla variabile e aggiorno lo stato.
		      \item Aggiorno lo stato
		      \item Provo a fare inferenze per semplificare il problema.
	      \end{enumerate}
	\item Se sono riuscito a fare inferenza aggiorno ulteriormente lo stato.
	\item Chiamo ricorsivamente \verb|Backtrack| col nuovo assegnamento.
	\item Se l'algoritmo ha successo ritorno la soluzione. Altrimenti torno al punto 3.
	\item Se non ho trovato valori consistente da assegnare l'algoritmo fallisce.
\end{enumerate}
\begin{lstlisting}[style=pseudo-style]
Backtrack(assignment, csp)
	if assignment.isComplete() then
		return assignment;
	var = csp.unassignedVariable();
	foreach value in csp.domain(var, assignment)
		if value.isConsistent(var, assignment) then
			var = value;
			assignment.add(var);
			inf = csp.inference(var);
			if inf != failure then
				assignment.addInference(inf);
				result = Backtrack(assignment, csp);
				if result != failure then
					return result;
	return failure;
\end{lstlisting}

\subsection{Euristiche per Backtrack}
L'algoritmo svolge certi passaggi in maniera meccanica, con l'uso di qualche buona euristica possiamo andare a ridurre di molto
il problema. Le euristiche che proporremo si offrono di migliorare i seguenti punti.
\begin{itemize}
	\item La scelta della variabile da assegnare.
	\item La scelta dei valori del dominio.
	\item Inferenza: influenza di un assegnamento sulle altre variabili, restrizione dei domini, propagazione dei vincoli.
	\item Backtrack intelligente: evitare di ripetere i fallimenti.
\end{itemize}

\subsubsection{Scelta delle variabili}
\begin{itemize}
	\item \textbf{MRV} (\emph{Minimum Remaining Values}): si sceglie la variabile che ha meno valori legali residui, ovvero la
	      variabile \textbf{pi\`u vincolata}. Questa euristica ci fa scoprire prima i fallimenti.
	\item \textbf{Grado}: si sceglie la variabile coinvolta in pi\`u vincoli con le altre variabili, ovvero la variabile
	      \textbf{pi\`u vincolante}. In caso di parit\`a di MRV si usa questa euristica.
\end{itemize}

\subsubsection{Scelta dei valori}
\begin{itemize}
	\item Si sceglie il valore \textbf{meno vincolante}, ossia quello che esclude meno valori per le altre variabili direttamente
	      collegate con la variabile scelta.
	\item \textbf{Conflitti minimi}: si sceglie il valore che crea meno conflitti.
\end{itemize}
\`E sempre meglio valutare prima un assegnamento che ha pi\`u probabilit\`a di successo.


\subsubsection{Inferenza e propagazione dei vincoli}
\begin{itemize}
	\item \textbf{Verifica in avanti} (\emph{Forward Checking}): una volta assegnato un valore ad una variabile si possono eliminare
	      i valori incompatibili per le altre variabili direttamente collegate da vincoli.
	\item \textbf{Consistenza di nodo e arco}: si restringono i valori dei domini delle variabili tenendo conto dei vincoli unari e
	      binari su tutto il grafo.
\end{itemize}

\subsubsection{Backtrack intelligente}
Si considerano alternative solo per le variabili che hanno causato il fallimento (quelle assegnate in precedenza per cui potrebbere andare
bene anche altri valori), ovvero per il cosiddetto \textbf{insieme dei conflitti}.

\subsubsection{Riparazione euristica}
Si parte con un assegnamento completo ma inconsitente. Ad ogni passo si modifica l'assegnamento ad una variabile per cui un vincolo \`e violato.
