\chapter{Modelli lineari}
I modelli lineari sono molto utili per risolvere \textbf{problemi di regressione}, che, come abbiamo gi\`a anticipato
hanno una funzione obbiettivo che restituisce un valore target di tipo \textbf{numerico}.

Il nostro obbiettivo \`e quello di riuscire ad approssimare una funzione a valori reali e continui che potrebbe restituire
anche dati imprecisi o \textbf{rumorosi}. Un tipico esempio di coppia presente nel training set sar\`a del tipo:
\[ (x, f(x) + rumore) \]

\section{Regressione lineare univariata}
Per utilizzare modelli lineari dobbiamo fare riferimento ad un sottoinsieme della classe di problemi descritta sopra, ovvero
la \textbf{regressione lineare univariata}, in cui si considerano modelli composti da funzioni lineari, che hanno una sola
variabile $x$ in input e una sola variabile $y$ in output.

Per questa classe di problemi si assume un modello espresso come
\[ h(x) = w_1 x + w_0 \]
dove i $w_i$ sono coefficienti reali o parametri liberi detti \textbf{pesi}.

Quello che vogliamo fare \`e un'operazione di \textbf{fitting}, ossia tracciare una retta che passi il pi\`u vicino
possibile ad ognuno dei dati presenti nel training set. Nello specifico, vogliamo costruire un algoritmo di apprendimento
che riesca a stimare $w_1$ e $w_0$.