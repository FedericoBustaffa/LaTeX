\chapter{Agenti Intelligenti}
Un \textbf{agente intelligente} \`e definito come un qualcosa che interagisce con un
ambiente, riceve delle \textbf{percezioni} e risponde a queste percezioni con degli
\textbf{attuatori}.

\section{Caratteristiche degli agenti}
Un agente intelligente pu\`o avere diverse caratteristiche:
\begin{itemize}
	\item \`E \textbf{situato}, ovvero riceve \emph{percezioni} dall'ambiente e reagisce tramite
	      delle azioni.
	\item Hanno abilit\`a \textbf{sociale}, ovvero comunicano, collaborano e si difendono da
	      altri agenti.
	\item Hanno opinioni, obbiettivi e intenzioni.
	\item Sono \textbf{embodied}: hanno un corpo e forse provano emozioni.
\end{itemize}

\subsection{Percezioni e azioni}
\begin{itemize}
	\item La \textbf{percezione} \`e un input che proviene dai sensori.
	\item La \textbf{sequenza percettiva} \`e la storia completa delle percezioni.
	\item La \textbf{scelta delle azioni} \`e una funzione che dipende \emph{unicamente}
	      dalla sequenza percettiva.
	\item La \textbf{funzione agente} definisce l'azione da compiere per ogni sequenza
	      percettiva ed \`e implementata da un \textbf{programma agente}.
	\item Il compito dell'IA \`e progettare il programma agente.
\end{itemize}

\section{Agenti razionali}
Un \textbf{agente razionale} interagisce con il suo ambiente in maniera "efficace", ovvero
fa la cosa "giusta". Per valutare un agente serve un criterio oggettivo dell'effetto delle
azioni che questo compie. Si deve avere anche una misura delle prestazioni dell'agente
valutata su pi\`u ambienti, e basato sul come vogliamo che questo si evolva.

La razionalit\`a \`e relativa e dipende da molti fattori come:
\begin{itemize}
	\item La misura delle prestazioni.
	\item La conoscenza pregressa dell'ambiente.
	\item Le percezioni presenti e passate.
	\item Le capacit\`a dell'agente.
\end{itemize}
In definitiva, un agente razionale, per ogni sequenza di percezioni compie l'azione che
massimizza il valore atteso della misura delle prestazioni, considerando le sue percezioni
passate e la sua conoscenza pregressa.

Raramente tutta la conoscenza sull'ambiente pu\`o essere fornita a priori. L'agente deve
essere in grado di modificare il proprio comportamento con l'esperienza.
\subsection{Agenti autonomi}
Un agente \`e \textbf{autonomo} nella misura in cui il suo comportamento dipende dalla sua
capacit\`a di ottenere esperienza.

\section{Ambienti}
Per definire un problema per un agente si deve caratterizzare l'ambiente in cui questo opera.
Di seguito vediamo quali sono i criteri con cui un ambiente viene descritto.
\begin{itemize}
	\item \textbf{Osservabilit\`a}: \`e \textbf{completamente osservabile}, quando
	      l'apparato percettivo \`e in grado di dare una conoscenza completa dell'ambiente,
	      \`e \textbf{parzialmente osservabile} quando sono presenti limiti
	      o inaccuratezze sull'apparato percettivo.
	\item \textbf{Ambienti singoli e multiagente}: \`e \textbf{singolo} quando
	      pu\`o variare per via di eventi o di azioni di altri agenti.
	      \`E \textbf{multiagente competitivo} se pi\`u agenti si ostacolano a vicenda
	      oppure \textbf{multiagente cooperativo} nel caso gli agenti si aiutino l'un
	      l'altro per risolvere il problema.
	\item \textbf{Predicibilit\`a}: \`e \textbf{deterministisco}, nel caso in cui lo stato
	      successivo \`e completamente determinato dallo stato corrente e dall'azione. \`E
	      \textbf{Stocastico} se esistono elementi di incertezza con associata probabilit\`a.
	      \`E \textbf{non deterministisco} se si tiene traccia di pi\`u stati possibili ma
	      non in base ad una probabilit\`a.
	\item \textbf{Ambienti episodici e sequenziali}: \`e \textbf{episodico} se l'esperienze
	      dell'agente \`e divisa in episodi atomici indipendenti. \`E \textbf{sequenziale}
	      se ogni decisione influenza le successive.
	\item \textbf{Ambienti statici e dinamici}: \`e \textbf{statico} se non cambia mentre
	      l'agente decide l'azione. \`E \textbf{dinamico} se cambia nel tempo. \`E
	      \textbf{Semi-dinamico} se l'ambiente non cambia ma la valutazione dell'agente s\`i.
	\item \textbf{Ambienti discreti e continui}: pu\`o essere \textbf{discreto} o
	      \textbf{continuo} in base ai valori che si possono assumere.
	\item \textbf{Ambienti noti e ignoti}: pu\`o essere \textbf{noto} o \textbf{ignoto} in
	      base allo stato di conoscenza dell'agente (noto non significa osservabile).
\end{itemize}

\section{Struttura degli agenti}
Come abbiamo gi\`a detto un agente \`e composto da un'architettura di base e da un programma
agente. L'architettura serve a capatare le percezioni dall'ambiente e a rispondere agli
stimoli con gli attuatori. Il programma agente implementa in concreto la risposta basata
sulle percezioni.

\subsection{Programma agente}
In generale, il programma agente, prende in input delle percezioni e ritorna delle azioni.
Per scegliere la migliore azione possibile si basa sulla memoria delle percezioni precedenti.
Dopo avere compiuto l'azione si aggiorna la memoria con l'azione appena compiuta e si ritorna
l'azione scelta.

\subsection{Agenti risolutori di problemi}
Gli \textbf{agenti risolutori di problemi} adottano il paradigma della risoluzione di
problemi come \textbf{ricerca} in uno \textbf{spazio di stati}.

Sono agenti basati su modello e con obbiettivo che adottano una rappresentazione
\textbf{atomica} dello stato e pianificano l'intera sequenza di mosse prima di agire.

\subsubsection{Processo di risoluzione}
Il processo di risoluzione di un problema si compone di quattro passi principali:
\begin{enumerate}
	\item Determinazione dell'obbiettivo.
	\item Formulazione del problema, che consiste nella rappresentazione di
	      \begin{itemize}
		      \item stati
		      \item azioni
	      \end{itemize}
	\item Determinazione della soluzione mediante \textbf{ricerca}
	\item Esecuzione del piano.
\end{enumerate}

\subsubsection{Assunzioni sull'ambiente}
Per adesso assumiamo che l'ambiente sia:
\begin{itemize}
	\item Statico
	\item Osservabile
	\item Discreto
	\item Deterministico
\end{itemize}

\subsubsection{Formulazione del problema}
Un problema pu\`o essere definito formalmente con queste cinque componenti:
\begin{description}
	\item[Stato iniziale:] Lo stato da cui si parte.
	\item[Azioni possibili:] Dipendono dallo stato corrente e portano in un nuovo stato.
	\item[Modello di transizione:] Descrizione dello spazio degli stati e come si effettua
	      la transizione dall'uno all'altro.
	\item[Test obbiettivo:] Ho un insieme di stati obbiettivo e devo verificare se gli stati
	      che ho raggiunto sono quelli nell'insieme obbiettivo oppure no.
	\item[Costo del cammino:] \`E la somma del costo delle azioni (transizioni di stato).
\end{description}