\chapter{Validazione}
Con la \textbf{validazione} vogliamo trattare il problema di capire quando un modello \`e un \textbf{buon} modello e
anche capire la differenza tra usare algoritmi di Machine Learning e usare \textbf{bene} algoritmi di Machine Learning.

Ricordiamo che fare apprendimento, nello spazio delle generalizzazioni, vuol dire trovare una buona funzione in uno
spazio delle funzioni, a partire da dati noti per fare predizioni su dati non ancora analizzati. Per "buona" si intende
una funzione con una basse errore di generalizzazione.

In questo capitolo cerchiamo di rispondere alle domande:
\begin{itemize}
	\item Una funzione che generalizza bene su dati di training approssimer\`a bene anche su dati non ancora visti ?
	\item \`E una funzione valida ?
\end{itemize}

Nell'ambito della validazione gli obbiettivi principali sono due:
\begin{itemize}
	\item \textbf{Selezione del modello}: si cercano di stimare le performance dei modelli per riuscire a scegliere il
	      migliore.
	\item \textbf{Stima del modello}: una volta scelto il modello, si cerca di stimare l'errore su dati non ancora
	      analizzati. Si cerca di calcolare la \textbf{qualit\`a} del modello.
\end{itemize}
La regola d'oro \`e avere data set separati per ognuno di questi due obbiettivi. Ognuno fatto su misura per misurare
ci\`o di cui abbiamo bisogno.

\section{Approcci alla validazione}
Uno degli approcci standard per la validazione prevede di avere tre data set separati: \textbf{training set},
\textbf{validation set} e \textbf{test set}.
\begin{itemize}
	\item Il training set serve ad allenare il modello per fare un buon fitting dei dati.
	\item Il validation set viene usato su pi\`u modelli per valutare le performance di ognuno e scegliere il migliore
	      (selezione del modello).
	\item Il training set e il validation set a volte sono chiamati con un nome unico: \textbf{development set}. Insieme
	      servono a costruire ed individuare il modello finale che andremo ad usare.
	\item Il test set stima l'errore di generalizzazione (per il modello scelto) per dati non ancora analizzati.
\end{itemize}
Tra gli errori pi\`u comuni abbiamo
\begin{itemize}
	\item Usare un data set per qualcosa di diverso dal proposito per cui \`e stato creato.
	\item Usare una delle due stime fatte sui modelli come valore di riferimento in un ambito diverso da quello pensato.
\end{itemize}