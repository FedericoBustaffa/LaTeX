\part{Machine Learning}
\chapter{Introduzione}
Il \textbf{Machine Learning} \`e un'area che combina l'esigenza di creare macchine in grado di apprendere con i nuovi
strumenti adattivi e statistici che di continuo vengono migliorati o inventati.

Il Machine Learning nasce dalla necessit\`a che abbiamo di analizzare una crescente quantit\`a di dati empirici combinata
con la difficolt\`a che si ha nel \emph{"programmare l'intelligenza"} a priori. L'unica scelta possibile \`e creare
macchine in grado di apprendere ed evolversi in modo da rendersi adattive in maniera autonoma e dunque senza avere bisogno
di essere programmate per risolvere problemi troppo specifici.

Tra i principali obbiettivi abbiamo:
\begin{itemize}
	\item \textbf{Intelligenza artificiale}: costruire \textbf{sistemi intelligenti adattivi}.
	\item \textbf{Apprendimento statistico}: costruire \textbf{sistemi di analisi dati} predittivi e computazionalmente
	      efficaci.
	\item \textbf{Innovazione in varie aree}: costruire \textbf{modelli} come strumento per problemi interdisciplinari
	      complessi.
\end{itemize}

