\part{Machine Learning}
\chapter{Introduzione}
Il \textbf{Machine Learning} \`e un'area che combina l'esigenza di creare macchine in grado di apprendere con i nuovi
strumenti adattivi e statistici che di continuo vengono migliorati o inventati.

Il Machine Learning nasce dalla necessit\`a che abbiamo di analizzare una crescente quantit\`a di dati empirici combinata
con la difficolt\`a che si ha nel \emph{"programmare l'intelligenza"} a priori. L'unica scelta possibile \`e creare
macchine in grado di apprendere ed evolversi in modo da rendersi adattive in maniera autonoma e dunque senza avere bisogno
di essere programmate per risolvere problemi troppo specifici.

Tra i principali obbiettivi abbiamo:
\begin{itemize}
	\item \textbf{Intelligenza artificiale}: costruire \textbf{sistemi intelligenti adattivi}.
	\item \textbf{Apprendimento statistico}: costruire \textbf{sistemi di analisi dati} predittivi e computazionalmente
	      efficaci.
	\item \textbf{Innovazione in varie aree}: costruire \textbf{modelli} come strumento per problemi interdisciplinari
	      complessi.
\end{itemize}
In generale si cerca di creare modelli in grado di evolversi con una conoscenza a priori molto limitata ma sufficiente
da permettere al modello di imparare e ampliare tale conoscenza.

Il Machine Learning \`e in generale molto utile quando
\begin{itemize}
	\item La conoscenza o la teoria che sta dietro certi fenomeni \`e poca o in alcuni casi assente.
	\item C'\`e incertezza nei dati che possono essere incompleti o difficilmente decifrabili.
	\item Si trattano ambienti dinamici, ovvero non conosciuti a priori.
\end{itemize}

Ci sono tutta via dei requisiti perch\`e il processo sia svolto in maniera efficace:
\begin{itemize}
	\item Ci deve essere una fonte di apprendimento in modo da avere abbastanza dati per riuscire a costruire un
	      meccanismo che dia risultanti soddisfacenti.
	\item I dati in ingresso non possono essere troppo incompleti o troppo pochi. Dobbiamo avere una base solida da
	      cui iniziare.
\end{itemize}

Possiamo affermare a questo punto che ci sia bisogno di un nuovo paradigma computazionale differente da quello di
programmazione standard, in grado di trattare dati incerti e imprecisi. Tipicamente si parla di \textbf{soft computing} o
\textbf{intelligenza computazionale}.

L'obbiettivo \`e trovare soluzioni approssimative per problemi difficili da formalizzare scrivendo a mano un algoritmo.
Chiariamo che non si tratta di una metodologia approssimata, ma di un approccio rigoroso con forti basi
matematiche in grado di trovare funzioni/metodi di approssimazione a problemi complessi.

\section{Sistemi predittivi}
In generale si crea un modello che prende dei dati in input e in base a questi dati, esso muta e si evolve per poter
fare una \textbf{predizione} pi\`u accurata. Il processo di miglioramento del modello \`e guidato dal \textbf{compito}
che deve svolgere l'agente, dall'\textbf{algoritmo di apprendimento} e da un processo di \textbf{validazione} del modello.

\section{Apprendimento supervisionato}
Nell'\textbf{apprendimento supervisionato} abbiamo un vettore di valori in input che, attraverso
una funzione, viene associato a diverse categorie o valori reali di uscita. Tale funzione

Ci\`o che conosciamo quando ci approcciamo ad un problema di questo tipo \`e la coppia di input-output
\[ (x, d) \]
per una funzione sconosciuta $c$. Ovvero
\[ c(x) = d \]

Abbiamo poi un \textbf{supervisore} che prende un insieme di valori iniziali, forniti prima che inizi il processo di
apprendimento, e decide a priori dei \textbf{valori target} che soddisfino $c$. Chiariamo che il supervisore non usa
$c$ per assegnare i valori. Anche perch\'e $c$ \`e sconosciuta ed \`e la funzione che dobbiamo riuscire a ricavare tramite
l'apprendimento.

Ora vogliamo trovare una \emph{buona} approssimazione di $c$. Vogliamo cio\`e trovare un'\textbf{ipotesi} $h$ che
potr\`a essere usata per fare predizioni su dati ancora sconosciuti.

Il valore target pu\`o avere natura
\begin{itemize}
	\item \textbf{Categorica}: problema di \textbf{classificazione} in cui $c(x)$ ritorna quella che \`e la classe
	      (assunta) corretta per $x$. In questo caso abbiamo che $c(x)$ \`e un funzione a valori discreti da 1 a $k$.
	\item \textbf{Numerica}: problema di \textbf{regressione}. In questo caso invece approssimiamo la funzione con
	      valori reali continui in $\mathbb{R}$ o $\mathbb{R}^k$.
\end{itemize}

\section{Apprendimento non supervisionato}
In questo caso ci vengono dati una serie di dati iniziali in input ma non abbiamo un \textbf{supervisore} che etichetta
a priori il valore target di questi dati.

In questo caso si cerca di trovare dati simili per riuscire a dare un senso al vettore di input (\textbf{clustering}).

\section{Modelli}
Un \textbf{modello} vuole catturare o descrivere relazioni tra i dati in base ai compiti dati. Definisce inoltre la classe
di funzioni che possono essere implementate (\emph{spazio delle ipotesi}).

In totale un modello \`e composto dalle seguenti componenti:
\begin{itemize}
	\item \textbf{Esempio di allenamento}: si tratta di una serie di dati in input con relativo valore target di output
	      che pu\`o anche essere inaccurato o impreciso.
	\item \textbf{Funzione obbiettivo}: L'effettiva funzione $c$ che stiamo stimando.
	\item \textbf{Ipotesi}: Una certa funzione $h$ che sar\`a una stima della funzione $c$ basata sui dati raccolti e
	      sull'obbiettivo dato.
	\item \textbf{Spazio delle ipotesi}: L'insieme di tutte le possibili ipotesi $h$ che potrebbero in qualche modo essere
	      derivate dall'algoritmo.
\end{itemize}

\section{Algoritmi di apprendimento}
Basandoci sui dati in nostro possesso, sull'obbiettivo che abbiamo definito e sul modello vogliamo fare una
\textbf{ricerca euristica}, attraverso lo spazio delle ipotesi, della miglior ipotesi possibile.

L'ipotesi $h$ potrebbe non coincidere con l'insieme di tutte le possibili funzioni e la ricerca potrebbe non essere
esaustiva poich\'e ha bisogno bisogno di fare assunzioni (\textbf{bias induttivi}).

\subsection{Generalizzazione}
Per costruire un buon algoritmo di apprendimento dobbiamo introdurre il concetto di generalizzazione. Con
\textbf{apprendimento} intendiamo la ricerca di una \emph{buona} funzione in uno spazio di funzioni, minimizzando una
funzione di errore.

In questo caso chiamiamo \textbf{buona}, una funzione, che \textbf{generalizza} l'errore, ovvero misura
quanto accuratamente il modello fa previsioni in base a nuovi dati in ingresso.

La \textbf{fase di apprendimento} si basa sul costruire il modello in base ai dati e ai bias.

La \textbf{fase predittiva} \`e la fase in cui applichiamo il risultato dell'apprendimento ad un nuovo campione di
dati mai analizzati prima.