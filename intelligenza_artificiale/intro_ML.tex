\part{Machine Learning}
\chapter{Introduzione}
Il \textbf{Machine Learning} \`e un'area che combina l'esigenza di creare macchine in grado di apprendere con i nuovi
strumenti adattivi e statistici che di continuo vengono migliorati o inventati.

Il Machine Learning nasce dalla necessit\`a che abbiamo di analizzare una crescente quantit\`a di dati empirici combinata
con la difficolt\`a che si ha nel \emph{"programmare l'intelligenza"} a priori. L'unica scelta possibile \`e creare
macchine in grado di apprendere ed evolversi in modo da rendersi adattive in maniera autonoma e dunque senza avere bisogno
di essere programmate per risolvere problemi troppo specifici.

Tra i principali obbiettivi abbiamo:
\begin{itemize}
	\item \textbf{Intelligenza artificiale}: costruire \textbf{sistemi intelligenti adattivi}.
	\item \textbf{Apprendimento statistico}: costruire \textbf{sistemi di analisi dati} predittivi e computazionalmente
	      efficaci.
	\item \textbf{Innovazione in varie aree}: costruire \textbf{modelli} come strumento per problemi interdisciplinari
	      complessi.
\end{itemize}
In generale si cerca di creare modelli in grado di evolversi con una conoscenza a priori molto limitata ma sufficiente
da permettere al modello di imparare e ampliare tale conoscenza.

Il Machine Learning \`e in generale molto utile quando
\begin{itemize}
	\item La conoscenza o la teoria che sta dietro certi fenomeni \`e poca o in alcuni casi assente.
	\item C'\`e incertezza nei dati che possono essere incompleti o difficilmente decifrabili.
	\item Si trattano ambienti dinamici, ovvero non conosciuti a priori.
\end{itemize}

Ci sono tutta via dei requisiti perch\`e il processo sia svolto in maniera efficace:
\begin{itemize}
	\item Ci deve essere una fonte di apprendimento in modo da avere abbastanza dati per riuscire a costruire un
	      meccanismo che dia risultanti soddisfacenti.
	\item I dati in ingresso non possono essere troppo incompleti o troppo pochi. Dobbiamo avere una base solida da
	      cui iniziare.
\end{itemize}

Possiamo affermare a questo punto che ci sia bisogno di un nuovo paradigma computazionale differente da quello di
programmazione standard, in grado di trattare dati incerti e imprecisi. Tipicamente si parla di \textbf{soft computing} o
\textbf{intelligenza computazionale}.

L'obbiettivo \`e trovare soluzioni approssimative per problemi difficili da formalizzare scrivendo a mano un algoritmo.
Chiariamo che non si tratta di una metodologia approssimata, ma di un approccio rigoroso con forti basi
matematiche in grado di trovare funzioni/metodi di approssimazione a problemi complessi.

\section{Sistemi predittivi}
In generale si crea un modello che prende dei dati in input e in base a questi dati, esso muta e si evolve per poter
fare una \textbf{predizione} pi\`u accurata. Il processo di miglioramento del modello \`e guidato dal \textbf{compito}
che deve svolgere l'agente, dall'\textbf{algoritmo di apprendimento} e da un processo di \textbf{validazione} del modello.

\section{Apprendimento supervisionato}
In questo caso abbiamo un vettore in input che, attraverso una funzione, viene associato a diverse categorie o valori
reali di uscita.

Ci\`o che conosciamo quando ci approcciamo ad un problema di questo tipo \`e la coppia
\[ <input, output> = (x, d) \]
per una funzione sconosciuta $f$ (pi\`u precisamente si conoscono solo i punti dati dalle coppie date).

Abbiamo poi un \textbf{valore target} che \`e dato da un \textbf{supervisore} che a priori decide il valore dei dati
in input in relazione a $f(x)$.

Ora vogliamo trovare una \emph{buona} approssimazione per $f$. Per farlo usiamo un'\textbf{ipotesi} $h$ che potr\`a essere
usata per fare predizioni su dati ancora sconosciuti.