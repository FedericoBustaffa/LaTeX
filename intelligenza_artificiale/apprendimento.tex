\chapter{Apprendimento}
Iniziamo in questo capitolo a parlare di \textbf{apprendimento}, trattandolo come un problema di ricerca in uno
\textbf{spazio delle ipotesi}.

\section{Problemi di classificazione}
Nel caso di problemi di classificazione il nostro valore target \`e una variabile di tipo \textbf{categorico}. In questo
caso l'\textbf{apprendimento} si basa sul riuscire a inferire una funzione \textbf{booleana} da una serie di
esempi di allenamento.

\begin{definition}
	Si dice che un'ipotesi \textbf{soddisfa} un certo valore in input $x$ se $h(x) = 1$, cio\`e se \`e vera.
\end{definition}

\begin{definition}\label{def: consistente}
	Si dice che un'ipotesi $h$ \`e \textbf{consistente} con un esempio se
	\[ h(x) = c(x) \]
	ovvero se ritorna gli stessi valori che ritorna anche la funzione target.
\end{definition}

Nel caso di sole funzioni booleane lo spazio delle ipotesi ha dimensione
\[ |H| = 2^{2^n} \]
dove $n$ \`e la dimensione dell'input.

In realt\`a lavoreremo con uno spazio delle funzioni considerevolmente ristretto, rispetto allo spazio di tutte le
possibili funzioni, questo per ovvie ragioni di efficienza.

Per restringere lo spazio delle funzioni useremo
\begin{itemize}
	\item \textbf{Regole congiuntive} per spazi delle ipotesi finiti e discreti.
	\item \textbf{Funzioni lineari} per spazi delle ipotesi infiniti e continui.
\end{itemize}

\subsection{Spazio delle ipotesi discreto}
Vediamo come organizzare efficientemente la ricerca in uno spazio delle ipotesi finito e discreto tramite un algoritmo che
lavora su un insieme molto ristretto.

Tutto questo sotto due principali condizioni:
\begin{itemize}
	\item Si utilizzano \textbf{regole congiuntive}.
	\item Si assume che non ci sia \emph{rumore} nei dati.
\end{itemize}

\subsubsection{Regole congiuntive}
Tramite regole congiuntive prendo funzioni booleane che mettono in \emph{and} tutte le diverse variabili in input in tutti
i possibili modi, ottenendo cos\`i uno spazio delle ipotesi di dimensione
\[ |H| = 2^n \]
Come possiamo vedere abbiamo ridotto notevolmente lo spazio, che prima aveva una dimesione $|H|$ pari a $2^{2^n}$.

Se volessimo aggiungere tutti i possibili input negati (\textbf{not}) avremmo uno spazio delle ipotesi di dimensione
\[ |H| = 3^n + 1 \]

\subsubsection{Rappresentazione delle ipotesi}
Un ipotesi $h$ \`e rappresentata come una congiunzione di vincoli sugli attributi. In particolare ogni vincolo pu\`o
essere un valore:
\begin{itemize}
	\item \textbf{Specifico}
	\item \textbf{Significativo}
	\item \textbf{Vietato}
\end{itemize}
In generale strutturare lo spazio di ricerca (anche se infito) aiuta molto nella ricerca.

\begin{definition}
	Si dice che una certa ipotesi $h_j$ \`e \textbf{pi\`u generale} di $h_k$ se e solo se per ogni $x \in X$ vale che
	\[ h_k(x) = 1 \quad \Rightarrow \quad h_j(x) = 1 \]
	In questo caso si indica con
	\[ h_j \geq h_k \]
\end{definition}

Tramite la definizione appena data possiamo stabilire un \textbf{ordinamento parziale} utile per la ricerca.
Nello specifico vedremo un metodo in cui si analizzer\`a l'ipotesi pi\`u specifica, arrivando poi a quella pi\`u generale.

Si \emph{generalizzano} quindi, le ipotesi pi\`u specifiche in maniera \textbf{conservativa}, ovvero facendo s\`i che, per
ogni esempio positivo, anche la $h$ sia positiva in modo che sia anche \emph{consistente} con i dati iniziali forniti in
input (\emph{training set}).

Vogliamo per\`o generalizzare non pi\`u di quanto sia lo stretto necessario.

\subsection{Algoritmo Find-S}
L'algoritmo trova l'ipotesi pi\`u \textbf{specifica} in $H$ che \`e consistente con gli esempi di allenamento
\textbf{positivi}. L'ipotesi trovata \`e consistente anche con gli esempi negativi a patto che anche $c$ sia
contenuto in $H$. Questo perch\'e $c \geq h$
\begin{enumerate}
	\item Si inizializza \verb|h| con l'ipotesi pi\`u specifica.
	\item Per ogni istanza positiva \verb|x|:
	      \begin{itemize}
		      \item Per ogni attributo in \verb|h|:
		            \begin{itemize}
			            \item Se l'attributo \`e soddisfatto da \verb|x| non si fa niente.
			            \item Altrimenti si rimpiazza l'attributo con il vincolo pi\`u generale che \`e soddisfatto da
			                  \verb|x|.
		            \end{itemize}
	      \end{itemize}
	\item Si ritorna l'ipotesi \verb|h| trovata (la pi\`u specifica).
\end{enumerate}

\subsection{Version spaces}
L'idea \`e quella di generare una \emph{descrizione} dell'insieme di tutte le ipotesi $h$ consistenti con l'insieme di
allenamento $D$.

\`E possibile farlo senza enumerarle tutte dato che abbiamo una definizione generale per la \emph{consistenza} di
un'ipotesi (vedere definizione \ref{def: consistente}).

