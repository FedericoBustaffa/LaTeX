\chapter{Apprendimento}
Iniziamo in questo capitolo a parlare di \textbf{apprendimento}, trattandolo come un problema di ricerca in uno
\textbf{spazio delle ipotesi}.

\section{Problemi di classificazione}
Nel caso di problemi di classificazione il nostro valore target \`e una variabile di tipo \textbf{categorico}. In questo
caso l'\textbf{apprendimento} si basa sul riuscire a inferire una funzione \textbf{booleana} da una serie di
esempi di allenamento.

\begin{definition}
	Si dice che un'ipotesi \textbf{soddisfa} un certo valore in input $x$ se $h(x) = 1$, cio\`e se \`e vera.
\end{definition}

\begin{definition}\label{def: consistente}
	Si dice che un'ipotesi $h$ \`e \textbf{consistente} con un \textbf{valore target} di esempio se
	\[ h(x) = c(x) \]
	ovvero se ritorna gli stessi valori che ritorna anche la funzione obbiettivo per i valori del training set.
\end{definition}

Nel caso di sole funzioni booleane lo spazio delle ipotesi ha dimensione
\[ |H| = 2^{2^n} \]
dove $n$ \`e la dimensione dell'input. Questo perch\'e abbiamo $2^n$ possibili istanze delle $n$ variabili in input e
per ognuna di esse possiamo avere in output 2 valori booleani.

In realt\`a lavoreremo con uno spazio delle funzioni ristretto, rispetto allo spazio di tutte le
possibili funzioni.

Per restringere lo spazio delle funzioni useremo
\begin{itemize}
	\item \textbf{Regole congiuntive} per spazi delle ipotesi finiti e discreti.
	\item \textbf{Funzioni lineari} per spazi delle ipotesi infiniti e continui.
\end{itemize}

\subsection{Spazio delle ipotesi discreto}
Vediamo come organizzare efficientemente la ricerca in uno spazio delle ipotesi finito e discreto tramite un algoritmo che
lavora su un insieme molto ristretto.

Tutto questo sotto due principali condizioni:
\begin{itemize}
	\item Si utilizzano \textbf{regole congiuntive}.
	\item Si assume che non ci sia \emph{rumore} nei dati.
\end{itemize}

\subsubsection{Regole congiuntive}
Le \textbf{regole congiuntive} mi permettono di prendere solo funzioni booleane che mettono in \emph{and} le
variabili, ottenendo cos\`i uno spazio delle ipotesi di dimensione
\[ |H| = 2^n \]
Come possiamo vedere abbiamo ridotto notevolmente lo spazio, che prima aveva una dimesione $|H|$ pari a $2^{2^n}$.

Se volessimo aggiungere tutti i possibili input negati (\textbf{not}) avremmo uno spazio delle ipotesi di dimensione
\[ |H| = 3^n + 1 \]

\subsubsection{Rappresentazione delle ipotesi}
Un ipotesi $h$ \`e rappresentata come una congiunzione di vincoli sugli attributi. In particolare ogni vincolo pu\`o
essere un valore:
\begin{itemize}
	\item \textbf{Specifico}
	\item \textbf{Significativo}
	\item \textbf{Vietato}
\end{itemize}
In generale strutturare lo spazio di ricerca (anche se infito) aiuta molto nella ricerca.

\begin{definition}
	Si dice che una certa ipotesi $h_j$ \`e \textbf{pi\`u generale} di $h_k$ se e solo se per ogni $x \in X$ vale che
	\[ h_k(x) = 1 \quad \Rightarrow \quad h_j(x) = 1 \]
	In questo caso si indica con
	\[ h_j \geq h_k \]
\end{definition}

Tramite la definizione appena data possiamo stabilire un \textbf{ordinamento parziale} utile per la ricerca.
Nello specifico vedremo un metodo in cui si analizzer\`a l'ipotesi pi\`u specifica, arrivando poi a quella pi\`u generale.

Si \emph{generalizzano} quindi, le ipotesi pi\`u specifiche in maniera \textbf{conservativa}, ovvero facendo s\`i che, per
ogni esempio positivo, anche la $h$ sia positiva in modo che sia anche \emph{consistente} con i dati iniziali forniti in
input (\emph{training set}).

Vogliamo per\`o generalizzare non pi\`u di quanto sia lo stretto necessario.

\subsection{Algoritmo Find-S}
L'algoritmo trova l'ipotesi pi\`u \textbf{specifica} in $H$ che \`e consistente con gli esempi di allenamento
\textbf{positivi}. L'ipotesi trovata \`e consistente anche con gli esempi negativi a patto che anche $c$ sia
contenuto in $H$. Questo perch\'e $c \geq h$
\begin{enumerate}
	\item Si inizializza \verb|h| con l'ipotesi pi\`u specifica.
	\item Per ogni istanza positiva \verb|x|:
	      \begin{itemize}
		      \item Per ogni attributo in \verb|h|:
		            \begin{itemize}
			            \item Se l'attributo \`e soddisfatto da \verb|x| non si fa niente.
			            \item Altrimenti si rimpiazza l'attributo con il vincolo pi\`u generale che \`e soddisfatto da
			                  \verb|x|.
		            \end{itemize}
	      \end{itemize}
	\item Si ritorna l'ipotesi \verb|h| trovata (la pi\`u specifica).
\end{enumerate}

\subsection{Spazio delle versioni}
L'idea \`e quella di generare una \emph{descrizione} dell'insieme di tutte le ipotesi $h$ consistenti con l'insieme di
allenamento $D$.

\`E possibile farlo senza enumerarle tutte dato che abbiamo una definizione generale per la \emph{consistenza} di
un'ipotesi (vedere definizione \ref{def: consistente}).

\subsection{Algoritmo di eliminazione con lista}
Questo algoritmo segue i seguenti passi:
\begin{enumerate}
	\item Si crea lo spazio delle versioni, ovvero una lista contenente tutte le possibili ipotesi in $H$.
	\item Per ogni esempio di allenamento si rimuove dallo spazio delle versioni ogni ipotesi inconsistente con l'esempio
	      di allenamento.
	\item Si ritorna la lista di ipotesi nello spazio delle versioni.
\end{enumerate}
Questo algoritmo \`e tuttavia irrealistico poich\'e si dovrebbero enumerare tutte le possibili ipotesi in $H$.

\subsection{Rappresentazione dello spazio delle versioni}
Introduciamo ora un po' di notazione per riuscire a rappresentare lo spazio delle versioni in maniera pi\`u efficiente.

\begin{definition}
	Il \textbf{confine generale}, $G$, dello spazio delle versioni $V_{H, D}$ \`e l'insieme dei membri
	\emph{massimamente generali} delle ipotesi consistenti in $D$.
\end{definition}

\begin{definition}
	Il \textbf{confine specifico}, $S$, dello spazio delle versioni $V_{H, D}$ \`e l'insieme dei membri
	\emph{massimamente specifici} delle ipotesi consistenti in $D$.
\end{definition}

\begin{theorem}
	Ogni membro dello spazio delle versioni sta tra i due confini appena definiti.
\end{theorem}

\subsection{Algoritmo di eliminazione}
Sia $G$ il confine generale e $S$ il confine specifico, l'algoritmo funziona in questo modo:
\begin{enumerate}
	\item Per ogni esempio di allenamento positivo $d$.
	\item Rimuovo da $G$ ogni ipotesi inconsistente con $d$.
	\item Per ogni ipotesi $s$ in $S$ inconsistente con $d$ generalizzo $S$:
	      \begin{enumerate}
		      \item Rimuovo $s$ da $S$.
		      \item Aggiungo a $S$ tutte le minime generalizzazioni $h$ tali che:
		            \begin{itemize}
			            \item $h$ sia consistente con $d$.
			            \item Qualche membro di $G$ sia pi\`u generale di $h$.
		            \end{itemize}
		      \item Rimuovo da $S$ ogni ipotesi pi\`u generale delle ipotesi in $S$.
	      \end{enumerate}
\end{enumerate}

Si pu\`o scrivere l'algoritmo anche per esempi negativi. Come vedremo tra poco \`e speculare a quello appena definito:
\begin{enumerate}
	\item Per ogni esempio di allenamento negativo $d$.
	\item Rimuovo da $S$ ogni ipotesi inconsistente con $d$.
	\item Per ogni ipotesi $g$ in $G$ inconsistente con $d$ specializzo $G$:
	      \begin{enumerate}
		      \item Rimuovo $g$ da $G$.
		      \item Aggiungo a $G$ tutte le minime specializzazioni $h$ tali che:
		            \begin{itemize}
			            \item $h$ sia consistente con $d$.
			            \item Qualche membro di $S$ sia pi\`u specifico di $h$.
		            \end{itemize}
		      \item Rimuovo da $G$ ogni ipotesi pi\`u specifica delle ipotesi in $G$.
	      \end{enumerate}
\end{enumerate}

\section{Bias induttivi}
Per ora abbiamo deciso di predere in considerazione solo regole congiuntive. Il nostro spazio delle ipotesi \`e dunque
incapace di rappresentare semplici disgiunzioni. Nonostante questa grossa limitazione non ci \`e possibile liberarci
di tali regole.

L'idea \`e quella di scegliere un $H$ che esprime tutti i possibili concetti tramite congiunzioni, disgiunzioni e negazioni.
Quindi in $H$ sar\`a contenuta anche la funzione obbiettivo.

Dato che il nostro obbiettivo \`e generalizzare lanciamo l'algoritmo di eliminazione ma quello che succederebbe \`e che non
tutti i nuovi esempi verrebbero classficati in maniera ambigua. I soli ad essere classificati in maniera non ambigua
sarebbero gli esempi di allenamento iniziali.

\begin{theorem}
	Un apprendimento che non fa preferenze di alcun tipo non \`e in grado di generalizzare.
	\begin{proof}
		Ogni istanza non osservata sar\`a classificata positiva da met\`a delle ipotesi nello spazio delle versioni e
		negativa dall'altra met\`a.
	\end{proof}
\end{theorem}

L'apprendimento senza assunzioni a priori non ha basi razionali per classificare istanze non ancora viste. Le assunzioni
a priori non sono utilizzate per una questione di efficienza ma per il bisogno di \emph{generalizzare}.

\begin{definition}
	Sia $L$ un algoritmo di apprendimento, $X$ l'insieme delle istanze, $c$ la funzione obbiettivo, $D_c = \{<x, c(x)>\}$
	l'insieme di esempi di allenamento e sia $L(x_i, D_c)$ la classificazione assegnata ad un'istanza $x_i$ da $L$ dopo
	aver lavorato sull'insieme $D_c$. Chiamo \textbf{bias induttivo} di $L$ ogni minimo insieme $B$ di asserzioni tali che,
	per ogni funzione obbiettivo $c$ e per il corrispondente insieme di esempi di allenamento $D_c$.
	\[ (\forall x_i \in X) [B \wedge D_c \wedge x_i] \vdash L(x_i, D_c) \]
	dove $A \vdash B$ significa che $A$ implica logicamente $B$.
\end{definition}