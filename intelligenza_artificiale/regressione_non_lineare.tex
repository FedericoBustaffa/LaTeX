\chapter{Regressione non lineare}
Per approssimare meglio casi di regressione \`e molto conveniente passare a relazioni di tipo \textbf{non lineare}.
Fino ad ora abbiamo trattato modelli lineari espressi dalla formula
\[ h_w(x) = \sum_{i = 0}^n w_i x_i \]
con le variabili e coefficienti di grado 1. Adesso vogliamo passare a modelli pi\`u flessibili con una maggior
capacit\`a di approssimazione. Per farlo possiamo passare ad un polinomio di questo tipo
\[ h_w(x) = \sum_{i = 0}^n w_i x^i \]
Quella appena scritta \`e detta \textbf{regressione polinomiale} ma il modello resta \textbf{lineare}. Questo perch\'e
ci\`o che rende lineare un modello \`e la linearit\`a nei pesi $w$ e non nelle variabili $x$.

Precisiamo anche che la formula appena scritta vale solo per \textbf{regressioni polinomiali univariate}. Infatti se
notiamo, la variabile in ingresso \`e una sola ma ha grado $n$.

Questo \`e molto comodo e possiamo usare ancora l'algoritmo LMS per trovare una buona approssimazione della funzione
dato che il modello rimane lineare.

\section{Espansione lineare della base}
Quel che vogliamo fare ora \`e trovare modelli lineari ancora pi\`u flessibili e funzionanti per pi\`u variabili in
ingresso. Ecco che viene introdotta l'\textbf{espansione lineare di base}, la cui formula, ancora pi\`u generale della
precedente  \`e definita in questo modo:
\[ h_w(x) = \sum_{k = 0}^K w_k \phi_k(x) \]
dove $x$ equivale al pattern $x = [x_1, x_2, \dots, x_n]$ e dove $\phi$ \`e una funzione da
$\mathbb{R}^n$ in $\mathbb{R}$

Il modello appena rappresentato continua ad essere linerare nei coefficienti $w$ e anche in $\phi$ ma non lo \`e pi\`u
in $x$. Come gi\`a spiegato in precedenza, questo fa s\`i che il modello sia ancora lineare, consentendoci cos\`i di
utilizzare ancora l'algoritmo LMS per riuscire a trovare una buona approssimazione.

Quello che abbiamo introdotto ora ci consente un maggior grado di libert\`a nell'approssimazione della funzione ma ha
di contro il fatto che il modello potrebbe diventare \textbf{troppo complesso}. Dove per \emph{complessit\`a} intendiamo
il grado di libert\`a del modello.

Un modello \`e tanto pi\`u espressivo, quanto pi\`u \`e complesso.

\subsection{Complessit\`a dei modelli}
Ora che abbiamo piena libert\`a di espressione per definire modelli flessibili a piacere, andiamo incontro a nuovi
problemi. Uno di questi \`e il problema della scelta per la $\phi$. La scelta di $\phi$ \`e fondamentale per costruire
un buon modello e in base a questa scelta andiamo incontro ad altri due problemi: l'\textbf{underfitting} e
l'\textbf{overfitting}.

Il primo esprime la scarsa capacit\`a di approssimazione del modello, il quale ammette molto rumore sui dati di test.

Il secondo, invece, denota una capacit\`a di approssimazione del modello altissima sui dati di test. Talmente alta da
non ammettere rumore su di essi ($E(w) = 0$ sul training set). Ma per riuscire ad essere perfetto sui dati di test crea
una funzione che non riesce ad approssimare in maniera consistente la funzione obbiettivo.

\subsubsection{Individuare underfitting e overfitting}
Mentre per l'individuazione di un problema di underfitting baster\`a calcolare $E(w)$ per tutti i dati di training,
la quale restituir\`a un alto valore di errore, individuare un problema di overfitting sar\`a pi\`u difficile.
Questo perch\'e l'errore sui dati di training calcolato da $E(w)$ sar\`a nullo.

\section{Regolarizzazione}
La \textbf{regolarizzazione} \`e un approccio che permette di usare un modello complesso e in seguito
\emph{regolarizzarlo} per rendere migliore la sua capacit\`a di approssimazione mitigando il pi\`u possibile il
problema di overfitting.

In generale \`e meglio usare un modello pi\`u flessibile e regolarizzarlo in seguito, piuttosto che usare un modello
troppo rigido che poi, non pu\`o pi\`u essere migliorato.

Nello specifico andremo a parlare di \textbf{regressione Ridge} o \textbf{regolarizzazione di Tikhonov}. In questo caso
sarebbe pi\`u appropriato chiamare la funzione $E$ di errore $loss$ poich\'e nel problema di overfitting l'errore \`e
nullo o comunque minimo sui dati di training. Qui ci interessa sapere quanta precisione perdiamo nell'approssimare la
funzione obbiettivo e dunque la funzione $loss$ sar\`a definita in questo modo:
\[ E(w) = \sum_{p = 1}^l (y_p - h_w(x_p))^2 + \lambda \| w \|^2 \]
dove $\lambda$ \`e detto coefficiente di regolarizzazione e dove
\[ \| w \|^2 = \sum_{i} w_i^2 \]
L'effetto \`e quello di ridurre i pesi $w$ ottenendo cos\`i una nuova regola delta:
\[ w_{\text{new}} = w + \eta \Delta w - 2 \lambda w \]
Va specificato che tanto pi\`u $\lambda$ \`e alto tanto pi\`u il modello si semplifica. Dunque anche la scelta di un
$\lambda$ corretto determina quanto il modello venga regolarizzato.

Se scelgo un alto $\lambda$ rischio il problema dell'underfitting, se lo scelgo troppo basso rimango nel problema di
overfitting che stavo cercando di risolvere con la regolarizzazione.

Questo metodo ci permette di regolarizzare anche modelli di complessit\`a non nota. Strumento quindi molto utile dato
che non \`e sempre possibile sapere quanto il nostro modello sia complesso.

