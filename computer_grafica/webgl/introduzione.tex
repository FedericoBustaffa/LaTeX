
\chapter{Introduzione a WebGL}
\section{Ambiente di lavoro}
\subsection{HTML}
Prima di tutto costruiamo un semplice file html che servir\`a ad eseguire il codice JavaScript
all'interno di un browser.
\begin{lstlisting}[language=html5]
<html>

<head>
	<script src="code.js"></script>
</head>

<body>
	<canvas id="CANVAS" width="500px" height="500px" style="border: 1px solid black"></canvas>
</body>

</html>
\end{lstlisting}
In questo caso il file \verb|code.js| \`e il file JavaScript in cui andremo a lavorare.
L'id \verb|CANVAS| servir\`a a collegare il canvas col codice JavaScript.

\subsection{Inizializzare WebGL}
Per utilizzare WebGL ci servir\`a un oggetto di tipo \verb|WebGLRenderingContext|.
\begin{lstlisting}[language=javascript]
var gl = null;
\end{lstlisting}
Per inizializzarlo usiamo l'elemento canvas dichiarato nel codice html usando il suo id.
\begin{lstlisting}[language=javascript, firstnumber=2]
function initWebGL() {
	canvas = document.GetElementById("CANVAS");
	gl = canvas.getContext("webgl");
}
\end{lstlisting}
Ora serve indicare una funzione che venga eseguita appena la finestra viene caricata. Lo
facciamo in questo modo:
\begin{lstlisting}[language=javascript, firstnumber=6]
function run() {
	initWebGL();
}

window.onload = run;
\end{lstlisting}
questa funzione \verb|run| verr\`a ampliata in seguito.
Se eseguito tutto correttamente verr\`a visualizzato un quadrato bianco con il contorno nero.

\section{Scena}
Prima di iniziare a disegnare va chiarito che la nostra geometria viene visualizzata
all'interno della finestra canvas creata nel codice html. Nel nostro caso \`e un quadrato
di dimensione $500 \times 500$ pixel. Per orientarci con WebGL dobbiamo immaginare
un diagramma cartesiano di questo tipo:
\begin{center}
	\begin{tikzpicture}[scale=2.5]
		\draw[->, dashed]
		(-1, 0) node[above] {-1.0} --
		(1, 0) node[above] {1.0} node[below] {$x$};

		\draw[->, dashed]
		(0, -1) node[right] {-1.0} --
		(0, 1) node[right] {1.0} node[left] {$y$};
	\end{tikzpicture}
\end{center}
che si trova in mezzo al quadrato. Un punto che ha come valore delle ascisse -1.0 si trover\`a
esattamente sul lato sinistro del quadrato, al contrario un punto con ascissa a valore 1.0
si trova sul lato destro. Il discorso vale in modo analogo per l'asse delle ordinate.

Se avessi invece un rettangolo, per esempio di formato $1500 \times 500$ pixel, il discorso
non cambia, dovremo solo tenere di conto che il diagramma cartesiano di riferimento avr\`a
l'asse $x$ pi\`u lunga ma sempre con valori compresi tra -1 e 1.
\begin{center}
	\begin{tikzpicture}[scale=2.5]
		\draw[->, dashed]
		(-2.0, 0) node[above] {-1.0} --
		(2.0, 0) node[above] {1.0} node[below] {$x$};

		\draw[->, dashed]
		(0, -1) node[right] {-1.0} --
		(0, 1) node[right] {1.0} node[left] {$y$};
	\end{tikzpicture}
\end{center}