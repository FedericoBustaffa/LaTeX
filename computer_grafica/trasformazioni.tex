\chapter{Trasformazioni}
\section{Algebra vettoriale}
\subsection{Entit\`a geometriche}
Le entit\`a utilizzate sono \textbf{punti} e \textbf{vettori}, questi ultimi definiti da
\emph{modulo}, \emph{direzione} e \emph{verso}.

\subsection{Operazioni su punti e vettori}
Definiamo qualche operazione utile per queste due entit\`a:
\begin{itemize}
	\item punto $+$ vettore ritorna un punto
	\item punto $-$ punto ritorna un vettore
	\item vettore $+$ vettore ritorna un vettore
	\item scalare $\cdot$ vettore ritorna un vettore
\end{itemize}

\subsection{Prodotto scalare tra due vettori}
Il prodotto tra due vettori $a$ e $b$ scritto come $a \cdot b$ equivale a:
\[ a \cdot b = \sum_{i=1}^n a_1 b_1 + \dots + a_n b_n  \]
Questo significa che
\[ a \cdot b = \| a \| \| b \| \cos{(\theta)} \]
da qui posso ricavarmi l'angolo in questo modo:
\[ \theta = \arccos{\left( \frac{a \cdot b}{\| a \| \| b \|} \right)} \]
Altre propriet\`a:
\begin{itemize}
	\item Il prodotto scalare tra due vettori \`e 0 quando i due vettori sono \emph{ortogonali}
	      fra loro.
	\item Il prodotto scalare \`e massimo per vettori paralleli (stesso verso) ed \`e minimo
	      per vettori anti-paralleli (verso opposto).
\end{itemize}

\subsection{Prodotto vettoriale tra due vettori}


\section{Trasformazioni geometriche}
Una \textbf{trasformazione geometrica} \`e una funzione che mappa punti in punti e vettori
in vettori. Quando parliamo vogliamo \emph{trasformare} un oggetto intendiamo l'applicazione
della stessa trasformazione a tutti i punti dell'oggetto.

\subsection{Traslazione}

\subsection{Scalatura}

\subsection{Rotazione}

\subsection{Taglio}

\section{Trasformazioni con matrici}

\subsection{Composizione di trasformazioni}

\section{Frame}
