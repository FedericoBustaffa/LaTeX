\chapter{Camera}
Fino ad ora abbiamo trattato le cosiddette \textbf{matrici di modellazione}, il cui scopo \`e quello di applicare delle
trasformazioni direttamente alla geometria disegnata oppure a qualcuno dei nostri frame.

In questo capitolo tratteremo invece le \textbf{matrici di vista} e le \textbf{matrici di proiezione}. Esse servono ad
applicare trasformazioni al punto di vista o, in gergo, ad effettuare \textbf{movimenti di camera} e creare effetti di
prospettiva differenti tra loro.

Prima di addentrarci nell'argomento dobbiamo precisare che quando si parla di movimenti di camera, il sistema di riferimento
lungo il quale ci muoviamo non \`e quello solito. Mentre per gli assi $x$ e $y$ le cose non cambiano, l'asse $z$ si
sviluppa in maniera inversa.

Di norma, tanto pi\`u grande \`e il valore $z$ di un oggetto, tanto pi\`u questo oggetto si muover\`a "all'interno dello
schermo". Ma nel caso volessimo spostare il punto di vista, ad un valore pi\`u grande di $z$, corrisponderebbe un
allontanamento dalla scena (come se ci stessimo muovendo all'indietro).