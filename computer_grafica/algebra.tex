\chapter{Algebra} \label{algebra}
Per comprendere al meglio gli argomenti trattati \`e necessaria una conoscenza di base
dell'algebra lineare e della trigonometria. Per quanto riguarda l'algebra, verranno largamente usati
\textbf{vettori} e \textbf{matrici}. Ci si limiter\`a inoltre ad un uso di base delle funzioni trigonometriche.

Un primo approccio sar\`a relativo al mondo bidimensionale, in seguito ci sposteremo in quello tridimensionale.
Non tratteremo dunque vettori e matrici molto grandi o comunque di dimensioni tali da non poter essere rappresentabili.

\section{Algebra vettoriale}
\subsection{Operazioni su punti e vettori}
Iniziamo col definire qualche operazione utile per \textbf{punti} e \textbf{vettori}:
\begin{itemize}
	\item La differenza tra due punti \`e uguale a un vettore
	      \[ p_1 - p_2 = v \]
	\item La somma tra un punto e un vettore \`e uguale a un punto
	      \[ p + v_1 = v_2 \]
	\item La somma tra due vettori \`e uguale a un vettore
	      \[ v_1 + v_2 = v_3 \]
	\item Il prodotto tra uno scalare e un vettore \`e uguale a un vettore
	      \[ \lambda \cdot v_1 = v_2 \]
\end{itemize}

\subsection{Prodotto scalare tra due vettori}
Il \textbf{prodotto scalare} tra due vettori $a$ e $b$, scritto come $a \cdot b$, equivale a
\[ a \cdot b = \sum_{i=1}^n a_1 b_1 + \dots + a_n b_n  \]
dove $a_i$ indica l'$i$-esima componente del vettore $a$.

Un altro strumento utile \`e la \textbf{norma} di un vettore $v$ ed \`e definita come segue
\[ \| v \| = \sqrt{v_1^2 + v_2^2 + \dots + v_n^2} \]
Ora possiamo definire il prodotto scalare in un altro modo utile per trattare gli angoli compresi tra due vettori, ossia:
\[ a \cdot b = \| a \| \| b \| \cos{(\theta)} \]
da qui posso ricavarmi l'angolo $\theta$ compreso tra $a$ e $b$ in questo modo:
\[ \theta = \arccos{\left( \frac{a \cdot b}{\| a \| \| b \|} \right)} \]
Altre propriet\`a:
\begin{itemize}
	\item Il prodotto scalare tra due vettori \`e 0 quando i due vettori sono \emph{ortogonali}
	      fra loro.
	\item Il prodotto scalare \`e massimo per vettori paralleli (stesso verso) ed \`e minimo
	      per vettori anti-paralleli (verso opposto).
	\item Il prodotto scalare gode della propriet\`a commutativa.
	      \[ a \cdot b = b \cdot a \]
\end{itemize}

\subsection{Prodotto vettoriale tra due vettori}
Il \textbf{prodotto vettoriale} tra due vettori $a$ e $b$ di $n$ componenti, scritto come $a \times b$, \`e un operatore
binario che restituisice un altro vettore lungo $n$ e ortogonale sia ad $a$ che a $b$. Per calcolare questo vettore
baster\`a calcolare il determinante della matrice
\[
	\begin{bmatrix}
		i_1 & i_2 & \dots & i_n \\
		a_1 & a_2 & \dots & a_n \\
		b_1 & b_2 & \dots & b_n
	\end{bmatrix}
\]
Abbiamo detto che tratteremo solo vettori tridimensionali, quindi possiamo ridurci al caso del calcolo del determinante
per la matrice
\[
	\begin{bmatrix}
		i   & j   & k   \\
		a_x & a_y & a_z \\
		b_x & b_y & b_z
	\end{bmatrix}
\]
Il calcolo ci dar\`a come risultato
\[
	\det{
		\left(
		\begin{bmatrix}
			i   & j   & k   \\
			a_x & a_y & a_z \\
			b_x & b_y & b_z
		\end{bmatrix}
		\right)
	} = i(a_y b_z - b_y a_z) + j(-a_x b_z + b_x a_z) + k(a_x b_y - b_x a_y)
\]
Che possiamo riscrivere in forma pi\`u compatta in questo modo:
\[
	a \times b = \begin{bmatrix}
		a_y b_z - b_y a_z \\ -a_x b_z + b_x a_z \\ a_x b_y - b_x a_y
	\end{bmatrix}
\]
Non vale la propriet\`a commutativa ma vale la distributiva per la somma.
\[ a \times (b + c) = a \times b + a \times c \]
La norma equivale a
\[ \| a \times b \| = \| a \| \| b \| \sin(\theta) \]

\subsection{Coordinate polari}
Le \textbf{coordinate polari} servono a esprimere punti e vettori tramite la tupla
$(\theta, \rho)$ dove
\begin{itemize}
	\item $\theta$ \`e l'angolo formato con l'asse $x$ dal vettore che va dall'origine al punto.
	\item $\rho$ \`e la distanza del punto dall'origine.
\end{itemize}
Di seguito due formule per passare da coordinate cartesiane a polari.
\begin{gather*}
	\theta = \arctan(2 \cdot (y, x)) \\
	\rho = \sqrt{x^2 + y^2}
\end{gather*}
Viceversa se vogliamo passare da coordinate polari a cartesiane ci baster\`a usare queste formule
\begin{gather*}
	x = \rho \cdot \cos(\theta) \\
	y = \rho \cdot \sin(\theta)
\end{gather*}
Quanto visto funziona per punti e vettori nello spazio bidimensionale. Nulla vieta di estendere il discorso allo spazio
tridimensionale ma avremo bisogno di un dato in pi\`u ovvero l'angolo compreso il vettore e l'asse $y$ (o $z$).

\subsection{Prodotto riga per colonna tra matrici}
Siano $A_{n \times m}, B_{m \times r}$ due matrici, il prodotto riga per colonna
sar\`a una matrice $C$ di formato $n \times r$. Per calcolare l'elemento $c_{ij}$
\[ c_{ij} = \sum_{k = 1}^m a_{i,k} \cdot b_{k,j} \]
Nel caso in cui si voglia moltiplicare una matrice $M$ di formato $p \times q$ e un
vettore $v$ di lunghezza $q$ si pu\`o vedere il prodotto come una combinazione lineare
delle colonne della matrice per le singole componenti del vettore.
\begin{gather*}
	M \cdot v =
	\begin{bmatrix}
		m_{11} & \dots & m_{1p} \\
		\dots  & \dots & \dots  \\
		m_{q1} & \dots & m_{qp}
	\end{bmatrix} \cdot
	\begin{bmatrix}
		v_1 \\ \dots \\ v_q
	\end{bmatrix} \\
	= v_1 \begin{bmatrix}
		m_{11} \\ \dots \\ m_{q1}
	\end{bmatrix} \cdot
	... \cdot
	v_q \begin{bmatrix}
		m_{1p} \\ \dots \\ m_{qp}
	\end{bmatrix}
\end{gather*}
\textbf{NOTA}: Il prodotto riga per colonna \textbf{non} gode della propriet\`a
commutativa, dunque la propriet\`a appena vista vale solo se la moltiplicazione viene
svolta con quell'ordine e il vettore $v$ \`e un vettore colonna.

Vedremo pi\`u avanti che la non commutativit\`a di questa operazione sar\`a fondamentale
per definire diversi tipi di trasformazione.