\section{Trasformazioni tridimensionali}
Tutto ci\`o che \`e stato detto fin ad ora per lo spazio bidimensionale va esteso al mondo
tridimensionale. Per farlo ci basta aggiungere una riga e una colonna alla matrice di
trasformazione.

D'ora in poi avremo a che fare con matrici $4 \times 4$ e vettori a 4 dimensioni.

\subsection{Rotazione in 3D}
Nello spazio tridimensionale \`e necessario determinare intorno a quale dei tre assi ruotare.
Quel che abbiamo fatto fino ad ora nello spazio bidimensionale era ruotare la nostra
geometria intorno all'asse $z$.

D'ora in poi le matrici di trasformazione per la rotazione cambieranno in questo modo:
\begin{itemize}
	\item Rotazione intorno all'asse $x$
	      \[
		      R_x(\theta) = \begin{bmatrix}
			      1 & 0            & 0             & 0 \\
			      0 & \cos(\theta) & -\sin(\theta) & 0 \\
			      0 & \sin(\theta) & \cos(\theta)  & 0 \\
			      0 & 0            & 0             & 1
		      \end{bmatrix}
	      \]
	\item Rotazione intorno all'asse $y$
	      \[
		      R_x(\theta) = \begin{bmatrix}
			      \cos(\theta) & 0 & -\sin(\theta) & 0 \\
			      0            & 1 & 0             & 0 \\
			      \sin(\theta) & 0 & \cos(\theta)  & 0 \\
			      0            & 0 & 0             & 1
		      \end{bmatrix}
	      \]
	\item Rotazione intorno all'asse $z$
	      \[
		      R_x(\theta) = \begin{bmatrix}
			      \cos(\theta) & -\sin(\theta) & 0 & 0 \\
			      \sin(\theta) & \cos(\theta)  & 0 & 0 \\
			      0            & 0             & 1 & 0 \\
			      0            & 0             & 0 & 1
		      \end{bmatrix}
	      \]
\end{itemize}

\subsubsection{Rotazione intorno ad un asse generico}
\begin{itemize}
	\item Il primo procedimento \`e analogo a quello nello spazio bidimensionale.
	      \begin{enumerate}
		      \item Considero un frame tale per cui l'asse $z$ \`e l'asse attorno al quel
		            voglio far ruotare la mia figura.
		      \item Applico una trasformazione che porti tale frame a combaciare col frame
		            canonico.
		      \item Effettuo la rotazione intorno all'asse $z$ del frame canonico.
		      \item Riporto il frame nella sua posizione iniziale.
	      \end{enumerate}
	\item Il secondo procedimento usa la formula di Rodriguez per ottenere il punto $p'$
	      \[ p' = \cos(\theta) p + (1 - \cos(\theta))(p \cdot r) r + \sin(\theta)(r \times p) \]
\end{itemize}

\subsubsection{Rotazione con i quaternioni}
Un altro strumento che ci permette ci effettuare rotazioni in 3D \`e un'estensione del numero
complesso, ossia il \textbf{quaternione}.
