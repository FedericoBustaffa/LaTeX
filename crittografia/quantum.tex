\chapter{Crittografia quantistica}
In questo capitolo non andremo a parlare di macchine quantistiche ma andremo a vedere gli effetti della
\textbf{meccanica quantistica} sulla crittografia.

Esistono infatti dei protocolli crittografici che sfruttano alcuni dei suoi principi per lo scambio delle chiavi in
contesti in cui \`e richiesta estrema sicurezza e ai quali si affianca un One-Time Pad come cifrario simmetrico per
le comunicazioni.

\section{Meccanica quantistica}
Introduciamo alcuni principi di meccanica quantistica necessari a comprendere il protocollo
\begin{itemize}
	\item \textbf{Sovrapposizione}: propriet\`a di un sistema quantistico di trovarsi in diversi stati
	      contemporaneamente.
	\item \textbf{Decoerenza}: la misurazione di un sistema quantistico disturba il sistema. Il sistema disturbato
	      perde la sovrapposizione degli stati e collassa in uno stato singolo.
	\item \textbf{No Cloning}: impossibilit\`a di copiare uno stato quantistico non noto.
	\item \textbf{Entanglement}: possibilit\`a che due o pi\`u elementi si trovino in uno stato quantistico correlato
	      in modo che, anche se portati a grande distanza, mantengono la correlazione.
\end{itemize}

\subsection{Fotoni polarizzati}
Per comprendere al meglio il protocollo che vedremo in seguito dobbiamo anche parlare di \textbf{fotoni polarizzati}.

Un fotone ha una propriet\`a, detta \textbf{polarizzazione}, che pu\`o assumere due valori e che pu\`o essere misurata
facendo riferimento ad una \textbf{base}, anch'essa di due tipi:
\begin{itemize}
	\item \textbf{Ortogonale}: si indica con $+$ e pu\`o assumere valore \textbf{verticale} (a $90^\circ$), indicato con
	      $v$ oppure \textbf{orizzontale} (a $0^\circ$), indicato con $h$.
	\item \textbf{Diagonale}: si indica con $\times$ e pu\`o valere $+45^\circ$ o $-45^\circ$.
\end{itemize}

\subsubsection{Misurazione}
Non \`e possibile distinguere, con una misura, uno dei quattro casi: si deve scegliere una delle due basi e, dopo la
misura, \`e possibile distinguere uno dei due casi relativi alla base scelta.

Per la misurazione viene usato uno strumento (\textbf{Polarizing Beam Splitter}) il quale, una volta scelta la base di
riferimento, misura il valore di polarizzazione relativo alla base.

Tuttavia, la misurazione \`e corretta solo nel caso in cui il fotone sia preparato con la stessa base del PBS. Nel caso
in cui la base non sia quella corretta si hanno pari probabilit\`a di misurare uno dei due valori.

Il PBS ha due uscite, A ed R, che stanno rispettivamente per \emph{assorbimento} e \emph{riflessione}, e un asse di
polarizzazione S. Anche il fotone ha un suo asse di polarizzazione F e indichiamo con $\theta$ l'angolo che questo
forma con S.

In fase di misurazione si hanno due possibili scenari:
\begin{itemize}
	\item Il fotone esce dall'uscita A con probabilit\`a $\cos^2 \theta$ e assume polarizzazione S.
	\item Il fotone esce dall'uscita R con probabilit\`a $\sin^2 \theta$ e assume polarizzazione perpendicolare a S
	      ($\text{S}^\perp$).
\end{itemize}
Procediamo per casi supponendo di usare un PBS impostato su base ortogonale e quindi con S = 0:
\begin{itemize}
	\item Se $\theta = 0$ (F = S), allora il fotone esce da A con probabilit\`a 1 con polarizzazione S = F.
	\item Se $\theta = 90^\circ$, allora il fotone esce da R con probabilit\`a 1 con polarizzazione
	      $\text{S}^\perp = \text{F}$.
	\item Se $\theta = \pm 45^\circ$, allora il fotone esce con pari probabilit\`a da A o da R con polarizzazione
	      a $0^\circ$ o $90^\circ$.
\end{itemize}
Dunque la lettura ha distrutto lo stato quantistico precedente, il quale non \`e pi\`u recuperabile.

Per fare riferimento ai principi elencati all'inizio, possiamo dire che si \`e perso lo stato di \emph{sovrapposizione}
e si \`e dunque verificata una situazione di \emph{decoerenza}.

Dato che il protocollo prevede l'invio di fotoni polarizzati tra i due utenti, l'azione di crittoanalista che tenta di
intercettare la comunicazione lascia tracce, proprio perch\'e rompe lo stato di \emph{sovrapposizione}.

\section{BB84}
Ideato da Bennet e Brassard, \`e un protocollo per lo scambio di chiavi che fa uso di fotoni polarizzati.

Supponiamo che un utente $A$ voglia comunicare con l'utente $B$ tramite il protocollo BB84. Una prima fase viene
effettuata sul \textbf{canale quantistico}, al quale \`e associato un valore QBER (Quantum Bit Error Rate), che indica
la percentuale di errori dovuti al rumore.
\begin{enumerate}
	\item L'utente $A$ genera un sequenza iniziale di $n$ bit $S_A$, da cui sar\`a estratta la chiave (rappresentata
	      con un codice a correzione di errori).
	\item Per $n$ volte
	      \begin{enumerate}[label=(\Roman*)]
		      \item $A$ sceglie una base a caso, codifica $S_A[i]$ e invia il fotone a $B$.
		      \item $B$ sceglie una base a caso, interpreta il fotone ricevuto e costruisce $S_B[i]$.
	      \end{enumerate}
\end{enumerate}
Queste due sequenze coincidono con certezza dove le basi scelte coincidono. Al contrario, dove le basi non coincidono
i bit avranno pari probabilit\`a di coincidere o di non coincidere.

A questo punto i due utenti passano ad una comunicazione sul canale standard e si accordano su una funzione hash
crittografica $h$.
\begin{enumerate}
	\item $B$ comunica ad $A$ le basi scelte per ogni bit.
	\item $A$ risponde a $B$ comunicando le basi comuni.
	\item $A$ e $B$ estraggono due sottosequenze $S_A'$ e $S_B'$ di $S_A$ ed $S_B$, corrispondenti alle basi comuni.
	\item $A$ e $B$ estraggono due sottosequenze di $S_A''$ ed $S_B''$ da $S_A'$ ed $S_B'$ in posizioni concordate per
	      accertarsi se c'\`e stato l'intervento di un crittoanalista.
	\item $A$ e $B$ si scambiano $S_A''$ e $S_B''$ e se la percentuale di bit diversi \`e maggiore di QBER allora
	      possiamo dedurre che ci sia stato l'intervento di un crittoanalista e dunque si ripete tutto il procedimento
	      da capo.
	\item $A$ e $B$ calcolano rispettivamente $S_A' \backslash S_A''$ e $S_B' \backslash S_B''$, le decifrano con un
	      codice di correzione degli errori e ottengono una sequenza comune $S_C$.
	\item $A$ e $B$ calcolano $h(S_C)$ e la usano come chiave.
\end{enumerate}

\subsection{Sicurezza}
Il crittoanalista vuole scoprire i bit della chiave e per farlo deve misurare la polarizzazione dei fotoni in transito
sul canale. Dato che nessuno apparte $A$ sa quale sia la base usata, il crittoanalista
\begin{enumerate}
	\item Intercetta il fotone.
	\item Lo misura con una base scelta a caso.
	\item Lo invia a $B$.
	\item Costruisce la sua sequenza $S$ (non necessariamente per ogni fotone inviato).
\end{enumerate}
Cos\`i facendo, nel caso in cui il crittoanalista abbia usato la stessa base di $A$, il fotone non viene perturbato
nel suo stato di polarizzazione, ma nel caso in cui la base sia diversa, il fotone cambia la sua polarizzazione.

Se fosse possibile copiare lo stato quantistico di un fotone sarebbe possibile, per il crittoanalista, copiare il
lo stato del fotone prima di misurarlo, inviare la copia a $B$ e poi misurare il fotone originale. Come detto
all'inizio del capitolo, il principio di \emph{no cloning} non permette che ci\`o avvenga.

Il crittoanalista, per provare a confondere le perturbazioni dovute alle misure che fa sui fotoni, potrebbe decidere
di intercettarne solo alcuni e nel caso riuscisse passare inosservato, conoscerebbe alcuni bit della chiave.

Questo \`e potenzialmente molto pericoloso dato che, per ogni bit conosciuto, lo spazio delle chiavi si dimezza e quindi
un attacco a \emph{forza bruta} molto meno costoso.

Ecco perch\'e $A$ e $B$ usano l'immagine hash di $S_C$: usando l'immagine hash, la sequenza cambia radicalmente e i bit
intercettati diventano dunque inutili per il crittoanalista.