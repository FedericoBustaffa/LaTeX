\chapter{Bitcoin}
Le normali valute come euro, dollari e cos\`i via sono gestite da una banca centrale, come la BCE per l'euro, la quale
gestisce il conio e altri parametri andando a regolare l'inflazione ecc.

Senza entrare nei dettagli finanziari, possiamo dunque dire che le valute comuni sono in mano a sistemi
\textbf{centralizzati}, i quali gestiscono tutto ci\`o che abbiamo detto prima.

Il sistema dei \textbf{bitcoin} nasce dall'esigenza di creare un sistema di valute digitali \textbf{decentralizzato},
ossia un sistema che \`e gestito solo dai suoi utenti.

\section{Introduzione}
Prima di addentrarci nell'argomento chiariamo cosa sia una valuta digitale e una criptovaluta:
\begin{itemize}
	\item \textbf{Valuta digitale}: valuta che esiste soltanto in forma digitale e che dunque, \`e utilizzabile
	      soltanto da un calcolatore.
	\item \textbf{Criptovaluta}: valuta digitale resa sicura grazie a tecniche di crittografia che rendono quasi
	      impossibile spendere due volte la stessa moneta.
\end{itemize}
Per capire il funzionamento di bitcoin dobbiamo prima introdurre due concetti: la transazione e il registro.
\begin{itemize}
	\item \textbf{Transazione}: passaggio di denaro fra due utenti.
	\item \textbf{Registro}: anche detto \textbf{libro contabile}, \`e un documento per il monitoraggio delle
	      transazioni.
\end{itemize}
Il bitcoin \`e una rete \emph{Peer-to-Peer} di nodi su cui gira il software \emph{bitcoin core}, con il quale
riescono a comunicare tra di loro e decidono come gestire le transazioni tra le monete.

Ogni nodo memorizza il registro di tutti i membri della rete. Quest'ultimo dev'essere quindi aggiornato periodicamente
in modo che sia uguale per tutti gli utenti. Affinch\'e questo avvenga il procedimento \`e il seguente
\begin{enumerate}
	\item Si sceglie un nodo \textbf{leader} tramite consenso: i nodi competono tra di loro cercando di risolvere un
	      problema complesso.
	\item Il \emph{leader} propone una nuova pagina del libro contabile sulla base delle varie richieste di transazione
	      tra gli utenti.
	\item Tutti i computer ricevono la pagina e, tramite la loro copia del registro, controllano che le transazioni
	      inserite siano corrette. Se lo sono accettano la pagina, altrimenti la rifiutano e si cerca un altro
	      \emph{leader}.
	\item Tutti i computer aggiornano il loro registro e il \emph{leader} viene ricompensato con un premio in bitcoin.
\end{enumerate}

\section{Blockchain}
Ogni pagina del registro \`e collegata alla precedente come in una lista linkata. Ogni pagina, detta \textbf{blocco}
(da qui il nome \textbf{blockchain}), ha un \textbf{header}, in cui sono presenti informazioni necessarie al
mantenimento della struttura dati, e un \textbf{body}, dove sono memorizzate le transazioni.

\subsection{Header}
Nell'\emph{header} sono presenti quattro parametri principali
\begin{itemize}
	\item \textbf{Timestamp}: chi crea la pagina mette un \emph{timestamp} nell'\emph{header} relativo al momento
	      della creazione di essa.
	\item \textbf{Nonce}: \`e il frutto della competizione di cui abbiamo parlato prima.

	      In pratica si fa una sorta di forza bruta cercando un valore \emph{nonce} tale che il valore della funzione
	      hash SHA256 calcolata sull'\emph{header} dell'ultima pagina della \emph{blockchain} e il \emph{nonce} sia
	      minore di un certo valore.

	      Trovare un \emph{nonce} di questo tipo equivale a vincere la competizione.
	\item \textbf{Merkle Root}: \`e la radice di un albero di Merkle costruito calcolando l'hash delle transazioni.
	      e serve a controllare che una certa transazione, all'interno della pagina, sia integra.
	\item \textbf{Hash previous block}: \`e il \emph{nonce} della pagina precedente.
\end{itemize}

\subsection{Transazioni}
Nel \emph{body} di un blocco sono salvate tutte le \textbf{transazioni} effettuate fino a quel momento.

Una transazione \`e un invio di bitcoin da un utente $A$ ad un utente $B$
\[ Tx(A, B, BTC) \]
In ambito bitcoin sono pi\`u complicate e fanno uso di firme digitali che hanno la propriet\`a di
\begin{itemize}
	\item \textbf{Autenticazione}: ogni volta che si effettua una transazione si deve essere in possesso di una coppia
	      di chiavi, privata e pubblica, che servono a produrre una firma digitale.
	\item \textbf{Integrit\`a}: una transazione non deve essere modificata da qualche attacco attivo o da problemi di
	      rete.
	\item \textbf{Non ripudio}: non si pu\`o negare di aver effettuato una transazione.
\end{itemize}
Supponiamo di avere due utenti $A$ e $B$ i quali possiedono rispettivamente le chiavi private $sk_A$ e $sk_B$ e le
chiavi pubbliche $pk_A$ e $pk_B$ e siano $Addr_A$ e $Addr_B$ rispettivamente i loro indirizzi, una transazione da $A$
a $B$ avviene in questo modo:
\begin{enumerate}
	\item $A$ firma la transazione con la sua chiave privata.
	\item $A$ invia il crittogramma a $B$.
	\item $B$ verifica che il messaggio sia inviato da $A$ tramite la chiave $pk_A$.
\end{enumerate}
Le transazioni vengono effettuate tramite programmi non Turing completi e la loro verifica coinvolge una terza parte
fidata che fa da \emph{garante}.

La transazione non avviene direttamente da $A$ a $B$ ma si invia il messaggio relativo alla transazione ad un
\textbf{indirizzo multisignature} che coinvolge questa terza parte fidata, la quale controlla che la transazione
sia correttamente effettuata e la firma.

Una transazione relativa ad un certo utente \`e strutturata come un blocco in cui sono presenti tre parametri
fondamentali:
\begin{itemize}
	\item \textbf{Input}: somma dei bitcoin ricevuti.
	\item \textbf{Output}: somma dei bitcoin inviati.
	\item \textbf{UTXO}: sta per \emph{Unspent Transaction Output} e sono i bitcoin non spesi che vengono comunque
	      mandati in output verso se stessi cos\`i da recuperarli.
\end{itemize}
Il risultato \`e la differenza tra input e output.

In definitiva, nessuno possiede dei bitcoin ma si possiede una chiave privata che consente di spendere bitcoin inviati
ad un certo indirizzo relativo ad una chiave pubblica. Se si perde la chiave privata si perdono tutti i bitcoin
legati ad essa.

\section{Conclusioni}
\subsection{Come ottenere bitcoin}
\subsection{Come utilizzare bitcoin}
\subsection{Valore del bitcoin}