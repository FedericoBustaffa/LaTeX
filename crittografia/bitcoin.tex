\chapter{Bitcoin}
Prima di parlare di bitcoin dobbiamo chiarire cosa sia una valuta digitale e una criptovaluta:
\begin{itemize}
	\item \textbf{Valuta digitale} \`e un valuta che esiste soltanto in forma digitale e che dunque, \`e
	      utilizzabile soltanto dal computer.
	\item \textbf{Criptovaluta} \`e una valuta digitale resa sicura grazie a tecniche di crittografia che rendono
	      quasi impossibile spendere due volte la stessa moneta.
\end{itemize}
\textbf{Bitcoin} \`e il primo esempio di criptovaluta.

\section{Introduzione}
Per capire il funzionamento di Bitcoin dobbiamo capire due concetti fondamentali
\begin{itemize}
	\item \textbf{Transazione}: passaggio di denaro fra due utenti.
	\item \textbf{Registro}: documento per il monitoraggio delle transazioni.
\end{itemize}
Bitcoin \`e una rete \emph{Peer-to-Peer} di nodi che eseguono il software \emph{Bitcoin core} con il quale comunicano
tra di loro e decidono come gestire le transazioni tra le monete. Ogni nodo memorizza il libro contabile di tutti i
membri della rete.

Il libro contabile dev'essere quindi aggiornato per tutti gli utenti ogni volta, per farlo
\begin{enumerate}
	\item Si sceglie un nodo \emph{leader} tramite consenso. I nodi competono tra di loro cercando di risolvere un
	      problema complesso.
	\item Il leader propone una nuova pagina del libro contabile sulla base delle varie richieste di transazione dei
	      vari utenti.
	\item Tutti i computer ricevono la pagina e, tramite la loro copia del registro, controllano che le transazioni
	      inserite siano corrette. Se lo sono accettano la pagina altrimenti la rifiutano e si cerca un altro leader.
	\item Tutti i computer aggiornano il loro registro e il leader viene ricompensato con un premio in Bitcoin.
\end{enumerate}

\section{Blockchain}
Ogni pagina del registro \`e collegata alla precedente come in una lista linkata. Ogni pagina, detta \textbf{blocco},
ha un \textbf{header}, in cui sono presenti informazioni necessarie al mantenimento della struttura dati, e un
\textbf{body}, dove sono memorizzate le transazioni.

Questa struttura dati \`e detta \textbf{blockchain}.

\subsection{Blocco}
Come abbiamo gi\`a detto, un blocco si divide in due parti pprincipali, l'\emph{header} e il \emph{body}.

\subsubsection{Header}
Nell'\emph{header} sono presenti quattro parametri principali
\begin{itemize}
	\item \textbf{Timestamp}: chi crea la pagina mette un \emph{timestamp} nell'\emph{header} relativo al momento
	      della creazione di essa.
	\item \textbf{Nonce}: \`e il frutto della competizione di cui abbiamo parlato prima.

	      In pratica si fa una sorta di forza bruta cercando un valore \emph{nonce} tale che il valore della funzione
	      hash SHA256 calcolata sull'\emph{header} dell'ultima pagina della \emph{blockchain} e il \emph{nonce} sia
	      minore di un certo valore.

	      Trovare un \emph{nonce} di questo tipo equivale a vincere la competizione.
	\item \textbf{Merkle Root}: \`e la radice di un albero di Merkle costruito calcolando l'hash delle transazioni.
	      e serve a controllare che una certa transazione, all'interno della pagina, sia integra.
	\item \textbf{Hash previous block}: \`e il \emph{nonce} della pagina precedente.
\end{itemize}

\subsubsection{Body}
Nel \emph{body} di un blocco sono salvate tutte le \textbf{transazioni} effettuate fino a quel momento.

Una transazione
\`e un invio di Bitcoin da un utente $A$ ad un utente $B$
\[ Tx(A, B, BTC) \]
In ambito Bitcoin sono pi\`u complicate e fanno uso di firme digitali che hanno la propriet\`a di
\begin{itemize}
	\item \textbf{Autenticazione}: ogni volta che si effettua una transazione si deve essere in possesso di una coppia
	      di chiavi, privata e pubblica, che servono a produrre una firma digitale.
	\item \textbf{Integrit\`a}: una transazione non deve essere modificata da qualche attacco attivo o da problemi di
	      rete.
	\item \textbf{Non ripudio}: non si pu\`o negare di aver effettuato una transazione.
\end{itemize}
Supponiamo di avere due utenti $A$ e $B$ i quali possiedono rispettivamente le chiavi private $sk_A$ e $sk_B$ e le
chiavi pubbliche $pk_A$ e $pk_B$ e siano $Addr_A$ e $Addr_B$ rispettivamente i loro indirizzi, una transazione da $A$
a $B$ avviene in questo modo:
\begin{enumerate}
	\item $A$ firma la transazione con la sua chiave privata.
	\item $A$ invia il crittogramma a $B$.
	\item $B$ verifica che il messaggio sia inviato da $A$ tramite la chiave $pk_A$.
\end{enumerate}