\chapter{Rappresentazione matematica di oggetti}\label{rappresentazione}
In crittografia vogliamo rappresentare degli oggetti, per farlo gli associamo un \textbf{nome}, ossia una stringa che
lo identifica in modo \textbf{univoco}. La stringa in questione \`e una sequenza di caratteri \textbf{ordinata}.

Viene fissato inizialmente un \textbf{alfabeto}, ossia un insieme di caratteri o simboli \textbf{finito} e vogliamo
rappresentare un oggetto tramite una sequenza di caratteri di questo alfabeto.

Possiamo osservare che, aumentando la lunghezza dei nomi che assegno ad ogni oggetto, il numero di oggetti che possiamo
rappresentare \`e, di fatto, infinito seppur l'alfabeto abbia un numero finito di caratteri.

\section{Alfabeti e sequenze}\label{alfabeti}
Introduciamo ora un po' di notazione per iniziare a rappresentare meglio gli alfabeti e le sequenze.

Possiamo definire un alfabeto $\Gamma$ di cardinalit\`a $s$ e, nel caso in cui dovessimo rappresentare $N$ oggetti,
vale che:
\begin{itemize}
	\item La lunghezza della sequenza pi\`u lunga che rappresenta un oggetto dell'insieme vale
	      \[ d(s, N) \]
	\item Il valore minimo di $d(s, N)$ tra tutte le rappresentazioni possibili vale
	      \[ d_{\min}(s, N) \]
	      Questo valore indica la lunghezza ideale per rappresentare gli $N$ oggetti, con un alfabeto di $s$ caratteri.
	\item Un metodo di rappresentazione \`e tanto migliore, quanto pi\`u $d(s, N)$ si avvicina a $d_{\min}(s, N)$.
	      Quindi, un metodo di rappresentazione, \`e migliore se usiamo stringhe "corte" per rappresentare gli oggetti.
\end{itemize}

\section{Rappresentazione unaria}\label{rappresentazione_unaria}
Prendiamo come primo esempio un tipo di rappresentazione \emph{sfavorevole}, che, nella pratica, non viene mai
utilizzata. Stiamo parlando della \textbf{rappresentazione unaria}.
\[ s = 1, \quad \Gamma = \{ 0 \} \]
L'unico modo per differenziare un nome dall'altro \`e aggiungere uno $0$ alla sequenza. Dunque, se dobbiamo
rappresentare $N$ oggetti, otteniamo
\[ d_{\min}(1, N) = N \]
La sequenza pi\`u lunga avr\`a anch'essa lunghezza
\[ d(s, N) = N \]

\section{Rappresentazione binaria}\label{rappresentazione_binaria}
Vediamo ora come l'aggiunta di un solo carattere ci faccia ottenere un metodo di rappresentazione migliore. Stiamo
parlando della \textbf{rappresentazione binaria}.
\[ s = 2, \quad \Gamma = \{ 0, 1 \} \]
Questo alfabeto ha due caratteristiche fondamentali:
\begin{itemize}
	\item Per ogni sequenza lunga $k$ con $k \geq 1$, abbiamo $2^k$ sequenze possibili.
	\item Il numero totale di sequenze lunghe da 1 a $k$ \`e dato da
	      \[ \sum_{i = 1}^k 2^i = 2^{k + 1} - 2 \]
\end{itemize}
Nel caso in cui volessimo rappresentare $N$ oggetti dobbiamo tenere di conto che
\[ 2^{k + 1} - 2 \geq N \]
con $k \geq \log (N + 2) - 1$. Possiamo quindi affermare che $d_{\min}(2, N)$ \`e il minimo intero $k$ che soddisfa
la relazione
\begin{gather*}
	d_{\min}(2, N) = \lceil \log (N + 2) - 1 \rceil \\
	\Downarrow \\
	\lceil \log N \rceil - 1 \leq d_{\min}(2, N) \leq \lceil \log N \rceil
\end{gather*}
Il numero $\lceil \log N \rceil$ sono i caratteri binari sufficienti per costruire $N$ sequenze differenti. In
sostanza possiamo costruire $N$ sequenze differenti, tutte di $\lceil \log N \rceil$ caratteri.

Per comodit\`a, \`e preferibile costruire $N$ sequenze, \emph{tutte} di $\lceil \log N \rceil$ caratteri e non $N$
sequenze \emph{al massimo} di $\lceil \log N \rceil$ caratteri.

\begin{example}
	Se volessimo rappresentare 7 oggetti con un alfabeto binario
	\[ N = 7, \quad \lceil \log 7 \rceil = 3 \]
	potremmo rappresentare i 7 oggetti come segue
	\[
		0 \quad
		1 \quad
		00 \quad
		01 \quad
		10 \quad
		11 \quad
		000
	\]
	\`E tuttavia pi\`u comodo rappresentare gli oggetti tramite sequenze tutte lunghe uguali come di seguito
	\[
		000 \quad
		001 \quad
		010 \quad
		100 \quad
		011 \quad
		101 \quad
		110 \quad
	\]
	cos\`i da non inserire caratteri di separazione che appesantirebbero la notazione.
\end{example}

\section{Rappresentazione generale}\label{rappresentazione_generale}
In generale, se si dispone di un alfabeto $\Gamma$ di cardinalit\`a $s$, possiamo costruire $N$ sequenze differenti,
tutte lunghe $\lceil \log_s N \rceil$ caratteri.

Per ottenere \textbf{rappresentazioni efficienti} si deve quindi usare un numero massimo di caratteri di
\textbf{ordine logaritmico} nella cardinalit\`a $N$ dell'insieme da rappresentare e tale che
\[ | \Gamma | \geq 2 \]

\subsection{Rappresentazione di interi}
La notazione posizionale per rappresentare numeri interi \`e una rappresentazione efficiente, indipedentemente dalla
base $s \geq 2$ scelta.

Un numero $N$ \`e rappresentato con un numero $d$ di cifre tale che
\[ \lceil \log N \rceil \leq d \leq \lceil \log N \rceil + 1 \]
C'\`e quindi una \textbf{riduzione logaritmica} tra il valore $N$ di un numero e la lunghezza $d$ della sua
rappresentazione.