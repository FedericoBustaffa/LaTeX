\chapter{Rappresentazione matematica di oggetti}\label{rappresentazione}
In crittografia vogliamo rappresentare ogni oggetto, per farlo gli associamo un \textbf{nome}, ossia una stringa che lo
identifica in modo \textbf{univoco}. La stringa in questione \`e una sequenza di caratteri \textbf{ordinata}.

Viene fissato inizialmente un \textbf{alfabeto}, ossia un insieme di caratteri o simboli \textbf{finito} e vogliamo
rappresentare un oggetto tramite una sequenza di caratteri di questo alfabeto.

Possiamo osservare che, aumentando la lunghezza dei nomi che assegno ad ogni oggetto, il numero di oggetti che posso
rappresentare \`e, di fatto, infinito seppur l'alfabeto abbia un numero finito di caratteri.

\section{Alfabeti e sequenze}\label{alfabeti}
Introduciamo ora un po' di notazione per iniziare a rappresentare meglio gli alfabeti e le sequenze.

Posso definire una alfabeto $\Gamma$ di cardinalit\`a $s$ e, nel caso in cui dovessi rappresentare $N$ oggetti, vale che:
\begin{itemize}
	\item La lunghezza della sequenza pi\`u lunga che rappresenta un oggetto dell'insieme vale
	      \[ d(s, N) \]
	\item Il valore minimo di $d(s, N)$ tra tutte le rappresentazioni possibili vale
	      \[ d_{\min}(s, N) \]
	      Questo valore indica la lunghezza ideale per rappresentare gli $N$ oggetti, con un alfabeto di $s$ caratteri.
	\item Un metodo di rappresentazione \`e tanto migliore, quanto pi\`u $d(s, N)$ si avvicina a $d_{\min}(s, N)$. Quindi
	      un metodo di rappresentazione \`e migliore se uso stringhe "corte" per rappresentare gli oggetti.
\end{itemize}

\section{Rappresentazione unaria}\label{rappresentazione_unaria}
Prendiamo come primo esempio un tipo di rappresentazione \emph{svaforevole}, che in pratica non viene mai utilizzato, che
\`e quello della \textbf{rappresentazione unaria}. In questo caso abbiamo
\[ s = 1, \quad \Gamma = \{ 0 \} \]
L'unico modo per differenziare un nome dall'altro \`e aggiungere uno $0$ alla sequenza. Dunque, se devo rappresentare $N$
oggetti ho che
\[ d_{\min}(1, N) = N \]
e la sequenza pi\`u lunga avr\`a anch'essa lunghezza $N$:
\[ d(s, N) = N \]

\section{Rappresentazione binaria}\label{rappresentazione_binaria}
Vediamo ora come l'aggiunta di un solo carattere ci faccia ottenere un metodo di rappresentazione migliore. In questo caso
abbiamo
\[ s = 2, \quad \Gamma = \{ 0, 1 \} \]
Questo alfabeto ha due caratteristiche fondamentali:
\begin{itemize}
	\item Per ogni sequenza lunga $k$ con $k \geq 1$, abbiamo $2^k$ sequenze.
	\item Il numero totale di sequenze lunghe da 1 a $k$ \`e dato da
	      \[ \sum_{i = 1}^k 2^i = 2^{k + 1} - 2 \]
\end{itemize}
Nel caso in cui avessi $N$ oggetti da rappresentare avrei che
\begin{itemize}
	\item $2^{k + 1} - 2 \geq N$
	\item $k \geq \log_2 (N + 2) - 1$
\end{itemize}
Posso quindi affermare che $d_{\min}(2, N)$ \`e il minimo intero $k$ che soddisfa la relazione:
\[ d_{\min}(2, N) = \lceil \log_2 (N + 2) - 1 \rceil \]
che implica
\[ \lceil \log_2 N \rceil - 1 \leq d_{\min}(2, N) \leq \lceil \log_2 N \rceil \]

Il numero $\lceil \log_2 N \rceil$ sono i caratteri binari sufficienti per costruire $N$ sequenze differenti. In
sostanza posso costruire $N$ sequenze differenti tutte di $\lceil \log_2 N \rceil$ caratteri.

Per comodit\`a, \`e preferibile costruire $N$ sequenze, \emph{tutte} di $\lceil \log_2 N \rceil$ caratteri e non $N$
sequenze \emph{al massimo} di $\lceil \log_2 N \rceil$ caratteri.

\begin{example}
	Se volessi rappresentare sette oggetti con un alfabeto binario avrei che
	\[ N = 7, \quad \lceil \log_2 7 \rceil = 3 \]
	e quindi potrei rappresentare i sette oggetti come segue
	\[
		0 \quad
		1 \quad
		00 \quad
		01 \quad
		10 \quad
		11 \quad
		000
	\]
	\`E tuttavia pi\`u comodo rappresentare gli oggetti tramite sequenze tutte lunghe uguali come di seguito
	\[
		000 \quad
		001 \quad
		010 \quad
		100 \quad
		011 \quad
		101 \quad
		110 \quad
	\]
\end{example}

\section{Rappresentazione generale}\label{rappresentazione_generale}
In generale, se dispongo di un alfabeto $\Gamma$ di cardinalit\`a $s$, posso costruire $N$ sequenze differenti \emph{tutte}
lunghe $\lceil \log_s N \rceil$ caratteri.

Il vantaggio di avere sequenze tutte di uguale lunghezza \`e che posso concatenarle senza dover aggiungere un simbolo di
separazione tra l'una e l'altra.

Per ottenere \textbf{rappresentazioni efficienti} si deve quindi usare un numero massimo di caratteri di
\textbf{ordine logaritmico} nella cardinalit\`a $N$ dell'insieme da rappresentare e $| \Gamma | \geq 2$.

\subsection{Rappresentazione di interi}
La notazione posizionale per rappresentare numeri interi \`e una rappresentazione efficiente, indipedentemente dalla base
$s \geq 2$ scelta.

Un numero $N$ \`e rappresentato con un numero $d$ di cifre tale che
\[ \lceil \log_2 N \rceil \leq d \leq \lceil \log_2 N \rceil + 1 \]
C'\`e quindi una \textbf{riduzione logaritmica} tra il valore $N$ di un numero e la lunghezza $d$ della sua
rappresentazione.