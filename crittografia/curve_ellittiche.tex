\chapter{Crittografia su curve ellittiche}
La \textbf{crittografia su curve ellittiche} nasce per alleggerire il carico computazionale che si porta dietro
la crittografia a chiave pubblica basata sull'algebra modulare.

I problemi su cui si basa la crittografia a chiave pubblica, come la fattorizzazione e il calcolo del logaritmo
discreto hanno due problemi:
\begin{itemize}
	\item Sebbene si risolvano in tempo polinomiale sono comunque \textbf{calcoli molto pesanti}.
	\item Gli algoritmi odierni per il calcolo di questi due problemi non hanno pi\`u costo esponenziale ma,
	      come abbiamo gi\`a visto, \textbf{subesponenziale}. Ne \`e una diretta conseguenza l'aumento della
	      lunghezza delle chiavi.
\end{itemize}
Come vedremo, la crittografia su curve ellittiche, propone i cifrari visti in precedenza (protocollo DH e cifrario
di ElGamal) in un contesto matematico diverso basati sul calcolo del logaritmo discreto e non pi\`u sulla
fattorizzazione.

Il risultato \`e stato quello di ottenere un costo del miglior attacco, come vedremo, \textbf{puramente esponenziale}.

\section{Curve ellittiche}
Le curve ellittiche sono curve algebriche descritte da equazioni cubiche (simili a quelle utilizzate nel calcolo
della lunghezza degli archi delle ellissi):
\[ E(a, b) = \{ (x, y) \in \mathbb{R}^2 \mid y^2 = x^3 + ax + b \} \]
qui sono definite sui reali, nella cosiddetta \emph{forma normale di Weierstrass}, ma possiamo scrivere l'equazione
su un campo $\mathbb{K}$ qualisiasi.

L'insieme appena descritto contiene anche il cosiddetto \textbf{punto all'infinito} $O$ in direzione dell'asse $y$
(la curva ha un asintoto verticale), il quale rappresenta l'\textbf{elemento neutro per l'addizione}.

La curva si pu\`o presentare in due forme
\begin{itemize}
	\item A \textbf{due componenti} nel caso in cui la cubica abbia tre radici reali.
	\item Ad \textbf{una componente} nel caso in cui la cubica abbia una sola radice reale e due complesse
	      coniugate.
\end{itemize}
Se la curva ha una di queste due forme, \`e sempre possibile mandare una \textbf{tangente}. Questo per\`o non \`e
vero per tutti i tipi di curva.

Per le applicazioni crittografiche si assume infatti che il discriminante della cubica sia
\[ 4 a^3 + 27 b^2 \neq 0 \]
Questo ci assicura che
\begin{itemize}
	\item La cubica non abbia radici multiple.
	\item La curva sia priva di punti singolari come \emph{cuspidi} o \emph{nodi}, dove non sarebbe definita in
	      modo univoco la tangente.
\end{itemize}
Le curve ellittiche rappresentano anche una \textbf{simmetria orizzontale}, ossia rispetto all'asse delle ascisse.
Questo ci permette di definire sempre, per ogni punto della curva, il suo \textbf{opposto} ancora sulla curva.

Preso il punto $p = \langle x, y \rangle$, il suo opposto ha coordinate $-p = \langle x, -y \rangle$.

\subsection{Somma di punti}
Prima di tutto notiamo che le curve ellittiche possiedono una propriet\`a: ogni retta interseca una curva ellittica
in al pi\`u tre punti.
\[
	\begin{cases}
		y = mx + q \\
		y^2 = x^3 + ax + b
	\end{cases}
\]
Sviluppando il sistema si ottiene una cubica in $x$ che ha al massimo tre radici. Le radici in questione possono
essere o tutte reali o una reale e due complesse coniugate.

Quindi se una retta interseca la curva nei punti $P$ e $Q$ (coincidenti se la retta \`e una tangente), allora la
retta interseca la curva anche in un terzo punto $R$.

Dati tre punti $P$, $Q$ ed $R$ su una curva, se sono anche tutti su una retta, vale
\[ P + Q + R = O \]
Da questa regola si ricava la regola per la somma di due punti $P$ e $Q$.
\begin{enumerate}
	\item Si scrive l'equazione della retta che passa per $P$ e $Q$.
	\item Dato che $P$ e $Q$ sono sulla curva e su una retta, esiste un terzo punto $R$ identificato dall'intersezione
	      della curva con la retta.
	\item L'opposto di $R$ \`e il punto somma di $P$ e $Q$.
\end{enumerate}
Per sommare un punto $P$ con se stesso
\begin{enumerate}
	\item Si manda la tangente alla curva nel punto $P$.
	\item Si trova il punto $R$ di intersezione.
	\item Il punto $-R$ equivale al punto $2P$.
\end{enumerate}
Dati due punti $P = (x_P, y_P)$ e $Q = (x_Q, y_Q)$, sulla curva $E(a, b)$, si distinguono tre casi:
\begin{itemize}
	\item Se $P \neq \pm Q$ allora $S = P + Q$ con
	      \[ \begin{matrix}
			      x_S = \lambda^2 - x_P - x_Q \\
			      y_S = -y_P + \lambda (x_P - x_S)
		      \end{matrix}
	      \]
	      dove
	      \[ \lambda = \frac{y_Q - y_P}{x_Q - x_P} \]
	      \`e il coefficiente angolare della retta per $P$ e $Q$.
	\item Se $P = Q$ allora $S = P + Q = 2P$ con
	      \[
		      \begin{matrix}
			      x_S = \lambda^2 - x_P - x_Q \\
			      y_S = -y_P + \lambda (x_P - x_S)
		      \end{matrix}
	      \]
	      dove
	      \[ \lambda = \frac{3_P^2 + a}{2yp} \]
	      \`e il coefficiente angolare della tangente alla curva in $P$ ($y_P \neq 0$).
	      Se $y_P = 0$ allora $S = 2P = O$.
	\item Se $Q = -P$ allora $S = P + Q = P + (-P) = O$.
\end{itemize}
I punti delle curve ellittiche con discriminante diverso da zero hanno la struttura algebrica di gruppo abeliano,
e gode dunque di alcune propriet\`a:
\begin{itemize}
	\item \textbf{Chiusura}: $\forall P, Q \in E(a, b)$ vale
	      \[ P + Q \in E(a, b) \]
	\item \textbf{Elemento neutro}: $\forall P \in E(a, b)$ vale
	      \[ P + O = O + P = P \]
	\item \textbf{Inverso}: $\forall P \in E(a, b)$ esiste un unico $Q = -P \in E(a, b)$ tale che
	      \[ P + Q = Q + P = O \]
	\item \textbf{Associativit\`a}: $\forall P, Q, R \in E(a, b)$ vale
	      \[ P + (Q + R) = (P + Q) + R \]
	\item \textbf{Commutativit\`a}: $\forall P, Q \in E(a, b)$ vale
	      \[ P + Q = Q + P \]
\end{itemize}

\section{Curve ellittiche su campi finiti}
Ora che abbiamo definito, in generale, cosa sono e come funzionano le curve ellittiche sui reali passiamo a trattare
le curve ellittiche su campi finiti, che sono quelle che poi servono di pi\`u all'atto pratico.

I campi finiti possono essere di due tipi fondamentalmente: quelli che hanno come cardinalit\`a un numero primo e
quelli che hanno come cardinalit\`a la potenza di un numero primo.

\subsection{Curve ellittiche prime}
Le \textbf{curve ellittiche prime} sono curve i cui coefficienti $a$ e $b$ sono presi in $\mathbb{Z}_p$ con $p > 3$
un numero primo e l'equazione che le descrive \`e
\[ E_p(a, b) = 	\{ (x, y) \in \mathbb{Z}_p^2 \mid y^2 \mod{p} = (x^3 + ax + b) \mod{p} \} \cup \{ O \} \]
Una curva ellittica prima contiene un numero \textbf{finito} di punti e non \`e pi\`u rappresentata da una curva
continua nel piano.

Una curva ellittica prima risulta simmetrica rispetto alla retta
\[ y = \frac{p}{2} \]
questo perch\'e si lavora sempre tra 0 e $p - 1$.

L'inverso di un punto $P = (x, y) \in E_p(a, b)$ \`e definito come
\[ -P = (x, -y) = (x, p - y) \in E_p(a, b) \]

Se il polinomio $x^3 + ax + b \mod{p}$ non ha radici multiple, ovvero se il discriminante della cubica \`e diverso
da zero, allora i punti della curva $E_p(a, b)$ formano un \textbf{gruppo abeliano additivo finito}.

\begin{example}
	Prendiamo la curva $E_5(4, 4)$. Affinch\'e i punti abbiamo la struttura di campo abeliano dobbiamo verificare
	che il discriminante sia diverso da zero.
	\[ 4 \cdot 4^3 + 27 \cdot 4^2 \mod{5} \neq 0 \]
	sviluppando un po' i calcoli si ottiene
	\[ 3 \mod{5} = 3 \neq 0 \]
	quindi i punti sulla curva hanno la struttura di campo abeliano. Ora vogliamo trovare quanti punti ha questa
	curva, per farlo dobbiamo considerare l'equazione
	\[ y^2 \equiv x^3 + 4x + 4 \mod{5} \]
	Dato che stiamo lavorando modulo 5, per trovare tutti i punti della curva, possiamo far variare l'ascissa con
	tutti i valori da 0 a 4 e valutare la cubica.

	In questa fase si deve prestare attenzione al fatto che un punto sull'ascissa non necessariamente identifica
	un punto sulla cubica. Questo perch\'e non tutti i numeri modulo $p$ hanno una radice (solo la met\`a).

	Per vederlo facciamo un semplice procedimento
	\begin{center}
		\begin{tabular}{ c | c c c c c }
			$y$           & 0 & 1 & 2 & 3 & 4 \\
			\hline
			$y^2 \mod{p}$ & 0 & 1 & 4 & 4 & 1
		\end{tabular}
	\end{center}
	da qui vediamo subito che solo 1 e 4 hanno una radice nel campo.

	Valutiamo ora tutti i punti
	\begin{center}
		\begin{tabular}{ c | c | c }
			$x$ & $y^2$ & $y$                    \\
			\hline
			0   & 4     & $\langle 2, 3 \rangle$ \\
			1   & 4     & $\langle 2, 3 \rangle$ \\
			2   & 0     & 0                      \\
			3   & 3     & /                      \\
			4   & 4     & $\langle 2, 3 \rangle$
		\end{tabular}
	\end{center}
	La curva contiene dunque 7 punti e il punto all'infinito, abbiamo dunque una curva di \textbf{ordine} 8.
\end{example}

\begin{theorem}[Teorema di Hasse]
	Se $n$ \`e l'ordine della curva prima $E_p(a, b)$ allora vale che
	\[ |n - (p + 1)| \leq 2 \sqrt{p} \]
\end{theorem}

\section{Funzioni one-way trap-door}
Il metodo di costruzione di funzioni one-way trap-door su curve ellittiche \`e molto simile a quello usato per
il protocollo DH e per il cifrario di ElGamal su algebra modulare.

Questo perch\'e l'addizione di punti su curve ellittiche si pu\`o mettere in corrispondenza con la moltiplicazione
di interi modulo $p$.

Si introduce quindi la \textbf{moltiplicazione scalare} di punti sulla curva che altro non \`e che una
generalizzazione dell'addizione nell'algebra modulare. Questa operazione coinvolge un punto $P \in E(a, b)$ e un
intero $k$ e si pu\`o mettere in corrispondenza con l'elevamento a potenza. Come vedremo $k P$ si calcola
in tempo polinomiale con $\Theta(\log k)$ addizioni e raddoppi di punti.

\subsection{Algoritmo dei raddoppi ripetuti}
Questo algoritmo, come vedremo, si pu\`o mettere in corrispondenza con l'algoritmo di esponenziazione veloce usato
nell'algebra modulare e consiste nel fare $\lfloor \log k \rfloor$ raddoppi e $O(\log k)$ somme di punti.

Sia $P$ un punto in $E_p(a, b)$ e sia $k$ un intero
\begin{enumerate}
	\item Si scompone come somma di potenze di 2
	      \[ k = \sum_{i=0}^{\lfloor \log k \rfloor} k_i 2^i \quad \quad k_i \in \{ 0, 1 \} \]
	\item Si calcolano $\lfloor \log k \rfloor$ punti, ognuno come raddoppio del precedente.
	      \[ 2P, \quad 2^2P, \quad 2^3 P, \quad \dots, \quad 2^{\lfloor\log k \rfloor} \]
	\item Si calcola $Q = kP$ calcolando
	      \[ \sum_{i \mid k_i = 1} 2^i P \]
	      in al pi\`u $\log k$ addizioni.
\end{enumerate}
Se notiamo, l'algoritmo calcola $kP$ con $\Theta(\log k)$ operazioni.

\begin{example}
	Supponiamo di voler calcolare $13P$. Non possiamo permetterci di fare 13 somme, scriviamo quindi il 13 come
	somma di potenze di 2
	\[ 13P = (8 + 4 + 1) P = 8P + 4P + P \]
	A questo punto, avendo $P$ possiamo calcolare tanti raddoppi quanti sono necessari
	\[
		\begin{matrix}
			P \rightarrow 2P          \\
			2P \rightarrow 2(2P) = 4P \\
			4P \rightarrow 2(4P) = 8P
		\end{matrix}
	\]
	Andiamo ora a prendere solo i termini che ci servono, ossia $P$, $4P$ e $8P$ e li sommiamo.

	Abbiamo cos\`i calcolato $13P$ con 3 raddoppi e 2 somme di punti.
\end{example}

\subsubsection{Inversione della funzione}
Invertire questa funzione \`e simile ad invertire il logaritmo discreto. Dati $Q$ e $P$, trovare, se esiste, il
pi\`u piccolo intero tale che
\[ Q = kP \]
\`e un problema \emph{difficile}. Se $k$ soddisfa l'equazione scritta sopra, allora $k$ si chiama il
\textbf{logaritmo discreto su curve ellittiche}
\[ k = \log_P Q \]
Il problema ha costo esponenziale dato che conosciamo solo algoritmi di tipo forza bruta che provano tutti i
possibili valori di $k$.

Quello che si fa all'atto pratico \`e una serie di somme di $P$ con se stesso finch\'e non troviamo $k$ tale che
\[ kP = Q \]

A differenza dell'algebra modulare, per cui si conoscono algoritmi di costo subesponenziale per il calcolo del
logaritmo discreto, per le curve ellittiche, conosciamo solo algoritmi di costo esponenziale.

\section{Protocollo DH su curve ellittiche}
Il protocollo DH, dato che basa la sua sicurezza sul calcolo del logaritmo discreto, pu\`o essere rivisitato in
una sua versione su curve ellittiche.

La conversione dall'algebra modulare si ottiene sostituendo la moltiplicazione di interi in modulo con la somma di
punti.

Prima di vedere come funziona il protocollo dobbiamo definire un'altra cosa, ossia l'\textbf{ordine di un punto}.

\begin{definition}
	L'\textbf{ordine di un punto} \`e il pi\`u piccolo intero $n$ tale che il punto per questo intero da come risultato
	il punto all'infinito.
	\[ nP = O \]
\end{definition}

Supponiamo che i due utenti $i$ e $j$ vogliano instaurare un protocollo di comunicazione.
\begin{enumerate}
	\item $i$ e $j$ scelgono una curva ellittica $E_p(a, b)$ e un punto $B \in E_p(a, b)$ speciale, detto
	      \textbf{punto base}, il quale deve avere ordine $n$ molto grande.
	\item $i$ e $j$ scelgono rispettivamente $n_i$ e $n_j$ casuali tali che
	      \[ n_i, n_j < n \]
	\item $i$ e $j$ calcolano rispettivamente
	      \[
		      \begin{matrix}
			      P_i = n_i B \\
			      P_j = n_j B
		      \end{matrix}
	      \]
	      con l'algoritmo dei raddoppi ripetuti.
	\item $i$ e $j$ inviano rispettivamente $P_i$ e $P_j$ sul canale.
	\item $i$ e $j$ calcolano rispettivamente
	      \[
		      \begin{matrix}
			      k & = & n_i P_j & = & n_i n_j B \\
			      k & = & n_j P_i & = & n_j n_i B
		      \end{matrix}
	      \]
\end{enumerate}

\subsection{Attacchi}
Per quanto riguarda gli attacchi passivi, per risalire a $n_i$ e $n_j$ si deve calcolare il logaritmo discreto
su curve ellittiche, che come sappiamo, richiede costo esponenziale sapendo solo $E_p(a, b)$, $P_i$, $P_j$ e $B$.

Il protocollo \`e comunque vulnerabile ad attacchi di tipo \emph{man in the middle} a meno che non si usino curve
estratte da certificati digitali.

\section{Protocollo di ElGamal su curve ellittiche}
A differenza dei protoccolli visti fino ad ora, su curve ellittiche e non, il \textbf{protocollo di ElGamal su curve
	ellittiche} \`e usato come un cifrario a chiave pubblica per lo scambio di messaggi e non \`e dunque ristretto
al solo scambio della chiave.

Cifrare un messaggio, in questo ambito, significa \emph{incapsulare} il messaggio in uno dei punti delle curve
ellittiche. Il problema \`e che non disponiamo di algoritmi deterministici, che, in tempo polinomiale, mappano
un messaggio su uno dei punti della curva.

Un'idea potrebbe essere quella di porre $x = m$ nella cubica della curva e prendere il relativo punto, calcolato
tramite l'equazione.

Il problema \`e che per trovare la coordinata $y$ del punto dobbiamo fare la radice del valore calcolato dalla
cubica ma, come abbiamo visto, non sempre questo \`e possibile.

\subsection{Algoritmo di Koblitz}
Sia $m$ il messaggio che vogliamo inviare e sia $p$ il numero primo con cui stiamo lavorando, l'\textbf{algoritmo
	di Koblitz} funziona in questo modo
\begin{enumerate}
	\item Si sceglie un intero $h$ tale che
	      \[ (m + 1) \cdot h < p \]
	\item Si pone
	      \[ x = m \cdot h + i \]
	      nella cubica, con $0 \leq i < h-1$.
	\item Si fanno $h$ tentativi.
\end{enumerate}
La probabilit\`a di fallire \`e di circa $1/2$ ogni volta, che si traduce in una probabilit\`a di circa $(1/2)^h$
di fallimento totale. Se $h$ \`e sufficientemente grande, la probabilit\`a di fallire \`e molto bassa.

\begin{lstlisting}[style=pseudo-style]
Koblitz(m, h, a, b, p)
	for i = 0 to h-1
		x = m * h + i;
		z = (pow(x, 3) + a * x + b) % p;
		if z.hasRadix() then 
			y = sqrt(z);
			return (x, y);

	return FAIL;
\end{lstlisting}
Il destinatario, in fase di decifrazione, ottiene il punto della curva di coordinate $\langle x, y \rangle$. Per
estrarre il messaggio calcola
\[
	m = \left\lfloor \frac{x}{h} \right\rfloor  =
	\left\lfloor \frac{mh + i}{h} \right\rfloor =
	\left\lfloor m + \frac{i}{h} \right\rfloor
	\quad \quad \text{con } \frac{i}{h} < 1
\]
Quello che si fa in pratica \`e dividere l'ascissa del punto per $h$.

\subsection{Scambio di messaggi}
Scelta una curva $E_p(a, b)$, un punto base $B \in E_p(a, b)$ di ordine $n$ elevato, ogni utente costruisce la sua
coppia $\langle k_\text{pub}, k_\text{priv} \rangle$.

L'utente $u$ forma la coppia di chiavi calcolandole in questo modo:
\[
	\begin{matrix}
		k_\text{pub} = n_u B = P_u \\
		k_\text{priv} = n_u < n
	\end{matrix}
\]
Se due utenti $i$ e $j$ volessero comunicare, il procedimento sarebbe il seguente:
\begin{enumerate}
	\item $i$ incapsula il messaggio $m$ in un punto $M$ della curva con l'algoritmo di Koblitz.
	\item $i$ sceglie $r < n$ casuale e da usare una volta sola.
	\item $i$ calcola un punto
	      \[ V = r \cdot B \]
	      della curva con l'algoritmo dei raddoppi ripetuti.
	\item $i$ cifra il messaggio con la chiave pubblica $P_j$ del destinatario calcolando
	      \[ W = M + r \cdot P_j \]
	      dove $W$ \`e un punto della curva.
	\item $i$ invia la coppia di punti $\langle V, W \rangle$ a $j$.
	\item $j$ riceve $\langle V, W \rangle$.
	\item $j$ calcola il punto
	      \[ M = W - n_j \cdot V \]
	      sulla curva dove risiede il messaggio.
	\item $j$ calcola
	      \[ m = \left\lfloor \frac{x}{h} \right\rfloor \]
\end{enumerate}

\subsubsection{Correttezza}
Per convincerci di ci\`o che stiamo facendo, facciamo qualche passo indietro.
\begin{itemize}
	\item Il punto $M$ si trova calcolando
	      \[ M = W - n_j \cdot V \]
	\item Il punto $W$ si ottiene calcolando
	      \[ W = M + r \cdot P_j \]
	      dove $P_j$ \`e la chiave pubblica di $j$.
	\item La chiave pubblica $P_j$ si ottiene calcolando
	      \[ n_j \cdot B \]
	\item Il punto $V$ si ottiene calcolando
	      \[ V = r \cdot B \]
\end{itemize}
Mettendo tutto insieme si ricava che
\[ W - n_j \cdot V \quad = \quad M + r (n_j \cdot B) - n_j (r \cdot B) \quad = \quad M \]

\subsection{Attacchi}
Essendo un sistema a chiave pubblica, \`e vulnerabile ad attacchi di tipo \emph{man in the middle} a meno che non
si estraggano le chiavi pubbliche da certificati digitali.

Gli attacchi passivi possono essere di due tipi
\begin{itemize}
	\item \textbf{Calcolo della chiave privata}: richiede il calcolo del logaritmo discreto su curve ellittiche
	      (costo esponenziale).
	\item \textbf{Attacco su r}: nel caso il parametro $r$ venga riusato il crittoanalista pu\`o scoprirlo e
	      decifrare facilmente il crittogramma.
\end{itemize}

\subsubsection{Attacco su r}
Conoscere $r$ ci permette di decifrare il crittogramma in tempo polinomiale perch\'e
\[ W = M + r \cdot P_j \]
quindi, se conosciamo $r$, possiamo calcolare
\[ M = W - r \cdot P_j \]
Estrarre $r$, calcolandolo con i dati pubblici forniti da mittente e destinatario \`e tuttavia un problema difficile
quanto il calcolo del logaritmo discreto.

Si dovrebbero sfruttare altre vulnerabilit\`a nel sistema come ad esempio capire quale sia il generatore di numeri
casuali utilizzato oppure sfruttare il fatto che $r$ possa essere utilizzato pi\`u volte.

\section{Sicurezza delle curve ellittiche}
La sicurezza della crittografia su curve ellittiche si basa sulla difficolt\`a nel calcolare il logaritmo discreto
di un punto della curva.

Nonostante non esista una dimostrazione formale di \emph{intrattabilit\`a}, questo problema \`e comunque considerato
molto \emph{difficile}.

In particolare, \`e un problema molto pi\`u difficile della fattorizzazione e del calcolo del logaritmo discreto
nell'algebra modulare, problemi per i quali esiste un algoritmo (Index Calculus) di costo
\[ O(2^{\sqrt{b \log b}}) \]
quindi subesponenziale, dove $b$ sono i bit del modulo su cui stiamo lavorando.

Questo algoritmo sfrutta il fatto che gli interi in modulo hanno la struttura algebrica di \textbf{campo} e su cui
\`e dunque definita l'operazione di moltiplicazione.

I punti di una curva ellittica hanno invece la struttura algebrica di \textbf{gruppo abeliano}, che ha propriet\`a
ben pi\`u deboli di un campo e dove non \`e definita la moltiplicazione. Questo non ha dunque permesso di
rimodellare l'Index Calculus su curve ellittiche.

Esiste un metodo pi\`u efficiente del forza bruta, l'algoritmo di Pollard, di costo
\[ O(2^{b/2}) \]
e quindi puramente \textbf{esponenziale}.

La crittografia su curve ellittiche ha quindi il grosso vantaggio di avere una sicurezza di $b/2$ bit dove $b$ \`e
la lunghezza della chiave.

Se per esempio sull'RSA, per avere 256 bit di sicurezza, necessitavamo di una chiave di pi\`u di 15.000 bit, per le
curve ellittiche ne basta una lunga 512.