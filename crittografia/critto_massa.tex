\chapter{Cifrari simmetrici per crittografia di massa}\label{critto_sim_massa}
Questi cifrari rientrano nel secondo gruppo di cifrari moderni, non hanno quindi un tipo di sicurezza incondizionata ma
di tipo computazionale.

Nascondono quindi l'informazione a patto che il crittoanalista abbia risorese computazionali limitate e che P $\neq$ NP.

La loro sicurezza si basa sui due \textbf{principi di Shannon}, i quali, rendono questi cifrari robusti alla
crittoanalisi statistica. I due principi in questione sono
\begin{itemize}
	\item \textbf{Diffusione}: Il testo in chiaro si deve distribuire su tutto il crittogramma. Ogni carattere del
	      crittogramma deve cio\`e dipendere da tutti i caratteri del blocco del messaggio.

	      Si ottiene cos\`i un istogramma delle frequenze piatto.
	\item \textbf{Confusione}: Messaggio e crittogramma sono combinati fra loro in modo molto complesso per non
	      permettere al crittoanalista di separare le due sequenze tramite l'analisi statistica del crittogramma.

	      Per far s\`i che questo avvenga devono essere vere due condizioni
	      \begin{itemize}
		      \item La chiave deve essere ben distribuita sul testo cifrato.
		      \item Ogni bit del crittogramma deve dipendere da tutti i bit della chiave.
	      \end{itemize}
\end{itemize}

\section{DES}\label{DES}
\`E stato il primo cifrario \textbf{certificato} proposto da IBM e che proponeva una struttura di questo tipo:
\begin{itemize}
	\item Il messaggio \`e diviso in blocchi, ciascuno cifrato e decifrato indipendetemente dall'altro.
	\item Ogni blocco \`e di 64 bit.
	\item Cifratura e decifrazione procedono in $r$ fasi o \textbf{round} in cui si ripetono le stesse operazioni. Per
	      questo cifrario i round sono 16.
	\item La chiave \`e composta da 8 byte. I primi sette bit di ciascun byte sono scelti arbitrariamente e l'ottavo
	      \`e aggiunto per il controllo di parit\`a.
	      \begin{itemize}
		      \item La chiave contiene dunque 64 bit: 56 arbitrari e 8 di parit\`a.
		      \item Dalla chiave vengono create $r$ \textbf{sottochiavi di fase}.
	      \end{itemize}
\end{itemize}

\subsection{Funzionamento}\label{funzionamento_DES}
Sia $m$ il messaggio da inviare, $c$ il rispettivo crittogramma e $k$ la chiave. Il processo di cifratura \`e il seguente
\begin{enumerate}
	\item Il bit del messaggio vengono permutati (blocco PI).
	\item La chiave viene private dei bit di controllo parit\`a e permuta i rimanenti (blocco T).
	\item Si dividono i bit del messaggio in due parti (S e D), ciascuna di 32 bit.
	\item Si entra in un ciclo di 16 fasi e per ogni fase $i$ abbiamo in input, l'output della fase precedente.

	      Alla chiave $k$ si applicano queste operazioni:
	      \begin{itemize}
		      \item I 56 bit della chiave vengono divisi in due parti da 28 bit ciascuna e si applica, a ciascuna delle
		            due parti, uno shift ciclico di 1 o 2 bit a seconda della fase in cui ci si trova.

		            Procedimento necessario affinch\'e vengano usati tutti i bit della chiave.
		      \item Si estraggono 48 bit dai due blocchi di 28 bit del punto precedente che andranno a formare la
		            sottochiave di fase.
		      \item Riconcateno le due sequenze shiftate che andranno poi a comporre la chiave per la fase successiva.
	      \end{itemize}
	      I due blocchi del messaggio vengono trattati in questo modo:
	      \begin{itemize}
		      \item Si mandano i 32 bit di destra (input) nella parte di sinistra (output)
		            \[ S[i] = D[i-1] \]
		      \item Sempre della parte di destra vengono copiati 16 bit andando a produrre un blocco da 48 bit.
		      \item Si fa lo XOR bit a bit tra questo blocco appena prodotto e la sottochiave di fase.
		      \item Il blocco di 48 bit vengono riportati a 32 bit grazie alla \textbf{S-box} (approfondimento pi\`u
		            avanti).
		      \item Si permutano i bit prodotti al passo precedente.
		      \item Si fa lo XOR bit a bit tra il blocco di bit appena prodotto e la parte sinistra in input andando a
		            comporre il nuovo blocco di destra.
	      \end{itemize}
	\item Parte destra e parte sinistra vengono unite di nuovo.
	\item Si permutano i bit del blocco ottenuto (blocco PF).
\end{enumerate}
La S-box \`e una funzione composta da 8 sotto-funzioni, ciascuna che prende in input 6 bit e ne restituisce 4.

Per farlo si prendono il primo e l'ultimo bit in input e se ne ricava un indice di riga mentre con i rimanenti bit
si ricava un indice di colonna.

Tramite questi due indici si ottiene un valore presente in una tabella, le cui righe contengono ognuna una permutazione
dei primi 16 interi, il quale sar\`a resituito in output di 4 bit.

\subsection{Sicurezza}\label{sicurezza_DES}
Un cifrario ha una sicurezza di $b$ bit se il costo del miglior attacco \`e di ordine $O(2^b)$ operazioni e richiede di
esplorare uno spazio delle chiavi di cardinalit\`a $2^b$.

Nel caso del DES abbiamo chiavi da 56 bit ma lo spazio delle chiavi ha cardinalit\`a $2^{55}$ dato che se cifriamo
il complemento del messaggio col complemento della chiave otteniamo il complemento del crittogramma. I bit di sicurezza
non sono quindi 56 ma 55.

Questo ha anche un'altra applicazione: se provando una chiave, questa non va bene allora posso escludere anche il suo
complemento.

\subsection{Attacchi}\label{attacchi_DES}
Uno degli attacchi di cui il DES \`e stato vittima \`e quello di tipo \textbf{distribuito}, ossia, attacchi di tipo
forza bruta distribuiti su pi\`u macchine i quali sono riusciti a forzare il cifrario in tempi sempre minori.

Un tipo di attacco che si potrebbe decidere di sferrare \`e di tipo \textbf{chosen plain text}.
\begin{enumerate}
	\item Si prende un messaggio $m$ e lo si cifra in $c_1$.
	\item Si cifra in $\overline{m}$ in $c_2$, ossia il complemento di $m$.
	\item Per ogni chiave $k$ si prova a cifrare $m$ con $k$.
	      \begin{itemize}
		      \item Se si ottiene $c_1$ molto probabilmente $k$ \`e la chiave (non sicuramente dato che potrebbero
		            esserci altre chiavi che mappano $m$ in $c_1$).
		      \item Se la cifratura ha invece prodotto $\overline{c_2}$ allora \`e probabile che $\overline{k}$ sia
		            la chiave.

		            Questo perch\'e provando a cifrare il complemento del messaggio col complemento della chiave si
		            ottiene il complemento del crittogramma. Nel nostro caso
		            \[ C(\overline{m}, \overline{k}) = \overline{\overline{c_2}} = c_2 \]
		      \item Altrimenti ne $k$ ne $\overline{k}$ sono le chiavi ma con una sola cifratura vengono scartate
		            due chiavi.
	      \end{itemize}
\end{enumerate}

Un altro attacco di tipo \emph{chosen plain text} si basa sulla \textbf{crittoanalisi differenziale}, la quale necessita
di almeno $2^{47}$ coppie messaggio-crittogramma per funzionare e sfrutta l'analisi probabilistica per stimare quale
chiave \`e stata usata andando a cercare variazioni nei vari crittogrammi.

Il costo di questo attacco \`e tuttavia dell'ordine di $O(2^{55})$ operazioni per via delle 16 fasi del cifrario che
rende l'attacco leggermente pi\`u dispendioso del forza bruta.

L'ultima tecnica di attacco che vediamo \`e basata sulla \textbf{crittoanalisi lineare} ed \`e un attacco di tipo
\emph{know plain text}.

Per effettuare l'attacco si necessita di $2^{43}$ coppie messaggio-crittogramma e serve a stimare alcuni bit della chiave.

Questa tecnica \`e meno costosa del forza bruta e dunque \`e vulnerabile.

\subsection{Alternative}\label{alternative_DES}
Si \`e provato a cambiare sempre le chiavi di fase, arrivando ad avere 768 bit di chiave complessivi. In realt\`a,
per attacchi basati su crittoanalisi differenziale, i bit di sicurezza sono 61.

Vengono aggiunti solo 6 bit di sicurezza al fronte di una chiave molto pi\`u lunga.

L'approccio che invece \`e stato adottato \`e stata la \textbf{cifratura multipla}, ossia, la composizione del DES con
se stesso. Scelte due chiavi $k_1$ e $k_2$ qualsiasi, vale che
\[ C(C(m, k_1), k_2) \neq C(m, k_3) \]
per qualunque chiave $k_3$ nello spazio delle chiavi. In questo modo otteniamo chiavi di 112 bit ma con 57 bit di
sicurezza.

Ci\`o che riduce molto i bit di sicurezza sono gli attacchi di tipo \textbf{meet in the middle}. Data una coppia
messaggio-crittogramma l'attacco funziona in questo modo:
\begin{enumerate}
	\item Per ogni $k_1$ si calcola e si salva
	      \[ C(m, k_1) \]
	\item Per ogni $k_2$ si calcola
	      \[ D(c, k_2) \]
	      e si cerca nella tabella.
	\item Se troviamo una corrispondenza $k_1$ e $k_2$ probabilmente sono le chiavi.
\end{enumerate}
L'attacco si basa sul fatto che se il crittogramma $c$ \`e generato da
\[ C(C(m, k_1), k_2) \]
allora vale
\[ D(c, k_2) = C(m, k_1) \]
Quello che di fatto andiamo a fare \`e enumerare tutte le chiavi due volte (non tutte le coppie di chiavi) e poi cerchiamo
una corrispondenza.

Se le chiavi sono $2^{56}$ basta moltiplicare per 2 e otteniamo cos\`i $2^{57}$ operazioni per forzare il cifrario.

\section{AES}\label{AES}
Si tratta di un cifrario simmetrico che fa uso di chiavi brevi (128 o 256 bit) e ripetute che deveono essere cambiate
per ogni nuova sessione di comunicazione.

Dato che la chiave \`e corta il messaggio \`e diviso in blocchi lunghi quanto la chiave.
