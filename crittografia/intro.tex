\chapter{Introduzione}\label{intro}
La \textbf{crittografia} si occupa dei metodi di cifratura mentre la \textbf{crittoanalisi} dei metodi di decifrazione
di un messaggio. Entrambe si possono riunire sotto la branca della \textbf{crittologia}.

Un tipico scenario \`e quello in cui due utenti sono impegnati nello scambio di messaggi su di un canale di trasmissione
\emph{insicuro}, ossia un canale da cui \`e possibile intercettare i messaggi che vi transitano.

Per proteggere la comunicazione, i due utenti, devono adottare un \textbf{metodo di cifratura} che impedisca ad un
crittoanalista di leggere i messaggi che si scambiano.

Ci\`o che viaggia sul canale di trasmissione non \`e quindi il messaggio in chiaro, bens\`i un \textbf{crittogramma},
il quale deve possedere due propriet\`a fondamentali:
\begin{itemize}
	\item Deve essere \textbf{incomprensibile} al crittoanalista.
	\item Deve essere \textbf{facilmente decifrabile} dal destinatario.
\end{itemize}

\section{Cifratura e decifrazione}\label{cifratura}
In crittografia si usa una \textbf{funzione di cifratura}
\[ C : \text{MSG} \rightarrow \text{CRITTO} \]
dove MSG \`e l'insieme dei messaggi $m$ in chiaro e CRITTO \`e l'insieme dei crittogrammi $c$ ottenuti calcolando
\[ c = C(m) \]
Per decifrare i crittogrammi su usa una \textbf{funzione di decifrazione}
\[ D : \text{CRITTO} \rightarrow \text{MSG} \]
e \textbf{inversa} di $C$, la quale ricava il messaggio $m$ dal crittogramma $c$ calcolando
\[ m = D(c) \]
La funzione $C$ deve essere \textbf{iniettiva} poich\'e deve esserci una corrispondenza \emph{biunivoca} tra un
messaggio e il relativo crittogramma. Se ad un crittogramma corrispondessero pi\`u messaggi in chiaro, il destinatario
non sarebbe pi\`u in grado di capire quale sia il messaggio inviato dal mittente.

Come gi\`a detto, $C$ e $D$ devono essere l'una l'inversa dell'altra, vale quindi la seguente relazione:
\[ D(c) = D(C(m)) = m \]

\section{Chiavi}\label{chiavi}
In alcuni cifrari, la segretezza dipende dalla conoscenza del metodo di cifratura. Una volta scoperto, il cifrario
diventa inutilizzabile poich\'e chiunque pu\`o decifrare.

Se un cifrario \`e utilizzato da molti utenti, la \textbf{parte segreta} del metodo non pu\`o essere la funzione di
cifratura e/o quella di decifrazione perch\'e sarebbe troppo difficile mantenerla segreta.

Si deve quindi introdurre un nuovo elemento, ossia la \textbf{chiave}, nota solo agli utenti che stanno comunicando e
tramite la quale cifrano e decifrano il messaggio.

Per garantire la sicurezza, il numero delle chiavi deve essere di ordine abbastanza grande da rendere impossibile o
molto dispendioso il metodo di provarle tutte (\textbf{forza bruta}) e inoltre la chiave deve essere scelta in modo
\textbf{casuale}.

Le funzioni di cifratura e decifrazione diventano quindi di questo tipo:
\[ c = C(m, k) \quad m = D(c, k) \]
dove $k$ \`e la chiave.

Ricapitolando, il crittoanalista, anche conoscendo $C$, $D$ e $c$, non deve essere in grado di risalire a $m$ se non
conosce $k$.

\section{Cifrari}\label{cifrari}
Vediamo ora quali sono le principali categorie di cifrario. Approfondiremo in seguito ognuno di questi aspetti,
per il momento solo accennati.

\subsection{Cifrari per uso ristretto e per uso generale}
\begin{itemize}
	\item I \textbf{cifrari per uso ristretto} hanno la caratteristica di tenere la funzione di cifratura e
	      decifrazione segrete in ogni loro aspetto.
	\item I \textbf{cifrari per uso generale}, al contrario, rendono pubbliche le funzioni di cifratura e decifrazione
	      ma fanno uso di una chiave segreta.
\end{itemize}

\subsection{Cifrari perfetti}
Un \textbf{cifrario perfetto} \`e tale se
\begin{itemize}
	\item La chiave \`e lunga quanto il messaggio e diversa per ogni messaggio.
	\item Il messaggio in chiaro e il crittogramma risultano del tutto scorrelati tra loro.
	\item Nessuna informazione sul testo in chiaro pu\`o filtrare dal crittogramma.
	\item La conoscenza del crittoanalista non cambia dopo aver osservato un crittogramma.
\end{itemize}
I cifrari perfetti sono tuttavia molto costosi al livello computazionale e quindi non vengono usati per la crittografia
di massa.

\subsection{Cifrari sicuri}
Ad oggi i cifrari usati non sono perfetti ma sono dichiarati \textbf{sicuri} perch\'e rimasti inviolati e perch\'e la
loro violazione si basa su problemi matematici molto complessi. Se si riuscissero a trovare delle soluzioni efficienti
per tali problemi (tempo polinomiale) la sicurezza di tali cifrari crollerebbe.

\subsection{Cifrari simmetrici}
In questi cifrari la \textbf{chiave di cifratura} \`e uguale alla \textbf{chiave di decifrazione} ed \`e nota solo ai
due utenti.

Il messaggio \`e diviso in \textbf{blocchi} lunghi come la chiave, la quale \`e utilizzata per trasformare un blocco
del messaggio in un blocco del crittogramma.

\subsection{Cifrari a chiave pubblica}
I \textbf{cifrari a chiave pubblica} o \textbf{asimmetrici} nascono dall'esigenza di risolvere il problema dello scambio
delle chiavi tra i due utenti su un canale insicuro.

In questo caso, mittente e destinatario hanno due chiavi diverse. La chiave di cifratura e le funzioni di cifratura e
decifrazione sono \textbf{pubbliche} mentre invece la chiave di decifrazione \`e \textbf{privata} e solo il destinatario
del messaggio ne \`e in possesso.

L'obbiettivo \`e quello di permettere a tutti di inviare messaggi cifrati ad un certo destinatario ma solo lui deve
essere in grado di decifrarli.

In questo tipo di cifrari, la funzione di cifratura dev'essere di tipo \textbf{one-way trap-door}, ovvero, facile da
calcolare (tempo polinomiale) ma difficile da invertire (tempo esponenziale) a meno che non si conosca la chiave.

\subsubsection{Vantaggi}
Si elimina cos\`i la necessit\`a di effettuare uno scambio di chiavi fisico e abbiamo inoltre una riduzione
sostanziale sul numero di chiavi da generare: se gli utenti di un sistema sono $n$, il numero complessivo di chiavi
pubbliche e private \`e $2n$ anzich\'e $n (n-1) / 2$ come nel caso di un cifrario simmetrico.

\subsubsection{Svantaggi}
D'altra parte abbiamo dei cifrari molto lenti, dunque difficilmente utilizzabili in crittografia di massa e sono inoltre
soggetti ad attacchi di tipo \emph{chosen plain text}.

Ecco perch\'e questi cifrari sono utilizzati solo per scambiarsi una chiave simmetrica di un cifrario simmetrico, il
quale \`e molto pi\`u performante.

\section{Crittoanalisi}\label{crittoanalisi}
Il crittoanalista pu\`o avere due tipi di comportamento nella fase di attacco ad un sistema crittografico:
\begin{itemize}
	\item \textbf{Passivo}: si limita ad ascoltare la comunicazione intercettando i crittogrammi inviati.
	\item \textbf{Attivo}: modifica il contenuto dei messaggi spacciandosi per uno dei due utenti.
\end{itemize}
L'obbiettivo di un attacco \`e ovviamente quello di forzare il sistema. Il metodo utilizzato e il livello di
pericolosit\`a dipendono dalle informazioni in possesso del crittoanalista.

\subsection{Attacchi passivi}
Negli attacchi passivi il crittoanalista si limita dunque ad intercettare i messaggi per provare a estrapolarne
informazioni. Tra gli attacchi passivi pi\`u comuni abbiamo:
\begin{itemize}
	\item \textbf{Cipher Text}: Il crittoanalista rileva solo crittogrammi sul canale e cerca di estrapolarne informazioni.
	\item \textbf{Known Plain Text}: Il crittoanalista conosce delle coppie ($m$, $c$).
	\item \textbf{Chosen Plain Text}: Il crittoanalista si procura delle coppie ($m$, $c$) relative
	      ai messaggi in chiaro da lui scelti.
	\item \textbf{Chosen Cipher Text}: Il crittoanalista si procura delle coppie ($m$, $c$) relative a crittogrammi
	      da lui scelti.
	\item \textbf{Forza bruta}: Il crittoanalista prova tutte le chiavi.
\end{itemize}

\subsubsection{Attacchi chosen plain-text}
Dato che la funzione di cifratura e anche la sua chiave sono pubbliche, il crittoanalista pu\`o cifrare un numero a
piacere di messaggi, confrontarli con i crittogrammi che sta provando a decifrare.

Nel caso in cui trovi una corrispondenza tra uno dei suoi crittogrammi e uno dei crittogrammi da decifrare significa
che ha trovato il messaggio in questione.

\subsection{Attacchi attivi}
In un attacco di tipo attivo, il crittoanalista si intromette tra le comunicazioni che arrivano tra i due utenti e
cerca di disturbare la comunicazione in vari modi.

\subsubsection{Man in the middle}
Uno degli attacchi di tipo attivo pi\`u comune \`e il \textbf{man in the middle}. In questo tipo di attacco il
crittoanalista interrompe le comunicazioni dirette tra i due utenti e sostituisce i loro messaggi con i propri
convincendo ciascun utente che tali messaggi arrivino dall'altro.