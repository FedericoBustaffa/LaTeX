\chapter{Cifrari storici}
In questo capitolo andremo a trattare i \textbf{cifrari storici}, chiamati con questo nome perch\'e ad oggi sono stati
tutti forzati e dunque non sono pi\`u cifrari sicuri.

I primi cifrari nascono in un periodo in cui cifratura e decifrazione si facevano "con carta e penna" o quasi e servivano
per cifrare frasi di senso compiuto in linguaggio naturale.

Da un certo punto in poi tutti i cifrari hanno cercato di seguire i cosiddetti \textbf{principi di Bacone}
\begin{itemize}
	\item Le funzioni $C$ e $D$ devonon essere \textbf{facili da calcolare}.
	\item \`E \textbf{impossibile} ricavare la $D$ se la $C$ non \`e nota.
	\item Il crittogramma $c = C(m)$ deve apparire "\textbf{innocente}", deve sembrare cio\`e un testo in chiaro e non
	      una sequenza di caratteri insensata.
\end{itemize}

\section{Cifrario di Cesare}
L'idea di base \`e che il crittogramma $c$ \`e ottenuto dal messaggio in chiaro $m$ sostituendo ogni lettera di $m$ con
quella tre posizioni pi\`u avanti nell'alfabeto.

\begin{center}
	A B C D \dots W X Y Z
	\[ \downarrow \]
	D E F G \dots Z A B C
\end{center}
La decifrazione \`e immediata: basta sostituire ogni lettera del crittogramma con la lettera tre posizioni pi\`u indietro
nell'alfabeto.

\`E un cifrario molto semplice e non utilizza una chiave di cifratura. Una volta scoperto il metodo di cifratura e
decifrazione diventa del tutto inutile.

\subsection{Cifrario di Cesare generalizzato}
Per rendere il cifrario un po' pi\`u robusto basterebbe inserire un chiave $1 \leq k \leq 25$ (26 lascia inalterato il
messaggio) e invece di traslare sempre di tre posizioni le lettere del messaggio in chiaro, le trasliamo di $k$
posizioni.

Ovviamente i due utenti devono possedere la solita chiave per cifrare e decifrare i messaggi.

\subsubsection{Cifratura e decifrazione}
Sia $x$ una lettera dell'alfabeto, $pos(x)$ la sua posizione nell'alfabeto e $k$ la chiave tale che $1 \leq k \leq 25$.

La funzione di cifratura ritorna la lettera $y$ tale che
\[ pos(y) = (pos(x) + k) \mod{26} \]

La funzione di decifrazione ritorna la lettera $x$ tale che
\[ pos(x) = (pos(y) - k) \mod{26} \]

\`E immediato effettuare un attacco a forza bruta per forzarlo: si provano tutte le 25 chiavi.

\subsubsection{Osservazioni}
Il cifrario gode della propriet\`a \textbf{commutativa}: data una sequenza di chiavi e di operazioni di cifratura e
decifrazione, l'ordine delle operazioni pu\`o essere permutato arbitrariamente senza modificare il crittogramma finale.

Date due chiavi $k_1$ e $k_2$, e una sequenza $s$ vale
\[
	\begin{matrix}
		C(C(s, k_2), k_1) & = & C(s, k_1 + k_2) \\
		D(D(s, k_2), k_1) & = & D(s, k_1 + k_2)
	\end{matrix}
\]
Quindi una sequenza di operazioni di cifratura e decifrazione pu\`o essere ridotta ad una sola operazione di cifratura
o decifrazione.

In generale, comporre pi\`u cifrari \textbf{non} aumenta la sicurezza del sistema.

\section{Cifrari a sostituzione}
I \textbf{cifrari a sostituzione} sostituiscono ogni lettera del messaggio in chiaro con una o pi\`u lettere
dell'alfabeto secondo una regola prefissata.

Questi cifrari si suddividono a loro volta in due sottocategorie
\begin{itemize}
	\item Cifrari a \textbf{sostituzione monoalfabetica}: alla stessa lettera del messaggio corrisponde sempre una
	      stessa lettera nel crittogramma.
	\item Cifrari a \textbf{sostituzione polialfabetica}: alla stessa lettera del messaggio corrisponde una lettera
	      scelta in un insieme di lettere possibili, secondo una regola opportuna.
\end{itemize}

\subsection{Cifrari a sostituzione monoalfabetica}
Per questo tipo di cifrari si possono usare funzioni di cifratura e decifrazione pi\`u complesse dell'addizione e
sottrazione in modulo.

Andiamo cos\`i ad ampliare molto lo spazio delle chiavi non ottenendo per\`o grandi  miglioramenti per quanto
riguarda la sicurezza.

\subsubsection{Cifrario affine}
Questo cifrario usa una chiave composta da due valori
\[ k = \langle a, b \rangle \]
e cifra una lettera $x$ del messaggio in chiaro con la lettera $y$ che occupa la posizione
\[ pos(y) = (a * pos(x) + b) \mod{26} \]
nell'alfabeto.

La funzione di decifrazione invece mappa la lettera cifrata $y$ nella lettera $x$ del messaggio in chiara che ha posizione
\[ pos(x) = a^{-1} * (pos(y) - b) \mod{26} \]
nell'alfabeto. Si deve quindi trovare l'\textbf{inverso} moltiplicativo di $a$, ossia, quel numero che moltiplicato per $a$
mi da 1 (ovviamente in modulo).

Se vi ricordate qualcosa del corso di matematica discreta noterete che sorgono due problemi.
\begin{itemize}
	\item Il primo \`e che se $a$ \`e nullo allora la lettera $x$, in fase di cifratura viene sempre mappata nella lettera
	      di posizione $b$.
	\item Il secondo problema \`e dovuto al fatto che, in fase di decifrazione, dobbiamo usare l'inverso modulare di $a$ ma,
	      l'esistenza di quest'ultimo, non \`e sempre e comunque garantita.

	      Affinch\'e $a$ abbia un inverso modulare esso deve essere coprimo col numero di simboli del mio alfabeto (26 nel
	      nostro caso).
\end{itemize}
Il parametro $a$ deve quindi soddisfare queste due propriet\`a affinch\'e il cifrario sia utilizzabile
\[ a \neq 0 \quad \wedge \quad (a, 26) = 1 \]
Se $a$ non \`e coprimo col numero di simboli dell'alfabeto la funzione di cifratura non \`e pi\`u iniettiva e la decifrazione
diventa impossibile.

Se consideriamo un alfabeto di 26 caratteri, dobbiamo contare gli interi minori di 26 coprimi con esso.
\begin{itemize}
	\item I fattori primi di 26 sono 2 e 13.
	\item 26 \`e pari, dunque $a$ non pu\`o essere pari.
\end{itemize}
Ne deduciamo che il valore di $a$ deve essere un numero dispari compreso tra 1 e 25 ad eccezione di 13. Abbiamo quindi 12
possibili valori.

Se siete tra quelli bravi a matematica discreta non vi sar\`a sfuggito che il numero di possibili valori di $a$, ossia il
numero di valori coprimi con un certo numero $n$, \`e equivalente al valore della \textbf{funzione di Eulero} calcolata
su $n$. Nel nostro caso abbiamo
\[ \phi(26) = 12 \]
Per coloro che invece hanno bisogno di un ripasso elenchiamo qualche propriet\`a comoda per il calcolo di $\phi$
\begin{itemize}
	\item Se $n$ \`e primo allora
	      \[ \phi(n) = n - 1 \]
	\item Se $n = p \cdot q$ con $p$ e $q$ primi allora
	      \[ \phi(n) = (p - 1)(q - 1) \]
\end{itemize}
Il parametro $b$ invece pu\`o assumere un qualsiasi valore tra 1 e 25. Abbiamo in totale un numero di chiavi pari a
\[ 12 * 25 = 311 \]
Abbiamo aumentato molto il numero delle chiavi ma il cifrario rimane comunque molto debole.

\subsubsection{Cifrario completo}
Con questo cifrario andiamo a rendere lo spazio delle chiavi molto pi\`u ampio rispetto ad un cifrario affine andando a
prendere una permutazione arbitraria dell'alfabeto come chiave.

Ora abbiamo quindi che la chiave non \`e pi\`u una coppia di valori ma una permutazione lunga quanto l'intero alfabeto che
sto utilizzando.

Con questo tipo di cifratura la lettera in posizione $i$ viene cifrata nella lettera di posizione $i$ nella permutazione.

Con un alfabeto di $n$ caratteri ottengo uno spazio delle chiavi di dimensione
\[ n! - 1 \]
La permutazione in cui l'alfabeto \`e nell'ordine "canonico" non viene contata.

Il cifrario \`e tuttavia molto semplice da forzare sfruttando la struttura logica del messaggio in chiaro e l'occorrenza
statistica delle lettere.

\subsection{Cifrari a sostituzione polialfabetica}
Questi cifrari, come gi\`a anticipato, cifrano una lettera $x$ in una lettera $y$ scegliendo quest'ultima in un insieme di
possibili lettere secondo una regola opportuna.

In generale, ad una lettera $x$ non corrisponde sempre la stessa lettera $y$ che magari abbiamo ottenuto in una cifratura
precedente. Ogni lettera pu\`o essere quindi cifrata in pi\`u lettere e questo rende il cifrario molto pi\`u robusto di un
cifrario a sostituzione monoalfabetica.

\subsubsection{Cifrario di Augusto}
Questo cifrario, ideato dall'imperatore Augusto, veniva usato da quest'ultimo per cifrare informazioni utili che egli voleva
custodire e mantenere segrete.

Il funzionamento si basava sul primo libro dell'Illiade
\begin{enumerate}
	\item Si prende la lettera in posizione $i$ nel documento che vogliamo cifrare $a$.
	\item Si prende la lettera in posizione $i$ nel primo libro dell'Illiade $b$.
	\item Si calcola la distanza fra $a$ e $b$ nell'alfabeto che stiamo considerando.
	\item Si scrive in posizione $i$ la lettera dell'alfabeto nella posizione identificata dalla distanza trovata al punto
	      precedente.
\end{enumerate}

\subsubsection{Disco cifrante di Leon Battista Alberti}
Il cifrario si compone di due dischi: uno esterno e uno interno.
\begin{itemize}
	\item \textbf{Disco esterno}: composto dai caratteri dell'alfabeto (in ordine) e in pi\`u aggiungo qualche carattere
	      speciale (in genere dei numeri) alla fine.
	      \begin{center}
		      A B C D E F G H I L M N O P Q R S T U V Z 1 2 3 4 5
	      \end{center}
	\item \textbf{Disco interno}: composto dallo stesso numero di caratteri del primo disco ma si usa un alfabeto
	      pi\`u ricco e i caratteri sono permutati.
	      \begin{center}
		      E Q H C W L M V P D N X A O G Y I B Z R J T S K U F
	      \end{center}
\end{itemize}
Per riuscire a comunicare, i due utenti, devono possedere ognuno una copia identica del cifrario.

Il cifrario fa uso di una chiave composta da una coppia di caratteri: il primo deve essere presente sul disco esterno, il
secondo su quello interno. Una volta decisa la chiave iniziale si impostano i due dischi.

Cifratura e decifrazione avvengono mettendo in corrispondenza disco esterno ed interno. Se per esempio sto cifrando il
messaggio mi baster\`a far corrispondere alla lettera del messaggio la lettera corrispondete sul disco permutato.

Ci\`o che rende il cifrario interessante e anche molto robusto \`e il \textbf{cambio di chiave} che si pu\`o effettuare
in qualsiasi momento. Anzi, pi\`u cambi di chiave effettuiamo, pi\`u il nostro messaggio sar\`a difficile da decifrare.

Per effettuare il cambio di chiave (in fase di cifratura) dobbiamo inserire, all'interno del messaggio, i caratteri speciali.

Di seguito due possibili tecniche per l'utilizzo efficace del cambio di chiave.
\begin{itemize}
	\item Nel primo metodo inseriamo un carattere speciale nel messaggio in chiaro e lo cifriamo secondo la chiave che stiamo
	      usando. A questo punto abbiamo cambiato chiave di cifratura: la prima lettera della chiave, ossia quella del disco
	      esterno rimane la stessa ma la seconda lettera diventa la lettera nella quale il carattere speciale viene cifrato.

	      Questo procedimento equivale a ruotare il disco interno e iniziare quindi a cifrare il messaggio in un modo diverso.
	\item Il secondo metodo, a "\textbf{indice mobile}", \`e pi\`u complesso e rende il cifrario un po' pi\`u robusto.

	      Come prima cosa inserisco un carattere speciale (stavolta dev'essere necessariamente un numero) all'interno del
	      messaggio e lo cifro con la chiave corrente. Quel numero mi dice quanti caratteri sono ancora cifrati con la chiave
	      corrente.

	      Dopo quel numero di caratteri avr\`o un carattere che non fa parte del messaggio ma che mi indicher\`a la nuova
	      chiave da usare.
\end{itemize}
Cambiando molto spesso la chiave, attacchi di tipo statistico, sono praticamente inutili.

\subsubsection{Cifrario di Vigen\`ere}
\`E un cifrario in cui si fa uso di una chiave \textbf{corta} e \textbf{ripetuta ciclicamente}.

Il cifrario funziona cos\`i
\begin{enumerate}
	\item Si prende una sequenza corta di caratteri dell'alfabeto, diciamo lunga $n$ caratteri.
	\item Si prende la posizione che ciascun carattere occupa nell'alfabeto. La posizione di ciascun carattere
	      nell'alfabeto indica una traslazione che si dovr\`a effettuare in fase di cifratura.
	\item Consideriamo il messaggio a gruppi di $n$ caratteri.
	\item Cifriamo la lettera in posizione $i$ della parte di messaggio in chiaro che sto considerando andando a traslarla
	      di un valore pari al valore di traslazione corrispondente alla lettera in posizione $i$ nella chiave.
	\item Ripeto il procedimento per tutto il messaggio cifrando sempre gruppi di $n$ caratteri.
\end{enumerate}
In questo modo, alla stessa lettera del messaggio, potrebbe corrispondere una traslazione diversa a seconda del valore di
traslazione indicato dalla chiave.

La debolezza del cifrario consiste nell'utilizzo di una chiave corta e ripetuta. Se si riesce infatti a stimare
correttamente la lunghezza della chiave si pu\`o capire si possono decifrare parti di messaggio andando anche a fare
analisi sulla frequenza dei caratteri nel linguaggio naturale.

Per ovviare al problema basta usare una chiave casuale e non riutilizzabile lunga quanto il messaggio in chiaro. In questo
modo il cifrario diventa \textbf{inattacabile} e si ottiene un \textbf{one-time pad}, che andremo a trattare pi\`u avanti.

\section{Cifrari a trasposizione}
I \textbf{cifrari a trasposizione} permutano le lettere del messaggio in chiaro secondo una regola prefissata ed
eventualmente aggiungono altre lettere che verranno poi ignorate in fase di decifrazione.

\subsection{Cifrario a permutazione semplice}
In questo cifrario utilizziamo un intero $h$ e una permutazione $\pi$ dei primi $h$ interi come chiave.

Per effettuare la cifratura si opera in questo modo
\begin{enumerate}
	\item Si suddivide il messaggio in blocchi di $h$ lettere.
	\item Si permutano le lettere di ciascun blocco secondo $\pi$, ovvero, si va a vedere il numero in posizione $i$
	      nella permutazione e si va a spostare il carattere in posizione $\pi[i]$ nel blocco alla posizione $i$.
\end{enumerate}
Le chiavi di cifratura sono $h! - 1$ dato che consideriamo tutte possibili permutazioni di $h$ elementi e togliamo quella
identica.

\subsection{Cifrario a permutazione di colonne}
In questo cifrario usiamo una chiave composta da due interi $c$ ed $r$ e da una permutazione $\pi$ dei primi $c$ interi.

I valori $c$ ed $r$ denotano il numero di righe e di colonne di una \textbf{tabella} di lavoro $T$.

\begin{enumerate}
	\item Il messaggio viene decomposto in blocchi di $c \times r$ caratteri.
	\item Scrivo ogni blocco per righe dall'alto verso il basso dentro $T$.
	\item Permuto le colonne di $T$ secondo $\pi$ come nel cifrario a permutazione semplice.
	\item Formo il crittogramma leggendo la tabella permutata per colonne.
\end{enumerate}
Il numero delle chiavi \`e teoricamente esponenziale dato che non ci sono limiti sulla scelta di $r$ e $c$.

\subsection{Cifrario a griglia}
La chiave segreta \`e una griglia quadrata $q \times q$ con $q$ pari.
\begin{enumerate}
	\item Si considera un numero
	      \[ s = q^2 / 4 \]
	      di celle (un quarto del totale).
	\item Si scrivono i primi $s$ caratteri del messaggio nelle posizioni corrispondenti alle $s$ celle scelte.
	\item Si ruota la griglia di 90 gradi e si scrivono altri $s$ caratteri del messaggio.
	\item Ripeto la rotazione per 3 volte.
\end{enumerate}
Le posizioni delle $s$ celle devono essere scelte in modo che non si sovrappongano mai nelle quattro rotazioni. Il numero
delle chiavi \`e dunque $4^s$ dato che ho 4 possibili scelte per ognuna delle $s$ celle.

\section{Crittoanalisi statistica}
Come abbiamo solo accennato in qualche occasione, la sicurezza di un cifrario \`e legata alla dimensione dello spazio
delle chiavi, tuttavia, questa \`e condizione necessaria ma non sufficiente a rendere il cifrario sicuro.

Uno spazio di chiavi molto grande protegge solamente da attacchi di tipo \emph{forza bruta}. Un crittoanalista ha tuttavia
molti metodi a disposizione per riuscire a forzare un cifrario: le chiavi potrebbero essere generate male, troppo corte,
prevedibili o riutilizzate oppure si potrebbe conoscere il formato del messaggio.

Un crittoanalista pu\`o aggrapparsi a tutte queste cose per riuscire a forzare il cifrario anche se dal punto di vista
algoritmico esso non presenta vulnerabilit\`a.

Un altro metodo con il quale si \`e riusciti a forzare molti dei cifrari storici \`e la \textbf{crittoanalisi statistica}.

Quello che si fa non \`e cercare una vulnerabilit\`a nell'algoritmo che ha prodotto il crittogramma ma si va a fare
analisi statistica sul crittogramma in nostro possesso.

Avendo a disposizione solo il crittogramma possiamo compiere attacchi di tipo \textbf{cipher text} andando per esempio
a fare \textbf{analisi delle frequenze}.

Si va cio\`e a vedere quali sono le lettere pi\`u frequenti nel crittogramma e le confrontiamo con le lettere pi\`u
frequenti del linguaggio naturale (ovviamente nella lingua in cui stiamo comunicando).

\subsection{Attacchi}
Facciamo prima qualche ipotesi affinch\'e abbia senso fare della crittoanalisi statistica.
\begin{itemize}
	\item Si deve conoscere il metodo impiegato per la cifratura e decifrazione.
	\item Si deve conoscere il linguaggio in cui \`e scritto il messaggio.
	\item Si ammette che il messaggio sia sufficientemente lungo da poter rilevare alcuni dati statistici sui caratteri
	      che compongono il crittogramma.
\end{itemize}
Premettiamo che la frequenza con cui appaiono in media le lettere dell'alfabeto \`e ben studiata in ogni lingua e sono
anche presenti dati simili per \emph{digrammi}, \emph{trigrammi} ecc.

\subsubsection{Sostituzione monoalfabetica}
Se il metodo di cifratura \`e di tipo \emph{monoalfabetico} significa che ad una lettera del crittogramma corrisponde
sempre la stessa lettera del messaggio e viceversa.

Questo significa anche che la frequenza della lettera cifrata nel crittogramma \`e uguale alla frequenza della lettera in
chiaro nel linguaggio naturale.

Basta dunque prendere un istogramma delle frequenze delle varie lettere e dopo pochi tentativi la procedura converge in
qualche corrispondenza significativa.

\subsubsection{Sostituzione polialfabetica}
In questo caso, l'istogramma delle frequenze del crittogramma risulta piatto dato che con questo metodo di sostituzione
una lettera del crittogramma potrebbe corrispondere a lettere diverse del messaggio.

Il cifrario di Vigen\`ere \`e stato tuttavia forzato sfruttando la debolezza relativa ad avere una chiave corta e ripetuta.

Se la chiave contiene $h$ caratteri, le apparizioni della stessa lettera, distanti un multiplo di $h$ nel messaggio, si
sovrappongono alla stessa lettera della chiave. Supponiamo quindi di conoscere la lunghezza $h$ della chiave
\begin{enumerate}
	\item Si costruisce un sottomessaggio formato dal carattere in posizione $i$ ($i \leq h$) e da tutti i caratteri
	      in posizione $i + k \cdot h$.
	\item Ognuno di questi sottomessaggi \`e cifrato in modo monoalfabetico (stesso valore di traslazione).
	\item Si effettua un'analisi delle frequenze per ciascun sottomessaggio.
\end{enumerate}

Se per\`o non conosciamo la lunghezza $h$ della chiave dobbiamo andare a sfruttare qualche propriet\`a del linguaggio
naturale per riuscire a stimarla pi\`u correttamente possibile.

Il messaggio contiene quasi sicuramente gruppi di lettere ripetuti pi\`u volte come ad esempio trigrammi pi\`u frequenti
nella lingua o parole ricorrenti nel testo.

Pu\`o capitare che questi $q$-grammi vengano cifrati con la stessa porzione di chiave e quindi in modo identico l'uno
all'altro.

Andiamo quindi a cercare nel crittogramma coppie di sequenze identiche. Se trovo due sequenze, in posizione $p_1$ e $p_2$,
con le caratteristiche citate in precedenza, \`e probabile che la distanza $p_2 - p_1$ sia la lunghezza della
chiave o ad un suo multiplo.

\subsubsection{Cifrari a trasposizione}
Come detto in precedenza questi cifrari permutano i caratteri del messaggio. Non ha dunque senso condurre un attacco di
tipo statistico basato sulle frequenze.

Sono tuttavia utili le propriet\`a sintattiche e lessicali del linguaggio, per esempio lo studio dei $q$-grammi.

Se si vuole forzare un cifrario a permutazione semplice e si conosce la lunghezza $h$ della chiave si opera in questo
modo:
\begin{enumerate}
	\item Si divide il crittogramma in blocchi di $h$ caratteri.
	\item Si cercano gruppi di $q$ lettere che formano i $q$-grammi pi\`u frequenti nel linguaggio.
	\item Se un blocco della permutazione deriva da un $q$-gramma si scopre parte della permutazione.
\end{enumerate}

\subsection{Conclusione}
In crittoanalisi statistica lo studio della frequenza delle varie lettere nel linguaggio \`e quindi molto utile per
capire il metodo di cifratura utilizzato
\begin{itemize}
	\item Nei cifrari a trasposizione l'istogramma delle frequenze coincide approssimativamente con quello proprio del
	      linguaggio.
	\item Nei cifrari a sostituzione monoalfabetica l'istogramma relativo al crittogramma e quello delle frequenze delle
	      varie lettere nel linguaggio naturale si somigliano a meno di una permutazione delle lettere.
	\item Nei cifrari a sostituzione polialfabetica l'istogramma delle frequenze \`e piatto.
\end{itemize}

\section{ENIGMA}
L'ultimo cifrario storico di cui andiamo a parlare \`e la macchina \textbf{ENIGMA}. Di grande rilievo storico \`e questo
cifrario perch\'e segna un punto di svolta verso sistemi crittografici automatizzati.

Storicamente fu impiegato dai tedeschi nella seconda guerra mondiale per l'invio di informazioni criptate.

Come vedremo si tratta di un'\emph{estensione} del cifrario di Alberti, visto precedentemente.

\subsection{Utilizzo}
La macchina aveva le sembianze di una macchina da scrivere. Quando si voleva formare il crittogramma si digitava sulla
tastiera il messaggio in chiaro ma ci\`o che veniva effettivamente era del testo cifrato in un modo che vedremo pi\`u
avanti.

Al contrario, quando si voleva decifrare un crittogramma, lo si digitava sulla macchina e questa forniva il relativo testo
in chiaro.

\subsection{Funzionamento}
Come gi\`a anticipato, quando si premeva un tasto sulla tastiera, il carattere corrispondente non era quello premuto ma
un altro che variava a seconda di un po' di cose.

Come prima cosa si scambiava lo spinotto che collegava il tasto di una lettera con una lettera diversa.

Per complicare ulteriormente le cose, ENIGMA, era dotata di alcuni \textbf{rotori}, i quali servivano a mappare una
lettera in un'altra lettera. In genere ogni macchina ENIGMA possedeva tre rotori.

Ogni volta che un rotore effettuava uno scambio, per contatto, andava a attivare la lettera di un altro rotore che a sua
volta effettuava un ulteriore scambio.

Una volta finiti i rotori si arrivava ad un \textbf{riflettore}, il quale compiva lo stesso scambio effettuato internamente
dai riflettori e innescava un ulteriore sequenza di scambi ripassando da tutti i rotori in senso inverso.

Detto cos\`i sembrerebbe un cifrario a permutazione semplice ma quello ci\`o che non abbiamo detto \`e che i rotori, ogni
volta che si batteva una lettera, avanzavano di un passo.

In realt\`a era solo un rotore ad avanzare, almeno finch\'e non aveva compiuto 26 passi. Una volta digitate 26 lettere
il rotore tornava in posizione iniziale e il rotore successivo iniziava la rotazione.

In questo modo la chiave cambiava ogni qual volta si digitava un carattere.

\subsection{Chiavi}
Analizziamo quante sono le possibili permutazioni generabili da ENIGMA.

Ogni rotore genera 26 permutazioni con tutte le sue rotazioni giungendo cos\`i ad un totale di $26^3$ permutazioni
possibili (circa $17.000$).

Un numero troppo basso per la comunicazione di messaggi in ambito militare. Vedremo infatti che la macchina verr\`a
complicata.

\subsection{Debolezze}
\begin{itemize}
	\item Numero di chiavi basso.
	\item I rotori erano immutabili e uguali su ogni macchina ENIGMA.
	\item Le $26^3$ permutazioni erano sempre le stesse ed erano applicate sempre nello stesso ordine.
\end{itemize}

\subsection{Modifiche}
Una prima modifica per aumentare il numero di possibili permutazioni \`e stata quella di permutare tra loro i rotori. In
questo modo giungiamo cos\`i ad un numero di possibili permutazioni pari a $3! \cdot 26^3$ (pi\`u di $10^5$).

Un'altra modifica \`e stata l'aggiunta del \textbf{plugboard} tra la tastiera e il primo rotore che consentiva di scambiare
i caratteri di 6 coppie scelte arbitrariamente.

Ogni cablaggio \`e descritto da una sequenza di 12 caratteri scelti da un insieme totale di 26. Abbiamo quindi un numero
di possibili combinazioni pari a $\begin{psmallmatrix} 26 \\ 12 \end{psmallmatrix}$ che \`e circa $10^7$.

Ogni gruppo di 12 lettere si pu\`o presentare in $12!$ permutazioni diverse, ma non tutte producono lo stesso effetto.
Dobbiamo quindi dividere $10^7$ per $12!$ e per $2^6$.

Il numero di chiavi totale \`e quindi dato da
\[
	3! \cdot 26^3 \cdot
	\begin{pmatrix}
		26 \\ 12
	\end{pmatrix}
	\frac{12!}{6! \cdot 2^6}
\]
Per un totale $10^{16}$ chiavi.

Durante la seconda guerra mondiale si davano in dotazione 8 rotori da cui sceglierne 3 e il numero di coppie scambiabili
con il plugboard \`e passato da 6 a 10.

\subsection{Decifrazione}
La decifrazione di ENIGMA \`e merito degli inglesi, tra cui anche Alan Turing, i quali erano venuti in possesso di un
modello di ENIGMA e grazie alla costruzione di un simulatore di quest'ultima, il famoso COLOSSUS, il quale provava a
capire quale fosse una possibile configurazione della macchina a partire dal crittogramma.

Ogni possessore di una macchina ENIGMA aveva un registro di \textbf{chiavi giornaliere}. Le chiavi nel registro erano usate
per impostare l'assetto iniziale della macchina. Con quell'assetto si comunicava un nuovo assetto da usare per quella
specifica trasmissione. Il nuovo assetto veniva usato per effettuare l'effetiva cifratura del messaggio.