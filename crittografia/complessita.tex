\chapter{Teoria della complessit\`a}\label{complessita}
La \textbf{complessit\`a} \`e quella branca dell'informatica teorica che classifica i problemi in base alla
difficolt\`a che si ha nel risolverli, ossia al numero di operazioni necessarie per risolverli.

Nel capitolo \ref{calcolabilita} abbiamo trattato i problemi decidibili e non. In questo capitolo restringiamo il
campo solo ai problemi decidibili ma li dividiamo in due sottocategorie: \textbf{trattabili} e \textbf{intrattabili}.

Si tratta in entrambi i casi di problemi risolvibili, ma nel caso degli intrattabili, il costo computazionale \`e
talmente alto (generalmente esponenziale o pi\`u) da renderli molto simili ai problemi indecidibili, dato che
la risoluzione richiederebbe decine se non centinaia o migliaia (in base alla dimensione dell'input) di anni di
calcolo.

Al contrario, i problemi trattabili, hanno algoritmi di costo polinomiale e dunque sono risolvibili in tempi brevi.

Ci sono infine problemi il cui stato non \`e noto: abbiamo solo algoritmi di costo esponenziale per la loro
risoluzione ma non abbiamo dimostrazione del fatto che siano intrattabili.

\section{Algoritmi polinomiali ed esponenziali}\label{alg_poly_exp}
Studiamo ora la dimensione dei dati trattabili in funzione dell'incremento della velocit\`a dei calcolatori.

\begin{example}
	Siano $C_1$ e $C_2$ due calcolatori, con $C_2$ che \`e $k$ volte pi\`u veloce di $C_1$ e testiamo quanti dati
	riescono a processare i due calcolatori in un tempo $t$.

	Il calcolatore $C_1$ processa $n_1$ dati nel tempo $t$ mentre $C_2$ processa $n_2$ dati sempre in tempo $t$.
	Possiamo osservare che, usare $C_2$, \`e come usare $C_1$ per un tempo $kt$.

	Se l'algoritmo che stiamo usando \`e \textbf{polinomiale} e risolve il problema in un tempo $cn^s$, dove
	$c$ e $s$ sono costanti, abbiamo che
	\[	\begin{array}{llcl}
			C_1 : & c n_1^s = t  & \rightarrow &
			n_1 = \left( \displaystyle\frac{t}{c} \right)^{1/s} \\

			C_2 : & c n_2^s = kt & \rightarrow &
			n_2 = k^{1/s}\left( \displaystyle\frac{t}{c} \right)^{1/s}
		\end{array}
	\]
	Otteniamo quindi
	\[ n_2 = k^{1/s} \cdot n_1 \]
	abbiamo dunque un miglioramento di un \textbf{fattore moltiplicativo} $k^{1/s}$ tanto pi\`u grande quanto pi\`u
	piccolo \`e il grado ($s$) del polinomio.

	Se invece facciamo girare sugli stessi calcolatori un algoritmo di costo esponenziale, che risolve il problema in
	un tempo $c \cdot 2^n$ con $c$ costante, otteniamo
	\[
		\begin{array}{llcl}
			C_1 : & c \cdot 2^{n_1} = t   & \rightarrow & 2^{n_1} = \displaystyle\frac{t}{c}         \\
			C_2 : & c \cdot 2^{n_2} = k t & \rightarrow & 2^{n_2} = k \cdot \displaystyle\frac{t}{c}
		\end{array}
	\]
	Otteniamo quindi
	\[ 2^{n_2} = k \cdot 2^{n_1} \]
	da cui ricaviamo
	\[ n_2 = n_1 + \log k \]
	In questo caso il miglioramento \`e solo di un \textbf{fattore additivo} $\log k$. Questo significa che per
	quanto si incrementi la velocit\`a di elaborazione $k$ della macchina $C_2$ rispetto a $C_1$, i dati elaborati in
	pi\`u sarebbero solo $\log k$.
\end{example}

\section{Problemi}\label{problemi}
Trattiamo ora le varie \textbf{classi di complessit\`a} e come si definiscono. Indichiamo con $\Pi$, un generico
\textbf{problema}, con $I$, l'insieme delle \textbf{istanze} in ingresso e con $S$ l'insieme delle \textbf{soluzioni}.

\subsection{Tipologie di problemi}
Abbiamo tre \textbf{classi} principali di problemi: di \textbf{ricerca}, di \textbf{ottimizzazione} e
\textbf{decisionali}.

\subsubsection{Problemi di ricerca}
Sono problemi in cui, data un'istanza $x$ in input, si richiede di restituire una soluzione $s$, presa da un insieme
$S$ di possibili soluzioni.

Un esempio di problema di ricerca \`e quello in cui si deve calcolare un cammino che unisce due nodi di un grafo. In
questo caso $S$ sar\`a uguale a tutti i possibili cammini che vanno dal nodo di partenza a quello di arrivo.

Le \textbf{istanze positive} saranno le coppie di nodi per le quali esiste almeno un cammino che li unisce, viceversa
le \textbf{istanze negative} saranno le coppie di nodi per le quali non esiste nemmeno un cammino che li unisce.

\subsubsection{Problemi di ottimizzazione}
Nei problemi di ottimizzazione, data un'istanza $x$ in input, si vuole trovare la \textbf{migliore soluzione} $s$ tra
tutte le soluzioni possibili.

In questo caso $S$ sar\`a l'insieme di tutte le soluzioni ottime al problema (diverse fra loro ma con stesso valore).

\subsubsection{Problemi decisionali}
I problemi decisionali richiedono una risposta binaria, in genere \verb|true| o \verb|false|, ossia con un insieme delle
soluzioni
\[ S = \{ 0, 1 \} \]
le \textbf{istanze positive} sono
\[ x \in I \mid \Pi(x) = 1 \]
mentre le \textbf{istanze negative} sono
\[ x \in I \mid \Pi(x) = 0 \]
Tipicamente sono problemi che indagano delle propriet\`a.

In complessit\`a si prende in considerazione solo quest'ultima classe di problemi per due motivi principalmente:
\begin{itemize}
	\item Tutto il tempo di calcolo \`e speso per trovare la soluzione e non per scriverla (la risposta \`e solo 0 o 1).
	\item La complessit\`a di un problema, riformulato in forma decisionale, non cambia rispetto alla sua forma non
	      decisionale.
\end{itemize}
In realt\`a tanti problemi interessanti sono problemi di ottimizzazione e per trattarli dobbiamo riformularli in
forma decisionale. Per farlo, basta verificare l'esistenza di una soluzione (al problema di ottimizzazione) che
soddisfa una certa propriet\`a.

Questo ci dice che il problema di ottimizzazione \`e almeno tanto difficile quanto il corrispondente problema
decisionale.

Caratterizzare il problema decisionale ci da un \textbf{limite inferiore} alla complessit\`a del relativo problema di
ottimizzazione.

\section{Classi di complessit\`a}\label{classi}
Un problema, in base alle risorse spese per la sua risoluzione, viene classificato in diverse
\textbf{classi di complessit\`a}.

\begin{theorem}
	Dato un problema decisionale $\Pi$ ed un algoritmo $A$, diciamo che $A$ \textbf{risolve} $\Pi$ se, data un'istanza di
	input $x$, vale
	\[ A(x) = 1 \quad \Leftrightarrow \quad \Pi(x) = 1 \]
\end{theorem}

\begin{theorem}
	Dato un problema decisionale $\Pi$, un algoritmo $A$ e la dimensione $n$ dell'input, diciamo che $A$ risolve $\Pi$
	in tempo $t(n)$ e spazio $s(n)$ se il tempo di esecuzione e l'occupazione di memoria di $A$ sono rispettivamente
	$t(n)$ e $s(n)$.
\end{theorem}

Data una funzione $f$ qualsiasi e la dimensione $n$ dell'input, possiamo definire due classi di complessit\`a:
\begin{itemize}
	\item \textbf{Time}: l'insieme dei problemi decisionali che possono essere risolti in tempo $O(f(n))$.
	\item \textbf{Space}: l'insieme dei problemi decisionali che possono essere risolti in spazio $O(f(n))$.
\end{itemize}
Da queste due classi di problemi possiamo derivare altre sottoclassi pi\`u specifiche
\begin{itemize}
	\item \textbf{P}: l'insieme dei problemi risolvibili in tempo polinomiale nella dimensione $n$ dell'istanza di input.

	      In questo tipo di problemi abbiamo due costanti: $c$ e $n_0 > 0$ tali che il numero di passi elementari \`e al
	      pi\`u $n^c$ per ogni input di dimensione $n$ e per ogni $n > n_0$.

	\item \textbf{P-Space}: l'insieme dei problemi risolvibili in spazio polinomiale nella dimensione $n$ dell'istanza
	      di input.

	      In questo tipo di problemi abbiamo due costanti: $c$ e $n_0 > 0$ tali che il numero di celle di memoria
	      utilizzate \`e al pi\`u $n^c$ per ogni input di dimensione $n$ e per ogni $n > n_0$.
	\item \textbf{Exp Time}: l'insieme dei problemi risolvibili in tempo esponenziale nella dimensione $n$ dell'istanza
	      di ingresso.
\end{itemize}

\subsection{Relazioni tra classi}
Si congettura che
\[ \text{P} \subseteq \text{P-Space} \]
questo perch\'e un algoritmo di costo polinomiale pu\`o avere accesso al pi\`u ad un numero polinomiale di locazioni di
memoria diverse (in ordine di grandezza).

Si congettura anche che
\[ \text{P-Space} \subseteq \text{Exp Time} \]
questo perch\'e non \`e detto che un algoritmo di costo esponenziale richieda un numero esponenziale di celle di memoria.

\subsection{Classe NP e certificati}
Introduciamo ora una nuova classe di problemi, la classe \textbf{NP}, dove NP sta per \textbf{P}olinomiale su modelli
\textbf{N}on deterministici.

Per le \textbf{istanze accettabili} (\textbf{positive}) $x$ di alcuni problemi, \`e possibile fornire un
\textbf{certificato polinomiale} $y$ che possa convincerci del fatto che l'istanza soddisfa la propriet\`a e dunque
\`e un'istanza \emph{accettabile}.

I certificati servono per dare una risposta in tempi pi\`u brevi a certi problemi decisionali. Un certificato \`e di
fatto una descrizione breve di una soluzione che possiede la propriet\`a cercata.

\begin{example}
	Prendiamo il problema della Clique per esempio. Una clique, dato un grafo non orientato, \`e il sottografo
	connesso pi\`u grande, rispetto al grafo che sto considerando.

	Esiste solo un algoritmo di costo esponenziale per trovare una clique in un grafo ma non sappiamo se possa essere
	risolto in tempo polinomiale.

	Vogliamo verificare che all'interno di questo grafo ci sia una clique di $k$ vertici.

	Se abbiamo un certificato polinomiale contenente un'insieme di $k$ vertici che formano una clique, non ci rimane
	che verificare se quei vertici formano effettivamente una clique (tempo polinomiale).
\end{example}

Un certificato \`e un \emph{attestato breve di esistenza} di una soluzione con determinate propriet\`a e si definisce solo
per istanze accettabili. Infatti, in generale, non \`e facile costruire attestati di non esistenza di una certa soluzione.

\subsubsection{Verifica}
L'\textbf{idea} \`e quella di utilizzare il costo della \textbf{verifica} del certificato per caratterizzare la
complessit\`a del problema.

Un problema $\Pi$ \`e \emph{verificabile in tempo polinomiale} se
\begin{itemize}
	\item Ogni istanza $x$ accettabile di $\Pi$ di dimensione $n$, ammette un certificato $y$ di dimensione
	      polinomiale in $n$.
	\item Esiste un algoritmo di verifica polinomiale in $n$, applicabile a ogni coppia $\langle x, y \rangle$, che
	      permette di attestare che $x$ \`e accettabile.
\end{itemize}
Un altro modo di definire i problemi della classe NP \`e quello di problemi \emph{verificabili in tempo polinomiale}.

\begin{observation}
	Un certificato viene trovato in tempo esponenziale e ci serve \textbf{solo} a caratterizzare un certo problema in
	base al costo della sua verifica.
\end{observation}

\begin{theorem}
	La classe P \`e un sottoinsieme di NP. Un problema risolvibile in tempo polinomiale \`e sicuramente verificabile in
	tempo polinomiale.
\end{theorem}

\subsubsection{Problemi NP-completi}
I problemi \textbf{NP-completi} sono i problemi \emph{pi\`u difficili} all'interno della classe NP. Se esistesse un
algoritmo polinomiale per risolvere uno solo di questi problemi, allora tutti i problemi in NP potrebbero essere risolti
in tempo polinomiale e dunque giungeremmo alla conclusione che $\text{P} = \text{NP}$.

\begin{theorem}\label{th: NP-completi_poly}
	Tutti i problemi NP-completi sono risolvibili in tempo polinomiale oppure nessuno lo \`e.
\end{theorem}

\subsection{Riduzioni polinomiali}
Introduciamo ora le \textbf{riduzioni polinomiali}, necessarie per comprendere meglio il teorema
\ref{th: NP-completi_poly}.

\begin{theorem}
	Siano $\Pi_1$ e $\Pi_2$ due problemi decisionali e siano $I_1$ e $I_2$ gli insiemi delle istanze di input di
	$\Pi_1$ e $\Pi_2$, allora $\Pi_1$ si \textbf{riduce} in tempo polinomiale a $\Pi_2$
	\[ \Pi_1 \leq_p \Pi_2 \]
	se esiste una funzione $f : I_1 \rightarrow I_2$, calcolabile in tempo polinomiale, tale che, per ogni istanza $x$
	di $\Pi_1$, $x$ \`e un'istanza accettabile di $\Pi_1$ se e solo se $f(x)$ \`e un'istanza accettabile di $\Pi_2$.
\end{theorem}

In sintesi, se $\Pi_1$ si riduce a $\Pi_2$, significa che in tempo polinomiale posso
\begin{enumerate}
	\item \emph{Tradurre} $\Pi_1$ in $\Pi_2$.
	\item Risolvere $\Pi_2$.
	\item La soluzione trovata \`e valida anche per $\Pi_1$.
\end{enumerate}

Supponiamo di avere un algoritmo di costo polinomiale per risolvere $\Pi_2$ e vogliamo usare lo stesso algoritmo per
risolvere $\Pi_1$.
\begin{enumerate}
	\item Prendiamo un'istanza $x$ di $\Pi_1$ e traduciamola in un'istanza accettabile per $\Pi_2$ tramite la funzione
	      $f$.
	\item A questo punto facciamo girare l'algoritmo per la risoluzione di $\Pi_2$ sull'istanza $f(x)$.
	\item La soluzione data dall'algoritmo vale anche per $\Pi_1$.
\end{enumerate}

\subsubsection{Problemi NP}

\begin{definition}
	Un problema decisionale $\Pi$ si dice \textbf{NP-arduo} se
	\[	\forall \; \Pi' \in \text{NP}, \quad \Pi' \leq_p \Pi \]
	Ovvero se ogni problema in NP \`e riducibile a $\Pi$.
\end{definition}

\begin{definition}
	Un problema decisionale $\Pi$ si dice \textbf{NP-completo} se
	\[ \Pi \in \text{NP} \quad \wedge \quad \forall \; \Pi' \in \text{NP}, \quad \Pi' \leq_p \Pi \]
	Ovvero se ogni problema in NP \`e riducibile a $\Pi$, anch'esso in NP.
\end{definition}

Dimostrare che un problema \`e in NP pu\`o essere facile: basta esibire un certificato polinomiale. Non \`e altrettanto
facile dimostrare che un problema \`e NP-arduo o NP-completo: bisognerebbe dimostrare che tutti i problemi in NP si
riducono polinomialmente a $\Pi$.

In realt\`a la prima dimostrazione di NP-completezza aggira il problema.

\begin{theorem}[Cook]
	Il problema SAT \`e NP-completo.
\end{theorem}

Cook ha mostrato che, dato un qualunque problema $\Pi \in \text{P}$ ed una qualunque istanza $x$ per $\Pi$, si pu\`o
costruire un'espressione booleana in forma normale congiuntiva, che descrive il calcolo di un algoritmo per risolvere
$\Pi$ su $x$. L'espressione \`e vera se e solo se l'algoritmo restituisce 1.

\begin{theorem}
	Un problema decisionale $\Pi$ \`e NP-completo se
	\[ \Pi \in \text{NP} \quad \wedge \quad \text{SAT} \leq_p \Pi \]
	o se $\Pi$ \`e riducibile ad qualsiasi altro problema NP-completo.
\end{theorem}

\subsubsection{Problemi NP equivalenti}
Il fatto che un problema $\Pi$ si possa ridurre a SAT lo rende NP-completo. Il fatto che $\Pi$ sia NP-completo rende
possibile ridurre SAT a $\Pi$. Questo rende SAT e $\Pi$ due problemi \textbf{NP-equivalenti}.

\begin{theorem}
	Tutti i problemi NP-completi sono tra loro NP-equivalenti.
\end{theorem}

\subsubsection{Classi co-P e co-NP}
Le ultime due classi di problemi che trattiamo sono le classe co-P e co-NP. Fanno parte di queste due classi i problemi
\textbf{complementari} ai problemi in P ed NP.

Dal punto di vista della complessit\`a, passare da un problema al suo complementare, cambia molto a seconda se il problema
sia in P oppure no.

Se ho un problema in P allora anche il suo complementare \`e in P: basta rispondere il contrario del problema di partenza.
Possiamo quindi affermare che P = co-P.

Discorso diverso per i problemi NP. Se un problema NP \`e tale in presenza di un certificato che lo verifichi in tempo
polinomiale, la verifica del suo complementare \`e, in generale, \emph{difficile} e richiede tempo esponenziale. Si
congettura quindi che NP $\neq$ co-NP.