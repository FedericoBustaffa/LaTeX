\chapter{Introduzione}

\section{Crittografia e cifrari}

\subsection{Crittografia e crittoanalisi}
La \textbf{crittografia} si occupa dei metodi di cifratura mentre la
\textbf{crittoanalisi} dei metodi di decifrazione di un messaggio.
Entrambe si possono riunire sotto la branca della \textbf{crittologia}.

\subsection{Cifratura}
La \textbf{funzione di cifratura}
\[ \mathcal{C}(m) = x \]
cifra una certo \textbf{messaggio} $m$ in un \textbf{crittogramma} $x$. Dev'essere
\emph{iniettiva} e la sua inversa sar\`a la cosidetta \textbf{funzione di decifrazione}.

\subsection{Cifrario di Cesare}
L'idea del cifrario \`e ottenere un crittogramma sostituendo ogni lettere del messaggio
con la lettera tre posizioni pi\`u avanti nell'alfabeto. Se si supera la Z, si riparte
dalla A.

La segretezza dipendeva dalla conoscenza del metodo. Una volta scoperto il metodo di
cifratura il cifrario diventa inutile.

\subsection{Cifrari per uso ristretto e per uso generale}
\begin{itemize}
	\item I \textbf{cifrari per uso ristretto} hanno la caratteristica di tenere la
	      funzione di cifratura e decifrazione, segrete in ogni loro aspetto.
	\item I \textbf{cifrari per uso generale}, al contrario, rendono pubbliche le
	      funzioni di cifratura e decifrazione. Tengono tuttavia una chiave segreta che
	      sar\`a diversa per ogni coppia di utenti e inserita come parametro nelle
	      funzioni di cifratura e decifrazione.

	      Se non si conosce la chiave non si deve poter risalire al messaggio in chiaro
	      neppure conoscendo le due funzioni.
\end{itemize}

\subsection{Chiavi}
Il numero delle chiavi dev'essere di ordine abbastanza grande da rendere impossibile
o molto dispendioso il metodo di provarle tutte (forza bruta). Inoltre la chiave deve
essere scelta in modo \textbf{casuale}.

\subsection{Attacchi}
L'obbiettivo di un attacco \`e ovviamente quello di forzare il sistema. Il metodo e il
livello di pericolosit\`a dipendono dalle informazioni in possesso del crittoanalista.
Di seguito alcune tipologie di attacco:
\begin{itemize}
	\item \textbf{Cipher Text}: Il crittoanalista rileva solo crittogrammi sul
	      canale e cerca di estrapolarne informazioni.
	\item \textbf{Know Plain-Text}: Il crittoanalista conosce una serie di coppie
	      (messaggio, crittogramma).
	\item \textbf{Chosen Plain-Text}: Il crittoanalista si procura una serie di coppie
	      (messaggio, crittogramma) relative ai messaggi in chiaro da lui scelti.
\end{itemize}

\subsection{Cifrario perfetto}
Un \textbf{cifrario perfetto} deve possedere le seguenti caratteristiche:
\begin{itemize}
	\item Il messaggio in chiaro e il crittogramma risultano del tutto scorrelati tra
	      loro.
	\item Nessuna informazione sul testo in chiaro pu\`o filtrare dal crittogramma.
	\item La conoscenza del crittoanalista non cambia dopo aver osservato un
	      crittogramma.
\end{itemize}
Ad oggi i cifrari usati non sono perfetti ma sono dichiarati \textbf{sicuri} perch\'e
rimasti inviolati e perch\'e la loro violazione si basa su problemi matematici molto
complessi. Se si riuscisse a trovare delle soluzioni efficienti per tali problemi 
la sicurezza di tali cifrari crollerebbe.

\subsection{Cifrari simmetrici}
In questi cifrari la \textbf{chiave di cifratura} \`e uguale alla
\textbf{chiave di decifrazione} ed \`e nota solo ai due comunicanti. Il messaggio \`e 
diviso in blocchi lunghi come la chiave. La chiave \`e utilizzata per trasformare un 
blocco del messaggio in un blocco del crittogramma.

\subsection{Cifrari a chiave pubblica}
In questo caso la \textbf{chiave di cifratura} \`e pubblica e anche le
\textbf{funzioni di cifratura e decifrazione} lo sono. Ci\`o che \`e privato \`e la
\textbf{chiave di decifrazione}.

In questo tipo di cifrari, la funzione di cifratura dev'essere di tipo
\textbf{one-way trap-door}, ovvero, cifrare dev'essere computazionalmente facile e
decifrare dev'essere computazionalmente difficile (a meno che non si conosca la chiave).

\subsection{Attacchi chosen plain-text}
Dato che la funzione di cifratura e anche la sua chiave sono pubbliche,
il crittoanalista pu\`o cifrare un numero a piacimento di messaggi, confrontarli con i
messaggi crittografati che sta provando a decifrare e, nel caso in cui trovi una
corrispondenza tra qualcuno dei suoi crittogrammi e qualcuno da decifrare significa che
ha trovato il messaggio in questione.