\chapter{Protocolli}
In questo capitolo andremo a trattare i protocolli di sicurezza tra i quali distingueremo
\begin{itemize}
	\item \textbf{Identificazione}: un sistema di elaborazione ha bisogno di accertarsi dell'\textbf{identit\`a}
	      di un utente che vuole accedere ai suoi servizi.
	\item \textbf{Autenticazione}: il destinatario di un messaggio deve essere in grado di accertare
	      l'\textbf{identit\`a del mittente} e l'\textbf{integrit\`a del crittogramma} ricevuto.
	\item \textbf{Firma digitale}: la firma digitale si pone tre obbiettivi:
	      \begin{itemize}
		      \item Il mittente non deve poter \textbf{negare l'invio} di un messaggio.
		      \item Il destinatario deve essere in grado di \textbf{autenticare} il messaggio.
		      \item Il destinatario non deve poter sostenere di aver ricevuto un messaggio diverso da quello inviato
		            dal mittente.
	      \end{itemize}
\end{itemize}
Come si pu\`o notare, ognuna di queste funzionalit\`a estende l'altra
\begin{itemize}
	\item L'autenticazione garantisce l'identificazione.
	\item La firma digitale garantisce l'autenticazione.
\end{itemize}
Ognuna di queste funzionalit\`a ha il compito di proteggere le comunicazioni da attacchi attivi, come per esempio
gli attacchi \emph{man in the middle}.

\section{Funzioni hash}
Per l'implementazione di queste funzionalit\`a faremo ricorso alle \textbf{funzioni hash}. Una funzione hash
\[ f : X \rightarrow Y \]
\`e una funzione tale che
\[ n = |X| >> m = |Y| \]
ossia, \`e tale se, definito il dominio $X$, il codominio $Y$ della funzione \`e molto pi\`u piccolo.

Inoltre, una funzione hash, ha molti elementi del dominio che vengono mappati nella stessa immagine del codominio.
Questo ci permette di \emph{partizionare} il dominio in sottoinsiemi
\[ X = X_1 \cup X_2 \cup \dots \cup X_m \]
tali che ogni elemento del sottoinsieme \`e mappato, dalla funzione hash, nella stessa immagine del codominio
\[ \forall i, \quad \forall x \in X_i \quad \quad f(x) = y \]
Una funzione hash \`e buona se la cardinalit\`a di ogni sottoinsieme \`e circa la stessa.

\subsection{Funzioni hash one-way}
Le funzioni hash usate in crittografia devono soddisfare tre requisiti principali
\begin{itemize}
	\item Per ogni $x \in X$ deve essere \emph{facile} calcolare
	      \[ y = f(x) \]
	\item \textbf{One-way}: per la maggior parte degli $y \in Y$ deve essere \emph{difficile} determinare $x \in X$
	      tale che
	      \[ f(x) = y \]
	\item \textbf{Claw-free}: deve essere \emph{difficile} determinare una coppia $x_1, x_2 \in X$ tali che
	      \[ f(x_1) = f(x_2) \]
\end{itemize}

\subsection{SHA-1}
Una delle funzioni hash, storicamente usata in crittografia, \`e \textbf{SHA-1}, la quale opera su sequenze lunghe
fino a $2^{64} - 1$ bit e produce immagini di 160 bit.

In particolare opera su blocchi di 160 bit, contenuti in un buffer di 5 registri da 32 bit ciascuno, in cui sono
caricati inizialmente dei valori pubblici.

Il messaggio viene concatenato con una sequenza di \emph{padding} che rende la sua lunghezza un multiplo di 512 bit.

Il contenuto dei registri varia nel corso dei cicli successivi in cui questi valori si combinano tra loro e con
blocchi di 32 bit provenienti dal messaggio.

La funzione sfrutta shift ciclici e una componente non lineare che varia per riuscire ad ottenere il valore hash per
ciascun messaggio in input.

\section{Identificazione}
Vediamo ora come viene applicato il protocollo di \textbf{identificazione} sia nel caso in cui ci trovassimo su un
canale sicuro sia nel caso in cui ci trovassimo su un canale insicuro.

\subsection{Canale sicuro}
Prendiamo per esempio la situazione in cui un utente voglia accedere ai propri file personali memorizzati su un
calcolatore ad accesso riservato ai membri della sua organizzazione.
\begin{enumerate}
	\item L'utente invia in chiaro nome utente e password.
	\item Dato che il canale \`e sicuro, un attacco pu\`o essere sferrato solo da un utente locale al sistema o da
	      un hacker.
\end{enumerate}
Il meccanismo di identifiaczione dovrebbe per\`o prevedere una \textbf{cifratura} della password tramite funzioni hash
one-way anche su canali sicuri.

Dobbiamo distinguere due casi: quando l'utente si registra e quando l'utente effettua tutti gli accessi successivi.

\subsubsection{Registrazione}
Possiamo associare la fase di registrazione alla fase di cifratura con una funzione hash one-way $h$:
\begin{enumerate}
	\item L'utente $u$ si registra fornendo per la prima volta la password $p$.
	\item Il sistema associa a $p$ due sequenze binarie che memorizza nel file delle password al posto di $p$.
	      \begin{itemize}
		      \item Un \textbf{seme} casuale $s$ prodotto da un generatore.
		      \item Il \textbf{valore hash} della concatenazione tra $p$ ed $s$.
		            \[ q = h(p s) \]
	      \end{itemize}
\end{enumerate}
Quello che quindi viene salvato dal sistema \`e l'\textbf{immagine hash} della password concatenata ad un seme casuale
e non la password in chiaro.

\subsubsection{Accesso}
In fase di accesso il sistema compie i seguenti passaggi
\begin{enumerate}
	\item Recupera $s$ dal file delle password.
	\item Concatena $s$ con la password $p$ fornita da $u$.
	\item Calcola il valore hash della sequenza $p s$.
	\item Se $q = h(p s)$ l'identificazione ha successo.
\end{enumerate}
Avere accesso al file delle password non fornisce informazioni interessanti perch\'e \`e computazionalmente difficile
ricavare la password dalla sua immagine hash.

\subsection{Canale insicuro}
Nel caso di canale insicuro non si pu\`o inviare la password in chiaro e dunque andremo a lavorare con sistemi a
chiave pubblica per l'invio della password.

In realt\`a un sistema non dovrebbe mai poter maneggiare password in chiaro ma solo una loro immagine hash.

Supponiamo che l'utente $u$ voglia accedere ai servizi forniti da un certo sistema $s$ e per farlo \`e necessario
che si identifichi.

Supponiamo che il sistema adotti un cifrario a chiave pubblica (per esempio l'RSA) per lo scambio sicuro dei dati
dell'utente. L'utente dispone quindi di una chiave pubblica $\langle e, n \rangle$ e di una chiave privata $d$.

Quello che avviene all'atto pratico \`e questo:
\begin{enumerate}
	\item $s$ genera un numero casuale $r < n$ e lo invia in chiaro a $u$.
	\item $u$ calcola
	      \[ f = r^d \mod{n} \]
	      che rappresenta la \textbf{firma} di $u$ su $s$ e lo invia a $s$.
	\item $s$ verifica che
	      \[ f^e \mod{n} = r \]
	      se l'uguaglianza \`e soddisfatta, l'identificazione ha successo.
\end{enumerate}
Come possiamo notare, le operazioni di cifratura e decifrazione sono invertite rispetto all'impiego standard dell'RSA
ma come sappiamo sono commutative:
\[ (x^e \mod{n})^d \mod{n} = (x^d \mod{n})^e \mod{n} \]
Chiariamo inoltre che solo $u$ pu\`o produrre $f$ dato che \`e l'unico che possiede il valore $d$.

Se il passo 3 va a buon fine, il sistema ha la garanzia che l'utente che ha richiesto l'identificazione sia
effettivamente $u$, anche se il canale \`e insicuro.

Il protocollo funziona bene a patto che il sistema sia onesto. Dato che \`e il sistema a fornire $r$, se questo \`e
effettivamente un valore casuale allora tutto va a buon fine, se invece \`e un valore con propriet\`a utili a ricavare
la chiave privata di $u$ si rischia una corruzione della firma.

\subsection{Protocollo a conoscenza zero}
Questo protocollo permette ad un utente di dimostrare la sua identit\`a o la sua conoscenza di un certo segreto,
convincendo un altro utente di essere in possesso di una certa capacit\`a, senza rivelare niente oltre alla
veridicit\`a di questa sua capacit\`a.

In questo nuovo paradigma abbiamo due utenti:
\begin{itemize}
	\item \textbf{Prover}: indicato con $P$, ha il compito di dimostrare la sua capacit\`a o conoscenza.
	\item \textbf{Verifier}: indicato con $V$, ha il compito di verificare la veridicit\`a di ci\`o che afferma $P$.
\end{itemize}

Supponiamo quindi che $P$ voglia dimostrare a $V$ di possedere una certa capacit\`a:
\begin{enumerate}
	\item $V$ mette alla prova $P$ chiedendogli di dimostrare la sua capacit\`a su un problema che lui gli pone.
	\item $P$ risponde.
	\item Se la risposta \`e corretta, $V$ genera un valore casuale e modifica il problema in base al valore generato.
	\item $V$ chiede a $P$ di risolvere il problema modificato.
	\item Se $P$ risponde bene, allora $V$ continua a modificare il problema e a chiedere a $P$ di risolverlo, fino ad
	      essere sicuro che $P$ possegga effettivamente la capacit\`a da lui dichiarata.
	\item Se $P$ sbaglia anche solo una risposta, allora $V$ pu\`o immediatamente dedurre che $P$ dica il falso.
\end{enumerate}
Se le sfide proposte da $V$ sono $k$, la probabilit\`a che $P$ stia dicendo il falso, \`e
\[ \left( \frac{1}{2} \right)^k \]
Possiamo quindi concludere che, pi\`u $k$ \`e alto, meno sono le probabilit\`a che $P$ stia dicendo il falso.

\subsubsection{Propriet\`a fondamentali}
Di seguito le propriet\`a fondamentali del protocollo:
\begin{itemize}
	\item \textbf{Completezza}: se $P$ \`e onesto, $V$ accetta sempre la dimostrazione.
	\item \textbf{Correttezza}: se $P$ \`e disonesto, la probabilit\`a che $P$ riesca ad ingannare $V$ \`e al pi\`u
	      $(1/2)^k$ con $k$ scelto da $V$.
	\item \textbf{Conoscenza zero}: se $P$ \`e onesto, un verificatore (anche disonesto), non pu\`o acquisire nessuna
	      informazione se non la veridicit\`a dell'affermazione.
\end{itemize}

\subsection{Protocollo di Fiat-Shamir}
Questa non \`e altro che un'applicazione del protocollo a conoscenza zero con chiavi pubbliche e private.

Il \emph{prover} \`e impersonato dall'utente che vuole dimostrare di essere il legittimo proprietario di una chiave
privata associata ad una certa chiave pubblica senza usarla su dati scelti dall'utente \emph{verifier}.

Il protocollo si basa sulla difficolt\`a del calcolo di una radice in modulo un numero composto.

L'utente $P$, in fase di preparazione
\begin{enumerate}
	\item Sceglie due numeri primi $p$ e $q$ molto grandi.
	\item Calcola $n = p \cdot q$.
	\item Sceglie una sorta di chiave privata $s < n$.
	\item Calcola
	      \[ t = s^2 \mod{n} \]
	\item Rende nota la coppia $\langle t, n \rangle$ e mantiene privata la tripla $\langle p, q, s \rangle$.
\end{enumerate}
A questo punto $P$ vuole dimostrare a $V$ di conoscere una radice di $t$, ovvero $s$, senza per\`o inviargliela.

Per $k$ volte, con $k$ scelto da $V$, si ripetono i seguenti passi
\begin{enumerate}
	\item $V$ chiede a $P$ di iniziare una iterazione.
	\item $P$ genera un numero casuale $r < n$, calcola
	      \[ u = r^2 \mod{n} \]
	      e lo invia a $V$.
	\item $V$ genera un bit casuale $e$ e lo invia a $P$.
	\item $P$ calcola
	      \[ z = r \cdot s^e \mod{n} \]
	      e lo invia a $V$.
	      \begin{itemize}
		      \item Se $e = 0$ allora $z = r \mod{n}$.
		      \item Se $e = 1$ allora $z = r \cdot s \mod{n}$
	      \end{itemize}
	\item $V$ calcola
	      \[ x = z^2 \mod{n} \]
	      e controlla se
	      \[ x = u \cdot t^e \mod{n} \]
	      Se l'uguaglianza \`e vera si torna al passo 1, altrimenti $P$ non \`e identificato.
\end{enumerate}

\subsubsection{Completezza}
In questo caso, se $P$ \`e davvero in possesso di una radice di $t$, il verificatore lo identifica.
\begin{itemize}
	\item Se $e = 0$ allora
	      \[ x \quad = \quad u \cdot t^e \mod{n} \quad = \quad u \mod{n} \]
	\item Se $e = 1$ allora
	      \[ x \quad = \quad z^2 \mod{n} \quad = \quad (r \cdot s^e)^2 \mod{n} \quad = \quad u \cdot t \mod{n} \]
\end{itemize}
Quindi $P$ supera tutte le iterazioni se conosce $s$.

\subsubsection{Correttezza}
Supponiamo che $P$ sia disonesto e che quindi non conosca $s$. Per ingannare $V$ deve riuscire a prevedere il bit
casuale generato da $V$ ad ogni iterazione.

Distinguiamo due casi:
\begin{itemize}
	\item Se $P$ prevede di ricevere $e = 0$ non modifica il protocollo e se la previsione \`e corretta supera il
	      test.
	\item Se $P$ prevede di ricevere $e = 1$
	      \begin{enumerate}
		      \item Al passo 2 del protocollo e invia
		            \[ r^2 \cdot t^{-1} \mod{n} \]
		      \item Al passo 4 del protocollo invia
		            \[ z = r \mod{n} \]
	      \end{enumerate}
	      Se al passo 5 la previsione \`e corretta, $P$ supera il test perch\'e $V$ controlla se
	      \[ x = z^2 = u \cdot t^e \]
	      e se $e = 1$ allora
	      \[ u \cdot t^e = u \cdot t \]
	      Dato che $z = r$ e che
	      \[ u = r^2 \cdot t^{-1} \]
	      otteniamo
	      \[ r^2 = r^2 \cdot t^{-1} \cdot t \]
	      e quindi
	      \[ r^2 = r^2 \]
\end{itemize}
Come possiamo vedere, il metodo funziona a patto che la previsione sul bit sia corretta e la previsione deve essere
fatta prima di ricevere $e$.

La probabilit\`a di prevedere il bit ad ogni passo \`e quindi di un $1/2$. Per $k$ passi abbiamo quindi una
probabilit\`a di $(1/2)^k$ di prevedere tutti i bit.

\section{Autenticazione}
L'autenticazione riguarda il messaggio e si occupa di accertare l'identit\`a del mittente e l'integrit\`a del
messaggio.

\subsection{Canale insicuro}
Per questo protocollo, su canale insicuro, useremo un sistema basato su cifrari simmetrici in cui mittente e
destinatario devono quindi accordarsi su una chiave segreta $k$.

Nella pratica il mittente
\begin{enumerate}
	\item Allega al messaggio un \textbf{MAC} (Message Authentication Code) $A(m, k)$, allo scopo di garantire la
	      provenienza e l'integrit\`a del messaggio.
	\item A questo punto ha due opzioni
	      \begin{itemize}
		      \item Invia la coppia $\langle m,\; A(m, k) \rangle$ in chiaro.
		      \item Cifra $m$ e spedisce $\langle C(m, k'),\; A(m, k) \rangle$ dove $C$ \`e una funzione di cifratura
		            e $k'$ la chiave pubblica o la chiave simmetrica segreta del cifrario scelto.
	      \end{itemize}
\end{enumerate}

Il destinatario invece
\begin{enumerate}
	\item Riceve il messaggio (se cifrato lo decifra).
	\item Dato che conosce $A$ e $k$ calcola a sua volta il MAC.
	\item Confronta il MAC calcolato con il MAC ricevuto.
\end{enumerate}
Se la verifica ha successo il messaggio \`e autenticato altrimenti lo si scarta.

\subsubsection{MAC}
Il MAC \`e un'immagine breve del messaggio che pu\`o essere generata solo da un mittente conosciuto dal destinatario
e pu\`o essere calcolato con cifrari asimettrici, simmetrici o funzioni hash one-way.

Quest'ultima opzione implementativa \`e la pi\`u frequente
\[ A(m, k) = h(m k) \]
dato che il calcolo di una funzione hash \`e molto veloce per chi sta cifrando ma computazionalmente molto dispendioso
per un crittoanalista, che, anche disponendo di $h$ e $m$ non \`e comunque in grado di risalire a $k$ in tempo
polinomiale.

Dato che $k$ viaggia all'interno di un MAC si dovrebbe invertire la funzione hash (costo esponenziale).

Il crittoanalista non pu\`o nemmeno sostituire (facilmente) il messaggio $m$ con un altro messaggio $m'$ perch\'e
dovrebbe allegare a $m'$ il MAC $A(m', k)$ che pu\`o produrre solo conoscendo $k$.

\section{Firma digitale}
Questo protocollo cerca di portare tutte le propriet\`a di una \textbf{firma manuale} (con carta e penna) in ambito
tecnologico. Una firma manuale infatti
\begin{itemize}
	\item \`E autentica e non falsificabile.
	\item Non \`e riutilizzabile.
	\item Non pu\`o essere ripudiata.
\end{itemize}
Anche il documento firmato deve essere \textbf{inalterabile}.

Come vedremo, una \textbf{firma digitale}
\begin{itemize}
	\item Non pu\`o consistere nella digitalizzazione di un documento scritto firmato manualmente perch\'e si
	      potrebbe facilmente contraffare.
	\item Deve avere una forma che dipenda dal documento su cui viene apposta, per essere inscindibile da
	      quest'ultimo.
	\item Pu\`o essere generata sia tramite cifrari simmetrici che asimmetrici.
\end{itemize}

\subsection{Protocollo 1: Diffie Hellman}
In questo protocollo il messaggio $m$ \`e in chiaro e firmato.

Supponiamo che un utente $u$ sia in possesso di una coppia di chiavi $\langle k_\text{pub}, k_\text{priv} \rangle$ e
che abbia a disposizione una funzione $C$ di cifratura e una funzione $D$ di decifrazione (commutative), voglia
firmare $m$ e inviarlo a $v$.

Per firmare $m$, l'utente $u$
\begin{enumerate}
	\item Genera la firma $f$ per $m$ calcolando
	      \[ f = D(m, k_\text{priv}) \]
	\item $u$ invia all'utente $v$ la tripla $\langle u, m, f \rangle$.
\end{enumerate}
L'utente $v$
\begin{enumerate}
	\item Riceve $\langle u, m, f \rangle$.
	\item Verifica l'autenticit\`a della firma calcolando e controllando che
	      \[ m = C(f, k_\text{pub}) \]
\end{enumerate}
Se la verifica va a buon fine allora $v$ accetta la firma.

\subsubsection{Limiti}
Questo protocollo ha il grosso limite di non riuscire a proteggere il messaggio in lettura, infatti anche se
inviassimo un crittogramma $c$ di $m$, sarebbe il risultato della cifratura di $m$ con la chiave pubblica.

Possiede tuttavia tutti i requisiti di una firma manuale
\begin{itemize}
	\item La chiave $k_\text{priv}$ \`e nota solo a $u$ e per ottenerla si fa un numero esponenziale di operazioni.
	\item Se $m$ venisse alterato dal crittoanalista non ci sarebbe pi\`u consistenza tra $m$ e $f$ e la verifica
	      fallirebbe.
	\item Poich\'e solo $u$ pu\`o aver prodotto $f$ non pu\`o ripudiarla.
	\item Dato che la firma \`e un'immagine di $m$ non \`e riutilizzabile su un altro messaggio $m'$.
\end{itemize}

\subsection{Protocollo 2}
Questo protocollo si propone di risolvere il problema del protocollo precedente, relativo all'impossibilit\`a di
proteggere il messaggio.

L'utente $u$
\begin{enumerate}
	\item Genera la firma $f$ per $m$ calcolando
	      \[ f = D(m, k_\text{priv}) \]
	\item Cifra la firma calcolando
	      \[ c = C(f, k) \]
	      dove $k$ pu\`o essere la chiave pubblica del destinatario oppure una chiave simmetrica segreta.
	\item Invia la coppia $\langle u, c \rangle$ a $v$.
\end{enumerate}
L'utente $v$
\begin{enumerate}
	\item Riceve $\langle u, c \rangle$.
	\item Ricava $f$ calcolando
	      \[ f = D(c, k) \]
	      con $k$ che pu\`o essere la sua chiave privata o una chiave simmetrica.
	\item Cifra $f$ con la chiave pubblica del mittente ottenendo il messaggio $m$
	      \[ m = C(f, k_\text{pub}) \]
\end{enumerate}
Se il messaggio \`e significativo l'identit\`a di $u$ \`e attestata altrimenti si butta via il messaggio.

\subsubsection{Algoritmo con RSA}
In questo caso abbiamo due coppie di chiavi
\begin{gather*}
	d_u, \quad \langle e_u, n_u \rangle \\
	d_v, \quad \langle e_v, n_v \rangle
\end{gather*}
La coppia del mittente \`e usata per produrre la firma e verificarla mentre la coppia del destinatario per decifrare
il crittogramma.

Supponiamo che $u$ sia il mittente e $v$ il destinatario, l'utente $u$
\begin{enumerate}
	\item Genera la firma del messaggio $m$ calcolando
	      \[ f = m^{d_u} \mod{n_u} \]
	\item Cifra $f$ con la chiave pubblica di $v$ calcolando
	      \[ c = f^{e_v} \mod{n_v} \]
	\item Invia la coppia $\langle u, c \rangle$ a $v$.
\end{enumerate}
L'utente $v$
\begin{enumerate}
	\item Riceve la coppia $\langle u, c \rangle$.
	\item Decifra $c$ calcolando
	      \[ f = c^{d_v} \mod{n_v} \]
	\item Decifra $f$ con la chiave pubblica di $u$
	      \[ m = f^{e_u} \mod{n_u} \]
\end{enumerate}
Se $m$ \`e significativo allora l'identit\`a del mittente \`e attestata.

Affinch\'e il procedimento venga effettuato correttamente \`e necessario che
\[ f < n_v \]
e perch\'e questo accada c'\`e bisogno che
\[ n_u \leq n_v \]
Questo impedisce a $v$ di inviare messaggi firmati e cifrati da $u$.

Di solito ogni utente ha due coppie di chiavi diverse, una per la firma e una per la cifratura, tali che le chiavi
per la firma siano per esempio minori di un certo valore $H$ e quelle di cifratura siano invece maggiori. Il valore
$H$ \`e un valore molto grande su cui i due utenti si devono accordare.

\subsubsection{Attacco}
Un crittoanalista potrebbe procurarsi la firma di un utente su messaggi apparentemente privi di senso.

Prendiamo uno scenario in cui un crittoanalista $x$ si procura la firma dell'utente da messaggi privi di senso
per l'utente.

Supponiamo che
\begin{itemize}
	\item Il destinatario $v$ di un messaggio invii sempre una risposta $ack$ al mittente $u$ ogni volta che
	      riceve un messaggio (prima della verifica della firma).
	\item Il segnale di $ack$ sia il crittogramma della firma di $v$ su $m$.
\end{itemize}
In queste condizioni, un attacco attivo, pu\`o avere successo se
\begin{enumerate}
	\item $x$ intercetta il crittogramma $c$ firmato e inviato da $u$ a $v$, lo rimuove dal canale e lo rispedisce a
	      $v$, facendogli credere che $c$ sia stato inviato da lui.
	\item $v$ spedisce automaticamente a $x$ un ack.
	\item $x$ usa l'ack ricevuto per risalire al messaggio originale applicando le funzioni del cifrario con le
	      chiavi pubbliche di $u$ e $v$.
\end{enumerate}
Per risalire a $m$, il crittoanalista compie dei passaggi algebrici che avranno complessivamente costo polinomiale.

Prima di tutto, il fatto che $u$ abbia inviato il crittogramma $c$ a $v$, significa che
\begin{gather*}
	c = C(f, k_{v [\text{pub}]}) \\
	f = D(m, k_{u [\text{priv}]})
\end{gather*}
A questo punto, dopo che $x$ ha intercettato $c$ e l'ha rispedito a $v$, l'utente $v$ decifra $c$ ottenendo
\[ f = D(c, k_{v [\text{priv}]}) \]
e verifica la firma con la chiave pubblica di $x$ ottenendo un messaggio
\[ m' = C(f, k_{x [\text{pub}]}) \]
Il messaggio $m'$, molto probabilmente, non \`e significativo ma l'ack $c'$ \`e gi\`a stato inviato in automatico
calcolando
\begin{gather*}
	f' = D(m', k_{v [\text{priv}]}) \\
	c' = C(f', k_{x [\text{pub}]})
\end{gather*}
A questo punto $x$ ha tutto ci\`o che serve:
\begin{enumerate}
	\item Decifra $c'$ e trova $f'$
	      \[ f' = D(c', k_{x [\text{priv}]}) = D(C(f', k_{x [\text{pub}]}), k_{x [\text{priv}]}) \]
	\item Verifica la firma $f'$ e trova $m'$ usando la chiave pubblica di $v$
	      \[ m' = C(f', k_{v [\text{pub}]}) = C(D(m', k_{v [\text{priv}]}), k_{v [\text{pub}]}) \]
	\item Da $m'$ ricava la firma $f$
	      \[ f = D(m', k_{x [\text{priv}]}) = D(C(f, k_{x [\text{pub}]}), k_{x [\text{priv}]}) \]
	\item Verifica $f$ con la chiave pubblica di $u$ e trova $m$
	      \[ m = C(f, k_{u [\text{pub}]}) = C(D(m, k_{u [\text{priv}]}), k_{u [\text{pub}]}) \]
\end{enumerate}
Come possiamo vedere si usano sempre le funzioni di cifratura e decifrazione che, come sappiamo, hanno sempre costo
polinomiale e quindi anche l'attacco ha costo complessivamente polinomiale.

Per proteggersi da questo attacco non si devono inviare ack automatici, almeno finch\'e non si \`e concluso la fase
di verifica e si deve sempre firmare un'immagine hash del messaggio.

\subsection{Protocollo 3}
Questo protocollo si propone di risolvere anche i problemi presentati del secondo protocollo. In questo caso il
messaggio $m$ \`e cifrato e firmato con una funzione hash.

Il mittente $u$
\begin{enumerate}
	\item Calcola l'hash del messaggio $h(m)$ e genera la firma calcolando
	      \[ f = D(h(m), k_{u [\text{priv}]}) \]
	\item Cifra il messaggio calcolando
	      \[ c = C(m, k_{v [\text{pub}]}) \]
	\item Invia la tripla $\langle u, c, f \rangle$ a $v$.
\end{enumerate}
Il destinatario $v$
\begin{enumerate}
	\item Riceve la tripla $\langle u, c, f \rangle$.
	\item Decifra $c$ calcolando
	      \[ m = D(c, k_{v [\text{priv}]}) \]
	\item Calcola $h(m)$.
	\item Verifica la firma calcolando e verificando che
	      \[ h(m) = C(f, k_{u [\text{pub}]}) \]
\end{enumerate}
Come per tutti i protocolli a chiave pubblica, anche questo \`e vulnerabile ad attacchi di tipo
\emph{man in the middle}.

\subsection{Certification Authority}
Un algoritmo \`e tanto robusto quanto la sicurezza delle sue chiavi ma lo scambio o la generazione della chiave \`e
un passo cruciale.

\`E proprio in questo frangente che i crittoanalisti sfruttano attacchi di tipo \emph{man in the middle} per
riuscire a forzare facilmente i sistemi crittografici.

Sono dunque nate delle infrastrutture, chiamate \textbf{certification authority}, adibite a garantire la validit\`a
delle chiavi pubbliche e a regolare il loro uso, gestendone la distribuzione.

Le CA rilasciano un \textbf{certificato digitale} che autentica l'associazione tra un utente e la sua chiave pubblica.

Un certificato digitale consiste della chiave pubblica di una lista di informazioni relative al suo proprietario,
opportunamente firmate dalla CA. Per falsificare un certificato si deve falsificare la firma delle CA.

Una CA mantiene un archivio di chiavi pubbliche sicuro, accessibile a tutti e protetto da attacchi in scrittura non
autorizzati.

La chiave pubblica della CA \`e nota a tutti gli utenti che la mantengono protetta sui propri dispositivi e la
utilizzano per verificare la firma della CA stessa sui certificati.

Le CA hanno in genere una struttura gerarchica ad albero, dunque quando si vuole verificare la forma di una CA si avvia
una sorta di verifica a catena risalendo la gerarchia delle varie CA coinvolte.

\subsubsection{Certificazione}
Supponiamo che $u$ voglia comunicare con $v$
\begin{enumerate}
	\item $u$ richiede la chiave pubblica di $v$ alla CA.
	\item La CA invia a $u$ il certificato digitale $c_v$ di $v$.
	\item Dato che $u$ conosce la chiave pubblica della CA, controlla l'autenticit\`a del certificato verificandone il
	      periodo di validit\`a e la firma della CA.
	\item $u$ estrae dal certificato la chiave pubblica di $v$ e inizia il protocollo di comunicazione.
\end{enumerate}
Gli attacchi \emph{man in the middle} sono sempre possibili ma devono essere effettuati falsificando la certificazione
ma si assume che la CA sia fidata e il suo archivio di chiavi inattaccabile.

\subsection{Protocollo 4}
L'ultimo protocollo che vediamo consiste nel cifrare, firmare e certificare un messaggio $m$.

Il mittente $u$
\begin{enumerate}
	\item Si procura il certificato $\text{cert}_v$ di $v$ e verifica che sia autentico.
	\item Calcola $h(m)$ e genera la firma calcolando
	      \[ f = D(h(m), k_{u [\text{priv}]}) \]
	\item Cifra $m$ calcolando
	      \[ c = C(m, k_{v [\text{pub}]}) \]
	\item Invia la tripla $\langle \text{cert}_u, c, f \rangle$ a $v$.
\end{enumerate}
Il destinatario $v$
\begin{enumerate}
	\item Riceve la tripla $\langle \text{cert}_u, c, f \rangle$ a $v$.
	\item Verifica l'autenticit\`a di $\text{cert}_u$ con la chiave pubblica della CA che tiene in locale.
	\item Decifra $c$ con la sua chiave privata calcolando
	      \[ m = D(c, k_{v [\text{priv}]}) \]
	\item Verifica che la firma sia autentica controllando che
	      \[ h(m) = C(f, k_{u [\text{pub}]}) \]
\end{enumerate}
L'unico punto debole di questo metodo \`e rappresentato dall'uso di certificati revocati.

\section{SSL}
Il \textbf{protocollo SSL} \`e molto usato per costruire comunicazioni sicure, garantisce \textbf{confidenzialit\`a}
e \textbf{affidabilit\`a} ed \`e progettato per permettere a due computer che non conoscono le reciproche
funzionalit\`a di comunicare.

Supponiamo che un utente $u$ voglia accedere ad un servizio fornito da un sistema $s$. Il protocollo SSL garantisce
\begin{itemize}
	\item \textbf{Confidenzialit\`a}: la trasmissione \`e cifrata mediante un sistema ibrido in cui si usa un
	      cifrario asimmetrico per costruire le chiavi e uno simmetrico per la comunicazione.
	\item \textbf{Autenticazione}: il protocollo accerta l'identit\`a dei due utenti tramite un cifrario asimmetrico
	      o facendo uso di certificati digitali e inserendo un MAC nei messaggi.
\end{itemize}
L'SSL sta tra il protocollo di trasporto (TCP/IP) e il protocollo applicativo (HTTP) ed \`e completamente
indipendente da quest'ultimo.

\subsection{Livelli}
Il protocollo \`e organizzato su due livelli: \textbf{record} e \textbf{handshake}.

\subsubsection{SSL Record}
Il livello \emph{record} \`e a livello pi\`u basso ed \`e connesso direttamente al protocollo di trasporto.

Ha come obbiettivo l'incapsulamento dei dati spediti dai protocolli dei livelli superiori, assicurando
confidenzialit\`a e integrit\`a della comunicazione.

Implementa fisicamente il canale su cui viaggiano i messaggi.

\subsubsection{SSL Handshake}
Il livello \emph{handshake} \`e responsabile dell'instaurazione e del mantenimento dei parametri usati dal livello
\emph{record} e permette al sistema di
\begin{itemize}
	\item Autenticarsi
	\item Negoziare gli algoritmi di cifratura e firma
	\item Stabilire le chiavi per i singoli algoritmi crittografici e per il MAC
\end{itemize}
In definitiva il livello \emph{handshake} crea un canale \textbf{sicuro}, \textbf{affidabile} e \textbf{autenticato}
tra utente e sistema, entro il quale il livello \emph{record} fa viaggiare i messaggi.

Affinch\'e avvenga l'\emph{handshake} deve esserci uno scambio preliminare di messaggi indispensabile alla creazione
di un canale sicuro. Attraverso questi messaggi, client e server si identificano a vicenda e cooperano per la
costruzione delle chiavi simmetriche usate per le comunicazioni successive.

\subsection{Creazione del canale}
Vediamo ora tutti i passi che compie il protocollo per la costruzione del canale sicuro.

\begin{enumerate}
	\item \textbf{Client hello}: $u$ invia a $s$ un messaggio di \emph{client hello} con cui
	      \begin{itemize}
		      \item Richiede la creazione di una connessione SSL
		      \item Specifica le prestazioni di sicurezza che desidera siano garantite durante la comunicazione
		            (\textbf{cipher suite}).
		      \item Invia una sequenza di byte casuali.
	      \end{itemize}
	\item \textbf{Server hello}: $s$ riceve il messaggio di \emph{client hello} e manda un messaggio di
	      \emph{server hello} con cui
	      \begin{itemize}
		      \item Specifica una \emph{cipher suite} che anch'esso \`e in grado di supportare.
		      \item Invia una sequenza di byte casuali.
	      \end{itemize}
	      Se $u$ non riceve il \emph{server hello} interrompe la comunicazione.
	\item \textbf{Invio del messaggio}: $s$ si autentica con $u$ inviandogli il proprio certificato digitale e, se
	      i servizi offerti da $s$ devono essere protetti negli accessi, $s$ pu\`o richiedere a $u$ di autenticarsi
	      inviando il suo certificato digitale.
	\item \textbf{Server hello done}: $s$ invia il messaggio \emph{server hello done} con cui sancisce la fine degli
	      accordi sulla \emph{cipher suite} e sui parametri crittografici a essa associati.
	\item \textbf{Autenticazione del sistema}: per accertare l'autenticit\`a del certificato ricevuto da $s$, $u$ deve
	      controllare che
	      \begin{itemize}
		      \item Il certificato sia ancora valido.
		      \item La CA che ha firmato il certificato sia tra quelle \emph{fidate}.
		      \item La firma apposta sul certificato sia autentica.
	      \end{itemize}
	\item \textbf{User master secret}: $u$ a questo punto
	      \begin{itemize}
		      \item Costruisce un \textbf{pre-master secret} costituito da una nuova sequenza di byte casuali.
		      \item Lo cifra con il cifrario a chiave pubblica su cui si \`e accordato con $s$.
		      \item Invia il crittogramma a $s$.
	      \end{itemize}
	      Il pre-master secret viene poi usato per calcolare un \textbf{master secret}.
	\item \textbf{System master secret}: $s$ riceve il crittogramma contenente il \emph{pre-master secret} inviato
	      da $u$ e calcola il \emph{master secret} compiendo le stesse operazioni compiute da $u$.
	\item \textbf{Invio del certificato}: questo passo \`e opzionale e si fa solo nel caso in cui $s$ richieda un
	      certificato al client.
	\item \textbf{Finished}: \`E il primo messaggio protetto mediante il \emph{master secret} e la \emph{cipher suite}
	      su cui le due parti si sono accordate.

	      Il messaggio viene prima costruito da $u$ e spedito a $s$ e dopo avviene il contrario ma il messaggio \`e
	      diverso.
\end{enumerate}
Se tutto questo processo \`e andato a buon fine si possono costruire le chiavi crittografiche simmetriche in modo
sicuro.

\subsection{Sicurezza}
Nei passi di \emph{hello} i due utenti si inviano due sequenze casuali per la costruzione del \emph{master secret},
che risulta cos\`i, diverso in ogni sessione di SSL.

Un crittoanalista non pu\`o riutilizzare i messaggi di \emph{handshake} catturati sul canale per sostituirsi a $s$
in una successiva comunicazione con $u$ perch\'e questi sono riferiti a valori \emph{usa e getta}.

I blocchi vengono cifrati con un MAC (anch'esso cifrato) e quindi un crittoanalista non dovrebbe riuscire ad invertire
una funzione hash.

Un attacco volto a modificare la comunicazione tra i due utenti \`e difficile quanto un attacco volto alla decriptazione
dei messaggi.

Il sistema \`e autenticato con un certificato ed \`e dunque robusto ad attacchi \emph{man in the middle}.

La sicurezza del protocollo SSL \`e alta tanto quanto la pi\`u debole \emph{cipher suite} supportata.