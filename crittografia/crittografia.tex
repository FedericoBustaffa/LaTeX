\documentclass[10pt, a4paper]{report}

\pdfpagewidth\paperwidth
\pdfpageheight\paperheight

\usepackage[italian]{babel}
\usepackage[utf8]{inputenc}
\usepackage[T1]{fontenc}
\usepackage[hidelinks]{hyperref}
\usepackage{amsmath, amssymb, amsthm, mathtools}
\usepackage{enumitem}

\usepackage{../packages/codestyle}
\usepackage{../packages/mathenv}

\title{Crittografia}
\author{Federico Bustaffa}
\date{14/02/2022}
	
\begin{document}
\maketitle
\tableofcontents

\chapter{Introduzione}
L'obbiettivo del corso è quello di progettare sistemi robusti che non crollino al primo attacco o davanti al
primo utente ingenuo che ne fa uso.

Gran parte degli oggetti di uso quotidiano sono stati resi "intelligenti" dall'aggiunta di una componente
informatica al loro interno. Tali oggetti, anche se dall'esterno possono sembrare estremamente limitati, sono
sistemi completi e flessibili in grado di eseguire qualsiasi software.

L'aggiornamento online del software all'interno di tali sistemi è un mezzo per mantenere l'oggetto all'avanguardia
ma implica che esso possa essere manipolato da remoto per eseguire qualsiasi software e dunque per modificare
il comportamento dell'oggetto a proprio piacimento.

\section{Proprietà di sicurezza}
Un sistema informatico è, ad ogni livello di implementazione, formato da un insieme di moduli connessi, ognuno dei
quali offre un certo numero di operazioni.

Queste operazioni permettono di leggere e manipolare informazioni, che hanno poi un impatto sul mondo esterno.

In ogni sistema informatico ci sono regole (\textbf{politica di sicurezza}) che definiscono chi può invocare una
certa operazione e quindi ha il diritto di leggere o manipolare informazioni. Tali regole vengono implementate da
un sottoinsieme dei moduli del sistema informatico.

In questo contesto le tre principali proprietà che ci interessano sono:
\begin{itemize}
	\item \textbf{Confidenzialità}: solo chi ha il diritto di leggere una certa informazione può farlo.
	\item \textbf{Integrità}: solo chi ha il diritto di aggiornare una certa informazione può farlo.
	\item \textbf{Disponibilitò}: chi ha un diritto e vuole esercitarlo riesce a farlo in un tempo finito.
\end{itemize}
Da queste tre proprietà fondamentali possiamo derivarne altre secondarie:
\begin{itemize}
	\item \textbf{Tracciabilità}: individuare chi ha invocato un'operazione.
	\item \textbf{Accountability}: addebitare l'uso delle risorse.
	\item \textbf{Auditability}: verificare l'efficacia dei meccanismi di \emph{enforcement} di una politica (come
	      viene realizzata).
	\item \textbf{Forensics}: provare che certe azioni hanno avuto luogo.
	\item \textbf{Privacy/GDPR}: individuare chi, come e se un utente può usare informazioni personali.
\end{itemize}

\section{Politiche di sicurezza}
Una \textbf{politica di sicurezza} è un insieme di regole definite dal proprietario del sistema o del processo
aziendale per decidere gli utenti che possono invocare un'operazione e quando possono farlo.

Esistono diverse categorie di politiche descrivibili come il risultato di due scelte indipendenti:
\begin{itemize}
	\item La prima scelta è relativa al come la politica viene descritta:
	      \begin{itemize}
		      \item \textbf{Default allow}: operazioni vietate.
		      \item \textbf{Default deny}: operazioni permesse.
	      \end{itemize}
	\item La seconda scelta definisce vincoli sul proprietario del sistema:
	      \begin{itemize}
		      \item \textbf{Discretionary Access Control}: decide il proprietario.
		      \item \textbf{Mandatory Access Control}: esistono vincoli globali a tutto il sistema che nemmeno il
		            proprietario può violare.
	      \end{itemize}
\end{itemize}

\subsection{Soggetti e oggetti}
Relativamente alla seconda caratteristica che una politica deve avere, si cerca di modellare le risorse condivise
come \textbf{oggetti} e gli utenti come \textbf{soggetti}.

I soggetti invocano le operazioni definite dagli oggetti, se un oggetto invoca le operazioni definite da altri
oggetti allora diventa esso stesso un soggetto. In sintesi
\begin{itemize}
	\item \textbf{Soggetto}: in genere è un utente, un processo, un thread, un'istruzione.
	\item \textbf{Oggetto}: in genere si tratta di tipi di dati astratti, procedure, parametri, risorse logiche o
	      fisiche.
\end{itemize}

\subsection{Discretionary Access Control - DAC}
In questo modello per ogni oggetto esiste un \textbf{proprietario} (del sistema o del processo), il quale decide i
diritti dei vari soggetti mentre lui non ha vincoli di alcun tipo. Questo modello è tipico del mondo industriale.

\subsection{Mandatory Access Control - MAC}
Questo modello prevede la divisione di soggetti (utenti) e oggetti (risorse) in \textbf{classi}. Le classi sono
ordinate parzialmente in \textbf{livelli} (1, 2, 3 e così via).

Il livello di un soggetto esprime il grado di libertà che vogliamo lasciare a tale soggetto. Tanto più alto è il
livello di un soggetto tanto maggiore sarà il livello degli oggetti con cui esso può interagire.

Il livello di un oggetto esprime invece il grado di importanza di tale oggetto. Tanto più alto è il livello di un
oggetto tanto maggiore dev'essere è il livello del soggetto perché esso possa interagire con tale oggetto.

\subsubsection{Information Flow I}
In questo tipo di politica un utente può
\begin{itemize}
	\item Leggere tutti i file che hanno classe minore o uguale alla sua.
	\item Modificare i record dei file che hanno classe uguale alla sua.
	\item Appendere un record ad un file che ha classe maggiore della sua.
\end{itemize}
Per compiere queste operazioni è necessario il permesso dell'\emph{owner} che può solo restringere ulteriormente ciò
che un utente può fare. Si tratta di una politica MAC di tipo \textbf{no write down} e privilegia la confidenzialità.

\subsubsection{Information Flow II}
In questo tipo di politica un utente può
\begin{itemize}
	\item Scrivere tutti i file di una classe minore o uguale alla sua.
	\item Leggere tutti i file di una classe maggiore o uguale.
\end{itemize}
Questo implica che un utente poco affidabile, ossia di basso livello, non può andare a modificare dati critici. Si
tratta di una politica MAC di tipo \textbf{no write up} e privilegia l'integrità.

\subsubsection{Chinese Wall}
Gli oggetti del sistema sono partizionati in sottoinsiemi. L'utente che ha operato su un oggetto di un insieme non
può operare su oggetti di un altro insieme.

Questa politica \textbf{dinamica} permette di gestire conflitti di interesse ed è compatibile con la politica MAC
poiché aggiunge vincoli.

\subsubsection{Watermark}
Questa politica non prevede che un soggetto abbia un livello fissato ma che varia in base alle operazioni che esso
compie sui vari oggetti e dal livello di questi ultimi.

Il livello di un soggetto aumenta dopo che questo legge un oggetto di livello più alto del suo, rimane invece invariato
se il soggetto legge un oggetto di livello più basso.

\section{Matrice di controllo degli accessi}
Si tratta di una matrice con un comportamento molto dinamico che ha tante righe quanti sono i soggetti e tante colonne
quanti sono gli oggetti.

Nella posizione identificata dal soggetto $i$ e dall'oggetto $j$ si trovano i \textbf{diritti} che quel soggetto ha
su quell'oggetto. In generale è bene ai fini di sicurezza che la matrice contenga pochi diritti e che dunque appaia
sostanzialmente vuota.

Una rappresentazione concreta di tale matrice è necessaria ma non sufficiente per il rispetto della politica.

\subsection{Linux}
Nei sistemi operativi Linux vi è una rappresentazione concreta di tale matrice nel file system. Per ogni file, Linux
fornisce una sequenza di bit (bitmask) che indica i diritti che l'utente ha su tali file (lettura, scrittura,
esecuzione e così via).

\section{Trusted Computing Base - TCB}
Una caratteristica dei sistemi informatici è quella di avere al loro interno componenti informatici per implementare
e gestire la loro politica di sicurezza interna.

Se uno di questi componenti ha un errore o c'è un errore nei componenti che esso utilizza, allora l'implementazione
della politica non è corretta e quasi sicuramente un soggetto potrebbe invocare operazioni per le quali non ha i
diritti necessari.

\subsection{Dimensioni del TCB}
Per quanto riguarda le dimensioni del TCB possiamo dire che, tanto minore è il numero delle sue componenti, tanto
maggiore è la sicurezza del sistema.

Un TCB con dimensioni contenute permette anche una dimostrazione matematica relativamente semplice della sua
correttezza.

La dimensione del TCB è anche un criterio che permette di confrontare strategie alternative nella realizzazione della
politica di sicurezza.

\section{Vulnerabilità}
Quando si parla di \textbf{vulnerabilità} si vuole indicare un \emph{difetto}
\begin{itemize}
	\item Hardware
	\item Software
	\item Nell'utente
	\item Nelle regole della politica
\end{itemize}
che permette di violare la politica di sicurezza del sistema permettendo ad un soggetto di compiere un'operazione per
la quale non ha diritti.

L'obbiettivo della sicurezza informatica è quello di costruire sistemi che funzionano anche con delle vulnerabilità
e non quello di costruire sistemi senza vulnerabilità. In genere le varie vulnerabilità dei vari componenti vengono
\emph{compensate} in qualche modo da altri componenti.

La vulnerabilità più frequente, spesso non è nel codice, ma nel fatto che chi implementa il sistema dia per scontato
che l'utente non faccia errori nell'interfacciarsi con esso.

\subsection{Intrusioni}
Un'\textbf{intrusione} è una sequenza di \textbf{azioni} e \textbf{attacchi} per riuscire a controllare gli oggetti
del sistema.

Non tutte le azioni che l'attaccante fa sono \emph{illegali}, alcune di esse possono essere perfettamente lecite ma
sfruttate poi per violare il sistema.

In un'intrusione si sfrutta una o più vulnerabilità, usando anche programmi automatizzati (\textbf{exploit}) per ognuna
di esse, per riuscire a sostituirsi all'\emph{owner} del sistema e dunque avere la possibilità di
\begin{itemize}
	\item Raccogliere informazioni
	\item Modificare informazioni
	\item Impedire ad altri di accedere alle informazioni
\end{itemize}

\subsubsection{Fasi di un'intrusione}
Vediamo nello specifico cosa fa un hacker quando tenta un'intrusione:
\begin{enumerate}
	\item Raccoglie di informazioni iniziali sul sistema.
	\item Individuazione delle vulnerabilità del sistema per compiere un accesso iniziale.
	\item Ripetizione di una sequenza di operazioni finché non ha successo:
	      \begin{itemize}
		      \item Raccolta di informazioni sul sistema.
		      \item Scoperta di vulnerabilità.
		      \item Costruzione di exploit.
		      \item Attacco: si eseguono gli exploit ed eventuali azioni manuali.
	      \end{itemize}
	\item Installazione di strumenti per il controllo.
	\item Cancellazione delle tracce dell'attacco.
	\item Accesso, modifica ecc. ad un \emph{sottoinsieme} delle informazioni del sistema o si compiono altri
	      tipi attacchi:
	      \begin{itemize}
		      \item Furto di informazioni.
		      \item Cifratura di dati per chiedere riscatto.
	      \end{itemize}
\end{enumerate}
Un'intrusione può essere vista come un'\textbf{escalation} nell'acquisizione di diritti e nella raccolta di
informazioni tramite vari attacchi ripetuti.

Un attaccante in genere cerca di acquisire delle informazioni che gli permettano di avere nuovi diritti. Una volta
ottenuti i diritti è in grado di acquisire nuove informazioni e così via finché non raggiunge il proprio obbiettivo.

\subsection{Approcci alla sicurezza}
Le intrusioni sono dunque possibili grazie alle vulnerabilità e questo ci porta a definire due approcci alla sicurezza:
\begin{itemize}
	\item \textbf{Sicurezza incondizionale}: si assume che qualsiasi sia la vulnerabilità nel sistema, esista qualcuno
	      interessato ad usarla ed è quindi necessario eliminarle tutte.
	\item \textbf{Sicurezza condizionale}: in questo tipo di approccio si fa un'analisi in cui si cerca di capire quali
	      siano le reali minacce per il sistema e si eliminano solo le vulnerabilità che tali minacce potrebbero usare
	      per attaccare il sistema.
\end{itemize}

\subsubsection{Analisi del rischio}
Il primo approccio comporta costi molto elevati e spesso inaccettabili, inoltre richiede una quantità di lavoro enorme
e spesso inutile.

Con il secondo approccio invece si cerca di capire quali componenti del sistema si possono difendere e soprattutto
quali componenti \emph{conviene} difendere.

Per capirlo è necessaria un'\textbf{analisi del rischio} con la quale si cerca di individuare la tipologia di attacco
più probabile in relazione al sistema che stiamo cercando di proteggere. 			% INTRODUZIONE
\chapter{Rappresentazione matematica di oggetti}\label{rappresentazione}
In crittografia vogliamo rappresentare ogni oggetto, per farlo gli associamo un \textbf{nome}, ossia una stringa che lo
identifica in modo \textbf{univoco}. La stringa in questione \`e una sequenza di caratteri \textbf{ordinata}.

Viene fissato inizialmente un \textbf{alfabeto}, ossia un insieme di caratteri o simboli \textbf{finito} e vogliamo
rappresentare un oggetto tramite una sequenza di caratteri di questo alfabeto.

Possiamo osservare che, aumentando la lunghezza dei nomi che assegno ad ogni oggetto, il numero di oggetti che posso
rappresentare \`e, di fatto, infinito seppur l'alfabeto abbia un numero finito di caratteri.

\section{Alfabeti e sequenze}\label{alfabeti}
Introduciamo ora un po' di notazione per iniziare a rappresentare meglio gli alfabeti e le sequenze.

Posso definire una alfabeto $\Gamma$ di cardinalit\`a $s$ e, nel caso in cui dovessi rappresentare $N$ oggetti, vale che:
\begin{itemize}
	\item La lunghezza della sequenza pi\`u lunga che rappresenta un oggetto dell'insieme vale
	      \[ d(s, N) \]
	\item Il valore minimo di $d(s, N)$ tra tutte le rappresentazioni possibili vale
	      \[ d_{\min}(s, N) \]
	      Questo valore indica la lunghezza ideale per rappresentare gli $N$ oggetti, con un alfabeto di $s$ caratteri.
	\item Un metodo di rappresentazione \`e tanto migliore, quanto pi\`u $d(s, N)$ si avvicina a $d_{\min}(s, N)$. Quindi
	      un metodo di rappresentazione \`e migliore se uso stringhe "corte" per rappresentare gli oggetti.
\end{itemize}

\section{Rappresentazione unaria}\label{rappresentazione_unaria}
Prendiamo come primo esempio un tipo di rappresentazione \emph{svaforevole}, che in pratica non viene mai utilizzato, che
\`e quello della \textbf{rappresentazione unaria}. In questo caso abbiamo
\[ s = 1, \quad \Gamma = \{ 0 \} \]
L'unico modo per differenziare un nome dall'altro \`e aggiungere uno $0$ alla sequenza. Dunque, se devo rappresentare $N$
oggetti ho che
\[ d_{\min}(1, N) = N \]
e la sequenza pi\`u lunga avr\`a anch'essa lunghezza $N$:
\[ d(s, N) = N \]

\section{Rappresentazione binaria}\label{rappresentazione_binaria}
Vediamo ora come l'aggiunta di un solo carattere ci faccia ottenere un metodo di rappresentazione migliore. In questo caso
abbiamo
\[ s = 2, \quad \Gamma = \{ 0, 1 \} \]
Questo alfabeto ha due caratteristiche fondamentali:
\begin{itemize}
	\item Per ogni sequenza lunga $k$ con $k \geq 1$, abbiamo $2^k$ sequenze.
	\item Il numero totale di sequenze lunghe da 1 a $k$ \`e dato da
	      \[ \sum_{i = 1}^k 2^i = 2^{k + 1} - 2 \]
\end{itemize}
Nel caso in cui avessi $N$ oggetti da rappresentare avrei che
\begin{itemize}
	\item $2^{k + 1} - 2 \geq N$
	\item $k \geq \log_2 (N + 2) - 1$
\end{itemize}
Posso quindi affermare che $d_{\min}(2, N)$ \`e il minimo intero $k$ che soddisfa la relazione:
\[ d_{\min}(2, N) = \lceil \log_2 (N + 2) - 1 \rceil \]
che implica
\[ \lceil \log_2 N \rceil - 1 \leq d_{\min}(2, N) \leq \lceil \log_2 N \rceil \]

Il numero $\lceil \log_2 N \rceil$ sono i caratteri binari sufficienti per costruire $N$ sequenze differenti. In
sostanza posso costruire $N$ sequenze differenti tutte di $\lceil \log_2 N \rceil$ caratteri.

Per comodit\`a, \`e preferibile costruire $N$ sequenze, \emph{tutte} di $\lceil \log_2 N \rceil$ caratteri e non $N$
sequenze \emph{al massimo} di $\lceil \log_2 N \rceil$ caratteri.

\begin{example}
	Se volessi rappresentare sette oggetti con un alfabeto binario avrei che
	\[ N = 7, \quad \lceil \log_2 7 \rceil = 3 \]
	e quindi potrei rappresentare i sette oggetti come segue
	\[
		0 \quad
		1 \quad
		00 \quad
		01 \quad
		10 \quad
		11 \quad
		000
	\]
	\`E tuttavia pi\`u comodo rappresentare gli oggetti tramite sequenze tutte lunghe uguali come di seguito
	\[
		000 \quad
		001 \quad
		010 \quad
		100 \quad
		011 \quad
		101 \quad
		110 \quad
	\]
\end{example}

\section{Rappresentazione generale}\label{rappresentazione_generale}
In generale, se dispongo di un alfabeto $\Gamma$ di cardinalit\`a $s$, posso costruire $N$ sequenze differenti \emph{tutte}
lunghe $\lceil \log_s N \rceil$ caratteri.

Il vantaggio di avere sequenze tutte di uguale lunghezza \`e che posso concatenarle senza dover aggiungere un simbolo di
separazione tra l'una e l'altra.

Per ottenere \textbf{rappresentazioni efficienti} si deve quindi usare un numero massimo di caratteri di
\textbf{ordine logaritmico} nella cardinalit\`a $N$ dell'insieme da rappresentare e $| \Gamma | \geq 2$.

\subsection{Rappresentazione di interi}
La notazione posizionale per rappresentare numeri interi \`e una rappresentazione efficiente, indipedentemente dalla base
$s \geq 2$ scelta.

Un numero $N$ \`e rappresentato con un numero $d$ di cifre tale che
\[ \lceil \log_2 N \rceil \leq d \leq \lceil \log_2 N \rceil + 1 \]
C'\`e quindi una \textbf{riduzione logaritmica} tra il valore $N$ di un numero e la lunghezza $d$ della sua
rappresentazione. 	% RAPPRESENTAZIONE OGGETTI
\chapter{Teoria della calcolabilit\`a}\label{calcolabilita}
La \textbf{calcolabilit\`a} si occupa delle questioni fondamentali circa la potenza e le limitazioni dei sistemi di
calcolo e cerca di definire le nozioni di \textbf{algoritmo} e di \textbf{problema non decidibile}.

In altre parole si occupa di classificare i problemi in \emph{risolvibili} e \emph{non risolvibili} per via algoritmica.

A differenza invece della \textbf{complessit\`a} che si occupa di definire la nozione di \text{algoritmo efficiente}
e di \textbf{problema intrattabile}.  In altre parole divide i problemi in \emph{facili} e \emph{difficili}.

\section{Insiemi numerabili}\label{insiemi_numerabili}
\begin{itemize}
	\item Due insiemi $A$ e $B$ hanno lo stesso numero di elementi se e solo se si pu\`o stabilire una
	      \textbf{corrispondenza biunivoca} tra i loro elementi.
	\item Un insieme \`e \textbf{numerabile} se e solo se i suoi elementi possono essere messi in
	      \textbf{corrispondenza biunivoca con i numeri naturali}.
\end{itemize}

\subsection{Enumerazione delle sequenze}
Se si volesse elencare, in un ordine ragionevole, le sequenze di lunghezza finita costruite su un alfabeto finito ci
scontreremmo con un problema: le sequenze non sono in numero finito, quindi non si potr\`a completare l'elenco.

Lo scopo in questo caso \`e raggiungere qualsiasi sequenza $\sigma$, arbitrariamente scelta, in un numero finito di passi.
Per fare ci\`o, $\sigma$ deve trovarsi a distanza \emph{finita} dall'inizio dell'elenco.

\subsubsection{Ordinamento canonico}
Per riuscire ad enumerare queste sequenze dobbiamo ricorrere al cosiddetto \textbf{ordinamento canonico}.
\begin{enumerate}
	\item Si ordinano le sequenze in ordine lunghezza crescente.
	\item A parit\`a di lunghezza si ordinano le sequenze secondo l'ordinamento tra i caratteri dell'alfabeto.
\end{enumerate}

A questo punto una sequenza $s$ arbitraria si trover\`a tra quelle di $| s |$ caratteri, in una posizione corrispondente
all'ordine alfabetico relativo all'alfabeto $\Gamma$ che sto utilizzando.

\begin{example}
	Se prendessi un alfabeto composto solo da lettere
	\[ \Gamma = \{ a, b, c, \dots, z \} \]
	e volessi scrivere le sequenze nell'ordine canonico otterrei
	\begin{gather*}
		a, \quad b, \quad c, \quad \dots, \quad z, \quad aa, \quad ab, \quad \dots, \quad az, \\
		ba, \quad \dots, \quad bz, \quad za, \quad \dots, \quad zz, \quad aaa, \quad \dots, \quad zzz, \quad \dots
	\end{gather*}
	come possiamo vedere, si pu\`o costruire un numero infinito di sequenze ma tutte di lunghezza finita e, soprattutto,
	che si possono mettere in corrispondenza biunivoca con l'insieme dei numeri naturali.
\end{example}

Osserviamo che l'enumerazione delle sequenze \`e possibile perch\'e esse sono di lunghezza finita anche se illimitata.
Ovvero, scelto un intero $d$, esistono sempre sequenze di lunghezza maggiore di $d$.

Se le sequenze fossero di lunghezza infinita l'insieme non sarebbe numerabile.

\subsection{Problemi computazionali}
\begin{theorem}
	L'insieme dei \textbf{problemi computazionali} non \`e numerabile.
\end{theorem}

Un problema computazionale pu\`o essere visto come una \emph{funzione matematica} che associa ad ogni insieme di dati,
espressi da $k$ numeri interi, il corrispondente risultato, espresso da $j$ numeri interi.
\[ f : N^k \rightarrow N^j \]
L'insieme delle funzioni $f : N^k \rightarrow N^j$ non \`e numerabile.

\subsection{Diagonalizzazione}
Per dimostrare quanto detto in precedenza, utilizziamo la \textbf{diagonalizzazione}.
\begin{theorem}
	L'insieme di funzioni
	\[ F = \{ f \mid f : N \rightarrow \{0, 1\} \} \]
	nel quale ogni $f \in F$ pu\`o essere rappresentata da una sequenza infinita del tipo
	\[
		\begin{matrix}
			x    &  & 0 & 1 & 2 & \dots & n & \dots \\
			f(x) &  & 0 & 1 & 0 & \dots & 0 & \dots
		\end{matrix}
	\]
	o, se possibile, da una regola finita di costruzione
	\[
		f(x) = \begin{cases}
			0 & x \text{ pari}    \\
			1 & x \text{ dispari}
		\end{cases}
	\]
	\textbf{non \`e numerabile}.
	\begin{proof}
		Vogliamo dimostrare il teorema per assurdo, ammettiamo quindi che $F$ sia numerabile.

		Dato che $F$ \`e enumerabile allora posso assegnare ad ogni funzione $f \in F$ un numero progressivo nella
		numerazione, e costruire una tabella (infinita) di tutte le funzioni.
		\begin{center}
			\begin{tabular}{c | c c c c c}
				$x$      & 0     & 1     & 2     & 3     & \dots \\
				\hline
				$f_0(x)$ & 1     & 0     & 1     & 0     & \dots \\
				$f_1(x)$ & 0     & 0     & 1     & 1     & \dots \\
				$f_2(x)$ & 1     & 1     & 0     & 1     & \dots \\
				$f_3(x)$ & 0     & 1     & 1     & 0     & \dots \\
				\dots    & \dots & \dots & \dots & \dots & \dots
			\end{tabular}
		\end{center}
		Consideriamo adesso la funzione $g \in F$
		\[
			g(x) = \begin{cases}
				0 & f_x (x) = 1 \\
				1 & f_x (x) = 0
			\end{cases}
		\]
		La funzione $g$, cos\`i definita, \textbf{non} corrisponde a nessuna delle $f_i$ in tabella. Questo perch\'e
		differisce sicuramente nei valori posti sulla diagonale principale.
		\begin{center}
			\begin{tabular}{c | c c c c c}
				$x$      & 0     & 1     & 2     & 3     & \dots \\
				\hline
				$f_0(x)$ & 1     & 0     & 1     & 0     & \dots \\
				$f_1(x)$ & 0     & 0     & 1     & 1     & \dots \\
				$f_2(x)$ & 1     & 1     & 0     & 1     & \dots \\
				$f_3(x)$ & 0     & 1     & 1     & 0     & \dots \\
				\dots    & \dots & \dots & \dots & \dots & \dots \\
				\hline
				$g(x)$   & 0     & 1     & 1     & 1     & \dots
			\end{tabular}
		\end{center}
		Ecco che si giunge ad una contraddizione.
	\end{proof}
\end{theorem}

\subsection{Il problema della rappresentazione}
L'informatica rappresenta tutte le sue entit\`a (quindi anche gli algoritmi) in forma digitale, come
\textbf{sequenze finite di simboli di alfabeti finiti}.

Lo stesso vale per gli \textbf{algoritmi}, i quali sono composti da una sequenza finita di operazioni, completamente
e univocamente determinate. Gli algoritmi sono potenzialmente infiniti ma comunque numerabili.

Come gi\`a detto, i problemi computazionali non sono numerabili e dunque abbiamo molti pi\`u problemi che algoritmi.
Questo implica necessariamente che esistano problemi \textbf{privi} di un algoritmo di calcolo per la loro risoluzione.

\section{Il problema dell'arresto}\label{arresto}
Formulato da Turing, consiste in un algoritmo che indaga le proprit\`a di un altro algoritmo, trattato come dato in
input.

Nello specifico il problema \`e cos\`i formulato:
\begin{center}
	Presi ad arbitrio un algoritmo $A$ e i suoi dati di input $D$, decidere in \textbf{tempo finito} se la computazione
	di $A$ su $D$ termina o no.
\end{center}

Sebbene il problema sia lecito, dato che un algoritmo e i relativi dati in ingresso sono codificati con lo stesso
alfabeto, Turing stesso ha dimostrato che \`e \textbf{impossibile} riuscire a stabilire (in tempo finito) se un
programma arbitrario si arresta e termina la sua esecuzione.

\begin{theorem}
	Il problema dell'arresto \`e \textbf{indecidibile}.
	\begin{proof}
		Per dimostrarlo proviamo a dobbiamo considerare un generico algoritmo \verb|A| con un generico input \verb|D|.
		L'algoritmo \verb|ARRESTO| prende in input \verb|A| e \verb|D| e ritorna \verb|true| se \verb|A(D)| termina
		oppure \verb|false| se non termina.

		Se \verb|A(D)| termina \verb|ARRESTO| ritorna \verb|true| e non ci sono problemi. Se invece non termina
		\verb|ARRESTO| non \`e in grado di rispondere \verb|false| in tempo finito.

		Questo perch\'e \verb|ARRESTO| non pu\`o non passare dalla simulazione di \verb|A| su \verb|D| e quindi, nel caso
		\verb|A| non termini, \verb|ARRESTO| non terminerebbe a sua volta.

		Se esistesse l'algoritmo \verb|ARRESTO|, esisterebbe anche il seguente algoritmo
		\begin{lstlisting}[style=pseudo-style]
PARADOSSO(A)
	while (ARRESTO(A, A));
		\end{lstlisting}
		\begin{center}
			\verb|PARADOSSO| termina
			\[ \Leftrightarrow \]
			\verb|ARRESTO(A, A) = 0|
			\[ \Leftrightarrow \]
			\verb|ARRESTO| non termina.
		\end{center}

		Ma cosa succede se provassimo a calcolare \verb|PARADOSSO(PARADOSSO)| ?
		\begin{center}
			\verb|PARADOSSO(PARADOSSO)| termina
			\[ \Leftrightarrow \]
			\verb|ARRESTO(PARADOSSO, PARADOSSO) = 0|
			\[ \Leftrightarrow \]
			\verb|PARADOSSO(PARADOSSO)| non termina
		\end{center}
		Ma ecco che si giunge ad una contraddizione.
	\end{proof}
\end{theorem}
 	% CALCOLABILITA'
\chapter{Teoria della complessit\`a}\label{complessita}
La \textbf{complessit\`a} \`e quella branca dell'informatica teorica che classifica i problemi in base alla
difficolt\`a che si ha nel risolverli, ossia al numero di operazioni necessarie per risolverli.

Nel capitolo \ref{calcolabilita} abbiamo trattato i problemi decidibili e non. In questo capitolo restringiamo il
campo solo ai problemi decidibili ma li dividiamo in due sottocategorie: \textbf{trattabili} e \textbf{intrattabili}.

Si tratta in entrambi i casi di problemi risolvibili, ma nel caso degli intrattabili, il costo computazionale \`e
talmente alto (generalmente esponenziale o pi\`u) da renderli molto simili ai problemi indecidibili, dato che
la risoluzione richiederebbe decine se non centinaia o migliaia (in base alla dimensione dell'input) di anni di
calcolo.

Al contrario, i problemi trattabili, hanno algoritmi di costo polinomiale e dunque sono risolvibili in tempi brevi.

Ci sono infine problemi il cui stato non \`e noto: abbiamo solo algoritmi di costo esponenziale per la loro
risoluzione ma non abbiamo dimostrazione del fatto che siano intrattabili.

\section{Algoritmi polinomiali ed esponenziali}\label{alg_poly_exp}
Studiamo ora la dimensione dei dati trattabili in funzione dell'incremento della velocit\`a dei calcolatori.

\begin{example}
	Siano $C_1$ e $C_2$ due calcolatori, con $C_2$ che \`e $k$ volte pi\`u veloce di $C_1$ e testiamo quanti dati
	riescono a processare i due calcolatori in un tempo $t$.

	Il calcolatore $C_1$ processa $n_1$ dati nel tempo $t$ mentre $C_2$ processa $n_2$ dati sempre in tempo $t$.
	Possiamo osservare che, usare $C_2$, \`e come usare $C_1$ per un tempo $kt$.

	Se l'algoritmo che stiamo usando \`e \textbf{polinomiale} e risolve il problema in un tempo $cn^s$, dove
	$c$ e $s$ sono costanti, abbiamo che
	\[	\begin{array}{llcl}
			C_1 : & c n_1^s = t  & \rightarrow &
			n_1 = \left( \displaystyle\frac{t}{c} \right)^{1/s} \\

			C_2 : & c n_2^s = kt & \rightarrow &
			n_2 = k^{1/s}\left( \displaystyle\frac{t}{c} \right)^{1/s}
		\end{array}
	\]
	Otteniamo quindi
	\[ n_2 = k^{1/s} \cdot n_1 \]
	abbiamo dunque un miglioramento di un \textbf{fattore moltiplicativo} $k^{1/s}$ tanto pi\`u grande quanto pi\`u
	piccolo \`e il grado ($s$) del polinomio.

	Se invece facciamo girare sugli stessi calcolatori un algoritmo di costo esponenziale, che risolve il problema in
	un tempo $c \cdot 2^n$ con $c$ costante, otteniamo
	\[
		\begin{array}{llcl}
			C_1 : & c \cdot 2^{n_1} = t   & \rightarrow & 2^{n_1} = \displaystyle\frac{t}{c}         \\
			C_2 : & c \cdot 2^{n_2} = k t & \rightarrow & 2^{n_2} = k \cdot \displaystyle\frac{t}{c}
		\end{array}
	\]
	Otteniamo quindi
	\[ 2^{n_2} = k \cdot 2^{n_1} \]
	da cui ricaviamo
	\[ n_2 = n_1 + \log k \]
	In questo caso il miglioramento \`e solo di un \textbf{fattore additivo} $\log k$. Questo significa che per
	quanto si incrementi la velocit\`a di elaborazione $k$ della macchina $C_2$ rispetto a $C_1$, i dati elaborati in
	pi\`u sarebbero solo $\log k$.
\end{example}

\section{Problemi}\label{problemi}
Trattiamo ora le varie \textbf{classi di complessit\`a} e come si definiscono. Indichiamo con $\Pi$, un generico
\textbf{problema}, con $I$, l'insieme delle \textbf{istanze} in ingresso e con $S$ l'insieme delle \textbf{soluzioni}.

\subsection{Tipologie di problemi}
Abbiamo tre \textbf{classi} principali di problemi: di \textbf{ricerca}, di \textbf{ottimizzazione} e
\textbf{decisionali}.

\subsubsection{Problemi di ricerca}
Sono problemi in cui, data un'istanza $x$ in input, si richiede di restituire una soluzione $s$, presa da un insieme
$S$ di possibili soluzioni.

Un esempio di problema di ricerca \`e quello in cui si deve calcolare un cammino che unisce due nodi di un grafo. In
questo caso $S$ sar\`a uguale a tutti i possibili cammini che vanno dal nodo di partenza a quello di arrivo.

Le \textbf{istanze positive} saranno le coppie di nodi per le quali esiste almeno un cammino che li unisce, viceversa
le \textbf{istanze negative} saranno le coppie di nodi per le quali non esiste nemmeno un cammino che li unisce.

\subsubsection{Problemi di ottimizzazione}
Nei problemi di ottimizzazione, data un'istanza $x$ in input, si vuole trovare la \textbf{migliore soluzione} $s$ tra
tutte le soluzioni possibili.

In questo caso $S$ sar\`a l'insieme di tutte le soluzioni ottime al problema (diverse fra loro ma con stesso valore).

\subsubsection{Problemi decisionali}
I problemi decisionali richiedono una risposta binaria, in genere \verb|true| o \verb|false|, ossia l'insieme delle
soluzioni \`e
\[ S = \{ 0, 1 \} \]
le \textbf{istanze positive} sono
\[ x \in I \mid \Pi(x) = 1 \]
mentre le \textbf{istanze negative} sono
\[ x \in I \mid \Pi(x) = 0 \]
Tipicamente sono problemi che indagano delle propriet\`a.

In complessit\`a si prende in considerazione solo quest'ultima classe di problemi per due motivi principalmente:
\begin{itemize}
	\item Tutto il tempo di calcolo \`e speso per trovare la soluzione e non per scriverla (la risposta \`e solo 0 o 1).
	\item La complessit\`a di un problema, riformulato in forma decisionale, non cambia rispetto alla sua forma non
	      decisionale.
\end{itemize}
In realt\`a tanti problemi interessanti sono problemi di ottimizzazione e per trattarli dobbiamo riformularli in
forma decisionale. Per farlo, basta verificare l'esistenza di una soluzione (al problema di ottimizzazione) che
soddisfa una certa propriet\`a.

Questo ci dice che il problema di ottimizzazione \`e almeno tanto difficile quanto il corrispondente problema
decisionale.

Caratterizzare il problema decisionale ci da un \textbf{limite inferiore} alla complessit\`a del relativo problema di
ottimizzazione.

\section{Classi di complessit\`a}\label{classi}
Un problema, in base alle risorse spese per la sua risoluzione, viene classificato in diverse
\textbf{classi di complessit\`a}.

\begin{theorem}
	Dato un problema decisionale $\Pi$ ed un algoritmo $A$, diciamo che $A$ \textbf{risolve} $\Pi$ se, data un'istanza di
	input $x$, vale
	\[ A(x) = 1 \quad \Leftrightarrow \quad \Pi(x) = 1 \]
\end{theorem}

\begin{theorem}
	Dato un problema decisionale $\Pi$, un algoritmo $A$ e la dimensione $n$ dell'input, diciamo che $A$ risolve $\Pi$
	in tempo $t(n)$ e spazio $s(n)$ se il tempo di esecuzione e l'occupazione di memoria di $A$ sono rispettivamente
	$t(n)$ e $s(n)$.
\end{theorem}

Data una funzione $f$ qualsiasi e la dimensione $n$ dell'input, possiamo definire due classi di complessit\`a:
\begin{itemize}
	\item \textbf{Time}: l'insieme dei problemi decisionali che possono essere risolti in tempo $O(f(n))$.
	\item \textbf{Space}: l'insieme dei problemi decisionali che possono essere risolti in spazio $O(f(n))$.
\end{itemize}
Da queste due classi di problemi possiamo derivare altre sottoclassi pi\`u specifiche
\begin{itemize}
	\item \textbf{P}: l'insieme dei problemi risolvibili in tempo polinomiale nella dimensione $n$ dell'istanza di input.

	      In questo tipo di problemi abbiamo due costanti: $c$ e $n_0 > 0$ tali che il numero di passi elementari \`e al
	      pi\`u $n^c$ per ogni input di dimensione $n$ e per ogni $n > n_0$.

	\item \textbf{P-Space}: l'insieme dei problemi risolvibili in spazio polinomiale nella dimensione $n$ dell'istanza
	      di input.

	      In questo tipo di problemi abbiamo due costanti: $c$ e $n_0 > 0$ tali che il numero di celle di memoria
	      utilizzate \`e al pi\`u $n^c$ per ogni input di dimensione $n$ e per ogni $n > n_0$.
	\item \textbf{Exp Time}: l'insieme dei problemi risolvibili in tempo esponenziale nella dimensione $n$ dell'istanza
	      di ingresso.
\end{itemize}

\subsection{Relazioni tra classi}
Si congettura che
\[ \text{P} \subseteq \text{P-Space} \]
questo perch\'e un algoritmo di costo polinomiale pu\`o avere accesso al pi\`u ad un numero polinomiale di locazioni di
memoria diverse (in ordine di grandezza).

Si congettura anche che
\[ \text{P-Space} \subseteq \text{Exp Time} \]
questo perch\'e non \`e detto che un algoritmo di costo esponenziale richieda un numero esponenziale di celle di memoria.

\subsection{Classe NP e certificati}
Introduciamo ora una nuova classe di problemi, la classe \textbf{NP}, dove NP sta per \textbf{P}olinomiale su modelli
\textbf{N}on deterministici.

Per le \textbf{istanze accettabili} (\textbf{positive}) $x$ di alcuni problemi, \`e possibile fornire un
\textbf{certificato polinomiale} $y$ che possa convincerci del fatto che l'istanza soddisfa la propriet\`a e dunque
\`e un'istanza \emph{accettabile}.

I certificati servono per dare una risposta in tempi pi\`u brevi a certi problemi decisionali. Un certificato \`e di
fatto una descrizione breve di una soluzione che possiede la propriet\`a cercata.

\begin{example}
	Prendiamo il problema della Clique per esempio. Una clique, dato un grafo non orientato, \`e il sottografo
	connesso pi\`u grande, rispetto al grafo che sto considerando.

	Esiste solo un algoritmo di costo esponenziale per trovare una clique in un grafo ma non sappiamo se possa essere
	risolto in tempo polinomiale.

	Vogliamo verificare che all'interno di questo grafo ci sia una clique di $k$ vertici.

	Se abbiamo un certificato polinomiale contenente un'insieme di $k$ vertici che formano una clique, non ci rimane
	che verificare se quei vertici formano effettivamente una clique (tempo polinomiale).
\end{example}

Un certificato \`e un \emph{attestato breve di esistenza} di una soluzione con determinate propriet\`a e si definisce solo
per istanze accettabili. Infatti, in generale, non \`e facile costruire attestati di non esistenza di una certa soluzione.

\subsubsection{Verifica}
L'\textbf{idea} \`e quella di utilizzare il costo della \textbf{verifica} del certificato per caratterizzare la
complessit\`a del problema.

Un problema $\Pi$ \`e \emph{verificabile in tempo polinomiale} se
\begin{itemize}
	\item Ogni istanza $x$ accettabile di $\Pi$ di dimensione $n$, ammette un certificato $y$ di dimensione
	      polinomiale in $n$.
	\item Esiste un algoritmo di verifica polinomiale in $n$, applicabile a ogni coppia $\langle x, y \rangle$, che
	      permette di attestare che $x$ \`e accettabile.
\end{itemize}
Un altro modo di definire i problemi della classe NP \`e quello di problemi \emph{verificabili in tempo polinomiale}.

\begin{observation}
	Un certificato viene trovato in tempo esponenziale e ci serve \textbf{solo} a caratterizzare un certo problema in
	base al costo della sua verifica.
\end{observation}

\begin{theorem}
	La classe P \`e un sottoinsieme di NP. Un problema risolvibile in tempo polinomiale \`e sicuramente verificabile in
	tempo polinomiale.
\end{theorem}

\subsubsection{Problemi NP-completi}
I problemi \textbf{NP-completi} sono i problemi \emph{pi\`u difficili} all'interno della classe NP. Se esistesse un
algoritmo polinomiale per risolvere uno solo di questi problemi, allora tutti i problemi in NP potrebbero essere risolti
in tempo polinomiale e dunque giungeremmo alla conclusione che $\text{P} = \text{NP}$.

\begin{theorem}\label{th: NP-completi_poly}
	Tutti i problemi NP-completi sono risolvibili in tempo polinomiale oppure nessuno lo \`e.
\end{theorem}

\subsection{Riduzioni polinomiali}
Introduciamo ora le \textbf{riduzioni polinomiali}, necessarie per comprendere meglio il teorema
\ref{th: NP-completi_poly}.

\begin{theorem}
	Siano $\Pi_1$ e $\Pi_2$ due problemi decisionali e siano $I_1$ e $I_2$ gli insiemi delle istanze di input di
	$\Pi_1$ e $\Pi_2$, allora $\Pi_1$ si \textbf{riduce} in tempo polinomiale a $\Pi_2$
	\[ \Pi_1 \leq_p \Pi_2 \]
	se esiste una funzione $f : I_1 \rightarrow I_2$, calcolabile in tempo polinomiale, tale che, per ogni istanza $x$
	di $\Pi_1$, $x$ \`e un'istanza accettabile di $\Pi_1$ se e solo se $f(x)$ \`e un'istanza accettabile di $\Pi_2$.
\end{theorem}

In sintesi, se $\Pi_1$ si riduce a $\Pi_2$, significa che in tempo polinomiale posso
\begin{enumerate}
	\item \emph{Tradurre} $\Pi_1$ in $\Pi_2$.
	\item Risolvere $\Pi_2$.
	\item La soluzione trovata \`e valida anche per $\Pi_1$.
\end{enumerate}

Supponiamo di avere un algoritmo di costo polinomiale per risolvere $\Pi_2$ e vogliamo usare lo stesso algoritmo per
risolvere $\Pi_1$.
\begin{enumerate}
	\item Prendiamo un'istanza $x$ di $\Pi_1$ e traduciamola in un'istanza accettabile per $\Pi_2$ tramite la funzione
	      $f$.
	\item A questo punto facciamo girare l'algoritmo per la risoluzione di $\Pi_2$ sull'istanza $f(x)$.
	\item La soluzione data dall'algoritmo vale anche per $\Pi_1$.
\end{enumerate}

\subsubsection{Problemi NP}

\begin{definition}
	Un problema decisionale $\Pi$ si dice \textbf{NP-arduo} se
	\[	\forall \; \Pi' \in \text{NP}, \quad \Pi' \leq_p \Pi \]
	Ovvero se ogni problema in NP \`e riducibile a $\Pi$.
\end{definition}

\begin{definition}
	Un problema decisionale $\Pi$ si dice \textbf{NP-completo} se
	\[ \Pi \in \text{NP} \quad \wedge \quad \forall \; \Pi' \in \text{NP}, \quad \Pi' \leq_p \Pi \]
	Ovvero se ogni problema in NP \`e riducibile a $\Pi$, anch'esso in NP.
\end{definition}

Dimostrare che un problema \`e in NP pu\`o essere facile: basta esibire un certificato polinomiale. Non \`e altrettanto
facile dimostrare che un problema \`e NP-arduo o NP-completo: bisognerebbe dimostrare che tutti i problemi in NP si
riducono polinomialmente a $\Pi$.

In realt\`a la prima dimostrazione di NP-completezza aggira il problema.

\begin{theorem}[Cook]
	Il problema SAT \`e NP-completo.
\end{theorem}

Cook ha mostrato che, dato un qualunque problema $\Pi \in \text{P}$ ed una qualunque istanza $x$ per $\Pi$, si pu\`o
costruire un'espressione booleana in forma normale congiuntiva, che descrive il calcolo di un algoritmo per risolvere
$\Pi$ su $x$. L'espressione \`e vera se e solo se l'algoritmo restituisce 1.

\begin{theorem}
	Un problema decisionale $\Pi$ \`e NP-completo se
	\[ \Pi \in \text{NP} \quad \wedge \quad \text{SAT} \leq_p \Pi \]
	o se $\Pi$ \`e riducibile ad qualsiasi altro problema NP-completo.
\end{theorem}

\subsubsection{Problemi NP equivalenti}
Il fatto che un problema $\Pi$ si possa ridurre a SAT lo rende NP-completo. Il fatto che $\Pi$ sia NP-completo rende
possibile ridurre SAT a $\Pi$. Questo rende SAT e $\Pi$ due problemi \textbf{NP-equivalenti}.

\begin{theorem}
	Tutti i problemi NP-completi sono tra loro NP-equivalenti.
\end{theorem}

\subsubsection{Classi co-P e co-NP}
Le ultime due classi di problemi che trattiamo sono le classe co-P e co-NP. Fanno parte di queste due classi i problemi
\textbf{complementari} ai problemi in P ed NP.

Dal punto di vista della complessit\`a, passare da un problema al suo complementare, cambia molto a seconda se il problema
sia in P oppure no.

Se ho un problema in P allora anche il suo complementare \`e in P: basta rispondere il contrario del problema di partenza.
Possiamo quindi affermare che P = co-P.

Discorso diverso per i problemi NP. Se un problema NP \`e tale in presenza di un certificato che lo verifichi in tempo
polinomiale, la verifica del suo complementare \`e, in generale, \emph{difficile} e richiede tempo esponenziale. Si
congettura quindi che NP $\neq$ co-NP. 		% COMPLESSITA'
\chapter{Sequenze casuali}\label{casualita}
Introdurre le \textbf{sequenze casuali} ci permette di alimentare gli \textbf{algoritmi randomizzati} e sono inoltre
molto utili per \textbf{generare le chiavi} di cifratura e decifrazione nel modo migliore possibile.

\begin{example}
	Prendiamo due sequenze $h_1$ e $h_2$, entrambe lunghe 20 cifre
	\[
		\begin{matrix}
			h_1 = 1 & 1 & 1 & 1 & 1 & \dots & 1 & \dots & 1 \\
			h_2 = 0 & 1 & 1 & 0 & 0 & \dots & 0 & \dots & 0
		\end{matrix}
	\]
	facciamo finta di aver ottenuto entrambe le sequenze lanciando 20 volte una moneta. La probabilit\`a di ottenere sia
	$h_1$ che $h_2$ con questo metodo \`e di $\frac{1}{2}^{20}$.

	Se dovessimo indicare quale delle due \`e quella \emph{casuale} sarebbe naturale indicare $h_2$. Ma proviamo a definire
	formalmente la \textbf{casualit\`a}.
\end{example}

Il matematico russo Kolmogorov ha descritto il \textbf{significato algoritmico di casualit\`a} per una sequenza binaria.

\begin{definition}[Kolmogorov]
	Una sequenza \emph{binaria} $h$ \`e \textbf{casuale} se non ammette un algoritmo $A$, in grado di descrivere $h$, la
	cui rappresentazione binaria sia pi\`u corta di $h$. Quindi $h$ \`e casuale se
	\[ |h| \leq |A_h| \]
	Possiamo dunque affermare che una sequenza non \`e casuale se c'\`e un algoritmo \emph{semplice} che la descrive.
\end{definition}

\begin{example}
	Prendiamo ora come esempio una sequenza $h$ di $n$ 1. Abbiamo che
	\[ |h| = n \]
	e un algoritmo $A_h$ che la descrive
	\begin{lstlisting}[style=pseudo-style]
	for i = 0 to n-1
		print(1);
	\end{lstlisting}
	La lunghezza della rappresentazione in binario di $A_h$ avr\`a lunghezza
	\[ |A_h| = \text{cost} + \Theta(\log n) \]
	dato che per rappresentare $n$ in binario ho bisogno di $\log n$ bit. Abbiamo quindi descritto con $\log n$	bit
	una sequenza di $n$ bit.
\end{example}

\begin{example}
	Considerando invece una sequenza $h$ che ci appare casuale e non presenta evidenti regolarit\`a, l'algoritmo di
	rappresentazione deve contenerla interamente al suo interno e la restituisce in output.
	\begin{lstlisting}[style=pseudo-style]
	print(1001011110...1010...);
	\end{lstlisting}
	In questo caso la lunghezza della rappresentazione in binario di $A_h$ avr\`a lunghezza
	\[ |A_h| = \text{cost} + \Theta(n) \]
	dato che non ho un modo compatto per descrivere $h$.
\end{example}

\section{Sistemi di calcolo}\label{sistemi_calcolo}
Iniziamo ora a mettere in relazione le sequenze con i \textbf{sistemi di calcolo} dai quali vengono generate.

Prima di addentrarci nell'argomento facciamo qualche precisazione. I sistemi di calcolo sono certamente \emph{infiniti}
ma anche \emph{numerabili} dato che devo poterli descrivere con sequenze finite di caratteri di un alfabeto finito.

\begin{definition}[Complessit\`a di Kolmogorov]
	Sia $h$ una sequenza, $S_i$ un generico sistema di calcolo e $p$ un programma scritto nel formalismo richiesto dal
	sistema di calcolo $S_i$. Indico con
	\[ K_{S_i}(h) = \min\{ |p| \mid S_i(p) = h \} \]
	la \textbf{complessit\`a di Kolmogorov}, ossia la lunghezza del programma pi\`u breve in grado di generare la
	sequenza $h$ nel sistema $S_i$.
\end{definition}

\subsection{Sistema di calcolo universale}
Ci\`o che vogliamo fare \`e rendere la definizione di casualit\`a indipendente dal sistema di calcolo adottato.

Tra i sistemi di calcolo esiste almeno un sistema di calcolo \textbf{universale} $S_u$ in grado di simulare tutti gli
altri sistemi di calcolo.

Questo sistema di calcolo prende in input una coppia $q$, formata da un indice $i$ (l'indice del sistema di calcolo da
simulare) e un programma $p$, da far girare sul sistema di calcolo $S_i$.
\[ S_u(q) =  S_u (\langle i, p \rangle) = S_i (p) = h \]
Se andiamo a valutare la lunghezza di $q$ in bit otteniamo che
\[ |q| = |i| + |p| = \Theta (\log i) + |p| \]
dato che per rappresentare $i$ ho bisogno di $\log i$ bit. Notiamo anche che $|i|$ dipende unicamente dal sistema di
calcolo e non dalla sequenza $h$.

Proviamo ora a valutare la complessit\`a di Kolmogorov su $S_u$. Possiamo dire certamente che
\[ K_{S_u}(h) \leq K_{S_i}(h) + c_i \]
dove $c_i$ \`e una costante che non dipende da $h$.

\begin{definition}
	La \textbf{complessit\`a} secondo Kolmogorov di una sequenza $h$ \`e
	\[ K(h) = K_{S_u}(h) \]
	infatti
	\[ \forall S_i \quad K_{S_i}(h) \geq K_{S_u}(h) \]
	a meno di una costante.
\end{definition}

\begin{definition}[Casualit\`a secondo Kolmogorov]
	Una sequenza $h$ \`e \textbf{casuale} se
	\[ K(h) \geq |h| - \lceil \log |h| \rceil \]
	La ragione per cui si sottrae un fattore logaritmico \`e per rilassare il vincolo di casualit\`a, altrimenti
	soddisfatto da ben poche sequenze.
\end{definition}

Notiamo che la casualit\`a \`e una \emph{propriet\`a} della sequenza, e non dipende quindi dal sistema di calcolo
che l'ha generata.

\section{Esistenza}\label{esistenza_casualita}
Dimostriamo adesso l'\textbf{esistenza} di sequenze casuali secondo la teoria appena descritta.

\begin{proof}
	Consideriamo solo sequenze binarie lunghe $n$ bit e indichiamo con $S$ il numero di sequenze binarie lunghe $n$.
	\[ S = 2^n \]
	Indichiamo inoltre con $T$ il numero di sequenze binarie lunghe $n$ \textbf{non casuali} secondo la definizione data
	in precedenza.

	Per dimostrare l'esistenza di sequenze casuali, dobbiamo dimostrare che
	\[ T < S\]
	ovvero che l'insieme $S$ non \`e composto esclusivamente da sequenze non casuali. Di conseguenza, le sequenze in
	$S$, che non sono in $T$ devono essere necessariamente casuali.

	Iniziamo con il ricordare una caratteristica detta in precedenza sulle sequenze in $T$ e quindi non casuali. I
	programmi che generano sequenze lunghe $n$ in $T$ hanno lunghezza
	\[ |p| < n - \lceil \log n \rceil \]
	perch\'e se la lunghezza \`e almeno $n - \lceil \log n \rceil$ allora la sequenza \`e casuale.

	Indichiamo con $N$ il numero di sequenze binarie $h$ tali che
	\[ |h| < n - \lceil \log n \rceil \]
	Tra queste ci saranno anche le sequenze che codificano i programmi che generano le $T$ sequenze non casuali. Possiamo
	ora definire $N$ come segue
	\[ N = \sum_{i=0}^{n - \lceil \log n \rceil - 1} 2^i = 2^{n - \lceil \log n \rceil} - 1 \]
	Se mettiamo a confronto $N$ con $S$ notiamo subito che
	\[ N < S \]
	ma abbiamo che in $N$ ci sono tutti i programmi che generano le $T$ sequenze non casuali, quindi
	\[ T \leq N < S \quad \Rightarrow \quad T < S \]
\end{proof}

Possiamo fare un'altra considerazione: se sviluppiamo il seguente limite
\[ \lim_{n \rightarrow +\infty} \frac{T}{S} \]
otteniamo
\[
	\lim_{n \rightarrow +\infty} \frac{2^{n - \lceil \log n \rceil} - 1}{2^n} =
	\lim_{n \rightarrow +\infty} \left( \frac{1}{2^{\lceil \log n \rceil}} - \frac{1}{2^n}\right) = 0
\]
Questo ci dice che al crescere di $n$ le sequenze non casuali sono molte meno di quelle casuali.

Fatta questa considerazione ci potrebbe venire in mente di prendere una sequenza qualsiasi lunga $n$ e vedere se \`e
casuale. Purtroppo non esiste un algoritmo che, data una sequenza arbitraria, sia in grado di dirci se questa sia
casuale o meno secondo Kolmogorov.

\begin{theorem}
	Il problema di stabilire se una sequenza sia casuale secondo Kolmogorov \`e indecidibile.
	\begin{proof}
		Procediamo con una dimostrazione per assurdo. Ipotizziamo che esista un algoritmo \verb|RANDOM| in grado di
		dire se $h$ sia casuale o meno.
		\begin{lstlisting}[style=pseudo-style]
RANDOM(h)
	return h.isRandom();
		\end{lstlisting}
		Costruiamo a questo punto l'algoritmo \verb|PARADOSSO| che enumera tutte le sequenze di lunghezza $h$, chiama
		\verb|RANDOM| su ognuna di esse	e infine restituisce la prima sequenza casuale $h$ tale che
		\[ |h| - \lceil \log |h| \rceil > 1 \]
		\begin{lstlisting}[style=pseudo-style]
PARADOSSO()
	for h = 1 to +inf
		if |h| - ceil(log(|h|)) > |P| and RANDOM(h) then
			return h;
		\end{lstlisting}
		\verb|P| \`e una stringa che rappresenta la codifica in binario di \verb|PARADOSSO| e di \verb|RANDOM|, vale
		quindi che
		\begin{center}
			|\verb|P|| \verb|=| |\verb|PARADOSSO|| \verb|+| |\verb|RANDOM||
		\end{center}
		e |\verb|P|| \`e una costante che non dipende da $h$. Questo perch\'e $h$ compare in \verb|PARADOSSO| solo come
		nome di variabile.

		\begin{enumerate}
			\item La prima condizione \`e vera quando incontro una sequenza $h$ tale che
			      \[ |h| - \lceil \log |h| \rceil > |\text{P}| \]
			      ma \verb|P| \`e un programma \textbf{breve} che \emph{genera} $h$, quindi
			      \[ |P| < |h| - \lceil \log |h| \rceil \]
			      ma questo equivale a dire che $h$ non \`e casuale secondo Kolmogorov.
			\item La seconda condizione \`e vera quando \verb|RANDOM| trova una sequenza casuale e, come detto in
			      precedenza, le sequenze casuali esistono e quindi prima o poi \verb|RANDOM| ne trover\`a una.
		\end{enumerate}
		Ecco che qui si ottiene una contraddizione dato che il programma termina su una sequenza che viene riconosciuta
		allo stesso tempo come casuale e come non casuale.
	\end{proof}
\end{theorem}

\section{Sorgente casuale binaria}\label{sorgente_binaria}
Iniziamo ora a parlare di come queste sequenze casuali vengono generate.

\begin{definition}
	Una \textbf{sorgente casuale binaria} produce un flusso di bit con due propriet\`a fondamentali:
	\begin{itemize}
		\item Lo 0 e l'1 si presentano con uguale probabilit\`a
		      \[ \begin{matrix} P(0) & = & P(1) & = & \displaystyle\frac{1}{2} \end{matrix} \]
		      quest'ipotesi, in realt\`a, non \`e cos\`i rigida. Non \`e necessario che le due probabilit\`a siano
		      uguali, basta che
		      \[ P(0) > 0 \quad \wedge \quad P(1) > 0 \]
		      ma devono essere \textbf{immutabili} durante il processo. Quello che si fa per far comparire gli 0
		      gli 1 con uguale frequenza \`e generare la sequenza con probabilit\`a $P(0)$ e $P(1)$ diverse fra loro.
		      Per esempio prendiamo questa sequenza la cui sorgente ha $P(1) > P(0)$
		      \[ \begin{matrix} 0 & 0 & 1 & 0 & 0 & 1 & 1 & 1 & 1 & 1 \end{matrix} \]
		      a questo punto prendiamo in considerazione le coppie di bit generate
		      \[ \begin{matrix} 00 & 10 & 01 & 11 & 11 \end{matrix} \]
		      eliminiamo le coppie in cui i bit sono uguali e valutiamo come 0 le coppie 01 e come 1 le coppie 10,
		      otteniamo cos\`i la sequenza
		      \[ \begin{matrix} 1 & 0 \end{matrix} \]
		      Eliminiamo le coppie in cui i bit sono uguali perch\'e le coppie 11 appaiono con maggiore probabilit\`a
		      e le coppie 00 con minor probabilit\`a. Al contrario le coppie 01 e le coppie 10 sono equiprobabili.
		\item La generazione di un bit \`e indipendente da quella dei bit precedenti. Questa ipotesi \`e molto pi\`u
		      stringente della precedente e non \`e chiaro se sia davvero possibile soddisfarla.
	\end{itemize}
\end{definition}

\subsection{Generatori pseudocasuali}
Dato che riuscire a generare numeri casuali cercando di soddisfare la seconda ipotesi descritta al paragrafo
\ref{sorgente_binaria} \`e molto complesso, quello che si fa nella pratica \`e utilizzare
\textbf{generatori pseudocasuali}, ossia generatori che generano casualit\`a mediante un algoritmo, ricercandola
all'interno di processi matematici.

Questi generatori sono definiti \emph{pseudocasuali} perch\'e, in genere, sono programmi brevi e che quindi generano
sequenze non casuali secondo Kolmogorov.

Un altro motivo \`e che il numero di sequenze possibili, dipende da un parametro di input ossia il \textbf{seme}. Le
possibili sequenze generate sono funzione della lunghezza del seme e sono molte meno di quelle generabili in realt\`a.

Il problema di questi generatori \`e che producono un flusso di bit che ad un certo punto si ripete. Questo avviene
perch\'e, quando in modo casuale, il seme iniziale viene rigenerato, la sequenza riparte da capo. La sequenza che
viene ripetuta all'interno del flusso \`e detta \textbf{periodo}.

Il seme lo otteniamo con un procedimento casuale (per esempio prendendo lo stato interno dell'orologio del calcolatore)
e il periodo lo otteniamo in modo deterministico a partire dal seme.

Possiamo pensare infine al generatore pseudocasuale come ad un \emph{amplificatore} di casualit\`a, dato che cerca di
rendere il periodo pi\`u lungo possibile.

Un generatore \`e tanto migliore quanto pi\`u lungo \`e il periodo che riesce a generare.

Precisiamo che, essendo deterministico, il generatore ha sempre bisogno di un seme diverso per generare una sequenza
diversa. Se forniamo sempre il solito seme otteniamo sempre la solita sequenza.

Supponiamo di partire da un seme $S$ di $s$ bit. Il generatore genera al pi\`u $2^s$ sequenze diverse lunghe quanto il
periodo del generatore.

\subsubsection{Generatore lineare}
\`E un generatore molto semplice che sfrutta, per la generazione del flusso di bit, la funzione
\[ x_i = (a \cdot x_{i - 1} + b) \mod{m} \]
dove i parametri $a$, $b$ ed $m$ sono interi positivi.

Questo generatore ha periodo minore di $m$, dato che, lavorando modulo $m$, possiamo ottenere al massimo $m$ sequenze
diverse. Nella pratica, i parametri $a$ e $b$, si scelgono appositamente per riuscire a far s\`i che il periodo sia
esattamente uguale ad $m$ e non minore.

Se i parametri sono scelti bene il generatore produce una permutazione degli interi da 0 a $m - 1$. Affinch\'e questo
avvenga, $a$, $b$ ed $m$, devono soddisfare alcune propriet\`a:
\begin{itemize}
	\item I parametri $b$ ed $m$ devono essere coprimi
	      \[ (b, m) = 1 \]
	\item Il parametro $a - 1$ deve essere divisibile per ogni fattore primo di $m$.
	\item Se $m$ \`e un multiplo di 4 allora anche $a-1$ deve esserlo
	      \[ 4 \mid m \quad \Rightarrow \quad 4 \mid a - 1 \]
\end{itemize}
Come possiamo vedere questo generatore non possiede la seconda propriet\`a fondamentale, dato che la generazione di un
numero \`e fortemente influenzata dalla generazione del precedente.

Per ora abbiamo usato il generatore per la generazione di numeri interi ma se volessimo generare sequenze binarie
basta calcolare
\[ x_i / m \]
e prendere la \textbf{parit\`a} della prima cifra decimale.

\subsubsection{Generatore polinomiale}
Il primo metodo per riuscire a \emph{generalizzare} un po' il generatore lineare, \`e aumentare il grado del polinomio.
Riscriviamo quindi la funzione precedente ma con un polinomio di grado $t$
\[ x_i = (a_1 x_{i-1}^t + a_2 x_{i-1}^{t-1} + \dots + a_t x_{i-1} + a_{t+1}) \mod{m} \]
Per quanto si possa aumentare il grado del polinomio, questo generatore non \`e tanto migliore del generatore lineare.

Anche per questo dobbiamo scegliere bene i parametri per rendere il periodo pi\`u lungo possibile (sempre uguale a $m$).
Una scelta di parametri molto comune \`e questa:
\[
	\begin{matrix}
		a & = & \pi    \\
		b & = & e      \\
		m & = & 2^{32}
	\end{matrix}
\]

\subsection{Test statistici}
Nel caso in cui volessimo costruire il nostro generatore personale, per riuscire a capire quanto sia effettivamente
valido o ben costruito, dobbiamo sottoporlo ad alcuni \textbf{test statistici}:
\begin{itemize}
	\item \textbf{Test di frequenza}: si verifica che i diversi simboli della sequenza appaiano con pari probabilit\`a.
	\item \textbf{Poker test}: si controlla che le sottosequenze della stessa lunghezza siano equiprobabili.
	\item \textbf{Test di autocorrelazione}: si verifica che, a distanze fissate, non appaia sempre lo stesso simbolo.
	\item \textbf{Run test}: la frequenza di sottosequenze composte interamente dallo stesso simbolo cali in modo
	      esponenziale, relativamente alla loro lunghezza.
\end{itemize}
Ovviamente, per effettuare un buon test, la sequenza deve essere sufficientemente lunga da permettere un'analisi
esauriente.

Sia il generatore lineare che quello polinomiale superano questi test ma non vanno comunque bene per la generazione di
chiavi crittografiche. Anche tenendo nascosti i parametri, esistono algoritmi di costo polinomiale che sono in grado di
stimare, con una probabilit\`a significativamente maggiore di $1/2$, quale sar\`a il prossimo bit generato.

\subsubsection{Test di prossimo bit}
Per le applicazioni crittografiche si richiede un test in pi\`u, il \textbf{test di prossimo bit}.

Un generatore binario supera il test di prossimo bit se non esiste un algoritmo polinomiale in grado di prevedere il
prossimo bit generato a partire dalla conoscenza dei bit gi\`a generati con probabilit\`a strettamente maggiore di $1/2$.

Se un generatore supera questo test, supera anche tutti gli altri test statistici ed \`e detto
\textbf{crittograficamente sicuro}.

\subsection{Generatori crittograficamente sicuri}
La costruzione di generatori sicuri si basa sulle cosiddette funzioni \textbf{one-way}, ossia funzioni \emph{facili} da
calcolare ma \emph{difficili} da invertire.
\begin{itemize}
	\item $y = f(x)$ si calcola in tempo polinomiale.
	\item $x = f^{-1}(y)$ si calcola in tempo esponenziale.
\end{itemize}
Per generare la sequenza si parte da un seme $x_0$ (mantenuto segreto) e si calcola in tempo polinomiale la sequenza
\[
	\begin{matrix}
		x_1   & = f(x_0)     &             &                \\
		x_2   & = f(x_1)     & = f(f(x_0)) & = f^{(2)}(x_0) \\
		\dots &              & \dots       &                \\
		x_i   & = f(x_{i-1}) & \dots       & = f^{(i)}(x_0) \\
		\dots &              & \dots       &
	\end{matrix}
\]
Affich\'e la sequenza non venga scoperta deve essere consumata al contrario. Questo perch\'e si deve sempre assumere
che il crittoanalista conosca $f$. Conoscendo $f$ e vedendo un generico $x_i$ \`e facile calcolare $f(x_{i+1})$.

Se invece consumiamo la sequenza al contrario, l'unico modo per il crittoanalista di prevedere il prossimo bit generato
\`e quello di invertire $f$, ma come abbiamo gi\`a detto, \`e un'operazione che richiede tempo esponenziale.

\subsubsection{Predicati hard-core delle funzioni one-way}
Per rendere il generatore appena descritto, un generatore \emph{binario}, che per\`o conservi le propriet\`a che lo
rendono crittograficamente sicuro, si fa ricorso ai \textbf{predicati hard-core} per le funzioni one-way.
\begin{definition}
	Chiamo $b(x)$ \textbf{predicato hard-core} di $f$ se
	\begin{itemize}
		\item \`e facile da calcolare conoscendo $x$.
		\item \`e difficile da prevedere con probabilit\`a maggiore di $1/2$ se si conosce $f(x)$.
	\end{itemize}
\end{definition}

\begin{example}
	Prendiamo $f$ definita come segue
	\[ f(x) = x^2 \mod{m} \]
	dove $m$ \`e un numero composto e prendiamo $b$, definito come la parit\`a di un numero $x$ in input. Poniamo
	$m = 77$ e $x = 10$, otteniamo
	\[ f(10) = 10^2 \mod{77} = 23 \]
	Arrivati a questo punto, se si conosce $x$ \`e immediato calcolarne la parit\`a ma per chi non lo conosce ed \`e
	in possesso solo di 23 e di $f$, le cose si complicano notevolmente. Si dovrebbe cercare quel numero $x$, che,
	elevato al quadrato modulo 77, faccia 23 e infine calcolarne la parit\`a. Questo procedimento per\`o non si pu\`o
	effettuare se non in tempo esponenziale, facendo una ricerca esaustiva e provando tutti i possibili valori da 0 a
	$m-1$.
\end{example}

\subsubsection{Generatore BBS}
Ideato da Blum, Blum e Shub nell'1986 funziona in questo modo ed \`e tra i generatori classificati sicuri. Consideriamo
\[ n = p \cdot q \]
con $p$ e $q$ che devono soddisfare alcuni requisiti per garantire la difficolt\`a nella fattorizzazione:
\begin{itemize}
	\item Devono essere numeri primi grandi.
	\item $p \mod{4} = 3$
	\item $q \mod{4} = 3$
	\item $2 \lfloor p / 4 \rfloor + 1$ e $2 \lfloor q / 4 \rfloor + 1$ devono essere coprimi.
\end{itemize}
A questo punto scegliamo un numero $y$, coprimo con $n$ e calcoliamo
\[ x_0 = y^2 \mod{n} \]
ottenendo cos\`i il seme $x_0$. Generiamo ora una successione di $m \leq n$ interi, calcolando ogni elemento come segue
\[ x_i = (x_{i-1})^2 \mod{n} \]
Per rendere la sequenza binaria prendiamo come predicato la parit\`a:
\[ b_i = 1 \quad \Leftrightarrow \quad x_{m-i} \text{ \`e dispari} \]

\begin{example}
	Prendiamo $p = 11$ e $q = 19$ e verifichiamo che siano soddisfatte le propriet\`a. La prima propriet\`a, per questo
	esempio non la consideriamo, dato che non \`e necessaria ai fini del funzionamento del generatore ma solo a fini di
	sicurezza crittografica.
	\begin{itemize}
		\item $11 \mod{4} = 3$
		\item $19 \mod{4} = 3$
		\item Calcoliamoci i due valori e vediamo se sono coprimi fra loro
		      \begin{gather*}
			      2 \lfloor 11 / 4 \rfloor + 1 = 7 \\
			      2 \lfloor 19 / 4 \rfloor + 1 = 9
		      \end{gather*}
		      7 e 9 sono coprimi quindi tutte le propriet\`a sono soddisfatte.
	\end{itemize}
	Con $p = 11$ e $q = 19$ abbiamo che
	\[ n = 209 \]
	Prendiamo ora un $y$ coprimo con $n$, diciamo
	\[ y = 30 \]
	e calcoliamo il seme $x_0$ come segue
	\[ x_0 = 30^2 \mod{209} = 64 \]
	Da qui possiamo iniziare a generare la nostra sequenza
	\[ \begin{matrix} x_0 = 64 & x_1 = 125 & x_2 = 159 & x_3 = 201 \end{matrix} \]
	Usiamo infine la parit\`a per ottenere la sequenza binaria
	\[ \begin{matrix} 0 & 1 & 1 & 1 \end{matrix} \]
\end{example}

\subsection{Generatori basati su cifrari simmetrici}
Il funzionamento di generatori di questo tipo si basa sui seguenti parametri:
\begin{itemize}
	\item Un cifrario simmetrico (DES, AES, ecc).
	\item $r$: numero di bit delle parole prodotte.
	\item $s$: seme casuale di $r$ bit.
	\item $m$: numero di parole di $r$ bit.
	\item $k$: chiave segreta del cifrario.
\end{itemize}
Di seguito quello che potrebbe essere un generatore basato su cifrari simmetrici:
\begin{lstlisting}[style=pseudo-style]
Generatore(s, m)
	d = time(); // in r bit 
	y = C(d, k); // C = funzione di cifratura 
	z = s;
	for i = 1 to n
		x_i = C(y ^ z, k); // ^ = xor bit a bit
		z = C(y ^ x_i, k);
		output(x_i); // comunico x_i all'esterno
\end{lstlisting} 			% SEQUENZE CASUALI
\chapter{Numeri primi}\label{primi}
Iniziamo ora a parlare di \textbf{primalit\`a} e di come costruire algoritmi efficienti per effettuare il
\textbf{test di primalit\`a}, ossia algoritmi in grado di dirci se un numero \`e primo o no e come fare per generare
numeri primi.

Iniziamo con un algoritmo semplice ma che come vedremo risulter\`a molto inefficiente.
\begin{lstlisting}[style=pseudo-style]
Primo(n)
	for i = 2 to sqrt(n)
		if n % i == 0 then
			return false;
	
	return true;
\end{lstlisting}
Come possiamo facilmente constatare, l'algoritmo
\begin{enumerate}
	\item Controlla se uno dei numeri da 2 a $\sqrt{n}$ divide $n$. Si parte da 2 dato che la divisione per 0 \`e in
	      generale indefinita o comunque tende all'infinito e tutti i numeri sono divisibili per 1.

	      Ci si ferma a $\sqrt{n}$ dato che un numero, se composto, possiede sicuramente un divisore minore della sua
	      radice.
	\item Se ne trova uno che divide $n$ allora ritorna \verb|false|.
	\item Se non ne trova nessuno ritorna \verb|true|.
\end{enumerate}
Un'analisi poco attenta potrebbe indurci a pensare che il costo di questo algoritmo sia $O(\sqrt{n})$ dato che faccio al
pi\`u $\sqrt{n}$ iterazioni. Questo \`e in parte vero ma dobbiamo considerare la dimensione dell'istanza di input, la
sua rappresentazione e il costo della divisione.

L'istanza di input, ossia $n$, richiede $\Theta(\log_2 n)$ bit per essere rappresentata mentre la divisione \`e, in
generale, un'operazione quadratica nel numero di cifre. Tutto questo fa salire la complessit\`a a
\[ O(\sqrt{n} \cdot \log^2 n) \]
Ma non \`e finita qui: come abbiamo detto, $n$, necessita di $\Theta(\log_2 n)$ bit per essere rappresentato, dunque $n$
si pu\`o scrivere come $2^{\log n}$ e questo fa diventare la complessit\`a
\[ O(2^\frac{\log n}{2} \cdot \log^2 n) \]
Come possiamo vedere, un algoritmo all'apparenza polinomiale \`e in realt\`a un algoritmo di costo esponenziale nella
dimensione dell'input. Si tratta di un algoritmo \textbf{pseudopolinomiale}.

\section{Algoritmo di esponenziazione veloce}\label{esponenziazione}
Introduciamo l'\textbf{algoritmo di esponenziazione veloce} o \textbf{algoritmo delle quadrature successive} che sar\`a
molto utile per svolgere elevamenti a potenza in modo efficiente. Essere in grado di svolgere elevamenti a potenza
in modo efficiente \`e necessario per avere algoritmi efficienti per il test di primalit\`a.

Sia $b$ la base e $n$ l'esponente e $m$ un intero qualsiasi, vogliamo calcolare
\[ x = b^n \mod{m} \]
con un numero di operazioni dell'ordine di $O(\log_2 n)$.
\begin{enumerate}
	\item Si scompone l'esponente $n$ come somma di potenze di 2.
	      \[ n = \sum_{i=0}^{\lfloor \log_2 n \rfloor} k_i \cdot 2^i \quad \quad k_i \in \{0,1\} \]
	\item Si calcolano tutte le potenze $b^{2^i} \mod{m}$ calcolando ogni potenza come quadrato delle precedente
	      \[ b^{2^i} \mod{m} = \left( b^{2^{i-1}} \right)^2 \mod{m} \]
	      con $1 \leq i \leq \lfloor \log_2 n \rfloor$
	\item Prendiamo, fra le potenze ricavate al punto 2, quelle che compaiono nella scomposizione fatta al
	      punto 1 e moltiplichiamole
	      \[ x = \prod_{i \mid k_i \neq 0} b^{2^i} \mod{m} \]
\end{enumerate}

Al punto 2 vengono compiute esattamente $\log_2 n$ operazioni per il calcolo delle varie potenze e al punto 3 svolgiamo
un numero di moltiplicazioni dell'ordine di $O(\log_2 n)$.

Se consideriamo che la moltiplicazione \`e un'operazione di costo polinomiale nel numero delle cifre \`e facile
stabilire che l'algoritmo sia complessivamente di costo polinomiale.

\begin{example}
	Vogliamo calcolare
	\[ 9^{45} \mod{11} \]
	Scriviamo l'esponente come somma di potenze di 2
	\[ 9^{32 + 8 + 4 + 1} \mod{11} \]
	Calcoliamo ora le potenze $9^{2^i}$ fino ad arrivare a $9^{32}$, ognuna calcolata come quadrato della precedente.
	\[
		\begin{matrix}
			9^2 \mod{11} =    &                & 4 \\
			9^4 \mod{11} =    & 4^2 \mod{11} = & 5 \\
			9^8 \mod{11} =    & 5^2 \mod{11} = & 3 \\
			9^{16} \mod{11} = & 3^2 \mod{11} = & 9 \\
			9^{32} \mod{11} = & 9^2 \mod{11} = & 4
		\end{matrix}
	\]
	Ora non ci rimane che prendere le potenze di cui abbiamo bisogno
	\[ 9^{45} \mod{11} = (9^{32} \mod{11}) \cdot (9^8 \mod{11}) \cdot (9^4 \mod{11}) \cdot (9^1 \mod{11}) \]
	che possiamo comodamente riscrivere come
	\[ 9^{45} \mod{11} = 4 \cdot 3 \cdot 5 \cdot 9 = 1 \]
\end{example}

\section{Algoritmi randomizzati}\label{algoritmi_random}
Gli algoritmi randomizzati sono fondamentali per effettuare test di primalit\`a efficienti dato che un algoritmo
deterministico e polinomiale nella dimensione dell'input esiste ma \`e comunque molto lento.

Questi algoritmi si dividono in due gruppi principali
\begin{itemize}
	\item \textbf{Las Vegas}: generano un risultato \emph{sicuramente corretto} in un tempo \emph{probabilmente breve}.
	\item \textbf{Monte Carlo}: generano un risultato \emph{probabilmente corretto} in un tempo \emph{sicuramente breve}.
\end{itemize}
L'algoritmo che vedremo per il test di primalit\`a \`e della tipologia Monte Carlo ma la probabilit\`a di errore
dev'essere \textbf{misurabile} e \textbf{arbitrariamente piccola}.

\subsection{Classe RP}
La \textbf{classe RP} comprende tutti quei problemi \emph{verificabili} in tempo polinomiale tramite algoritmi
randomizzati.

Sia $\Pi$ un problema decisionale, $x$ un istanza di input di $\Pi$ allora $y$ \`e un \textbf{certificato probabilistico}
di $x$ se
\begin{itemize}
	\item \`E di lunghezza al pi\`u polinomiale in $|x|$ (devo leggerlo in tempo polinomiale).
	\item \`E estratto perfettamente a caso da un insieme associato a $x$.
\end{itemize}
Possiamo anche dire che $A$ \`e un \textbf{algoritmo di verifica} che prende in input $x$ e $y$ se, in tempo polinomiale,
riesce ad attestare che
\begin{itemize}
	\item $x$ non possiede la propriet\`a.
	\item $x$ possiede la propriet\`a con probabilit\`a $> 1/2$.
\end{itemize}
Si congettura che
\[ \text{P} \subset \text{RP} \subset \text{NP} \]

\section{Test di Miller-Rabin}\label{Miller_Rabin}
La prima parte dell'algoritmo \`e composta dei seguenti passi.
\begin{enumerate}
	\item Prendiamo $n$ intero e dispari di cui vogliamo testare la primalit\`a.
	\item Prendiamo $n-1$ (sicuramente pari) e cerchiamo la massima potenza di 2 che lo divide, cos\`i da rappresentare
	      $n-1$ in questo modo
	      \[ n-1 = 2^w \cdot z \quad \text{con $z$ dispari} \]
	      questo \`e sempre possibile perch\'e un numero pari \`e sempre rappresentabile come potenza di 2 che moltiplica
	      un numero dispari.

	      Per determinare $w$ e $z$ impieghiamo in media $O(\log n)$ passi.
	\item Scegliamo un intero $y$ compreso tra 2 e $n-1$.
\end{enumerate}
Se $n$ \`e primo allora valgono i due predicati
\begin{itemize}
	\item $(n, y) = 1$
	\item $y^z \mod{n} \equiv 1$

	      oppure

	      $\exists i, \quad 0 \leq i \leq w-1 \mid y^{2^i} \cdot z \mod{n} \equiv -1$
\end{itemize}
Chiariamo che questi due predicati sono condizioni necessarie alla primalit\`a ma non sufficienti.

\begin{lemma}[Miller-Rabin]
	Se $n$ \`e un numero composto, il numero di interi $y$ compresi tra 2 ed $n-1$, che soddisfano entrambi i predicati
	\`e minore di $n / 4$.
	\[ \# \{ 2 \leq y \leq n-1 \mid P_1(y) = \text{ true} \quad \wedge \quad P_2(y) = \text{ true} \} < \frac{n}{4} \]
\end{lemma}
Questo lemma ci dice che la probabilit\`a di scegliere un $y$ che soddisfa entrambi i predicati \`e minore di $1 / 4$.
Questo \`e banale dato che ho $n - 2$ possibili scelte e $n / 4$ di queste rendono veri entrambi i predicati.
\[ \frac{n/4}{n-2} < \frac{1}{4} \]
Quando scelgo un $y$ vado a testare i due predicati: se anche solo uno \`e falso allora possiamo affermare con certezza
che $y$ \`e composto, se invece sono entrambi soddisfatti \emph{molto probabilmente} \`e primo con probabilit\`a di
errore al pi\`u del $25\%$.

A questo punto possiamo iterare $k$ volte con $k$ scelte casuali e indipendenti di $y$, la probabilit\`a di errore scende
a $(1/4)^k$.

Possiamo quindi concludere l'algoritmo in questo modo
\begin{enumerate}
	\setcounter{enumi}{3}
	\item Verifico che i due predicati siano soddisfatti.
	\item Itero $k$ volte su $k$ scelte diverse e casuali di $y$.
	      \begin{itemize}
		      \item Se anche solo una volta un predicato non \`e soddisfatto allora il numero non \`e primo.
		      \item Altrimenti possiamo affermare che lo sia con probabilit\`a di errore inferiore a $(1/4)^k$
	      \end{itemize}
\end{enumerate}

\begin{lstlisting}[style=pseudo-style]
Verifica(n, y) // true se n e' composto
	if not P1(n, y) or not P2(n, y) then
		return false;
	else
		return true;
\end{lstlisting}

\begin{lstlisting}[style=pseudo-style]
TestMR(n, k)
	for i = 1 to k
		y = random(2, n - 1);
		if Verifica(y, n) then 
			return false;

	return true;
\end{lstlisting}

\subsection{Verifica dei predicati}
Possiamo valutare il costo del ciclo come costante ($k$ cicli) mentre \`e fondamentale capire il costo della verifica
dei predicati.
\begin{itemize}
	\item Per la verifica del primo predicato dobbiamo calcolare il massimo comun divisore con l'algoritmo di Euclide.
	      Un'operazione che richiede costo cubico nel numero di cifre
	      \[ O(\log^3 n) \]
	\item La verifica del secondo predicato \`e pi\`u complessa per via degli elevamenti a potenza.

	      Prima di tutto dobbiamo calcolare $y^z$ e per capire quanto sia costosa questa operazione dobbiamo capire quanto
	      \`e grande $z$. Se andiamo a vedere come si ottiene $z$ possiamo ricavare che $z$ abbia al massimo valore
	      $\frac{n-1}{2}$ dunque $z$ \`e dell'ordine di $n$.

	      Per fare quindi $y^z$ non posso moltiplicare $y$ per se stesso $\frac{n-1}{2}$ volte perch\'e farei un numero
	      di operazioni proporzionale al valore $n$ e non possiamo permettercelo in termini computazionali.

	      Il nostro obbiettivo \`e quello di compiere un numero di operazioni che \`e proporzionale al numero delle cifre
	      ossia $\log_2 n$ e come sappiamo, questo si pu\`o fare con l'\textbf{algoritmo di esponenziazione veloce},
	      esposto al paragrafo \ref{esponenziazione}.

	      Per la verifica della seconda parte del predicato basta semplicemente elevare al quadrato il risultato ottenuto
	      per la verifica della prima parte del predicato. Anche questa seconda verifica ha quindi costo polinomiale.
\end{itemize}
Come possiamo vedere, la verifica dei predicati, ha complessivamente costo polinomiale.

\section{Generazione di numeri primi}\label{generazione_primi}
La generazione di numeri primi casuali si effettua semplicemente generando un numero casuale e in seguito si effettua
il test di primalit\`a di Miller-Rabin. Ripeto la generazione del numero casuale finch\'e non ne trovo uno che posso
dichiarare primo con un possibilit\`a di errore pi\`u bassa possibile.

\begin{theorem}
	Il numero di numeri primi minori di un certo $n$ tende a
	\[ \frac{n}{\ln n} \]
	per $n$ che tende all'infinito.
\end{theorem}

Questo teorema ci dice che, per $n$ sufficientemente grande, nel suo intorno (di ampiezza $\ln n$) cade mediamente un
numero primo.

Mediamente dovremo quindi fare un numero $\ln n$ di tentativi, il che ci va bene dato che \`e polinomiale nel numero
delle cifre.

\subsection{Algoritmo}
Costruiamo ora un algoritmo per generare un numero primo di $n$ bit.

\begin{lstlisting}[style=pseudo-style]
Primo(n)
	S = randomSeq(n-2); // sequenza casuale di n-2 bit
	N = 1 + S + 1; // numero dispari con n bit significativi
	while TestMR(N, k) == 0 
		N = N + 2;

	return N;
\end{lstlisting}
Il costo complessivo \`e $O(n^4)$ dato che facciamo $O(n)$ volte il test di Miller-Rabin, il quale aveva un costo
complessivamente di $O(n^3)$.
 			% NUMERI PRIMI
\chapter{Cifrari storici}
In questo capitolo andremo a trattare i \textbf{cifrari storici}, chiamati con questo nome perch\'e ad oggi sono stati
tutti forzati e dunque non sono pi\`u cifrari sicuri.

I primi cifrari nascono in un periodo in cui cifratura e decifrazione si facevano "con carta e penna" o quasi e servivano
per cifrare frasi di senso compiuto in linguaggio naturale.

Da un certo punto in poi tutti i cifrari hanno cercato di seguire i cosiddetti \textbf{principi di Bacone}
\begin{itemize}
	\item Le funzioni $C$ e $D$ devonon essere \textbf{facili da calcolare}.
	\item \`E \textbf{impossibile} ricavare la $D$ se la $C$ non \`e nota.
	\item Il crittogramma $c = C(m)$ deve apparire "\textbf{innocente}", deve sembrare cio\`e un testo in chiaro e non
	      una sequenza di caratteri insensata.
\end{itemize}

\section{Cifrario di Cesare}
L'idea di base \`e che il crittogramma $c$ \`e ottenuto dal messaggio in chiaro $m$ sostituendo ogni lettera di $m$ con
quella tre posizioni pi\`u avanti nell'alfabeto.

\begin{center}
	A B C D \dots W X Y Z
	\[ \downarrow \]
	D E F G \dots Z A B C
\end{center}
La decifrazione \`e immediata: basta sostituire ogni lettera del crittogramma con la lettera tre posizioni pi\`u indietro
nell'alfabeto.

\`E un cifrario molto semplice e non utilizza una chiave di cifratura. Una volta scoperto il metodo di cifratura e
decifrazione diventa del tutto inutile.

\subsection{Cifrario di Cesare generalizzato}
Per rendere il cifrario un po' pi\`u robusto basterebbe inserire un chiave $1 \leq k \leq 25$ (26 lascia inalterato il
messaggio) e invece di traslare sempre di tre posizioni le lettere del messaggio in chiaro, le trasliamo di $k$
posizioni.

Ovviamente i due utenti devono possedere la solita chiave per cifrare e decifrare i messaggi.

\subsubsection{Cifratura e decifrazione}
Sia $x$ una lettera dell'alfabeto, $pos(x)$ la sua posizione nell'alfabeto e $k$ la chiave tale che $1 \leq k \leq 25$.

La funzione di cifratura ritorna la lettera $y$ tale che
\[ pos(y) = (pos(x) + k) \mod{26} \]

La funzione di decifrazione ritorna la lettera $x$ tale che
\[ pos(x) = (pos(y) - k) \mod{26} \]

Con un calcolatore moderno \`e immediato effettuare un attacco a forza bruta: si provano tutte le 25 chiavi. 			% CIFRARI STORICI
\chapter{Cifrari perfetti}\label{perfetti}
I \textbf{cifrari perfetti}, detti anche \textbf{cifrari a sicurezza incondizionata}: sono cifrari per uso ristretto e
nascondono l'informazione con certezza assoluta (anche per macchine quantistiche).

Un \textbf{cifrario perfetto} \`e tale se non si riesce ad estrapolare alcuna informazione dall'analisi del crittogramma.

Proviamo a formalizzare matematicamente quanto appena detto. Per farlo dobbiamo considerare
\begin{itemize}
	\item \textbf{MSG}: spazio dei messaggi.
	\item \textbf{CRITTO}: spazio dei crittogrammi.
	\item \textbf{M}: variabile aleatoria che descrive il comportamento del	mittente e assume i valori in MSG.
	\item \textbf{C}: variabile aleatoria che descrive la comunicazione sul canale.
\end{itemize}
Indichiamo ora con
\[ P(M = m) \]
la probabilit\`a che il mittente voglia inviare il messaggio $m \in$ MSG. Indichiamo invece con
\[ P(M = m \mid C = c) \]
la probabilit\`a condizionata che il messaggio inviato sia proprio $m$, posto che sul canale stia transitando il
crittogramma $c \in$ CRITTO. In altre parole quest'ultima espressione indica la probabilit\`a che $c$ sia $m$ cifrato.

\begin{theorem}\label{th: cifrario_perfetto}
	Un cifrario \`e \textbf{perfetto} se $\forall m \in \text{MSG}$ e $\forall c \in \text{CRITTO}$ vale che
	\[ P(M = m \mid C = c) = P(M = m) \]
\end{theorem}

\begin{example}
	Mettiamoci per un attimo in uno scenario di massimo pessimismo in cui il crittoanalista sa:
	\begin{itemize}
		\item La distribuzione di probabilit\`a con cui il mittente invia messaggi.
		\item Il cifrario utilizzato.
		\item Lo spazio delle chiavi.
	\end{itemize}
	Supponiamo inoltre che di voler inviare un messaggio $m$ con probabilit\`a
	\[ P(M = m) = p > 0 \quad \quad \text{con } 0 < p < 1 \]
	e analizziamo due casi estremi e opposti l'uno all'altro. Nel primo caso diciamo che esiste un crittogramma $c$ tale
	che
	\[ P(M = m \mid C = c) = 1 \]
	e nel secondo caso diciamo che esiste un crittogramma $c$ tale che
	\[ P(M = m \mid C = c) = 0 \]
	In entrambi i casi, vedere il crittogramma, raffina la conoscenza del crittoanalista. L'unico caso in cui il
	crittoanalista non ricava nulla dal crittogramma \`e il caso descritto dal teorema \ref{th: cifrario_perfetto}.
\end{example}

\section{Svantaggi}\label{svantaggi_perfetti}
L'estrema solidit\`a di un cifrario perfetto ha per\`o un costo in termini di numero di chiavi.

\begin{theorem}[Shannon]
	In un cifrario perfetto l'insieme delle chiavi deve essere grande almeno quanto l'insieme dei messaggi possibili.
	Dove per \textbf{messaggio possibile} indichiamo un messaggio $m \in$ MSG tale che
	\[ P(M = m) > 0 \]
	Questa \`e condizione necessaria ma non sufficiente affinch\'e il cifrario sia perfetto.
	\begin{proof}
		Dimostriamo il teorema per assurdo e andiamo ad indicare con $N_k$ il numero delle chiavi e con $N_m$ il numero
		dei messaggi possibili.

		Supponiamo per assurdo che
		\[ N_m > N_k \]
		e consideriamo ora un crittogramma $c$ che pu\`o transitare sul canale con probabilit\`a
		\[ P(C = c) > 0 \]
		Se provassimo a decifrare $c$ con una generica chiave $k_i$ otterremo un messaggio $m_i$. Facciamo per\`o
		attenzione al fatto che cifrando $c$ con una chiave $k_j$ potremmo ottenere il messaggio $m_i$, ottenibile anche
		con la chiave $k_i$.

		Indichiamo quindi con $s$ tale che
		\[ s \leq N_k \]
		il numero dei messaggi che potrebbero corrispondere al crittogramma $c$. Ma per ipotesi abbiamo che
		\[ N_k < N_m \]
		quindi
		\[ s \leq N_k < N_m \]
		Ho ottenuto che il numero dei messaggi che possono corrispondere al crittogramma $c$ \`e strettamente minore
		del numero dei messaggi possibili.

		Questo vuol dire che esiste un messaggio $m'$ appartenente allo spazio dei messaggi possibili che non pu\`o
		corrispondere a quel crittogramma.
		\[ P(M = m' \mid C = c) = 0 \]
		Giungiamo quindi all'assurdo dato che un cifrario \`e perfetto se un crittogramma pu\`o corrispondere ad uno
		qualsiasi dei messaggi possibili.
	\end{proof}
\end{theorem}

\section{One-Time Pad}\label{one_time_pad}
Come abbiamo in parte anticipato, il cifrario \textbf{One-Time Pad} altro non \`e che un cifrario di Vigen\`ere che
cifra e decifra sequenze binarie e dove la chiave, al posto di essere corta e ripetuta, \`e lunga quanto il messaggio
e dunque non \`e mai ripetuta.

La prima parte del nome (One-Time) \`e relativa alla chiave: ogni chiave dev'essere utilizzata una sola volta e poi
buttata via.

\subsection{Funzionamento}\label{funzionamento_otp}
Consideriamo
\begin{itemize}
	\item \textbf{MSG}: lo spazio dei messaggi.
	\item \textbf{CRITTO}: lo spazio dei crittogrammi.
	\item \textbf{KEY}: lo spazio delle chiavi.
\end{itemize}
Sia il messaggio, che la chiave, che il crittogramma saranno una sequenza di $n$ bit. Il crittogramma si compone facendo
lo XOR bit a bit di messaggio e chiave
\[ c = m \oplus k \]
Lo XOR ritorna 1 se i bit che sto confrontando sono uguali, ritorna 0 altrimenti. Il crittoanalista, vedendo il
crittogramma, sa che
\begin{itemize}
	\item quando vede uno 0 in posizione $i$, allora i bit di messaggio e chiave in posizione $i$ sono uguali ma
	      non si sa se siano tutti e due 0 o tutti e due 1.
	\item quando vede un 1 in posizione $i$, allora i bit di messaggio e chiave in posizione $i$ sono diversi ma
	      non si sa quale sia 1 e quale sia 0.
\end{itemize}
Per effettuare la decifrazione basta rifare lo XOR bit a bit di crittogramma e chiave
\[ c_i \oplus k_i = m_i \]
si vede facilmente che il procedimento funziona
\begin{gather*}
	c_i = m_i \oplus k_i \\
	\Downarrow \\
	m_i \oplus k_i \oplus k_i = m_i
\end{gather*}
ma $k_i \oplus k_i$ \`e un sequenza di 0 e quindi deduciamo facilmente che
\[ m_i \oplus 0 = m_i \]

\subsection{Debolezza}\label{debolezza_otp}
La debolezza si ha dal punto di vista della generazione delle chiavi. Come abbiamo detto, la chiave dev'essere monouso.

Prendiamo come esempio il caso in cui due messaggi, $m_1$ ed $m_2$ siano cifrati, con la stessa chiave $k$, in due
crittogrammi
\begin{gather*}
	c_1 = m_1 \oplus k \\
	c_2 = m_2 \oplus k
\end{gather*}
A questo punto il crittoanalista potrebbe applicare lo XOR bit a bit fra i due crittogrammi per ottenere
\[ c_1 \oplus c_2 = (m_1 \oplus k) \oplus (m_2 \oplus k) \]
dato che vale la propriet\`a associativa posso scrivere
\[ c_1 \oplus c_2 = (m_1 \oplus m_2) \oplus (k \oplus k) \]
Come prima $k \oplus k$ \`e una sequenza di 0 e quindi otteniamo
\[ c_1 \oplus c_2 = m_1 \oplus m_2 \]
Dalla sequenza di bit ottenuta, si pu\`o raffinare la propria conoscenza del messaggio andando a cercare lunghe sequenze
di 0, le quali indicano che quella parte di messaggio \`e stata inviata due volte.

\subsection{Sicurezza}\label{sicurezza_otp}
Vogliamo ora dimostrare che il cifrario \`e perfetto. Per farlo lavoriamo sotto alcune ipotesi
\begin{itemize}
	\item Tutti i messaggi sono lunghi $n$. Se il messaggio \`e pi\`u corto di $n$ faccio un po' di \emph{padding}.
	      Se invece il messaggio \`e pi\`u lungo di $n$ faccio una cifratura a blocchi.
	\item Tutte le sequenze di $n$ bit sono messaggi possibili.
	\item I messaggi privi di significato vengono utilizzati per confondere la crittoanalisi e ognuno di essi ha una
	      probabilit\`a molto bassa, ma comunque maggiore di 0, di essere inviati.
	\item La chiave deve essere casuale e unica per ogni messaggio.
\end{itemize}

\begin{theorem}
	Sotto le ipotesi appena elencate, One-Time Pad \`e un cifrario perfetto e impiega un numero minimo di chiavi.
	\begin{proof}
		Dimostriamo per prima cosa la \textbf{minimalit\`a}, ossia
		\[ N_n = N_k = 2^n \]
		ma questo \`e immediato dato che le chiavi sono sequenze di bit lunghe quanto i messaggi.
	\end{proof}

	\begin{proof}
		Dimostriamo ora che il cifrario \`e perfetto. Come sappiamo, un cifrario \`e perfetto se per ogni $m \in$ MSG e
		per ogni $c \in$ CRITTO vale
		\[ P(M = m \mid C = c) = P(M = m) \]
		Partiamo dicendo che
		\[ P(M = m \mid C = c) = \frac{P(M = m \wedge C = c)}{P(C = c)} \]
		dove il valore al numeratore \`e la probabilit\`a che il messaggio inviato sia $m$ e che sia stato cifrato in $c$.

		Per come \`e fatto lo XOR, fissato $m$, chiavi diverse producono crittogrammi diversi. Esiste dunque
		un'\textbf{unica} chiave $k$ che cifra $m$ in $c$. Pi\`u formalmente possiamo affermare che la probabilit\`a che
		il crittogramma sia $c$ \`e uguale alla probabilit\`a di scegliere a caso l'unica chiave $k$ che cifra $m$ in $c$
		\[ P(C = c) = \frac{1}{2^n} \]
		Se la probabilit\`a di ottenere il crittogramma $c$ dipende solo dalla chiave, allora i due eventi sono al
		numeratore indipendenti e possiamo quindi riscrivere la formula iniziale in questo modo
		\[ P(M = m \mid C = c) = \frac{P(M = m) \cdot P(C = c)}{P(C = c)} \]
		Semplificando \`e immediato ottenere
		\[ P(M = m \mid C = c) = P(M = m) \]
	\end{proof}
\end{theorem}

\subsection{Scambio delle chiavi}\label{chiavi_otp}
Un metodo ragionevole per lo scambio di chiavi \`e quello che prevede lo scambio tra i due utenti del generatore casuale
e del suo assetto iniziale (seme).

In questo modo il procedimento di cifratura e decifrazione funziona in questo modo
\begin{enumerate}
	\item I due generatori vengono impostati allo stesso modo con lo stesso seme.
	\item Si scrive un messaggio $m$ e si genera un chiave $k$ lunga $|m|$ con il generatore.
	\item Si cifra il messaggio con $k$.
	\item Si genera la chiave $k$ di $|c|$ bit con il secondo generatore, che ricordiamo essere uguale al primo e
	      inizializzato con lo stesso seme.
	\item Si decifra il crittogramma $c$ con la chiave $k$ generata dal secondo generatore.
\end{enumerate}
Alla fine di questo processo si ha che i due generatori sono impostati di nuovo alla stessa maniera e si pu\`o quindi
continuare la comunicazione.

Il generatore deve essere crittograficamente sicuro e il seme deve essere molto lungo in modo da essere al riparo da
attacchi a forza bruta sul seme.

\subsection{Conclusioni}\label{conclusioni_otp}
In conclusione proviamo a rimuovere l'ipotesi secondo cui ogni messaggio sia possibile, anche quelli non significativi.

Dato che le chiavi devono essere tante quante i messaggi possibili. Se restringessimo l'insieme dei messaggi possibili
anche lo spazio delle chiavi diventerebbe pi\`u piccolo e con esso anche la lunghezza delle chiavi diminuirebbe.

In lingua inglese i messaggi significativi lunghi $n$ bit sono circa $\alpha^n$ con
\[ \alpha = 1.1 \]
Se considerassimo quindi solo l'insieme dei messaggi significativi in inglese, la cardinalit\`a dell'insieme di chiavi
passerebbe da $2^n$ a $1.1^n$.

Il numero delle chiavi dev'essere almeno quanto il numero dei messaggi
\[ N_k \geq N_m = \alpha^n < 2^n \]
e dato che $\alpha^n < 2^n$ posso descrivere le chiavi con $t$ bit con $t$ tale che
\[ 2^t \geq \alpha^n \]
quindi
\[ t \quad \geq \quad n \log_2 \alpha \quad \tilde{=} \quad 0.12 \cdot n \]
Abbiamo cos\`i ridotto il numero di chiavi possibili a circa un decimo del numero di chiavi che avevamo prima.

Il problema \`e che avendo ridotto cos\`i tanto l'insieme delle chiavi, un attacco di tipo forza bruta torna ad avere
senso.

Quello che si fa in genere per riuscire a mitigare il problema riuscendo comunque a diminuire un po' il numero di chiavi
e far s\`i che decifrando un crittogramma con chiavi diverse si riesca a risalire a diversi messaggi significativi.

In altre parole, cifrando messaggi diversi con chiavi diverse si ottiene lo stesso crittogramma.

Per fare questo il numero di coppie $(m, k)$ deve essere molto maggiore del numero di crittogrammi. Supponiamo di usare
chiavi casuali di $t$ bit. Se il numero di messaggi significativi \`e $\alpha^n$ abbiamo
\[ \alpha^n \cdot 2^t \]
possibili coppie $(m, k)$ mentre il numero dei crittogrammi rimane $2^n$. Otteniamo dunque che
\[ \alpha^n \cdot 2^t >> 2^n \]
che equivale a
\[ n \log_2 \alpha + t >> n \]
sviluppando ancora i calcoli otteniamo che le chiavi devono essere lunghe
\[ t >> n - n\log_2 \alpha \quad \rightarrow \quad t >> 0.88 \cdot n \]
affinch\'e si verifichi il fenomeno descritto in precedenza, ossia che a pi\`u coppie messaggio-chiave corrisponda lo
stesso crittogramma.

La rimozione dell'ipotesi non ci permette quindi di risparmiare sui bit della chiave se si vuole mantenere un buon grado
di sicurezza. Siamo comunque riusciti a diminuire il numero delle chiavi. 			% CIFRARI PERFETTI
\chapter{Cifrari simmetrici}\label{critto_sim_massa}
Questi cifrari basano la loro sicurezza sulla difficolt\`a nel risolvere problemi complessi. Si dice quindi che hanno
una sicurezza di tipo computazionale e nascondono l'informazione a patto che il crittoanalista abbia risorse
computazionali limitate e sotto l'ipotesi che P $\neq$ NP.

La loro sicurezza si basa sui due \textbf{principi di Shannon}, i quali, rendono questi cifrari robusti alla
crittoanalisi statistica:
\begin{itemize}
	\item \textbf{Diffusione}: Il testo in chiaro si deve distribuire su tutto il crittogramma.

	      Ogni carattere del crittogramma deve cio\`e dipendere da tutti i caratteri del blocco del messaggio ottenendo
	      cos\`i un istogramma delle frequenze piatto.
	\item \textbf{Confusione}: Messaggio e crittogramma sono combinati fra loro in modo molto complesso per non
	      permettere al crittoanalista di separare le due sequenze tramite l'analisi statistica del crittogramma.

	      Per far s\`i che questo avvenga devono essere vere due condizioni:
	      \begin{itemize}
		      \item La chiave deve essere ben distribuita sul testo cifrato.
		      \item Ogni bit del crittogramma deve dipendere da tutti i bit della chiave.
	      \end{itemize}
\end{itemize}

\section{DES}\label{DES}
\`E stato il primo cifrario \textbf{certificato} proposto da IBM e che proponeva una struttura di questo tipo:
\begin{itemize}
	\item Il messaggio \`e diviso in blocchi, ciascuno cifrato e decifrato indipendetemente dall'altro.
	\item Ogni blocco \`e di 64 bit.
	\item Cifratura e decifrazione procedono in $r$ fasi o \textbf{round} in cui si ripetono le stesse operazioni. Noi
	      considereremo la versione del cifrario con 16 round.
	\item La chiave \`e composta da 8 byte. I primi 7 bit di ciascun byte sono scelti arbitrariamente e l'ottavo \`e
	      aggiunto per il controllo di parit\`a.
	      \begin{itemize}
		      \item La chiave contiene dunque 64 bit: 56 arbitrari e 8 di parit\`a.
		      \item Dalla chiave vengono create $r$ \textbf{sottochiavi di fase}.
	      \end{itemize}
\end{itemize}

\subsection{Funzionamento}\label{funzionamento_DES}
Sia $m$ il messaggio da inviare, $c$ il rispettivo crittogramma e $k$ la chiave. Il processo di cifratura \`e il seguente
\begin{enumerate}
	\item I bit del messaggio vengono permutati (blocco PI).
	\item La chiave viene privata dei bit di controllo parit\`a e i rimanenti vengono permutati (blocco T).
	\item Si dividono i bit del messaggio in due parti (S e D), ciascuna di 32 bit.
	\item Si entra in un ciclo di 16 fasi e per ogni fase $i$ abbiamo in input, l'output della fase precedente.

	      Alla chiave $k$ si applicano queste operazioni:
	      \begin{itemize}
		      \item I 56 bit della chiave vengono divisi in due parti da 28 bit ciascuna e si applica, a ciascuna delle
		            due parti, uno shift ciclico di 1 o 2 bit a seconda della fase in cui ci si trova.

		            Procedimento necessario affinch\'e vengano usati tutti i bit della chiave (\emph{confusione}).
		      \item Si estraggono 48 bit dai due blocchi di 28 bit del punto precedente, i quali andranno a formare la
		            sottochiave di fase.
		      \item Riconcateniamo le due sequenze shiftate che andranno poi a comporre la chiave per la fase
		            successiva.
	      \end{itemize}
	      I due blocchi del messaggio vengono trattati in questo modo:
	      \begin{itemize}
		      \item Si mandano i 32 bit di destra (input) nella parte di sinistra (output)
		            \[ S[i] = D[i-1] \]
		      \item Vengono copiati 16 bit della parte di destra, andando cos\`i a produrre un blocco da 48 bit.
		      \item Si fa lo XOR bit a bit tra il blocco appena prodotto e la sottochiave di fase.
		      \item I blocchi di 48 bit vengono riportati a 32 bit grazie alla \textbf{S-box} (approfondimento pi\`u
		            avanti).
		      \item Si permutano i bit prodotti al passo precedente.
		      \item Si fa lo XOR bit a bit tra il blocco appena prodotto e la parte sinistra in input, andando cos\`i
		            a comporre il nuovo blocco di destra.
	      \end{itemize}
	\item Parte destra e parte sinistra vengono unite di nuovo.
	\item Si permutano i bit del blocco ottenuto (blocco PF).
\end{enumerate}

\subsubsection{S-Box}
La \textbf{S-Box} \`e una funzione composta da 8 sotto-funzioni, ciascuna che prende in input 6 bit e ne restituisce
4.

Per farlo si prendono il primo e l'ultimo bit in input e se ne ricava un indice di riga, mentre con i rimanenti bit
si ricava un indice di colonna.

Tramite questi due indici si ottiene un valore presente in una tabella, le cui righe contengono ognuna una permutazione
dei primi 16 interi. Il valore identificato dai due indici \`e resituito in output di 4 bit.

\subsection{Sicurezza}\label{sicurezza_DES}
Un cifrario ha una sicurezza di $b$ bit se il costo del miglior attacco \`e di ordine $O(2^b)$ operazioni e richiede di
esplorare uno spazio delle chiavi di cardinalit\`a $2^b$.

Nel caso del DES abbiamo chiavi da 56 bit ma lo spazio delle chiavi ha cardinalit\`a $2^{55}$ dato che, se cifriamo
il complemento del messaggio col complemento della chiave, otteniamo il complemento del crittogramma. I bit di
sicurezza non sono quindi 56 ma 55.

In sostanza escludere una chiave ci permette di escludere anche il suo complemento.

\subsection{Attacchi}\label{attacchi_DES}
Il DES, per quanto complesso, si \`e rivelato vulnerabile a diversi attacchi di diversa natura.

\subsubsection{Attachi distribuiti}
Uno degli attacchi di cui il DES \`e stato vittima \`e quello di tipo \textbf{distribuito}, ossia, un attacco a
forza bruta, distribuito su pi\`u macchine. Con questo tipo di attacco si \`e riusciti a forzare il cifrario in tempi
sempre pi\`u brevi.

\subsubsection{Chosen plain text}
\begin{enumerate}
	\item Si prende un messaggio $m$ e lo si cifra in $c_1$.
	\item Si cifra $\overline{m}$, ossia il complemento di $m$, in $c_2$.
	\item Per ogni chiave $k$ si prova a cifrare $m$ con $k$.
	      \begin{itemize}
		      \item Se si ottiene $c_1$ molto probabilmente $k$ \`e la chiave (non sicuramente dato che potrebbero
		            esserci altre chiavi che mappano $m$ in $c_1$).
		      \item Se la cifratura ha invece prodotto $\overline{c_2}$ allora \`e probabile che $\overline{k}$ sia
		            la chiave.

		            Questo perch\'e provando a cifrare il complemento del messaggio col complemento della chiave si
		            ottiene il complemento del crittogramma. Nel nostro caso
		            \[ C(\overline{m}, \overline{k}) = \overline{\overline{c_2}} = c_2 \]
		      \item Altrimenti n\'e $k$ n\'e $\overline{k}$ sono le chiavi ma con una sola cifratura vengono scartate
		            due chiavi.
	      \end{itemize}
\end{enumerate}

\subsubsection{Crittoanalisi differenziale}
Un altro attacco di tipo \emph{chosen plain text} si basa sulla \textbf{crittoanalisi differenziale}, la quale necessita
di almeno $2^{47}$ coppie $\langle m, c \rangle$ per funzionare e sfrutta l'analisi probabilistica per stimare quale
chiave \`e stata usata andando a cercare variazioni nei vari crittogrammi.

Il costo di questo attacco \`e tuttavia dell'ordine di $O(2^{55})$ operazioni per via delle 16 fasi del cifrario, le
quali, rendono l'attacco leggermente pi\`u dispendioso del forza bruta.

\subsubsection{Crittoanalisi lineare}
L'ultima tecnica di attacco che vediamo \`e basata sulla \textbf{crittoanalisi lineare}. \`E un attacco di tipo
\emph{know plain text} e serve a stimare alcuni bit della chiave.

Per effettuare l'attacco si necessita di $2^{43}$ coppie $\langle m, c \rangle$ ed \`e meno costosa del forza bruta.

\subsection{Miglioramenti}\label{miglioramenti_DES}
Visti i problemi del DES e le sue, ormai note vulnerabilit\`a, si \`e provato a migliorarlo apportando qualche modifica.

\subsubsection{Chiavi}
Si \`e provato a cambiare sempre le chiavi di fase, arrivando ad avere 768 bit di chiave complessivi. In realt\`a,
per attacchi basati su crittoanalisi differenziale, i bit di sicurezza sono 61, aggiungendo cos\`i solo 6 bit di
sicurezza al fronte di una chiave molto pi\`u lunga.

\subsubsection{Cifratura doppia}
L'approccio che invece \`e stato adottato \`e stata la \textbf{cifratura multipla}, ossia, la composizione del DES con
se stesso. Scelte due chiavi $k_1$ e $k_2$ qualsiasi, vale che
\[ C(C(m, k_1), k_2) \neq C(m, k_3) \]
per qualunque chiave $k_3$ nello spazio delle chiavi. In questo modo otteniamo chiavi di 112 bit ma con 57 bit di
sicurezza.

Ci\`o che riduce molto i bit di sicurezza sono gli attacchi di tipo \textbf{meet in the middle}: data una coppia
$\langle m, c \rangle$
\begin{enumerate}
	\item Per ogni $k_1$ si calcola e si salva in una tabella
	      \[ C(m, k_1) \]
	\item Per ogni $k_2$ si calcola e si cerca nella tabella
	      \[ D(c, k_2) \]
	\item Se troviamo una corrispondenza $k_1$ e $k_2$ probabilmente sono le chiavi.
\end{enumerate}
L'attacco si basa sul fatto che se il crittogramma $c$ \`e generato da
\[ C(C(m, k_1), k_2) \]
allora vale
\[ D(c, k_2) = C(m, k_1) \]
Quello che di fatto andiamo a fare \`e enumerare tutte le chiavi due volte (non tutte le coppie di chiavi) e poi
cerchiamo una corrispondenza.

Se le chiavi sono $2^{56}$ basta moltiplicare per 2 e otteniamo cos\`i $2^{57}$ operazioni per forzare il cifrario al
fronte di una chiave lunga 112 bit.

\subsubsection{Cifratura tripla}
Per giungere ad una sicurezza significativa si \`e arrivati a comporre il DES con se stesso tre volte. Si parla di
3TDEA, nel caso si utilizzino tre chiavi
\[ c = C(D(C(m, k_1), k_2), k_3) \]
o di 2TDEA, nel caso si utilizzino due chiavi
\[ c = C(D(C(m, k_1), k_2), k_1) \]
\textbf{NOTA}: usare la funzione di decifrazione tra le due cifrature non aumenta n\'e diminuisce la sicurezza, \`e
solo per rendere il sistema retrocompatibile con l'applicazione singola del DES.

Con questo nuovo metodo andiamo ad ottenere, in entrambi i casi, 112 bit di sicurezza ma, come vedremo fra poco, la
versione a due chiavi si rivela pi\`u conveniente.

Un attacco di tipo \emph{meet in the middle} sul 3TDEA sfrutta la relazione
\[ C(D(C(m, k_1), k_2), k_3) \]
la quale pu\`o essere riscritta come
\[ D(c, k_3) = D(C(m, k_1), k_2) \]
e se cifriamo con chiave $k_2$ entrambi i membri otteniamo
\[ C(D(c, k_3), k_2) = C(m, k_2) \]
Sapendo questo e data una coppia $\langle m, c \rangle$, l'attacco si compone delle seguenti fasi
\begin{enumerate}
	\item Si enumerano tutte le $2^{56}$ possibili chiavi $k_1$ e si calcola
	      \[ C(m, k_1) \]
	      salvando i risultati in una tabella.
	\item Si enumerano tutte le $2^{112}$ possibili coppie di chiavi $\langle k_2, k_3 \rangle$ e si calcola
	      \[ C(D(c, k_3), k_2) \]
	\item Se si trova una corrispondeza, allora le chiavi $k_1$, $k_2$ e $k_3$, molto probabilmente, sono le chiavi
	      usate.
\end{enumerate}
L'attacco ha quindi un costo complessivo di $O(2^{112})$ operazioni mentre un attacco di tipo forza bruta ne richiede
$O(2^{168})$.

La versione 2TDEA ha comunque 112 bit di sicurezza ma utilizza solo due chiavi da 56 bit ciascuna e dunque, una chiave
complessiva di 112 bit cos\`i da avere tutti i bit della chiave come bit di sicurezza.

\section{AES}\label{AES}
Si tratta di un cifrario che fa uso di chiavi brevi (128, 192 o 256 bit) e ripetute, le quali devono essere cambiate
per ogni nuova sessione di comunicazione.

Il messaggio \`e diviso in blocchi lunghi sempre 128 bit (a prescindere dalla lunghezza della chiave) ma cambia il
numero di fasi di cui si compone il processo di cifratura: 10 per chiavi da 128 bit, 12 per chiavi da 192 bit e 14
per chiavi da 256 bit.

\subsection{Gestore delle chiavi}
Per ora vediamo solamente la versione con 128 bit di chiave, la quale viene caricata per colonne in una matrice $W$ da
16 byte. Tale matrice viene ampliata aggiungendo ricorsivamente 40 colonne a partire dalle 4 iniziali per generare le
10 sottochiavi di fase secondo questa regola
\[
	W[i] = \begin{cases}
		W[i - 4] \oplus W[i - 1]    & \text{se } 4 \nmid i \\
		W[i - 4] \oplus T(W[i - 1]) & \text{se } 4 \mid i
	\end{cases}
\]
dove $T$ \`e una trasformazione \emph{non lineare} (S-Box) che rende il cifrario robusto ad attacchi basati su
\emph{crittoanalisi lineare}.

\subsection{Operazioni}
Dato che ogni blocco \`e di 128 bit, il cifrario lo organizza in una matrice $B$ di dimensione $4 \times 4$ da 16 byte.
Dopo di che iniziano le varie fasi, le quali si compongono di 4 operazioni principali.

\subsubsection{S-Box}
La \textbf{S-Box}, in questo caso, \`e una matrice di $16 \times 16$ byte, che contiene una permutazione di tutti i 256
interi rappresentabili con 8 bit.

Sia $B$ la matrice che contiene il blocco del messaggio e sia $b_{i, j}$ un byte in posizione $(i, j)$ di tale matrice,
la S-Box mappa $b_{i, j}$ in $a_{i, j}$ usando i primi 4 bit di $b_{i, j}$ per ricavare un indice $r$ di riga e gli
ultimi 4 per ricavare un indice $c$ di colonna.

Il valore $a_{i, j}$ restituito, \`e il valore presente nella S-box in posizione $(r, c)$ scritto con 8 bit.

\paragraph{Formulazione matematica}
La S-box calcola l'\textbf{inverso moltiplicativo} di ogni byte $b_{i, j}$, considerando il byte come un elemento del
campo \emph{finito} GF($2^8$). Questa operazione \`e ci\`o che rende la funzione $T$, che abbiamo citato in precedenza,
\emph{non lineare}.

Ogni byte $b_{i, j}$ viene prima sostituito con il suo inverso moltiplicativo in GF($2^8$) e poi moltiplicato per una
matrice di $8 \times 8$ bit sommato con un vettore colonna.

\subsubsection{Shift delle righe}
I byte di ogni riga vengono shiftati ciclicamente verso sinistra di 0, 1, 2, e 3 posizioni, rispettivamente. In questo
modo i 4 byte di ogni colonna si disperdono su 4 colonne diverse.

\subsubsection{Rimescolamento delle colonne}
Ogni colonna del blocco, trattata come un vettore di 4 elementi, viene moltiplicata per una matrice $M$ prefissata di
$4 \times 4$ byte.

La moltiplicazione \`e eseguita modulo $2^8$ e la somma modulo 2 (operazioni del campo GF($2^8$)).

Questo fa s\`i che ogni byte della colonna rimescolata dipenda da tutti i byte della colonna di partenza.

\subsubsection{Somma della chiave di fase}
L'ultima operazione di ogni fase \`e la somma della chiave di fase con lo XOR bit a bit.

Si utilizza la chiave di fase fornita dal gestore delle chiavi e si fa lo XOR bit a bit con il blocco in output
dall'operazione di rimescolamento delle colonne e ottengo cos\`i un nuovo blocco da dare in input alla fase successiva.

\subsection{Sicurezza}
Tutti i bit sono di sicurezza e, ad oggi, nessun attacco \`e stato in grado di compromettere AES anche nella sua
versione pi\`u semplice con chiave a 128 bit.

Esistono attacchi pi\`u efficienti di un attacco esauriente sulle chiavi per le versioni con 6 fasi, ma nessun attacco
\`e pi\`u efficiente se le fasi sono almeno 7.

Si conoscono attacchi \textbf{side-channel} che sfruttano debolezze della piattaforma sui cui esso \`e implementato.

\section{Cifrario a blocchi}
In un \textbf{cifrario a blocchi}, per trattare messaggi con lunghezza diversa da un multiplo della lunghezza del
messaggio, si fa del \textbf{padding}, ovvero si inseriscono bit casuali per riempire il blocco finale.

Cifrare a blocchi espone tuttavia la comunicazione ad attacchi dato che
\begin{itemize}
	\item Blocchi uguali nel messaggio producono blocchi uguali nel crittogramma (se cifrati con la stessa
	      chiave).
	\item C'\`e \textbf{poca diffusione} tra un blocco e l'altro.
	\item C'\`e \textbf{periodicit\`a} nel crittogramma.
\end{itemize}
Per risolvere il problema \`e necessario che a blocchi uguali del messaggio corrispondano blocchi diversi del
crittogramma. Per riuscire ad ottenere questo risultato, si fa ricorso ad una tecnica chiamata
\textbf{composizione di	blocchi}, in cui, la cifratura di un blocco, deve dipendere dalla cifratura dei precedenti.

\subsection{Cipher Block Chaining - CBC}
Per la cifratura del blocco $m_i$ si calcola
\[ c_i = C(m_i \oplus c_{i-1}, k) \]
dove $c_{i-1}$ \`e il blocco cifrato al passo precedente.

Al primo passo si deve usare una sequenza $c_0$ per effettuare la prima cifratura. Questa sequenza pu\`o essere
scambiata in chiaro e pu\`o essere una qualsiasi sequenza di 128 bit, a patto che sia cambiata per ogni blocco che
viene cifrato.

Per la decifrazione del blocco $c_i$ si calcola
\[ m_i = c_{i-1} \oplus D(c_i, k) \]
La fase di decifrazione pu\`o essere fatta in parallelo, a differenza della cifratura che deve essere sequenziale. 			% CIFRARI SIMMETRICI PER CRITTOGRAFIA DI MASSA
\chapter{Cifrari a chiave pubblica}
Nei cifrari simmetrici si ha un grosso problema, ovvero, lo \textbf{scambio della chiave}, che, come sappiamo, deve
essere la stessa per entrambi gli utenti.

Fino ad ora abbiamo sempre assunto che i due utenti fossero gi\`a in possesso della chiave e abbiamo solo parlato del
metodo di cifratura e decifrazione. Come abbiamo visto i messaggi scambiati sono cifrati e dunque la comunicazione \`e
sicura ma come avviene lo scambio della chiave ?
\begin{itemize}
	\item Se avviene di persona allora tanto vale scambiarsi direttamente il messaggio.
	\item Se avviene in chiaro non ha pi\`u senso cifrare il messaggio dato che chiunque potrebbe intercettare la
	      chiave e decifrarlo senza sforzo.
	\item Se inviassimo la chiave cifrata si innescherebbe lo stesso problema all'infinito: il destinatario avrebbe
	      bisogno della chiave di cifratura per decifrare e dunque si dovrebbe inviare un'altra chiave e cos\`i via.
\end{itemize}
I \textbf{cifrari a chiave pubblica} risolvono il problema.

\section{Soluzione ingenua}
Se abbiamo un sistema con $N$ utenti, ogni utente pu\`o memorizzare $N-1$ chiavi diverse e condivise con ciascun altro
utente.

In questo modo abbiamo un numero quadratico di chiavi nel numero di utenti del sistema.

Quando un utente $i$ vuole comunicare con l'utente $j$ manda un messaggio a $j$ cifrandolo con la chiave $k_j$.
L'utente $j$ a questo punto decifra il messaggio con la sua chiave $k_j$ e invia un messaggio a $i$ cifrandolo con
la chiave $k_i$.

\section{TTP}
Una soluzione migliore \`e rappresentata dal \textbf{TTP} o \textbf{trusted 3rd party}, ossia una terza parte
\emph{fidata}, a cui gli utenti si appoggiano per comunicare.

Ogni utente deve ricordarsi una sola chiave mentre TTP gestisce la creazione e lo scambio delle chiavi condivise tra i
due utenti.

Siano $A$ e $B$ i due utenti che vogliono comunicare, il processo di scambio funziona in questo modo:
\begin{enumerate}
	\item $A$ e $B$ possiedono rispettivamente $k_A$ e $k_B$, due chiavi generate da loro stessi.
	\item $A$ comunica a TTP di voler comunicare con $B$.
	\item TTP genera casualmente una chiave $k_{AB}$ che potranno usare i due utenti per quella comunicazione.
	\item TTP cifra genera due crittogrammi $c_A$ e $c_B$ in questo modo
	      \[
		      \begin{matrix}
			      c_A & = & C(k_{AB}, k_A) \\
			      c_B & = & C(k_{AB}, k_B)
		      \end{matrix}
	      \]
	\item TTP invia $c_A$ e $c_B$ ad $A$.
	\item $A$ decifra $c_A$ con $k_A$ e invia a $c_B$ a $B$.
	\item $B$ decifra $c_B$ con $k_B$.
\end{enumerate}
Alla fine di questo processo i due utenti sono in possesso di una chiave $k_{AB}$ che potranno usare per quella
comunicazione in un cifrario simmetrico.

Le problematiche di questo sistema sono due:
\begin{itemize}
	\item TTP deve essere sempre online.
	\item TTP conosce tutte le chiavi.
\end{itemize}
\`E un approccio utilizzabile solo in un sistema ristretto come un'universit\`a o un'azienda.

\section{Chiave pubblica}
In questo tipo di cifratura si vuole implementare un meccanismo che permette a chiunque di inviare messaggi cifrati a
un certo utente ma permettere solo a quell'utente di decifrarli.

Le operazioni di cifratura e decifrazione sono pubbliche e utilizzano due chiavi diverse:
\begin{itemize}
	\item \textbf{Chiave pubblica}: \`e nota a tutti.
	\item \textbf{Chiave private}: nota solo al destinatario.
\end{itemize}
Questa coppia di chiavi \`e generata dall'utente in veste di destinatario, il rende nota la sua chiave pubblica e
mantiene segreta la sua chiave privata.

Se volessimo inviare un messaggio all'utente $i$ si dovrebbe cifrare il messaggio con la sua chiave pubblica. L'utente
$i$ decifra il crittogramma con la sua chiave privata.

In questo sistema l'unica cosa non nota a tutti \`e la chiave privata del destinatario. Le funzioni di cifratura e
decifrazione e la chiave pubblica sono note a qualsiasi utente.

I cifrari a chiave pubblica hanno due principali vantaggi
\begin{itemize}
	\item Se ci sono $n$ utenti nel sistema, il numero complessivo di chiavi pubbliche e private \`e $2n$ anzich\'e
	      $\frac{n(n-1)}{2}$
	\item Non \`e richiesto alcuno scambio segreto di chiavi.
\end{itemize}
ma possiede anche altri due principali svantaggi
\begin{itemize}
	\item Sono molto pi\`u lenti dei cifrari simmetrici.
	\item Sono esposti ad attacchi di tipo \emph{chosen plain text}.
\end{itemize}

\subsection{Attacchi chosen plain text}
Un crittoanalista quando effettua un'attacco di questo tipo
\begin{enumerate}
	\item Si procura un po' di messaggi in chiaro.
	\item Li cifra con la funzione pubblica $C$ e con la chiave pubblica $k_{\text{pub}}$ del destinatario.
	\item Confronta infine i crittogrammi in suo possesso con i crittogrammi che passano sul canale.
\end{enumerate}
Un attacco di questo tipo \`e molto pericoloso nel caso in cui il crittoanalista sospetta che un certo messaggio debba
transitare sul canale.

Ecco perch\'e questi cifrari non sono usati per comunicare ma solo per scambiarsi la chiave di un cifrario simmetrico
(AES).

In questo modo risolviamo in un colpo solo tutti i problemi elencati in precedenza:
\begin{itemize}
	\item Il processo \`e "lento" solo per lo scambio della chiave (256 bit) ma la comunicazione viene fatta con un
	      cifrario simmetrico molto veloce.
	\item L'attacco di tipo \emph{chosen plain text} diventa inutile dato che la chiave \`e una sequenza di bit casuale.
\end{itemize}

\subsection{Requisiti}
Perch\'e la cifratura a chiave pubblica devono essere soddisfatti alcuni requisiti.
\begin{itemize}
	\item Il procedimento di cifratura e decifrazione devono essere implementati correttamente. Il destinatario deve
	      essere in grado di decifrare qualsiasi messaggio con la propria chiave privata.
	      \[ D(C(m, k_{\text{pub}}), k_{\text{priv}}) = m \]
	\item Efficienza e sicurezza del sistema:
	      \begin{itemize}
		      \item La coppia di chiavi \`e \emph{facile} da generare e deve risultare praticamente impossibile che due
		            utenti scelgano la stessa chiave.
		      \item Dati $m$ e $k_{\text{pub}}$ \`e \emph{facile} per il mittente produrre il crittogramma.
		      \item Dati $c$ e $k_{\text{priv}}$ \`e \emph{facile} per il destinatario produrre il messaggio originale.
		      \item Pur conoscendo la chiave pubblica e le funzioni di cifratura e decifrazione deve essere
		            \emph{difficile} per il crittoanalista risalire al messaggio in chiaro.
	      \end{itemize}
\end{itemize}
La soluzione risiede nel trovare una funzione di tipo \textbf{one-way trap-door}, ovvero, una funzione facile da
calcolare e difficile da invertire a meno che non si conosca la chiave privata.

\section{RSA}
Questo cifrario usa l'algebra modulare e si basa sulla moltiplicazione di due numeri primi $p$ e $q$ poich\'e calcolare
\[ n = p \cdot q \]
richiede tempo quadratico nella lunghezza della loro rappresentazione ma, ricostruire $p$ e $q$ a partire da $n$
richiede tempo esponenziale se $p$ e $q$ sono primi.

Se si conosce tuttavia uno dei due fattori, risalire all'altro \`e facile (basta fare una divisione).

\subsection{Generazione delle chiavi}
Per la generazione delle due chiavi, il destinatario deve
\begin{enumerate}
	\item Scegliere due numeri primi $p$ e $q$ molto grandi, dove per "molto grandi" intendiamo tali che $p \cdot q$
	      sia un numero di circa 2000 bit per una protezione fino al 2030 o di circa 3000 bit se vogliamo una
	      protezione oltre il 2030.

	      Per farlo dobbiamo generare numeri di circa 1500 bit ed effettuare il test di Miller-Rabin per la
	      primalit\`a finch\'e non otteniamo due numeri primi (tempo polinomiale).
	\item Calcolare
	      \[ n = p \cdot q \]
	      e la relativa funzione di Eulero
	      \[ \phi(n) = (p - 1)(q - 1) \]
	      il tutto in tempo polinomiale.
	\item Scegliere un intero $e$ tale che
	      \[ e < \phi(n) \quad \wedge \quad (e, \phi(n)) = 1 \]
	\item Calcolare
	      \[ d = e^{-1} \mod{\phi(n)} \]
	      ossia l'inverso di $e$ modulo $\phi(n)$.
\end{enumerate} 	% CIFRARI A CHIAVE PUBBLICA
\chapter{Crittografia su curve ellittiche}
La \textbf{crittografia su curve ellittiche} nasce per alleggerire il carico computazionale che si porta dietro
la crittografia a chiave pubblica basata sull'algebra modulare.

I problemi su cui si basa la crittografia a chiave pubblica, come la fattorizzazione e il calcolo del logaritmo
discreto hanno due problemi:
\begin{itemize}
	\item Sebbene si risolvano in tempo polinomiale sono comunque \textbf{calcoli molto pesanti}.
	\item Gli algoritmi odierni per il calcolo di questi due problemi non hanno pi\`u costo esponenziale ma,
	      come abbiamo gi\`a visto, \textbf{subesponenziale}. Ne \`e una diretta conseguenza l'aumento della
	      lunghezza delle chiavi.
\end{itemize}
Come vedremo, la crittografia su curve ellittiche, propone i cifrari visti in precedenza (protocollo DH e cifrario
di ElGamal) in un contesto matematico diverso e basati sul calcolo del logaritmo discreto e non pi\`u sulla
fattorizzazione.

Il risultato \`e stato quello di ottenere un costo del miglior attacco, come vedremo, \textbf{puramente esponenziale}.

\section{Curve ellittiche}
Prima di vedere i dettagli implementativi definiamo cosa sono le curve ellittiche.

Le curve ellittiche sono curve algebriche descritte da equazioni cubiche (simili a quelle utilizzate nel calcolo
della lunghezza degli archi delle ellissi).
\[ E(a, b) = \{ (x, y) \in \mathbb{R}^2 \mid y^2 = x^3 + ax + b \} \]
qui sono definite sui reali, nella cosiddetta \emph{forma normale di Weierstrass}, ma possiamo scrivere l'equazione
su un campo $\mathbb{K}$ qualisiasi.

L'insieme appena descritto contiene anche il cosiddetto \textbf{punto all'infinito} $O$ in direzione dell'asse $y$
(la curva ha un asintoto verticale), il quale rappresenta l'\textbf{elemento neutro per l'addizione}.

La curva si pu\`o presentare in due forme
\begin{itemize}
	\item A \textbf{due componenti} nel caso in cui la cubica abbia tre radici reali.
	\item Ad \textbf{una componente} nel caso in cui la cubica abbia una sola radice reale e due complesse
	      coniugate.
\end{itemize}
Se la curva ha una di queste due forme, \`e sempre possibile mandare una \textbf{tangente}. Questo per\`o non \`e
vero per tutti i tipi di curva.

Per le applicazioni crittografiche si assume infatti che il discriminante della cubica sia
\[ 4 a^3 + 27 b^2 \neq 0 \]
Questo ci assicura che
\begin{itemize}
	\item La cubica non abbia radici multiple.
	\item La curva sia priva di punti singolari come \emph{cuspidi} o \emph{nodi}, dove non sarebbe definita in
	      modo univoco la tangente.
\end{itemize}
Le curve ellittiche rappresentano anche una \textbf{simmetri orizzontale}, ossia rispetto all'asse delle ascisse.
Questo ci permette di definire sempre, per ogni punto della curva, il suo \textbf{opposto} sempre sulla curva.

Preso il punto $p = \langle x, y \rangle$, il suo opposto ha coordinate $-p = \langle x, -y \rangle$.

\subsection{Somma di punti}
Prima di tutto notiamo che le curve ellittiche possiedono una propriet\`a: ogni retta interseca una curva ellittica
in al pi\`u tre punti.
\[
	\begin{cases}
		y = mx + q \\
		y^2 = x^3 + ax + b
	\end{cases}
\]
Sviluppando il sistema si ottiene una cubica in $x$ che ha al massimo tre radici. Le radici in questione possono
essere o tutte reali o una reale e due complesse coniugate.

Quindi se una retta interseca la curva nei punti $P$ e $Q$, coincidenti se la retta \`e una tangente, allora la
retta interseca la curva anche in un terzo punti $R$.

Dati tre punti $P, Q, R$ su una curva, se sono anche tutti su una retta, vale
\[ P + Q + R = O \]
Da questa regola si ricava la regola per la somma di due punti $P$ e $Q$.
\begin{enumerate}
	\item Si scrive l'equazione della retta che passa per $P$ e per $Q$.
	\item Dato che $P$ e $Q$ sono sulla curva e su una retta, esiste un terzo punto $R$.
	\item L'opposto di $R$ \`e il punto somma di $P$ e $Q$.
\end{enumerate}
Per sommare un punto $P$ con se stesso
\begin{enumerate}
	\item Si manda la tangente alla curva nel punto $P$.
	\item Si trova il punto $R$ di intersezione.
	\item Il punto $-R$ equivale al punto $2P$.
\end{enumerate}
Dati due punti $P = (x_P, y_P)$ e $Q = (x_Q, y_Q)$, sulla curva $E(a, b)$, si distinguono tre casi:
\begin{itemize}
	\item Se $P \neq \pm Q$ allora $S = P + Q$ con
	      \[ \begin{matrix}
			      x_S = \lambda^2 - x_P - x_Q \\
			      y_S = -y_P + \lambda (x_P - x_S)
		      \end{matrix}
	      \]
	      dove
	      \[ \lambda = \frac{y_Q - y_P}{x_Q - x_P} \]
	      \`e il coefficiente angolare della retta per $P$ e $Q$.
	\item Se $P = Q$ allora $S = P + Q = 2P$ con
	      \[
		      \begin{matrix}
			      x_S = \lambda^2 - x_P - x_Q \\
			      y_S = -y_P + \lambda (x_P - x_S)
		      \end{matrix}
	      \]
	      dove
	      \[ \lambda = \frac{3_P^2 + a}{2yp} \]
	      \`e il coefficiente angolare della tangente alla curva in $P$ ($y_P \neq 0$).
	      Se $y_P = 0$ allora $S = 2P = O$.
	\item Se $Q = -P$ allora $S = P + Q = P + (-P) = O$.
\end{itemize}
I punti delle curve ellittiche con discriminante diverso da zero hanno la struttura algebrica di gruppo abeliano,
e gode dunque di alcune propriet\`a:
\begin{itemize}
	\item \textbf{Chiusura}: $\forall P, Q \in E(a, b)$ vale
	      \[ P + Q \in E(a, b) \]
	\item \textbf{Elemento neutro}: $\forall P \in E(a, b)$ vale
	      \[ P + O = O + P = P \]
	\item \textbf{Inverso}: $\forall P \in E(a, b)$ esiste un unico $Q = -P \in E(a, b)$ tale che
	      \[ P + Q = Q + P = O \]
	\item \textbf{Associativit\`a}: $\forall P, Q, R \in E(a, b)$ vale
	      \[ P + (Q + R) = (P + Q) + R \]
	\item \textbf{Commutativit\`a}: $\forall P, Q \in E(a, b)$ vale
	      \[ P + Q = Q + P \]
\end{itemize}

\section{Curve ellittiche su campi finiti}
Ora che abbiamo definito, in generale, cosa sono e come funzionano le curve ellittiche sui reali passiamo a trattare
le curve ellittiche su campi finiti, che sono quelle che poi servono di pi\`u all'atto pratico.

I campi finiti possono essere di due tipi fondamentalmente: quelli che hanno come cardinalit\`a un numero primo e
quelli che hanno come cardinalit\`a la potenza di un numero primo.

\subsection{Curve ellittiche prime}
Le \textbf{curve ellittiche prime} sono curve i cui coefficienti $a, b$ sono presi in $\mathbb{Z}_p$ con $p > 3$
un numero primo e l'equazione che le descrive \`e
\[ E_p(a, b) = 	\{ (x, y) \in \mathbb{Z}_p^2 \mid y^2 \mod{p} = (x^3 + ax + b) \mod{p} \} \cup \{ O \} \]
Una curva ellittica prima contiene un numero \textbf{finito} di punti e non \`e pi\`u rappresentata da una curva
continua nel piano.

Una curva ellittica prima risulta simmetrica rispetto alla retta
\[ y = \frac{p}{2} \]
questo perch\'e si lavora sempre tra 0 e $p - 1$.

L'inverso di un punto $P = (x, y) \in E_p(a, b)$ \`e definito come
\[ -P = (x, -y) = (x, p - y) \in E_p(a, b) \]

Se il polinomio $x^3 + ax + b \mod{p}$ non ha radici multiple, ovvero se il discriminante della cubica \`e diverso
da zero, allora i punti della cruva $E_p(a, b)$ formano un \textbf{gruppo abeliano additivo finito}.

\begin{example}
	Prendiamo la curva $E_5(4, 4)$. Affinch\'e i punti abbiamo la struttura di campo abeliano dobbiamo verificare
	che il discriminante sia diverso da 0.
	\[ 4 \cdot 4^3 + 27 \cdot 4^2 \mod{5} \neq 0 \]
	sviluppando un po' i calcoli si ottiene
	\[ 3 \mod{5} = 3 \neq 0 \]
	quindi i punti sulla curva hanno la struttura di campo abeliano. Ora vogliamo trovare quanti punti ha questa
	curva, per farlo dobbiamo considerare l'equazione
	\[ y^2 \equiv x^3 + 4x + 4 \mod{5} \]
	Dato che stiamo lavorando modulo 5, per trovare tutti i punti della curva, possiamo far variare l'ascissa con
	tutti i valori da 0 a 4 e valutare la cubica.

	In questa fase si deve prestare attenzione al fatto che un punto sull'ascissa non necessariamente identifica
	un punto sulla cubica. Questo perch\'e, lavorando in modulo $p$, non tutti i numeri modulo $p$ hanno una radice
	(solo la met\`a).

	Per vederlo facciamo un semplice procedimento
	\begin{center}
		\begin{tabular}{ c | c c c c c }
			$y$           & 0 & 1 & 2 & 3 & 4 \\
			\hline
			$y^2 \mod{p}$ & 0 & 1 & 4 & 4 & 1
		\end{tabular}
	\end{center}
	da qui vediamo subito che solo 1 e 4 hanno una radice nel campo.

	Valutiamo ora tutti i punti
	\begin{center}
		\begin{tabular}{ c | c | c }
			$x$ & $y^2$ & $y$                    \\
			\hline
			0   & 4     & $\langle 2, 3 \rangle$ \\
			1   & 4     & $\langle 2, 3 \rangle$ \\
			2   & 0     & 0                      \\
			3   & 3     & /                      \\
			4   & 4     & $\langle 2, 3 \rangle$
		\end{tabular}
	\end{center}
	La curva contiene dunque 7 punti e il punto all'infinito, abbiamo dunque una curva di \textbf{ordine} 8.
\end{example}

\begin{theorem}[Teorema di Hasse]
	Se $n$ \`e l'ordine della curva prima $E_p(a, b)$ allora vale che
	\[ |n - (p + 1)| \leq 2 \sqrt{p} \]
\end{theorem}

\section{Funzioni one-way trap-door}
Il metodo di costruzione di funzioni one-way trap-door su curve ellittiche \`e molto simile a quello usato per
il protocollo DH e per il cifrario di ElGamal su algebra modulare.

Questo perch\'e l'addizione di punti su curve ellittiche si pu\`o mettere in corrispondenza con la moltiplicazione
di interi modulo $p$.

Si introduce anche la \textbf{moltiplicazione scalare} di punti sulla curva che altro non \`e che una
generalizzazione dell'addizione nell'algebra modulare. Questa operazione coinvolge un punto $P \in E(a, b)$ e un
intero $k$ e si pu\`o mettere in corrispondenza con l'elevamento a potenza. Come vedremo $k P$ si calcola
in tempo polinomiale con $\Theta(\log k)$ addizioni e raddoppi di punti.

\subsection{Algoritmo dei raddoppi ripetuti}
Questo algoritmo, come vedremo, si pu\`o mettere in corrispondenza con l'algoritmo di esponenziazione veloce usato
nell'algebra modulare e consiste nel fare $\lfloor \log k \rfloor$ raddoppi e $O(\log k)$ somme di punti.

Sia $P$ un punto in $E_p(a, b)$ e sia $k$ un intero
\begin{enumerate}
	\item Si scompone come somma di potenze di 2
	      \[ k = \sum_{i=0}^{\lfloor \log k \rfloor} k_i 2^i \quad \quad k_i \in \{ 0, 1 \} \]
	\item Si calcolano $\lfloor \log k \rfloor$ punti, ognuno come raddoppio del precedente.
	      \[ 2P, \quad 2^2P, \quad 2^3 P, \quad \dots, \quad 2^{\lfloor\log k \rfloor} \]
	\item Si calcola $Q = kP$ calcolando
	      \[ \sum_{i \mid k_i = 1} 2^i P \]
	      in al pi\`u $\log k$ addizioni.
\end{enumerate}
Se notiamo, l'algoritmo calcola $kP$ con $\Theta(\log k)$ operazioni.

\begin{example}
	Supponiamo di voler calcolare $13P$. Non possiamo permetterci di fare 13 somme, scriviamo quindi il 13 come
	somma di potenze di 2
	\[ 13P = (8 + 4 + 1) P = 8P + 4P + P \]
	A questo punto, avendo $P$ possiamo calcolare tanti raddoppi quanti sono necessari
	\[
		\begin{matrix}
			P \rightarrow 2P          \\
			2P \rightarrow 2(2P) = 4P \\
			4P \rightarrow 2(4P) = 8P
		\end{matrix}
	\]
	Andiamo ora a prendere solo i termini che ci servono, ossia $P$, $4P$ e $8P$ e li sommiamo.

	Abbiamo cos\`i calcolato $13P$ con 3 raddoppi e 2 somme di punti.
\end{example}

\subsubsection{Inversione della funzione}
Invertire questa funzione \`e simile ad invertire il logaritmo discreto. Dati $Q$ e $P$, trovare, se esiste, il
pi\`u piccolo intero tale che
\[ Q = kP \]
\`e un problema \emph{difficile}. Se $k$ soddisfa l'equazione scritta sopra, allora $k$ si chiama il
\textbf{logaritmo discreto su curve ellittiche}
\[ k = \log_P Q \]
Il problema ha costo esponenziale dato che conosciamo solo algoritmi di tipo forza bruta che provano tutti i
possibili valori di $k$.

Quello che si fa all'atto pratico \`e una serie di somme di $P$ con se stesso finch\'e non troviamo $k$ tale che
\[ kP = Q \]

A differenza dell'algebra modulare, per cui si conoscono algoritmi di costo subesponenziale per il calcolo del
logaritmo discreto, per le curve ellittiche, conosciamo solo algoritmi di costo esponenziale.

\section{Protocollo DH su curve ellittiche}
Il protocollo DH, dato che basa la sua sicurezza sul calcolo del logaritmo discreto, pu\`o essere rivisitato in
una sua versione su curve ellittiche.

La conversione dall'algebra modulare si ottiene molto semplicemente, sostituendo la moltiplicazione di interi in
modulo con la somma di punti.

Prima di vedere come funziona il protocollo dobbiamo definire un'altra cosa ossia l'\textbf{ordine di un punto}.
L'ordine di un punto \`e il pi\`u piccolo intero $n$ tale che il punto per questo intero da come risultato il
punto all'infinito.
\[ nP = O \]

Supponiamo che i due utenti $i$ e $j$ vogliano instaurare un protocollo di comunicazione
\begin{enumerate}
	\item $i$ e $j$ scelgono una curva ellittica $E_p(a, b)$ e un punto $B \in E_p(a, b)$ speciale, detto
	      \textbf{punto base}, il quale deve avere ordine $n$ molto grande.
	\item $i$ e $j$ scelgono rispettivamente $n_i$ e $n_j$ casuali tali che
	      \[ n_i, n_j < n \]
	\item $i$ e $j$ calcolano rispettivamente
	      \[
		      \begin{matrix}
			      P_i = n_i B \\
			      P_j = n_j B
		      \end{matrix}
	      \]
	      con l'algoritmo dei raddoppi ripetuti.
	\item $i$ e $j$ inviano rispettivamente $P_i$ e $P_j$ sul canale.
	\item $i$ e $j$ calcolano rispettivamente
	      \[
		      \begin{matrix}
			      k & = & n_i P_j & = & n_i n_j B \\
			      k & = & n_j P_i & = & n_j n_i B
		      \end{matrix}
	      \]
\end{enumerate}

\subsection{Attacchi}
Per quanto riguarda gli attacchi passivi, per risalire a $n_i$ e $n_j$ si deve calcolare il logaritmo discreto
su curve ellittiche, che, come sappiamo, richiede costo esponenziale sapendo solo $E_p(a, b)$, $P_i$, $P_j$ e $B$.

Il protocollo \`e comunque vulnerabile ad attacchi di tipo \emph{man in the middle} a meno che non si usino curve
estratte da certificati digitali.

\section{Protocollo di ElGamal su curve ellittiche}
A differenza dei protoccolli visti fino ad ora, su curve ellittiche e non, il \textbf{protocollo di ElGamal su curve
	ellittiche} \`e usato come un cifrario a chiave pubblica per lo scambio di messaggi e non \`e dunque ristretto
al solo scambio della chiave.

Cifrare un messaggio, in questo ambito, significa \emph{incapsulare} il messaggio in uno dei punti delle curve
ellittiche. Il problema \`e che non disponiamo di algoritmi deterministici, che in tempo polinomiale, mappano
un messaggio su uno dei punti della curva.

Un'idea potrebbe essere quella di porre $x = m$ nella cubica della curva e prendere il relativo punto, calcolato
tramite l'equazione.

Il problema \`e che per trovare la coordinata $y$ del punto dobbiamo fare la radice del valore calcolato dalla
cubica, ma, come abbiamo visto, non sempre questo \`e possibile.

\subsection{Algoritmo di Koblitz}
Sia $m$ il messaggio che vogliamo inviare e sia $p$ il numero primo con cui stiamo lavorando,l'\textbf{algoritmo
	di Koblitz} funziona in questo modo
\begin{enumerate}
	\item Si sceglie un intero $h$ tale che
	      \[ (m + 1) \cdot h < p \]
	\item Si pone
	      \[ x = m \cdot h + i \]
	      nella cubica, con $0 \leq i < h-1$.
	\item Si fanno $h$ tentativi.
\end{enumerate}
La probabilit\`a di fallire \`e di circa $1/2$ ogni volta, che si traduce in una probabilit\`a di circa $(1/2)^h$
di fallimento totale. Se $h$ \`e sufficientemente grande, la probabilit\`a di fallire \`e molto bassa.

\begin{lstlisting}[style=pseudo-style]
Koblitz(m, h, a, b, p)
	for i = 0 to h-1
		x = m * h + i;
		z = (pow(x, 3) + a * x + b) % p;
		if z.hasRadix() then 
			y = sqrt(z);
			return (x, y);

	return FAIL;
\end{lstlisting}
Il destinatario, in fase di decifrazione, ottiene il punto della curva di coordinate $\langle x, y \rangle$. Per
estrarre il messaggio calcola
\[
	m = \left\lfloor \frac{x}{h} \right\rfloor  =
	\left\lfloor \frac{mh + i}{h} \right\rfloor =
	\left\lfloor m + \frac{i}{h} \right\rfloor
	\quad \quad \text{con } \frac{i}{h} < 1
\]
Quello che si fa in pratica \`e dividere l'ascissa del punto per $h$.

\subsection{Scambio di messaggi}
Scelta una curva $E_p(a, b)$, un punto base $B \in E_p(a, b)$ di ordine $n$ elevato, ogni utente costruisce la sua
coppia $\langle k_\text{pub}, k_\text{priv} \rangle$.

L'utente $u$ forma la coppia di chiavi calcolandole in questo modo:
\[
	\begin{matrix}
		k_\text{pub} = n_u B = P_u \\
		k_\text{priv} = n_u < n
	\end{matrix}
\]
Se due utenti, $i$ e $j$, volessero comunicare, il procedimento sarebbe il seguente
\begin{enumerate}
	\item $i$ incapsula il messaggio $m$ in un punto $M$ della curva con l'algoritmo di Koblitz.
	\item $i$ sceglie $r < n$ casuale e da usare una volta sola.
	\item $i$ calcola un punto
	      \[ V = r \cdot B \]
	      della curva con l'algoritmo dei raddoppi ripetuti.
	\item $i$ cifra il messaggio con la chiave pubblica $P_j$ del destinatario calcolando
	      \[ W = M + r \cdot P_j \]
	      dove $W$ \`e un punto della curva.
	\item $i$ invia la coppia di punti $\langle V, W \rangle$ a $j$.
	\item $j$ riceve $\langle V, W \rangle$.
	\item $j$ calcola il punto
	      \[ M = W - n_j \cdot V \]
	      sulla curva dove risiede il messaggio.
	\item $j$ calcola
	      \[ m = \left\lfloor \frac{x}{h} \right\rfloor \]
\end{enumerate}

\subsubsection{Correttezza}
Per convincerci che ci\`o che stiamo facendo facciamo qualche passo indietro.
\begin{itemize}
	\item Il punto $M$ si trova calcolando
	      \[ M = W - n_j \cdot V \]
	\item Il punto $W$ si ottiene calcolando
	      \[ W = M + r \cdot P_j \]
	      dove $P_j$ \`e la chiave pubblica di $j$.
	\item La chiave pubblica $P_j$ si ottiene calcolando
	      \[ n_j \cdot B \]
	\item Il punto $V$ si ottiene calcolando
	      \[ V = r \cdot B \]
\end{itemize}
Mettendo tutto insieme si ricava che
\[ W - n_j \cdot V \quad = \quad M + r (n_j \cdot B) - n_j (r \cdot B) \quad = \quad M \]

\subsection{Attacchi}
Essendo un sistema a chiave pubblica, \`e vulnerabile ad attacchi di tipo \emph{man in the middle} a meno che non
si estraggano le chiavi pubbliche da certificati digitali.

Gli attacchi passivi possono essere di due tipi
\begin{itemize}
	\item \textbf{Calcolo della chiave privata}: richiede il calcolo del logaritmo discreto su curve ellittiche
	      (costo esponenziale).
	\item \textbf{Attacco su r}: nel caso il parametro $r$ venga riusato il crittoanalista pu\`o scoprirlo e
	      decifrare facilmente il crittogramma.
\end{itemize}

\subsubsection{Attacco su r}
Conoscere $r$ ci permette di decifrare il crittogramma in tempo polinomiale perch\'e
\[ W = M + r \cdot P_j \]
quindi, se conosciamo $r$, possiamo calcolare
\[ M = W - r \cdot P_j \]
Estrarre $r$, calcolandolo con i dati pubblici forniti da mittente e destinatario \`e tuttavia un problema difficile
quanto il calcolo del logaritmo discreto.

Si dovrebbe sfruttare altre vulnerabilit\`a nel sistema come ad esempio capire quale sia il generatore di numeri
casuali utilizzato oppure sfruttare il fatto che $r$ possa essere utilizzato pi\`u volte.

\section{Sicurezza delle curve ellittiche}
La sicurezza della crittografia su curve ellittiche si basa sulla difficolt\`a nel calcolare il logaritmo discreto
si un punto della curva.

Nonostante non esista una dimostrazione formale di \emph{intrattabilit\`a} questo problema \`e considerato molto
\emph{difficile}.

In particolare, \`e un problema molto pi\`u difficile della fattorizzazione e del calcolo del logaritmo discreto
nell'algebra modulare, problemi per i quali esiste un algoritmo (Index Calculus) di costo
\[ O(2^{\sqrt{b \log b}}) \]
quindi subesponenziale, dove $b$ sono i bit del modulo su cui stiamo lavorando.

Questo algoritmo sfrutta il fatto che gli interi in modulo hanno la struttura algebrica di \textbf{campo} e su cui
\`e dunque definita l'operazione di moltiplicazione.

I punti di una curva ellittica hanno invece la struttura algebrica di \textbf{gruppo abeliano}, che ha propriet\`a
ben pi\`u deboli di un campo e dove non \`e definita la moltiplicazione. Questo non ha dunque permesso di
rimodellare l'Index Calculus su curve ellittiche.

Esiste un metodo pi\`u efficiente del forza bruta, l'algoritmo di Pollard, di costo
\[ O(2^{b/2}) \]
e quindi puramente \textbf{esponenziale}.

La crittografia su curve ellittiche ha quindi il grosso vantaggio di avere una sicurezza di $b/2$ bit dove $b$ \`e
la lunghezza della chiave.

Se per esempio sull'RSA, per avere 256 bit di sicurezza, necessitavamo di una chiave di pi\`u di 15.000 bit, per le
curve ellittiche ne basta una lunga 512. 	% CIFRARI SU CURVE ELLITTICHE
\chapter{Protocolli}
In questo capitolo andremo a trattare i protocolli di sicurezza tra i quali distingueremo
\begin{itemize}
	\item \textbf{Identificazione}: un sistema di elaborazione ha bisogno di accertarsi dell'\textbf{identit\`a}
	      di un utente che vuole accedere ai suoi servizi.
	\item \textbf{Autenticazione}: il destinatario di un messaggio deve essere in grado di accertare
	      l'\textbf{identit\`a del mittente} e l'\textbf{integrit\`a del crittogramma} ricevuto.
	\item \textbf{Firma digitale}: la firma digitale si pone tre obbiettivi:
	      \begin{itemize}
		      \item Il mittente non deve poter \textbf{negare l'invio} di un messaggio.
		      \item Il destinatario deve essere in grado di \textbf{autenticare} il messaggio.
		      \item Il destinatario non deve poter sostenere di aver ricevuto un messaggio diverso da quello inviato
		            dal mittente.
	      \end{itemize}
\end{itemize}
Come si pu\`o notare, ognuna di queste funzionalit\`a, estende l'altra
\begin{itemize}
	\item L'autenticazione garantisce l'identificazione.
	\item La firma digitale garantisce l'autenticazione.
\end{itemize}
Ognuna di queste funzionalit\`a ha il compito di proteggere le comunicazioni da attacchi attivi, come per esempio
gli attacchi \emph{man in the middle}.

\section{Funzioni hash}
Per l'implementazione di queste funzionalit\`a faremo ricorso alla \textbf{funzioni hash}. Una funzione hash
\[ f : X \rightarrow Y \]
\`e una funzione tale che
\[ n = |X| >> m = |Y| \]
ossia, \`e tale se, definito il dominio $X$, il codominio $Y$ della funzione \`e molto pi\`u piccolo.

Inoltre, una funzione hash, ha molti elementi del dominio che vengono mappati nella stessa immagine del codominio.
Questo ci permette di \emph{partizionare} il dominio in sottoinsiemi
\[ X = X_1 \cup X_2 \cup \dots \cup X_m \]
tali che ogni elemento del sottoinsieme \`e mappato, dalla funzione hash, nella stessa immagine del codominio
\[ \forall i, \quad \forall x \in X_i \quad \quad f(x) = y \]
Una funzione hash \`e buona se la cardinalit\`a di ogni sottoinsieme sia circa la stessa.

\subsection{Funzioni hash one-way}
Le funzioni hash usate in crittografia devono soddisfare tre requisiti principali
\begin{itemize}
	\item Per ogni $x \in X$ deve essere \emph{facile} calcolare
	      \[ y = f(x) \]
	\item \textbf{One-way}: per la maggior parte degli $y \in Y$ deve essere \emph{difficile} determinare $x \in X$
	      tale che
	      \[ f(x) = y \]
	\item \textbf{Claw-free}: deve essere \emph{difficile} determinare una coppia $x_1, x_2 \in X$ tali che
	      \[ f(x_1, x_2) \]
\end{itemize}

\subsection{Funzione hash SHA-1}
Una delle funzioni, storicamente usata in crittografia, \`e \textbf{SHA-1}, la quale opera su sequenze lunghe fino
a $2^{64} - 1$ bit e produce immagini di 160 bit.

In particolare opera su blocchi di 160 bit, contenuti in un buffer di 5 registri da 32 bit ciascuno, in cui sono
caricati inizialmente dei valori pubblici.

Il messaggio viene concatenato con una sequenza di \emph{padding} che rende la sua lunghezza un multiplo di 512 bit.

Il contenuto dei registri varia nel corso dei cicli successivi in cui questi valori si combinano tra loro e con
blocchi di 32 bit provenienti dal messaggio.

La funzione sfrutta shift ciclici e una componente non lineare che varia per riuscire ad ottenere il valore hash per
ciascun messaggio in input.

\section{Identificazione}
Vediamo ora come viene applicato il protocollo di \textbf{identificazione} sia nel caso in cui ci trovassimo su un
canale sicuro sia nel caso in cui ci trovassimo su un canale insicuro.

\subsection{Canale sicuro}
Prendiamo per esempio la situazione in cui un utente voglia accedere ai propri file personali memorizzati su un
calcolatore ad accesso riservato ai membri della sua organizzazione.
\begin{enumerate}
	\item L'utente invia in chiaro nome utente e password.
	\item Dato che il canale \`e sicuro, un attacco pu\`o essere sferrato solo da un utente locale al sistema o da
	      un hacker.
\end{enumerate}
Il meccanismo di identifiaczione dovrebbe per\`o prevedere una \textbf{cifratura} della password tramite funzioni hash
one-way anche su canali sicuri.

Dobbiamo distinguere due casi: quando l'utente si registra e quando l'utente effettua tutti gli accessi successivi.

\subsubsection{Registrazione}
Possiamo associare la fase di registrazione alla fase di cifratura con una funzione hash one-way $h$:
\begin{enumerate}
	\item L'utente $u$ si registra fornendo per la prima volta la password $p$.
	\item Il sistema associa a $p$ due sequenze binarie che memorizza nel file delle password al posto di $p$.
	      \begin{itemize}
		      \item Un \textbf{seme} casuale $s$ prodotto da un generatore.
		      \item Il \textbf{valore hash} della concatenazione tra $p$ ed $s$.
		            \[ q = h(p s) \]
	      \end{itemize}
\end{enumerate}
Quello che quindi viene salvato dal sistema \`e l'\textbf{immagine hash} della password concatenata ad un seme casuale
e non la password in chiaro.

\subsubsection{Accesso}
In fase di accesso il sistema compie i seguenti passaggi
\begin{enumerate}
	\item Recupera $s$ dal file delle password.
	\item Concatena $s$ con la password $p$ fornita da $u$.
	\item Calcola il valore hash della sequenza $p s$.
	\item Se $q = h(p s)$ l'identificazione ha successo.
\end{enumerate}
Avere accesso al file delle password non fornisce informazioni interessanti perch\'e \`e computazionalmente difficile
ricavare la password dalla sua immagine hash.

\subsection{Canale insicuro}
Nel caso di canale insicuro non si pu\`o inviare la password in chiaro e dunque andremo a lavorare con sistemi a
chiave pubblica per l'invio della password.

In realt\`a un sistema non dovrebbe mai poter maneggiare password in chiaro ma solo una loro immagine hash.

Supponiamo che l'utente $u$ voglia accedere ai servizi forniti da un certo sistema $s$ e per farlo \`e necessario
che si identifichi.

Supponiamo che il sistema adotti un cifrario a chiave pubblica (per esempio l'RSA) per lo scambio sicuro dei dati
dell'utente. L'utente dispone quindi di una chiave pubblica $\langle e, n \rangle$ e di una chiave privata $d$.

Quello che avviene all'atto pratico \`e questo:
\begin{enumerate}
	\item $s$ genera un numero casuale $r < n$ e lo invia in chiaro a $u$.
	\item $u$ calcola
	      \[ f = r^d \mod{n} \]
	      che rappresenta la \textbf{firma} di $u$ su $s$ e lo invia a $s$.
	\item $s$ verifica che
	      \[ f^e \mod{n} = r \]
	      se l'uguaglianza \`e soddisfatta, l'identificazione ha successo.
\end{enumerate}
Come possiamo notare, le operazioni di cifratura e decifrazione sono invertite rispetto all'impiego standard dell'RSA
ed \`e possibile farlo dato che sono commutative
\[ (x^e \mod{n})^d \mod{n} = (x^d \mod{n})^e \mod{n} \]
Chiariamo inoltre che solo $u$ pu\`o produrre $f$ dato che \`e l'unico che possiede il valore $d$.

Se il passo 3 va a buon fine, il sistema ha la garanzia che l'utente che ha richiesto l'identificazione sia
effettivamente $u$, anche se il canale \`e insicuro.

Il protocollo funziona bene a patto che il sistema sia onesto. Dato che \`e il sistema che fornisce $r$, se $r$ \`e
effettivamente un valore casuale allora tutto va a buon fine, se invece $r$ \`e un dato con particolari propriet\`a
utili a ricavare la chiave privata di $u$.

\subsection{Protocollo a conoscenza zero}
Questo protocollo permette ad un utente di dimostrare la sua identit\`a o la sua conoscenza di un certo segreto,
convincendo un altro utente di essere in possesso di una certa capacit\`a, senza rivelare niente oltre alla
veridicit\`a di questa sua capacit\`a.

In questo nuovo paradigma abbiamo due utenti:
\begin{itemize}
	\item \textbf{Prover}: indicato con $P$, ha il compito di dimostrare la sua capacit\`a o conoscenza.
	\item \textbf{Verifier}: indicato con $V$, ha il compito di verificare la veridicit\`a di ci\`o che afferma $P$.
\end{itemize}
Supponiamo quindi che $P$ voglia dimostrare a $V$ di possedere una certa capacit\`a, un protocollo \`e il seguente
\begin{enumerate}
	\item $V$ mette alla prova $P$ chiedendogli di dimostrare la sua capacit\`a su un problema che lui gli pone.
	\item $P$ risponde.
	\item Se la risposta \`e corretta, $V$ genera un valore casuale e modifica il problema in base al valore generato.
	\item $V$ chiede a $P$ di risolvere il problema modificato.
	\item Se $P$ risponde bene, allora $V$ continua a modificare il problema e a chiedere a $P$ di risolverlo, fino ad
	      essere sicuro che $P$ possegga effettivamente la capacit\`a da lui dichiarata.
	\item Se $P$ sbaglia anche solo una risposta, allora $V$ pu\`o immediatamente dedurre che $P$ dica il falso.
\end{enumerate}
Se le sfide proposte da $V$ sono $k$, la probabilit\`a che $P$ stia dicendo il falso, \`e
\[ \left( \frac{1}{2} \right)^k \]
Possiamo quindi concludere che, pi\`u $k$ \`e alto, meno sono le probabilit\`a che $P$ stia dicendo il falso.

\subsubsection{Propriet\`a fondamentali}
Di seguito le propriet\`a fondamentali del protocollo:
\begin{itemize}
	\item \textbf{Completezza}: se $P$ \`e onesto, $V$ accetta sempre la dimostrazione.
	\item \textbf{Correttezza}: se $P$ \`e disonesto, la probabilit\`a che $P$ riesca ad ingannare $V$ \`e al pi\`u
	      $(1/2)^k$ con $k$ scelto da $V$.
	\item \textbf{Conoscenza zero}: se $P$ \`e onesto, un verificatore (anche disonesto), non pu\`o acquisire nessuna
	      informazione se non la veridicit\`a dell'affermazione.
\end{itemize}

\subsection{Protocollo di Fiat-Shamir}
Questa non \`e altro che un'applicazione del protocollo a conoscenza zero con chiavi pubbliche e private.

Il \emph{prover} \`e impersonato dall'utente che vuole dimostrare di essere il legittimo proprietario di una chiave
privata associata ad una certa chiave pubblica senza usarla su dati scelti dall'utente \emph{verifier}.

Il protocollo si basa sulla difficolt\`a del calcolo di una radice in modulo un numero composto.

L'utente $P$, in fase di preparazione
\begin{enumerate}
	\item Sceglie due numeri primi $p$ e $q$ molto grandi.
	\item Calcola $n = pq$.
	\item Sceglie una sorta di chiave privata $s < n$.
	\item Calcola
	      \[ t = s^2 \mod{n} \]
	\item Rende nota la coppia $\langle t, n \rangle$ e mantiene privata la tripla $\langle p, q, s \rangle$.
\end{enumerate}
A questo punto $P$ vuole dimostrare a $V$ di conoscere una radice di $t$, ovvero $s$ senza per\`o inviargliela.

Per $k$ volte, con $k$ scelto da $V$, si ripetono i seguenti passi
\begin{enumerate}
	\item $V$ chiede a $P$ di iniziare una iterazione.
	\item $P$ genera un numero casuale $r < n$, calcola
	      \[ u = r^2 \mod{n} \]
	      e lo invia a $V$.
	\item $V$ genera un bit casuale $e$ e lo invia a $P$.
	\item $P$ calcola
	      \[ z = r \cdot s^e \mod{n} \]
	      e lo invia a $V$.
	      \begin{itemize}
		      \item Se $e = 0$ allora $z = r \mod{n}$.
		      \item Se $e = 1$ allora $z = r \cdot s \mod{n}$
	      \end{itemize}
	\item $V$ calcola
	      \[ x = z^2 \mod{n} \]
	      e controlla se
	      \[ x = u \cdot t^e \mod{n} \]
	      Se l'uguaglianza \`e vera si torna al passo 1, altrimenti $P$ non \`e identificato.
\end{enumerate}

\subsubsection{Completezza}
In questo caso, se $P$ \`e davvero in possesso di una radice di $t$, il verificatore lo identifica.
\begin{itemize}
	\item Se $e = 0$ allora
	      \[ x \quad = \quad u \cdot t^e \mod{n} \quad = \quad u \mod{n} \]
	\item Se $e = 1$ allora
	      \[ x \quad = \quad z^2 \mod{n} \quad = \quad (r \cdot s^e)^2 \mod{n} \quad = \quad u \cdot t \mod{n} \]
\end{itemize}
Quindi $P$ supera tutte le iterazioni se conosce $s$.

\subsubsection{Correttezza}
Supponiamo che $P$ sia disonesto e che quindi non conosca $s$. Per ingannare $V$ deve riuscire a prevedere il bit
casuale generato da $V$ ad ogni iterazione.

Distinguiamo due casi:
\begin{itemize}
	\item Se $P$ prevede di ricevere $e = 0$ non modifica il protocollo e se la previsione \`e corretta supera il
	      test.
	\item Se $P$ prevede di ricevere $e = 1$
	      \begin{enumerate}
		      \item Al passo 2 del protocollo e invia
		            \[ r^2 \cdot t^{-1} \mod{n} \]
		      \item Al passo 4 del protocollo invia
		            \[ z = r \mod{n} \]
	      \end{enumerate}
	      Se al passo 5 la previsione \`e corretta, $P$ supera il test perch\'e $V$ controlla se
	      \[ x = z^2 = u \cdot t^e \]
	      e se $e = 1$ allora
	      \[ u \cdot t^e = u \cdot t \]
	      Dato che $z = r$ e che
	      \[ u = r^2 \cdot t^{-1} \]
	      otteniamo
	      \[ r^2 = r^2 \cdot t^{-1} \cdot t \]
	      e quindi
	      \[ r^2 = r^2 \]
\end{itemize}
Come possiamo vedere, il metodo funziona a patto che la previsione sul bit sia corretta e la previsione deve essere
fatta prima di ricevere $e$.

La probabilit\`a di prevedere il bit ad ogni passo \`e quindi di un $1/2$. Per $k$ passi abbiamo quindi una
probabilit\`a di $(1/2)^k$ di prevedere tutti i bit.

\section{Autenticazione}
L'autenticazione riguarda il messaggio e si occupa di accertare l'identit\`a del mittente e l'integrit\`a del
messaggio.

\subsection{Canale insicuro}
Per questo protocollo, su canale insicuro, useremo un sistema basato su cifrari simmetrici in cui mittente e
destinatario devono quindi accordarsi su una chiave segreta $k$.

Nella pratica il mittente
\begin{enumerate}
	\item Allega al messaggio un \textbf{MAC} (Message Authentication Code) $A(m, k)$, allo scopo di garantire la
	      provenienza e l'integrit\`a del messaggio.
	\item A questo punto ha due opzioni
	      \begin{itemize}
		      \item Invia la coppia $\langle m, A(m, k) \rangle$ in chiaro.
		      \item Cifra $m$ e spedisce $\langle C(m, k'), A(m, k) \rangle$ dove $C$ \`e una funzione di cifratura
		            e $k'$ la chiave pubblica o la chiave simmetrica segreta del cifrario scelto.
	      \end{itemize}
\end{enumerate}

Il destinatario invece
\begin{enumerate}
	\item Riceve il messaggio (se cifrato lo decifra).
	\item Dato che conosce $A$ e $k$ calcola a sua volta il MAC.
	\item Confronta il MAC calcolato con il MAC ricevuto.
\end{enumerate}
Se la verifica ha successo il messaggio \`e autenticato altrimenti lo si scarta.

\subsubsection{MAC}
Il MAC \`e un'immagine breve del messaggio che pu\`o essere generata solo da un mittente conosciuto dal destinatario
e pu\`o essere calcolato con cifrari asimettrici, simmetrici o funzioni hash one-way.

Quest'ultima opzione implementativa \`e la pi\`u frequente
\[ A(m, k) = h(m k) \]
dato che il calcolo di una funzione hash \`e molto veloce per chi sta cifrando ma computazionalmente molto dispendioso
per un crittoanalista, che, anche disponendo di $h$ e $m$ non \`e comunque in grado di risalire a $k$ in tempo
polinomiale dato che $k$ viaggia all'interno di un MAC e quindi si dovrebbe invertire la funzione hash (costo
esponenziale).

Il crittoanalista non pu\`o nemmeno sostituire (facilmente) il messaggio $m$ con un altro messaggio $m'$ perch\'e
dovrebbe allegare a $m'$ il MAC $A(m', k)$ che pu\`o produrre solo conoscendo $k$.

\section{Firma digitale}
Questo protocollo cerca di portare tutte le propriet\`a di una \textbf{firma manuale} (con carta e penna) in ambito
tecnologico. Una firma manuale infatti
\begin{itemize}
	\item \`E autentica e non falsificabile.
	\item Non \`e riutilizzabile.
	\item Non pu\`o essere ripudiata.
\end{itemize}
Anche il documento firmato deve essere \textbf{inalterabile}.

Come vedremo, una \textbf{firma digitale}
\begin{itemize}
	\item Non pu\`o consistere nella digitalizzazione di un documento scritto firmato manualmente perch\'e si
	      potrebbe facilmente contraffare.
	\item Deve avere una forma che dipenda dal documento su cui viene apposta, per essere inscindibile da
	      quest'ultimo.
	\item Pu\`o essere generata sia tramite cifrari simmetrici che asimmetrici.
\end{itemize}

\subsection{Protocollo 1: Diffie Hellman}
In questo protocollo il messaggio $m$ \`e in chiaro e firmato.

Supponiamo che un utente $u$, in possesso di una coppia di chiavi $\langle k_\text{pub}, k_\text{priv} \rangle$ e
che ha a disposizione una funzione $C$ di cifratura e una funzione $D$ di decifrazione (commutative), voglia firmare
$m$ e inviarlo a $v$.

Per firmare $m$, l'utente $u$
\begin{enumerate}
	\item Genera la firma $f$ per $m$ calcolando
	      \[ f = D(m, k_\text{priv}) \]
	\item $u$ invia all'utente $v$ la tripla $\langle u, m, f \rangle$.
\end{enumerate}
L'utente $v$
\begin{enumerate}
	\item Riceve $\langle u, m, f \rangle$.
	\item Verifica l'autenticit\`a della firma calcolando e controllando che
	      \[ m = C(f, k_\text{pub}) \]
\end{enumerate}
Se la verifica va a buon fine allora $v$ accetta la firma.

\subsubsection{Limiti}
Questo protocollo ha il grosso limite di non riuscire a proteggere il messaggio in lettura, infatti anche se
inviassimo un crittogramma $c$ di $m$, sarebbe il risultato della cifratura di $m$ con la chiave pubblica.

Possiede tuttavia tutti i requisiti di una firma manuale
\begin{itemize}
	\item La chiave $k_\text{priv}$ \`e nota solo a $u$ e per ottenerla si fa un numero esponenziale di operazioni.
	\item Se $m$ venisse alterato dal crittoanalista non ci sarebbe pi\`u consistenza tra $m$ e $f$ e la verifica
	      fallirebbe.
	\item Poich\'e solo $u$ pu\`o aver prodotto $f$ non pu\`o ripudiarla.
	\item Dato che la firma \`e un'immagine di $m$ non \`e riutilizzabile su un altro messaggio $m'$.
\end{itemize}

\subsection{Protocollo 2}
Questo protocollo si propone di risolvere il problema del precedente relativo all'impossibilit\`a di proteggere il
messaggio.

L'utente $u$
\begin{enumerate}
	\item Genera la firma $f$ per $m$ calcolando
	      \[ f = D(m, k_\text{priv}) \]
	\item Cifra la firma calcolando
	      \[ c = C(f, k) \]
	      dove $k$ pu\`o essere la chiave pubblica del destinatario oppure una chiave simmetrica segreta.
	\item Invia la coppia $\langle u, c \rangle$ a $v$.
\end{enumerate}
L'utente $v$
\begin{enumerate}
	\item Riceve $\langle u, c \rangle$.
	\item Ricava $f$ calcolando
	      \[ f = D(c, k) \]
	      con $k$ che pu\`o essere la sua chiave privata o una chiave simmetrica.
	\item Cifra $f$ con la chiave pubblica del mittente ottenendo il messaggio $m$
	      \[ m = C(f, k_\text{pub}) \]
\end{enumerate}
Se il messaggio \`e significativo l'identit\`a di $u$ \`e attestata altrimenti si butta via il messaggio.

\subsubsection{Algoritmo con RSA}
In questo caso abbiamo due coppie di chiavi
\begin{gather*}
	d_u, \quad \langle e_u, n_u \rangle \\
	d_v, \quad \langle e_v, n_v \rangle
\end{gather*}
La coppia del mittente \`e usata per produrre la firma e verificarla mentre la coppia del destinatario per decifrare
il crittogramma.

Supponiamo che $u$ sia il mittente e $v$ il destinatario, l'utente $u$
\begin{enumerate}
	\item Genera la firma del messaggio $m$ calcolando
	      \[ f = m^{d_u} \mod{n_u} \]
	\item Cifra $f$ con la chiave pubblica di $v$ calcolando
	      \[ c = f^{e_v} \mod{n_v} \]
	\item Invia la coppia $\langle u, c \rangle$ a $v$.
\end{enumerate}
L'utente $v$
\begin{enumerate}
	\item Riceve la coppia $\langle u, c \rangle$.
	\item Decifra $c$ calcolando
	      \[ f = c^{d_v} \mod{n_v} \]
	\item Decifra $f$ con la chiave pubblica di $u$
	      \[ m = f^{e_u} \mod{n_u} \]
\end{enumerate}
Se $m$ \`e significativo allora l'identit\`a del mittente \`e attestata.

Affinch\'e il procedimento venga effettuato correttamente \`e necessario che
\[ f < n_v \]
e perch\'e questo accada c'\`e bisogno che
\[ n_u \leq n_v \]
Questo impedisce a $v$ di inviare messaggi firmati e cifrati da $u$.

Di solito ogni utente ha due coppie di chiavi diverse, una per la firma e una per la cifratura, tali che le chiavi
per la firma siano per esempio minori di un certo valore $H$ e quelle di cifratura siano invece maggiori. Il valore
$H$ \`e un valore molto grande su cui i due utenti si devono accordare.

\subsubsection{Attacco}
Un crittoanalista potrebbe procurarsi la firma di un utente su messaggi apparentemente privi di senso.

Prendiamo uno scenario in cui un crittoanalista $x$ si procura la firma dell'utente da messaggi privi di senso
per l'utente.

Supponiamo che
\begin{itemize}
	\item Il destinatario $v$ di un messaggio invii sempre una risposta $ack$ al mittente $u$ ogni volta che
	      riceve un messaggio (prima della verifica della firma).
	\item Il segnale di $ack$ sia il crittogramma della firma di $v$ su $m$.
\end{itemize}
In queste condizioni, un attacco attivo, pu\`o avere successo se
\begin{enumerate}
	\item $x$ intercetta il crittogramma $c$ firmato inviato da $u$ a $v$, lo rimuove dal canale e lo rispedisce a
	      $v$, facendogli credere che $c$ sia stato inviato da lui.
	\item $v$ spedisce automaticamente a $x$ un ack.
	\item $x$ usa l'ack ricevuto per risalire al messaggio originale applicando le funzioni del cifrario con le
	      chiavi pubbliche di $u$ e $v$.
\end{enumerate}
Per risalire a $m$, il crittoanalista compie dei passaggi algebrici che avranno complessivamente costo polinomiale.

Prima di tutto, il fatto che $u$ abbia inviato il crittogramma $c$ a $v$, significa che
\begin{gather*}
	c = C(f, k_{v [\text{pub}]}) \\
	f = D(m, k_{u [\text{priv}]})
\end{gather*}
A questo punto, dopo che $x$ ha intercettato $c$ e l'ha rispedito a $v$, l'utente $v$ decifra $c$ ottenendo
\[ f = D(c, k_{v [\text{priv}]}) \]
e verifica la firma con la chiave pubblica di $x$ ottenendo un messaggio
\[ m' = C(f, k_{x [\text{pub}]}) \]
Il messaggio $m'$, molto probabilmente, non \`e significativo ma l'ack $c'$ \`e gi\`a stato inviato in automatico
calcolando
\begin{gather*}
	f' = D(m', k_{v [\text{priv}]}) \\
	c' = C(f', k_{x [\text{pub}]})
\end{gather*}
A questo punto $x$ ha tutto ci\`o che serve:
\begin{enumerate}
	\item Decifra $c'$ e trova $f'$
	      \[ f' = D(c', k_{x [\text{priv}]}) = D(C(f', k_{x [\text{pub}]}), k_{x [\text{priv}]}) \]
	\item Verifica la firma $f'$ e trova $m'$ usando la chiave pubblica di $v$
	      \[ m' = C(f', k_{v [\text{pub}]}) = C(D(m', k_{v [\text{priv}]}), k_{v [\text{pub}]}) \]
	\item Da $m'$ ricava la firma $f$
	      \[ f = D(m', k_{x [\text{priv}]}) = D(C(f, k_{x [\text{pub}]}), k_{x [\text{priv}]}) \]
	\item Verifica $f$ con la chiave pubblica di $u$ e trova $m$
	      \[ m = C(f, k_{u [\text{pub}]}) = C(D(m, k_{u [\text{priv}]}), k_{u [\text{pub}]}) \]
\end{enumerate}
Come possiamo vedere si usano sempre le funzioni di cifratura e decifrazione che, come sappiamo, hanno sempre costo
polinomiale e quindi anche l'attacco ha costo complessivamente polinomiale.

Per proteggersi da questo attacco non si devono inviare ack automatici, almeno finch\'e non si \`e concluso la fase
di verifica e si deve sempre firmare un'immagine hash del messaggio.

\subsection{Protocollo 3}
Questo protocollo si propone di risolvere anche i problemi presentati del secondo. In questo caso il messaggio
$m$ \`e cifrato e firmato con una funzione hash.

Il mittente $u$
\begin{enumerate}
	\item Calcola l'hash del messaggio $h(m)$ e genera la firma calcolando
	      \[ f = D(h(m), k_{u [\text{priv}]}) \]
	\item Cifra il messaggio calcolando
	      \[ c = C(m, k_{v [\text{pub}]}) \]
	\item Invia la tripla $\langle u, c, f \rangle$ a $v$.
\end{enumerate}
Il destinatario $v$
\begin{enumerate}
	\item Riceve la tripla $\langle u, c, f \rangle$.
	\item Decifra $c$ calcolando
	      \[ m = D(c, k_{v [\text{priv}]}) \]
	\item Calcola $h(m)$.
	\item Verifica la firma calcolando e verificando che
	      \[ h(m) = C(f, k_{u [\text{pub}]}) \]
\end{enumerate}
Come per tutti i protocolli a chiave pubblica, anche questo \`e vulnerabile ad attacchi di tipo
\emph{man in the middle}.

\subsection{Certification Authority}
Un algoritmo \`e tanto robusto quanto la sicurezza delle sue chiavi ma lo scambio o la generazione della chiave \`e
un passo cruciale.

\`E proprio in questo frangente che i crittoanalisti sfruttano attacchi di tipo \emph{man in the middle} per
riuscire a forzare facilmente i sistemi crittografici.

Sono dunque nate delle infrastrutture, chiamate \textbf{certification authority}, adibite a garantire la validit\`a
delle chiavi pubbliche e a regolare il loro uso, gestendone la distribuzione.

Le CA rilasciano un \textbf{certificato digitale} che autentica l'associazione tra un utente e la sua chiave pubblica.

Un certificato digitale consiste della chiave pubblica di una lista di informazioni relative al suo proprietario,
opportunamente firmate dalla CA. Per falsificare un certificato si deve falsificare la firma delle CA.

Una CA mantiene un archivio di chiavi pubbliche sicuro, accessibile a tutti e protetto da attacchi in scrittura non
autorizzati.

La chiave pubblica della CA \`e nota a tutti gli utenti che la mantengono protetta sui propri dispositivi e la
utilizzano per verificare la firma della CA stessa sui certificati.

Le CA hanno in genere una struttura gerarchica ad albero e dunque si avvia una sorta di verifica a catena risalendo
le varie CA.

\subsubsection{Certificazione}
Supponiamo che $u$ voglia comunicare con $v$
\begin{enumerate}
	\item $u$ richiede la chiave pubblica di $v$ alla CA.
	\item La CA invia a $u$ il certificato digitale $c_v$ di $v$.
	\item Dato che $u$ conosce la chiave pubblica della CA, controlla l'autenticit\`a del certificato verificandone il
	      periodo di validit\`a e la firma della CA.
	\item $u$ estrae dal certificato la chiave pubblica di $v$ e inizia il protocollo di comunicazione.
\end{enumerate}
Gli attacchi \emph{man in the middle} sono sempre possibili ma devono essere effettuati falsificando la certificazione
ma si assume che la CA sia fidata e il suo archivio di chiavi inattaccabile.

\subsection{Protocollo 4}
Ultimo protocollo che vediamo consiste nel cifrare, firmare e certificare un messaggio $m$.

Il mittente $u$
\begin{enumerate}
	\item Si procura il certificato $\text{cert}_v$ di $v$ e verifica che sia autentico.
	\item Calcola $h(m)$ e genera la firma calcolando
	      \[ f = D(h(m), k_{u [\text{priv}]}) \]
	\item Cifra $m$ calcolando
	      \[ c = C(m, k_{v [\text{pub}]}) \]
	\item Invia la tripla $\langle \text{cert}_u, c, f \rangle$ a $v$.
\end{enumerate}
Il destinatario $v$
\begin{enumerate}
	\item Riceve la tripla $\langle \text{cert}_u, c, f \rangle$ a $v$.
	\item Verifica l'autenticit\`a di $\text{cert}_u$ con la chiave pubblica della CA che tiene in locale.
	\item Decifra $c$ con la sua chiave privata calcolando
	      \[ m = D(c, k_{v [\text{priv}]}) \]
	\item Verifica che la firma sia autentica controllando che
	      \[ h(m) = C(f, k_{u [\text{pub}]}) \]
\end{enumerate}
L'unico punto debole di questo metodo \`e rappresentato dall'uso di certificati revocati.

\section{Protocollo SSL}
\`E un protocollo molto usato nelle comunicazioni sicure e garantisce \textbf{confidenzialit\`a} e
\textbf{affidabilit\`a} nelle comunicazioni su internet ed \`e progettato per permettere a due computer che non
conoscono le reciproche funzionalit\`a di comunicare.

Supponiamo che un utente $u$ voglia accedere ad un servizio fornito da un sistema $s$. Il protocollo SSL garantisce
\begin{itemize}
	\item \textbf{Confidenzialit\`a}: la trasmissione \`e cifrata mediante un sistema ibrido in cui si usa un
	      cifrario asimmetrico per costruire le chiavi e uno simmetrico per la comunicazione.
	\item \textbf{Autenticazione}: il protocollo accerta l'identit\`a dei due utenti tramite un cifrario asimmetrico
	      o facendo uso di certificati digitali e inserendo un MAC nei messaggi.
\end{itemize}
L'SSL sta tra il protocollo di trasporto (TCP/IP) e il protocollo applicativo (HTTP) ed \`e completamente
indipendente da quest'ultimo.

\subsection{Livelli}
Il protocollo \`e organizzato su due livelli: \textbf{record} e \textbf{handshake}.

\subsubsection{SSL Record}
Il livello \emph{record} \`e a livello pi\`u basso ed \`e connesso direttamente al protocollo di trasporto.

Ha come obbiettivo incapsulare i dati spediti dai protocolli dei livelli superiori, assicurando confidenzialit\`a
e integrit\`a della comunicazione.

Implementa fisicamente il canale su cui viaggiano i messaggi.

\subsubsection{SSL Handshake}
Il livello \emph{handshake} \`e responsabile dell'instaurazione e del mantenimento dei parametri usati dal livello
\emph{record} e permette al sistema di
\begin{itemize}
	\item Autenticarsi
	\item Negoziare gli algoritmi di cifratura e firma
	\item Stabilire le chiavi per i singoli algoritmi crittografici e per il MAC
\end{itemize}
In definitiva il livello \emph{handshake} crea un canale \textbf{sicuro}, \textbf{affidabile} e \textbf{autenticato}
tra utente e sistema, entro il quale il livello \emph{record} fa viaggiare i messaggi.

Affinch\'e avvenga l'\emph{handshake} deve esserci uno scambio preliminare di messaggi indispensabile alla creazione
di un canale sicuro. Attraverso questi messaggi, client e server si identificano a vicenda e cooperano per la
costruzione delle chiavi simmetriche usate per le comunicazioni successive.

\subsection{Creazione del canale}
Vediamo ora tutti i passi che compie il protocollo per la costruzione del canale sicuro.

\begin{enumerate}
	\item \textbf{Client hello}: $u$ invia a $s$ un messaggio di \emph{client hello} con cui
	      \begin{itemize}
		      \item Richiede la creazione di una connessione SSL
		      \item Specifica le prestazioni di sicurezza che desidera siano garantite durante la comunicazione
		            (\textbf{cipher suite}).
		      \item Invia una sequenza di byte casuali.
	      \end{itemize}
	\item \textbf{Server hello}: $s$ riceve il messaggio di \emph{client hello} e manda un messaggio di
	      \emph{server hello} con cui
	      \begin{itemize}
		      \item Specifica una \emph{cipher suite} che anch'esso \`e in grado di supportare.
		      \item Invia una sequenza di byte casuali.
	      \end{itemize}
	      Se $u$ non riceve il \emph{server hello} interrompe la comunicazione.
	\item \textbf{Invio del messaggio}: $s$ si autentica con $u$ inviandogli il proprio certificato digitale e se
	      i servizi offerti da $s$ devono essere protetti negli accessi, $s$ pu\`o richiedere a $u$ di autenticarsi
	      inviando il suo certificato digitale.
	\item \textbf{Server hello done}: $s$ invia il messaggio \emph{server hello done} con cui sancisce la fine degli
	      accordi sulla \emph{cipher suite} e sui parametri crittografici a essa associati.
	\item \textbf{Autenticazione del sistema}: per accerta l'autenticit\`a del certificato ricevuto da $s$, $u$ deve
	      controllare che
	      \begin{itemize}
		      \item Il certificato sia ancora valido.
		      \item La CA che ha firmato il certificato sia tra quelle \emph{fidate}.
		      \item La firma apposta sul certificato sia autentica.
	      \end{itemize}
	\item \textbf{User master secret}: $u$ a questo punto
	      \begin{itemize}
		      \item Costruisce un \textbf{pre-master secret} costituito da una nuova sequenza di byte casuali.
		      \item Lo cifra con il cifrario a chiave pubblica su cui si \`e accordato con $s$.
		      \item Invia il crittogramma a $s$.
	      \end{itemize}
	      Il pre-master secret viene poi usato per calcolare un \textbf{master secret}.
	\item \textbf{System master secret}: $s$ riceve il crittogramma contenente il \emph{pre-master secret} inviato
	      da $u$ e calcola il \emph{master secret} compiendo le stesse operazioni compiute da $u$.
	\item \textbf{Invio del certificato}: questo passo \`e opzionale e si fa solo nel caso in cui $s$ richieda un
	      certificato al client.
	\item \textbf{Finished}: \`E il primo messaggio protetto mediante il \emph{master secret} e la \emph{cipher suite}
	      su cui le due parti si sono accordate.

	      Il messaggio viene prima costruito da $u$ e spedito a $s$ e dopo avviene il contrario ma il messaggio \`e
	      diverso.
\end{enumerate}
Se tutto questo processo \`e andato a buon fine si possono costruire le chiavi crittografiche simmetriche in modo
sicuro.

\subsection{Sicurezza}
Nei passi di \emph{hello} i due utenti si inviano due sequenze casuali per la costruzione del \emph{master secret},
che risulta cos\`i, diverso in ogni sessione di SSL.

Un crittoanalista non pu\`o riutilizzare i messaggi di \emph{handshake} catturati sul canale per sostituirsi a $s$
in una successiva comunicazione con $u$ perch\'e questi sono riferiti a valori \emph{usa e getta}.

I blocchi vengono cifrati con un MAC (anch'esso cifrato) e quindi un crittoanalista dovrebbe riuscire ad invertire
una funzione hash.

Un attacco volto a modificare la comunicazione tra i due utenti \`e difficile quanto un attacco volto alla decriptazione
dei messaggi.

Il sistema \`e autenticato con un certificato ed \`e dunque robusto ad attacchi \emph{man in the middle}.

La sicurezza del protocollo SSL \`e sicuro almeno quanto il pi\`u debole \emph{cipher suite} supportato. 		% PROTOCOLLI DI SICUREZZA
\chapter{Bitcoin}
Le normali valute come euro, dollari e cos\`i via sono gestite da una banca centrale, come la BCE per l'euro, la quale
gestisce il conio e altri parametri andando a regolare l'inflazione ecc.

Senza entrare nei dettagli finanziari, possiamo dunque dire che le valute comuni sono in mano a sistemi
\textbf{centralizzati}, i quali gestiscono tutto ci\`o che abbiamo detto prima.

Il sistema dei \textbf{bitcoin} nasce dall'esigenza di creare un sistema di valute digitali \textbf{decentralizzato},
ossia un sistema che \`e gestito solo dai suoi utenti.

\section{Introduzione}
Prima di addentrarci nell'argomento chiariamo cosa sia una \emph{valuta digitale} e una \emph{criptovaluta}:
\begin{itemize}
	\item \textbf{Valuta digitale}: valuta che esiste soltanto in forma digitale e che dunque, \`e utilizzabile
	      soltanto da un calcolatore.
	\item \textbf{Criptovaluta}: valuta digitale resa sicura grazie a tecniche di crittografia che rendono quasi
	      impossibile spendere due volte la stessa moneta.
\end{itemize}
Per capire il funzionamento di bitcoin dobbiamo prima introdurre due concetti: la transazione e il registro.
\begin{itemize}
	\item \textbf{Transazione}: passaggio di denaro fra due utenti.
	\item \textbf{Registro}: anche detto \textbf{libro contabile}, \`e un documento per il monitoraggio delle
	      transazioni.
\end{itemize}
Bitcoin \`e una rete \emph{Peer-to-Peer} di nodi su cui gira il software \emph{bitcoin core}, tramite il quale
i vari nodi riescono a comunicare tra di loro e a decidere come gestire le transazioni tra le monete.

Ogni nodo memorizza il registro di tutti i membri della rete. Quest'ultimo dev'essere quindi aggiornato periodicamente
in modo che sia uguale per tutti gli utenti. Affinch\'e questo avvenga il procedimento \`e il seguente
\begin{enumerate}
	\item Si sceglie un nodo \textbf{leader} tramite consenso: i nodi competono tra di loro cercando di risolvere un
	      problema complesso. In media viene scelto un nuovo \emph{leader} ogni 15 minuti.
	\item Il \emph{leader} propone una nuova pagina del libro contabile sulla base delle varie richieste di transazione
	      che gli utenti hanno fatto durante il periodo di selezione del \emph{leader}.
	\item Tutti i computer ricevono la pagina e, tramite la loro copia del registro, controllano che le transazioni
	      inserite siano corrette. Se lo sono accettano la pagina, altrimenti la rifiutano e si cerca un altro
	      \emph{leader}.
	\item Se la pagina viene accettata tutti i computer aggiornano il loro registro e il \emph{leader} viene ricompensato
	      con un premio in bitcoin.
\end{enumerate}

\section{Blockchain}
Ogni pagina del registro \`e collegata alla precedente come in una lista linkata. Ogni pagina, detta \textbf{blocco}
(da qui il nome \textbf{blockchain}), ha un \textbf{header}, in cui sono presenti informazioni necessarie al
mantenimento della struttura dati, e un \textbf{body}, dove sono memorizzate le transazioni.

Nell'\emph{header} sono presenti quattro parametri principali
\begin{itemize}
	\item \textbf{Timestamp}: chi crea la pagina mette un \emph{timestamp} nell'\emph{header} relativo al momento
	      della sua creazione.
	\item \textbf{Nonce}: \`e il frutto della competizione di cui abbiamo parlato prima, ossia il valore ricavato
	      dalla risoluzione del problema posto. Come vedremo pi\`u avanti si tratta del risultato dell'operazione
	      detta \textbf{mining}.
	\item \textbf{Merkle Root}: \`e la radice di un albero di Merkle costruito calcolando l'hash delle transazioni
	      e serve a controllare che una certa transazione, all'interno della pagina, sia integra.
	\item \textbf{Hash Previous Block}: \`e il valore \emph{hash} dell'\emph{header} del blocco precedente.
\end{itemize}

\subsection{Timestamp}
Il \textbf{timestamp} \`e molto importante per riuscire a dare un ordine temporale alle varie pagine il che aiuta nella
verifica della correttezza delle stesse.

\subsection{Nonce}
Come gi\`a anticipato, all'interno dell'\emph{header} di un blocco della \emph{blockchain}, \`e presente un valore
\textbf{nonce}. Tale valore \`e il frutto della competizione a cui gli utenti partecipano per diventare \emph{leader}.

La competizione consiste nel trovare il valore \emph{nonce}, tale che la funzione hash crittografica SHA256 applicata
sull'header del blocco e sul \emph{nonce} sia inferiore di un certo valore detto \textbf{difficulty target}.
\begin{center}
	\verb|SHA256(header, nonce) < difficulty_target|
\end{center}
Pi\`u il \emph{difficulty target} \`e piccolo, pi\`u \`e difficile trovare un \emph{nonce} che soddisfi la disequazione.

Quello che si fa in pratica \`e un forza bruta sui valori del \emph{nonce} fich\'e non si trova quello giusto. Ecco
perch\'e questo procedimento, in gergo, si chiama \textbf{mining}.

Il primo utente che trova un \emph{nonce} valido lo mostra agli altri utenti che ne verificano la correttezza. Se
il valore \`e corretto allora l'utente diventa il nuovo \emph{leader} e propone una nuova pagina del registro contabile.

\subsection{Merkle Root}
Un altro valore presente nell'\emph{header} di un blocco \`e la \textbf{Merkle root}, ossia la radice di un
\textbf{albero di Merkle} calcolato sulle transazioni di un blocco:
\begin{enumerate}
	\item Si calcola l'hash delle singole transazioni.
	\item I valori hash calcolati vengono concatenati a due a due.
	\item Si calcola l'hash di tutte le concatenazioni.
	\item Si continua finch\'e non rimane un solo valore, ossia la \emph{Merkle root}.
\end{enumerate}
Il risultato \`e un albero binario bilanciato che ha come foglie il valore hash delle transazioni e dove il nodo padre
di un generico nodo \`e il valore hash della concatenazione dei due nodi figli. La radice dell'albero \`e ovviamente la
\emph{Merkle root}.

La \emph{Merkle root} serve a controllare l'integrit\`a di una transazione: all'utente viene fornita la
\emph{Merkle root} e vengono forniti i valori hash necessari a calcolarla. Se l'utente, calcolando la \emph{Merkle root}
ottiene la \emph{Merkle root} fornita dal software allora sa che la transazione \`e corretta.

Quando il \emph{leader} deve creare il blocco con le nuove transazioni calcola il nuovo \emph{albero di Merkle} e quindi
la nuova \emph{Merkle root} che va ad inserire nell'\emph{header} del blocco.

Gli altri utenti ricalcolano a loro volta l'albero e controllano che la radice sia corretta.

\subsection{Hash Previous Block}
Questo valore serve a rendere la \emph{blockchain} immutabile. Se provassimo a modificare una o pi\`u transazioni di
un blocco della \emph{blockchain}, la \emph{Merkle root} nell'\emph{header} del blocco non sarebbe pi\`u consistente
con quest'ultimo.

Ricalcoliamo allora l'albero e di conseguenza la nuova radice. A questo punto per\`o \`e praticamente certo che il
valore del \emph{nonce} violi la disequazione e dobbiamo quindi ricalcolarlo.

Supponiamo di riuscire a calcolare un nuovo \emph{nonce}: il valore hash dell'\emph{header} della pagina corrente
(presente nel campo \emph{hash previous block} dell'\emph{header} della pagina successiva) non sar\`a pi\`u consistente
con la pagina corrente.

Si innesca cos\`i un processo di modifiche a catena fino ad arrivare alla pagina pi\`u recente nella blockchain. Anche
riuscendo a modificare tutti i blocchi necessari e a diventare \emph{leader} nell'ultima competizione, una volta
proposta la nuova pagina, questa verrebbe respinta dagli altri utenti in quanto non sarebbe consistente col registro
in loro possesso.

\section{Transazioni}
Nel \emph{body} di un blocco sono salvate tutte le \textbf{transazioni} effettuate fino a quel momento. Una transazione
\`e un invio di bitcoin da un utente $A$ ad un utente $B$ ma in ambito bitcoin sono pi\`u complicate e fanno uso di
firme digitali che hanno la propriet\`a di
\begin{itemize}
	\item \textbf{Autenticazione}: ogni volta che si effettua una transazione si deve essere in possesso di una coppia
	      di chiavi, privata e pubblica, che servono a produrre una firma digitale.
	\item \textbf{Integrit\`a}: una transazione non deve essere modificata da qualche attacco attivo o da problemi di
	      rete.
	\item \textbf{Non ripudio}: non si pu\`o negare di aver effettuato una transazione.
\end{itemize}
Supponiamo di avere due utenti $A$ e $B$ i quali possiedono rispettivamente le chiavi private $k_{A\text{ [priv]}}$ e
$k_{B\text{ [priv]}}$ e le chiavi pubbliche $k_{A\text{ [pub]}}$ e $k_{B\text{ [pub]}}$ e siano $\text{Addr}_A$ e
$\text{Addr}_B$ rispettivamente i loro indirizzi, una transazione da $A$ a $B$ avviene in questo modo:
\begin{enumerate}
	\item $A$ firma la transazione con la sua chiave privata.
	\item $A$ invia il crittogramma a $B$.
	\item $B$ verifica che il messaggio sia inviato da $A$ tramite la chiave $k_{A\text{ [pub]}}$.
\end{enumerate}
Le transazioni vengono effettuate tramite programmi non Turing completi e la loro verifica coinvolge una terza parte
fidata che fa da \emph{garante}.

La transazione non avviene direttamente da $A$ a $B$ ma si invia il messaggio relativo alla transazione ad un
\textbf{indirizzo multisignature} che coinvolge questa terza parte fidata, la quale controlla che la transazione
sia correttamente effettuata e la firma.

Una transazione relativa ad un certo utente \`e strutturata come un blocco in cui sono presenti tre parametri
fondamentali:
\begin{itemize}
	\item \textbf{Input}: somma dei bitcoin ricevuti.
	\item \textbf{Output}: somma dei bitcoin inviati.
	\item \textbf{Unspent Transaction Output}: bitcoin non spesi che vengono comunque mandati in output verso se stessi
	      cos\`i da recuperarli.
\end{itemize}
Il risultato \`e la differenza tra input e output.

In definitiva, nessuno possiede dei bitcoin ma si possiede una chiave privata che consente di spendere bitcoin inviati
ad un certo indirizzo relativo ad una chiave pubblica. Se si perde la chiave privata si perdono tutti i bitcoin legati
ad essa.
 			% BITCOIN
\chapter{Crittografia quantistica}
In questo capitolo non andremo a parlare di macchine quantistiche ma andremo a vedere gli effetti della
\textbf{meccanica quantistica} sulla crittografia.

Esistono infatti dei protocolli crittografici che sfruttano alcuni dei suoi principi per lo scambio delle chiavi in
contesti in cui \`e richiesta estrema sicurezza e ai quali si affianca un One-Time Pad come cifrario simmetrico per
le comunicazioni.

\section{Meccanica quantistica}
Introduciamo alcuni principi di meccanica quantistica necessari a comprendere il protocollo
\begin{itemize}
	\item \textbf{Sovrapposizione}: propriet\`a di un sistema quantistico di trovarsi in diversi stati
	      contemporaneamente.
	\item \textbf{Decoerenza}: la misurazione di un sistema quantistico disturba il sistema. Il sistema disturbato
	      perde la sovrapposizione degli stati e collassa in uno stato singolo.
	\item \textbf{No Cloning}: impossibilit\`a di copiare uno stato quantistico non noto.
	\item \textbf{Entanglement}: possibilit\`a che due o pi\`u elementi si trovino in uno stato quantistico correlato
	      in modo che, anche se portati a grande distanza, mantengono la correlazione.
\end{itemize}

\subsection{Fotoni polarizzati}
Per comprendere al meglio il protocollo che vedremo in seguito dobbiamo anche parlare di \textbf{fotoni polarizzati}.

Un fotone ha una propriet\`a, detta \textbf{polarizzazione}, che pu\`o assumere due valori e che pu\`o essere misurata
facendo riferimento ad una \textbf{base}, anch'essa di due tipi:
\begin{itemize}
	\item \textbf{Ortogonale}: si indica con $+$ e pu\`o assumere valore \textbf{verticale} (a $90^\circ$), indicato con
	      $v$ oppure \textbf{orizzontale} (a $0^\circ$), indicato con $h$.
	\item \textbf{Diagonale}: si indica con $\times$ e pu\`o valere $+45^\circ$ o $-45^\circ$.
\end{itemize}

\subsubsection{Misurazione}
Non \`e possibile distinguere, con una misura, uno dei quattro casi: si deve scegliere una delle due basi e, dopo la
misura, \`e possibile distinguere uno dei due casi relativi alla base scelta.

Per la misurazione viene usato uno strumento (\textbf{Polarizing Beam Splitter}) il quale, una volta scelta la base di
riferimento, misura il valore di polarizzazione relativo alla base.

Tuttavia, la misurazione \`e corretta solo nel caso in cui il fotone sia preparato con la stessa base del PBS. Nel caso
in cui la base non sia quella corretta si hanno pari probabilit\`a di misurare uno dei due valori.

Il PBS ha due uscite, A ed R, che stanno rispettivamente per \emph{assorbimento} e \emph{riflessione}, e un asse di
polarizzazione S. Anche il fotone ha un suo asse di polarizzazione F e indichiamo con $\theta$ l'angolo che questo
forma con S.

In fase di misurazione si hanno due possibili scenari:
\begin{itemize}
	\item Il fotone esce dall'uscita A con probabilit\`a $\cos^2 \theta$ e assume polarizzazione S.
	\item Il fotone esce dall'uscita R con probabilit\`a $\sin^2 \theta$ e assume polarizzazione perpendicolare a S
	      ($\text{S}^\perp$).
\end{itemize}
Procediamo per casi supponendo di usare un PBS impostato su base ortogonale e quindi con S = 0:
\begin{itemize}
	\item Se $\theta = 0$ (F = S), allora il fotone esce da A con probabilit\`a 1 con polarizzazione S = F.
	\item Se $\theta = 90^\circ$, allora il fotone esce da R con probabilit\`a 1 con polarizzazione
	      $\text{S}^\perp = \text{F}$.
	\item Se $\theta = \pm 45^\circ$, allora il fotone esce con pari probabilit\`a da A o da R con polarizzazione
	      a $0^\circ$ o $90^\circ$.
\end{itemize}
Dunque la lettura ha distrutto lo stato quantistico precedente, il quale non \`e pi\`u recuperabile.

Per fare riferimento ai principi elencati all'inizio, possiamo dire che si \`e perso lo stato di \emph{sovrapposizione}
e si \`e dunque verificata una situazione di \emph{decoerenza}.

Dato che il protocollo prevede l'invio di fotoni polarizzati tra i due utenti, l'azione di crittoanalista che tenta di
intercettare la comunicazione lascia tracce, proprio perch\'e rompe lo stato di \emph{sovrapposizione}.

\section{BB84}
Ideato da Bennet e Brassard, \`e un protocollo per lo scambio di chiavi che fa uso di fotoni polarizzati.

Supponiamo che un utente $A$ voglia comunicare con l'utente $B$ tramite il protocollo BB84. Una prima fase viene
effettuata sul \textbf{canale quantistico}, al quale \`e associato un valore QBER (Quantum Bit Error Rate), che indica
la percentuale di errori dovuti al rumore.
\begin{enumerate}
	\item L'utente $A$ genera un sequenza iniziale di $n$ bit $S_A$, da cui sar\`a estratta la chiave (rappresentata
	      con un codice a correzione di errori).
	\item Per $n$ volte
	      \begin{enumerate}[label=(\Roman*)]
		      \item $A$ sceglie una base a caso, codifica $S_A[i]$ e invia il fotone a $B$.
		      \item $B$ sceglie una base a caso, interpreta il fotone ricevuto e costruisce $S_B[i]$.
	      \end{enumerate}
\end{enumerate}
Queste due sequenze coincidono con certezza dove le basi scelte coincidono. Al contrario, dove le basi non coincidono
i bit avranno pari probabilit\`a di coincidere o di non coincidere.

A questo punto i due utenti passano ad una comunicazione sul canale standard e si accordano su una funzione hash $h$
crittografica.
\begin{enumerate}
	\item $B$ comunica ad $A$ le basi scelte per ogni bit.
	\item $A$ risponde a $B$ comunicando le basi comuni.
	\item $A$ e $B$ estraggono due sottosequenze $S_A'$ e $S_B'$ di $S_A$ ed $S_B$, corrispondenti alle basi comuni.
	\item $A$ e $B$ estraggono due sottosequenze di $S_A''$ ed $S_B''$ da $S_A'$ ed $S_B'$ in posizioni concordate per
	      accertarsi se c'\`e stato l'intervento di un crittoanalista.
	\item $A$ e $B$ si scambiano $S_A''$ e $S_B''$ e se la percentuale di bit diversi \`e maggiore di QBER allora
	      possiamo dedurre che ci sia stato l'intervento di un crittoanalista e dunque si ripete tutto il procedimento
	      da capo.
	\item $A$ e $B$ calcolano rispettivamente $S_A' \backslash S_A''$ e $S_B' \backslash S_B''$, le decifrano con un
	      codice di correzione degli errori e ottengono una sequenza comune $S_C$.
	\item $A$ e $B$ calcolano $h(S_C)$ e la usano come chiave.
\end{enumerate}

\subsection{Sicurezza}
Il crittoanalista vuole scoprire i bit della chiave e per farlo deve misurare la polarizzazione dei fotoni in transito
sul canale. Dato che nessuno apparte $A$ sa quale sia la base usata, il crittoanalista
\begin{enumerate}
	\item Intercetta il fotone.
	\item Lo misura con una base scelta a caso.
	\item Lo invia a $B$.
	\item Costruisce la sua sequenza $S$ (non necessariamente per ogni fotone inviato).
\end{enumerate}
Cos\`i facendo, nel caso in cui il crittoanalista abbia usato la stessa base di $A$, il fotone non viene perturbato
nel suo stato di polarizzazione, ma nel caso in cui la base sia diversa, il fotone cambia la sua polarizzazione.

Se fosse possibile copiare lo stato quantistico di un fotone sarebbe possibile, per il crittoanalista, copiare il
lo stato del fotone prima di misurarlo, inviare la copia a $B$ e poi misurare il fotone originale. Come detto
all'inizio del capitolo, il principio di \emph{no cloning} non permette che ci\`o avvenga.

Il crittoanalista, per provare a confondere le perturbazioni dovute alle misure che fa sui fotoni, potrebbe decidere
di intercettarne solo alcuni e nel caso riuscisse passare inosservato, conoscerebbe alcuni bit della chiave.

Questo \`e potenzialmente molto pericoloso dato che, per ogni bit conosciuto, lo spazio delle chiavi si dimezza e quindi
un attacco a \emph{forza bruta} molto meno costoso.

Ecco perch\'e $A$ e $B$ usano l'immagine hash di $S_C$: usando l'immagine hash, la sequenza cambia radicalmente e i bit
intercettati diventano dunque inutili per il crittoanalista. 			% CRITTOGRAFIA QUANTISTICA

\end{document}