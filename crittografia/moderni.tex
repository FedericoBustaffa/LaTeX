\chapter{Cifrari moderni}\label{moderni}
Passiamo ora ai cosiddetti \textbf{cifrari moderni} i quali si dividono in due grandi gruppi
\begin{itemize}
	\item \textbf{Cifrari a sicurezza incondizionata}: sono cifrari per uso ristretto e nascondono l'informazione con
	      certezza assoluta (anche per macchine quantistiche).
	\item \textbf{Cifrari a sicurezza computazionale}: sono cifrari adibiti alla crittografia di massa e nascondono
	      l'informazione solo se il crittoanalista ha accesso a risorse computazionali limitate (su macchine quantistiche
	      si forzano in tempo polinomiale). Anche nel caso in cui si riuscisse a dimostrare che P = NP ognuno di questi
	      cifrari crollerebbe e si potrebbe forzare in tempo polinomiale.
\end{itemize}

\section{Cifrari perfetti}
Un \textbf{cifrario perfetto} \`e tale se non si riesce ad estrapolare alcuna informazione dall'analisi del crittogramma.

Proviamo a formalizzare matematicamente quanto appena detto. Per farlo dobbiamo considerare
\begin{itemize}
	\item \textbf{MSG}: spazio dei messaggi.
	\item \textbf{CRITTO}: spazio dei crittogrammi.
	\item \textbf{M}: variabile aleatoria che descrive il comportamento del	mittente e assume i valori in MSG.
	\item \textbf{C}: variabile aleatoria che descrive la comunicazione sul canale.
\end{itemize}
Indichiamo ora con
\[ P(M = m) \]
la probabilit\`a che il mittente voglia inviare il messaggio $m \in$ MSG. Indichiamo invece con
\[ P(M = m \mid C = c) \]
la probabilit\`a condizionata che il messaggio inviato sia proprio $m$, posto che sul canale stia transitando il
crittogramma $c \in$ CRITTO. In altre parole quest'ultima espressione indica la probabilit\`a che $c$ sia $m$ cifrato.

\begin{theorem}\label{th: cifrario_perfetto}
	Un cifrario \`e \textbf{perfetto} se $\forall m \in \text{MSG}$ e $\forall c \in \text{CRITTO}$ vale che
	\[ P(M = m \mid C = c) = P(M = m) \]
\end{theorem}

\begin{example}
	Mettiamoci per un attimo in uno scenario di massimo pessimismo in cui il crittoanalista sa:
	\begin{itemize}
		\item La distribuzione di probabilit\`a con cui il mittente invia messaggi.
		\item Il cifrario utilizzato.
		\item Lo spazio delle chiavi.
	\end{itemize}
	Supponiamo inoltre che di voler inviare un messaggio $m$ con probabilit\`a
	\[ P(M = m) = p > 0 \quad \quad \text{con } 0 < p < 1 \]
	e analizziamo due casi estremi e opposti l'uno all'altro. Nel primo caso diciamo che esiste un crittogramma $c$ tale
	che
	\[ P(M = m \mid C = c) = 1 \]
	e nel secondo caso diciamo che esiste un crittogramma $c$ tale che
	\[ P(M = m \mid C = c) = 0 \]
	In entrambi i casi, vedere il crittogramma, raffina la conoscenza del crittoanalista. L'unico caso in cui il
	crittoanalista non ricava nulla dal crittogramma \`e il caso descritto dal teorema \ref{th: cifrario_perfetto}.
\end{example}

\subsection{Svantaggi}
L'estrema solidit\`a di un cifrario perfetto ha per\`o un costo in termini di numero di chiavi.

\begin{theorem}[Shannon]
	In un cifrario perfetto l'insieme delle chiavi deve essere grande almeno quanto l'insieme dei messaggi possibili.
	Dove per \textbf{messaggio possibile} indichiamo un messaggio $m \in$ MSG tale che
	\[ P(M = m) > 0 \]
	Questa \`e condizione necessaria ma non sufficiente affinch\'e il cifrario sia perfetto.
	\begin{proof}
		Dimostriamo il teorema per assurdo e andiamo ad indicare con $N_k$ il numero delle chiavi e con $N_m$ il numero
		dei messaggi possibili.

		Supponiamo per assurdo che
		\[ N_m > N_k \]
		e consideriamo ora un crittogramma $c$ che pu\`o transitare sul canale con probabilit\`a
		\[ P(C = c) > 0 \]
		Se provassimo a decifrare $c$ con una generica chiave $k_i$ otterremo un messaggio $m_i$. Facciamo per\`o
		attenzione al fatto che cifrando $c$ con una chiave $k_j$ potremmo ottenere il messaggio $m_i$, ottenibile anche
		con la chiave $k_i$.

		Indichiamo quindi con $s$ tale che
		\[ s \leq N_k \]
		il numero dei messaggi che potrebbero corrispondere al crittogramma $c$. Ma per ipotesi abbiamo che
		\[ N_k < N_m \]
		quindi
		\[ s \leq N_k < N_m \]
		Ho ottenuto che il numero dei messaggi che possono corrispondere al crittogramma $c$ \`e strettamente minore
		del numero dei messaggi possibili.

		Questo vuol dire che esiste un messaggio $m'$ appartenente allo spazio dei messaggi possibili che non pu\`o
		corrispondere a quel crittogramma.
		\[ P(M = m' \mid C = c) = 0 \]
		Giungiamo quindi all'assurdo dato che un cifrario \`e perfetto se un crittogramma pu\`o corrispondere ad uno
		qualsiasi dei messaggi possibili.
	\end{proof}
\end{theorem}

\subsection{One-Time Pad}
Come abbiamo in parte anticipato, il cifrario \textbf{One-Time Pad} altro non \`e che un cifrario di Vigen\`ere che
cifra e decifra sequenze binarie e dove la chiave, al posto di essere corta e ripetuta, \`e lunga quanto il messaggio
e dunque non \`e mai ripetuta.

La prima parte del nome (One-Time) \`e relativa alla chiave: ogni chiave dev'essere utilizzata una sola volta e poi
buttata via.

\subsubsection{Funzionamento}
Consideriamo
\begin{itemize}
	\item \textbf{MSG}: lo spazio dei messaggi.
	\item \textbf{CRITTO}: lo spazio dei crittogrammi.
	\item \textbf{KEY}: lo spazio delle chiavi.
\end{itemize}
Sia il messaggio, che la chiave, che il crittogramma saranno una sequenza di $n$ bit. Il crittogramma si compone facendo
lo XOR bit a bit di messaggio e chiave
\[ c = m \oplus k \]
Lo XOR ritorna 1 se i bit che sto confrontando sono uguali, ritorna 0 altrimenti. Il crittoanalista, vedendo il
crittogramma, sa che
\begin{itemize}
	\item quando vede uno 0 in posizione $i$, allora i bit di messaggio e chiave in posizione $i$ sono uguali ma
	      non si sa se siano tutti e due 0 o tutti e due 1.
	\item quando vede un 1 in posizione $i$, allora i bit di messaggio e chiave in posizione $i$ sono diversi ma
	      non si sa quale sia 1 e quale sia 0.
\end{itemize}
Per effettuare la decifrazione basta rifare lo XOR bit a bit di crittogramma e chiave
\[ c_i \oplus k_i = m_i \]
si vede facilmente che il procedimento funziona
\begin{gather*}
	c_i = m_i \oplus k_i \\
	\Downarrow \\
	m_i \oplus k_i \oplus k_i = m_i
\end{gather*}
ma $k_i \oplus k_i$ \`e un sequenza di 0 e quindi deduciamo facilmente che
\[ m_i \oplus 0 = m_i \]

\subsubsection{Debolezza}
La debolezza si ha dal punto di vista della generazione delle chiavi. Come abbiamo detto, la chiave dev'essere monouso.

Prendiamo come esempio il caso in cui due messaggi, $m_1$ ed $m_2$ siano cifrati, con la stessa chiave $k$, in due
crittogrammi
\begin{gather*}
	c_1 = m_1 \oplus k \\
	c_2 = m_2 \oplus k
\end{gather*}
A questo punto il crittoanalista potrebbe applicare lo XOR bit a bit fra i due crittogrammi per ottenere
\[ c_1 \oplus c_2 = (m_1 \oplus k) \oplus (m_2 \oplus k) \]
dato che vale la propriet\`a associativa posso scrivere
\[ c_1 \oplus c_2 = (m_1 \oplus m_2) \oplus (k \oplus k) \]
Come prima $k \oplus k$ \`e una sequenza di 0 e quindi otteniamo
\[ c_1 \oplus c_2 = m_1 \oplus m_2 \]
Dalla sequenza di bit ottenuta, si pu\`o raffinare la propria conoscenza del messaggio andando a cercare lunghe sequenze
di 0, le quali indicano che quella parte di messaggio \`e stata inviata due volte.

\subsubsection{Sicurezza}
Vogliamo ora dimostrare che il cifrario \`e perfetto. Per farlo lavoriamo sotto alcune ipotesi
\begin{itemize}
	\item Tutti i messaggi sono lunghi $n$. Se il messaggio \`e pi\`u corto di $n$ faccio un po' di \emph{padding}.
	      Se invece il messaggio \`e pi\`u lungo di $n$ faccio una cifratura a blocchi.
	\item Tutte le sequenze di $n$ bit sono messaggi possibili.
	\item I messaggi privi di significato vengono utilizzati per confondere la crittoanalisi e ognuno di essi ha una
	      probabilit\`a molto bassa, ma comunque maggiore di 0, di essere inviati.
	\item La chiave deve essere casuale e unica per ogni messaggio.
\end{itemize}

\begin{theorem}
	Sotto le ipotesi appena elencate, One-Time Pad \`e un cifrario perfetto e impiega un numero minimo di chiavi.
	\begin{proof}
		Dimostriamo per prima cosa la \textbf{minimalit\`a}, ossia
		\[ N_n = N_k = 2^n \]
		ma questo \`e immediato dato che le chiavi sono sequenze di bit lunghe quanto i messaggi.
	\end{proof}

	\begin{proof}
		Dimostriamo ora che il cifrario \`e perfetto. Come sappiamo, un cifrario \`e perfetto se per ogni $m \in$ MSG e
		per ogni $c \in$ CRITTO vale
		\[ P(M = m \mid C = c) = P(M = m) \]
		Partiamo dicendo che
		\[ P(M = m \mid C = c) = \frac{P(M = m \wedge C = c)}{P(C = c)} \]
		dove il valore al numeratore \`e la probabilit\`a che il messaggio inviato sia $m$ e che sia stato cifrato in $c$.

		Per come \`e fatto lo XOR, fissato $m$, chiavi diverse producono crittogrammi diversi. Esiste dunque
		un'\textbf{unica} chiave $k$ che cifra $m$ in $c$. Pi\`u formalmente possiamo affermare che la probabilit\`a che
		il crittogramma sia $c$ \`e uguale alla probabilit\`a di scegliere a caso l'unica chiave $k$ che cifra $m$ in $c$
		\[ P(C = c) = \frac{1}{2^n} \]
		Se la probabilit\`a di ottenere il crittogramma $c$ dipende solo dalla chiave, allora i due eventi sono al
		numeratore indipendenti e possiamo quindi riscrivere la formula iniziale in questo modo
		\[ P(M = m \mid C = c) = \frac{P(M = m) \cdot P(C = c)}{P(C = c)} \]
		Semplificando \`e immediato ottenere
		\[ P(M = m \mid C = c) = P(M = m) \]
	\end{proof}
\end{theorem}

\subsubsection{Scambio delle chiavi}
Un metodo ragionevole per lo scambio di chiavi \`e quello che prevede lo scambio tra i due utenti del generatore casuale
e del suo assetto iniziale (seme).

In questo modo il procedimento di cifratura e decifrazione funziona in questo modo
\begin{enumerate}
	\item I due generatori vengono impostati allo stesso modo con lo stesso seme.
	\item Si scrive un messaggio $m$ e si genera un chiave $k$ lunga $|m|$ con il generatore.
	\item Si cifra il messaggio con $k$.
	\item Si genera la chiave $k$ di $|c|$ bit con il secondo generatore, che ricordiamo essere uguale al primo e
	      inizializzato con lo stesso seme.
	\item Si decifra il crittogramma $c$ con la chiave $k$ generata dal secondo generatore.
\end{enumerate}
Alla fine di questo processo si ha che i due generatori sono impostati di nuovo alla stessa maniera e si pu\`o quindi
continuare la comunicazione.

Il generatore deve essere crittograficamente sicuro e il seme deve essere molto lungo in modo da essere al riparo da
attacchi a forza bruta sul seme.

\subsubsection{Conclusioni}
In conclusione proviamo a rimuovere l'ipotesi secondo cui ogni messaggio sia possibile, anche quelli non significativi.

Dato che le chiavi devono essere tante quante i messaggi possibili. Se restringessimo l'insieme dei messaggi possibili
anche lo spazio delle chiavi diventerebbe pi\`u piccolo e con esso anche la lunghezza delle chiavi diminuirebbe.

In lingua inglese i messaggi significativi lunghi $n$ bit sono circa $\alpha^n$ con
\[ \alpha = 1.1 \]
Se considerassimo quindi solo l'insieme dei messaggi significativi in inglese, la cardinalit\`a dell'insieme di chiavi
passerebbe da $2^n$ a $1.1^n$.

Il numero delle chiavi dev'essere almeno quanto il numero dei messaggi
\[ N_k \geq N_m = \alpha^n < 2^n \]
e dato che $\alpha^n < 2^n$ posso descrivere le chiavi con $t$ bit con $t$ tale che
\[ 2^t \geq \alpha^n \]
quindi
\[ t \quad \geq \quad n \log_2 \alpha \quad \tilde{=} \quad 0.12 \cdot n \]
Abbiamo cos\`i ridotto il numero di chiavi possibili a circa un decimo del numero di chiavi che avevamo prima.

Il problema \`e che avendo ridotto cos\`i tanto l'insieme delle chiavi, un attacco di tipo forza bruta torna ad avere
senso.

Quello che si fa in genere per riuscire a mitigare il problema riuscendo comunque a diminuire un po' il numero di chiavi
e far s\`i che decifrando un crittogramma con chiavi diverse si riesca a risalire a diversi messaggi significativi.

In altre parole, cifrando messaggi diversi con chiavi diverse si ottiene lo stesso crittogramma.

Per fare questo il numero di coppie $(m, k)$ deve essere molto maggiore del numero di crittogrammi. Supponiamo di usare
chiavi casuali di $t$ bit. Se il numero di messaggi significativi \`e $\alpha^n$ abbiamo
\[ \alpha^n \cdot 2^t \]
possibili coppie $(m, k)$ mentre il numero dei crittogrammi rimane $2^n$. Otteniamo dunque che
\[ \alpha^n \cdot 2^t >> 2^n \]
che equivale a
\[ n \log_2 \alpha + t >> n \]
sviluppando ancora i calcoli otteniamo che le chiavi devono essere lunghe
\[ t >> n - n\log_2 \alpha \quad \rightarrow \quad t >> 0.88 \cdot n \]
affinch\'e si verifichi il fenomeno descritto in precedenza, ossia che a pi\`u coppie messaggio-chiave corrisponda lo
stesso crittogramma.

La rimozione dell'ipotesi non ci permette quindi di risparmiare sui bit della chiave se si vuole mantenere un buon grado
di sicurezza. Siamo comunque riusciti a diminuire il numero delle chiavi.