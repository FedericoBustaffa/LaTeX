\section{Valutazione di espressioni}
La \textbf{valutazione} di una $\lambda$-espressione consiste nel calcolare un risultato a partire dall'applicazione
di un'espressione su di un'altra espressione. Per esempio la valutazione di
\[ (\lambda x.e_1) \; e_2 \]
consiste nel valutare $e_1$ dove ogni occorrenza di $x$ in $e_1$ è rimpiazzata da $e_2$.

\subsection{Beta riduzione e redex}
In particolare quando valutiamo un'espressione dobbiamo individuare quale sia l'operazione da eseguire e sostituire
nell'espressione il suo risultato. Nel $\lambda$-calcolo l'unica operazione possibile è l'applicazione funzione.

L'operazione di applicazione funzionale è detta \textbf{$\beta$-riduzione} e la porzione di $\lambda$-espressione su
cui si esegue tale operazione è detta \textbf{redex}.

La stessa espressione potrebbe essere valutata in ordine diverso, l'importante è che ogni valutazione dia sempre il
solito risultato.

Un caso comune in cui si hanno più \emph{redex} valutabili allo stesso tempo è quello delle funzioni a più parametri.
Queste funzioni possono essere trattate in due modi:
\begin{itemize}
	\item In \textbf{ampiezza}:
	      \begin{enumerate}
		      \item Si sostituisce ad ogni variabile legata il parametro in input.
		      \item Si passa alla valutazione dello strato successivo.
	      \end{enumerate}
	\item In \textbf{profondità}:
	      \begin{enumerate}
		      \item Si sostituisce una sola delle variabili con il corrispondente parametro in input.
		      \item Si continua la valutazione nella sotto-espressione ottenuta.
	      \end{enumerate}
\end{itemize}
Possiamo quindi definire una \emph{redex} come una struttura del tipo
\[ (\lambda x.e_1) \; e_2 \]
che deve essere individuata nell'espressione per poter applicare la $\beta$-riduzione in modo corretto.

\subsection{Capture avoiding substitution}
La sostituzione sintattica che viene svolta in ogni $\beta$-riduzione può causare problemi come ad esempio andare a
legare una variabile libera (sempre da evitare per evitare di cambiare il significato dell'espressione).

Ecco che introduciamo la \textbf{capture avoiding substitution}. Consideriamo sempre l'espressione genereale
$(\lambda x.e_1) \; e_2$, possiamo passare alla notazione \emph{capture avoiding substitution} scrivendo
\[ (\lambda x.e_1) \{ x := e_2 \} \]
In questo modo andiamo ad esplicitare che ogni occorrenza di $x$ in $e_1$ deve essere sostituita con $e_2$.

La \emph{capture avoiding substitution} è definita ricorsivamente per andare a catturare ogni casistica possibile. Di
seguito i vari casi sono scritti nella forma $e_1 \; \{ x := e_2 \}$ dove $e_1$ è il corpo della funzione.
\begin{itemize}
	\item Se il corpo è esattamente $x$ questo viene sostituito interamente da $e_2$
	      \[ x \; \{ x := e_2 \} = e_2 \]
	\item Se il corpo è qualcosa di diverso da $x$ non si applica alcuna sostituzione
	      \[ y \; \{ x := e_2 \} = y \]
	\item Se il corpo è costituito dall'applicazione di $e'$ su $e''$ si sostituisce ad ogni occorrenza di $x$ in
	      $e'$ ed $e''$ l'espressione $e_2$
	      \[ (e' e'') \; \{ x := e_2 \} = (e' \; \{ x := e_2 \})(e'' \; \{ x := e_2 \}) \]
	\item Se il corpo è un'astrazione funzionale che prende un generico parametro $y \neq x$, con corpo $e$ e $y$ non
	      appartiene all'insieme delle variabili libere in $e$, si va a sostituire ogni occorrenza di $x$ in $e$ con
	      $e_2$
	      \[ (\lambda y.e) \; \{ x := e_2 \} = \lambda y.(e \; \{ x := e_2 \}) \]
	\item Se il corpo è un'astrazione funzionale che prende un generico parametro $y \neq x$, con corpo $e$ e $y$
	      appartiene all'insieme delle variabili libere in $e$, si introduce una variabile fresca $z$, andando ad
	      eseguire una $\alpha$-conversione sostituendo ogni occorrenza di $y$ in $e$ con $z$.

	      All'espressione risultante si sostituisce ogni occorrenza di $x$ con $e_2$.
	      \[ (\lambda y.e) \; \{ x := e_2 \} = \lambda z.((e \; \{ y := z \}) \; \{ x := e_2 \}) \]
\end{itemize}
