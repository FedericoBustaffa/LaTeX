\section{Alpha equivalenza}
Le espressioni
\[ \lambda a.ac \]
e
\[ \lambda b.bc \]
sono dette \textbf{$\alpha$-equivalenti}. La sostituzione di una variabile legata con un'altra opportuna variabile
\emph{fresca} costituisce una trasformazione sintattica di programmi che, solitamente in matematica, viene chiamata
\textbf{$\alpha$-conversione}.
\begin{center}
	\verb| function(a) { return a + 1; } |
\end{center}
è $\alpha$-equivalente a
\begin{center}
	\verb| function(b) { return b + 1; } |
\end{center}
Espressioni $\alpha$-equivalenti rappresentano lo stesso programma.

\subsection{Valutazione di lambda espressioni}
Eseguire una lambda espressione significa valutarla e comporta solamente la chiamata di funzioni.

Per esempio la valutazione di
\[ \lambda x.e_1 \quad e_2 \]
consiste nel valutare $e_1$ dove ogni occorrenza di $x$ in $e_1$ è rimpiazzata da $e_2$.

La notazione
\[ e_1 \{ x := e_2 \} \]
è introdotta per descrivere la lambda espressione $e_1$ in cui ogni occorrenza della variabile $x$ è sostituita dalla
lambda espressione $e_2$.

Nel caso in cui $e_2$ contenga una variabile libera $y$ che è legata in $e_1$ c'è il rischio che $y$ venga catturata
dal legatore $\lambda$ di $e_1$. Legare una variabile libera è sempre da evitare.

La soluzione è $\alpha$-convertire prima $e_1$ introducendo una variabile fresca per ottenere una sostituzione che
eviti il problema di legare una variabile libera.

Per esempio
\[ (\lambda x.(x * y)) \{ y := (x + x) \} = (\lambda x.(x * (x + x))) \]
In questo caso la variabile libera $x$ nell'espressione $y$ viene è una variabile legata nel corpo della funzione e
come vediamo verrebbe catturata dal legatore.

Per risolvere introduciamo una variabile fresca $z$ per riuscire a fare una $\alpha$-conversione:
\[ (\lambda x.(x * y)) \{ y := (x + x) \} = (\lambda z.(z * (x + x))) \]
Ora l'unico parametro è $z$ e abbiamo una variabile libera $x$ che rimane tale senza che venga catturata dal legatore.