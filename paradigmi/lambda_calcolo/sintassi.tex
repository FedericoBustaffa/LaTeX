\section{Sintassi}
Il \textbf{lambda calcolo} può essere visto come il primo \textbf{linguaggio funzionale} e consente di definire solo
funzioni anonime con una sintassi minimale e molto semplice.

Un programma nel lambda calcolo è di fatto un'espressione, detta \textbf{lambda espressione}, la cui sintassi ha 3
componenti principali:
\begin{itemize}
	\item \textbf{Identificatori}
	      \[ e ::= x \]
	      In questo modo l'\textbf{identificatore} $x$ assume come valore l'espressione $e$.
	\item \textbf{Astrazione funzionale}
	      \[ \lambda x.e \]
	      Indica una funzione (anonima) con parametro $x$ e corpo $e$.
	\item \textbf{Applicazione funzionale}
	      \[ e_1 \; e_2 \]
	      In questo modo si applica l'espressione $e_1$ al parametro $e_2$.
\end{itemize}

Una funzione può essere definita e applicata scrivendo tutto nella solita riga.
\[ (\lambda x.(\lambda y.xy)) \quad (\lambda z.z) \]

\subsection{Costruttore lambda}
Lo \textbf{scope} del lambda si estende il più a destra possibile. Scrivere
\[ \lambda x. \lambda y. x y \]
equivale a
\[ \lambda x. (\lambda y. (x y)) \]

In entrambi i casi stiamo definendo una funzione che prende $x$ come parametro e come corpo ha una funzione con
parametro $y$ e corpo l'applicazione di $x$ a $y$, ovvero $xy$.

\subsection{Applicazione funzionale}
L'applicazione funzionale associa a sinistra. Scrivere
\[ e_1 e_2 e_3 \]
è come scrivere
\[ (e_1 e_2) e_3 \]
In entrambi i casi stiamo applicando $e_3$ all'applicazione di $e_1$ su $e_2$.

\subsection{Variabili libere e legate}
Le variabili in una lambda espressione che sono introdotte (dichiarate) in un qualche $\lambda$ sono \textbf{legate}
da quel $\lambda$.

Tutte le altre variabili che non sono associate a una dichiarazione con un qualche $\lambda$ sono \textbf{libere}.