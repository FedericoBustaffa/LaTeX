\section{Programmazione funzionale}
Vediamo ora come simulare costrutti di programmazione funzionale in $\lambda$-calcolo.

\subsection{Ricorsione}
In $\lambda$-calcolo esiste un costrutto che permette di implementare la ricorsione. Questo costrutto è il
\textbf{combinatore} $Y$, definito come segue
\[ Y = \lambda f.(\lambda x.f(x x)) (\lambda x.f(x x)) \]
e vale, qualunque sia $F$, la seguente relazione di $\beta$-equivalenza
\[ Y F \equiv_\beta F (Y F) \]
\begin{proof}
	Si può anche dimostrare la relazione in questo modo
	\[
		\begin{array}{rcl}
			Y F & =           & \lambda x.f(x x) (\lambda x.f(x x)) F     \\
			    & \rightarrow & (\lambda x. F (x x)) (\lambda x.F(x x))   \\
			    & \rightarrow & F ((\lambda x.F(x x))) (\lambda x.F(x x))
		\end{array}
	\]
\end{proof}
Per definire una funzione ricorsiva il procedimento è il seguente:
\begin{enumerate}
	\item Si definisce $G = \lambda f.\langle \text{espressione che contiene } f \rangle$.
	\item Si definisce $F = Y G$.
\end{enumerate}

\subsection{Tipi di dato}
In $\lambda$-calcolo è possibile definire o per meglio dire \emph{codificare} delle funzioni che rappresentano
costanti, valori booleani, interi e relative operazioni.

\subsubsection{Booleani}
Come primo esempio vediamo come codificare i due valori booleani \verb|True| e \verb|False|:
\begin{gather*}
	True = \lambda t.\lambda f.t \\
	False = \lambda t. \lambda f.f
\end{gather*}
Possiamo codificare anche il costrutto \verb|IF| come segue
\[ IF = \lambda e.\lambda then.\lambda else.(e \; then \; else) \]
L'\verb|IF|, come vediamo è codificato come una funzione che prende tre parametri, $e$ è l'espressione booleana
che viene valutata mentre $then$ è il valore restituito se la guardia è vera mentre $else$ è il valore restituito
se la guardia è falsa.
\[
	\begin{array}{rcl}
		IF \; True \; e_1 \; e_2 & =           &
		(\lambda e.\lambda then.\lambda else.(e \; then \; else)) \; True \; e_1 \; e_2                         \\
		                         & \rightarrow & (\lambda then.\lambda else.True \; then \; else) \; e_1 \; e_2 \\
		                         & \rightarrow & (\lambda else.True \; e_1 \; else) \; e_2                      \\
		                         & \rightarrow & True \; e_1 \; e_2                                             \\
		                         & =           & (\lambda t.\lambda f.t) \; e_1 \; e_2                          \\
		                         & \rightarrow & (\lambda f.e_1) \; e_2                                         \\
		                         & \rightarrow & e_1
	\end{array}
\]
Come c'era da aspettarsi un \verb|IF| con guardia \verb|True| ritorna il valore nel ramo \verb|then|, ossia $e_1$.

\subsubsection{Numeri naturali}
Per rappresentare i \textbf{numeri naturali} si assume di avere un costrutto libero $z$ che rappresenta lo $0$ e un
costrutto $s$ che rappresenta il successore. La codifica dello $0$ è rappresentata come segue
\[ C_0 = \lambda z.\lambda s.z \]
La codifica dell'$1$ è definita come il successore di zero, ossia
\[ C_1 = \lambda z.\lambda s.s \; z \]
In generale possiamo rappresentare un qualsiasi numero naturale $n$ come una serie di $n$ funzioni successore infine
applicate a $z$.

La \textbf{somma} è definita come una funzione a quattro parametri nella quale il primo e secondo parametro sono gli
addendi, il terzo valore è lo zero e l'ultimo è la funzione successore.
\[ Plus = \lambda m.\lambda n.\lambda z.\lambda s. m \; (n \; z \; s) \; s \]
Il \textbf{prodotto} fa uso della somma per essere definito più comodamente:
\[ Times = \lambda m.\lambda n.m \; C_0 \; (Plus \; n) \]

\subsection{Passaggio di parametri}
Andiamo ora a vedere le regole per la valutazione dell'applicazione di funzione in tre passi.
\begin{enumerate}
	\item Valutazione dell'espressione che definisce la funzione.
	\item Definizione del meccanismo di passaggio dei parametri. Abbiamo due alternative:
	      \begin{itemize}
		      \item \textbf{Call by value}: tutte le occorrenze della variabile legata alla funzione nell'espressione
		            del corpo della funzione sono sostituite dall'\textbf{espressione valutata} che definisce il
		            parametro formale.
		      \item \textbf{Call by name}: tutte le occorrenze della variabile legata alla funzione nell'espressione
		            del corpo della funzione sono sostituite dall'\textbf{espressione non valutata} che definisce il
		            parametro attuale.
	      \end{itemize}
	\item Passaggio dei parametri e valutazione della funzione.
\end{enumerate}

\subsubsection{Regole di inferenza per beta riduzione}
Prima di proseguire definiamo tre regole di inferenza per l'operazione di $\beta$-riduzione:
\begin{itemize}
	\item Se $e_1$ può essere ridotta in $e'$, l'applicazione di $e_1$ a $e_2$ è equivalente all'applicazione di $e'$
	      su $e_2$.
	      \[ \frac{e_1 \rightarrow e'}{e_1 \; e_2 \rightarrow e' \; e_2} \]
	\item Se $e_2$ può essere ridotta in $e'$, l'applicazione di $e_1$ su $e_2$ è equivalente all'applicazione di $e_1$
	      su $e'$.
	      \[ \frac{e_2 \rightarrow e'}{e_1 \; e_2 \rightarrow e_1 \; e'} \]
	\item Se il corpo $e$, di una funzione $\lambda x.e$, può essere ridotto in $e'$, allora l'intera espressione si
	      può riscrivere con $e'$ al posto di $e$ come corpo della funzione.
	      \[ \frac{e \rightarrow e'}{\lambda x.e \rightarrow \lambda x.e'} \]
\end{itemize}

\subsubsection{Riduzione call by value}
Prima si valuta completamente l'argomento e poi si applica la funzione. In questo caso la notazione per la
$\beta$-riduzione diventa questa:
\[ (\lambda x.e_1) \; v \rightarrow e_1 \; \{ x := v \} \]
dove $v$ è un'espressione non ulteriormente riducibile. Le regole di inferenza per il passaggio \emph{call by value}
diventano le seguenti:
\[
	\frac{e_1 \rightarrow e'}{e_1 \; e_2 \rightarrow e' \; e_2} \quad \quad
	\frac{e_2 \rightarrow e'}{v \; e_2 \rightarrow v \; e'}
\]

\subsubsection{Riduzione call by name}
Si applica la funzione il prima possibile portandola prima in forma $\lambda x.e_1$ e non scende nel corpo delle
funzioni come invece avviene nei linguaggi di programmazione. La notazione ritorna quella vista in precedenza per la
$\beta$-riduzione.
\[ (\lambda x.e_1) \; e_2 \rightarrow e_1 \; \{ x := e_2 \} \]
L'unica regola di inferenza è questa
\[ \frac{e_1 \rightarrow e'}{e_1 \; e_2 \rightarrow e' \; e_2} \]
Il vantaggio di questa strategia è la \textbf{posticipazione} della valutazione dell'espressione. L'espressione viene
valutata solo quando e se necessario, non prima.

\subsubsection{Non terminazione}
In molti casi, qualunque sia la scelta, il risultato finale non cambia. Esistono tuttavia situazioni in cui una riduzione
\emph{call by value} piuttosto che una riduzione \emph{call by name} potrebbe influire sull'esito della computazione.

Per esempio, alcune computazioni, a seconda della scelta fatta, potrebbero terminare o non terminare.
\begin{center}
	\verb|IF| \verb|True| \verb|True| $(\Omega \Omega)$
\end{center}
Nel caso adottassimo una riduzione \emph{call by name} la computazione terminerebbe subito:
\begin{center}
	\verb|IF| \verb|True| \verb|True| $(\Omega \Omega) \Rightarrow$ \verb|True|
\end{center}
Nel caso della \emph{call by value} la computazione invece non termina:
\begin{center}
	\verb|IF| \verb|True| \verb|True| $(\Omega \Omega)$

	$\Rightarrow$ \verb|IF| \verb|True| \verb|True| $(\Omega \Omega)$

	$\Rightarrow$ \verb|IF| \verb|True| \verb|True| $(\Omega \Omega)$

	...
\end{center}

\section{Lambda calcolo tipato}
