\section{Metriche}
Vogliamo ora capire come misurare le prestezioni della rete. Per farlo 
dobbiamo introdurre concetti come
\begin{itemize}
	\item Ampiezza di banda e bitrate
	\item Throughput
	\item Latenza
	\item Perdita di pacchetti
\end{itemize}
Tali concetti saranno utili più avanti quando andremo a trattare la
gestione della congestione del traffico o del flusso di dati 
trasmessi.

\subsection{Larghezza di banda e bitrate}
Quando si parla di \textbf{larghezza di banda} si parla della larghezza
dell'intervallo delle frequenze utilizzato dal mezzo trasmissivo e si 
misura in Hertz. Questa indica, in un certo senso, la "\emph{capacità}"
del mezzo trasmissivo.

Il \textbf{bitrate} è invece la velocità di trasmissione del mezzo 
trasmissivo e indica quanti bit possono essere trasmessi nell'unità 
di tempo e si misura in bit al secondo (bps).

\subsection{Throughput}
Il \textbf{throughput} indica la quantità di dati che possono essere 
trasmessi da un nodo sorgente a un nodo destinazione in un certo 
intervallo di tempo. Da non confondere con il bitrate in quanto il 
throughput indica la velocità con cui trasferiamo i dati tenendo di 
conto di eventuali variazioni del mezzo trasmissivo, duplicazioni, 
protocolli ecc.

Se per esempio fino a un certo nodo abbiamo un bitrate $R_1$ e da quel 
nodo in poi abbiamo un bitrate $R_2$, il throughput sarà
\[ \text{throughput} = \min (R_1, R_2) \]
In generale quando ci sono più connessioni i cui host interagiscono 
l'uno con l'altro si ha spesso una situazione in cui il throughput 
finale equivale alla capacità trasmissiva peggiore all'interno della 
rete.

\subsection{Latenza e perdite}
Con \textbf{latenza} indichiamo il tempo richiesto affinché un 
messaggio arrivi a destinazione dal momento in cui il primo bit parte 
dalla sorgente. La latenza si può ricavare tramite la seguente formula
\[ l = rp + rt + ra + re \]
dove gli elementi a secondo membro indicano rispettivamente i ritardi 
di propagazione, trasmissione accodamento ed elaborazione.

\subsubsection{Accodamento}
Come abbiamo già visto, i pacchetti che non possono essere elaborati 
dal router vengono messi in attesa in un buffer. Questo ovviamente 
genera ritardo, ossia il \textbf{ritardo di accodamento}, che equivale 
al tempo di permanenza del pacchetto nel buffer. Quando il buffer del 
router è pieno, tutti i pacchetti in arrivo vengono scartati.

\subsubsection{Elaborazione}
Altro ritardo introdotto dalla commutazione di pacchetto è il 
\textbf{ritardo di elaborazione} dovuto a controllo di errori sui bit 
e all'elaborazione di una scelta del canale d'uscita del pacchetto.

\subsubsection{Trasmissione}
Il \textbf{ritardo di trasmissione} dipende dalla velocità di 
trasmissione del collegamento e dalla lunghezza del pacchetto. 
Rappresenta il tempo impiegato per immettere completamente il 
pacchetto sulla linea.
\[ rt = \frac{l}{r} \]
dove $l$ è la lunghezza del pacchetto e $r$ è la velocità di 
trasmissione (rate) del mezzo trasmissivo.

\subsubsection{Propagazione}
Tempo impiegato da un bit per viaggiare da un punto $A$ a un punto $B$
sul mezzo trasmissivo. Esso dipende dalla lunghezza del collegamento 
fisico e dalla velocità di propagazione nel mezzo.
\[ rp = \frac{d}{s} \]
dove $d$ è la lunghezza del mezzo trasmissivo e $s$ è la velocità di 
propagazione di tale mezzo.

\subsubsection{End-to-End}
Proviamo ora a fare qualche esempio pratico per riuscire a comprendere 
meglio tutti questi concetti. Supponiamo di voler inviare un file di 
1 Mbit su un datalink di lunghezza 4800 km. Il ritardo di propagazione,
in questo caso è
\[
	rp = \frac{4800 \cdot 10^3 [\text{m}]}{3 \cdot 
	10^8 [\text{m/s}]} = 0.016 [\text{s}]
\]
Con una velocità di trasmissione di 64 kbps abbiamo un ritardo di
trasmissione pari a
\[ 
	rt = \frac{10^6 [\text{b}]}{64 \cdot 10^3 [\text{bps}]} = 
	15.625 [\text{s}]
\]
Con una velocità di trasmissione di 1 Gbps abbiamo invece un ritardo di
trasmissione pari a
\[ rt = \frac{10^6 [\text{b}]}{10^9 [\text{bps}]} = 0.001 [\text{s}] \]
Il ritardo \textbf{end-to-end} è il ritardo complessivo che c'è da 
origine a destinazione. Per calcolarlo supponiamo che ci siano $n-1$
nodi tra origine e destinazione. Per prima cosa dobbiamo calcolare il 
ritardo accumulato su ogni nodo andando a sommare tutti i tipi di 
ritardo visti fino ad ora.
\[ R_{\text{nodo}} = re + rp + rt + ra \]
A questo punto è facile intuire che il ritardo end-to-end sarà 
equivalente alla somma di tutti i ritardi accumulati sui singoli nodi
\[ R_\text{end-to-end} = \sum_{i=1}^n R_i \]

\subsubsection{Traceroute}
Uno strumento in grado di tracciare il percorso compiuto da un 
pacchetto da un nodo sorgente ad un nodo destinazione, mostrando il 
numero di passaggi necessari per raggiungerlo insieme al ritardo 
sperimentato da un nodo all'altro. Su Windows si utilizza digitando 
il comando
\begin{center}
	\verb|tracert <URL>|
\end{center}
mentre su Linux è possibile utilizzarlo tramite il comando 
\begin{center}
	\verb|traceroute <URL>|
\end{center}

\subsubsection{Volume di trasmissione}
Il prodotto tra il bitrate e il ritardo ci fornisce un \textbf{volume} 
di trasmissione che indica quanti possono essere contenuti nel mezzo 
trasmissivo in una certa unità di tempo.

