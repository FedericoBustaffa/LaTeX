\section{Internet}
Quando parliamo di \textbf{Internet} si fa riferimento alla rete più famosa la quale interconnette
migliaia di reti e che normalmente usiamo per navigare.

Essa è un rete a commutazione di pacchetto e ogni rete connessa a Internet deve usare un
\textbf{IP} (Internet Protocol) e rispettare certe convenzioni su nomi e indirizzi.

Internet ha avuto molto successo perché permette di erogare servizi di comunicazione alle
applicazioni come email, giochi, chiamate. Tali servizi possono essere
\begin{itemize}
	\item \textbf{Connectionless}: senza garanzia di consegna.
	\item \textbf{Connection-oriented}: garantiti in integrità, completezza e ordine.
\end{itemize}
Come vedremo più avanti, Internet è composto da varie entità software quali
\begin{itemize}
	\item \textbf{Applicazioni}: le quali elaborano e scambiano informazioni.
	\item \textbf{Protocolli}: organizzati come una pila, si occupano di regolamentare la
		trasmissione e la ricezione di messaggi.
	\item \textbf{Interfacce}: separano gli strati delle \emph{pila protocollare}.
	\item \textbf{Standard}: sono gli standard di Internet e del Web.
\end{itemize}
Possiamo inoltre immaginare l'organizzazione della rete come \emph{stratificata}. Al livello più
basso ci sono le reti private le quali si connettono a degli \textbf{Internet Service Provider}
(ISP) i quali forniscono dei servizi connettendosi alle \textbf{dorsali} (ISP di livello 1
interconnesse tra loro).

Il collegamento che connette l'utente al primo router di Internet è detto \textbf{rete di accesso}
e può essere effettuato
\begin{itemize}
	\item Via rete telefonica.
	\item Tramite reti wireless.
	\item Tramite collegamento diretto.
\end{itemize}

