\section{Tecnologie di livello collegamento}
In quest'ultima parte del corso andremo a trattare alcune delle
tecnologie più usate a livello collegamento come ad esempio
\textbf{ethernet}, \textbf{switch} e \textbf{VLAN}.

\subsection{Ethernet}
La tecnologia \textbf{ethernet} detiene una posizione nel mercato delle
LAN cablate poiché:
\begin{itemize}
	\item \`E stata la prima LAN cablata ad alta velocità con vasta
		diffusione.
	\item \`E più semplice e meno costosa di tecnologie come token
		ring, FDDI e ATM.
	\item Sta sempre al passo con i tempi tramite un tasso trasmissivo
		sempre più elevato.
\end{itemize}
Il tasso trasmissivo dipende anche dalla tecnologia usata: cavo 
coassiale, cavo di rame e fibra ottica.

\subsubsection{Topologia fisica}
In principio la topologia di rete era a \textbf{bus}, tutti i 
dispositivi si collegavano quindi sul collegamento (condiviso). Questo
aveva il grosso problema delle collisioni.

La topologia moderna prevede una topologia a \textbf{stella}: al centro
della stella è presente uno switch al quale si collegano tutti i 
dispositivi sulla rete. Sarà poi compito dello switch gestire le
collisioni.

\subsubsection{Pacchetti Ethernet}
L'adattatore trasmittente incapsula i datagrammi IP in un pacchetto
Ethernet costituito da
\begin{itemize}
	\item Un \textbf{preambolo} di 8 byte con i bit settati in modo da
		sincronizzare l'orologio del destinatario con quello del
		mittente.
	\item \textbf{Indirizzi} di sorgente e destinatario.
	\item \textbf{Tipo}: identificativo del protocollo a cui il livello
		collegamento dovrà consegnare il payload (Multiplexing).
	\item \textbf{Payload}
	\item \textbf{Bit di controllo errori}: in Ethernet viene
		implementato solo il controllo degli errori e non la
		correzione.
\end{itemize}
La tecnologia Ethernet è di tipo connectionless in quanto non avviene
alcun \emph{handshake} e non è neanche affidabile in quanto la 
ricezione non scatena nessun \verb|ACK| di risposta. Implementa il 
protocollo MAC con rilevamento delle collisioni.

\subsection{Switch}
La topologia a stella a cui abbiamo accennato in poco fa si poteva
già ottenere tramite un \textbf{hub}. Un hub altro non è che un 
\textbf{ripetitore} multiporta. Mentre il ripetitore si occupa di 
rogenerare il segnale lungo il mezzo fisico in un collegamento di tipo
punto-punto, l'hub ha lo stesso compito ma propaga il segnale su più
porte in uscita.

L'evoluzione dell'hub, attualmente molto usata, è lo \textbf{switch}
che connette più dispositivi andando a anche a lavorare a livello 
collegamento verificando indirizzi MAC e contenuto dei frame.

Possiedono una tabella contente l'associazione tra l'indirizzo MAC di
destinazione del frame e la porta su cui questo dev'essere inoltrato.
Sono quindi in grado di effettuare un'operazione di filtraggio dei 
frame che gli arrivano.

Tale tabella viene costruita mediante un meccanismo di 
\textbf{autoapprendimento}: all'inizio è vuota e man mano che i vari
dispositivi si collegano e comunicano, lo switch riesce a capire su
quali porte sono collegati i dispositivi.
\begin{enumerate}
	\item Quando la tabella è vuota lo switch inoltra il datagramma su
		tutte le porte in uscita e si salva il MAC del mittente e la
		porta che questo ha usato per comunicare.
	\item Una volta che tutti i dispositivi hanno effettuato almeno
		una trasmissione, lo switch avrà completato la tabella.
	\item Nel caso in cui un dispositivo non comunichi per molto 
		tempo, il riferimento nella tabella verrà cancellato.
\end{enumerate}
Lo switch permette soprattutto di evitare collisioni e di effettuare
trasmissioni multiple simultanee in quanto gli host hanno connessioni
dedicate verso lo switch.

\subsection{VLAN}
Una rete aziendale può avere una topologia molto semplice, la quale 
comprende un'unica sottorete, un mail server, un web server, degli 
switch e un router per le comunicazione da e verso l'esterno.

La differenza principale tra router e switch è che il router può 
lavorare anche a livello rete andando a svolgere le funzioni di inoltro
a più alto livello.

Questo tipo di topologia implementa un unico dominio di broadcast. Se
infatti un host invia un messaggio di broadcast (per esempio tramite 
protocollo ARP), questo viene ricevuto da tutti i componenti della 
sottorete. Questo potrebbe creare problemi di sicurezza e/o di privacy
poiché andiamo a propagare un messaggio su tutta la sottorete anche 
dove non è necessario.

Il nostro obbiettivo è quindi andare a creare più domini di broadcast
per andare ad evitare questa situazione. Uno dei metodi che implementa
questa strategia è la \textbf{VLAN}.
