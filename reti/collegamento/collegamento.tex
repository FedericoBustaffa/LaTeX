\chapter{Livello di collegamento}
Il livello di collegamento si occupa di muovere i datagrammi da un
nodo al nodo adiacente su un singolo link di comunicazione.

Un datagramma può essere gestito da più protocolli su collegamenti 
diversi poiché i link che connettono i nodi della rete possono essere
eterogenei.

Anche i servizi erogati dai protocolli del livello di collegamento 
possono essere differenti, per esempio nell'affidabilità. I 
collegamenti si possono inoltre dividere in due categorie principali:
\begin{itemize}
	\item \textbf{Punto-punto}: collegamento dedicato a due soli 
		dispositivi.
	\item \textbf{Broadcast}: collegamento condiviso tra più nodi che,
		nel momento in cui un nodo trasmette un frame, il canale lo
		diffonde e tutti i nodi ne ricevono una copia.
\end{itemize}
Il livello collegamento si potrebbe in realtà suddividere in due 
sottolivelli:
\begin{itemize}
	\item \textbf{Data-Link Control}: incapsulamento del datagramma in
		un frame, controllo e correzione errori.
	\item \textbf{Medium Access Control}: più comunemente detto MAC, 
		definisce il protocollo per accedere al mezzo.
\end{itemize}

\section{Servizi}
I servizi offerti a livello collegamento sono molteplici e dipendono
anche dal tipo di collegamento che si utilizza:
\begin{itemize}
	\item \textbf{Framing}: i protocolli incapsulano i datagrammi del
		livello rete in un frame al livello collegamento. Un frame è
		composto dal payload, un'intestazione ed eventuale trailer
		(usato per rilevazione e controllo errori).

		Il servizio di framing serve a separare i messaggi durante la
		trasmissione da un nodo sorgente ad un nodo destinazione. Per
		identificare sorgente e destinazione vengono utilizzati gli
		indirizzi MAC.
	\item \textbf{Consegna affidabile}: in alcuni protocolli, nel caso
		ci sia una perdita si effettua la ritrasmissione.
	\item \textbf{Controllo del flusso}: evita che il nodo trasmittente
		saturi quello ricevente.
	\item \textbf{Rilevazione degli errori}: permette, tramite bit 
		di errore, di rilevare errori e perdite dovuti ad attenuazione
		del segnale o da rumore elettromagnetico.
	\item \textbf{Correzione errori}: il nodo ricevente determina il
		punto in cui si è verificato l'errore e lo corregge.
\end{itemize}

\section{Implementazione}
Il livello di collegamento è implementato in una combinazione di 
hardware, software e firmware tramite la \textbf{scheda di rete} o 
\textbf{adaptor}.

La scheda di rete possiede una componente chiamata \textbf{controller},
che si occupa di implementare gran parte dei servizi citati in 
precedenza. Si occupa infatti dell'operazione di framing, controllo
dei bit di errore e della gestione di accesso al mezzo.

La parte software si occupa invece di interfacciarsi con il livello di
rete e gestire eventuali errori.

\subsection{Comunicazione tra adaptor}
Lato mittente il datagramma viene incapsulato in un frame, si 
aggiungono bit di controllo errori e, se necessario, si implementano
trasmissione affidabile e controllo del flusso.

Lato destinatario si usa l'intestazione del frame per fare controllo
degli errori, gestire un eventuale ritrasmissione ed effettuare 
controllo di flusso. Si estrae infine il datagramma e lo si passa al
livello superiore.

\subsection{Collegamenti ad accesso multiplo}
Il tipo di collegamento \textbf{broadcast} è \emph{condiviso}. Questo
crea un problema di \textbf{collisioni} quando avvengono due o più 
trasmissioni simultanee tra i nodi creando \emph{interferenza}.

Più nello specifico, un nodo che sta in ascolto sul canale condiviso,
non riesce a distinguere messaggi in arrivo da un nodo piuttosto che
da un altro. \`E quindi necessario definire dei protocolli che 
gestiscano l'accesso multiplo alla risorsa.

Un tipo di soluzione \emph{ideale} prevede l'utilizzo di un canale 
broadcast con rate $R$ bps e la gestione di tale canale dovrebbe 
avvenire in questo modo:
\begin{enumerate}
	\item Quando un solo nodo trasmette può inviare dati con un rate 
		di $R$ bps.
	\item Quando $M$ nodi vogliono trasmettere, ciascuno trasmette ad
		un rate medio di $R / M$ bps.
	\item 
\end{enumerate}
