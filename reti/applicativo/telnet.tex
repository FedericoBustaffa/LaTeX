\section{TELNET}
Il protocollo \textbf{TELNET} (TErminaL NETwork) fornisce la 
possibilità di controllare un terminale da remoto. Si tratta di un 
protocollo non specifico per un'applicazione, è infatti possibile 
collegarsi da remoto ad una macchina e poi eseguire le applicazioni 
che preferiamo.

\subsection{Funzionamento}
Per riuscire a fare tutto questo, TELNET maschera sia la rete che i 
sistemi operativi e utilizza un'interfaccia semplice (a caratteri) ma 
veloce. Abbiamo quindi la possibilità di
\begin{enumerate}
	\item Connetterci ad un server remoto, interagendo con il 
		terminale.
	\item Inviare comandi alla macchina server.
	\item L'output della macchina remota viene trasportato al terminale
		dell'utente.
\end{enumerate}
In generale il protocollo stabilisce una connessione TCP con il server 
remoto, in seguito accetta l'output prodotto dalla macchina server e lo
visualizza sul terminale utente.

Dall'altro lato il server accetta la connessione TCP e trasmette i 
dati al sistema operativo locale. La connessione persiste per tutta 
la durata della sessione di login.

\subsection{Problematiche}
Per poter funzionare, TELNET deve poter operare con il massimo numero
di sistemi e quindi gestire dettagli di sistemi operativi eterogenei.

I terminali possono differire gli uni dagli altri per codifica dei 
caratteri, larghezza della linea e lunghezza della pagina e per i tasti
funzione individuati da diverse sequenze di caratteri.

Per risolvere questo problema è stato definito uno standard comune, 
ossia un \textbf{terminale virtuale} (NVT), con delle regole per la
codifica dei caratteri e dei comandi.
Il terminale virtuale definisce un set di caratteri e comandi 
universale che permette di trasformare il set localmente in uso in un 
set universale.

TELNET non è un protocollo sicuro in quanto fa viaggiare tutte le 
informazioni in chiaro. Ad oggi esiste un protocollo che agisce in modo
molto simile ma cifra la comunicazione, ossia \textbf{SSH} (Secure 
SHell).
