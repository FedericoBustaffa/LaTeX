\section{SMTP}
Il protocollo \textbf{SMTP} (Simple Mail Transfer Protocol) serve ad inviare un messaggio da un
utente \emph{mittente} ad un utente \emph{destinatario} in modo \textbf{asincrono}. Non è infatti
necessario che i due utenti siano contemporaneamente connessi alla rete.

Questo protocollo di fatto implementa un servizio di \textbf{posta elettronica} che si appoggia su
componenti \emph{intermediari} per trasferire i messaggi. Gli intermediari, in questo caso, sono i
\textbf{mail server} che archiviano le mail sul server.

Per comunicare con il mail server si utilizzano user agent come Outlook o Thunderbird, i quali ci
permettono di inviare, editare e leggere le varie mail.

Il server si compone di tante \textbf{mailbox} quanti sono gli utenti connessi a quel mail server,
contenenti messaggi da leggere e di una coda di  messaggi in uscita, contenente messaggi che devono
ancora essere inviati.

Se due utenti sono connessi a due mail server differenti, questi comunicano tra di loro
scambiandosi le email.

La comunicazione tra mail server non utilizza il protocollo SMTP, è il client che per richiedere
di inviare una mail al mail server, deve usare tale protocollo.

\subsection{Funzionamento}
Un mail server, per sapere a quale server deve inviare la mail, ha bisogno dell'\textbf{indirizzo
email} composto in questo modo
\begin{center} \verb|local-part@domain-name| \end{center}
dove \verb|domain-name| specifica un mail server determinando il nome di dominio di una
destinazione a cui la mail dovrebbe essere recapitata. La \verb|local-part| specifica la mailbox
nel mail server. I passaggi compiuti dal protocollo sono i seguenti:
\begin{enumerate}
	\item L'utente, tramite uno user agent fa una richiesta al suo mail server di inviare una mail
		specificando un destinatario tramite un indirizzo.
	\item Il mail server prende in carico la mail e ne salva una copia in locale. A questo punto
		si presentano due scenari:
		\begin{itemize}
			\item Il dominio specificato nell'indirizzo non è quello del server e quindi la mail
				viene inoltrata ad un altro mail server.
			\item Il dominio specificato è quello del mail server e dunque la mail viene inserita
				nella mailbox del destinatario.
		\end{itemize}
	\item Il destinatario, tramite uno user agent, accede al suo server di mail che gli fornisce
		la mail.
\end{enumerate}
Ogni volta che una mail viene inoltrata da un server, questo ne tiene una copia in locale fin
quando non riceve una notifica di avvenuta ricezione dal destinatario.

\subsubsection{Alias}
L'\textbf{alias} è una cassetta di posta \emph{virtuale} che serve a ridistribuire i messaggi
verso uno o più indirizzi di posta elettronica personali. L'alias permette
\begin{itemize}
	\item Ad un singolo utente di avere identificatori multipli, assegnando un set di
		identificatori ad una singola persona.
	\item Di associare un gruppo di destinatari ad un singolo identificatore.
\end{itemize}
Per riuscire a fare questo serve un \textbf{espansore degli alias}, ossia un componente che, una 
volta arrivata una mail, verifica consultando un \textbf{database degli alias} se deve espandere
gli alias.

Supponiamo che venga inviata una mail ad un gruppo di utenti. L'espansore non fa altro che
espandere l'indirizzo mail (arrivato sotto forma di alias) negli indirizzi veri e propri di tutti
gli utenti ottenuti tramite l'espansione.

Nell'altro caso è ad esempio possibile definire un alias per l'indirizzo mail di un certo utente.
L'espansore non fa altro che riscrivere l'indirizzo mail associato a quell'alias.

\subsection{Trasferimento}
Il protocollo di trasferimento di SMTP è di tipo \textbf{push}, si cerca infatti di inserire un 
messaggio in una coda (all'interno dei vari server). L'obbiettivo di SMTP è affidabilità ed
efficienza, ecco perché il protocollo di trasporto utilizzato è TCP.

\subsubsection{Protocollo}
Nella pratica SMTP usa il protocollo TCP per consegnare in modo affidabile messaggi dal client al
server. Il trasferimento si divide in tre fasi:
\begin{enumerate}
	\item \textbf{Handshaking}: il client stabilisce la connessione con il server e attende il
		codice \verb|220 READY FOR MAIL|. Il client risponde con il comando \verb|HELO| e il server
		risponde identificandosi. A questo punto il client può trasmettere il messaggio.
	\item \textbf{Trasferimento messaggio}
	\item \textbf{Chiusura della connessione}
\end{enumerate}
L'interazione è di tipo comando/risposta. Il comando è un testo ASCII mentre la risposta è un
codice di stato e descrizione. Il messaggio vero e proprio è del testo con caratteri ASCII a 7 bit.

\subsubsection{Comandi}
Tra i comandi che il protocollo SMTP mette a disposizione abbiamo:
\begin{itemize}
	\item \verb|HELO <client identifier>|: identificativo del client.
	\item \verb|MAIL FROM:<reverse-path><CRLF>|: identificativo di chi invia la mail.
	\item \verb|RCPT TO:<forward-path><CRLF>|: identificativo di chi riceve la mail.
	\item \verb|DATA|: inizio del trasferimento del messaggio.
	\item \verb|QUIT|: chiusura della sessione.
\end{itemize}
Per determinare la fine del messaggio si usa \verb|<CRLF>|. Uno degli standard di formato per
messaggi SMTP è il RFC 2822 che prevede
\begin{itemize}
	\item \textbf{Intestazione}: nella quale specificare a chi si vuole inviare il messaggio, chi
		sta inviando il messaggio e il soggetto del messaggio.
	\item \textbf{Corpo}: in cui è presente il messaggio vero e proprio.
\end{itemize}

\subsection{MIME}
Il protocollo appena descritto permetteva di inviare messaggi con caratteri ASCII a 7 bit e quindi
tagliava fuori caratteri accentati, caratteri provenienti da alfabeti differenti o simboli tipici
di lingue senza alfabeto (come il cinese). Inoltre non era possibile inviare contenuti multimediali
che richiedevano un formato binario.

Nasce quindi il \textbf{MIME} (Multipurpose Internet Mail Extension) la cui idea era quella di
continuare a usare il formato specificato dallo standard RFC 822 ma aggiungendo una struttura al
corpo del messaggio e definendo regole di  codifica (\textbf{encoding}) per il trasferimento del
testo in ASCII.

In questo modo si è riusciti ad inviare messaggi MIME usando protocolli e mail server già
esistenti andando a definire un insieme di metodi per rappresentare i dati binari in formato ASCII
nel messaggio inviato. Quello che viene fatto nella pratica è questo:
\begin{enumerate}
	\item Il messaggio viene scritto con la codifica desiderata.
	\item Si specifica la versione MIME utilizzata.
	\item Vengono specificati i parametri che definiscono come il messaggio è stato codificato e
		il tipo di messaggio che si desidera inviare.
	\item Il modulo MIME codifica il messaggio in formato ASCII a 7 bit e lo invia al mail server.
	\item Una volta che il messaggio è stato recapitato, il destinatario utilizza il proprio modulo
		MIME e, tramite la codifica specificata, decodifica il messaggio riportandolo nel formato
		corretto.
\end{enumerate}

\subsection{Protocolli di accesso alla mail}
Una volta che la mail viene inserita nella mailbox del destinatario questa non gli viene inoltrata
ma rimane lì finché non è lui a richiederla.

Il motivo risiede nell'idea che sta dietro al protocollo SMTP, ossia poter accedere la mailbox in
un secondo momento rispetto all'invio senza dover stare sempre in ascolto.

Ecco che vi è la necessità di definire un nuovo protocollo che permette al destinatario di
richiedere la mail nella mailbox del server. I protocolli messi a disposizione sono 3:
\begin{itemize}
	\item \textbf{POP3}: si apre una connessione TCP con il server, seguono fasi di autorizzazione,
		scambio e aggiornamento.
	\item \textbf{IMAP}: più evoluto di POP3, offre più funzionalità per la gestione della posta.
	\item \textbf{HTTP}: se lo user agent è un browser.
\end{itemize}
Non di particolare interesse, semplicemente sono definiti comandi specifici per poter interagire
con il server e gestire la propria posta.
