\section{World Wide Web}
Possiamo definire il \textbf{Web} come un sistema di condivisione di 
informazioni tramite collegamenti in rete.

Una pagina web è tipicamente formata da un file HTML di base, da 
diversi oggetti referenziati (altre pagine, file multimediali ecc.).
Ciascuno di questi oggetti è \emph{indirizzabile} tramite una 
\textbf{URL} (Uniform Resource Locator).

\subsection{URI, URL e URN}
Quando si parla di \textbf{URI} (Uniform Resource Identifier) si fa 
riferimento all'identificativo di una risorsa sulla rete. Cerchiamo 
di capire meglio perché si chiama così:
\begin{itemize}
	\item \textbf{Uniform}: uniformità della sintassi 
		dell'identificatore anche se i meccanismi per accedere alle 
		risorse possono variare.
	\item \textbf{Resource}: qualsiasi cosa abbia un'\emph{identità}.
	\item \textbf{Identifier}: informazioni che permettono di 
		distinguere un oggetto dall'altro.
\end{itemize}
Un tipo di URI è la \textbf{URN}, ossia un sottoinsieme di URI che 
devono rimanere globalmente unici e persistenti anche quando la 
risorsa cessa di esistere.

Il tipo di URI che invece ci interessa trattare per andare a 
comprendere meglio il protocollo HTTP è la \textbf{URL}, ossia un sotto
insieme di URI che identifica le risorse attraverso il loro meccanismo 
di accesso. La sintassi di una URL è la seguente
\begin{center}
	\verb|<scheme>://<user>/<password>@<host>:<port>/<path>| 
\end{center}
Vediamo a cosa corrispondono ognuno di questi nomi:
\begin{itemize}
	\item \textbf{Schema}: indica il meccanismo di accesso alla risorsa
		(HTTP, FTP).
	\item \textbf{Utente} e \textbf{password}: sono opzionali e ad 
		oggi deprecati.
	\item \textbf{Host}: indica il nome di un dominio o l'indirizzo IP 
		(www.google.it).
	\item \textbf{Porta}: numero di porta opzionale in quanto ad ogni 
		servizio è associata una porta di default (HTTP sulla porta 
		80).
	\item \textbf{Path}: percorso per accedere alla risorsa sulla 
		macchina host in cui risiede la risorsa.
\end{itemize}
Una URL può essere \textbf{assoluta} o \textbf{relativa}:
\begin{itemize}
	\item \textbf{Assoluta}: identifica una risorsa indipendetemente 
		dal contesto in cui si viene usata.
	\item \textbf{Relativa}: informazioni per identificare una risorsa
		in relazione ad un'altra URL (è priva di schema e di 
		authority).
\end{itemize}
Le URL relative non viaggiano in rete ma sono interpretate dal browser 
in relazione al documento di partenza.
