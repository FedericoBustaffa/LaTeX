\section{FTP}
Ultimo protocollo che vediamo per lo strato applicativo è \textbf{FTP}.
Questo protocollo nasce dalla necessità di trasferire file da/verso un
host remoto.

Adotta un paradigma di tipo client/server in cui il client richiede
il trasferimento al server remoto. Si tratta di un servizio diverso
dall'accesso condiviso on-line. In questo caso si ottiene una copia
locale ed eventualmente si ha un trasferimento del file modificato
sul server remoto.

Il servizio fornisce funzionalità aggiuntive rispetto al semplice 
trasferimento file:
\begin{itemize}
	\item Navigazione e modifica del filesystem remoto.
	\item Specifica del formato dei dati da trasferire.
	\item Il client può specificare login e password.
\end{itemize}

\subsection{Modello FTP}
Il modello FTP si compone di due tipi di connessione diversi:
\begin{itemize}
	\item \textbf{Control connection}: scambio di comandi e risposte 
		tra client e server (TELNET).
	\item \textbf{Data connection}: connessione su cui i dati sono 
		trasferiti con modi e tipi specificati. I dati trasferiti 
		possono essere parte di un file o un set di file.
\end{itemize}
Per entrambe le connessioni il protocollo di trasporto usato è TCP e, a
differenza di HTTP, FTP è protocollo di tipo \textbf{stateful}, in 
quanto il server deve tenere traccia dello stato in cui si trova 
l'utente.

\subsubsection{Connessione di controllo}
La connessione di controllo avviene nel seguente modo:
\begin{enumerate}
	\item Il client contatta il server FTP alla porta 21.
	\item Il client ottiene l'autorizzazione sulla connessione di 
		controllo.
	\item Il client invia i comandi sulla connessione di controllo.
\end{enumerate}
La connessione di controllo è \textbf{persistente}.

\subsubsection{Connessioni dati}
Quando il server riceve un comando per trasferire un file (da o verso 
il client) sulla connessione di controllo
\begin{enumerate}
	\item Il server apre una connessione TCP con il client sulla 
		porta 20 (\textbf{Active Mode}).
	\item Il server trasferisce il file sulla connessione dati.
	\item Dopo il trasferimento di un file, il server chiude la 
		connessione.
\end{enumerate}
La connessione dati non \textbf{non è persistente}.

\subsubsection{Active vs Passive Mode}
Nella modalità attiva appena descritta è il server ad aprire la 
connessione TCP verso il client. Per riuscire a farlo deve essere a 
conoscenza della porta su cui il client è in ascolto, ed è proprio 
quest'ultimo a comunicargliela con la connessione di controllo. In
alternativa è possibile stabilire la connessione dati in 
\textbf{Passive Mode}:
\begin{enumerate}
	\item Il client chiede al server di di mettersi in ascolto su una
		qualche porta (non specificata).
	\item Il server fornisce il numero di porta al client che lo usa 
		per aprire la connessione.
\end{enumerate}
Questo metodo può essere più solido dal punto di vista della sicurezza.

\subsubsection{Conclusioni}
I dipositivi dove risiedono client e server FTP sono diversi, ecco che 
il client, per risolvere i problemi di eterogeneità tra lui e il
server, deve definire il tipo di file, la struttura dati e la modalità 
di trasmissione.

Tutto questo viene preparato attraverso uno scambio di informazioni 
lungo la connessione di controllo. Tra le modalità di trasmissione dati
disponibili abbiamo diverse opzioni:
\begin{itemize}
	\item \textbf{Stream}: flusso di bit continuo.
	\item \textbf{Block}: dati suddivisi in blocchi, ognuno dei quali 
		preceduto da un header.
	\item \textbf{Compressed}: trasmissione di un file compresso.
\end{itemize}

\subsection{Comandi FTP}
Come ogni protocollo visto fino ad ora abbiamo una serie di comandi FTP
che si dividono in \textbf{comandi di controllo}:
\begin{itemize}
	\item \verb|USER <username>|
	\item \verb|PASS <password>|
	\item \verb|LIST|: elenco file nella directory corrente.
	\item \verb|NLST|: elenco file e directory nella directory 
		corrente.
	\item \verb|RETR <filename>|: recupera un file dalla directory 
		corrente.
	\item \verb|STOR <filename>|: memorizza un file nell'host remoto.
	\item \verb|ABOR|: interrompe l'ultimo comando ed i trasferimenti 
		in corso.
	\item \verb|PORT|: indirizzo e numero di porta del client.
	\item \verb|SYST|: tipologia di sistema server.
	\item \verb|QUIT|: chiude la connessione.
\end{itemize}
Il server risponde a questi comandi tramite \textbf{codici} di 
risposta che non staremo ad elencare ma tra i quali è presente anche 
il 227 che indica che il server è entrato in Passive Mode.

\subsection{Conclusioni}
Il trasferimento può essere anonimo se il server FTP supporta 
connessioni senza autenticazione. Tipicamente si restringe la porzione 
di filesystem accessibile per questi utenti e si limitano i comandi 
utilizzabili.

Il servizio FTP, di per sé, non offre un sistema di sicurezza, ecco che
viene utilizzato spesso in coppia con il protocollo TLS andando a 
definire un protocollo a tutti gli effetti che è il protocollo FTPS.
