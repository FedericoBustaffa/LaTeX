\section{HTTP}
L'\textbf{HTTP} (Hyper Text Transfer Protocol) è il protocollo generalmente usato per navigare in
rete e dunque per fare richieste di contenuti sul web. Si tratta di un protocollo
\textbf{stateless}, ossia senza memoria di stato, in quanto le coppie richiesta/risposta sono
indipendenti.

Il protocollo non è limitato all'ipertesto ma supporta lo scambio di informazioni di qualsiasi 
genere. Questo grazie un sistema di tipizzazione e negoziazione sulla rappresentazione dei dati.

Il protocollo utilizza TCP come servizio di trasporto in quanto non può permettersi la perdita di
dati.

Per il protocollo HTTP possiamo andare ad aggiungere un ulteriore parametro alle URL, ossia la
\verb|query|, una stringa di informazioni che deve essere interpretata dal server.
\begin{center} \verb|http://<host>:<port>/<path>?<query>| \end{center}
In questo modo possiamo andare a manipolare la risorsa richiesta (sempre nei limiti di ciò che il
server ci rende disponibile).

\subsection{Connessioni}
Una \textbf{connessione} è un circuito logico di livello trasporto stabilito tra due programmi
applicativi per comunicare. Le connessioni possono essere
\begin{itemize}
	\item \textbf{Non persistenti}: ad ogni richiesta del client viene aperta una nuova connessione
		TCP.
	\item \textbf{Persistenti}: la connessione TCP non viene chiusa fin quando non viene conclusa 
		la sessione di scambio delle informazioni.
\end{itemize}
Ad oggi HTTP implementa un tipo di connessione persistente che è diventato uno standard de facto.

\subsubsection{Pipelining}
Il \textbf{pipelining} serve per migliorare ulteriormente le prestazioni. Nella pratica consiste 
nell'invio, da parte del client, di molteplici richieste senza aspettare la ricezioni di nessuna
risposta. Il server deve inviare le risposte nello stesso ordine in cui sono arrivate le richieste.

Il client non può inviare in pipeline richieste che usano metodi HTTP non \emph{idempotenti}, ossia
operazioni che, anche se fatte più volte, non variano il risultato finale.

Il problema principale di questo sistema è il vincolo sull'ordine di risposta, in quanto si
rischiano rallentamenti pesanti nel caso in cui una richiesta che necessita un tempo lungo di 
elaborazione è la prima da soddisfare.

\subsection{Messaggi HTTP}
Un \textbf{messaggio} HTTP può essere di due tipi: \textbf{request} o \textbf{response} ed è la
riga iniziale che distingue la richiesta dalla risposta. Seguono una serie di \textbf{header},
ossia coppie nome/valore. Infine (anche se opzionale) c'è il corpo del messaggio. In realtà
la risposta può essere anche solo la pagina html richiesta. La \textbf{request line} è composta da
tre parti principali:
\begin{itemize}
	\item \textbf{Metodo}: l'operazione che il client richiede venga eseguita.
	\item \textbf{Request URI}: l'oggetto dell'operazione specificata con il metodo.
	\item \textbf{Version HTTP}: formato del messaggio e capacità di comprendere ulteriori
		comunicazioni.
\end{itemize}
Come abbiamo detto, nei messaggi HTTP possono essere specificati anche una serie di \emph{header},
i quali servono a specificare alcuni parametri del messaggio trasmesso o ricevuto. Possiamo avere 
header di vario tipo:
\begin{itemize}
	\item \textbf{Generali}: contengono informazioni sulla trasmissione come data, codifica o
		connessione
	\item \textbf{Entità}: relativi all'entità trasmessa come la lunghezza e il tipo del contenuto
		trasmesso.
	\item \textbf{Richiesta}: relativi alla richiesta come informazioni su chi fa la richiesta, a 
		chi viene fatta la richiesta, autorizzazioni ecc.
	\item \textbf{Risposta}: si trova nel messaggio di risposta e contiene informazioni sul server
		come ad esempio l'autorizzazione richiesta.
\end{itemize}
La \textbf{response line} è invece composta in questo modo:
\begin{itemize}
	\item \textbf{Status line}: contiene tre informazioni:
		\begin{itemize}
			\item Versione HTTP.
			\item Lo \emph{status code} (codice di 3 cifre).
			\item Una descrizione testuale del codice di stato.
		\end{itemize}
	\item \textbf{headers}: header informativi sulla risposta che non possono essere inseriti
		nella status line.
\end{itemize}
Gli status codes si dividono in 5 categorie:
\begin{itemize}
	\item 1xx: informativi.
	\item 2xx: successo.
	\item 3xx: redirezione ad azioni necessarie per completare correttamente la richiesta.
	\item 4xx: errori del client nella richiesta.
	\item 5xx: errori del server nel gestire una richiesta apparentemente corretta.
\end{itemize}
Sono i codici informativi che vengono inseriti nel messaggio di risposta dal server.

\subsubsection{Metodi di richiesta}
Nel messaggio di richiesta, come detto in precedenza, dobbiamo inserire un metodo, a seconda del 
metodo, il messaggio di risposta sarà differente:
\begin{itemize}
	\item \verb|OPTIONS|: la risposta conterrà i metodi disponibili per l'URI corrente.
	\item \verb|GET|: richiede il trasferimento di una risorsa identificata da una URL e operazioni
		associate alla URL stessa. Sono possibili anche delle \emph{conditional get} in cui si
		aggiungono delle condizioni sulla risorsa richiesta.
	\item \verb|HEAD|: simile a \verb|GET| restituisce solo l'intestazione del messaggio.
	\item \verb|POST|: metodo usato dal client per inviare contenuti (nel corpo del messaggio) al
		server.
	\item \verb|PUT|: richiesta di creare o modificare una risorsa (sono necessari i permessi).
	\item \verb|DELETE|: si chiede di eliminare qualcosa nella risorsa.
\end{itemize}
I metodi si dividono in \textbf{safe} e \textbf{idempotenti}. I metodi safe non hanno effetti
collaterali, ossia non si modifica in alcun modo la risorsa. I metodi idempotenti sono metodi che
possono avere effetti collaterali ma, anche se eseguite diverse volte hanno lo stesso effetto di
una richiesta singola.

\subsection{Web Caching}
L'obbiettivo del \textbf{web caching} è quello di soddisfare le richieste del client senza
contattare il server. Per farlo, il client può mantenere delle copie locali delle risorse che aveva
ottenuto in precedenza da altre richieste.
In alternativa è possibile utilizzare un \textbf{proxy} che intercetta il traffico e lo inoltra al 
server. Una volta ricevuta una risposta ne mantiene una copia e sarà quindi lui a fornire la
risorsa al client.

\subsection{Cookies}
Come abbiamo detto HTTP è stateless, ossia non mantiene informazioni sul client. Con questo tipo
di implementazione del protocollo nasce però un problema, che è quello di riconoscere un utente
di un'applicazione Web. Si devono quindi creare applicazioni web con stato tenendo di conto che 
ogni utente si connette ogni volta con indirizzo IP e porta differenti.

Ecco che viene creato il meccanismo dei \textbf{cookie}, il quale permette di creare delle sessioni
sopra un protocollo stateless:
\begin{enumerate}
	\item Il client invia una richiesta al server.
	\item Il server manda un messaggio di risposta con una linea
		\begin{center} \verb|Set-cookie : <id>| \end{center}
	\item Il client memorizza il cookie in un file associandolo al server e lo aggiunge con una 
		linea
		\begin{center} \verb|cookie : <id>| \end{center}
		a tutte le sue successive richieste a quel sito.
	\item Il server confronta il cookie presentato con l'informazione associata ad esso.
\end{enumerate}
I cookie vengono utilizzati per autenticazione, ricordare il profilo utente, scelte precedenti ecc.
