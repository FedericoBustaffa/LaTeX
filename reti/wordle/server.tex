\section{Server}
Il \textbf{server}, dopo l'avvio si occupa di leggere il file json di configurazione
\verb|server_config.json| e di inizializzare le strutture necessarie per la connessione da parte
dei client:
\begin{enumerate}
	\item Ripristina la lista degli utenti tramite il file \verb|users.json|.
	\item Tramite tale lista crea una classifica degli utenti in ordine di punteggio.
	\item Crea un'istanza di un oggetto \verb|Wordle| che si occupa di gestire le interazioni
	      dell'utente prettamente relative al gioco.
	\item Crea un thread \emph{estrattore} che si occupa di estrarre una parola dal vocabolario
	      ogni qual volta scade un timeout.
	\item Pubblica l'oggetto remoto accessibile tramite RMI dai client per effettuare l'operazione
	      di registrazione.
	\item Inizializzazione di un server TCP tramite Java NIO.
	\item Inizializzazione di un socket multicast per inviare messaggi sul gruppo sociale.
\end{enumerate}

\subsection{Strutture dati}
Dato che il server fa uso di \textbf{multi-threading} per una maggiore efficienza nell'esecuzione
delle richieste dei vari client, sono necessarie strutture dati che garantiscano l'accesso in
mutua esclusione ai diversi thread.

\subsubsection{Utenti}
Prima fra tutti è la \textbf{tabella hash} che gestisce gli utenti. Ho deciso di implementarla
tramite una \verb|ConcurrentHashMap| le cui chiavi fossero gli \verb|username| degli utenti (poiché
univoci per ogni utente) e i relativi valori fossero istanze della classe \verb|User| contenente
tutti i vari dati relativi allo stato dell'utente e al punteggio di quest'ultimo.

\subsubsection{Classifica}
La seconda struttura dati è la classifica degli utenti in base al loro punteggio, implementata
tramite una \verb|LinkedList| e sincronizzata con il metodo \verb|synchronizedList| della
classe \verb|Collections|.

\subsubsection{Notificatori}
Si tratta di una lista sincronizzata di oggetti remoti, i quali implementano la classe
\verb|NotifyService|, passati dai client durante la fase di login. Ogni qual volta c'è un
cambiamento nelle prima tre posizioni della classifica il server scorre tutta la lista notificando
i client tramite tali oggetti.

In fase di logout l'utente invoca un metodo che si occupa di rimuovere il \emph{notificatore}
a esso associato.

\subsubsection{Sessioni}
L'altra struttura dati utilizzata è una \verb|ConcurrentHashMap| che mette in correlazione un
utente con un'istanza della class \verb|Session|, la quale incapsula la parola che l'utente sta
cercando di indovinare, il numero di tentativi che questo ha fatto per indovinarla e altre
informzioni relative allo stato della sessione. I motivi dietro a questa scelta sono molteplici:
\begin{itemize}
	\item In primo luogo ho deciso di gestire la casistica in cui un utente si ritrovi a giocare
	      con una certa parola e nel frattempo il server ne estrae una nuova. In quel caso la
	      vecchia parola andrebbe persa e quindi l'utente non potrebbe terminare la partita. Ecco
	      che nasce quindi la necessità di creare delle \emph{sessioni} in grado di tenere
	      traccia della partite ancora in corso in modo che ogni utente sia in grado di
	      terminarle a prescindere da nuove parole estratte.
	\item Altra casistica d'interesse è quella in cui un utente voglia condividere il risultato
	      finale di una partita appena fatta. Avendo a disposizione una struttura dati simile è
	      possibile, tramite il proprio nome utente, accedere all'ultima sessione presente e
	      condividerla senza che vada persa in seguito ad una successiva estrazione da parte del
	      server.
\end{itemize}
Per alleggerire il carico del server avrei potuto creare la sessione e inviarla al client come
messaggio di risposta. Sarebbe stato poi compito del client aggiornarla man mano che inviava le
\verb|guessed words|. Questo implica però l'invio della parola segreta al client andando così ad
aprire la strada a possibili violazioni.

\subsection{Wordle}
Una delle entità principali lato server cuore del gioco è la classe \verb|Wordle| la cui istanza
contiene strutture dati e metodi fondamentali per la corretta interazione tra utente e server
quando si compiono azioni prettamente di gioco.

In particolare è la classe \verb|Wordle| che si occupa di gestire le sessioni di gioco nel modo
corretto e qualora un utente fare dei tentativi per indovinare la parole segreta si occupa di
verificare se il tentativo è corretto o, in caso contrario, di fornire i suggerimenti all'utente.

Tutto questo è affiancato da un thread \emph{estrattore}, il quale si occupa di estrarre una nuova
parola casuale ogni qual volta scade il timeout.

\subsection{Concorrenza}
Lato server si fa uso di multi-threading per riuscire a gestire contemporaneamente molteplici
richieste utente. Diventa quindi necessario proteggere le strutture dati soggette a conflitti
tramite meccanismi di concorrenza.

Come è possibile notare, tutte le strutture dati coinvolte, sono sincronizzate o concorrenti. I
thread principalmente coinvolti nella modifica delle strutture dati sono di tre tipi:
\begin{itemize}
	\item \textbf{Thread RMI}: quando più client interagiscono contemporaneamente con l'oggetto
	      remoto per effettuare una registrazione, vengono avviati più thread per gestire le varie
	      richieste.
	\item \textbf{Receiver}: quando il server deve ricevere dati su di una connessione TCP,
	      avvia un thread \verb|Receiver| in grado di leggere il messaggio inviato sulla
	      connessione TCP e svolgere le operazioni richieste dal client.
	\item \textbf{Sender}: quando il server deve inviare dati sulla connessione TCP (e non solo),
	      avvia un thread \verb|Sender| in grado di rispondere ai client non solo tramite TCP ma
	      anche tramite altri protocolli di comunicazione, a seconda dei casi.
\end{itemize}
Tutti questi thread possono modificare attivamente le strutture dati precedentemente elencate
oppure possono essere soltanto interessati a leggere i dati al loro interno. In ogni caso è
necessario che i dati all'interno di tali strutture siano sempre consistenti.

\subsection{Comunicazioni}
Lato server abbiamo diversi protocolli di comunicazione in grado di fornire servizi agli utenti.
Per quanto possibile ho provato ad affidare le operazioni di lettura solo al thread \verb|Receiver|,
mentre il thread \verb|Sender| si occupa di tutto ciò che il server deve inviare (a prescindere
dal protocollo).

\subsubsection{RMI}
Tramite \verb|RMI| il server pubblica un oggetto remoto implementato dalla classe
\verb|RegistrationService|, la quale mette a disposizione dell'utente un metodo (\verb|register|)
che va ad aggiungere nuovi utenti alla tabella hash degli utenti mantenuta dal server.

Tale oggetto contiene altri due metodi per la registrazione o la cancellazione di un utente dal
servizio di notifica che si occupa di notificare all'utente cambiamenti in classifica nelle prime
tre posizioni. Questi due metodi non fanno altro che aggiungere o rimuovere un oggetto remoto,
implementato dalla classe \verb|NotifyService|, dalla list dei \emph{notificatori} di cui ho
parlato nel paragrafo relativo alle strutture dati.

\subsubsection{TCP}
Per quanto riguarda le connessioni TCP il server fa uso dei \verb|Channel| per comunicare e sfrutta
il meccanismo del \verb|Selector| alleggerire l'impiego di risorse necessarie alla gestione delle
varie connessioni TCP con i client.

Il modo in cui ho deciso di implementare la gestione delle connessioni TCP da parte del server
prevede che, in seguito alla possibilità di inviare o ricevere dati, questo avvii o un
\verb|Receiver| o un \verb|Sender|.

Per evitare errori è necessario, ogni qual volta si avvia un thread, azzerare le operazioni di
interesse per quel \verb|channel| registrato sul \verb|selector|. Sarà il thread, una volta
terminate le operazioni, a resettare l'operazione di interesse (lettura o scrittura) e a svegliare
il \verb|selector| tramite il metodo \verb|wakeup|.

Questo è necessario poiché il \verb|selector| continua a vedere il \verb|channel| come pronto a
leggere o scrivere dato che è possibile che venga invocata nuovamente il metotodo \verb|select|
prima che il thread abbia terminato il suo compito.

\subsubsection{UDP}
Il server fa anche utilizzo di socket multicast usato per inviare i risultati delle partite che
gli utenti vogliono condividere tramite messaggi UDP.

Inoltre, in caso l'utente effettui il logout o la disconnessione dal server, quest'ultimo invia
un messaggio sul gruppo multicast che segnali all'utente in questione che può terminare la
connessione su tale gruppo.

\subsubsection{Attachment}
Ogni volta che un canale viene registrato sul \verb|selector|, gli viene allegato un oggetto di
tipo \verb|Attachment| contenente tutte le strutture dati necessarie a svolgere i propri compiti
e un \verb|ByteBuffer| necessario alla lettura e scrittura da e sui canali.

Il \verb|ByteBuffer| è anche usato come mezzo di comunicazione tra \verb|Receiver| e
\verb|Sender|. Ogni qual volta \verb|Receiver| riceve un comando elabora una risposta ma si limita
soltanto a scriverla sul \verb|ByteBuffer|.

Il \verb|Sender| si occupa di inviare la risposta sulla connessione TCP, analizzando prima il suo
contenuto e nel caso sia necessario, utilizzando RMI per inviare notifiche o invia un messaggio
UDP al gruppo di multicast.

\subsection{Spegnimento}
In avvio il server genera un thread in attesa di un input da tastiera. Per spegnere il server è
sufficiente recarsi nel terminale su cui questo è in esecuzione e premere \verb|INVIO|. In questo
modo il thread avviato in precedenza setterà la variabile booleana \verb|running| a \verb|false|.

Se ci sono ancora client connessi il server continuerà fornire i soliti servizi ma non permetterà
ad ulteriori client di connettersi.

Una volta che tutti i client si saranno disconnessi, il server terminerà la sua esecuzione tramite
il metodo \verb|shutdown| il quale si occuperà di chiudere tutte le strutture di comunicazione
in funzione e si occuperà di scrivere sul file json \verb|users.json| un backup degli utenti con
relative statistiche.