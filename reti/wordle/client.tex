\section{Client}
Per quanto riguarda il client ho deciso di mantenerlo più semplice possibile. Non si occupa di
svolgere attivamente molti compiti. Per quanto possibile ho lasciato che il server si occupasse
interamente della gestione delle richieste e dell'elaborazione delle risposte.

\subsection{Strutture dati}
Il client non mantiene molte strutture dati in quanto tutto ciò di cui necessita gli viene fornito
dal server tramite richieste.

\subsubsection{Risultati partite}
Unica struttura dati di rilievo presente lato client è quella in cui vengono memorizzati i
messaggi in arrivo sul gruppo sociale. Tale lista è gestita dal thread che si occupa di stare in
ascolto sul gruppo e ogni qual volta arriva un nuovo messaggio contenente la partita di un altro
utente lo memorizza in una \verb|LinkedList| sincronizzata.

\subsection{Concorrenza}
La concorrenza in questo caso è minima poiché l'unico thread avviato in aggiunta a quello
principale è il \verb|MulticastReceiver|. Esso si occupa di stare in ascolto sul gruppo di
multicast e, in caso riceva risultati di partite condivisi da altri utenti li aggiunge alla lista
dei risultati di cui abbiamo parlato nel paragrafo dedicato alle strutture dati.

Nel caso in cui ci siano utenti che condividono i loro risultanti mentre l'utente richiede di
visionare il contenuto della lista, potrebbero verificarsi conflitti e/o inconsistenze. \`E quindi
necessario rendere la lista sincronizzata per avere un accesso in mutua esclusione alla struttura.

\subsection{Comunicazioni}
I protocolli di comunicazione, come per il server, sono molteplici, ognuno necessario diversi
compiti.

\subsubsection{RMI}
Il client, una volta avviato, accede al servizio remoto pubblicato dal server per essere in grado
di registrarsi tramite l'interfaccia \verb|Registration|.

Se si effettua il login utilizza la medesima interfaccia per registrarsi al servizio di notifica
inviando un'istanza della classe \verb|NotifyService|. Dato che più thread lato server potrebbero
usare tale oggetto per inviare notifiche al client è necessario rendere i suoi metodi
sincronizzati.

\subsubsection{TCP}
Il client non utilizza Java NIO per le connessioni TCP ma dei semplici \verb|Socket| con i quali
si connette al server ed è in grado di inviare e ricevere messaggi in forma testuale tramite i
metodi \verb|send| e \verb|receive|.

\subsection{Spegnimento}
Ci sono due metodi per riuscire ad chiudere l'applicazione correttamente:
\begin{itemize}
	\item \verb|exit|: invia una comando di \verb|exit| al server che provvede ad effettuare il
	      logout se necessario e a decrementare il numero di connessioni attive.

	      Il client termina poi la sua esecuzione chiudendo le varie connessioni tramite il metodo
	      shutdown.
	\item \verb|SIGINT|: se si invia una segnale di \verb|SIGINT| al processo questo non fa altro
	      che scatenare l'invio di un comando di \verb|exit| al server generando lo stesso effetto
	      descritto in precedenza,
\end{itemize}
