\section{World Wide Web}
Possiamo definire il \textbf{Web} come un sistema di condivisione di informazioni tramite
collegamenti in rete.

Una pagine web, tipicamente è formata da un file HTML di base, da diversi oggetti referenziati
(altre pagine, file multimediali ecc.) e ciascuno di questi oggetti è \emph{indirizzabile} tramite
una \textbf{URL} (Uniform Resource Locator).

\subsection{URI, URL e URN}
Quando si parla di \textbf{URI} (Uniform Resource Identifier) si fa riferimento all'identificativo
di una risorsa sulla rete. Cerchiamo di capire meglio perché si chiama così:
\begin{itemize}
	\item \textbf{Uniform}: uniformità della sintassi dell'identificatore anche se i meccanismi
		per accedere alle risorse possono variare.
	\item \textbf{Resource}: qualsiasi cosa abbia un'\emph{identità}.
	\item \textbf{Identifier}: informazioni che permettono di distinguere un oggetto dall'altro.
\end{itemize}
Un tipo di URI (quello più usato e conosciuto) è l'\textbf{URL}, ossia un sottoinsieme di URI che 
identifica le risorse attraverso il loro meccanismo di accesso.

Altro tipo di URI l'\textbf{URN}, ossia un sottoinsieme di URI che devono rimanere globalmente
unici e persistenti anche quando la risorsa cessa di esistere.
