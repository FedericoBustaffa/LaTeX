\section{IP}
Il \textbf{protocollo IP} (Internet Protocol) è un servizio di tipo 
connectionless in quanto non c'è un circuito né logico né fisico tra i
due sistemi terminali.

Il servizio offerto non da alcuna garanzia di consegna dei datagrammi,
che l'ordine di invio sia rispettato e inoltre non è presente alcun 
meccanismo di recupero di eventuali datagrammi andati persi.

Quello che fa IP quando riceve un segmento TCP o un datagramma UDP è
aggiungere una sua intestazione a supporto dell'implementazione dei
servizi.

\subsection{Formato del datagramma IP}
Il formato del datagramma IP prevede due parti, l'header e il payload. 
L'header ha una dimensione variabile tra 20 e 60 byte e l'intero
datagramma può avere una dimensione totale di 65535 byte. L'header si
compone dei seguenti campi:
\begin{itemize}
	\item Versione IP.
	\item Lunghezza header.
	\item Tipo di servizio.
	\item Lunghezza totale del datagramma.
	\item Identificativo a 16 bit.
	\item TTL: massimo numero \emph{hop} rimanenti (decrementato ogni
		volta che passa da un router).
	\item Flag per la frammentazione.
	\item Checksum.
	\item Demultiplexing.
	\item IP mittente e destinatario.
	\item Opzioni.
\end{itemize}
I bit indicati nel campo "tipo di servizio" vanno ad indicare le 
caratteristiche del datagramma in base alla \emph{classe} del servizio
(telefonata, streaming, dati a bassa priorità ecc.) così da 
implementare politiche di accodamento diverse.

Alcuni dei bit in questo campo servono a notificare eventi di
congestione a livello rete e trasporto.

\subsection{Frammentazione}
L'operazione di \textbf{frammentazione} consiste nella suddivisione dei
datagrammi IP in datagrammi più piccoli, ognuno con la propria
intestazione e payload.

Il motivo per cui si ricorre alla frammentazione risiede in un 
problema dovuto alla capacità del metodo di collegamento. Quando 
abbiamo parlato del TCP abbiamo introdotto la Maximum Segment Size,
come la dimensione massima di un segmento TCP tale che possa stare
dentro un frame che rispetta il vincolo di Maximum Transfer Unit, il
quale dipende a sua volta dalla tecnologia di collegamento che viene
utilizzata.

Quando un datagramma IP viaggia sulla rete potrebbe attraverso diversi
mezzi di collegamento, ognuno con un valore di Maximum Transfer Unit
differente. Ecco che, se durante il viaggio, il datagramma passa da un
valore più alto a un più basso di MTU si è costretti a frammentare il
pacchetto.

Ciascun frammento è a tutti gli effetti un datagramma IP indipendente
dagli altri in grado di viaggiare su reti fisiche differenti.

I \textbf{campi di frammentazione} nell'header del datagramma IP
danno indicazioni su come frammentare e ricomporre il messaggio. Per
ricomporre i frammenti infatti il protocollo IP fa uso dei campi 
\textbf{identificatore}, \textbf{flag} e \textbf{offset}:
\begin{itemize}
	\item \textbf{Identificatore}: l'identificatore non dipende dal 
		frammento analizzato ma dal datagramma a cui appartiene tale
		frammento. In questo modo è possibile ricomporre correttamente
		il datagramma raggruppando frammenti con lo stesso 
		identificatore.
	\item \textbf{Offset}: indica la posizione relativa del frammento
		rispetto agli altri frammenti come multiplo di 8 byte in modo 
		da riordinarli quando necessario.
	\item \textbf{Flag}: campo di 3 bit, ognuno con un significato
		diverso:
		\begin{itemize}
			\item \emph{Reserved}: sempre a 0 per ora.
			\item \emph{Do not fragment}: vale 0 se il datagramma può 
				essere frammentato, 1 se il datagramma non deve essere
				frammentato.
			\item \emph{More fragments}: vale 0 se il pacchetto è
				l'ultimo frammento, vale 1 altrimenti.
		\end{itemize}
\end{itemize}
Visto che la frammentazione aumenta la probabilità di perdita di 
pacchetti, si è deciso di limitarla al minimo nei router intermedi.
In IPv4 si fa utilizzo del flag \emph{do not fragment} e in IPv6 non 
è proprio disponibile.

Nella pratica, quando un router non riesce più ad inoltrare un
datagramma, lo comunica al mittente che si occupa di frammentarlo o
cerca di inviarne uno più piccolo.

In questo modo si aggiunge un po' di carico sui nodi sorgente ma si
riesce ad alleggerire di molto il carico sui vari router evitando
la frammentazione, a meno che non sia strettamente necessaria.

\subsection{Indirizzo IPv4}
Ogni host è connesso alla rete tramite un'interfaccia di rete che è il
confine tra l'host e il collegamento su cui vengono inviati i dati. 
Ad ogni interfaccia è assegnato un \textbf{indirizzo IP}.

Per IPv4 abbiamo 4 byte (32 bit) per la composizione dell'IP. I primi 
$n$ bit identificano la rete su internet (\textbf{prefisso}), mentre 
i restanti identificano l'host all'interno di tale rete 
(\textbf{suffisso}).

\subsubsection{Assegnamento indirizzi}
Si pone il problema di assegnare gli indirizzi IP in modo che non ci
siano due host connessi alla stessa rete con lo stesso IP pubblico.

Un primo metodo prevede la suddivisione degli indirizzi IP di rete in 
5 classi: A, B, C, D, E. Queste differiscono nel numero di bit
(assegnato in modo statico) che dedicano all'identificativo di rete.

La classe A ha meno bit per l'identificazione delle reti ma ne ha
molti per l'identificazione dei vari host, le successive hanno più bit
per l'identificazione della rete e meno per l'identificazione degli
host.

Questo metodo è però poco flessibile per via dello scarso range di 
indirizzi disponibili, si è quindi deciso di rimuovere le classi e
passare ad un meccanismo di tipo \textbf{classless addressing}.

Si è quindi aggiunto un valore $n$ all'indirizzo IP che indica quanti 
bit sono stati utilizzati per il prefisso, ottenendo un identificativo
di rete composto da 32 bit dei quali, solo i primi $n$ sono 
significativi e gli altri sono messi a 0.

Per l'assegnamento dell'indirizzo host rimangono $2^{32-n}$ indirizzi 
da cui dobbiamo togliere l'indirizzo di rete e l'indirizzo di
broadcast (suffisso con tutti i bit a 1). Ecco che rimaniamo con
\[ 2^{32-n}-2 \]
indirizzi disponibili per gli host. Tali indirizzi non sono 
identificati solo dal suffisso ma da tutti e 32 i bit, ovvero 
dall'unione di indirizzo di rete e host.

Un altro modo per indicare l'indirizzo della sottorete è con la 
\textbf{subnet mask} (maschera di sottorete), un numero di 32 bit con
i primi $n$ bit messi a 1 e con i rimanenti a 0. Per riuscire ad
ottenere l'indirizzo di rete tramite la subnet mask e l'indirizzo IP
completo basta mettere in \verb|AND| bit a bit questi due.

Gli indirizzi IP sono gestiti dall'ICANN che a sua volta assegna dei 
blocchi di indirizzi ai vari ISP, i quali assegnano sottoblocchi ai 
vari client che ne fanno richiesta. Quando un ISP deve assegnare 
sottoblocchi di indirizzi a delle sottoreti:
\begin{itemize}
	\item Il numero $N$ degli indirizzi in ogni sottorete deve essere
		una potenza di 2.
	\item La lunghezza del prefisso di ogni sottorete si calcola 
		tramite la formula
		\[ n = 32 - \log_2 N \]
		dove $N$ è il numero di indirizzi della sottorete.
\end{itemize}
L'ISP si occupa poi di \emph{annunciare} su Internet la possibilità di
raggiungere un certo blocco di indirizzi da lui distribuito.

\subsubsection{Indirizzi speciali}
Esistono alcuni indirizzi speciali come ad esempio come:
\begin{itemize}
	\item \verb|0.0.0.0|: che viene utilizzato quando un host ha 
		necessità di inviare un datagramma ma non conosce ancora il 
		suo indirizzo IP.
	\item \verb|255.255.255.255|: usato quando un router o un host 
		devono inviare un datagramma a tutti i dispositivi presenti 
		nella rete.
	\item \verb|127.0.0.1|: il datagramma con questo indirizzo 
		destinazione non lascia l'host locale.
	\item Abbiamo infine 4 blocchi:
		\begin{itemize}
			\item \verb|10.0.0.0/8|
			\item \verb|172.16.0.0/12|
			\item \verb|192.168.0.0/16|
			\item \verb|169.254.0.0/16|
		\end{itemize}
		riservati per reti private locali.
	\item \verb|224.0.0.0/4|: indirizzi multicast.
\end{itemize}

\subsection{DHCP}
L'assegnazione di indirizzi IP può avvenire i due modi:
\begin{itemize}
	\item \textbf{Configurazione manuale}: l'amministratore configura 
		direttamente l'IP dell'host e aggiunge eventuali informazioni 
		di servizio.
	\item \textbf{DHCP}: l'host ottiene il proprio indirizzo IP in 
		modo automatico.
\end{itemize}
Il DHCP (Dynamic Host Configuration Protocol) ha come obbiettivo 
l'assegnazione dinamica di indirizzi IP all'interno di una rete ogni
qual volta un dispositivo si \emph{aggiunge} ad essa tramite un 
programma server presente in rete.

Questo metodo è nato per evitare che indirizzi IP appartenenti a 
dispostivi non connessi alla rete andassero sprecati. Si tratta 
di fatto di un protocollo client-server in cui
\begin{enumerate}
	\item L'host invia in \emph{broadcast} un messaggio 
		\verb|DHCP discover|.
	\item Il server DHCP risponde con un messaggio \verb|DHCP offer|.
	\item L'host richiede un indirizzo IP con un messaggio 
		\verb|DHCP request|.
	\item Il server risponde con un messaggio \verb|DHCP ack| se la 
		richiesta va a buon fine.
\end{enumerate}
In questa fase, dato che non è possibile stabilire una connessione 
verso un IP non ancora assegnato, si utilizza il protocollo UDP per 
il trasporto.

\subsection{IP forwarding}
Una delle funzioni principali di IP è il \textbf{forwarding} 
(inoltro), ossia il trasferimento del pacchetto in ingresso 
sull'appropriato collegamento in uscita.

In questa fase il router legge l'indirizzo di destinazione 
dall'header del datagramma e, per riuscire a prendere una decisione 
sul collegamento di uscita, fa riferimento ad una 
\textbf{tabella di inoltro} nella quale è specificata
l'\textbf{interfaccia di output}.

La tabella di inoltro viene costruita servendosi di una
\textbf{funzione di instradamento}, la quale fa uso di informazioni
raccolte attraverso comunicazioni con altri router, per risolvere un
problema decisionale, ossia la scelta di un percorso verso una certa
destinazione.

Anche a livello rete si fa Multiplexing e Demultiplexing in quanto
nell'header del datagramma viene codificato il valore del protocollo 
al quale viene consegnato il payload.

Ogni datagramma IP che è soggetto a \textbf{forwarding} da parte 
dell'host origine e del router che sta attraversando. Questo può 
essere \textbf{diretto} o \textbf{indiretto}:
\begin{itemize}
	\item \textbf{Diretto}: il pacchetto IP ha come destinazione un
		host nella propria rete (o sottorete) IP. In questo caso 
		l'indirizzo di destinazione a livello di collegamento è 
		proprio quello del destinatario in quanto l'invio è diretto 
		su di esso e non viene interpellata nessun'altra entità.
	\item \textbf{Indiretto}: in questo caso il datagramma IP ha come 
		destinazione un nodo di un'altra rete IP. L'invio in questo
		caso viene delegato ad un router e l'indirizzo di destinazione
		stavolta sarà quello del router.
\end{itemize}
In generale, le condizioni necessarie affinché la comunicazione 
avvenga con successo sono due:
\begin{itemize}
	\item Deve esistere un cammino a livello collegamento tra tutti 
		gli host appartenenti alla stessa sottorete.
	\item Ogni host deve avere un indirizzo IP di rete uguale ma con
		ID host univoco all'interno della stessa sottorete.
\end{itemize}
Ad ogni interfaccia nella la rete IP viene assegnato un indirizzo IP 
distinto.

\subsubsection{Inoltro diretto}
Quando il datagramma deve essere inoltrato ad un host della stessa 
sottorete del mittente, il mittente è a conoscenza dell'indirizzo IP
della propria interfaccia di rete, della propria subnet mask e 
dell'indirizzo IP di destinazione.

Con tutte queste informazioni a disposizione il mittente è in grado di
capire se il destinatario si trova nella stessa sottorete. Se così
fosse può effettuare un inoltro di tipo diretto.

A questo punto il mittente fa riferimento ad una tabella in cui poter
trovare l'indirizzo \textbf{MAC} del destinatario e passa il pacchetto
a livello inferiore che incapsula il pacchetto con destinazione il 
MAC del destinatario.

\subsubsection{Inoltro indiretto}
Supponiamo ora che l'indirizzo di destinazione del pacchetto non 
appartenga alla stessa sottorete.

Come prima il mittente è a conoscenza delle informazioni necessarie a
capire se il destinatario si trova sulla stessa sottorete o meno.

Se si trovano su reti differenti, l'host delega l'inoltro del 
pacchetto al router della propria sottorete ma indica come IP
destinazione sempre il solito indirizzo IP del destinatario.

A livello collegamento il frame inviato avrà come indirizzo sorgente
l'indirizzo MAC dell'host e come indirizzo destinazione l'indirizzo 
MAC del router.

\subsubsection{Tabella di inoltro}
Nella tabella di inoltro sono presenti diverse informazioni tra cui
\begin{itemize}
	\item Indirizzi IP delle reti raggiungibili.
	\item L'interfaccia per effettuare l'inoltro verso una specifica
		rete.
	\item Un campo che, nel caso ci sia bisogno di effettuare un 
		inoltro indiretto, conterrà il nome o l'indirizzo IP delegato
		a fare l'inoltro. In caso si tratti di inoltro diretto il 
		campo viene lasciato vuoto.
\end{itemize}
Nel caso in cui ci sia ad esempio un router che gestisce delle 
sottoreti, la sua tabella di inoltro avrà una una serie di entry con
il campo dell'indirizzo di inoltro nullo (inoltro diretto) e ad 
esempio con una o più entry che indicano l'indirizzo di inoltro di 
un router
esterno.

Il router esterno avrà anch'esso lo stesso tipo di tabella solo che 
non conterrà l'indirizzo di tutte le sottoreti raggiungibili tramite
il primo router. Si terrà invece solamente un indirizzo IP di rete in
grado di includere tutte le sottoreti raggiungibili attraverso il 
primo router (\textbf{aggregazione degli indirizzi}).

Supponiamo ora di avere due router che si possono raggiungere l'un
l'altro e tramite questi due router si possono raggiungere delle
sottoreti.

Nella tabella di inoltro possono essere presenti indirizzi di rete che
sono uno più specifico dell'altro. Nel momento in cui arriva un
pacchetto, per decidere su quale rete inoltrarlo si deve fare un match
tra gli indirizzi di rete presenti nella tabella di inoltro e 
l'indirizzo IP di destinazione del pacchetto.

L'indirizzo di rete scelto per l'inoltro è quello più specifico, 
ovvero quello che tra i possibili match ha la maschera più lunga
(\textbf{longer mask matching}).

