\section{IP}
Il \textbf{protocollo IP} (Internet Protocol) è un servizio di tipo connectionless in quanto non
c'è un circuito né logico né fisico tra i due sistemi terminali.

Il servizio offerto non da alcuna garanzia (un po' come UDP) in quanto non vi è alcuna certezza 
che i datagrammi arrivino a destinazione, che siano nello stesso ordine di invio e non vi è alcun
meccanismo di recupero di eventuali datagrammi andati persi.

Quello che fa IP quando riceve un segmento TCP o un datagramma UDP è aggiungere una sua 
intestazione a supporto dell'implementazione dei servizi.

\subsection{Forwarding}
Una delle funzioni principali di IP è l'\textbf{inoltro} (forwarding), ossia il trasferimento del
pacchetto sull'appropriato collegamento in uscita.

Per riuscire a farlo, il router, legge  dall'intestazione, presente all'inizio del datagramma,
l'indirizzo di destinazione. Per riuscire a prendere una decisione sul collegamento di uscita fa 
riferimento ad una \textbf{tabella di inoltro} nella quale è specificata \textbf{ l'interfaccia di
output}.

La tabella di inoltro viene costruita servendosi di una \textbf{funzione di instradamento}, la
quale fa uso di informazioni raccolte attraverso comunicazioni con altri router, per risolvere
un problema decisionale, ossia la scelta di un percorso verso una certa destinazione.

Anche a livello rete si fa Multiplexing e Demultiplexing in quanto nell'header del datagramma
viene codificato il valore del protocollo al quale viene consegnato il payload.

\subsection{Formato del datagramma IP}
Il formato del datagramma IP prevede due parti, l'header e il payload. L'header ha una dimensione
variabile tra 20 e 60 byte e l'intero datagramma può avere una dimensione totale di 65535 byte.
L'header si compone dei seguenti campi:
\begin{itemize}
	\item Versione IP.
	\item Lunghezza header.
	\item Tipo di servizio.
	\item Lunghezza totale del datagram.
	\item Identificativo a 16 bit.
	\item TTL: massimo numero \emph{hop} rimanenti (decrementato ogni volta che passa da un 
		router).
	\item Flag per la frammentazione.
	\item Checksum.
	\item Demultiplexing.
	\item IP mittente e destinatario.
	\item Opzioni.
\end{itemize}
I bit indicati nel campo "tipo di servizio" vanno a supporto del protocollo per indicare le 
caratteristiche del datagramma in base alla \emph{classe} del servizio (telefonata, streaming, 
dati a bassa priorità ecc.) in modo da poter implementare politiche di accodamento diverse.

Alcuni dei bit in questo campo servono a notificare eventi di congestione a livello rete e 
trasporto.

\subsection{Frammentazione}
L'operazione di \textbf{frammentazione} consiste nella suddivisione dei datagrammi IP in 
datagrammi più piccoli, ognuno con la propria intestazione e payload.

Il motivo per cui si ricorre alla frammentazione risiede in un problema dovuto alla capacità del
metodo di collegamento. Quando abbiamo parlato del TCP abbiamo introdotto la Maximum Segment Size,
come la dimensione massima di un segmento TCP tale che possa stare dentro un frame che rispetta il
vincolo di Maximum Transfer Unit, il quale dipende a sua volta dalla tecnologia di collegamento 
che viene utilizzata.

Quando un datagramma IP viaggia sulla rete potrebbe attraverso diversi mezzi di collegamento,
ognuno con un valore di Maximum Transfer Unit differente. Ecco che se durante il viaggio il 
datagramma passa da un valore più alto a un più basso di MTU si è costretti a frammentare il 
pacchetto.

Le informazioni nell'header del datagramma IP danno indicazioni su come frammentare e ricomporre
il messaggio.
