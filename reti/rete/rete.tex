\chapter{Livello di rete}
Il livello di \textbf{rete} offre servizi al livello di trasporto e si 
occupa della consegna dei datagrammi tra gli host appoggiandosi a 
servizi offerti dallo strato di collegamento.
I passi compiuti da una comunicazione a livello di rete sono tipicamente
quattro:
\begin{enumerate}
	\item L'entità a livello rete riceve i segmenti dal livello di 
		trasporto nell'host mittente e li incapsula in 
		\textbf{datagrammi}.
	\item I datagrammi sono inoltrati al nodo successivo sulla rete.
	\item Il router esamina i campi d'intenstazione presenti in tutti i
		datagrammi IP che lo attraversano e li inoltra da un 
		collegamento di ingresso ad uno di uscita.
	\item Consegna i segmenti al livello di trasporto dopo averli 
		decapsulati.
\end{enumerate}
Come possiamo intuire, a differenza dei livelli applicativo e di 
trasporto, il livello di rete è implementato anche nei nodi intermedi e 
riesce a interconnettere reti \emph{eterogenee}.

I servizi offerti dal livello di rete sono molti ma per il momento ci 
concentreremo solo su quelli di maggiore importanza per capire meglio il
funzionamento di questo strato, ossia:
\begin{itemize}
	\item Suddivisione in pacchetti.
	\item Instradamento (routing).
	\item Inoltro (forwarding).
\end{itemize}
Il primo e fondamentale protocollo che trattiamo è il \textbf{protocollo
IP}, molto semplice ma andremo anche a toccare altri protocolli definiti
via via nel tempo per riuscire a tenere il passo con l'evoluzione della
tecnologia.

