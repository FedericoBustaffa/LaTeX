\section{Protocolli di routing}
I \textbf{protocolli di routing} si occupa di determinare un percorso
\emph{buono} da mittente a destinatario passando per nodi intermedi.

In altri termini vogliamo trovare il cammino più veloce basandoci
su congestione, latenze e numero di nodi intermedi da un nodo sorgente 
ad un nodo destinazione, utilizzando le informazioni fornite dai vari
router sulla rete.

Il problema del calcolo del percorso da sorgente a destinazione è 
rappresentabile come il calcolo del percorso ottimale in un grafo.

Gli host e i router sono rappresentati come nodi del grafo mentre i
collegamenti sono gli archi che collegano tali nodi. In generale avremo
un grafo \textbf{pesato} che rappresenta il costo per arrivare da un
nodo all'altro.

Il costo può tenere di conto dei fattori sopra citati come ad esempio
congestione o distanza da un router all'altro.

Ovviamente è possibile impostare degli \emph{obbiettivi di traffico},
vogliamo quindi gestire il traffico per riuscire ad arrivare in un
certo stato di quest'ultimo (per esempio vogliamo minimizzare la 
congestione su certi archi).

\subsection{Algoritmi di routing}
In generale però a noi interessano principalmente gli algoritmi che
calcolano il percorso di costo minimo da un nodo sorgente ad un nodo
destinazione.

Non esiste una sola implementazione di tali algoritmi, essi si infatti
si dividono in base a
\begin{itemize}
	\item Modalità di acquisizione dei dati, che può essere
		\begin{itemize}
			\item \textbf{Globale}: ogni router acquisisce informazioni
				sullo stato di tutti i collegamenti ottenendo una
				vista completa della topologia di rete.
			\item \textbf{Decentralizzata}: 
		\end{itemize}
	\item sdsa
		\begin{itemize}
			\item 
		\end{itemize}
\end{itemize}
