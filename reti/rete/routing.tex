\section{Protocolli di routing}
I \textbf{protocolli di routing} si occupano di determinare un percorso
\emph{buono} da mittente a destinatario passando per nodi intermedi.

In altri termini, vogliamo trovare il cammino più veloce basandoci
su congestione, latenze e numero di nodi intermedi da un nodo sorgente 
ad un nodo destinazione, utilizzando le informazioni fornite dai vari
router.

Il problema del calcolo del percorso da sorgente a destinazione è 
rappresentabile come il calcolo del percorso ottimale in un grafo.

Gli host e i router sono rappresentati come nodi del grafo mentre i
collegamenti sono gli archi che collegano tali nodi. In generale avremo
un grafo \textbf{pesato} con il valore su ogni arco che rappresenta il 
costo per arrivare da un nodo all'altro.

A seconda della metrica che scegliamo per i pesi del nostro grafo è
possibile avere una rappresentazione dinamica di quest'ultimo. 

Supponiamo di voler misurare il costo di un certo cammino in base al 
grado di congestione della rete. Questo non è un valore fisso come ad
esempio la distanza da un nodo all'altro ma può variare nel tempo.

Dato che l'immissione di nuovi pacchetti sulla rete modifica questi 
valori, è anche possibile andare impostare degli \emph{obbiettivi di 
traffico}. In altre parole vogliamo immettere datagrammi in rete in 
modo da raggiungere un certo stato (di congestione ad esempio).

\subsection{Algoritmi di routing}
Gli algoritmi di routing che trattiamo non si occupano di raggiungere
degli obbiettivi di traffico, bensì calcolano il percorso di costo 
minimo da un nodo sorgente ad un nodo destinazione.

Non esiste una sola implementazione di tali algoritmi, essi infatti si 
dividono in base a:
\begin{itemize}
	\item Modalità di \textbf{acquisizione dei dati}, che può essere
		\begin{itemize}
			\item \textbf{Globale}: ogni router acquisisce informazioni
				sullo stato di tutti i collegamenti ottenendo una
				vista completa della topologia di rete.
			\item \textbf{Decentralizzata}: Ogni router acquisisce
				informazioni mediante lo scambio di messaggi con 
				router vicini.
		\end{itemize}
	\item \textbf{Vista nel tempo}, che può essere
		\begin{itemize}
			\item \textbf{Statica}: quando le rotte vengono gestite
				manualmente dagli amministratori (generalmente nel caso
				di reti piccole).
			\item \textbf{Dinamica}: quando i router si scambiano 
				informazioni per riuscire ad aggiornare dinamicamente 
				i percorsi.
		\end{itemize}
\end{itemize}
Per quanto ci riguarda il routing di tipo statico non è particolarmente
interessante. Andremo quindi a trattare algoritmi di tipo dinamico.

\subsection{Distance Vector}
Si tratta di un algortimo con modalità di acquisizione dei dati
decentralizzata e quindi \textbf{distribuito}. Ciascun nodo riceve
infatti delle informazioni da uno o più nodi vicini per effettuare
i calcoli e comunicarli agli altri nodi.

Si tratta di un tipo di algoritmo \textbf{iterativo} in quanto ci
servono più iterazioni per convergere ad una soluzione ed è un 
algoritmo \textbf{asincrono} per il calcolo del cammino costo minimo 
\textbf{asincrono} poiché ogni nodo può fare dei calcoli in modo
indipendente.

\subsubsection{Implementazione}
Si basa sull'equazione di \textbf{Bellman-Ford} per il calcolo di un
cammino di costo minimo tra un nodo sorgente $x$ ad un nodo 
destinazione $y$, passando per nodi intermedi. Questo algoritmo assume
che:
\begin{itemize}
	\item Il costo $c(x,z)$ tra nodo sorgente $x$ e il vicino $z$ sia
		noto.
	\item La distanza minima tra il vicino $z$ e la destinazione sia
		nota.
\end{itemize}
Nella pratica il router $x$ conosce la distanza tra lui e i suoi vicini
mentre i suoi vicini (non tutti) conoscono la distanza tra loro e il
nodo destinazione.

Il router a questo punto calcola il percorso migliore basandosi su un 
insieme di distanze fornitegli dai nodi adiacenti e sulla distanza tra
lui e tali nodi adiacenti.

La scelta ricade ovviamente sul cammino minimo da lui alla destinazione
anche se il router non conosce effettivamente il percorso ma solo il 
suo costo.
\[ d_{xy} = \min \{ c(x, v) + d_{vy} \} \]
Tale informazione viene poi comunicata ad eventuali altri router vicini
precedentemente interpellati nel calcolo del percorso.

Un'iterazione viene scatenata da un cambio nel costo di uno dei 
collegamenti locali oppure dalla ricezione di un \emph{vettore 
distanza} aggiornato.

Ogni nodo aggiorna i suoi vicini solo quando il proprio \emph{distance
vector} cambia e i vicini avvisano i rispettivi vicini solo se
necessario.

\subsubsection{Problematiche}
Il problema di questo algoritmo è il cosiddetto \textbf{count to
infinity}, che si presenta nel momento in cui un nodo si accorge che
il costo verso uno dei suoi vicini è aumentato ma non riesce a
comunicarlo agli altri suoi vicini poiché avviene uno scambio di
informazioni errato dovuto all'asincronia del metodo.

Per aggirare il problema, il nodo che notifica l'avvenuta degradazione
del collegamento, non riceve informazioni dagli altri nodi sui relativi
percorsi.

In questo modo i nodi che devono ricevere la notifica possono farlo 
correttamente e, se necessario, possono aggiornare i propri vettori 
distanza (\textbf{Split to Horizon}).

\subsection{Link state}
Questo algoritmo sfrutta l'acquisizione centralizzata e globale delle 
informazioni da tutti i nodi di una certa rete.

In questo modo è possibile acquisire una vista globale della topologia
di rete tramite messaggi (\textbf{link state broadcast}) per ogni nodo
nella rete.

Si tratta di un algoritmo iterativo che usa l'algoritmo di Dijkstra
per calcolare il cammino a costo minimo da un nodo origine a tutti gli
altri nodi della rete.

\subsection{Sistemi autonomi}
I router sono organizzati in \textbf{sistemi autonomi}, ossia gruppi di
router sotto lo stesso controllo amministrativo, il quale si occupa di
decidere in modo autonomo i protocolli e le politiche di routing.
Andiamo quindi a distinguere i protocolli di routing in due categorie:
\begin{itemize}
	\item \textbf{IGP}: usati all'interno di sistemi autonomi 
		(Interior Gateway Protocol).
	\item \textbf{EGP}: usati all'esterno di sistemi autonomi 
		(Exterior Gateway Protocol).
\end{itemize}
All'interno dello stesso sistema autonomo è necessario che ci sia un 
unico protocollo.

\subsubsection{RIP}
Il protocollo di interesse usato internamente da un sistema autonomo
è il \textbf{RIP} (Routing Information Protocol).

Questo protocollo fa uso dell'algoritmo \emph{distance vector} nella 
sua versione in grado di prevenire scambio di informazioni errate, così
da evitare inutili ritardi per l'instradamento.

In questo caso abbiamo che il costo da un nodo all'altro è sempre 1.
Di conseguenza il cammino di costo minimo è il cammino più breve, ossia
quello che ci permette di compiere il minor numero di salti.

Gli aggiornamenti sono inviati ogni 30 secondi tramite UDP da un 
processo demone che si occupa di inviare tali aggiornamenti.

\subsubsection{OSPF}
L'altro protocollo interno al sistema autonomo, che invece implementa 
l'algoritmo \emph{link state}, è l'\textbf{OSPF} (Open Shortest Path
First). In questo caso è l'amministratore di rete a decidere la metrica
per il calcolo del percorso.

Il problema è che questo metodo invia messaggi in broadcast a tutti i
router della rete andandola a caricare di messaggi di controllo. \`E
dunque necessario alleggerire il carico dividendo la rete in modo
gerarchico.

\subsubsection{BGP}
I sistemi autonomi possiedono dei \textbf{router di frontiera} che li
mettono in comunicazione con altri sistemi autonomi. Tali router si 
scambiano informazioni, basate sulla conoscenza che hanno del proprio 
sistema autonomo. In questo modo possono comunicare ad altri sistemi 
autonomi che attraverso di loro è possibile raggiungere
\begin{itemize}
	\item Altri sistemi autonomi.
	\item Router all'interno del sistema autonomo.
\end{itemize}
I router di frontiera avranno quindi un algoritmo per il calcolo dei
percorsi interni al sistema autonomo e uno per il calcolo dei percorsi 
da un sistema autonomo all'altro.

I router di frontiera comunicano in broadcast a tutti i router del 
proprio sistema autonomo la possibilità di raggiungere nodi 
appartenenti ad altri sistemi autonomi.

Nella pratica, coppie di router si scambiano informazioni tramite
connessioni TCP in sessioni BGP \textbf{esterne} ed \textbf{interne}.
Gli attributi più importanti sono due:
\begin{itemize}
	\item \verb|AS_PATH|: sequenza dei sistemi autonomi.
	\item \verb|NEXT_HOP|: indirizzo IP del primo router lungo un
		percorso annunciato a un dato prefisso di rete.
\end{itemize}
Il protocollo è di per se molto complesso e fa una scelta dei percorsi
anche in base a scelte amministrative e accordi tra paesi e/o 
compagnie.
