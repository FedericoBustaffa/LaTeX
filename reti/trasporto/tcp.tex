\section{TCP}
Il protocollo \textbf{TCP} è un tipo di protocollo orientato allo \textbf{stream}. Uno stream è
un flusso di byte la cui lunghezza non si può definire a priori. Le caratteristiche principali di
questo tipo di comunicazione sono:
\begin{itemize}
	\item I byte ricevuti dal destinatario sono esattamente gli stessi che ha inviato il mittente.
	      Senza perdite ne modifiche.
	\item Il protocollo vede il flusso di byte come una sequenza ordinata ma non strutturata di
	      byte.
\end{itemize}
Come già detto si tratta di un servizio connection-oriented con le seguenti caratteristiche:
\begin{itemize}
	\item I processi effetuano un \emph{handshake} prima dello scambio dei dati.
	\item Si dice \textbf{orientato} poiché lo stato della connessione risiede sui nodi terminali
	      e non sui nodi intermedi (router).
	\item La connessione è vista dagli applicativi come un circuito fisico dedicato.
	\item La connessione è di tipo \textbf{full-duplex}: a prescindere da chi invia il messaggio
	      di richiesta per la connessione, una volta instaurata la connessione, questa è
	      bidirezionale.
	\item La connessione è di tipo \textbf{punto-punto} in quanto si vuole connettere solo due
	      sistemi terminali l'un con l'altro.
\end{itemize}
Il trasferimento è inoltre \textbf{bufferizzato}, il software TCP può infatti suddividere il flusso
di byte in segmenti indipendentemente da come gli vengono inviati dal programma applicativo.

Per farlo è necessario un \textbf{buffer} dove immagazzinare le sequenze di byte. Non appena i dati
sono sufficienti per riempire un segmento ragionevolmente grande, questo viene trasmesso attraverso
la rete.

Questo metodo permette di "ottimizzare" il numero di segmenti spediti sulla rete andando a ridurre
di conseguenza il traffico. Permette inoltre di ritrasmettere eventuali segmenti andati persi.

\subsection{Segmenti TCP}
Il flusso di byte viene quindi partizionato in \textbf{segmenti}, ognuno con il proprio header, e
ognuno consegnato a livello IP.

Il protocollo, per garantire che i segmenti siano composti da byte ordinati, numera i byte stessi
con \textbf{numeri di sequenza} e \textbf{numeri di riscontro}.

Ogni segmento possiede un campo che indica il numero di sequenza, ossia il numero del primo byte in
tale segmento.

Il numero di riscontro è il numero dell'ultimo byte correttamente ricevuto più 1. Il numero di
riscontro indica quindi quanti byte sono giunti a destinazione e qual'è il prossimo byte che si è
in attesa di ricevere.

\subsubsection{Formato dei segmenti}
Il formato dei segmenti TCP comprende un \emph{header} e i dati che vogliano inviare. L'header del
segmento, con una lunghezza variabile tra i 20 e i 60 byte contiene:
\begin{itemize}
	\item Porta del sorgente e del destinatario.
	\item Numero di sequenza e di riscontro.
	\item Lunghezza dell'header.
	\item Campi riservati.
	\item Flag: \verb|URG|, \verb|ACK|, \verb|PSH|, \verb|RST|, \verb|SYN| e \verb|FIN|. Di
	      particolare rilevanza sono i flag
	      \begin{itemize}
		      \item \verb|ACK|: se settato a 1 indica che nel campo di riscontro è presente
		            un'informazione significativa.
		      \item \verb|SYN|: sincronizza il numero di sequenza.
		      \item \verb|FIN|: Indica che non ci sono altri dati in arrivo dal mittente e quindi
		            la connessione può essere chiusa.
	      \end{itemize}
	\item Dimensione della finestra: spazio disponibile nella finestra di ricezione.
	\item Checksum per la verifica dell'integrità.
\end{itemize}