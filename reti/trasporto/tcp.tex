\section{TCP}
Il protocollo \textbf{TCP} è un tipo di protocollo orientato allo \textbf{stream}. Uno stream è
un flusso di byte la cui lunghezza non si può definire a priori. Le caratteristiche principali di
questo tipo di comunicazione sono:
\begin{itemize}
	\item I byte ricevuti dal destinatario sono esattamente gli stessi che ha inviato il mittente.
	      Senza perdite ne modifiche.
	\item Il protocollo vede il flusso di byte come una sequenza ordinata ma non strutturata di
	      byte.
\end{itemize}
Come già detto si tratta di un servizio connection-oriented con le seguenti caratteristiche:
\begin{itemize}
	\item I processi effetuano un \emph{handshake} prima dello scambio dei dati.
	\item Si dice \textbf{orientato} poiché lo stato della connessione risiede sui nodi terminali
	      e non sui nodi intermedi (router).
	\item La connessione è vista dagli applicativi come un circuito fisico dedicato.
	\item La connessione è di tipo \textbf{full-duplex}: a prescindere da chi invia il messaggio
	      di richiesta per la connessione, una volta instaurata la connessione, questa è
	      bidirezionale.
	\item La connessione è di tipo \textbf{punto-punto} in quanto si vuole connettere solo due
	      sistemi terminali l'un con l'altro.
\end{itemize}
Il trasferimento è inoltre \textbf{bufferizzato}, il software TCP può infatti suddividere il flusso
di byte in segmenti indipendentemente da come gli vengono inviati dal programma applicativo.

Per farlo è necessario un \textbf{buffer} dove immagazzinare le sequenze di byte. Non appena i dati
sono sufficienti per riempire un segmento ragionevolmente grande, questo viene trasmesso attraverso
la rete.

Questo metodo permette di "ottimizzare" il numero di segmenti spediti sulla rete andando a ridurre
di conseguenza il traffico. Permette inoltre di ritrasmettere eventuali segmenti andati persi.