\section{TCP}
Il protocollo \textbf{TCP} è un tipo di protocollo orientato allo \textbf{stream}. Uno stream è
un flusso di byte la cui lunghezza non si può definire a priori. Le caratteristiche principali di
questo tipo di comunicazione sono:
\begin{itemize}
	\item I byte ricevuti dal destinatario sono esattamente gli stessi che ha inviato il mittente.
	      Senza perdite ne modifiche.
	\item Il protocollo vede il flusso di byte come una sequenza ordinata ma non strutturata di
	      byte.
\end{itemize}
Come già detto si tratta di un servizio connection-oriented con le seguenti caratteristiche:
\begin{itemize}
	\item I processi effetuano un \emph{handshake} prima dello scambio dei dati.
	\item Si dice \textbf{orientato} poiché lo stato della connessione risiede sui nodi terminali
	      e non sui nodi intermedi (router).
	\item La connessione è vista dagli applicativi come un circuito fisico dedicato.
	\item La connessione è di tipo \textbf{full-duplex}: a prescindere da chi invia il messaggio
	      di richiesta per la connessione, una volta instaurata la connessione, questa è
	      bidirezionale.
	\item La connessione è di tipo \textbf{punto-punto} in quanto si vuole connettere solo due
	      sistemi terminali l'un con l'altro.
\end{itemize}
Il trasferimento è inoltre \textbf{bufferizzato}, il software TCP può infatti suddividere il flusso
di byte in segmenti indipendentemente da come gli vengono inviati dal programma applicativo.

Per farlo è necessario un \textbf{buffer} dove immagazzinare le sequenze di byte. Non appena i dati
sono sufficienti per riempire un segmento ragionevolmente grande, questo viene trasmesso attraverso
la rete.

Questo metodo permette di "ottimizzare" il numero di segmenti spediti sulla rete andando a ridurre
di conseguenza il traffico. Permette inoltre di ritrasmettere eventuali segmenti andati persi.

\subsection{Segmenti TCP}
Il flusso di byte viene quindi partizionato in \textbf{segmenti}, ognuno con il proprio header, e
ognuno consegnato a livello IP.

Il protocollo, per garantire che i segmenti siano composti da byte ordinati, numera i byte stessi
con \textbf{numeri di sequenza} e \textbf{numeri di riscontro}.

Ogni segmento possiede un campo che indica il numero di sequenza, ossia il numero del primo byte in
tale segmento.

Il numero di riscontro è il numero dell'ultimo byte correttamente ricevuto più 1. Il numero di
riscontro indica quindi quanti byte sono giunti a destinazione e qual'è il prossimo byte che si è
in attesa di ricevere.

\subsubsection{Formato dei segmenti}
Il formato dei segmenti TCP comprende un \emph{header} e i dati che vogliano inviare. L'header del
segmento, con una lunghezza variabile tra i 20 e i 60 byte contiene:
\begin{itemize}
	\item Porta del sorgente e del destinatario.
	\item Numero di sequenza e di riscontro.
	\item Lunghezza dell'header.
	\item Campi riservati.
	\item Flag: \verb|URG|, \verb|ACK|, \verb|PSH|, \verb|RST|, \verb|SYN| e \verb|FIN|. Di
	      particolare rilevanza sono i flag
	      \begin{itemize}
		      \item \verb|ACK|: se settato a 1 indica che nel campo di riscontro è presente
		            un'informazione significativa.
		      \item \verb|SYN|: sincronizza il numero di sequenza in fase di instaurazione della
		            connessione.
		      \item \verb|FIN|: Indica che non ci sono altri dati in arrivo dal mittente e quindi
		            la connessione può essere chiusa.
	      \end{itemize}
	\item Dimensione della finestra: spazio disponibile nella finestra di ricezione.
	\item Checksum per la verifica dell'integrità.
\end{itemize}

\subsection{Gestione della connessione}
Come già detto, il servizio TCP è un servizio orientato alla connessione. Tale connessione viene
gestita in tre passi principali:
\begin{itemize}
	\item \textbf{Handshake}: instaurazione della connessione.
	\item \textbf{Trasferimento dati}: gestione del flusso di trasferimento.
	\item \textbf{Chiusura}: come la connessione viene chiusa da entrambi i lati.
\end{itemize}

\subsubsection{Handshake a tre vie}
L'\textbf{handshake} è la fase di instaurazione della connessione. Dopo questa fase non si fa più
distinzione tra client e server in quanto la comunicazione avviene in maniera bidirezionale. In
questa fase ogni messaggio è di 1 byte:
\begin{enumerate}
	\item Il client invia una richiesta di connessione settando a 1 il flag \verb|SYN| e con un
	      numero di sequenza $n$ generato casualmente.
	\item Se la richiesta di connessione viene accettata il server invia al client un segmento di
	      autorizzazione con il flag di \verb|ACK| settato a 1 e con il numero di riscontro uguale
	      a $n+1$. Nel segmento inviato sarà inoltre presente il flag \verb|SYN| settato a 1 e un
	      nuovo numero di sequenza $m$.
	\item Il client a questo punto invia un messaggio di risposta al messaggio di autorizzazione
	      settando il flag di \verb|SYN| a 0, il numero di sequenza a $n+1$ e il campo di riscontro
	      a $m+1$.
\end{enumerate}
Questi tre passaggio servono a impostare un tipo di connessione bidirezionale e paritaria tra
client e server. Se svolgessimo solo i primi due passaggi, il client non avrebbe la possibilità
di notificare la disponibilità a ricevere messaggi dal server.

\subsubsection{Chiusura della connessione}
La chiusura della connessione deve avvenire da entrambi i lati, i quali devono essere a conoscenza
del fatto che l'altro sta per chiudere la connessione. Per farlo i due processi si inviano un
segmento con il flag di \verb|FIN| settato a 1 secondo la seguente modalità:
\begin{itemize}
	\item Il client invia un segmento al server con il flag di \verb|FIN| settato a 1 e un numero
	      di sequenza $x$.
	\item Il server risponde tramite un segmento con il flag di \verb|ACK| settato a 1 e con il
	      campo di riscontro a $x+1$.
	\item Il server può continuare a inviare dati.
	\item Il server invia un segmento con il flag di \verb|FIN| settato a 1 e con un numero di
	      sequenza $y$.
	\item Il client risponde con un segmento di \verb|ACK| e con il campo di riscontro impostato
	      a $y+1$.
	\item Il client entra in uno stato di \verb|TIME_WAIT| dalla durata variabile prima di
	      chiudere definitivamente la connessione.
	\item Se il server riceve l'ultimo segmento di \verb|ACK| allora chiude la connessione,
	      altrimenti, dopo un certo timeout, invia di nuovo un segmento con il flag di \verb|FIN|.
	\item Il client, se dopo il timeout, non ha ricevuto ulteriori richieste chiude la
	      connessione.
\end{itemize}
Lo stato di \verb|TIME_WAIT| dipende da un altro parametro, ossia il \textbf{MSL} (Maximum Segment
Lifetime). Questo può variare tra le varie implementazioni TCP e indica il tempo massimo che un
pacchetto TCP può rimanere in rete. Dopo quel tempo o è arrivato a destinazione o è stato
cancellato.

Lo stato di \verb|TIME_WAIT| ha un durata che tipicamente viene calcolata come il doppio del MSL.

Nel caso in cui il client invii il segmento di \verb|FIN| mentre il server sta continuando a
inviare dati siamo in un stato \textbf{half-close}.

\subsection{Trasferimento dati affidabile}
Un segmento può essere perso o corrotto e TCP implementa un servizio di trasferimento dati
affidabile per riuscire a contenere queste problematiche.

Per verficare l'integrità del pacchetto esiste un meccanismo di checksum, mentre per stabilire se
qualche pacchetto è andato perso c'è il meccanismo dei \textbf{riscontri} implementato tramite
numeri di sequenza e di riscontro.

In particolari casi, per esempio con il meccanismo di \textbf{pipeline}, possiamo inviare più
segmenti senza attendere un loro riscontro.

Gli eventi che portano alla corruzione o perdita di segmenti TCP possono essere molteplici e
possono avvenire sia da parte del mittente che da parte del destinatario.

Inziamo a vedere cosa succede dal lato del mittente e proviamo a individuare possibili criticità:
\begin{enumerate}
	\item Il protocollo TCP riceve i dati dall'applicazione.
	\item I dati ricevuti vengono incapsulati in uno o più segmenti e si assegnano i numeri di
		sequenza.
	\item A questo punto il mittente avvia un \textbf{timer di ritrasmissione} (RTO).
	\item Il mittente ritrasmette il segmento in due casi:
		\begin{itemize}
			\item Timeout.
			\item Ricezione di tre \verb|ACK| duplicati.
		\end{itemize}
\end{enumerate}
I segmenti possono anche arrivare fuori sequenza e quindi memorizzati temporaneamente dal 
destinatario. Il TCP non implementa la gestione di tali pacchetti, questo dipende dal destinatario.

Il destinatario, per applicare i riscontri in modo efficiente, implementa diversi metodi:
\begin{itemize}
	\item Nel caso in cui il destinatario debba inviare dati, include nel messaggio da inviare
		il numero di riscontro per evitare di inviare due segmenti separati.
	\item Se il destinatario non ha dati da inviare e riceve un segmento in ordine ritarda l'invio
		dell'\verb|ACK| di 500ms a meno che non riceva un nuovo segmento. Questo serve a 
		riscontrare più segmenti in una sola volta.
	\item Se il destinatario riceve un segmento atteso e il precedente non è stato riscontrato,
		allora invia immediatamente l'\verb|ACK|.
	\item Nei casi in cui il destinatario riceva un segmento fuori sequenza, mancante oppure 
		duplicato invia immediatamente l'\verb|ACK| indicando il prossimo numero atteso.
\end{itemize}
Le varie casistiche possono verificarsi o meno anche a seconda dell'implementazione del TCP, che
può variare da host a host.

\subsubsection{Calcolo del timeout}
Il timeout di ritrasmissione RTO è fondamentale per il corretto funzionamento del TCP che, se
troppo breve potrebbe causare ritrasmissioni indesiderate e se troppo lungo potrebbe causare
rallentamenti.

Questo timeout deve essere sicuramente superiore all'\textbf{RTT} (Round Trip Time), ossia il
tempo che intercorre tra l'invio di un pacchetto e la ricezione del riscontro. Viene calcolato
analizzando gli RTT dei segmenti non ritrasmessi andando in primo luogo a misurare l'RTT tramite 
la seguente formula
\[
	\text{RTT}_\text{stimato} = (1 - \alpha) \cdot \text{RTT}_\text{stimato} +
	\alpha \cdot \text{RTT}_\text{misurato}
\]
La stima dell'RTT viene compiuta periodicamente e viene definita come un'aggiornamento dell'ultimo
RTT calcolato. La prima volta la stima dell'RTT corrisponderà alla misura esatta dell'RTT campione
amplificato di un fattore $\alpha$.

Il valore che in genere viene dato ad $\alpha$ è $1/8$ in modo da rendere via via meno importanti 
gli RTT dei pacchetti più vecchi.

La sola stima dell'RTT non è sufficiente a calcolare l'RTO. \`E infatti necessaria anche una stima
della variabilità dell'RTT, calcolabile come segue
\[
	\text{RTT}_\text{dev} = (1 - \beta) \cdot \text{RTT}_\text{dev} + 
	\beta \cdot |\text{RTT}_\text{misurato} - \text{RTT}_\text{stimato}|
\]
Con $\beta$ generalmente settato a $1/4$ questo calcolo ci dice quanto l'RTT misurato si discosta
dall'RTT stimato.

Una volta calcolati questi valori si calcola l'RTO semplicemente svolgendo la somma 
\[ \text{RTO} = \text{RTT}_\text{stimato} + 4 \cdot \text{RTT}_\text{dev} \]
In molte implementazioni, dopo un errore, si raddoppia il timeout in modo da avere in primo
meccanismo di controllo della congestione.

\subsubsection{Finestra di trasmissione}
I dati inviati dal processo applicativo sono mantenuti nel \textbf{buffer di invio}, la cui
dimensione e organizzazione decreta la frequenza con cui TCP riesce ad inviare segmenti. In questo
buffer, i segmenti si dividono in 4 sezioni:
\begin{itemize}
	\item Trasmessi e riscontrati.
	\item Trasmessi e non riscontrati.
	\item Trasmissibili.
	\item Non trasmissibili.
\end{itemize}
La \textbf{finestra di trasmissione} è composta dalla somma delle due sezioni del buffer di invio
in cui risiedono i dati trasmessi ma non riscontrati e quelli trasmissibili. Questa finestra ha 
dimensione variabile e può \emph{scorrere} man mano che alcuni byte vengono riscontrati andando a
eliminare questi ultimi e includendo tra i byte trasmissibili altri byte in attesa di essere
trasmessi.

La sezione dei byte trasmessi ma non riscontrati avrà come primo elemento il byte con numero di
sequenza $S_f$ (Send first), ossia il primo byte in attesa di essere riscontrato. La sezione dei
byte trasmissibili avrà invece come primo elemento il prossimo byte da inviare con numero di 
sequenza $S_n$ (Send next).

La dimensione della finestra varia in base ad altri due meccanismi che vedremo più avanti, ossia
il controllo di flusso e il controllo di congestione.

\subsubsection{Finestra di ricezione}
I dati ricevuti dal destinatario vengono memorizzati su un \textbf{buffer di ricezione} definito
dalle seguenti sezioni:
\begin{itemize}
	\item I byte che sono già stati consumati dal processo.
	\item I byte ricevuti in attesa di essere consumati.
	\item I byte che possono essere ricevuti.
	\item I byte che non possono essere ricevuti.
\end{itemize}
La dimensione \textbf{finestra di ricezione} equivale alla dimensione della sezione del buffer di 
ricezione in cui risiedono i byte che possono essere ricevuti. Tale dimensione rappresenta un
limite per l'invio di dati da parte del mittente.

\subsubsection{Controllo di flusso}
Per evitare di inondare il destinatario di dati entra in gioco il meccanismo di \textbf{controllo
di flusso}, che permette di regolare l'invio dei pacchetti, per adattarsi alla capacità del
mittente di riceverli.

Per implementare questo meccanismo, il mittente tiene una variabile $rwnd$, che da un'idea di
quanto spazio sia ancora disponibile nel buffer del destinatario. Tale valore viene comunicato
nel campo \verb|window| dell'header TCP ogni volta che il destinatario invia un messaggio al
mittente (anche solo di \verb|ACK|). Il valore di $rwnd$ viene calcolato sottraendo alla dimensione
totale del buffer di ricezione la differenza tra i numeri di sequenza dell'ultimo byte ricevuto e 
l'ultimo byte letto
\[ rwnd = \text{RcvBuffer} - (\text{LastReceived} - \text{LastRead}) \]
Il mittente si assicura ad ogni invio che
\[ \text{LastSent} - \text{LastAcked} \leq rwnd \]
Nel caso in cui $rwnd$ sia uguale a 0 il mittente può inviare dei segmenti \textbf{sonda} di 1 byte
per ricevere dal destinatario nuovi aggiornamenti sulla dimensione della finestra.

\subsubsection{Controllo di congestione}
Il \textbf{controllo di congestione} serve a regolare la quantità di dati inviati in funzione del
traffico presente sulla rete. Il fenomeno di congestione è originato dal tentativo delle sorgenti
di richiedere più banda di quella disponibile sul percorso.

Il protocollo TCP impone un limite a ciascun mittente sulla frequenza dei pacchetti inviati sulla
connessione, andando a definire una \textbf{finestra di congestione}, in funzione della
congestione percepita.

Il controllo di congestione TCP è di tipo punto-punto in quanto non vi è alcun supporto esplicito
della rete e la congestione è dedotta dai sistemi terminali.

\subsubsection{Regolazione della frequenza di invio}
Il protocollo TCP, per regolare la frequenza di invio dei dati, combina i due tipi di controllo
andando a ridimensionare la finestra di invio prendendo il minimo tra la dimensione della finestra
di ricezione e la dimensione della finestra di congestione.

Il meccanismo con cui TCP regola la propria frequenza di invio in funzione della congestione 
rilevata è composto di tre punti:
\begin{enumerate}
	\item Partenza lenta.
	\item Incremento additivo e decremento moltiplicativo (AIMD).
	\item Ripresa veloce e reazione ai timeout.
\end{enumerate}
Il valore della finestra di congestione influisce sulla frequenza di invio dei dati in quanto 
questa è sempre inferiore a
\[ cwnd / RTT \]
La dimensione della finestra di congestione si misura tipicamente in MSS (Maximum Segment Size),
ossia la massima quantità di dati trasportabile da un segmento. Questa misura varia a seconda del
collegamento, il quale fornisce una misura per la lunghezza massima del payload del frame di 
collegamento inviabile dall'host mittente e si misura in MTU (Maximum Transmissive Unit).

Nel meccanismo AIMD il TCP del mittente aumenta proporzionalmente la propria finestra di 
congestione ad ogni \verb|ACK| ricevuto in modo che si abbia un crescita pari ad 1 MSS per ogni 
RTT
\[ cwnd = cwnd + \frac{1}{cwnd} \]
Nel caso in cui si verifichi invece una perdita la finestra di congestione viene dimezzata.

Tipicamente, all'inizio della connessione, la finestra di congestione ha dimensione pari a 1MSS.
Stavolta però incrementiamo la sua dimensione di 1MSS ad ogni \verb|ACK| non duplicato, ottenendo
così un raddoppiamento della finestra ad ogni RTT.

Questo meccanismo prende il nome di \textbf{slow start} che punta ad una partenza lenta per 
riuscire ad evitare il dimezzamento della dimensione della finestra per più tempo possibile e
avere una crescita più rapida man mano che si va avanti.

C'è da tenere di conto che gli \verb|ACK| ritardati rallentano la velocità di incremento della
finestra.

\subsection{TCP Reno}
Questa implementazione di TCP prevede di definire una variabile \textbf{soglia} alla quale è 
assegnato un valore alto. Tale variabile determina la fine della fase di partenza lenta e quindi 
l'inizio della \emph{congestion avoidance}.

Nel caso invece si raggiunga un timeout si pone la soglia a metà della $cwnd$ e ridimensioniamo la
$cwnd$ a 1MSS in modo da ricominciare con un approccio \emph{slow start}.

\subsubsection{Fast recovery}
Nel caso in cui si ricevano tre \verb|ACK| duplicati si pone la soglia a metà della $cwnd$ e poi
si ridimensiona la finestra di congestione dandole il valore della soglia più 3MSS. In questo modo
si ha un meccanismo di \textbf{fast recovery}. In questa fase, se avviene un timeout si torna in 
slow start e se si riceve un \verb|ACK| non duplicato si ridimensiona la finestra di congestione
con il valore soglia. Finché si continuano a ricevere \verb|ACK| duplicati si rimane in fast 
recovery incrementando la finestra di congestione di 1MSS.

\subsubsection{TCP Tahoe}
Altra implementazione del controllo di congestione antecedente a TCP Reno è \textbf{TCP Tahoe} la
quale non prevede la fase \emph{fast recovery}. I casi di perdita di pacchetti sono gestiti allo 
stesso modo andando a dimezzare la soglia e tornando nella fase di \emph{slow start}.

\subsection{Throughput TCP}
Per stimare il throughput medio della connessione TCP possiamo indicare con $W$ il valore massimo
in byte della finestra di congestione e ricavandolo tramite la seguente formula
\[ T = \frac{0.75 \cdot W}{RTT} \]
Questo risultato si ottiene considerando che quando la finestra ha dimensione $W$ il throughput 
equivale a $\frac{W}{RTT}$ ma quando c'è una perdita la finestra va a $\frac{W}{2}$ modificando il
valore di throughput a $\frac{W}{2 \cdot RTT}$.

\subsubsection{Equità}
Poniamoci nell'ipotesi che ci siano $K$ connessioni TCP che insistono su un unico link di capacità
$R$ bit/s, ognuna con gli stessi valori di MSS e RTT e senza nessun altro protocollo che insiste
sullo stesso link.

Come risultato avremo che ogni connessione TCP tende a trasmettere $\frac{R}{K}$ bit/s andando
quindi a dividersi equamente le risorse di rete.

Nella pratica le connessioni con RTT più basso, variano più velocemente la loro finestra di
congestione e raggiungono throughput superiori a connessioni con RTT maggiore.
