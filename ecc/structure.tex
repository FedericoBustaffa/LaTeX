% --------------- STYLE ---------------
\usepackage[margin=1.25in]{geometry}
\usepackage{xcolor}

\usepackage{fancyhdr}
\usepackage[Sonny]{fncychap}
\usepackage[most]{tcolorbox}

% Font
\renewcommand{\familydefault}{\sfdefault}
\usepackage{sansmath}
\sansmath

\pagestyle{fancy}
\setlength{\headheight}{15pt}
\rhead{\thepage}


% --------------- MATH ---------------
\usepackage{amsmath, amssymb, amsthm, amsfonts, mathtools}

\usepackage{mdframed}
\newtheoremstyle{th_style}
{0pt}{0pt}
{\normalfont}
{}
{\color{green!40!black}}
{\;}{0.25em}
{\thmname{\textbf{#1}}\thmnumber{ \textbf{#2}}{\color{black}\thmnote{\textbf{ -- #3}}}}

\newmdenv[
	rightline=false,
	leftline=true,
	topline=false,
	bottomline=false,
	linecolor=green!40!black,
	innerleftmargin=5pt,
	innerrightmargin=5pt,
	innertopmargin=5pt,
	innerbottommargin=5pt,
	leftmargin=0cm,
	rightmargin=0cm,
	linewidth=4pt
]{dBox}

\newmdenv[
	rightline=false,
	leftline=true,
	topline=false,
	bottomline=false,
	linecolor=green!40!black,
	backgroundcolor=black!5,
	innerleftmargin=5pt,
	innerrightmargin=5pt,
	innertopmargin=5pt,
	innerbottommargin=5pt,
	leftmargin=0cm,
	rightmargin=0cm,
	linewidth=4pt
]{pBox}

\theoremstyle{th_style}
\newtheorem{theoremeT}{Teorema}[chapter]
\newtheorem{definitionT}{Definizione}[chapter]
\newtheorem{propositionT}{Proposizione}[chapter]
\newtheorem{corollaryT}{Corollario}[chapter]
\newtheorem{lemmaT}{Lemma}[chapter]
\newtheorem{observation}{Osservazione}[chapter]
\newtheorem{exampleT}{Esempio}[section]

\newenvironment{theorem}{\begin{pBox}\begin{theoremeT}}{\end{theoremeT}\end{pBox}}
\newenvironment{definition}{\begin{dBox}\begin{definitionT}}{\end{definitionT}\end{dBox}}
\newenvironment{proposition}{\begin{pBox}\begin{propositionT}}{\end{propositionT}\end{pBox}}
\newenvironment{corollary}{\begin{pBox}\begin{corollaryT}}{\end{corollaryT}\end{pBox}}
\newenvironment{lemma}{\begin{pBox}\begin{lemmaT}}{\end{lemmaT}\end{pBox}}
\newenvironment{example}{\begin{dBox}\begin{exampleT}}{\end{exampleT}\end{dBox}}
