\part{Complessità}

\chapter{Introduzione alla complessità}

In questo capitolo andiamo ad introdurre la teoria della
\textbf{complessità}, in particolare limitandoci ai problemi
decidibili. La complessità che prendiamo noi in esame studia
quindi problemi ricorsivi limitando le risorse per il calcolo,
in particolare parleremo di limitazioni in tempo e spazio.

Tratteremo quindi problemi più \emph{concreti} formalizzandoli
nella loro forma decisionale e andando quindi a studiare di
quante risorse un certo problema ha bisogno per essere deciso.

Quando ad esempio ci troviamo davanti ad un'instanza $x$ del
problema vogliamo determinare se $x \in I$ dove $I$ è la classe
di complessità di quel problema. Per farlo dobbiamo quindi
cercare una funzione $f$ calcolabile totale che riesca a stimare
il tempo (o lo spazio) necessario alla decisione.

Non siamo però interessati ad una funzione che determini
precisamente il quantitativo di risorse necessarie alla
decisione, bensì una funzione che maggiori il tempo (o lo spazio)
necessario al calcolo. In particolare cerchiamo $f$ tale che
\[ f(|x|) \geq \text{tempo o spazio} \]
dove $|x|$ indica la \textbf{taglia} del problema ovvero, detto
in termini più semplici, la dimensione dell'input.

Ovviamente per avere una stima più accurata possibile vogliamo
la minima $f$ che maggiora il tempo per una certa classe di
problemi. Vogliamo inoltre che tale funzione sia monotona
crescente, di modo da studiare la complessità
\textbf{asintotica}. Questo significa che tale funzione, da
un certo punto in poi deve essere sempre maggiore del tempo
necessario alla decisione di un certo problema per la classe
di problemi considerata. Altra considerazione da fare è che
a noi interessa solo lo studio del caso pessimo, in quanto è
il più rilevante e caratterizza meglio la classe di complessità
considerata.

Una volta trovata tale $f$, diciamo che questa \textbf{determina}
la classe di complessità. Andiamo quindi a definire meglio una
classe di complessità

\begin{definition}
	Una \textbf{classe di complessità} è l'insieme dei problemi
	per cui una certa $f$ maggiore tempo o spazio.
\end{definition}

In una classe di complessità ci saranno quindi tutti quei
problemi che richiedono almeno quella misura di complessitò, che
sia in tempo o spazio.
