\section{Macchine di Turing I/O}

Una macchina di Turing di tipo I/O rappresenta un'estensione
della MdT a $k$ nastri le quali ci consentono una trattazione
più agevole della complessità in spazio.

In questa specifica macchina abbiamo $k$ nastri di cui
\begin{itemize}
	\item Il primo, in sola lettura, è dedicato a contenere
	      l'input.
	\item Il secondo, in sola scrittura è dedicato a contenere
	      l'output.
	\item I $k-2$ nastri intermedi sono detti nastri di lavoro.
\end{itemize}

\begin{definition}
	Una MdT a $k$ nastri è di tipo \textbf{I/O} se e solo se la
	funzione di transizione $\delta$ è tale che, tutte le volte
	che
	\[
		\delta (q, \sigma_1, \dots, \sigma_k) =
		(q', (\sigma_1', D_1), \dots, (\sigma_k', D_k))
	\]
	deve valere che
	\begin{itemize}
		\item $\sigma_1' = \sigma_1$: primo nastro in sola
		      lettura.
		\item O $D_k = R$ o, quando $D_k = -$,
		      $\sigma_k' = \sigma_k$: ultimo nastro in sola
		      scrittura.
		\item Se $\sigma_1 = \#$ allora $D_1 \in \{ L, - \}$: la
		      macchina visita al massimo una cella bianca a
		      destra del dato d'ingresso.
	\end{itemize}
\end{definition}

Passiamo ad introdurre subito una proprietà interessante per
queste macchine.

\begin{property}
	Per ogni MdT a $k$ nastri $M$ che decide $I$ in tempo
	deterministico $f$ esiste una MdT $M'$ a $k+2$ nastri di
	tipo I/O che decide $I$ in tempo deterministico $c \cdot f$
	per qualche costante $c$.
	\begin{proof}
		La macchina $M'$ copia il primo nastro di $M$ sul suo
		secondo nastro, impiegando $|x| + 1$ passi e opera come
		$M$ senza più toccare il suo primo nastro. Quando $M$ si
		arresta, $M'$ si arresta dopo aver copiato il risultato
		sull'ultimo nastro, in al più $f(|x|)$ passi.
	\end{proof}
\end{property}

Le MdT si dimostrano ancora robuste per trasformazioni del
modello secondo modifiche algoritmicamente ragionevoli.