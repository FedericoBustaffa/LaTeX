\section{Complessità in tempo deterministico}

Introduciamo adesso il modo necessario a determinare il tempo
necessario alla soluzione di un problema, dove per problema
intendiamo l'appartenenza o meno ad un insieme.

\begin{definition}
	Diciamo che $t$ è il tempo richiesto da una MdT $M$ a $k$
	nastri per decidere il caso $x \in I$ se e solo se
	\[
		(q_0, \underline{\start} x, \underline{\start}, \dots,
		\underline{\start}) \to (H, w_1, \dots, w_k)
	\]
	con $H \in \{ SI, NO \}$
\end{definition}

In altre parole diciamo che $t$ è il tempo necessario a decidere
il caso $x$ se la macchina termina in $t$ passi in uno stato di
terminazione (positivo o negativo che sia).

Quello che però interessa a noi è ottenere una misura del tempo
necessario alla decisione in funzione della taglia del problema.
Vorremmo cioè, dato $x$ in ingresso, una funzione $f(|x|)$ che
ci indica il tempo necessario alla risoluzione del problema.

In ogni caso vogliamo che tale funzione sia calcolabile totale
e che ci restituisca un numero naturale. Siamo inoltre
interessati ad una funzione che non stimi il tempo \emph{esatto}
alla risoluzione del problema ma una che lo maggiori. In sintesi
vogliamo una funzione $f$ tale che $f(|x|)$ identifichi un limite
superiore al numero di passi che la macchina $M$ compie per
risolvere il problema sull'input $x$.

\begin{definition}
	Diciamo che $M$ decide $I$ in \textbf{tempo deterministico}
	$f$ se per ogni dato di ingresso $x$, il tempo $t$ richiesto
	da $M$ per decidere $x$ è minore o uguale a $f(|x|)$.
\end{definition}

Siamo quindi pronti per introdurre il concetto di classe di
complessità in tempo deterministico.

\begin{definition}
	La classe di complessità in tempo deterministico
	\[
		\TIME (f) = \{ I \mid \exists M \text{ che decide }
		I \text{ in tempo deterministico } f \}
	\]
	è quindi l'insieme dei problemi $I$ tali che esiste almeno
	una macchina $M$ che li decide in tempo deterministico $f$.
\end{definition}

Chiariamo brevemente che cosa intendiamo con \emph{ordine} di
una funzione. Con la notazione $\O(f)$ intendiamo l'insieme
\[
	\{ g \mid \exists r \in \R^+ . g(x) < r \cdot f(x)
	\text{ quasi ovunque} \}
\]
di tutte le funzioni tali per cui esiste una costante
moltiplicativa per cui la funzione $f$ è maggiore di $g$
da un certo punto in poi.

Siamo ora interessati a capire quale sia il guadagno in tempo
fornito da una MdT a $k$ nastri. Il risultato può risultare
controintuitivo se non si è capito che avere $k$ nastri non
vuol dire assolutamente avere $k$ processori.

\begin{theorem}[Riduzione del numero dei nastri] \label{th: red_nastri}
	Data una MdT $M$ con $k$ nastri che decide $I$ in tempo
	deterministico $f$, allora $\exists M'$ con $1$ nastro che
	decide $I$ in tempo deterministico $\O (f^2)$.
	\begin{proof}
		Costruiamo $M'$ in modo che simuli la data $M$, in modo
		analogo a quanto fatto per la costruzione della MdT
		universale. Ogni configurazione di $M$ della forma
		\[
			(q, \start w_1 \sigma_1 u_1, \dots,
			\start w_k \sigma_k u_k )
		\]
		viene simulata da
		\[
			(q', \start \start' w_1 \overline{\sigma_1} u_1
			\triangleleft' \dots
			\start' w_k \sigma_k u_k \triangleleft')
		\]
		racchiudiamo cioè ognuno dei nastri $w_i \sigma_i u_i$
		tra due nuove parentesi $\start'$ e $\triangleleft'$ e
		usiamo $\# \Sigma$ nuovi simboli $\overline{\sigma}_i$
		per ricordarci qual era la posizione della testina
		sull'$i$-esimo nastro.

		% da finire
	\end{proof}
\end{theorem}

Il teorema mostra che le MdT sono stabili, aggiungere dunque
nastri e processori non modifica le funzione calcolate o il
tempo deterministico richiesto se non polinomialmente.

Facciamo ora un'osservazione fondamentale che mette in relazione
spazio e tempo necessari a terminare un calcolo. Se una MdT
richiede tempo $f(|x|)$ per decidere $x \in I$ significa che
terminerà in un numero di passi inferiore a $f(|x|)$, e quindi
non può aver visitato, in nessuno dei suoi nastri, un numero di
caselle maggiore di $f(|x|)$ e quindi

\begin{observation}
	Non si può usare più spazio che tempo.
\end{observation}

Riprendendo il teorema appena dimostrato, è possibile fare una
ulteriore osservazione riguardo al vantaggio che introducono le
macchine parallele.

\begin{corollary}
	Le macchine parallele sono polinomialmente più veloci di
	quelle sequenziali.
\end{corollary}

Fino ad ora abbiamo sempre trascurato le costanti poiché
lavoriamo con ordini di grandezza. Se si volessero effettuare
però stime più precise le costanti contano ed è dunque di
possibile interesse cercare di ridurle. Noi continueremo tuttavia
ad ignorarle poiché per $n$ grande le costanti tendono a valere
poco in relazione a $n$ e perché macchine sempre più potenti
tendono a far rimpicciolire le costanti.

Quest'ultimo fatto è sostenuto da un teorema la cui idea è che
se $I \in \TIME (f)$ (quindi esiste una macchina che lo decide
in tempo deterministico $f(n)$), allora $I$ appartiene anche a
$\TIME (\epsilon \cdot f)$, qualunque sia la scelta di
$\epsilon > 0$. In sostanza il teorema ci dice che, dato un
algoritmo che decide un problema, è sempre possibile trovarne
uno equivalente che è più veloce per una costante moltiplicativa
$\epsilon < 1$.

Ovviamente dobbiamo notare che se $I \in \TIME(2^n)$, ovvero se
c'è un algoritmo che decide $I$ in tempo esponenziale, non è
possibile trovare un algoritmo che decide $I$ in tempo
deterministico polinomiale tramite il teorema che enunceremo tra
poco. Analoghe osservazioni valgono per lo spazio.

Il \emph{trucco} sta nel codificare l'alfabeto $\Sigma$ in un
alfabeto più ricco $\Sigma^m$, con $m$ arbitario. Questo si
traduce in macchine con parole di dimensioni maggiori (per
esempio $2^m$ bit).

\begin{theorem} [Accelerazione lineare] \label{th: acc_lin}
	Se $I \in \TIME (f)$, allora $\forall \epsilon \in \R^+$ si
	ha che $I \in \TIME(\epsilon \cdot f(n) + n + 2)$.
\end{theorem}

\begin{definition}
	La classe dei problemi decidibili da MdT deterministiche
	in tempo deterministico polinomiale è
	\[ \P = \bigcup_{k \geq 1} \TIME (n^k) \]
\end{definition}

Questo definisce quindi l'insieme dei problemi decidibili in
tempo polinomiale come i problemi per cui esiste una macchina
che li decide in tempo definito da un polinomio, per qualunque
polinomio di grado maggiore o uguale a $1$.
