\section{Complessità in tempo deterministico}

Introduciamo adesso il modo necessario a determinare il tempo
necessario alla soluzione di un problema, dove per problema
intendiamo l'appartenenza o meno ad un insieme.

\begin{definition}
	Diciamo che $t$ è il tempo richiesto da una MdT $M$ a $k$
	nastri per decidere il caso $x \in I$ se e solo se
	\[
		(q_0, \underline{\start} x, \underline{\start}, \dots,
		\underline{\start}) \to (H, w_1, \dots, w_k)
	\]
	con $H \in \{ SI, NO \}$
\end{definition}

In altre parole diciamo che $t$ è il tempo necessario a decidere
il caso $x$ se la macchina termina in $t$ passi in uno stato di
terminazione (positivo o negativo che sia).

Quello che però interessa a noi è ottenere una misura del tempo
necessario alla decisione in funzione della taglia del problema.
Vorremmo cioè, dato $x$ in ingresso, una funzione $f(|x|)$ che
ci indica il tempo necessario alla risoluzione del problema.

In ogni caso vogliamo che tale funzione sia calcolabile totale
e che ci restituisca un numero naturale. Siamo inoltre
interessati ad una funzione che non stimi il tempo \emph{esatto}
alla risoluzione del problema ma una che lo maggiori. In sintesi
vogliamo una funzione $f$ tale che $f(|x|)$ identifichi un limite
superiore al numero di passi che la macchina $M$ compie per
risolvere il problema sull'input $x$.

\begin{definition}
	Diciamo che $M$ decide $I$ in \textbf{tempo deterministico}
	$f$ se per ogni dato di ingresso $x$, il tempo $t$ richiesto
	da $M$ per decidere $x$ è minore o uguale a $f(|x|)$.
\end{definition}

Siamo quindi pronti per introdurre il concetto di classe di
complessità in tempo deterministico.

\begin{definition}
	La classe di complessità in tempo deterministico
	\[
		\text{TIME}(f) = \{ I \mid \exists M \text{ che decide }
		I \text{ in tempo deterministico } f \}
	\]
	è quindi l'insieme dei problemi $I$ tali che esiste almeno
	una macchina $M$ che li decide in tempo deterministico $f$.
\end{definition}

Chiariamo brevemente che cosa intendiamo con \emph{ordine} di
una funzione. Con la notazione $\O(f)$ intendiamo l'insieme
\[
	\{ g \mid \exists r \in \R^+ . g(x) < r \cdot f(x)
	\text{ quasi ovunque} \}
\]
di tutte le funzioni tali per cui esiste una costante
moltiplicativa per cui la funzione $f$ è maggiore di $g$
da un certo punto in poi.

Siamo ora interessati a capire quale sia il guadagno in tempo
fornito da una MdT a $k$ nastri. Il risultato può risultare
controintuitivo se non si è capito che avere $k$ nastri non
vuol dire assolutamente avere $k$ processori.

\begin{theorem}[Riduzione del numero dei nastri] \label{th: red_nastri}
	Data una MdT $M$ con $k$ nastri che decide $I$ in tempo
	deterministico $f$, allora $\exists M'$ con $1$ nastro che
	decide $I$ in tempo deterministico $\O (f^2)$.
	\begin{proof}
		Costruiamo $M'$ in modo che simuli la data $M$, in modo
		analogo a quanto fatto per la costruzione della MdT
		universale. Ogni configurazione di $M$ della forma
		\[
			(q, \start w_1 \sigma_1 u_1, \dots,
			\start w_k \sigma_k u_k )
		\]
		viene simulata da
		\[
			(q', \start \start' w_1 \overline{\sigma_1} u_1
			\triangleleft' \dots
			\start' w_k \sigma_k u_k \triangleleft')
		\]
		racchiudiamo cioè ognuno dei nastri $w_i \sigma_i u_i$
		tra due nuove parentesi $\start'$ e $\triangleleft'$ e
		usiamo $\# \Sigma$ nuovi simboli $\overline{\sigma}_i$
		per ricordarci qual era la posizione della testina
		sull'$i$-esimo nastro.
	\end{proof}
\end{theorem}