\section{Macchine di Turing a k nastri}

Similmente a quanto fatto per la definizione di una macchina di
Turing universale, la quale aveva tre nastri, generalizziamo
ancora di più l'hardware e introduciamo una macchina a $k$
nastri. Dato che trattiamo solo problemi di decisione, spezziamo
lo stato $h$ di terminazione in due stati $SI$ e $NO$.

\begin{definition} \label{def: mdt a k nastri}
	Dato un numero naturale $k$, una MdT a $k$ nastri è una
	quadrupla
	\[ M = (Q, \Sigma, \delta, q_0) \]
	con
	\begin{itemize}
		\item I simboli $\#, \start \in \Sigma$ e le direzioni
		      $L, R, - \notin \Sigma$.
		\item Gli stati $SI, NO \notin Q$.
		\item La funzione di transizione
		      \[
			      \delta : Q \times \Sigma^k \to
			      Q \cup \{ SI, NO \} \times
			      (\Sigma \times \{ L, R, - \})^k
		      \]
		      segue le stesse regole di una funzione di
		      transizione di una MdT ad un solo nastro, con
		      la differenza che questa prende in input $k$
		      simboli e restituisce $k$ simboli e $k$ direzioni.
	\end{itemize}
\end{definition}

Possiamo vedere la nuova funzione di transizione a $k$ simboli
in questo modo
\[
	\delta (q, \delta_1, \dots, \delta_k) =
	(q', (\delta_1', D_1), \dots, (\delta_k', D_k'))
\]
Di conseguenza una configurazione di una MdT a $k$ nastri ha la
seguente forma
\[ (q, u_1 \sigma_1 v_1, \dots, u_k \sigma_k v_k) \]
dove il carattere corrente sull'$i$-esimo nastro è $\sigma_i$.

Sia chiaro che tale macchina non aumenta in alcun modo la
capacità espressiva di una macchina di Turing, né ci mette in
un contesto di calcolo concorrente o distribuito. Questa
definizione della macchina rappresenta quelle che prendono il
nome di macchine parallele sincrone, in cui non c'è concorrenza
per le risorse. Notiamo infatti che lo stato rimane uno solo,
possiamo infatti immaginarci una macchina che è in grado di
leggere e scrivere dati in modo parallelo ma i passi della
computazione rimangono gli stessi di una macchina ad un nastro,
così come la sequenza degli stati che essa assume.
