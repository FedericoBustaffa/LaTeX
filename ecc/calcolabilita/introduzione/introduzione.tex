\part{Calcolabilità}

\chapter{Introduzione}
Iniziamo con la \textbf{teoria della calcolabilità} la quale
si pone come obbiettivo quello di definire cosa siano problemi,
funzioni e algoritmi, cercando di dare una definizione formale
di questi ultimi. Una volta definiti questi concetti sarà di
nostro interesse capire quali sono i problemi
\textbf{calcolabili} e quali invece no.

In questa prima parte non è di nostro interesse tenere di conto
le limitazioni che hanno i calcolatori reali. Ragioneremo quindi
supponendo che non questi non abbiamo limiti in tempo o spazio
per effettuare il calcolo.

Cercheremo quindi di capire quali sono i problemi
\textbf{calcolabili} mediante una \textbf{procedura effettiva},
quali invece \textbf{non} sono calcolabili per capire se ce ne
sono di interessanti, se ne esistono di reali o se sono solo
artificiali e puramente teorici.
