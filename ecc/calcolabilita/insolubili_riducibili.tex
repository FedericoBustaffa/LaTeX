\section{Problemi insolubili e riducibilità}
In quest'ultimo capitolo di questa prima parte di calcolabilità
andremo a trattare i problemi in cui si cerca di capire se un
elemento appartiene o no ad un dato insieme.

Fino ad ora ci siamo concentrati sulla risoluzione di problemi
tramite il calcolo di una funzione. In questa fase proviamo
invece un approccio alternativo, che però vedremo in seguito
essere correlato al metodo a cui siamo stati abituati fino ad
ora.

Un esempio molto semplice di correlazione tra i due metodi è
quello dato dalla correlazione tra una funzione e il suo dominio.
\[
	\lambda x . 2 x \leftrightarrow \{ \N \} \qquad
	\lambda x . x / 2 \leftrightarrow \{ 2n \mid n \in \N \}
\]
Oppure tra una funzione e la suna immagine
\[
	\lambda x . 2 x \leftrightarrow \{ 2 n \mid n \in \N \}
	\qquad \lambda x . x / 2 \leftrightarrow \{ \N \}
\]
Si noti che ci sono infinite numerabili funzioni che hanno i
naturali come dominio, ossia quelle totali, e altrettante che
hanno i naturali come immagine. Si noti come la funzione che
calcola il doppio è totale e che la sua immagine sono i numeri
pari, quindi non tutti i naturali. In ogni caso per questi
esempi sia i domini che le immagini sono insiemi ricorsivi
di cui abbiamo dato la definizione precedentemente
(\ref{def: relazione ricorsiva}) ma che ripetiamo qui. Un
insieme $I$ è ricorsivo (ovvero \emph{decidibile}) se e solo se
la sua funzione caratteristica
\[
	\chi_I (x) = \begin{cases}
		1 & \text{se } x \in I    \\
		0 & \text{se } x \notin I
	\end{cases}
\]
è calcolabile totale. Un altro esempio di insieme ricorsivo è
costituito dai numeri che soddisfano la condizione di Goldbach,
ovvero quelli che sono somma di due numeri primi
\[ \{ n = p + q \mid p, q \text{ primi} \} \]
Facciamo ora un esemppio di insiemi che si possono definire
quando limitiamo il numero di passi nel calcolo di una MdT.

\begin{example}
	Siano dati $k, z \in \N$. Gli insiemi
	\begin{align*}
		I & = \{ (i, x, k) \mid \exists y, n . (x,n < k)
		\land (M_i \text{ calcola } y = \varphi_i (x)
		\text{ in } n \text{ passi}) \}                  \\
		J & = \{ (i,x,k,z) \mid \exists n . (n < k)
		\land (M_i \text{ calcola } z = \varphi_i (x)
		\text{ in } n \text{ passi}) \}
	\end{align*}
	sono entrambi ricorsivi infatti la procedura calcolabile
	totale che decide $I$ è la seguente:
	\begin{enumerate}
		\item Muoviamo il cursore della MdT $M_i$ su $x < k$
		      per al massimo $n-1$ passi.
	\end{enumerate}
\end{example}
