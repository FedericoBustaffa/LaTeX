\section{Diagonalizzazione per funzioni parziali}
A questo punto non ci rimane che capire se i formalismi che
abbiamo definito fino ad ora possono esprimere almeno tutte
le funzioni parziali calcolabili.

Come sappiamo dalla definizione \ref{def: funzione parziale}
le funzioni parziali non richiedono di essere definite ovunque
e questo ci tornerà comodo.

Cerchiamo quindi di dimostrare che la diagonalizzazione non si
applica anche a queste funzioni. Come prima possiamo enumerare
tutte le funzioni e prendere $\psi_n$, ossia l'$n$-esima
funzione nella lista, e applichiamo la diagonalizzazione.
Poniamo dunque
\[ \varphi (x) = \psi_x (x) + 1 \]
Supponiamo ora che $\varphi$ sia rappresentata dall'$n$-esimo
algoritmo: non possiamo tuttavia concludere che
\[ \varphi \neq \psi_n \]
perché $\psi_n(n)$ potrebbe non essere definita e dunque
divergere. Se $\psi_n(n)$ diverge, lo fa anche $\psi_n(n)+1$
e dunque le due funzioni sono equivalenti.

Le funzioni parziali, per fortuna, hanno senso, prendiamo ad
esempio la seguente funzione che definisce la divisione
\[ div (x,y) = \lfloor x / y \rfloor \]
che è definita solo se $y \neq 0$. Ci chiediamo quindi se sia
possibile estendere tutte le funzioni parziali a totali. Come
vedremo più avanti la risposta è no perché non sempre c'è un
algoritmo che calcola la versione estesa. Torniamo però al caso
della divisione. Se ad esempio definissimo accuratamente il
dominio della funzione, per esempio in questo modo
\[ div : \N \times \N \backslash \{ 0 \} \to \N \]
avremmo comunque il problema che non tutte le coppie di naturali
sono coperte. Introduciamo quindi un simbolo speciale
$* \notin \N$ in modo da avere ancora una funzione per tutte le
possibili coppie di naturali
\[ div^* : \N \times \N \to \N \cup \{ * \} \]
definita in questo modo
\[
	div^* (x, y) = \begin{cases}
		div(x, y) & \text{se } y \neq 0 \\
		*         & \text{se } y = 0
	\end{cases}
\]
Non vogliamo però liberarci della parzialità, dato che, come
abbiamo appena visto, la diagonalizzazione non funziona in
questo caso. Abbiamo quindi bisogno di un formalismo per
definire anche le funzioni parziali.
