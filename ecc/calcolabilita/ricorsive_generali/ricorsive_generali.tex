\chapter{Funzioni ricorsive generali}
Introduciamo ora le \textbf{funzioni ricorsive generali} come
un'estensione delle ricorsive primitive. In particolare vogliamo
aggiungere, agli schemi primitivi di base di \emph{combinazione}
e \emph{ricorsione primitiva} (\ref{def: ricorsive primitive}),
lo schema di \textbf{minimizzazione}.

Dobbiamo prima però introdurre un po' di notazione: l'operatore
$\mu$, detto \textbf{operatore di minimizzazione}, applicato
ad un insieme di numeri naturali, ne restituisce il minimo, se
presente.

\begin{definition} \label{def: mu ricorsive}
	La classe delle funzioni \textbf{$\mu$-ricorsive} (o
	\textbf{ricorsive generali}) è la minima classe
	$\RR$ tale che soddisfa le condizioni di
	\begin{itemize}
		\item \emph{Zero} e \emph{ricorsione primitiva}.
		\item \textbf{Minimizzazione}: se
		      $\varphi (x_1, \dots, x_n, y) \in \RR$ in $n+1$
		      variabili, allora la funzione $\psi$ in $n$
		      variabili è in $\RR$ se è definita dal minimo $y$
		      tale che
		      \begin{itemize}
			      \item $\varphi(x_1, \dots, x_n, y) = 0$
			      \item $\forall z \leq y$ vale $\varphi(x_1,
				            \dots, x_n, z) \downarrow$.
		      \end{itemize}
		      In forma più compatta la condizione di
		      minimizzazione è la seguente
		      \[
			      \psi (x_1, \dots, x_n) = \mu y [
					      \forall z \leq y, \;
					      \varphi(x_1, \dots, x_n, z)
					      \downarrow]
		      \]
	\end{itemize}
\end{definition}

Una funzione $\mu$-ricorsiva è intuitivamente calcolabile poiché
l'algoritmo \emph{intuitivo} che la calcola è composto da un
ciclo in cui si incrementa la variabile $y$ (inizialmente posta
a $0$), si calcola la $\varphi$ e si ripetono questi passi
finché il risultato non è $0$. I primi passi dell'esecuzione
di questo algoritmo potrebbero dunque essere:
\begin{enumerate}
	\item Calcolare $\varphi(x_1, \dots, x_n, 0)$. Se il
	      risultato è $0$, allora $\psi (x_1, \dots, x_n) = 0$.
	\item Altrimenti si calcola $\varphi(x_1, \dots, x_n, 1)$,
	      se il risultato è $0$, allora
	      $\psi(x_1, \dots, x_n)=1$.
	\item ...
\end{enumerate}
Intuitivamente l'algoritmo potrebbe non terminare mai perché
\begin{itemize}
	\item Per ogni valore di $y$ esiste un $m_y$ tale che
	      \[ \varphi (x_1, \dots, x_n, y) = m_y \neq 0 \]
	\item Per i primi $k$ numeri naturali vale che
	      \[
		      \varphi (x_1, \dots, x_n, z) = n_z \neq 0
		      \quad \land \quad
		      \varphi (x_1, \dots, x_n, k) \uparrow
	      \]
\end{itemize}
Nel primo caso infatti continuiamo a calcolare
$\varphi(x_1, \dots, x_n, y)$ per valori crescenti di $y$ senza
quindi terminare mai. Nel secondo caso non ci arrestiamo mai nel
calcolo di $\varphi (x_1, \dots, x_n, k) \uparrow$ poiché la
funzione diverge: da qui la parzialità di $\psi$.

\begin{example}
	Prendiamo ad esempio la seguente funzione
	\begin{align*}
		\varphi       & = \lambda x, y . 3 \\
		\psi_\uparrow & =
		\lambda x . (\mu y . \varphi(x, y) = 0)
	\end{align*}
	In questo caso possiamo facilmente notare che $\forall x$ il
	calcolo di $\varphi$ termina. Altrettanto facile è
	verificare che, per nessun $x$, esiste un $y$ tale che
	$\varphi (x, y) = 0$ e quindi la funzione $\psi$ è indefinita
	per ogni valore di $x$.
\end{example}

Dobbiamo quindi partire da una funzione ricorsiva primitiva, a
cui si applica l'operatore di minimizzazione $\mu$ per ottenere
una funzione $\mu$-ricorsiva. Tutto ciò a patto che la condizione
che per ogni $z \leq y$, la funzione $\varphi (x_1, \dots, x_n)$
converge. In caso contrario la funzione $\psi$ potrebbe non
essere $\mu$-ricorsiva. Si noti anche che
\[
	f(x) = \begin{cases}
		\mu y [y < g(x), \; h(x, y) = 0] &
		\text{se esiste tale } y                             \\
		0                                & \text{altrimenti}
	\end{cases}
\]
è ricorsiva primitiva se $g$ e $h$ lo sono. La ragione è che $g$
impone un limite ai tentativi di ricercare il minimo $y$, e
quindi o lo si trova in meno di $g(x)$ applicazioni di $h$ o
diamo risultato $0$.

\begin{theorem}[Tesi di Church-Turing] \label{th: church-turing}
	Le funzioni intuitivamente calcolabili sono tutte e sole
	le funzioni parziali T-calcolabili.
\end{theorem}

Più che un teorema, questo appena enunciato, è una tesi che è
stata dimostrata per i formalismi già definiti ma che non può
essere dimostrata per quelli non ancora definiti. Si tratta
comunque di una tesi talmente forte che viene trattata come un
teorema.

