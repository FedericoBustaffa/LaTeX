\section{Forma normale ed equivalenze}
Come abbiamo già visto, è possibile enumerare le MdT, associando
loro un indice. Analogamente è possibile enumerare le funzioni
ricorsive primitive e non è difficile pensare ad un'estensione
per l'enumerazione di funzioni $\mu$-ricorsive.

I due metodi hanno in comune il fatto che si basano solamente
sui simboli usati per definire gli algoritmi. Infatti, sotto
ragionevoli ipotesi, per i nostri scopi non c'è differenza tra
un metodo di enumerazione e l'altro, purché sia \emph{effettivo}.

Data un'enumerazione effettiva, indicheremo con $\varphi_i$ la
funzione parziale che la macchina, o meglio l'algoritmo, $M_i$
calcola e chiameremo $i$ \emph{indice}. L'indice è riferito
alla macchina e non alla funzione, infatti potrebbe darsi che
per $i \neq j$, valga $\varphi_i = \varphi_j$, ma sicuramente
vale $M_i \neq M_j$.

\begin{theorem}[Padding lemma] \label{th: padding lemma}
	Ogni funzione calcolabile $\varphi_x$ ha $\# (N)$ indici.
	Vale inoltre che $\forall x$ si può costruire, mediante una
	funzione ricorsiva primitiva, un insieme infinito $A_x$ di
	indici tale che $\forall y \in A_x$, vale
	\[ \varphi_y = \varphi_x \]
	\begin{proof}
		Per ogni macchina $M_x$, se $Q = \{ q_0, \dots, q_k \}$
		è l'insieme degli stati possibili di $M_x$. Aggiungendo
		lo stato $q_{k+1}$ e la quintupla
		\[ (q_{k+1}, \#, q_{k+1}, \#, -) \]
		a tale macchina, si ottiene la macchina $M_{x_1}$ con
		$x_1 \in A_x$. Possiamo continuare all'infinito
		aggiungendo stati e quintuple che di fatto sono inutili,
		o per meglio dire non vengono mai raggiunti dalla
		macchina e non cambiano quindi cosa essa calcola.
	\end{proof}
\end{theorem}

Questo teorema ci dice che esistono infiniti algoritmi
numerabili che calcolano la stessa funzione e che alcuni di
loro sono ottenibili \emph{facilmente} da un algoritmo dato.

\begin{theorem}[Forma normale] \label{th: fn}
	Esistono un predicato $T(i, x, y)$ e una funzione $U(y)$
	calcolabili totali tali che $\forall i,x$ vale
	\[ \varphi_i(x) = U(\mu y . T (i, x, y)) \]
	Inoltre $T$ e $U$ sono ricorsive primitive.
	\begin{proof}
		Definiamo $T(i,x,y)$, detto comunemente
		\textbf{predicato di Kleene}, vero se e solo se $y$ è
		la codifica di una computazione terminante di $M_i$
		con dato iniziale $x$. Per calcolare $T$ dato $i$,
		recuperiamo $M_i$ dalla lista e cominciamo a scandire
		i valori $y$. Decodifichiamo ognuno di essi e, avendo
		come ingresso $x$ controlliamo se il risultato è una
		computazione terminante della forma
		\[ M_i(x) = c_0, c_1, \dots, c_n \]
		Se lo è, allora $c_n = (h, \start z \underline{\#})$ e
		definiamo $U$ in modo che $U(y) = z$.

		Il procedimento è effettivo e quindi $T$ e $U$ sono
		calcolabili per la tesi di Church-Turing, inoltre tale
		procedimento termina sempre e dunque $T$ e $U$ sono
		totali. Abbiamo inoltre che $T$ e $U$ sono ricorsive
		primitive perché sia le codifiche usate, che i controlli
		effettuati lo sono.
	\end{proof}
\end{theorem}

Questo teorema ci dice che tra tutti gli algoritmi che calcolano
la stessa funzione, uno di questi ha una forma privilegiata,
ossia quella \emph{normale} e di conseguenza ogni funzione ha
una forma normale.

Proviamo ad uscire un minimo dal formalismo del teorema stesso
procedendo per step. Iniziamo dal predicato di Kleene: esso è
una funzione che semplicemente ritorna vero se $y$ è la codifica
(in questo caso possiamo considerare l'enumerazione di G\"odel)
di una computazione terminante della macchina $M_i$ che prende
in input $x$.

Come abbiamo già visto una computazione, vista come una sequenza
finita di passi è anch'essa enumerabile ed è quindi possibile
immaginare che valga
\[ M_i (x) = c_0, c_1, \dots, c_n \]
dove i vari $c_i$ sono le varie configurazioni raggiunte dalla
macchina durante la computazione. Se la computazione termina
allora abbiamo che $T$ ritorna vero e che sicuramente l'ultima
configurazione, ossia $c_n$ è equivalente a
\[ c_n = (h, \start z \underline{\#}) \]
poiché siamo sicuramente giunti nello stato speciale $h$. Ho
qualche dubbio sul fatto che il cursore debba essere perforza
posizionato su un valore bianco in quanto, da quel che abbiamo
visto fino ad ora, l'unico requisito affinché una computazione
venga considerata terminata è che giunga nello stato $h$.

Ciò che è importante capire è che $U$ è la parte di stringa
della configurazione finale, privata del respingente.

\begin{example}
	Riprendiamo l'esempio della MdT in grado di riconoscere la
	presenza di una sequenza di due $1$ in una stringa data in
	input.

	In quell'esempio avevamo definito una funzione di
	transizione che sovrascriveva qualunque valore incontrasse
	con un $\#$. Per qualsiasi stringa in input che contenesse
	una sequenza di due $1$ consecutivi, la macchina si sarebbe
	sempre arrestata con la seguente configurazione finale
	\[ h / \start \# \dots \underline{\#} \]
	Per il teorema dobbiamo andare a prendere il \emph{minimo}
	$y$ che rappresenta una computazione terminante della
	macchina $i$-esima che prende un certo input $x$.

	Nel caso delle macchina di Turing ci sarà in realtà un solo
	$y$ del genere (discorso differente ad esempio per le
	funzioni ricorsive in cui possiamo avere diverse politiche
	di valutazione). Nel nostro specifico, tale $y$, sarà quello
	che codifica la computazione della macchina $i$ (quella che
	abbiamo scritto noi) che prende in input la stringa
	$x = 01011$ e che quindi avrà come configurazione finale
	\[ h / \start \# \# \# \# \underline{\#} \]
	Secondo il teorema vale che
	\[ \varphi_i (x) = \varphi (01011) = U(y) = \# \# \# \# \# \]
	che in realtà andrebbe convertito in nel formalismo
	ricorsivo perché abbiamo effettivamente più senso.
\end{example}

\begin{corollary}
	Le funzioni T-calcolabili sono $\mu$-ricorsive.
\end{corollary}

Questo corollario è una diretta conseguenza del teorema di
forma normale. Possiamo quindi dire che ogni funzione calcolata
da una MdT ammette una definizione $\mu$-ricorsiva.

\begin{lemma}
	Le funzioni $\mu$-calcolabili sono T-calcolabili.
\end{lemma}

Possiamo a questo punto concludere che l'equivalenza tra MdT
e funzioni $\mu$-ricorsive.

\begin{theorem}
	Una funzione è T-calcolabili se e solo se è
	$\mu$-calcolabile.
\end{theorem}

Il teorema di forma normale e quello d'equivalenza tra MdT e
funzioni $\mu$-ricorsive ha il seguente corollario interessante
dal punto di vista informatico. La sua rilevanza nel nostro
campo è legata al fatto che le funzioni primitive ricorsive
si possono rappresentare come un programma in linguaggio
\verb|FOR|, mentre le $\mu$-ricorsive con un programma in
linguaggio \verb|WHILE|.

\begin{corollary}
	Ogni funzione calcolabile parziale può essere può essere
	ottenuta da due funzioni ricorsive primitive e una sola
	applicazione dell'operatore $\mu$.
\end{corollary}