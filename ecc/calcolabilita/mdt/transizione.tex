\section{Funzione di transizione}
La \textbf{funzione di transizione} $\delta$ è necessaria a
far progredire il calcolo. Come possiamo vedere dalla
definizione che ne abbiamo dato, questa prende in input una
coppia di valori.
\begin{itemize}
	\item Lo \textbf{stato corrente} $q$ della macchina.
	\item Un simbolo $\sigma$ dell'alfabeto
\end{itemize}

Notiamo inoltre che la funzione $\delta$ prende in input una
coppia di valori ma ritorna una tripla composta da uno stato
$q'$, che può essere anche $h$ in caso il calcolo sia terminato
con successo, un simbolo $\sigma'$ e una tra le 3 mosse $L$,
$R$ e $-$.

Ad ogni passo del calcolo, $\delta$ elabora la coppia in input
e restituisce una tripla contente il nuovo stato, il nuovo
simbolo da scrivere alla posizione attuale del cursore e come
ci si deve muovere al passo successivo.

Qualcuno potrebbe aver notato la similitudine con un automa a
stati finiti. In quanto vi è sempre uno stato corrente e, in
base a quello e ad una nuova parte dell'input ci si muove in
un nuovo stato o si rimane nello stesso. Non è tuttavia
corretto dire che una MdT è un automa.

\subsection{Considerazioni}
Ora che abbiamo le idee più chiare, facciamo qualche
considerazione in più su $\delta$. Essa è \textbf{iniettiva},
vale cioè che, prese due triple $(q', \sigma', D')$ e
$(q'', \sigma'', D'')$ tali che
\begin{gather*}
	\delta (q, \sigma) =  (q', \sigma', D') \\
	\delta (q, \sigma) = (q'', \sigma'', D'')
\end{gather*}
allora $q' = q''$, $\sigma' = \sigma''$ e $D' = D''$.
Questo ci dice sostanzialmente che, dato uno stato e un simbolo
c'è un solo altro stato in cui possiamo andare. Un'altra cosa
da specificare è che per $\delta$ vale sempre che se
\[ \delta(q, \start) = (q', \sigma, D) \]
allora $\sigma = \start$ e $D = R$. Questo ci dice che se ci
troviamo all'inizio del nastro possiamo andare solo a destra.

Ritornando velocemente
all'\hyperref[sec: algoritmo]{idea intuitiva di algoritmo},
possiamo facilmente verificare la prima e seconda condizione
sono verificate poiché, sia $Q$ che $\Sigma$ sono insiemi
finiti e di conseguenza anche $\delta$ contiene un numero
finito di elementi.

\subsection{Alfabeto}
I dati su cui opera una MdT sono stringhe $w$ di caratteri
appartenenti a $\Sigma$, più precisamente $w \in \Sigma^*$,
dove $\Sigma^*$ comprende anche la stringa vuota $\epsilon$.

Senza stare a incasinarsi con inutili formalismi matematici,
se $\Sigma$ è l'alfabeto, allora $\Sigma^*$ è l'insieme di
tutte le possibili stringhe generabili con quell'alfabeto e
la stringa vuota.

\begin{example}
	Prendiamo ad esempio l'alfabeto binario
	\begin{gather*}
		\Sigma = \{ 0, 1 \} \\
		\Downarrow \\
		\Sigma^* = \{ \epsilon, 0, 1, 01, 10, 11, \dots \}
	\end{gather*}
\end{example}