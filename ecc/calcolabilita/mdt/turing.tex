\chapter{Macchine di Turing}
Introduciamo il primo formalismo per esprimere algoritmi, ideato
da Alan Turing nel 1936 e che prende il nome di
\textbf{macchina di Turing}.

L'idea di base è quella di avere un \textbf{nastro} di lunghezza
infinita su cui disporre una stringa di caratteri, rappresentante
l'input dell'algoritmo. Tale macchina è dotata di un
\textbf{cursore} che, partendo dall'inizio della stringa, legge
i caratteri che incontra e, in base a cosa legge, cambia stato
e si muove. La computazione termina quando si giunge finalmente
in uno stato speciale di terminazione.

\begin{definition}
	Una \textbf{macchina di Turing} (MdT) è definita come una
	quadrupla di questo tipo
	\[
		M = \begin{pmatrix}
			Q, & \Sigma, & \delta, & q_0
		\end{pmatrix}
	\]
	dove
	\begin{itemize}
		\item $Q$ è l'\textbf{insieme finito degli stati} in
		      cui si può trovare la macchina. Tra gli stati
		      abbiamo uno stato speciale $h$, con cui
		      indicheremo la corretta terminazione del calcolo
		      della macchina $M$.
		\item $\Sigma = \{ \sigma, \sigma', \dots \}$ è
		      l'\textbf{insieme finito dei simboli}, ossia,
		      l'\textbf{alfabeto} utilizzato per esprimere
		      gli algoritmi. Il numero e e i simboli stessi
		      possono variare da una MdT all'altra, tuttavia
		      ci sono dei simboli considerato speciali che sono
		      sempre presenti:
		      \begin{itemize}
			      \item \textbf{Bianco}: indicato con $\#$ è
			            come un carattere nullo.
			      \item \textbf{Respingente}: indicato con
			            $\start$, simboleggia l'inizio delle
			            stringa. Il cursore della macchina non
			            può andare mai a sinistra di questo
			            simbolo.
			      \item \textbf{Spostamento}: simboli che non
			            appartengono a $\Sigma$ e indicano come
			            e in che direzione muovere il cursore.
			            Sono rispettivamnete $L$, $R$ e $-$ che
			            stanno per \verb|Left|, \verb|Right| e
			            \verb|Hold| (non muovere il cursore).
		      \end{itemize}
		\item \textbf{Stato iniziale}: indicato con $q_0 \in Q$
		      è lo stato iniziale in cui si trova la macchina.
		\item \textbf{Funzione di transizione}: indicata con
		      $\delta$ è la funzione che definisce la
		      transizione da uno stato all'altro della macchina
		      in base a quel che viene letto dal cursore e in
		      base a quel che è stato letto fino a quel momento.
		      \[
			      \delta \subseteq (Q \times \Sigma) \to
			      (Q \cup \{ h \}) \times \Sigma \times
			      \{ L, R, - \}
		      \]
		      che ci permette per l'appunto di cambiare lo
		      stato della macchina e progredire nel calcolo.
	\end{itemize}
\end{definition}
