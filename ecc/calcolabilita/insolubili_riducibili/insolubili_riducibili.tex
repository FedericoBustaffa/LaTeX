\chapter{Problemi insolubili e riducibilità}
In quest'ultimo capitolo di questa prima parte di calcolabilità
andremo a trattare i problemi in cui si cerca di capire se un
elemento appartiene o no ad un dato insieme.

Introdurremo infatti  nuovi insiemi con caratteristiche che ci
saranno utili in futuro e andremo a riprendere alcuni dei vecchi
argomenti come la diagonalizzazione e la codifica a coda di
colomba.

Fino ad ora ci siamo concentrati sulla risoluzione di problemi
tramite il calcolo di una funzione. In questa fase proviamo
invece un approccio alternativo, che però vedremo in seguito
essere correlato al metodo a cui siamo stati abituati fino ad
ora.

Calcolare una funzione o verificare l'appartenenza di un
elemento ad un insieme sono due metodi che in realtà sono
strettamente correlati tra di loro, ma quest'ultimo ci può
tornare più comodo per dimostrare alcuni risultati.
