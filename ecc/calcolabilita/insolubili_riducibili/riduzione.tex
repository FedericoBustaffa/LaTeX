\section{Riduzioni}
Il concetto di riduzione è fondamentale per riuscire a interagire
con classi di problemi differenti. Una \textbf{riduzione} è una
funzione che \emph{transforma} un problema (ovvero un insieme o
una classe) $A$ in un altro problema $B$, in modo da mantenerne
inalterata la caratteristica principale.

\begin{definition} \label{def: riduzione}
	Dati due problemi $A$ e $B$, diciamo che $A$ \textbf{si riduce}
	a $B$ secondo la \textbf{riduzione} $f$ e si indica con
	\[ A \leq_f B \]
	se e solo se
	\[ a \in A \iff f(a) \in B \]
	ovvero
	\[
		f(A) \subseteq B \quad \land
		\quad f(\overline{A}) \subseteq \overline{B}
	\]
\end{definition}

Vediamo ora una la prima proprietà interessante delle riduzioni
rispetto agli insiemi complementari degli insiemi di interesse.

\begin{property} \label{prop: riduzione_comp}
	$A$ si riduce a $B$ secondo $f$ se e solo se
	$\overline{A} \leq_f \overline{B}$.
	\begin{proof}
		Si ha che
		\[
			x \in \overline{A} \iff x \notin A \iff
			f(x) \notin B \iff f(x) \in \overline{B}
		\]
	\end{proof}
\end{property}

Più in generale si definisce una \emph{relazione di riduzioni}
o \emph{famiglia} $\leq_F$ dove $F$ è una particolare classe di
funzioni. Allora scriveremo
\[ A \leq_F B \iff \exists f \in F \mid A \leq_f B \]
Noi siamo interessati a quelle riduzioni $\leq_F$ che danno
origine a classi di problemi in qualche modo \emph{omogenei}.

\begin{lemma} \label{def: classificazione}
	Siano $\mathcal{D}$ e $\mathcal{E}$ due classi di problemi
	con $\mathcal{D} \subseteq \mathcal{E}$. Una relazione di
	riduzione $\leq_F$ \textbf{classifica} $\mathcal{D}$ e
	$\mathcal{E}$ se e solo se, per ogni problema $A$, $B$ e $C$
	\begin{itemize}
		\item $A \leq_F A$
		\item $A \leq_F B$ e $B \leq_F C$ implica $A \leq_F C$
		\item $A \leq_F B$ e $B \in \mathcal{D}$ implica
		      $A \in \mathcal{D}$
		\item $A \leq_F B$ e $B \in \mathcal{E}$ implica
		      $A \in \mathcal{E}$
	\end{itemize}
\end{lemma}

Di seguito vediamo una caratterizzazione differente ma del tutto
equivalente per le riduzioni che classificano coppie di classi,
l'una inclusa nell'altra.

\begin{lemma} \label{lemma: classificazione}
	Una relazione di riduzione $\leq_F$ classifica $\mathcal{D}$
	e $\mathcal{E}$, con $\mathcal{D} \subseteq \mathcal{E}$, se
	e solo se
	\begin{itemize}
		\item $id \in F$
		\item $f, g \in F \implies f \circ g \in F$
		\item $f \in F, \; B \in \mathcal{D} \implies
			      \{ x \mid f(x) \in B \} \in \mathcal{D}$
		\item $f \in F, \; B \in \mathcal{E} \implies
			      \{ x \mid f(x) \in B \} \in \mathcal{E}$
	\end{itemize}
\end{lemma}

Attraverso il concetto di relazione di riduzione che classifica
due classi di problemi si possono definire le seguenti nozioni
molto importanti.

\begin{definition}
	Se $\leq_F$ classifica $\mathcal{D}$ e $\mathcal{E}$, vale
	che per ogni problema $A, B, H$
	\begin{itemize}
		\item $A \equiv_F B$ se $A \leq_F B$ e $B \leq_F A$.
		\item $H$ è \textbf{$\leq_F$-arduo} per $\mathcal{E}$ se
		      $\forall A \in \mathcal{E}$ vale $A \leq_F H$.
		\item $H$ è \textbf{$\leq_F$-completo} per $\mathcal{E}$
		      se $H$ è $\leq_F$-arduo per $\mathcal{E}$ e
		      $H \in \mathcal{E}$.
	\end{itemize}
\end{definition}

Questa definizione sarà fondamentale anche in teoria della
complessità, prendiamoci quindi un attimo per comprenderla
meglio:
\begin{itemize}
	\item La prima definizione ci dice che due problemi $A$ e
	      $B$ sono equivalenti se uno si riduce all'altro ed è
	      vero anche il viceversa.
	\item Un problema $H$ è \emph{arduo} per $\mathcal{E}$ se
	      tutti i problemi appartenenti a $\mathcal{E}$ si
	      riducono ad $H$ stesso.
	\item Un problema $H$ è \emph{completo} per $\mathcal{E}$ se
	      è arduo per $\mathcal{E}$ e fa inoltre parte di
	      $\mathcal{E}$ stesso.
\end{itemize}
Si dice anche che l'insieme
\[ \{ B \mid A \equiv_F B \} \]
è il \textbf{grado} di $A$ o anche che $A$ è \textbf{equivalente}
a $B$ rispetto alla riduzione $\leq_F$. Quando la classe di
riduzioni $F$ è fissata, diremo semplicemente $\mathcal{E}$-arduo
o $\mathcal{E}$-completo

La relazione di riduzione $\leq_F$ è quello che viene chiamato
\textbf{pre-ordine parziale}, cioè un ordinamento parziale,
riflessivo e transitivo ma non anti-simmetrico. Nello specifico
\begin{itemize}
	\item Ogni elemento è in relazione con se stesso: $a \leq a$.
	\item Vale la transitività: $a \leq b$ e $b \leq c$ implicano
	      $a \leq c$.
	\item Non vale l'anti-simmetria: se $a \leq b$ e $b \leq a$
	      non possiamo dedurre che $a = b$.
\end{itemize}

\begin{property}
	Se $\leq_F$ \textbf{classifica} $\mathcal{D}$ ed $\mathcal{E}$
	con $\mathcal{D} \subseteq \mathcal{E}$ e $C$ è completo per
	$\mathcal{E}$, allora
	\[ C \in \mathcal{D} \iff \mathcal{D} = \mathcal{E} \]
	\begin{proof}
		La parte
		\[ \mathcal{D} = \mathcal{E} \implies C \in \mathcal{D} \]
		è ovvia poiché, $C \in \mathcal{E}$ dato che è
		$\mathcal{E}$-completo.
		Vogliamo invece dimostrare che
		\[ C \in \mathcal{D} \implies \mathcal{D} = \mathcal{E} \]
		Partiamo dal fatto che $C \in \mathcal{E}$ per condizione
		di completezza. Dato che $\leq_F$ classifica
		$\mathcal{D}$ ed $\mathcal{E}$ valgono in particolare
		due proprietà per ogni problema $A$:
		\begin{gather*}
			A \leq_F C \land C \in \mathcal{E} \implies A \in \mathcal{E} \\
			A \leq_F C \land C \in \mathcal{D} \implies A \in \mathcal{D}
		\end{gather*}
		Stiamo quindi considerando tutti i problemi $A$ che si
		riducono a $C$ secondo $F$. Ma come possiamo vedere sono
		tutti parte sia di $\mathcal{D}$ che di $\mathcal{E}$.
	\end{proof}
\end{property}

Inoltre è facile capire quali siano gli elementi del grado di
$A$, problema $\leq_F$-completo per $\mathcal{E}$.

\begin{property}
	Se $A$ è $\mathcal{E}$-completo, $A \leq_F B$ e
	$B \in \mathcal{E}$, allora $B$ è $\mathcal{E}$-completo.
	\begin{proof}
		Il fatto che $A$ sia $\mathcal{E}$-completo implica che
		per ogni $C \in \mathcal{E}$ vale
		\[ C \leq_F A \]
		Inoltre, dato che $\leq_F$ classifica $\mathcal{E}$ vale
		la transitività e quindi
		\[
			C \leq_F A \quad \land \quad A \leq_F B
			\implies C \leq_F B
		\]
		Deduciamo quindi che $B$ è $\mathcal{E}$-arduo ma dato
		che $B \in \mathcal{E}$ allora è $\mathcal{E}$-completo.
	\end{proof}
\end{property}

Un problema completo per $\mathcal{E}$ rappresenta la difficoltà
massima dei problemi di $\mathcal{E}$. Infatti, è facile vedere
che il grado di un problema $A$ completo per $\mathcal{E}$ è il
grado massimo di $\mathcal{E}$ in  $\leq_F$. Valgono inoltre le
seguenti affermazioni (anche per problemi non completi):
\begin{itemize}
	\item Se $B \leq_F A$ allora $B$ al più ha il grado di $A$,
	      cioè è più facile o difficile quanto $A$.
	\item Se $A \leq_F B$ allora $B$ ha almeno il grado di $A$,
	      cioè è almeno difficile quanto $A$.
\end{itemize}
Passiamo ora riformulare le definizioni di
\hyperref[def: riduzione]{riduzione} e
\hyperref[def: classificazione]{classificazione} per ottenere
il concetto di riducibilità che ci interessa. Per farlo diamo
un nome alla classe di funzioni calcolabili
\[ rec = \{ \varphi_x \mid \forall y \in \N . \varphi_x(y) \downarrow \} \]

\begin{definition}
	Il problema $A$ è \textbf{riducibile} a $B$ se e solo se
	esiste una funzione calcolabile totale $f : \N \to \N$ tale
	che
	\[ x \in A \iff f(x) \in B \]
\end{definition}

Vogliamo vedere ora come le relazioni di riduzione conservano la
ricorsività e la ricorsiva enumerabilità. D'ora in poi ci
riferiremo agli insiemi $\mathcal{R}$ ed $\mathcal{RE}$ come le
classi di problemi rispettivamente ricorsivi e ricorsivamente
enumerabili.

\begin{theorem}
	La relazione di riduzione $\leq_{rec}$ classifica
	$\mathcal{R}$ ed $\mathcal{RE}$.
	\begin{proof}
		Sappiamo già che $\mathcal{R} \subseteq \mathcal{RE}$
		grazie alla proprietà \ref{prop: r_re}. Possiamo quindi
		usare il lemma \ref{lemma: classificazione} verificando
		che tutte le proprietà siano soddisfatte:
		\begin{itemize}
			\item L'identità appartiene a $rec$ poiché una
			      funzione calcolabile è $\mu$-ricorsiva e tra
			      i costrutti di $\mu$-ricorsione vi è anche
			      l'identità.
			\item La composizione preserva la totalità.
			\item La funzione caratteristica di
			      $\{ x \mid f(x) \in B \}$ è $\chi_B \circ f$
			      che è calcolabile totale perché $f$ e $\chi_B$
			      lo sono.
			\item Analogo al punto precedente ma con la
			      semi-caratteristica di $B$.
		\end{itemize}
	\end{proof}
\end{theorem}

Nei teoremi e negli esercizi useremo principalmente funzioni
di riduzione iniettive di solito ricavate tramite il teorema
del parametro.

Il fatto che $\leq_{rec}$ classifichi $\mathcal{R}$ ed
$\mathcal{RE}$ può essere visto come la capacità che le
riduzioni con funzioni calcolabili totali hanno di
\textbf{separare} i problemi ricorsivi da quelli ricorsivamente
enumerabili.

Come abbiamo già visto, questo viene fatto giocando sul tempo
necessario a decidere un problema: \textbf{finito} nel caso
in cui sia ricorsivo, \textbf{infinito} in caso contrario.

Notiamo anche che ci basta trovare un problema che sia
$\leq_{rec}$-completo per $\mathcal{R}$ per poter vedere quali
problemi sono decidibili e quali no.

Ancora più interessante è trovare un problema
$\leq_{rec}$-completo per $\mathcal{RE}$, così da capire quali
problemi sono al più semi-decidibili e quali nemmeno
semi-decidibili. Basta infatti ridurre il problema da studiare a
quello completo per sapere che è ricorsivamente enumerabile
oppure ridurre il problema completo a quello da studiare e
sapremo che quest'ultimo, ben che vada, è ricorsivamente
enumerabile.

Prima di andare avanti vediamo che $K$ non si riduce, con
funzioni calcolabili totali, a $\overline{K}$ né il viceversa.
Ci interessa vedere che un insieme ricorsivamente enumerabile
può non essere confrontabile con uno non ricorsivamente
enumerabile, usando $\leq_{rec}$. Dobbiamo quindi mostrare che
\[
	\overline{K} \not\leq_{rec} K \qquad \text{e}
	\qquad K \not\leq_{rec} \overline{K}
\]
La prima disuguaglianza dev'essere vera, altrimenti per la
proprietà \ref{prop: r_re}, la quale ci dice che un insieme e
il suo complemento sono ricorsivamente enumerabili se e solo se
sono ricorsivi, l'insieme $\overline{K}$ sarebbe ricorsivamente
enumarabile e anche ricorsivo. Abbiamo però dimostrato nel
teorema \ref{th: K_re_not_r} che $K$ non è ricorsivo.

La seconda disuguaglianza si dimostra vera ricorrendo alla
proprietà \ref{prop: riduzione_comp} la quale ci dice che
\[ A \leq_f B \iff \overline{A} \leq_f \overline{B} \]
e dunque se $K \leq_{rec} \overline{K}$ allora anche
$\overline{K} \leq_{rec} K$ che però è appena stato dimostrato
falso.

\begin{theorem}
	L'insieme $K$ è $\mathcal{RE}$-completo, ovvero
	$\leq_{rec}$-completo per $\mathcal{RE}$.
	\begin{proof}
		Dobbiamo dimostrare che se $A \in \mathcal{RE}$ allora
		$A \leq_{rec} K$. Per definizione $A$ è il dominio di
		una funzione calcolabile $\psi$, ovvero
		\[ A = \{ x \mid \psi (x) \downarrow \} \]
		A partire da $\psi$ definiamo la funzione $\psi'$ a due
		variabili di cui ignoriamo la seconda
		\[ \psi' (x, y) = \psi(x) \]
		che è calcolabile, dunque ha un indice $i$ tale che
		$\psi' = \varphi_i$, allora per il teorema del parametro
		\[ \psi' (x, y) = \varphi_i(x, y) = \varphi_{s(i, x)} (y) \]
		con $s$ calcolabile, iniettiva e totale. Possiamo quindi
		scrivere la definizione di $A$ come segue
		\begin{align*}
			A = & \{ x \mid \psi(x) \downarrow \}         \\
			=   & \{ x \mid \psi'(x, y) \downarrow \}     \\
			=   & \{ x \mid \varphi_i(x, y) \downarrow \}
		\end{align*}
		Adesso applichiamo il teorema del parametro e buttiamo
		via $y$ sostituendola con $s(i, x)$:
		\begin{align*}
			= & \{ x \mid \varphi_{s(i, x)}(y) \downarrow \}       \\
			= & \{ x \mid \varphi_{s(i, x)}(s(i, x)) \downarrow \}
		\end{align*}
		A questo punto è immediato notare che abbiamo ricreato
		la stessa struttura degli elementi che appartengono
		all'insieme $K$ e possiamo quindi concludere che
		\[
			\{ x \mid \varphi_{s(i, x)}(s(i, x)) \downarrow \}
			= \{ x \mid s(i, x) \in K \}
		\]
		A questo punto, dato che vogliamo dimostrare che
		\[ A \in \mathcal{RE} \implies A \leq_{rec} K \]
		possiamo dire che $x \in A$ implica $f(x) \in K$ con
		$f \in rec$ ovvero $f$ calcolabile totale. Abbiamo quindi
		bisogno che $K$ sia immagine di una funzione calcolabile
		totale. Ma abbiamo appena dimostrato che lo è per $s$
		che abbiamo già visto essere calcolabile, totale (e
		iniettiva). Rimane il problema che $s$ ha due argomenti
		ma possiamo prendere 
		\[ f = \lambda x . s(i, x) \]
		che rimane calcolabile e totale perché $s$ lo è.
	\end{proof}
\end{theorem}

Questo teorema ci dice che tutti gli insiemi ricorsivamente
enumerabili si riducono, per mezzo di una funzione calcolabile
totale, a $K$.

\begin{definition}
	L'insieme $A$ è un \textbf{insieme di indici} che
	rappresentano le funzioni se e solo se $\forall x, y$
	\[
		x \in A \quad \land \quad \varphi_x = \varphi_y
		\quad \implies \quad y \in A
	\]
\end{definition}

Passiamo ora a studiare gli insiemi di indici $A$ che
rappresentano le funzioni e tali per cui $K \leq_{rec} A$ (oppure
$K \leq_{rec} \overline{A}$).

In questo modo sapremo quali classi di funzioni sono al massimo
semi-decidibili (ma non decidibili), perché, gli indici in $A$
individuano esattamente le funzioni calcolate dalle macchine con
indice in $A$.

\begin{theorem} \label{th: indici}
	Sia $A$ un insieme di indici che rappresentano le funzioni
	tale che
	\[ \emptyset \neq A \neq \N \]
	allora $K \leq_{rec} A$ oppure $K \leq_{rec} \overline{A}$.
	\begin{proof}
		Iniziamo con il prendere l'indice $i_0$ tale che
		$\varphi_{i_0}(y)$ sia ovunque indefinita. Supponiamo
		che $i_0 \in \overline{A}$ e dimostriamo che
		$K \leq_{rec} A$. Poiché $A \leq \emptyset$ prendiamo un
		altro indice $i_1 \in A$.

		Siamo d'accordo che $\varphi_{i_0} \neq \varphi_{i_1}$
		poiché gli indici provengono da due insiemi di indici
		tra di loro complementari. Definiamo ora la seguente
		funzione
		\[
			\psi (x, y) = \varphi_{f(x)} (y) = \begin{cases}
				\varphi_{i_1} (y)                      & \text{se } x \in K \\
				\text{indefinita } = \varphi_{i_0} (y) & \text{altrimenti}
			\end{cases}
		\]
		dove usando il \hyperref[th: s-m-n]{teorema del parametro}
		abbiamo determinato la $f$ funzione calcolabile totale e
		iniettiva. Vale allora la seguente catena di implicazioni
		\[
			x \in K \implies \varphi_{f(x)} = \varphi_{i_1}
			\implies f(x) \in A
		\]
		perché $i_1 \in A$ e $A$ è un insieme di indici che
		rappresentano le funzioni, dunque anche $f(x) \in A$ e
		dunque abbiamo dimostrato vera la prima disuguaglianza.
		Viceversa, dato che $i_0 \in \overline{A}$ vale questa
		catena di uguaglianze
		\[
			x \notin K \implies \varphi_{f(x)} = \varphi_{i_0}
			\implies f(x) \in \overline{A} \implies f(x) \notin A
		\]
	\end{proof}
\end{theorem}

In realtà nella dimostrazione appena vista si può procedere in
modo simmetrico e del tutto analogo prendendo $i_0 \in A$.

Esistono anche insiemi $B$ tali che $K \leq_{rec} B$ \textbf{e}
$K \leq_{rec} \overline{B}$ e cioè esistono funzioni $f$ e $g$
calcolabili totali e iniettive tali che $x \in K$ se e solo se
$f(x) \in B$ e $x \in K$ se e solo se $g(x) \in \overline{B}$.

\begin{theorem} [Rice] \label{th: rice}
	Sia $\mathcal{A}$ una classe di funzioni calcolabili.
	L'insieme
	\[ A = \{ n \mid \varphi_n \in \mathcal{A} \} \]
	è ricorsivo se e solo se $\mathcal{A} = \emptyset$ oppure
	$\mathcal{A}$ è la classe di tutte le funzioni calcolabili.
	\begin{proof}
		Si noti che $A$ è un insieme di indici mentre
		$\mathcal{A}$ è una classe di funzioni. La dimostrazione
		è immediata per i casi banali in cui
		$\mathcal{A} = \emptyset$ oppure quando è la classe di
		tutte le funzioni calcolabili.

		Negli altri casi, basta applicare il teorema
		\ref{th: indici} appena visto. Dato che $A$ è un insieme
		di indici non vuoto perché $\mathcal{A}$ contiene almeno
		una funzione e non coincide con $\N$, dunque non contiene
		tutte le funzioni calcolabili.
	\end{proof}
\end{theorem}

Questo teorema si ripercuote sulle proprietà dimostrabili sui
programmi in quanto ogni metodo di prova si scontra
inevitabilmente con il problema della fermata.

In realtà, per aggirare il problema sono stati sviluppati metodi
di \textbf{analisi statica} del codice, i quali riescono ad
analizzare il testo del programma parsandolo e raccogliendo
informazioni sugli oggetti che compaiono nel programma e come
questi verrano usati a tempo di esecuzione.

Se per esempio (in contesto single-thread) volessimo verificare
se un programma con all'interno un \verb|while| entra in un
ciclo infinito oppure no possiamo fare dei controlli sulle
variabili che decretano l'uscita dal ciclo. Se a queste variabili
non vengono mai assegnati nuovi valori dentro il ciclo allora
possiamo affermare con un certo grado di sicurezza che il
programma andrà in ciclo.
