\section{Riduzione}
Il concetto di \textbf{riduzione} è fondamentale per riuscire
a interagire con classi di problemi differenti.

Una \textbf{riduzione} è una funzione che \emph{transforma}
un problema $A$ (ovvero un insieme o una classe) $A$ in un
altro problema $B$, in modo da mantenerne inalterata la
caratteristica principale.

\begin{definition}
	Dati due problemi $A$ e $B$, diciamo che $A$ \textbf{si riduce}
	a $B$ secondo la \textbf{riduzione} $f$ e si indica con
	\[ A \leq_f B \]
	se e solo se
	\[ a \in A \iff f(a) \in B \]
	ovvero
	\[
		f(A) \subseteq B \qquad \land
		\qquad f(\overline{A}) \subseteq \overline{B}
	\]
\end{definition}

\begin{property}
	$A$ si riduce a $B$ secondo $f$ se e solo se
	$\overline{A} \leq_f \overline{B}$.
	\begin{proof}
		Si ha che
		\[
			x \in \overline{A} \iff x \notin A \iff
			f(x) \notin B \iff f(x) \in \overline{B}
		\]
	\end{proof}
\end{property}

Più in generale si può una \emph{relazione di riduzioni} o
\emph{famiglia} $\leq_F$ dove $F$ è una particolare classe di
funzioni. Allora scriveremo
\[ A \leq_F B \iff \exists f \in F \mid A \leq_f B \]
Noi siamo interessati a quelle riduzioni che $\leq_F$ che danno
origine a classi di problemi in qualche modo \emph{omogenei}.

\begin{definition}
	Siano $\mathcal{D}$ e $\mathcal{E}$ due classi di problemi
	con $\mathcal{D} \subseteq \mathcal{E}$. Una relazione di
	riduzione $\leq_F$ \textbf{classifica} $\mathcal{D}$ e
	$\mathcal{E}$ se e solo se, per ogni problema $A$, $B$ e $C$
	\begin{itemize}
		\item $A \leq_F A$
		\item $A \leq_F B$ e $B \leq_F C$ implica $A \leq_F C$
		\item $A \leq_F B$ e $B \in \mathcal{D}$ implica
		      $A \in \mathcal{D}$
		\item $A \leq_F B$ e $B \in \mathcal{E}$ implica
		      $A \in \mathcal{E}$
	\end{itemize}
\end{definition}

Di seguito vediamo una caratterizzazione differente ma del tutto
equivalente per le riduzioni che classificano coppie di classi,
l'una inclusa nell'altra.

\begin{lemma}
	Una relazione di riduzione $\leq_F$ classifica $\mathcal{D}$
	e $\mathcal{E}$, con $\mathcal{D} \subseteq \mathcal{E}$, se
	e solo se
	\begin{itemize}
		\item $id \in F$
		\item $f, g \in F \implies f \circ g \in F$
		\item $f \in F, \; B \in \mathcal{D} \implies
			      \{ x \mid f(x) \in B \} \in \mathcal{D}$
		\item $f \in F, \; B \in \mathcal{E} \implies
			      \{ x \mid f(x) \in B \} \in \mathcal{E}$
	\end{itemize}
\end{lemma}

Attraverso il concetto di relazione di riduzione che classifica
due classi di problemi si possono definire le seguenti nozioni
molto importanti.

\begin{definition}
	Se $\leq_F$ classifica $\mathcal{D}$ e $\mathcal{E}$, vale
	che per ogni problema $A, B, H$
	\begin{itemize}
		\item $A \equiv B$ se $A \leq_F B$ e $B \leq_F A$.
		\item $H$ è \textbf{$\leq_F$-arduo} per $\mathcal{E}$ se
		      $\forall A \in \mathcal{E}$ vale $A \leq_F H$.
		\item $H$ è \textbf{$\leq_F$-completo} per $\mathcal{E}$
		      se $H$ è $\leq_F$-arduo per $\mathcal{E}$ e
		      $H \in \mathcal{E}$.
	\end{itemize}
\end{definition}

Diremo semplicemente $\mathcal{E}$-arduo o $\mathcal{E}$-completo
quando la classe di riduzioni $F$ è fissata.
