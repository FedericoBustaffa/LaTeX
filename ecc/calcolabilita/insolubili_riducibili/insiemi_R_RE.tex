\section{Insiemi ricorsivi e ricorsivamente enumerabili}
Un esempio molto semplice di correlazione tra i due metodi è
quello tra una funzione e il suo dominio.
\[
	\lambda x . 2 x \leftrightarrow \{ \N \} \qquad
	\lambda x . x / 2 \leftrightarrow \{ 2n \mid n \in \N \}
\]
Oppure tra una funzione e la sua immagine
\[
	\lambda x . 2 x \leftrightarrow \{ 2 n \mid n \in \N \}
	\qquad \lambda x . x / 2 \leftrightarrow \{ \N \}
\]
Si noti che ci sono infinite numerabili funzioni che hanno i
naturali come dominio, ossia quelle totali, e altrettante che
hanno i naturali come immagine. Si noti come la funzione che
calcola il doppio è totale e che la sua immagine sono i numeri
pari, quindi non tutti i naturali. In ogni caso per questi
esempi sia i domini che le immagini sono
\hyperref[def: relazione ricorsiva]{insiemi ricorsivi}.

Facciamo ora un esempio di insiemi che si possono definire
quando limitiamo il numero di passi nel calcolo di una MdT.

\begin{example}
	Dato $k \in \N$, l'insieme
	\[
		I = \{ (i, x, k) \mid \exists y, n . (x,n < k)
		\land (M_i \text{ calcola } y = \varphi_i (x)
		\text{ in meno di } n \text{ passi}) \}
	\]
	è ricorsivo. Per determinarlo dobbiamo determinare se la
	sua funzione caratteristica lo è. A tal proposito procediamo
	in questo modo
	\begin{enumerate}
		\item Recuperiamo $M_i$ e calcoliamo $M_i(x)$.
		\item Facciamo proseguire il calcolo per al massimo $n$
		      passi.
		\item Se la computazione termina in meno di $n$ passi
		      e l'ouput è proprio $y$, allora l'insieme è
		      ricorsivo.
	\end{enumerate}
\end{example}

\begin{example}
	Dato $z \in \N$, l'insieme
	\[
		J = \{ (i,x,k,z) \mid \exists n . (n < k)
		\land (M_i \text{ calcola } z = \varphi_i (x)
		\text{ in meno di } n \text{ passi}) \}
	\]
	è ricorsivo, infatti la procedura calcolabile
	totale che decide $I$ è la seguente:
	\begin{enumerate}
		\item Muoviamo il cursore della MdT $M_i$ su $x < k$
		      per al massimo $n-1$ passi.
	\end{enumerate}
\end{example}
