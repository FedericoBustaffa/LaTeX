\section{Insiemi ricorsivi e ricorsivamente enumerabili}
Un esempio molto semplice di correlazione tra i due metodi è
quello tra una funzione e il suo dominio.
\[
	\lambda x . 2 x \leftrightarrow \{ \N \} \qquad
	\lambda x . x / 2 \leftrightarrow \{ 2n \mid n \in \N \}
\]
Oppure tra una funzione e la sua immagine
\[
	\lambda x . 2 x \leftrightarrow \{ 2 n \mid n \in \N \}
	\qquad \lambda x . x / 2 \leftrightarrow \{ \N \}
\]
Si noti che ci sono infinite numerabili funzioni che hanno i
naturali come dominio, ossia quelle totali, e altrettante che
hanno i naturali come immagine. Si noti come la funzione che
calcola il doppio è totale e che la sua immagine sono i numeri
pari, quindi non tutti i naturali. In ogni caso per questi
esempi sia i domini che le immagini sono
\hyperref[def: relazione ricorsiva]{insiemi ricorsivi}, i quali
possono essere chiamati anche \textbf{decidibili}.

\begin{example}
	Dato $k \in \N$, l'insieme
	\[
		I = \{ (i, x, k) \mid \exists y, n . (x, n < k)
		\land (M_i \text{ calcola } y = \varphi_i (x)
		\text{ in meno di } n \text{ passi}) \}
	\]
	rappresenta l'insieme delle triple tali che, dato $k$,
	esistono $y$ ed $n$ tali che
	\begin{itemize}
		\item $x$ e $n$ sono minori di $k$. Stiamo quindi
		      limitando i possibili valori $x$ in input e stiamo
		      limitando il numero massimo di passi che la
		      macchina $M_i$ può svolgere per terminare la
		      computazione.
		\item $M_i$ calcola $y = \varphi_i (x)$ in meno di $n$
		      passi.
	\end{itemize}
	Per determinare se tale insieme è ricorsivo dobbiamo
	determinare se la sua funzione caratteristica lo è. A tal
	proposito procediamo in questo modo
	\begin{enumerate}
		\item Recuperiamo $M_i$ e calcoliamo $M_i(x)$.
		\item Facciamo proseguire il calcolo per al massimo $n$
		      passi.
		\item Se la computazione termina in meno di $n$ passi
		      allora $(i, x, k) \in I$ e quindi
		      \[ \chi_I (i, x, k) = 1 \]
		      se invece la computazione non termina la funzione
		      caratteristica ritorna $0$.
	\end{enumerate}
	Questa, come è facile notare, è una procedura finita di al
	massimo $n$ passi, alla fine dei quali possiamo andare a
	controllare se tutti i vincoli sono rispettati di modo da
	stabilire se la tripla appartiene all'insieme o meno. Ne
	segue che la funzione caratteristica è ricorsiva e dunque
	anche l'insieme $I$ lo è.
\end{example}

Prendiamo ora un altro esempio, molto simile, leggermente meno
restrittivo, ma che ci permette di non effettuare il controllo
sull'output.

\begin{example}
	Dati $k, z \in \N$, l'insieme
	\[
		J = \{ (i,x,k,z) \mid \exists n . (n < k)
		\land (M_i \text{ calcola } z = \varphi_i (x)
		\text{ in meno di } n \text{ passi}) \}
	\]
	è ricorsivo per lo stesso motivo di prima. Stavolta però
	l'insieme è definito come l'insieme delle quadruple in cui
	abbiamo aggiunto $z$ che è il risultato della funzione. In
	questo caso siamo solo interessati a vedere se $M_i$ termina
	in meno di $n$ passi.
\end{example}

Notiamo che la proposizione $\varphi_i (x) \downarrow$ è vera se
e solo se esiste un $n$ (senza limitazioni) tale che
$M_i(x)$ converge in meno di $n$ passi. Volendo è possibile
vedere questo insieme come l'unione infita degli insiemi
definiti come nell'ultimo esempio con $k$ che assume tutti i
possibili valori in $\N$. Otteniamo così l'insieme di tutte le
macchine che convergono su $x$.

Come sappiamo, il dominio di una funzione è l'insieme dei punti
sui quali la funzione converge. Quello che possiamo fare adesso
è mettere in relazione una funzione con il suo dominio.

\begin{definition} \label{def: rec_enum}
	Diciamo che l'insieme $I$ è \textbf{ricorsivamente enumerabile}
	se e solo se esiste $i$ tale che
	\[ I = \text{dom}(\varphi_i) \]
	dove $\varphi_i$ è una funzione calcolabile.
\end{definition}

Un insieme ricorsivamente enumerabile (detto anche
\textbf{semi-decidibile}) è quindi il dominio di una funzione
calcolabile (il più delle volte parziale, infatti se fosse
totale $I = \N$) che è anche detta \textbf{semi-caratteristica}.
di $I$. Similmente a quest'ultima classe di insiemi, anche gli
insiemi ricorsivi vengono spesso chiamati \textbf{decidibili}.


\begin{property} \label{prop: r_re}
	Le nozioni di insieme ricorsivo e ricorsivamente enumerabile
	sono leggermente differenti. Andiamo quindi ad elencare
	alcune delle proprietà principali:
	\begin{itemize}
		\item Se l'insieme $I$ è ricorsivo, allora è anche
		      ricorsivamente enumerabile.
		      \begin{proof}
			      Questo è banale poiché $\varphi_i$ restituisce
			      $1$ su $x$ se
			      \[ \chi_I (x) = 1 \]
			      altrimenti diverge.
		      \end{proof}
		\item L'insieme $I$ e il suo complemento $\overline{I}$
		      sono ricorsivamente enumerabili se e solo se sono
		      ricorsivi.
		      \begin{proof}
			      Prendiamo $\varphi_i$ con dominio $I$ e
			      $\varphi_{\overline{i}}$ con dominio
			      $\overline{I}$. A questo punto eseguiamo il
			      seguente ciclo.
			      \begin{enumerate}
				      \item Si esegue un passo nel calcolo di
				            $\varphi_i(x)$.
				      \item Se $\varphi_i(x) \downarrow$ allora
				            $x \in I$ e si pone $\chi_I(x) = 1$.
				      \item Altrimenti si esegue un passo di
				            $\varphi_{\overline{i}}(x)$
				      \item Se $\varphi_{\overline{i}}(x) \downarrow$
				            allora $x \notin I$ e si pone
				            $\chi_I(x) = 0$.
			      \end{enumerate}
		      \end{proof}
	\end{itemize}
\end{property}


\begin{theorem}
	L'insieme $I$ è ricorsivamente enumerabile se e solo se è
	vuoto oppure se è l'immagine di una funzione calcolabile
	totale.
	\begin{proof}
		Nel caso in cui
		\[ I = \text{dom}(\varphi_i) \neq \emptyset \]
		è necessario costruire una funzione calcolabile totale
		$f$ tale che
		\[ I = \text{imm}(f) \]
		a partire da $\varphi_i$. Dobbiamo quindi trovare un
		elemento $\overline{n} \in \text{dom}(\varphi_i)$
		tramite i seguenti passi:
		\begin{enumerate}
			\item Si cerca un elemento di
			      $I = \text{dom}(\varphi_i)$ mediante un
			      procedimento a coda di colomba in cui
			      \begin{itemize}
				      \item L'indice di riga $m$ rappresenta il
				            numero dei passi del calcolo di
				            $\varphi_i$.
				      \item L'indice di colonna $n$ rappresenta
				            il suo argomento.
			      \end{itemize}
			\item Si eseguono $m$ passi nel calcolo di
			      $\varphi_i(n)$ finché per qualche valore di
			      $m$ ed $n$ il caclolo si arresta. In caso di
			      arresto abbiamo trovato un elemento $n$ nel
			      dominio di $\varphi_i$, che chiamiamo
			      $\overline{n}$.
		\end{enumerate}
		A questo punto, rappresentando con $\langle m,n \rangle$
		la codifica della coppia $(m, n)$, si inizia un secondo
		procedimento in cui
		\begin{enumerate}
			\item Si esegue $\varphi_i (n)$ per $m$ passi.
			\item Se converge poniamo
			      \[ f(\langle m,n \rangle) = n \]
			      altrimenti si pone
			      \[ f(\langle m,n \rangle) = \overline{n} \]
			\item Si itera incrementando la codifica
			      $\langle m,n \rangle$, considerando quindi
			      $\langle m,n \rangle + 1$
		\end{enumerate}
		In questo modo vengono generati tutti gli elementi di
		$I$.
	\end{proof}
\end{theorem}

Questa dimostrazione si basa sul fatto che se $I$ è un insieme
ricorsivamente enumerabile, ed è quindi il dominio di una certa
funzione $\varphi_i$, è sempre possibile, tramite il procedimento
illustrato, costruire una funzione $f$ calcolabile totale che ha
$I$ come immagine.

\begin{theorem} \label{th: K_re_not_r}
	L'insieme
	\[ K = \{ x \mid \varphi_x(x) \downarrow \} \]
	degli elementi $x$ tali che $\varphi_x (x)$ converge, è
	ricorsivamente enumerabile ma non è ricorsivo.
	\begin{proof}
		Iniziamo con il dimostrare che l'insieme $K$ è
		ricorsivamente enumerabile. Possiamo dire che $K$ è il
		dominio della funzione
		\[
			\psi(x) = \begin{cases}
				x                 & \text{se } \varphi_x(x) \downarrow \\
				\text{indefinita} & \text{altrimenti}
			\end{cases}
		\]
		che è calcolabile poiché possiamo prendere $M_x$ e
		dargli $x$. Se e quando $M_x$ si arresta si restituisce
		$x$.
	\end{proof}
	\begin{proof}
		Per dimostrare che $K$ non è ricorsivo prendiamo la sua
		la funzione caratteristica $\chi_K$ di $K$ e per
		supponiamo per assurdo che sia calcolabile totale. Se
		così fosse sarebbe possibile costruire la seguente
		funzione
		\[
			f(x) = \begin{cases}
				\varphi_x (x) + 1 & \text{se } x \in K    \\
				0                 & \text{se } x \notin K
			\end{cases}
		\]
		anch'essa calcolabile totale in quanto, nel caso in cui
		$x \in K$, ovvero $\chi_K(x) = 1$, avremmo che la
		funzione è definita come
		\[ \varphi_x (x) + 1 \]
		Se invece $x \notin K$, la funzione è definita come $0$.
		In questo modo otteniamo una contraddizione perché
		$\forall x$ vale che
		\[ f(x) \neq \varphi_x (x) \]
		e dunque non troviamo alcun indice per $f$ che quindi
		non è calcolabile.
	\end{proof}
\end{theorem}

Il teorema appena enunciato ci dice che non esiste un algoritmo
per decidere se $x \in K$ o no. Questo problema viene quindi
definito \textbf{insolubile} anche se dovremmo dire che è
\textbf{semi-decidibile}.

Se chiamiamo $\mathcal{R}$ la classe degli insiemi ricorsivi e
$\mathcal{RE}$ la classe degli insiemi ricorsivamente
enumerabili, possiamo dire che vale la seguente relazione
gerarchica
\[ \mathcal{R} \subset	\mathcal{RE} \]
Altra considerazione da fare è che $\overline{K}$ non è
ricorsivamente enumerabile poiché se lo fosse allora sia $K$
che $\overline{K}$ sarebbe ricorsivi (\ref{prop: r_re}).
Possiamo quindi dire che
\[ \mathcal{R} \subset \mathcal{RE} \subset non\mathcal{RE} \]
dove $non \mathcal{RE}$ è la classe degli insiemi non
ricorsivamente enumerabili.

In realtà, per essere più precisi dovremmo dire che l'insieme
$K \in co-\mathcal{RE}$, ossia l'insieme dei problemi il cui
complemento è ricorsivamente enumerabile ma non ricorsivo.

Il problema dell'insieme $K$ potrebbe sembrare un po' artificiale
in quanto si applica un programma a se stesso. Ma se ci pensiamo,
questo è proprio quello che avviene con i \emph{compilatori}.

Supponiamo di aver scritto un \emph{cross-compiler} in un certo
linguaggio $L$ che traduce programmi scritti in $L$ in un altro
linguaggio $A$. Rappresentiamo tale compilatore come
\[ C_L^{L \rightarrow A} \]
Supponiamo ora di voler scrivere il compilatore
\[ C_A^{L \rightarrow A} \]
scritto in $A$ e che fa la stessa cosa del compilatore
precedente. Per ottenere quest'ultimo compilatore possiamo
compilare il primo compilatore con se stesso
\[
	C_L^{L \rightarrow A} (C_L^{L \rightarrow A}) =
	C_A^{L \rightarrow A}
\]
poiché il primo compilatore prende programmi scritti in $L$ e li
traduce in programmi scritti in $A$. Questa procedura si chiama
\textbf{Bootstrapping} e dovrebbe darci un esempio più concreto
di cosa sia l'insieme $K$.

Vogliamo a questo punto scrivere un programma $P$ che, dato un
altro programma $Q$ (individuato dal suo indice $y$) e un
argomento $x$ ci assicura che la computazione di $Q$ su $x$
termina.

Questo problema prende il nome di \textbf{problema della fermata}
ed è formulato come segue: dati $x, y$, possiamo dire che
$\varphi_y (x)$ converge? In altre parole ci stiamo chiedendo se
il programma $P_y (x)$ termina.

Il problema della fermata si studia introducendo un altro
problema, ossia
\[ K_0 = \{ (x, y) \mid \varphi_y (x) \downarrow \} \]
ovvero
\[ K_0 = \{ (x, y) \mid \exists z \mid T(y,x,z) \} \]
dove $T$ è il predicato di Kleene introdotto nel teorema
\ref{th: fn} che ritorna vero se e solo se $z$ è la codifica di
una computazione terminante di $M_y$ con $x$ in input.

\begin{corollary}
	L'insieme $K_0$ non è ricorsivo.
	\begin{proof}
		Si ha che $x \in K$ se e solamente se $(x, x) \in K_0$,
		quindi se $K_0$ fosse ricorsivo lo sarebbe anche $K$.
	\end{proof}
\end{corollary}

La dimostrazione in realtà si basa sulla nozione di
\textbf{riducibilità} che introdurremo fra poco. Non appena
definiremo meglio tale concetto sarà più chiaro anche come
$K$ e $K_0$ sono collegati.

Come abbiamo visto, il problema della fermata, formalizzato con
$K_0$ è strettamente collegato a al problema di decidere
l'appartenenza di un elemento a $K$, tanto da risultarne
\textbf{equivalente}.