\section{Insiemi ricorsivi e ricorsivamente enumerabili}
Un esempio molto semplice di correlazione tra i due metodi è
quello tra una funzione e il suo dominio.
\[
	\lambda x . 2 x \leftrightarrow \{ \N \} \qquad
	\lambda x . x / 2 \leftrightarrow \{ 2n \mid n \in \N \}
\]
Oppure tra una funzione e la sua immagine
\[
	\lambda x . 2 x \leftrightarrow \{ 2 n \mid n \in \N \}
	\qquad \lambda x . x / 2 \leftrightarrow \{ \N \}
\]
Si noti che ci sono infinite numerabili funzioni che hanno i
naturali come dominio, ossia quelle totali, e altrettante che
hanno i naturali come immagine. Si noti come la funzione che
calcola il doppio è totale e che la sua immagine sono i numeri
pari, quindi non tutti i naturali. In ogni caso per questi
esempi sia i domini che le immagini sono
\hyperref[def: relazione ricorsiva]{insiemi ricorsivi}.

Facciamo ora un esempio di insiemi che si possono definire
quando limitiamo il numero di passi nel calcolo di una MdT.

\begin{example}
	Dato $k \in \N$, l'insieme
	\[
		I = \{ (i, x, k) \mid \exists y, n . (x, n < k)
		\land (M_i \text{ calcola } y = \varphi_i (x)
		\text{ in meno di } n \text{ passi}) \}
	\]
	rappresenta l'insieme delle triple tali che, dato $k$,
	esistono $y$ ed $n$ tali che
	\begin{itemize}
		\item $x$ e $n$ sono minori di $k$. Stiamo quindi
		      limitando i possibili valori $x$ in input e stiamo
		      limitando il numero massimo di passi che la
		      macchina $M_i$ può svolgere per terminare la
		      computazione.
		\item $M_i$ calcola $y = \varphi_i (x)$ in meno di $n$
		      passi.
	\end{itemize}
	Per determinare se tale insieme è ricorsivo dobbiamo
	determinare se la sua funzione caratteristica lo è. A tal
	proposito procediamo in questo modo
	\begin{enumerate}
		\item Recuperiamo $M_i$ e calcoliamo $M_i(x)$.
		\item Facciamo proseguire il calcolo per al massimo $n$
		      passi.
		\item Se la computazione termina in meno di $n$ passi
		      e l'ouput è proprio $y$, allora l'insieme è
		      ricorsivo.
	\end{enumerate}
	Questa, come è facile notare, è una procedura finita di al
	massimo $n$ passi, alla fine dei quali possiamo andare a
	controllare se tutti i vincoli sono rispettati di modo da
	stabilire se la tripla appartiene all'insieme o meno.
\end{example}

Prendiamo ora un altro esempio, molto simile, leggermente meno
restrittivo, ma che ci permette di non effettuare il controllo
sull'output.

\begin{example}
	Dati $k, z \in \N$, l'insieme
	\[
		J = \{ (i,x,k,z) \mid \exists n . (n < k)
		\land (M_i \text{ calcola } z = \varphi_i (x)
		\text{ in meno di } n \text{ passi}) \}
	\]
	è ricorsivo per lo stesso motivo di prima. Stavolta però
	l'insieme è definito come l'insieme delle quadruple in cui
	abbiamo aggiunto alla tupla $z$ che è il risultato della
	funzione. In questo caso siamo solo interessati a vedere se
	$M_i$ termina in meno di $n$ passi.
\end{example}

Notiamo che la proposizione $\varphi_i (x) \downarrow$ è vera se
e solo se esiste un $n$ (senza limitazioni) tale che
$M_i(x) \downarrow$ converge in meno di $n$ passi. Volendo è
possibile vedere questo insieme come l'unione infita degli
insiemi definiti come nell'ultimo esempio con $k$ che assume
tutti i possibili valori in $\N$. Otteniamo così l'insieme di
tutte le macchine che convergono su $x$.

Come sappiamo, il dominio di una funzione è l'insieme dei punti
sui quali la funzione converge. Quello che possiamo fare adesso
è mettere in relazione una funzione con il suo dominio.

\begin{definition} \label{def: rec_enum}
	Diciamo che l'insieme $J$ è \textbf{ricorsivamente enumerabile}
	se e solo se esiste $i$ tale che
	\[ J = \text{dom}(\varphi_i) \]
\end{definition}

Le nozioni di insieme ricorsivo e ricorsivamente enumerabile
sono leggermente differenti. Di seguito andiamo ad elencare
alcune delle proprietà principali:
\begin{itemize}
	\item Se l'insieme $I$ è ricorsivo, allora è anche
	      ricorsivamente enumerabile.
	      \begin{proof}
		      Questo è banale poiché $\varphi_i$ restituisce
		      $1$ su $x$ se
		      \[ \chi_I (x) = 1 \]
		      altrimenti diverge.
	      \end{proof}
	\item L'insieme $I$ e il suo complemento $\overline{I}$
	      sono ricorisvamente enumerabili se e solo se sono
	      ricorsivi.
	      \begin{proof}
		      Prendiamo $\varphi_i$ con dominio $I$ e
		      $\varphi_{\overline{i}}$ con dominio
		      $\overline{I}$. A questo punto eseguiamo il
		      seguente ciclo.
		      \begin{enumerate}
			      \item Si esegue un passo nel calcolo di
			            $\varphi_i(x)$.
			      \item Se $\varphi_i(x) \downarrow$ allora
			            $x \in I$ e si pone $\chi_I(x) = 1$.
			      \item Altrimenti si esegue un passo di
			            $\varphi_{\overline{i}}(x)$
			      \item Se $\varphi_{\overline{i}}(x) \downarrow$
			            allora $x \notin I$ e si pone
			            $\chi_I(x) = 0$.
		      \end{enumerate}
	      \end{proof}
\end{itemize}

\begin{theorem}
	L'insieme $I$ è ricorsivamente enumerabile se e solo se è
	vuoto oppure se è l'immagine di una funzione calcolabile
	totale.
	\begin{proof}
		Nel caso in cui
		\[ I = \text{dom}(\varphi_i) \neq \emptyset \]
		è necessario costruire una funzione calcolabile totale
		$f$ tale che
		\[ I = \text{imm}(f) \]
		a partire da $\varphi_i$. Seguiamo quindi i seguenti
		passi
		\begin{enumerate}
			\item Si cerca un elemento di $I$ mediante un
			      procedimento a coda di colomba in cui
			      \begin{itemize}
				      \item L'indice di riga $m$ rappresenta il
				            numero dei passi del calcolo di
				            $\varphi_i$.
				      \item L'indice di colonna $n$ rappresenta
				            il suo argomento.
			      \end{itemize}
			\item Si eseguono $m$ passi nel calcolo di
			      $\varphi_i(n)$ finché per qualche valore di
			      $m$ e dell'argomento, sia $\overline{n}$ il
			      calcolo si arresta.
		\end{enumerate}
	\end{proof}
\end{theorem}
