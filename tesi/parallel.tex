\section{Calcolo parallelo}

In questa parte cerchiamo di dare forma ad un sistema in grado di prendere
in input un qualsiasi algoritmo genetico (nel nostro caso uno in grado di
risolvere il problema del commesso viaggiatore) e, sfruttando il parallelismo,
ottenere uno \emph{speed-up} significativo.

\subsection{Generazione della popolazione iniziale}

In questa prima fase andremo a generare una popolazione iniziale in modo del
tutto casuale. In questa fase si generano gli individui ed è consigliabile
inserire un controllo per evitare la generazione di duplicati. Questo perché
vogliamo ottenere la massiamo \emph{biodiversità} possibile.

Nel problema del commesso viaggiatore, così come lo abbiamo formulato, ogni
individuo è un vettore di interi, tutti diversi, i cui valori vanno da $0$ a
$T-1$, dove $T$ è il numero di città che stiamo considerando. Se consideriamo
$T = 10$, un possibile individuo della popolazione è il seguente
\begin{center}
	\includesvg[inkscapelatex=false, scale=0.4]{images/chromosome_example.svg}
\end{center}
Per generare individui nel caso del problema del commesso viaggiatore possiamo
\begin{enumerate}
	\item Prendere i valori da $0$ a $T-1$.
	\item Inserirli in un array.
	\item Mescolare l'array.
	\item In caso si generi un cromosoma già presente si ripete il processo.
\end{enumerate}
Questo processo è in genere abbastanza rapido e di conseguenza non richiede
una parallelizzazione. Potrebbe essere comunque interessante capire se,
lavorando in scenari in cui si necessita una popolazione iniziale molto ampia,
possa aver senso ricorrere al parallelismo.
