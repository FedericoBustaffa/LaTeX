\documentclass[12pt]{article}

% % --------------- PACKAGES ---------------
\usepackage{inputenc}
\usepackage[T1]{fontenc}
\usepackage[italian]{babel}
\usepackage[hidelinks]{hyperref}

\hypersetup{
	colorlinks=true,
	linkcolor=blue!70!black
}

% --------------- STYLE ---------------
\usepackage[margin=1.25in]{geometry}
\usepackage[most]{tcolorbox}

% Font
\usepackage{sansmath}

\renewcommand{\familydefault}{\sfdefault}
\sansmath

% page style
\usepackage{fancyhdr}
\usepackage[Sonny]{fncychap}

\pagestyle{fancy}
\setlength{\headheight}{15pt}
\rhead{\thepage}
\cfoot{\thepage}

% --------------- MATH ---------------
\usepackage{amsmath}
\usepackage{amssymb}
\usepackage{amsthm}
\usepackage{amsfonts}
\usepackage{mathtools}
\usepackage{mdframed}

\newcommand{\N}{\mathbb{N}}
\newcommand{\Z}{\mathbb{Z}}
\newcommand{\R}{\mathbb{R}}
\newcommand{\C}{\mathbb{C}}
\newcommand{\E}{\mathbb{E}}

\newcommand{\F}{\mathcal{F}}

\DeclareMathOperator{\Var}{Var}
\DeclareMathOperator{\Cov}{Cov}

% Boxes for theorem, definitions and examples
\newtheoremstyle{math_box}
{0pt}
{0pt}
{\normalfont}
{}
{\color{orange}}
{\;}
{0.25em}
{\thmname{\textbf{#1}}\thmnumber{ \textbf{#2}}{\color{black}\thmnote{\textbf{ -- #3.}}}}

\newmdenv[
	rightline=false,
	leftline=true,
	topline=false,
	bottomline=false,
	linecolor=orange!40,
	innerleftmargin=5pt,
	innerrightmargin=5pt,
	innertopmargin=0pt,
	innerbottommargin=0pt,
	leftmargin=0cm,
	rightmargin=0cm,
	linewidth=3pt
]{dBox}

\newmdenv[
	rightline=false,
	leftline=false,
	topline=false,
	bottomline=false,
	backgroundcolor=orange!15,
	innerleftmargin=5pt,
	innerrightmargin=5pt,
	innertopmargin=5pt,
	innerbottommargin=5pt,
	leftmargin=0cm,
	rightmargin=0cm,
]{pBox}

\theoremstyle{math_box}
\newtheorem{theoremeT}{Teorema}[chapter]
\newtheorem{definitionT}{Definizione}[chapter]
\newtheorem{propositionT}{Proposizione}[chapter]
\newtheorem{corollary}{Corollario}[chapter]
\newtheorem{lemma}{Lemma}[chapter]
\newtheorem{observation}{Osservazione}[chapter]
\newtheorem{exampleT}{Esempio}[section]

\newenvironment{theorem}{\begin{pBox}\begin{theoremeT}}{\end{theoremeT}\end{pBox}}
\newenvironment{definition}{\begin{dBox}\begin{definitionT}}{\end{definitionT}\end{dBox}}
\newenvironment{proposition}{\begin{pBox}\begin{propositionT}}{\end{propositionT}\end{pBox}}
\newenvironment{example}{\begin{dBox}\begin{exampleT}}{\end{exampleT}\end{dBox}}

\usepackage{tikz, pgfplots, pgf-pie}
\usepackage{caption, subcaption}
\usepackage{scalerel}
\usepackage{pict2e}
\usepackage{tkz-euclide}
\usepackage{pgfplots, pgfplotstable, pgf-pie}
\usepackage{circuitikz}

\usetikzlibrary{calc}
\usetikzlibrary{patterns, arrows}
\usetikzlibrary{shadows}
\usetikzlibrary{external}

\pgfplotsset{compat=newest}
\usepgfplotslibrary{statistics, fillbetween}

\tikzstyle{branch}=[
fill,
shape=circle,
minimum size=3pt,
inner sep=0pt
]


\usepackage[T1]{fontenc}
\usepackage[italian]{babel}
\usepackage[hidelinks]{hyperref}
\usepackage[margin=1in]{geometry}
\usepackage{minted}
\usepackage{diagbox}
\usepackage{svg}
\usepackage{wrapfig}

\definecolor{minted_bg}{rgb}{0.9, 0.9, 0.9}
\usemintedstyle{colorful}

\setminted[py]{
	tabsize=4,
	linenos=true,
	bgcolor=minted_bg,
	fontsize=\small,
	mathescape=true
}

\setminted[cpp]{
	tabsize=4,
	linenos=true,
	bgcolor=minted_bg,
	fontsize=\small,
	mathescape=true
}

\setminted[bash]{
	tabsize=4,
	% linenos=true,
	bgcolor=minted_bg,
	fontsize=\small,
	mathescape=true
}

\title{Parallelizzazione di algoritmi genetici}
\author{Federico Bustaffa}
\date{01/09/2024}

\begin{document}

\maketitle

\begin{abstract}
	L'obbiettivo di questo progetto è quello di andare a studiare varie alternative
	per rendere più efficienti algoritmi genetici, soprattutto andando a lavorare
	in ambito parallelo per migliorare le fasi che costituiscono un collo di
	bottiglia per le prestazioni. In generale un algoritmo genetico ha sei
	componenti fondamentali:
	\begin{enumerate}
		\item \textbf{Generazione}: si genera in modo casuale la popolazione
		      iniziale.
		\item \textbf{Selezione}: si selezionano gli individui per l'accoppiamento
		      e la generazione di nuovi individui.
		\item \textbf{Crossover}: gli individui selezionati vengono fatti
		      accoppiare e se ne generano di nuovi.
		\item \textbf{Mutazione}: ogni nuovo individuo ha un certa probabilità di
		      subire una mutazione.
		\item \textbf{Valutazione}: si valuta il valore di fitness degli individui.
		\item \textbf{Rimpiazzo}: per mantenere omogeneo il numero di individui
		      nella popolazione si adottano politiche di rimpiazzo per scartare
		      alcuni individui.
	\end{enumerate}
	Non entriamo nel merito di quali siano possibili tecniche per implementare un
	algoritmo genetico. Quello che ci interessa è individuare la struttura di base.

	Le fasi che più ci preme ottimizzare andando a lavorare in parallelo sono
	quelle di crossover, mutazione e valutazione.

	La mutazione non è un passo computazionalmente dispendioso ma si presta bene ad
	essere parallelizzato. Le altre due fasi invece potrebbero richiedere molto
	tempo e costituire un grosso limite per le performance.
\end{abstract}

\tableofcontents

\section{Calcolo parallelo}

In questa parte cerchiamo di dare forma ad un sistema in grado di prendere
in input un qualsiasi algoritmo genetico (nel nostro caso uno in grado di
risolvere il problema del commesso viaggiatore) e, sfruttando il parallelismo,
ottenere uno \emph{speed-up} significativo.

\subsection{Generazione della popolazione iniziale}

In questa prima fase andremo a \textbf{generare} una popolazione iniziale in
modo del tutto casuale. Evitare di generare duplicati è buona norma, almeno
in questa fase, così da garantire un alto grado di \emph{biodiversità} iniziale.

Nel problema del commesso viaggiatore, così come lo abbiamo formulato, ogni
individuo è un vettore di interi, tutti diversi, i cui valori vanno da $0$ a
$T-1$, dove $T$ è il numero di città che stiamo considerando. Se consideriamo
$T = 10$, un possibile individuo della popolazione è il seguente
\begin{center}
	\includesvg[inkscapelatex=false, scale=0.4]{images/chromosome_example.svg}
\end{center}
Per generare individui nel caso del problema del commesso viaggiatore possiamo
\begin{enumerate}
	\item Prendere i valori da $0$ a $T-1$.
	\item Inserirli in un array.
	\item Mescolare l'array.
	\item In caso si generi un cromosoma già presente si ripete il processo.
\end{enumerate}
Questo processo è in genere abbastanza rapido e di conseguenza non richiede
una parallelizzazione. Potrebbe essere comunque interessante capire se,
lavorando in scenari in cui si necessita una popolazione iniziale molto ampia,
possa aver senso ricorrere al parallelismo.

\subsection{Valutazione}

Per la valutazione

% \subsection{Crossover}

% La prossima fase riguarda la fase di \textbf{crossover}, in cui andiamo a
% combinare tra di loro i migliori individui per generarne altri che,
% presumibilmente, saranno migliori.

\end{document}
