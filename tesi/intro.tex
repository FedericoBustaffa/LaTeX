\section{Algoritmi genetici}

L'obbiettivo di questo progetto è quello di andare a studiare varie alternative
per rendere più efficienti algoritmi genetici, soprattutto andando a lavorare
in ambito parallelo per migliorare le fasi che costituiscono un collo di
bottiglia per le prestazioni. In generale un algoritmo genetico ha sei
componenti fondamentali:
\begin{enumerate}
	\item \textbf{Generazione}: si genera in modo casuale la popolazione
	      iniziale.
	\item \textbf{Selezione}: si selezionano gli individui per l'accoppiamento
	      e la generazione di nuovi individui.
	\item \textbf{Crossover}: gli individui selezionati vengono fatti
	      accoppiare e se ne generano di nuovi.
	\item \textbf{Mutazione}: ogni nuovo individuo ha un certa probabilità di
	      subire una mutazione.
	\item \textbf{Valutazione}: si valuta il valore di fitness degli individui.
	\item \textbf{Rimpiazzo}: per mantenere omogeneo il numero di individui
	      nella popolazione si adottano politiche di rimpiazzo per scartare
	      alcuni individui.
\end{enumerate}
Non entriamo nel merito di quali siano possibili tecniche per implementare un
algoritmo genetico. Quello che ci interessa è individuare la struttura di base.

Le fasi che più ci preme ottimizzare andando a lavorare in parallelo sono
quelle di crossover, mutazione e valutazione.

La mutazione non è un passo computazionalmente dispendioso ma si presta bene ad
essere parallelizzato. Le altre due fasi invece potrebbero richiedere molto
tempo e costituire un grosso limite per le performance.