\chapter*{Introduzione}

Il progetto di tesi verte sull'implementazione di una libreria di algoritmi
genetici, in grado di sfruttare architetture multi-core per il calcolo
parallelo. La necessità di un algoritmo genetico con questa struttura nasce in
un contesto di explainable AI, in particolare si fa riferimento al progetto
\textit{LORE}~\cite{guidotti2018LORE}, il quale utilizza un'implementazione
fornita dalla libreria di algoritmi genetici \textit{DEAP}~\cite{fortin2012DEAP}.

\textit{LORE} si propone di generare spiegazioni a decisioni o predizioni fatte
da modelli di machine learning, spesso difficili da interpretare per via della
loro elevata complessità, come ad esempio reti neurali profonde o foreste
casuali. In una prima fase, del metodo si sfrutta un algoritmo genetico per la
generazione di dati sintetici con determinate caratteristiche, fondamentali in
seguito per la costruzione delle spiegazioni.

Tale approccio risulta però essere particolarmente dispendioso, soprattutto in
casi in cui la quantità di dati sintetici che si desidera generare è molto
grande oppure quando il modello è particolarmente lento in fase di predizione
(o entrambe).

Da qui la necessità di implementare una libreria di algoritmi genetici in grado
di produrre risultati molto simili a \textit{DEAP} ma più performante e in
grado di sfruttare al meglio architetture multi-core.

Dato che \textit{LORE} e \textit{DEAP}, così come la maggior parte delle
librerie di machine learning, sono implementate (o forniscono un'API) in Python,
la prima fase del lavoro si è focalizzata sull'esplorazione e la ricerca di
possibili framework, in grado di lavorare in sinergia con Python. Le opzioni
esplorate vanno da librerie standard di Python e moduli di terze parti, passando
per l'implementazione di estensioni in linguaggi a basso livello come C/C++,
fino ad arrivare alla sperimentazione di feature sperimentali come
\textit{subinterpreters} e una delle prime versioni \textit{free-threaded}
proposta con Python 3.13.

Una volta implementata la libreria si è passati ad una fase di test per
valutarne la correttezza, impiegando l'algoritmo nella risoluzione di problemi
ben noti in letteratura facilmente rappresentabili tramite grafici. Tra questi
abbiamo il problema dello zaino, del commesso viaggiatore e un semplice caso di
regressione lineare. Si è infine testata la correttezza sul caso d'uso di
\textit{LORE} precedentemente descritto.

Il lavoro si è concluso con due fasi di test, la prima riguardante la qualità
delle soluzioni prodotte e la seconda riguardante le performance. Come
anticipato, l'obiettivo della tesi è stato quello di produrre qualcosa che
funzionasse qualitativamente come \textit{DEAP} ma che fosse più performante.

La libreria implementata sembra soddisfare le aspettative fornendo risultati
qualitativi in linea con \textit{DEAP} ma dimostrando prestazioni e capacità
di sfruttare architetture multi-core generalmente migliori.
