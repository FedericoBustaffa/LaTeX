\chapter{Sicurezza DNS}
\section{DNS}
Il \textbf{DNS} o \textbf{Domain Name System} è un database gerarchico che associa nomi degli host ad indirizzi IP.

Esso permette ad un utente di trovare un sistema senza conoscere il suo indirizzo IP. Per riuscire a fornire questo
tipo di servizio organizza la rete in \textbf{domini} organizzati logicamente come un albero invertito.

La \textbf{risoluzione} dei nomi è affidata ad un sistema distribuito. I membri del sistema sono detti
\textbf{Name Server} ed ogni applicazione deve contattare un Name Server per risolvere un nome.

Ogni autorità deve avere almeno un Name Server che gestisce la risoluzione dei nomi per quel dominio.

\subsection{Name Server e Resolver}
Un \textbf{Name Server} mantiene un database delle informazioni sugli host per il suo dominio e contatta il Name
Server di altri domini quando deve reperire informazioni (come l'indirizzo IP) sull'host a cui ci si vuole connettere.

Le informazioni devono essere aggiornate quando quelle dell'host cambiano. Questi aggiornamenti dinamici cambiano i
dati del DNS senza la necessità di ricostruire ogni altra parte dell'albero DNS.

La comunicazione con il Name Server è gestita dal \textbf{resolver}, il quale permette all'utente di fornire nomi
parziali, i quali vengono estesi per provare ad ottenere nomi completamente qualificati.
