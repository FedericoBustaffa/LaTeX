\chapter{Strategie di progetto}
Vediamo in questo capitolo come sia possibile progettare un sistema, tenendo conto di tutto ciò che abbiamo detto in
precedenza.

\section{Progettare con il privilegio minimo}
Come sappiamo, il principio del privilegio minimo, dice che ogni utente o componente del sistema dovrebbe avere il
minimo numero di diritti per svolgere il proprio compito.

Questa restrizione è molto più efficace se sta alla base della progettazione e non se viene aggiunta in seguito. La
sua successiva aggiunta potrebbe anzi essere motivo di comportamenti indesiderati e anche di vulneabilità.

La concessione di diritti non necessari porta ad una crescente probabilità di errore, bug e costringe i progettisti a
compromessi andando a creare problemi di sicurezza difficili da contenere o minimizzare.

Il progettista deve essere anche in grado du  capire dove e come applicare il principio del privilegio minimo, valutando
vari fattori come rischio, possibili danni ecc.

In generale la figura dell'\textbf{amministratore onnipotente} è sempre motivo di problemi di sicurezza ed è per questo
molto utile applicare il principio di \emph{compromise recording}

\section{API}
Per \textbf{API} si intende qualsiasi operazione in grado di modificare lo stato interno del sistema. Le API di livello
amministrativo sono quelle più importanti e pericolose in quanto, in genere, hanno gli impatti più pesanti sul sistema
e sono quindi il bersaglio preferito degli attaccanti.

Se un utente necessita di privilegi di amministrazione per eseguire una o più operazioni, i controlli di autenticazione
devono essere più pesanti e la API in questione deve essere ristretta il più possibile per riuscire a contenere il
eventuali comportamenti scorretti.

Adottare questo approccio di progettazione potrebbe impedire ad un certo operatore di risolvere un particolare problema,
ecco perché è possibile i cosiddetti \textbf{breakglass mechanism}, i quali permettono di violare il controllo degli
accessi in caso di necessità.

Questo tipo di meccanismo (molto potente) può essere molto utile nella risoluzione di problemi ma per evitare che
qualcuno ne abusi, il loro utilizzo deve essere \emph{regolamentato} e \emph{verificato} (\emph{audit}).

Un modo per rendere questo meccanismo più sicuro consiste nel permettere il suo utilizzo solo tramite determinate
macchine poste in ambienti controllati.

\section{Testing}
Un sistema che cerca di soddisfare il principio del privilegio minimo è più complesso di uno che concede pieni diritti
a tutti gli utenti.

In generale è più difficile anche scoprire se c'è qualche vulnerabilità in un sistema che adotta il privilegio minimo,
dato che questa viene scoperta molte volte solo dopo che un'intrusione ha avuto successo.

Il caso più comune è quello in cui si fornisce ad un utente un diritto che non dovrebbe avere. Fin tanto che l'utente
si comporta nel modo corretto è difficile che l'errore venga scoperto.

In questo caso si hanno due tipi di testing:
\begin{itemize}
	\item \textbf{Test del privilegio minimo}: verifica che ogni utente riesca ad eseguire solo le operazioni per le
	      quali ha diritto e, al contrario, non riesca ad eseguire quelle per le quali non possiede diritti.
	\item \textbf{Test con il privilegio minimo}: verifica che l'ambiente dedicato al test abbia solo i diritti
	      necessari.
\end{itemize}

\section{Autenticazione e autorizzazione}
Come già detto in precedenza i meccanismi di \textbf{autenticazione} e \textbf{autorizzazione} sono fondamentali per
la costruzione di un sistema sicuro.
\begin{itemize}
	\item L'\textbf{autenticazione} consiste nel verificare che l'utente richiedente l'utilizzo di una API, sia davvero
	      chi afferma di essere.
	\item L'\textbf{autorizzazione} consiste nel verificare che l'utente richiedente l'utilizzo di una API, abbia i
	      diritti necessari affinché tale servizio gli venga concesso. Si parla di \textbf{multi party authorization}
	      quando più persone o entità del sistema sono coinvolte nella concessione di un diritto o di un servizio.
\end{itemize}

\subsection{MPA e 3FA}
Si tratta di due protocolli di autorizzazione multi-fattore in cui l'idea di base è quella di verificare
l'autorizzazione per un'operazione, tramite uno o più nodi specializzati.

Questi nodi sono progettati per essere più robusti e hanno il ruolo di confermare le decisioni prese dagli altri nodi.