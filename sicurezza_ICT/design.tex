\chapter{Strategie di progetto}
Vediamo in questo capitolo come sia possibile progettare un sistema, tenendo conto di tutto ciò che abbiamo detto in
precedenza.

\section{Progettare con il privilegio minimo}
Come sappiamo, il principio del privilegio minimo, dice che ogni utente o componente del sistema dovrebbe avere il
minimo numero di diritti per svolgere il proprio compito.

Questa restrizione è molto più efficace se sta alla base della progettazione e non se viene aggiunta in seguito. La
sua successiva aggiunta potrebbe anzi essere motivo di comportamenti indesiderati e anche di vulneabilità.

La concessione di diritti non necessari porta ad una crescente probabilità di errore, bug e costringe i progettisti a
compromessi andando a creare problemi di sicurezza difficili da contenere o minimizzare.

Il progettista deve essere anche in grado du  capire dove e come applicare il principio del privilegio minimo, valutando
vari fattori come rischio, possibili danni ecc.

In generale la figura dell'\textbf{amministratore onnipotente} è sempre motivo di problemi di sicurezza ed è per questo
molto utile applicare il principio di \emph{compromise recording}

\subsection{API}
Per \textbf{API} si intende qualsiasi operazione in grado di modificare lo stato interno del sistema. Le API di livello
amministrativo sono quelle più importanti e pericolose in quanto, in genere, hanno gli impatti più pesanti sul sistema
e sono quindi il bersaglio preferito degli attaccanti.

Se un utente necessita di privilegi di amministrazione per eseguire una o più operazioni, i controlli di autenticazione
devono essere più pesanti e la API in questione deve essere ristretta il più possibile per riuscire a contenere il
eventuali comportamenti scorretti.

Adottare questo approccio di progettazione potrebbe impedire ad un certo operatore di risolvere un particolare problema,
ecco perché è possibile i cosiddetti \textbf{breakglass mechanism}, i quali permettono di violare il controllo degli
accessi in caso di necessità.

Questo tipo di meccanismo (molto potente) può essere molto utile nella risoluzione di problemi ma per evitare che
qualcuno ne abusi, il loro utilizzo deve essere \emph{regolamentato} e \emph{verificato} (\emph{audit}).

Un modo per rendere questo meccanismo più sicuro consiste nel permettere il suo utilizzo solo tramite determinate
macchine poste in ambienti controllati.

\subsection{Testing}
Un sistema che cerca di soddisfare il principio del privilegio minimo è più complesso di uno che concede pieni diritti
a tutti gli utenti.

In generale è più difficile anche scoprire se c'è qualche vulnerabilità in un sistema che adotta il privilegio minimo,
dato che questa viene scoperta molte volte solo dopo che un'intrusione ha avuto successo.

Il caso più comune è quello in cui si fornisce ad un utente un diritto che non dovrebbe avere. Fin tanto che l'utente
si comporta nel modo corretto è difficile che l'errore venga scoperto.

In questo caso si hanno due tipi di testing:
\begin{itemize}
	\item \textbf{Test del privilegio minimo}: verifica che ogni utente riesca ad eseguire solo le operazioni per le
	      quali ha diritto e, al contrario, non riesca ad eseguire quelle per le quali non possiede diritti.
	\item \textbf{Test con il privilegio minimo}: verifica che l'ambiente dedicato al test abbia solo i diritti
	      necessari.
\end{itemize}

\subsection{Autenticazione e autorizzazione}
Come già detto in precedenza i meccanismi di \textbf{autenticazione} e \textbf{autorizzazione} sono fondamentali per
la costruzione di un sistema sicuro.
\begin{itemize}
	\item L'\textbf{autenticazione} consiste nel verificare che l'utente richiedente l'utilizzo di una API, sia davvero
	      chi afferma di essere.
	\item L'\textbf{autorizzazione} consiste nel verificare che l'utente richiedente l'utilizzo di una API, abbia i
	      diritti necessari affinché tale servizio gli venga concesso.
\end{itemize}

\subsubsection{MPA e 3FA}
Si tratta di due protocolli di autorizzazione multi-fattore in cui l'idea di base è quella di verificare
l'autorizzazione per un'operazione, tramite uno o più nodi specializzati.

Questi nodi sono progettati per essere più robusti e hanno il ruolo di confermare le decisioni prese dagli altri nodi.

Si parla di \textbf{MPA} o \textbf{Multi Party Authorization}, quando più persone o entità del sistema sono coinvolte
nella concessione di un diritto o di un servizio.

Si parla di \textbf{3FA} quando, in aggiunta ad una \emph{two factor authentication}, si chiede all'utente, tramite il
suo smartphone, di confermare la richiesta di accesso.

\subsubsection{Accesso temporaneo}
Quando non si possono avere controlli esaurienti sulle operazioni per ragioni di tempo e non è possibile applicare il
principio del privilegio minimo, è possibile concedere il diritto all'operazione per un tempo limitato.

Una volta esaurito il tempo si interrompe il servizio e si forza l'utente a ripetere la richiesta riducendo così gli
impatti di uso scorretto della API.

\subsection{Conclusioni}
Per riuscire a progettare un sistema che soddisfi il principio del privilegio minimo serve:
\begin{itemize}
	\item Una buona conoscenza del sistema. La conoscenza del sistema è massima quando siamo noi stessi a progettarlo,
	      al contrario è molto complicato e in molti casi impossibile applicare il privilegio minimo ad un sistema
	      progettato da altri e già terminato.
	\item Un sistema di autenticazione per validare i tentativi di accesso.
	\item Un sistema di autorizzazione per rinforzare la politica di sicurezza che vogliamo implementare.
	\item Un insieme di controlli avanzati per l'autorizzazione temporanea.
	\item Alcune capacità tra cui:
	      \begin{itemize}
		      \item Capacità di regolamentare gli accessi.
		      \item Capacità di progettare, definire e correggere la politica di sicurezza.
		      \item Un meccanismo di \emph{breakglass} in caso di comportamenti inattesi da parte del sistema.
	      \end{itemize}
\end{itemize}

\section{Progettare per la robustezza}
Un requisito fondamentale per avere un sistema \textbf{robusto} è quello di avere un sistema che sia anche
\textbf{comprensibile}. Tra i motivi principali per volere questo abbiamo:
\begin{itemize}
	\item Nel caso di modifiche al sistema la probabilità di generare nuove vulnerabilità o la probabilità che il
	      meccanismo di \emph{backup} fallisca dopo un intrusione si riducono.
	\item La risposta agli attacchi è molto più semplice e agevole quando il sistema è semplice.
	\item Un sistema semplice permette una più facile individuazione dei casiddetti \textbf{invarianti}, ossia
	      proprietà che il sistema deve possedere sempre.
\end{itemize}

\subsection{Invarianti}
Un'\textbf{invariante} è una proprietà di sistema sempre vera anche se il sistema si comporta in modo inaspettato o
scorretto.

Il sistema assicura che una certa proprietà sia un invariante e riuscire a capire se una proprietà è un invariante
o meno è molto complesso e richiede analisi e conoscenza del sistema.

Quando si vuole definire un invariante si deve tener conto del fatto che, in genere, ci sono diversi modi di soddisfare
un invariante. Principalmente ci sono due modi di soddisfare un invariante:
\begin{itemize}
	\item Si eseguono dei test semplici ma anche poco affidabili per riuscire a scoprire se l'invariante viene violato.
	      Il problema di questi test è che spesso non scoprono alcuna violazione dell'invariante ma vengono eseguiti
	      per contenere i costi.
	\item Si cerca una dimostrazione formale del fatto che l'invariante è sempre soddisfatto. Soluzione decisamente
	      migliore ma anche molto più costosa.
\end{itemize}
La tecnica più usata si basa sulla costruzione di \textbf{modelli mentali} del sistema, ossia modelli parziali con i
quali si prende in esame la parte di sistema di cui si vuole validare l'invariante per poi effettuare dei test su di
essa.

Il problema di questo approccio è che spesso prende in esame solo i casi più frequenti che sono anche quelli dove è
più semplice che tutto vada nel modo migliore, quando invece i problemi di sicurezza si verificano molto spesso nelle
casistiche meno frequenti.

\subsection{Comprensibilità e complessità}
La \textbf{comprensibilità} aumenta se la \textbf{complessità} viene gestita decomponendo un sistema in modo
orizzontale o verticale creando moduli e gerarchie di maccine virtuali.

Un approccio molto utilizzato è quello di avere moduli che iteragiscono e cooperano con altri moduli ma che allo
stesso riescono a scoprire anomalie dei moduli con cui cooperano.

In generale è bene che un modulo sia più \textbf{autonomo} possibile, dove per modulo autonomo si intende un modulo
che non fa assunzioni sul comportamento di altri moduli. Minori sono le assunzioni di un modulo sul comportamento
degli altri, più facile è prevedere il suo comportamento.

Altri modi per ridurre la complessità del sistema sono:
\begin{itemize}
	\item Preferire interfacce \textbf{strette} che riducono quindi i gradi di libertà dell'interpretazione.
	\item Un'interfaccia per gestire risorse di tipo diverso si semplifica nel caso sia possibile definire un modello
	      dei dati \textbf{comune} ai vari moduli.
	\item Riconoscere operazioni \textbf{idempotenti}, le quali possono essere invocate e ripetute senza errori.
\end{itemize}

\subsection{Identità}
Favorire la comprensibilità permette anche di avere sistemi più sicuri per accertare l'\textbf{identità} dei vari
utenti.

Identità ben riconoscibili rendono più semplici le fasi di autenticazione e controllo degli accessi al sistema.
La robustezza del sistema dipende in gran parte dalla robustezza del meccanismo per l'identità. Se è molto facile
spacciarsi per un altro utente allora non ha senso avere un meccanismo di controllo o di autenticazione molto
potente.

Un identificatore robusto in termini di sicurezza informatica deve essere
\begin{itemize}
	\item Comprensibile agli essere umani.
	\item Robusto allo spoofing.
	\item Non riutilizzabile.
\end{itemize}

\subsubsection{Identità in Google}
In Google ci sono quattro cose che hanno un'identità:
\begin{itemize}
	\item Utenti
	\item Amministratori
	\item Macchine, le quali hanno come
	      \begin{itemize}
		      \item Identificatore il nome DNS.
		      \item Identità il ruolo che ha quella macchina.
	      \end{itemize}
	\item \textbf{Workload} suddivisi in
	      \begin{itemize}
		      \item Richiesti dall'utente.
		      \item Assegnati ad una macchina in un certo insieme.
		      \item Gestito da un sistema di orchestrazione simile a \emph{kubernetes} che offre:
		            \begin{itemize}
			            \item Load balancing
			            \item Resource optimization
		            \end{itemize}
	      \end{itemize}
\end{itemize}
Il meccanismo di controllo degli accessi che ne deriva è strutturato come un framework che coordina i vari workload
a partire da una richiesta utente. I dati vengono poi passati tra i vari workload ed infine il framework decide se
i controlli degli accessi sono risolti in base all'identità dell'utente richiedente o del workload.

\section{Trusted Computing Base - TCB}
Il TCB di un sistema è un sottoinsieme dei componenti del sistema che determina la \textbf{politica di sicurezza}.

In un sistema possiamo avere sottoinsiemi diversi, ognuno con un TCB diverso, per riuscire ad avere maggiore
modularità e scalabilità.

Ogni TCB dovrebbe essere \emph{sospettoso} di qualsiasi cosa arrivi dall'esterno della sua area di competenza, sia
che si tratti di qualcosa che arriva dall'esterno del sistema sia che si tratti di qualcosa interno al sistema.
Questo permette di trovare ad esempio guasti di altri moduli interni al sistema.

Decomporre un sistema in parallelo fa sì che si soddisfi quasi automaticamente il principio del privilegio minimo.

\section{Controllo dei tipi}
Definito un tipo di dato conviene definire una funzione associata per il controllo e la validazione del tipo stesso.
Il controllo dei tipi si decompone in:
\begin{itemize}
	\item Verifica della correttezza della funzione associata.
	\item Verifica che ogni modulo che utilizza il tipo invochi la funzione associata.
	\item Altri controlli sul codice dei vari moduli.
\end{itemize}
In molti linguaggi il meccanismo di incapsulamento e definizione di tipi non è sicuro quanto si pensa in quanto,
tramite \emph{reflection}, \emph{casting} e altri meccanismi, è possibile violare i tipi stessi.

\section{Progettare per la resilienza}
Un sistema dovrebbe essere progettato per semplificare future modifiche, dovute alla scoperta di nuove vulnerabilità
o per riuscire a soddisfare nuovi requisiti di sicurezza.

Un altro requisito fondamentale per la sicurezza di un sistema è la \textbf{resilienza} agli attacchi. A tal fine,
tra le pratiche più utili abbiamo:
\begin{itemize}
	\item Differenziazione e partizionamento di parti di sistema per riuscire a favorire l'indipendenza e la
	      sicurezza dei diversi moduli.
	\item Implementazione di funzioni di sicurezza automatiche per ridurre le risposte agli attacchi.
\end{itemize}

\section{Degradazione}
Dato che non è possibile sapere sempre dove o quando si presenterà una vulnerabilità, quello che si può fare è
limitare il danno che un attacco può fare.

Il sistema dovrebbe essere progettato in modo tale che gli strati più interni, in quanto meno complessi, siano
anche quelli più robusti in quanto il progettista dovrebbe riuscire meglio ad individuare tutte le vulnerabilità.

\subsection{Failing Safe e Failing Secure}
Si parla di \textbf{failing safe} quando il sistema continua a funzionare anche quando le componenti di sicurezza
hanno dei malfunzionamenti.

Per esempio si disattiva un firewall che non riesce più ad eseguire i suoi controlli di sicurezza.

Si parla invece di \textbf{failing secure} quando il sistema viene disattivato in modo da impedire ogni
comportamento scorretto ma anche impedendo ogni funzionamento.