\chapter{Analisi e attacchi}
L'approccio incondizionale comporta costi molto elevati e spesso inaccettabili, inoltre richiede una quantità di lavoro
enorme e spesso inutile.

Con l'approccio condizionale invece si cerca di capire quali componenti del sistema si possono difendere e soprattutto
quali componenti \emph{conviene} difendere.

Per capirlo è necessaria un'\textbf{analisi del rischio} con la quale si cerca di individuare la tipologia di attacco
più probabile in relazione al sistema che stiamo cercando di proteggere.

\section{Tipologie di analisi}
L'analisi del rischio si divide in diverse sottocategorie più specifiche che ci permettono di individuare in modo più
mirato eventuali problemi di diversa natura. In particolare parliamo di
\begin{itemize}
	\item Analisi delle risorse da proteggere (asset)
	\item Analisi delle minacce
	\item Analisi delle vulnerabilità
	\item Analisi degli attacchi
	\item Analisi degli impatti
	\item Individuazione del rischio, rischio accettabile ed introduzione di contromisure
\end{itemize}

\subsection{Analisi delle risorse}
In questa fase si cerca di individuare un insieme di oggetti ed alcune proprietà di sicurezza. Si definisce in seguito
una politica su questi oggetti in termini delle proprietà precedenti come diritti di lettura, scrittura, esecuzione
e così via.

\subsection{Analisi delle minacce}
In questa fase cerchiamo di capire chi è interessato ad attaccare il nostro sistema per rubare o modificare informazioni
o per impedire agli utenti di utilizzare il sistema.

Le possibili minacce possono arrivare sia da attaccanti che vogliono violare il sistema sia da eventi naturali che
potrebbero in qualche modo comprometterne l'integrità (in questo caso parliamo di \emph{safety}).

\subsubsection{Safety e Security}
Come anticipato, quando ci riferiamo alla capacità di un sistema di resistere ad eventi di origine non umana e casuale
come terremoti, crolli e così via, parliamo di \textbf{safety}.

Parliamo invece di \textbf{security} quando ci riferiamo alla capacità del sistema di resistere ad attacchi umani con
uno scopo malizioso per raggiungere un obbiettivo.

\subsection{Analisi delle vulnerabilità}
In questo caso cerchiamo di individuare quali sono le vulnerabilità che permettono ad un attaccante di ottenere, con
un numero finito di passi, l'accesso a risorse di suo interesse.

Ogni vulnerabilità abilita delle azioni dell'attaccante e per scoprirle possiamo usare metodi di ricerca automatici o
manuali condotti in ambiente locale o remoto.

Un attacco sfrutta le vulnerabilità ma non sempre va a buon fine, ha infatti una certa \textbf{probabilità di successo}
che aumenta o diminuisce in base al tipo di vulnerabilità che ha il sistema.

Per evitare che un attacco vada a buon fine è fondamentale che chi costruisce il sistema scopra le vulnerabilità più
critiche prima dell'attaccante.

\subsection{Analisi degli attacchi}
Un attacco ha vari \textbf{attributi} con i quali viene valutato:
\begin{itemize}
	\item Diritti richiesti / Diritti ottenuti
	\item Probabilità di successo
	\item Competenze / Abilità richieste
	\item Rumore generato
	\item Automatizzato / Automatizzabile
	\item Locale / Remoto
\end{itemize}
In generale un attacco ha delle precondizioni, ovvero dei requisiti per poter essere effettuato e delle postcondizioni,
ossia i diritti che si intende ottenere con tale attacco.

Un attacco che va a buon fine con requisiti minimi e tramite il quale si ottengono molti diritti è ciò che vogliamo
evitare.

L'obbiettivo è quello di rendere ogni attacco il più dispendioso possibile e anche nel caso in cui vada a buon fine i
diritti acquisiti devono essere minimi.

Alcuni approcci associano una valutazione di pericolosità ad un attacco ma in questo modo non abbiamo un indice
veritiero di quanto sia pericoloso un attacco. Ogni attacco va valutato in tutte le sue fasi e il sistema sul quale
viene effettuato è un parametro fondamentale che non può essere trascurato. Un attacco molto efficace su un certo tipo
di sistema potrebbe fallire subito su altri sistemi ad esempio.

\section{Attacchi}
Esistono varie tipologie di attacco e, come vedremo, varie tipologie di programmi per compierli, ognuno specializzato
in un particolare tipo di attacco.

\subsection{Attacco automatizzabile}
Gli attacchi automatizzabili, ovvero quelli per cui non sono necessarie azioni umane per i quali è sufficiente eseguire
un programma, sono molto più pericolosi di quelli non automatizzabili.

Si parla di intrusione automatizzabile se tutte le azioni che la compongono sono automatizzabili.

Gli attacchi automatizzabili costituiscono la proprietà caratterizzante della sicurezza informatica e possiedono due
vantaggi importanti per l'attaccante:
\begin{itemize}
	\item Tempo di esecuzione ridotto rispetto ad un attacco manuale
	\item Scarse competenze richieste dall'attaccante
\end{itemize}

\subsection{Attacco locale / remoto}
Un attacco che ha come bersaglio un certo sottosistema e che può essere eseguito
\begin{itemize}
	\item A patto che si disponga di un account sul sottosistema è detto \textbf{locale}.
	\item Senza disporre di un account sul sottosistema è detto \textbf{remoto}.
\end{itemize}
In generale un attacco automatizzabile eseguito da remoto è più pericoloso perché può essere eseguito da chiunque e
da un qualsiasi dispositivo connesso in rete.

\subsection{Malware}
Un \textbf{malware} è un software atto a provocare danni ad un dispositivo e a corrompere i dati al suo interno. I
tipici esempi di malware sono
\begin{itemize}
	\item Spyware
	\item Worm
	\item Trojan Horse
	\item Virus
	\item Miners
	\item Ransomware
\end{itemize}

\subsubsection{Spyware}
Uno \textbf{spyware} è un modulo hardware o software che raccoglie informazioni sull'attività di utente senza il suo
consenso.

In seguito trasmette queste informazioni tramite internet all'attaccante in modo che questo possa farne u utilizzo
illecito.

\subsubsection{Trojan Horse e Backdoors}
Questi due tipi di malware si introducono nel sistema sfruttando il fatto di sembrare innocui per poi compiere azioni
dannose come: furto di informazioni, criptazione dei dati (\textbf{ransomeware}) e installazione di strumenti in grado
di facilitare futuri attacchi da remoto.

\subsubsection{Worm}
Si tratta di un modulo composto da due parti:
\begin{itemize}
	\item \textbf{Attack vector}: codice che permette di attaccare nodi con cui interagisce il nodo su cui è in
	      esecuzione il worm.
	\item \textbf{Payload}: codice eseguito su tutti i nodi attaccati.
\end{itemize}
Lo scopo di un worm è dunque quello di usare un nodo attaccato per attaccarne altri e potersi così "\emph{riprodurre}".

\subsubsection{Virus}
Un \textbf{virus} può essere visto come un worm passivo, dato che non è in grado di diffondersi autonomamente, ma
richiede un meccanismo di trasporto per passare da un nodo all'altro come posta elettronica e chiavette USB.

\subsection{Intrusione mirata / non mirata}
In un'intrusione \textbf{mirata}, un attaccante stabilisce un obbiettivo e lo persegue fino a raggiungerlo cercando
di non essere scoperto e cercando di cambiare il suo comportamento in base all'obbiettivo.

Un'intrusione \textbf{non mirata} è realizzata mediante l'esecuzione di worm sulla rete a partire da un nodo iniziale.
In questo caso non è interesse dell'attaccante sapere dove va a finire il worm.

Un'intrusione può essere una fusione delle due tipologie descritte poco fa:
\begin{itemize}
	\item Un worm può attaccare tutti i sistemi che incontra fino a quando non raggiunge un obbiettivo descritto nel
	      suo payload.
	\item Si fa un'intrusione mirata per controllare il punto di lancio di un worm.
\end{itemize}