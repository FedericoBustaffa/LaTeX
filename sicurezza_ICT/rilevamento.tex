\chapter{Rilevamento intrusioni}
Per il rilevamento di intrusioni si fa uso di una componente del sistema, l'\textbf{intrusion detection system}
(\textbf{IDS}), che esamina informazioni e stato dei vari moduli per scoprire un attacco.

Non scopre tuttavia l'intrusione vera e propria, per quello serve costruire una correlazione tra i vari attacchi al
sistema.

Gli IDS si dividono in due tipologie principali:
\begin{itemize}
	\item \textbf{Local IDS}: esaminano le informazioni su un singolo nodo del sistema per scoprire attacchi avvenuti
	      su di esso.
	\item \textbf{Network IDS}: esaminano le informazioni scambiate e ricevute dai nodi di una sottorete per scoprire
	      attacchi contro uno dei nodi della sottorete.

	      Si tratta fondamentalmente di \emph{sniffer} che catturano (in modo trasparente) pacchetti e li esaminano
	      ricostruendo i protocolli di livello superiore.

	      Il problema di questi IDS è che la velocità di elaborazione e cattura dei pacchetti spesso non sta al passo
	      con la quantità di pacchetti inviati e si finisce per perderne alcuni.
\end{itemize}

\section{Rilevamento basato su anomalie}
Questa tecnica sfrutta il machine learning per riuscire ad apprendere il comportamento più comune del sistema.
Richiede quindi una fase iniziale di apprendimento che può durare anche dei mesi nella quale vengono monitorati
\begin{itemize}
	\item Invocazioni ai servizi
	\item Logging degli utenti
	\item Volume di richieste e traffico
\end{itemize}
Dopo l'apprendimento, qualsiasi comportamento troppo distante da quello appreso, viene identificato come
\textbf{anomalia}.

Ci sono tre possibili approcci per questo metodo:
\begin{itemize}
	\item \textbf{Statico}: le informazioni sul comportamento atteso sono prodotte dall'analisi statica della
	      struttura di un programma. Non richiede tempo ma è meno efficace.
	\item \textbf{Dinamico}: le informazioni sul comportamento di un programma vengono raccolte dalle esecuzioni e
	      quindi confrontate con il comportamento.
	\item \textbf{Ibrido}: il comportamento atteso del programma viene raccolto per colmare il divario dovuto ad
	      un'analisi statica e l'output viene confrontato con il comportamento.
\end{itemize}
Una volta appreso il comportamento, per riuscire a definire le anomalie, è necessario fornire un valore
\textbf{soglia}.

\subsection{Rilevamento basato su specifica}
Si tratta sempre di una tecnica basata sull'apprendimento di un comportamento atteso tramite il monitoraggio
dell'esecuzione ma questo è specificato dalla politica di sicurezza.

L'intrusione è dedotta tramite l'analisi delle specifiche confrontata con l'analisi dei diritti dei vari utenti. Se
un utente viola la specifica è perché sta provando ad attaccare il sistema.

Anche qui abbiamo i soliti tre approcci visti in precedenza ma definiti come segue:
\begin{itemize}
	\item \textbf{Statico}: un programma viene analizzato staticamente per calcolare le specifiche e il comportamento
	      viene confrontato con le specifiche.
	\item \textbf{Dinamico}: le informazioni sul comportamento del programma vengono raccolte e confrontate con le sue
	      specifiche.
	\item \textbf{Ibrido}: il compilatore crea una specifica e le osservazioni vengono raccolte per essere confrontate
	      con il comportamento del programma integrando quei casi che non possono essere risolti staticamente.
\end{itemize}

\section{Rilevamento basato su firma}
Con \textbf{firma} si fa riferimento a comportamenti e dati costanti che caratterizzano e identificano completamente
un malware.

Tutte le firme vengono raccolte in un database che guida il rilevamento. In questo modo si vengono a creare due
problemi: la \textbf{scoperta} di una firma e l'\textbf{aggiornamento} del database.

Questo metodo funziona solo se la firma del malware utilizzato è nota. Non è possibile rilevare un exploit mai
utilizzato prima o la cui firma non è presente nel database.

Il meccanismo prevede una politica \emph{default allow} e tutto ciò che differisce dalle firme nel database è
consentito. Definiamo di nuovo i tre approcci visti in precedenza:
\begin{itemize}
	\item \textbf{Statico}: il codice del programma viene analizzato e confrontato con la firma.
	\item \textbf{Dinamico}: le informazioni sul comportamento del programma vengono raccolte e confrontate con la
	      firma.
	\item \textbf{Ibrido}: si uniscono l'approccio statico e quello dinamico compiendo un'analisi statica che
	      seleziona i programmi sospetti. Questi vengono poi eseguiti e monitorati in un ambiente protetto.
\end{itemize}

\subsection{Firma locale e di rete}
Una \textbf{firma locale} è formata da una sequenza di
\begin{itemize}
	\item Informazioni di un log.
	\item Istruzioni in memoria.
	\item Invocazioni al sistema operativo.
\end{itemize}

Una \textbf{firma di rete} è una sequenza di informazioni che si possono trovare in uno o più pacchetti IP o in
una connessione TCP. In questo casi l'IDS fa un'ipotesi su come i pacchetti verrano ricevuti ed elaborati dal
destinatario.

\subsection{Polimorfismo di malware e virus}
Per ottenere virus sempre diversi si potrebbe cifrare il loro corpo in modo da ottenere virus sempre diversi dai quali
non è possibile rilevare una firma.

In questo ambito parliamo di virus o malware
\begin{itemize}
	\item \textbf{Crittografati}: il virus è composto da una parte detta \textbf{decryptor} e il codice del virus è
	      inizialmente criptato. Il virus sfrutta la cifratura del codice per eludere i controlli di firma. Una volta
	      dentro il sistema, il corpo del virus viene decifrato dal \emph{decryptor} ed eseguito.
	\item \textbf{Polimorfici}: si tratta di un'estensione dei virus crittografati. Una volta avvenuta la decifrazione,
	      il \emph{decryptor} genera una nuova chiave di cifratura casuale e cifra di nuovo il corpo del virus.

	      In questo modo non si riesce ad estrarre una firma dal virus poiché questo continua a mutare.
\end{itemize}

\section{Sandbox}
Una \textbf{sandbox} altro non è che una macchina virtuale dove eseguire il codice di un attaccante in modo sicuro. I
malware moderni tuttavia riescono a capire di essere eseguiti in una \emph{sandbox} e non rivelano il loro
comportamento pericoloso fino a che non superano i test per essere eseguiti sul sistema operativo (magari con permessi
di \emph{root}).

\section{Rilevamento basato su regole}
Si tratta di un modo per generalizzare il metodo basato su firma in cui si usa un insieme di regole che \emph{matchano}
i pacchetti ricevuti.

In questo modo abbiamo regole più astratte che riescono a gestire anche modifiche nel contenuto dei pacchetti.

\subsection{Snort}
Un IDS di questo tipo molto popolare è \textbf{Snort} composto dai seguenti moduli:
\begin{itemize}
	\item \textbf{Sniffer}: cattura i pacchetti che passano sul canale e possiede un motore di decifrazione.
	\item \textbf{Preprocessori}: ogni preprocessore esamina e manipola i pacchetti.
	\item \textbf{Motore di rilevamento}: analizza i pacchetti e li confronta con le regole dell'IDS.
	\item \textbf{Plugin per il rilevamento}: consentono ulteriori controlli sui pacchetti.
\end{itemize}

\subsubsection{Regole Snort}
Le \textbf{regole Snort} sono divise in due sezioni logiche:
\begin{itemize}
	\item \textbf{Intestazione} contenente
	      \begin{itemize}
		      \item Azione della regola
		      \item Protocollo utilizzato
		      \item Indirizzi IP di origine e destinazione
		      \item Maschere di rete
		      \item Informazioni sulle porte di origine e destinazione
	      \end{itemize}
	\item \textbf{Opzioni} contenente
	      \begin{itemize}
		      \item Messaggio di avviso
		      \item Informazioni sulle parti del pacchetto da ispezionare
	      \end{itemize}
\end{itemize}
Le azioni predefinite di Snort sono
\begin{itemize}
	\item \verb|alert|: genera un alert e logga il pacchetto.
	\item \verb|log|: logga il pacchetto.
	\item \verb|pass|: ignora il pacchetto.
	\item \verb|activate|: alert e poi attiva una regola dinamica.
	\item \verb|dynamic|: idle fino a quando non viene attivata poi logga i pacchetti.
	\item \verb|drop|: blocca e logga il pacchetto.
	\item \verb|reject|: blocca e logga il pacchetto e invia un \emph{TCP reset} o \emph{ICMP port unreacheable}.
	\item \verb|sdrop|: blocca il pacchetto senza loggarlo.
\end{itemize}
Un pacchetto dovrebbe essere controllato nell'ordine
\begin{center}
	\verb|drop > pass > alert > log|
\end{center}
in quanto è il modo più sicuro di operare e non permette a nessun pacchetto di passare se prima non è verificato
secondo le varie regole. Si tratta tuttavia di un metodo molto dispendioso, è per questo che si preferisce verificare
i pacchetti tramite il seguente ordine di regole
\begin{center}
	\verb|pass > drop > alert > log|
\end{center}

\section{Offuscamento del codice}
L'obbiettivo è rendere inefficienti gli strumenti di \emph{reverse engineering} che rendono possibile risalire al
codice ad alto livello dal codice in linguaggio macchina.

Una tecnica consiste nel cancellare parti di codice o fare uso di crittografia per rendere il codice polimorfico.

\section{Endpoint Detection \& Response - EDR}
Per \textbf{endpoint} si intende un nodo terminale della rete e l'EDR è un metodo di contromisura che si basa sul
monitoraggio continuo e la raccolta di dati dagli endpoint con funzionalità di analisi e risposta automatizzate.

Le funzioni principali di un sistema EDR sono:
\begin{itemize}
	\item \textbf{Monitaraggio e raccolta} dati.
	\item \textbf{Analisi} dati per identificare modelli di minaccia.
	\item \textbf{Risposta automatica} alle minacce identificate.
	\item \textbf{Ricerca} di minacce e attività sospette anche fra più nodi.
\end{itemize}
Un sistema EDR ha tre aspetti fondamentali:
\begin{itemize}
	\item \textbf{Agenti} software che monitorano sistemi, raccolgono informazioni su attacchi e le trasmettono ad un
	      database centrale.
	\item Un \textbf{motore analitico} in grado di analizzare i dati raccolti e, in caso di rilevamento di un minaccia
	      nota, si attiva una risposta automatica.
	\item \textbf{Analisi} per l'individuazione di minacce non note.
\end{itemize}

\subsection{Reazione ad un attacco}
Quando si subisce un attacco si può reagire in due possibili modi:
\begin{itemize}
	\item Si aggiungono contromisure sul sistema attaccato, cercando di migliorare le proprie difese.
	\item Si attacca il sistema dell'attaccante.
\end{itemize}
La seconda opzione è generalmente da evitare poiché gli attaccanti usano infrastrutture di \emph{command and control}
costruite tramite intrusioni precedenti.

L'unico modo utile di agire sarebbe quello di smantellare l'intera infrastruttura ma questo ci pone davanti a problemi
tecnici e legali di rilievo.

\subsubsection{Attribuzione}
Per reagire in modo offensivo si deve risolvere il problema dell'\textbf{attribuzione}, ossia il problema di
determinare chi ha condotto l'attacco.

In genere per riuscirci si deve monitorare in anticipo la struttura da cui proviene l'attacco. Il problema è che
rivelare il monitoraggio per l'attribuzione informa gli attaccanti che sono stati monitorati.

Un altro modo è quello dell'attribuzione delle TTPs utilizzate e sulla similarità del codice degli exploit utilizzati.

\subsubsection{Act of War}
Le assicurazioni non coprono gli \emph{act of war}. Se un attacco è sponsorizzato o eseguito da uno stato le
assicurazioni lo classificano come \emph{act of war} che non è coperto da assicurazione per furto o distruzione di
dati.