\chapter{Contromisure}
Quando parliamo di \textbf{contromisure} facciamo riferimento a delle modifiche che il difensore fa al sistema in
risposta alle intrusioni. Una prima classificazione delle contromisure le divide in tre grandi categorie:
\begin{itemize}
	\item \textbf{Proattive}: Applicate prima di un'intrusione per impedirla.
	\item \textbf{Dinamiche}: Applicate durante un'intrusione per fermarla.
	\item \textbf{Reattive}: Applicate dopo un'intrusione per fermare quelle successive.
\end{itemize}
Il processo per l'eliminazione delle vulnerabilità prevede i seguenti passi:
\begin{enumerate}
	\item Prevenire
	\item Resistere
	\item Scoprire
	\item Ripristinare
	\item Reagire
\end{enumerate}

\section{Robustezza e resilienza}
Due punti cardine della sicurezza sono \textbf{robustezza} e \textbf{resilienza}.

Per \emph{robustezza} si intende la capacità di un sistema di \emph{respingere} le intrusioni ma nel caso in cui il
sistema venga violato le prestazioni (in termini di sicurezza) calano. La capacità di un sistema di ripristinare la
robustezza dopo un'intrusione è detta \emph{resilienza}.

\subsection{Ridondanza ed eterogeneità}
Due punti importanti su cui si basano i principi di robustezza e resilienza sono \textbf{ridondanza} ed
\textbf{eterogeneità}.

\subsubsection{Ridondanza}
La \emph{ridondanza} si ottiene sovradimensionando il sistema con componenti o procedure \emph{inutili} ai fini di
funzionamento del sistema ma che sono molto utili nel momento in cui il sistema viene attaccato o violato per
resistere all'attacco o ripristinare lo stato sicuro. La ridondanza può essere di vario tipo:
\begin{itemize}
	\item \textbf{Ridondanza fredda}: componenti inattive che vengono attivate solo quando il sistema viene attaccato
	      per riuscire a ripristinare lo stato sicuro.
	\item \textbf{Ridondanza calda}: componenti sempre attive in modo da riuscire a sopportare meglio gli attacchi.
	\item \textbf{Triple Modular Redundancy}: ridondanza calda dove tre copie di uno stesso modulo prendono lo stesso
	      input ed eseguono la stessa computazione \emph{votando} infine il risultato. Aumenta la safety ma non sempre
	      la security.
\end{itemize}

\subsubsection{Eterogeneità}
Per \emph{eterogeneità} si intende l'uso di componenti diversi che svolgono compiti simili per evitare la caduta del
sistema su una singola vulnerabilità. Se un componente ha una vulnerabilità e più nodi del sistema hanno quel
componente, tutti i nodi con quel componente avranno quella vulnerabilità.

Un tipico esempio di eterogeneità è quello in cui si usano sistemi operativi diversi su una stessa rete.

\subsection{Sistema minimale}
Una buona pratica per la sicurezza è quella di costruire un sottoinsieme del sistema, formato da componenti
particolarmente robusti e spesso eterogenei rispetto a quelli del sistema principale.

Il sistema minimale è quello da cui parte il ripristino del normale funzionamento del sistema.

La violazione del sistema minimale comporta l'incapacità di ripristinare il sistema dopo un attacco.

\section{Meccanismi per contromisure}
L'implementazione di contromisure è basata di meccanismi condivisi i quali devono essere molto robusti dato che una
loro vulnerabilità potrebbe influenzare molte contromisure del sistema.

Tali meccanismi sono implementati su un security kernel che fa parte del TCB e che gestisce: l'\textbf{identità},
l'\textbf{autenticazione} e i \textbf{diritti} degli utenti.

\subsection{Autenticazione}
Per implementare un buon meccanismo di autenticazione si fa sempre riferimento alla tripla
\begin{center}
	\verb|<soggetto, oggetto, operazione>|
\end{center}
la quale associa ad un soggetto un certo diritto. Possiamo associare due tipi di controlli a questa tupla:
\begin{itemize}
	\item Controlli sull'identità del soggetto.
	\item Controlli sul possesso del diritto \verb|<oggetto, operazione>|.
\end{itemize}
Il controllo sul possesso dei diritti dei vari utenti e la loro gestione è compito del sistema operativo. Anche
l'autenticazione è compito del sistema operativo ma attualmente esistono componenti specializzati per l'autenticazione
che forniscono un token in grado di certificare in qualche modo l'avvenuta autenticazione.

\subsubsection{Tipi di autenticazione}
Esistono tre tipi principali di autenticazione:
\begin{itemize}
	\item \textbf{Debole statica}: include password e altre tecniche soggette ad attacchi che cercano di riprodurre
	      le sequenze di autenticazione.
	\item \textbf{Debole non statica}: fa uso di meccanismi crittografici per creare password utilizzabili per una
	      singola sessione. Può essere compromessa da attacchi di tipi \emph{session hijacking}, ossia dal
	      \emph{furto} della sessione.
	\item \textbf{Forte}: utilizza forti meccanismi crittografici per prevenire i principali problemi legati
	      all'autenticazione debole.
\end{itemize}

\subsubsection{Meccanismi di autenticazione}
Tra i meccanismi di autenticazione più comuni abbiamo:
\begin{itemize}
	\item L'utilizzo di frasi o numeri che solo l'utente conosce. Per difendersi da possibili attacchi, in genere si
	      salva un hash di queste informazioni e non l'informazione in chiaro. Un altro metodo consiste nell'uso di
	      database diversi per impedire attacchi che precalcolano l'hash.
	\item L'utilizzo di qualcosa che solo l'utente possiede, come una chiave di cifratura. Con questo metodo di
	      autenticazione si può
	      \begin{itemize}
		      \item Generare il prossima bit di una sequenza pseudocasuale.
		      \item Applicare una funzione condivisa ad un valore ricevuto dal server e restituire il risultato.
	      \end{itemize}
	\item L'utilizzo di parametri biometrici. Questo tipo di autenticazione porta con sé problemi di falsi
	      positivi/negativi, attacchi alla trasmissione di informazioni biometriche e difficoltà nel cambiare la
	      password.
\end{itemize}

\subsubsection{Kerberos}
Il sistema \textbf{Kerberos} permette ad un client di autenticarsi con un server tramite due entità: un \textbf{KDC}
(Key Distribution Center) e un \textbf{TGS}	(Ticket Granting Service).

Il KDC condivide con ogno entità del sistema (client, server e TGS) una chiave di cifratura. Per un client, conoscere
tale chiave equivale ad essere autenticato.

Un client che si connette al sistema deve essere in grado di decifrare qualsiasi cosa cifrata con la chiave che il KDC
gli ha fornito. Dato che il sistema comunica con gli utenti tramite messaggi crittografati, un utente malizioso che si
finge qualcun altro non sarebbe in grado di decifrare tali messaggi e dunque non riuscirebbe ad autenticarsi.

Supponiamo ora che un utente $A$ voglia autenticarsi ad un server tramite Kerberos:
\begin{enumerate}
	\item In una fase preliminare il KDC condivide due chiavi simmetriche $k_{TGS}$ e $k_A$ rispettivamente con il TGS
	      e con $A$.
	\item L'utente $A$ richiede i servizi del server $S$ e fa richiesta al sistema identificandosi come $A$.
	\item Il KDC invia informazioni cifrate con la chiave $k_A$ ad $A$ per interagire con il TGS.
	\item L'utente $A$ decifra le informazioni e le invia al TGS. Ciò che $A$ invia al TGS è cifrato con la chiave
	      del TGS.
	\item Il TGS decifra ciò che $A$ gli ha inviato e se ottiene informazioni consistenti fornisce un ticket di
	      validità temporanea ad $A$.
	\item $A$ usa il ticket per autenticarsi con $S$ ed essere quindi in grado di usufruire dei suoi servizi.
\end{enumerate}
Il KDC e il TGS gestiscono dei database in cui sono memorizzate le \emph{master key} dei client e dei server registrati.
Le \emph{master keys} all'interno del database sono criptate con la chiave privata rispettivamente del KDC e del TGS.
Se qualcuno venisse a conoscenza di queste due chiavi, l'intero sistema sarebbe compromesso.

\subsection{Controllo e gestione dei diritti}
Una volta avvenuta l'autenticazione l'utente è riconosciuto dal sistema ma ciò che può fare all'interno di esso dipende
dai diritti che possiede.

Per controllare quali diritti un soggetto $S_i$ ha nel sistema si fa utilizzo della \textbf{matrice di controllo degli
	accessi}, la quale mette in relazione il soggetto $S_i$ che vuole compiere un'azione e l'oggetto $O_j$ su cui viene
compiuta e ricava i diritti (o operazioni $op_{ij}$) che quel soggetto ha (o può fare) sull'oggetto.
\begin{center}
	\begin{tabular}{ c | c | c | c | c |}
		      & $O_1$ & $O_2$     & \dots     & $O_n$ \\
		\hline
		$S_1$ &       & $op_{12}$ &           &       \\
		\hline
		$S_2$ &       &           &           &       \\
		\hline
		\dots &       &           & $op_{ij}$ &       \\
		\hline
		$S_m$ &       &           &           &       \\
		\hline
	\end{tabular}
\end{center}
In genere una matrice del genere è presente a livello di sistema operativo per protezione di zone di memoria, file,
dispositivi di I/O e porte dei vari protocolli. Possono esserci matrici di controllo anche a livello applicativo per
le specifiche applicazioni nel caso fosse necessario.

In generale la matrice del sistema operativo determina quali utenti possono interagire con un'applicazione. Questo si
può controllare mediante il meccanismo delle porte o dei canali di comunicazione.

Per gli utenti che possono interagire la matrice determina quali operazioni nello specifico possono essere eseguite
sull'oggetto. Se invece un utente non può interagire conviene impedire l'interazione al livello più basso possibile
e scartare la richiesta.

Tutte queste operazioni di controllo comportano un \emph{overhead}, in particolare l'operazione di scarto di una
richiesta e rende il sistema più soggetto ad attacchi di tipo DOS. Ecco perché è fondamentale compiere ottimizzazioni
per scartare le richieste il prima possibile.

\subsection{Implementazione della matrice}
La matrice di controllo degli accessi può essere implementata principalmente in due modi:
\begin{itemize}
	\item Per \textbf{righe}: soluzione a \emph{capability} o \emph{ticket}.
	\item Per \textbf{colonne}: soluzione a \emph{lista di controllo} degli accessi (ACL).
\end{itemize}
La soluzione a \emph{capability} memorizza le informazioni relative ad un oggetto e alle relative operazioni che si
possono effettuare su esso ad un certo indirizzo (logico o fisico).

Questo indirizzo viene memorizzato nell'ambiente del soggetto. Quando un soggetto vuole eseguire un'operazione deve
indicare l'operazione che vuole eseguire e la \emph{capability} di cui dispone.

\subsection{Capability}
La soluzione a \emph{capability} memorizza le informazioni nel soggetto che poi presenta all'oggetto le operazioni
che può fare.

La soluzione a \emph{capability} più nota è quella per la protezione dei segmenti di memoria. Questa tecnica è
implementata tramite la tabella usata per la traduzione da indirizzi logici a fisici, nella quale troviamo anche un
gruppo di bit che indicano le operazioni possibili su quel segmento.

Nel caso più generale, ogni puntatore ad un oggetto diventa un puntatore protetto che indica anche quali sono le
operazioni possibili sull'oggetto puntato.

I puntatori vengono memorizzati nell'ambiente del soggetto e devono essere adeguatamente protetti per evitare che il
soggetto li manipoli in modo improprio. Un tipico approccio per la protezione è la cifratura dei puntatori non usati
e la successiva decifrazione solo in fase di utilizzo.

\subsubsection{Implementazione centralizzata}
Sono architetture progettate e costruite dove tutti i puntatori sono \emph{capability} protette dall'hardware e/o dal
sistema operativo tramite registri specializzati.

In questo caso una funzione viene denominata da un indice in un segmento o in una tabella protetta dal sistema
operativo.

\subsubsection{Implementazione distribuita}
Questa soluzione non prevede una protezione delle \emph{capability} a livello hardware o di sistema operativo dato che
queste devono essere trasmesse in rete.

In questo caso si fa uso di crittografia per la protezione delle \emph{capability}. Chi fornisce un servizio o gestisce
un oggetto gestisce anche un \textbf{segreto}. L'utente che richiede il servizio \emph{firma} la richiesta o parte di
essa con la sua chiave privata così che chiunque possa verificare l'identità tramite la sua chiave pubblica.

Questo tipo di implementazione possiede le seguenti proprietà:
\begin{itemize}
	\item \textbf{Protezione} dal \emph{tampering} (non si può manipolare) e dal \emph{furto} (se chi la presenta
	      viene autenticato).
	\item \textbf{Controllo della propagazione}: la \emph{capability} può essere creata solo dal gestore.
	\item \textbf{Delega}: un soggetto può trasmettere ad altri la capability ma si possono creare problemi con la
	      protezione.
	\item \textbf{Revoca}: si revoca una \emph{capability} ad un soggetto.
\end{itemize}

\subsection{ACL}
L'\textbf{ACL} non è altro che una lista associata ad ogni oggetto in cui sono memorizzati i diritti necessari per
interagire con quell'oggetto.

Un soggetto per poter interagire presenta i diritti di cui è in possesso e nel caso sia idonei all'interazione con
l'oggetto l'operazione viene effettuata, in caso contrario la richiesta viene rifiutata.

Un tipico esempio è quello dei router, i quali contengono delle \emph{regole} come ad esempio una lista di IP dai
quali si possono accettare pacchetti in ingresso e una lista di IP da cui rifiutare qualsiasi pacchetto ricevuto.

\subsection{Controllo degli accessi basato su ruoli}
Questo tipo di controllo non ragiona in termini di soggetti ma in termini di \textbf{ruoli}, ossia di profili
professionali diversi come ad esempio in un'azienda in cui abbiamo da un lato l'amministratore delegato e dall'altro
il dipendente.

La matrice specifica i diritti per i singoli ruoli e non per i singoli utenti andando a risparmiare memoria e
semplificando la gestione. Il meccanismo prevede una funzione o una tabella che mappa ogni utente con un ruolo e
relativi diritti.

\subsubsection{SE - Linux}
Si tratta di una versione di Linux definita da NSA per permettere all'utente di definire una qualunque politica di
sicurezza come un qualsiasi parametro del sistema operativo configurabile dall'utente. In particolare si possono
definire
\begin{itemize}
	\item I diritti utente sulle risorse.
	\item I diritti di ogni programma sulle risorse.
	\item I programmi che ogni utente può eseguire.
\end{itemize}

\subsection{Firewall}
Un \textbf{firewall} è un componente che separa due reti di diversa criticità e concentra tutte le linee in ingresso o
uscita tra le due reti e filtra le comunicazioni per implementare una certa politica di sicurezza.

Possiamo definire un firewall in base a due caratteristiche ortogonali:
\begin{itemize}
	\item I protocolli che conosce e che filtra, ossia il tipo di controlli che il firewall implementa.
	\item La sua implementazione, la quale va a determinare la robustezza del firewall.
\end{itemize}
Il firewall determina un punto di centralizzazione delle comunicazioni e quindi un potenziale collo di bottiglia per
le prestazioni.

\subsubsection{Controlli}
In firewall per protocolli dello stack TCP/IP i controlli dipendono dal livello dello stack a cui operano.

Un firewall che lavora al livello 3, ovvero il \textbf{livello IP}, opera solo sui bit nell'header del pacchetto e può
controllare
\begin{itemize}
	\item Mittente
	\item Destinatario
	\item Porte
	\item Protocollo usato
	\item Origine di connessioni di livello 4
\end{itemize}
Un firewall che invece lavora al livello 4, ossia il \textbf{livello TCP}, è detto \textbf{circuit level gateway} ed in
grado di controllare
\begin{itemize}
	\item Che la connessione esista
	\item Il contenuto del pacchetto (\emph{stateful inspection})
\end{itemize}
I firewall di \textbf{livello applicativo} sfruttano la conoscenza del protocollo applicativo e possono anche scoprire
attacchi che sfruttano vulnerabilità di tale protocollo.

\subsubsection{Circuit level gateway}
Un \textbf{circuit level gateway} fa uso di una tabella delle connessioni per riuscire a validare un pacchetto in
arrivo. Questo meccanismo serve a riconoscere se il pacchetto in arrivo rappresenta una richiesta di connessione al
nodo o se si tratta di un pacchetto in transito e che quindi deve solo passare dal nodo.
\begin{enumerate}
	\item Quando una connessione TCP viene instaurata viene memorizzata nella tabella.
	\item Il firewall cerca di determinare, tramite	l'ispezione del pacchetto in arrivo e le sue regole interne, se
	      esso possa entrare nel nodo o se debba essere rifiutato.
	\item Esamina la connessione da cui proviene il pacchetto e controlla se questa è presente nella tabella.
	\item Se è presente allora si tratta di una connessione valida e accetta il pacchetto altrimenti lo scarta.
	\item Una volta terminata la sessione si elimina anche la connessione dalla tabella.
\end{enumerate}
Questo metodo funziona anche con UDP sebbene sia un protocollo \emph{connectionless} è comunque possibile ricostruire
i circuiti logici.

Un possibile attacco a questo firewall tenta di saturare la tabella delle connessioni. Quando la tabella è satura non è
possibile eliminare una connessione ancora attiva e si vengono a creare due possibili scenari:
\begin{itemize}
	\item \textbf{Default allow}: Il firewall accetta qualsiasi richiesta di connessione.
	\item \textbf{Default deny}: Il firewall rifiuta qualsiasi richiesta di connessione.
\end{itemize}

\subsubsection{Proxy e reverse proxy}
Per \textbf{proxy} si intende un firewall specializzato nel proteggere un client da attacchi esterni. Controlla ad
esempio che il servizio richiesto ad un determinato server non sia pericoloso per l'integrità del sistema.

Il \textbf{reverse proxy} è invece un firewall specializzato nel proteggere un server da attacchi client. Per esempio
controlla che le richieste dei client non sia formattate o formulate in modo da apportare danni al server o in modo da
permettere in qualche modo un'intrusione.

\subsubsection{Stateful firewall}
Sono firewall che lavorano su tre livelli (IP, TCP e applicativo) e adottano un metodo di filtraggio e controllo
dinamico dei pacchetti.

Il firewall mantiene una tabella di stati che tiene traccia di tutte le sessioni e le caratteristiche di ogni pacchetto
che passa dal firewall. Se il pacchetto in arrivo possiede le caratteristiche nella tabella il firewall permette che
questo passi nel nodo.

\subsubsection{NAT firewall}
In una rete privata con un certo numero di nodi spesso si utilizza un \textbf{NAT firewall} che ha il compito di
fornire e gestire pochi indirizzi IP per la comunicazione all'esterno della rete.

In questo modo più nodi possono usare (a turno) lo stesso indirizzo IP globale per parlare all'esterno della rete mentre
usano un indirizzo locale per comunicare sulla rete interna.

\subsubsection{Egress filtering}
Un firewall può interrompere una connessione nel momento in cui l'analisi del contenuto rivela che c'è un'intrusione in
atto.

Con l'\textbf{egress filtering} si fa un controllo non solo delle comunicazioni in ingresso ma anche di quelle in
uscita. Questo meccanismo è fondamentale per rilevare \emph{botnet} connesse a nodi protetti della rete e per prevenire
attacchi di tipo DDOS.

\subsubsection{Micro segmentazione}
Un firewall potrebbe essere usato per difendere un'applicazione invece dell'interno nodo su cui questa gira.
\begin{itemize}
	\item Si fanno girare le varie applicazioni su una o più macchine virtuali.
	\item Si usa una terza macchina virtuale per eseguire un firewall che protegge le altre macchine virtuali.
\end{itemize}
Questo tipo di soluzione rappresenta un problema in termini di prestazioni ma un ottimo compromesso rispetto ad
eseguire ogni applicazione e il firewall su macchine fisiche indipendenti.

Con l'utilizzo di \emph{container} rinunciamo ad un po' di sicurezza ma favoriamo le prestazioni.

\subsection{Honeypot e Honeynet}
Le \textbf{honeypot} e \textbf{honeynet} sono sistemi o reti la cui funzione è quella di essere attaccati al posto del
sistema o della rete principale. Le motivazioni principali per l'utilizzo di contromisura sono due:
\begin{itemize}
	\item Scoprire come un sistema viene attaccato riuscendo a capire quali sia le vulnerabilità e come queste vengano
	      sfruttate dall'attaccante.
	\item Rallentare e scoprire un attaccante deviandolo verso un sistema che non provoca alcun danno.
\end{itemize}

Le differenze sostanziali tra i due tipi di contromisura sono queste:
\begin{itemize}
	\item Quando parliamo di \textbf{honeypot} stiamo facendo riferimento ad un sistema progettato appositamente per
	      essere attaccato e che comprende meccanismi per ricordare il comportamento dell'attaccante e fermarlo.

	      Il suo principale obbiettivo è quello di rallentare l'attaccante e scoprire come questo opera.
	\item Quando invece parliamo di \textbf{honeynet} stiamo facendo riferimento ad una rete vera e propria collegata
	      ad un sistema reale su cui viene usato il software vero e proprio.
\end{itemize}

\subsubsection{Honeypot}
Una \textbf{honeypot} può essere classificata con metodi alternativi:
\begin{itemize}
	\item Livello di \textbf{interazione}
	      \begin{itemize}
		      \item \textbf{Basso}: si tratta di un \emph{port-listener}.
		      \item \textbf{Medio}: un servizio emulato che analizza le comunicazioni e restituisce risposte simulate
		            per comportarsi come un servizio reale (\emph{honeyd}).
		      \item \textbf{Alto}: richiede l'uso di servizi reali e host completamente operativi o reti complete per
		            ingannare l'attaccante. Sebbene sia migliore per ingannare l'attaccante comporta maggiori rischi
		            per la sicurezza e dev'essere monitorato e protetto a dovere (\emph{honeynet}).
	      \end{itemize}
	\item Livello di \textbf{implementazione}
	      \begin{itemize}
		      \item \textbf{Virtuale}: viene simulata una rete tramite macchine virtuali (\emph{honeypot}).
		      \item \textbf{Fisico}: viene utilizzata una rete fisica (\emph{honeynet})
	      \end{itemize}
	\item \textbf{Obbiettivo}
	      \begin{itemize}
		      \item \textbf{Produzione}: rallentare l'attaccante.
		      \item \textbf{Ricerca}: capire modalità d'azione dell'attaccante.
	      \end{itemize}
\end{itemize}

\subsubsection{Honeyd}
Un \textbf{honeyd} è un singolo host che simula reti di grosse dimensioni e permette un alto livello di
configurabilità.

Un \emph{honeyd} rileva attività non autorizzate monitorando gli indirizzi IP non utilizzati di uno specifico
intervallo, detto \textbf{dark space}. Ogni connessione verso il \emph{dark space} viene considerata come un possibile
attacco o comunque come il risultato di una scansione.

Un \emph{honeyd} opera in questo modo
\begin{enumerate}
	\item Un \textbf{packet dispatcher} processa i pacchetti in arrivo e controlla indirizzo IP e integrità.
	\item Accetta solo TCP , UDP e ICMP.
	\item Interroga file di configurazione alla ricerca del modello associato all'IP di destinazione.
	\item Se necessario interagisce con applicazioni esterne capaci di gestire il flusso dati.
	\item Un meccanismo di reindirizzamento gira le richieste ad un sistema reale se necessario.
	\item Un \textbf{personality engine} elabora un pacchetto di risposta e lo modifica per renderlo compatibile con
	      lo stack utilizzato dal sistema operativo simulato.
\end{enumerate}

\section{VPN}
Si tratta di una rete che emula una connessione sicura sopra una rete pubblica (non sicura). Nella pratica connette
due sottoreti (protette da firewall) cifrando e autenticando tutto il traffico tra ogni coppia di firewall.

\`E una meccanismo soggetto ad attacchi di tipo DDOS poiché l'unico modo che si ha di scoprire traffico malevolo
in transito sul canale è decifrando.

\subsection{IPSEC}
Si tratta di un'estensione di IPv4 che cifra e autentica flussi di informazioni e che offre due possibili protocolli:
\begin{itemize}
	\item \textbf{Authentication Mode}: è previsto un \emph{header} di autenticazione nel pacchetto in transito.
	\item \textbf{Encapsulated Security Payload}: le informazioni nel pacchetto sono cifrate.
\end{itemize}
Entrambi i protocolli possono essere usati in due modi:
\begin{itemize}
	\item \textbf{Transport Mode}: si inferiscono nuovi campi nel pacchetto originale e dopo lo si invia. \`E
	      rischioso poiché le informazioni dei nodi terminali come l'indirizzo IP viaggiano in chiaro.
	\item \textbf{Tunnel Mode}: il pacchetto IP è protetto e diventa il contenuto informativo di un nuovo pacchetto,
	      nascondendo così le informazioni private degli utenti.
\end{itemize}
Ogni connessione è protetta mediante crittografia simmetrica mentre i vari nodi che interagiscono possono essere
dotati di una coppia chiave pubblica e privata.

In una fase iniziale si adotta un protocollo per lo scambio di chiavi tramite crittografia asimmetrica. Una volta
stabilita la connessione, questa viene protetta tramite crittografia simmetrica.

Le chiavi simmetriche vengono aggiornate periodicamente:
\begin{itemize}
	\item Per limitare l'impatto che avrebbe un attacco in grado di scoprire scoprire la chiave.
	\item \`E possibile sfruttare il volume di dati trasmessi per condurre attacchi che sfruttano particolari
	      configurazioni di bit per violare il sistema.
\end{itemize}
IPSEC definisce quindi 4 nuovi protocolli:
\begin{itemize}
	\item \textbf{Authentication Header}: autentica gli utenti e protegge l'integrità.
	\item \textbf{Encpsulating Security Payload}: garantisce la confidenzialità proteggendo il contenuto informativo
	      della comunicazione.
	\item \textbf{Internet Key Exchange}: due utenti concordano la chiave da usare e per quanto tempo usarla.
	\item \textbf{Internet Security Association and Key Management Protocol}: protocollo necessario per stabilire e
	      aggiornare la Security Association e i suoi attributi.
\end{itemize}

\subsubsection{Security Association}
Una \textbf{security association} o \textbf{SA} descrive una connessione diretta con i servizi associati al traffico
che la connessione trasmette. Per assicurare una connessione bidirezionale sono richieste due security association
(una per ogni direzione).

La SA comprende anche tutte le informazioni per rendere sicura la comunicazione.
