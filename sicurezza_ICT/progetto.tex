\chapter{Verifica del progetto di sistema}
Nel 1975, Saltzer \& Schroeder hanno definito dei principi che permettono di scoprire vulnerabilità non legati ad
errori di programmazione:
\begin{enumerate}
	\item \textbf{Economia dei meccanismi}: l'implementazione dei sistemi di
	      sicurezza devono essere semplici e compatti.
	\item \textbf{Fail-safe Default}: i meccanismi di protezione dovrebbero
	      vietare l'esecuzione di qualsiasia azione in assenza di diritti
	\item \textbf{Mediazione completa}: il meccanismo di protezione dovrebbe
	      controllare l'accesso ogni volta ad ogni oggetto.
	\item \textbf{Open Design}: il sistema deve rimanere sicuro finché
	      l'attaccante non scopre la chiave di cifratura.
	\item \textbf{Privilegio di separazione}: il meccanismo di protezione
	      dovrebbe permettere l'accesso tramite più di un pezzo di
	      informazione.
	\item \textbf{Privilegio minimo}: ogni processo dovrebbe essere eseguito
	      con il numero minimo di diritti.
	\item \textbf{Meccanismo comune}: si deve ridurre al minimo la
	      condivisione di informazioni tra utenti. Ogni canale di
	      condivisione può essere causa di vulnerabilità.
	\item \textbf{Psychological Acceptability}: Il meccanismo di protezione dovrebbe essere facile da utilizzare
	      per l'utente finale.
\end{enumerate}
Ogni sistema deve stabilire un compromesso tra verifica di questi principi e prestazioni, la regola fondamentale è
questa: possono mancare solo controlli che peggiorano delle prestazioni di interesse, un controllo che non peggiora
le prestazioni deve essere presente.

