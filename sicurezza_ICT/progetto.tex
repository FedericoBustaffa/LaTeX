\chapter{Verifica del progetto di sistema}
In questo capitolo andiamo a vedere come un sistema dovrebbe essere progettato per limitare al minimo le
vulnerabilità.

\section{Principi di Saltzer \& Schroeder}
Nel 1975, Saltzer \& Schroeder hanno definito dei principi che permettono di scoprire vulnerabilità non legati ad
errori di programmazione:
\begin{enumerate}
	\item \textbf{Economia dei meccanismi}: l'implementazione dei sistemi di sicurezza devono essere semplici e
	      compatti.
	\item \textbf{Fail-safe Default}: i meccanismi di protezione dovrebbero vietare l'esecuzione di qualsiasia
	      azione in assenza di diritti
	\item \textbf{Mediazione completa}: il meccanismo di protezione dovrebbe controllare l'accesso ogni volta ad
	      ogni oggetto.
	\item \textbf{Open Design}: il sistema deve rimanere sicuro finché l'attaccante non scopre la chiave di
	      cifratura.
	\item \textbf{Principio di separazione}: il meccanismo di protezione dovrebbe permettere l'accesso tramite più
	      di un pezzo di informazione.
	\item \textbf{Privilegio minimo}: ogni processo dovrebbe essere eseguito con il numero minimo di diritti.
	\item \textbf{Meccanismo comune}: si deve ridurre al minimo la condivisione di informazioni tra utenti. Ogni
	      canale di condivisione può essere causa di vulnerabilità.
	\item \textbf{Psychological Acceptability}: Il meccanismo di protezione dovrebbe essere facile da utilizzare
	      per l'utente finale.
\end{enumerate}
Ogni sistema deve stabilire un compromesso tra verifica di questi principi e prestazioni, la regola fondamentale è
questa: possono mancare solo controlli che peggiorano delle prestazioni di interesse, un controllo che non peggiora
le prestazioni deve essere presente.

\subsection{Fail-safe default}
Il sistema deve essere strutturato in modo che un certo soggetto debba perforza avere un \emph{ragione} per compiere
una determinata azione su un certo oggetto.

Se non c'è ragione per un soggetto di compiere una determinata azione non deve poterla compiere. Se non si è in grado
di spiegare perché un soggetto possa compiere un'operazione si sta violando il principio del privilegio minimo.

\subsection{Mediazione completa}
Recentemente sono stati aggiunti dei supporti hardware più veloci ed efficienti delle architetture a capability.

La memoria virtuale viene partizionata in zone ed ogni zona viene dedicata ad un certo tipo di dato e nessuna zona
di memoria può essere utilizzata da tipi di dato diversi.

Alcuni bit di ogni puntatore di memoria (\textbf{tag}) sono dedicati ad informazioni sul tipo di dato puntato e ad
ogni accesso si controlla che il tipo del puntatore sia coerente con quello dell'area di memoria puntata.

\subsection{Open design}
Un attaccante che scopre una chiave di cifratura non apprende nulla di utile per decifrare informazioni cifrate con
chiavi diverse.

Il sistema dovrebbe essere progettato assumendo che prima o poi l'attaccante scoprirà come questo è implementato. In
altre parole, il sistema deve rimanere sicuro anche quando l'attaccante scopre come è fatto. Per garantire questa
sicurezza è fondamentale che non venga scoperta la chiave.

In ogni caso è sempre bene lasciare all'attaccante il lavoro di scoprire come il sistema è definito.

L'unico caso in cui è utile rivelare informazioni è quello in cui viene effettuata una \textbf{peer review}, ossia
un'analisi, condotta da esperti, che rivela le vulnerabilità del sistema.

\subsection{Principio di separazione}
Ogni sistema dovrebbe implementare un meccanismo di \textbf{difesa in profondità} segmentando la rete in sottoreti,
ognuna protetta da un firewall, in modo da impedire che l'attacco ad un qualsiasi nodo della rete venga usato come
punto di appoggio per attacchi ad altri nodi della rete.

In una rete segmentata l'attaccante si muove più lentamente perché per raggiungere un nodo qualsiasi deve attaccare
prima una serie di altri nodi (\textbf{pivoting}). Più firewall differenti proteggono i nodi intermedi più lavoro sarà
richiesto all'attaccante per compiere l'attacco.

\subsection{Privilegio minimo: Zero trust network}
Una VPN può essere molto pericolosa se il client è stato attaccato con successo dato che fornisce un canale sicuro e
criptato all'attaccante.

Nelle reti \textbf{zero trust}, l'esito di una richiesta di un dispositivo, dipende da informazioni disponibili sul
dispositivo e dall'utente che lo usa, non dalla posizione fisica del dispositivo.

Le richieste sono elaborate da un \textbf{trust engine} che riceve informazioni su utente/dispositivo/servizio e sulla
rete e decide di accettare o rifiutare in base al rischio.

I fondamenti su cui si basa il meccanismo di \emph{zero trust} sono
\begin{itemize}
	\item La conoscenza di dispositivi, utenti e risorse.
	\item Tutte le comunicazioni (anche all'interno della stessa sottorete) sono protette.
	\item La comunicazione vale solo per la singola richiesta (privilegio minimo) poi viene distrutta.
	\item Autenticazione multi-factor.
\end{itemize}
Con questo meccanismo non c'è più un \emph{perimetro} ma si assume che nel sistema ci sia sempre un attaccante.

\subsection{Principio del meccanismo comune}
Si basa sul minimizzare la condivisione di risorse tra utenti poiché ogni canale di condivisione potrebbe essere
usato per violare il modello di sicurezza a livelli.

Nella pratica un attaccante potrebbe sfruttare il fatto che una risorsa potrebbe essere condivisa tra utenti di
livelli diversi e il fatto che tale risorsa ha sempre e comunque uno stato. Lo stato della risorsa potrebbe essere
usato per trasmettere informazioni.

Il problema si risolve eliminando risorse condivise tra utenti di livello diverso là dove non sono necessarie. Ogni
utente avrà quel tipo di risorsa per il suo livello e nessun utente di livello diverso potrà interagire con essa.

\subsection{Work factor}
Per valutare la robustezza di un sistema si misura la \textbf{quantità di lavoro} necessaria ad un attaccante per
violare il sistema. Per fare ciò si deve \textbf{emulare} l'attaccante e cercare di scoprire l'ordine in cui questo
compie le azioni della sua intrusione.

L'emulazione di attacchi ha una serie di problemi come
\begin{itemize}
	\item Necessità di \textbf{prove ripetute} per avere una valutazione consistente delle vulnerabilità.
	\item \textbf{Eventi stocastici} che provocano violazioni al sistema molto efficienti o inefficienti, dunque non
	      utili per una valutazione generale dell'attacco.
	\item Attaccare il sistema provoca \textbf{rumore} che disturba il sistema.
	\item Il \textbf{costo} di emulazione di un attacco è alto.
\end{itemize}

\subsection{Compromise recording}
Questo principio afferma che a volte è più conveniente essere attaccati e riportare il sistema in uno stato
consistente piuttosto che impedire totalmente l'attacco.

Per riuscire a fare questo si fa utilizzo di \textbf{file di log} usati per registrare tutte le azioni compiute sul
sistema dai vari soggetti per riuscire a ricostruire uno stato consistente del sistema.

I file di log permette di controllare abusi di privilegi che il soggetto deve possedere ma uso in modo inappropriato.

I file di log \textbf{non} devono essere memorizzati sullo stesso nodo di rete del quale vogliamo ricordare gli
eventi.

L'utilizzo di una \emph{blockchain} è utile per riuscire ad avere più copie per un file di log, il quale deve essere
poi accettato come valido da tutti i membri della blockchain.

\section{Approccio sistemico}
In generale una violazione dei principi precedentemente elencati indica una \textbf{debolezza} e non una
vulnerabilità.

Una debolezza è un difetto del sistema che può diventare una vulneabilità nel momento in cui qualcuno scopre come
sfruttarla per violare il sistema stesso.

Quando si applicano i principi ad un modulo, questi ci dicono quanto questo modulo sia robusto. Quando però li
applichiamo ad un sistema. Non è detto che le vulnerabilità del singolo modulo si trasferiscano all'intero sistema.
Questo perché si possono effettuare i cosiddetti \textbf{controlli compensativi o delegati}, in cui un componente
rimedia ai difetti di un altro.

Questo potrebbe portare anche un incremento delle prestazioni nel momento in cui il sistema comprenda un modulo
specializzato a fare controlli in modo efficiente.

\subsection{NIST Cybersecurity framework}
Si tratta di un framework che propone una modalità di miglioramento della sicurezza per il proprio sistema, piccolo o
grande che sia, diviso in cinque punti fondamentali:
\begin{itemize}
	\item \textbf{Identificare} cosa si vuola proteggere.
	\item \textbf{Proteggere} ciò che si è individuato al punto precedente.
	\item \textbf{Scoprire} attacchi in corso.
	\item \textbf{Reagire} all'intrusione.
	\item \textbf{Recupero} di uno stato consistente dopo che si viene attaccati.
\end{itemize}
