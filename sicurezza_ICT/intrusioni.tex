\chapter{Intrusioni}
Un punto fondamentale per la sicurezza è conoscere quali sono le possibili intrusioni che un attaccante potrebbe fare
nel nostro sistema.

Contrariamente a quanto si potrebbe pensare, conoscere solo alcune intrusioni e modificare il sistema in modo da
bloccare solo il sottoinsieme noto, la probabilità di successo di un'intrusione può aumentare (Teorema di Bayes).

Quanto appena detto è stato dimostrato e spiega una potenziale causa degli insuccessi nei metodi di valutazione della
sicurezza basati sulla scoperta di un numero limitato di intrusioni.

\section{Costruzione di un'intrusione}
Per riuscire a scoprire un'intrusione è fondamentale la tecnica chiamata \textbf{adversary emulation} che consiste nel
capire come l'attaccante riesce a sequenzializzare correttamente le sue azione per riuscire a raggiungere un certo
obbiettivo.

Il punto di partenza per modellare le azioni di un attaccante è quello di identificare \textbf{diritti iniziali} e i
\textbf{diritti finali}. Dopo questa fase iniziale si deve riuscire a capire quale sia la catena di azione che
permette, a partire dai diritti iniziali, di acquisire i diritti finali.

Una volta scoperti gli attacchi possiamo scoprire anche quali siano le informazioni necessarie per eseguire gli
attacchi.

\subsection{Attack graph}
L'\textbf{attack graph} è un grafo orientato privo di cicli associato ad uno specifico sistema in cui
\begin{itemize}
	\item Ad ogni nodo è associato un insieme di diritti che l'attaccante possiede in un certo momento.
	\item Ogni arco è etichettato con un attacco.
	\item I diritti necessari per eseguire un attacco si trovano nei nodi del grafo.
\end{itemize}
Il nodo iniziale del grafo è composto dai diritti \emph{legali} che tutti possiedono sul sistema, mentre in ognuno dei
nodi finali sono presenti i diritti che un attaccante ha acquisito.

Possiamo vedere l'intrusione come il \textbf{cammino} da un nodo iniziale ad uno finale con conseguente aumento dei
diritti acquisiti dall'attaccante.

L'acquisizione dei diritti è monotona quindi in un nodo si possono eseguire tutti gli attacchi possibili nel cammino
fino a quel nodo purché
\begin{itemize}
	\item Non siano già stati eseguiti.
	\item Concedano almeno un diritto non ancora posseduto.
\end{itemize}
L'acquisizione monotona dei diritti è dovuta al fatto che si modellano solo i possibili attacchi e non le possibili
azioni di difesa, portando così a grafi molto ricchi e di difficile costruzione nel caso di sistemi ragionevolmente
complessi.

Conviene considerare solo gli attacchi che sono previsti dalla strategia dell'attaccante e non tutti gli attacchi
possibili.

Questo metodo è utile per capire se un certo obbiettivo è raggiungibile ma non per capire come fare per raggiungerlo.

Chiariamo che solo il difensore di un sistema può costruire l'attack graph senza eseguire l'attacco in quanto conosce
le componenti del sistema e tutti i diritti necessari per compiere le varie azioni.

\subsection{Attack tree}
Si tratta di una versione semplificata dell'attack graph in cui un'intrusione viene rappresentata come un albero i cui
nodi rappresentano gli attacchi ed i sottoalberi rappresentano il come un certo attacco può essere eseguito.

L'albero può anche essere costruito in una versione \verb|AND/OR|:
\begin{itemize}
	\item \verb|AND|: tutti gli attacchi nei nodi figli devono essere eseguiti.
	\item \verb|OR|: basta un solo attacco tra quelli presenti nei nodi figli.
\end{itemize}
Si tratta di un metodo utile a rappresentare la decomposizione dell'intrusione e non a costruirla.

\section{Fermare le intrusioni}
Per fermare una minaccia occorre fermare un attacco in ogni intrusione che la minaccia può eseguire. Fermando un
attacco la catena dell'intrusione considerata viene interrotta.

Se si riesce a fermare un atacco che appare in più intrusioni si fermano tutte quelle intrusioni.

\subsection{Ranking delle vulnerabilità}
Il problema di ottimizzazione descritto in precedenza indica come sia possibile ottenere uno scoring delle
vulnerabilità relativo al sistema considerato.

Una vulnerabilità dipende dal numero di intrusioni che si riescono a fermare eliminando tale vulnerabilità. In
generale, si cerca di procedere eliminando prima la vulnerabilità che appare nel maggior numero di intrusioni non
ancora fermate.

\section{Persistenza}
Il problema della \textbf{persistenza} si verifica quando attaccanti interessati al furto di informazioni o
all'attivazione di malware all'interno del sistema continuano ad essere persistenti all'interno del sistema nonostante
le azioni dei difensori.

Queste azioni vengono eseguite una volta completata l'intrusione ossia una volta raggiunto un certo obbiettivo.

\section{Descrizione di una minaccia}
Riuscire a descrivere un attaccante è diventato un compito di sempre maggiore importantanza, ecco perché, a tal fine,
sono nati strumenti come la \textbf{Mitre Attack Matrix} costruita in questo modo:
\begin{itemize}
	\item Ogni colonna indica una \textbf{tattica}, ossia un obbiettivo a breve termine durante un'intrusione.
	\item Ogni riga indica una \textbf{tecnica}, ovvero il come viene raggiunto un obbiettivo a breve termine. Ogni
	      tattica è composta di più tecniche per riuscire a raggiungere l'obbiettivo.
	\item L'implementazione dettagliata di una tecnica è chiamata \textbf{procedura}.
\end{itemize}
Questa matrice permette un'emulazione molto dettagliata di un avversario ma è carente sul tipo di strategia utilizzata,
ossia sull'ordine con cui vengono effettuate le varie azioni.

\subsection{Attacchi al cloud}
Secondo il MITRE le intrusioni in sistemi cloud, di solito, non comprendono malware perché il provider è in grado di
rilevarlo.

Ecco che in questo ambito sono nati nuovi comportamenti possibili come furto di credenziali, creazione di nuove
macchine virtuali. Un'operazione come l'\textbf{esfiltrazione} può essere implementata come trasferimento di dati
da un account all'altro.

\subsection{Infrastruttura di comando e controllo}
La matrice citata in precedenza trascura azioni dell'attaccante per creare una propria infrastruttura di comando e
controllo che può servire sia per lanciare i primi passi di intrusione che per interagire con il sistema attaccato
quando intrusione o persistenza hanno successo.

Le azioni per creare tale infrastruttura non riguardano il sistema target e quindi non è possibile tenerne conto per
rilevare e sconfiggere un'intrusione. La sua creazione è semplificata se esiste un grande numero di sistemi a bassa
sicurezza per riuscire a creare facilmente su ognuno di essi una botnet.