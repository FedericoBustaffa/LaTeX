\section{Schema concettuale}
Di seguito lo schema concettuale descritto nella sezione precedente.

\begin{center}
	\includesvg[inkscapelatex=false, scale=0.725] {schema/concettuale.svg}
\end{center}

\subsection{Vincoli intrarelazionali}
Ogni prodotto può rimanere all'interno del carrello per un periodo di trenta minuti, a meno che
non sia stata richiesta una proroga, che aumenta il tempo massimo fino a 2 ore. Una volta
scaduto il tempo disponibile all'interno del carrello, questo viene eliminato automaticamente.

Per quanto riguarda gli ordini ho immaginato che ognuno di essi potesse avere diversi stati:
AttesaConferma, Confermato, InViaggio e Consegnato.

All'interno dell'entità Ordine abbiamo anche un campo DataConsegna che non indica la data prevista
ma la data in cui è effettivamente avvenuta la consegna.

Abbiamo quindi che l'ordine ha il campo DataConsegna nullo finché lo stato dell'ordine è diverso
da Consegnato. Una volta avvenuta la consegna lo stato deve passare in Consegnato e DataConsegna
deve assumere un valore non nullo.

\subsection{Vincoli interrelazionali}
In caso di proroga, un modello mantenuto più di due ore nel carrello viene rimosso e non può
essere reinserito nel caso in cui sia sotto il livello di scorta. In tal caso si devono attendere
altre due ore prima di poter reinserire il prodotto.

La notifica che il sistema invia agli amministratori, nel caso ci sia un'impennata di vendite,
deve avvenire nel caso in cui si verifichi un aumento del 20\% in 3 giorni.

Per quanto riguarda la gerarchia delle promozioni abbiamo un vincolo di copertura in quanto ogni
promozione deve essere di uno e uno soltanto dei tre tipi specificati dalle sottoclassi e non può
essere una promozione generica.