\documentclass[12pt, a4paper]{article}

\usepackage[T1]{fontenc}
\usepackage[italian]{babel}
\usepackage[hidelinks]{hyperref}
\usepackage[margin=1.25in]{geometry}

\usepackage{amsmath}
\usepackage{amssymb}
\usepackage{amsthm}
\usepackage{amsfonts}
\usepackage{mathtools}

\usepackage{minted}
\usepackage{svg}

\definecolor{minted_bg}{rgb}{0.9, 0.9, 0.9}
\usemintedstyle{colorful}

\setminted[sql]{
	tabsize=4,
	% linenos=true,
	bgcolor=minted_bg,
	fontsize=\small,
	mathescape=true
}

\title{Progetto basi di dati}
\author{Federico Bustaffa 583309}
\date{01/03/2024}

\begin{document}

\maketitle
\tableofcontents

\section{Descrizione del dominio}
Il sistema ha il compito di gestire la vendita online di occhiali. La prima entità rappresenta gli
\textbf{utenti} registrati nel sistema e i loro dati. Quello che interessa principalmente è
l'indirizzo email, diverso per ogni utente.

Ogni utente può \emph{visualizzare} i vari \textbf{modelli} di occhiali, ciascuno dei quali
possiede delle caratteristiche come la categoria, il marchio, il materiale, prezzo, quantità
disponibile e così via.

L'utente può inoltre \textbf{personalizzare} il modello che sta visualizzando specificando colore
e forma di lenti e montatura tra quelli disponibili per quel modello. Ogni modello ha almeno una
personalizzazione possibile e all'utente ne viene mostrata una di default.

Ogni utente possiede un \textbf{carrello} che può \emph{modificare}, come preferisce, andando ad
aggiungere o eliminare prodotti e andando prorogare, se lo desidera, il tempo in cui un certo
prodotto può rimanere nel carrello.

Ogni carrello possiede quindi una lista di modelli con il relativo timestamp del momento in cui
sono stati inseriti e un valore booleano che indica se è stata specificata la volontà di prorogare
il tempo oltre il quale il prodotto verrebbe eliminato automaticamente dal carrello.

Sono inoltre presenti valori come il costo totale dei prodotti nel carrello e il costo di
spedizione.

L'utente può inoltre \emph{ordinare} ciò che ha nel carrello \emph{generando} un \textbf{ordine},
contenente le informazioni relative all'acquisto come il metodo di pagamento, la data, lo stato e
la data in cui è avvenuta la consegna. Se l'acquisto viene confermato sarà il sistema ad inviare
un'email di conferma all'utente.

L'utente è anche in grado di \emph{effettuare} una \textbf{recensione} se lo desidera, dando una
valutazione e, se lo desidera scrivendo un commento. Ogni recensione \emph{riguarda} un solo
modello di occhiali.

Abbiamo infine un'entità \textbf{amministratore} con cui ci riferiamo alle figure che gestiscono
il sistema e che dunque sono in grado di \emph{gestire} vari tipi di \textbf{promozione}. Le
promozioni possono essere di tre tipologie:
\begin{itemize}
	\item \textbf{Sconti}: quando si vuole applicare una certa percentuale di sconto su uno o più
	      modelli.
	\item \textbf{Offerte}: sono della tipologia in cui se si fanno almeno un certo numero di
	      acquisti si riceve qualcosa in omaggio o un sconto sul totale.
	\item \textbf{Codice promozionale}: sono codici che l'utente può usare in un secondo momento
	      per ottenere vantaggi su futuri acquisti.
\end{itemize}
Non ci possono essere promozioni generiche ma solo di una di queste tre tipologie e ognuna riguarda
uno o più modelli di occhiali.

Il sistema è inoltre in grado di \emph{segnalare} agli amministratori quando si verificano cali o
impennate delle vendite per un certo modello.

\section{Schema concettuale}
Di seguito lo schema concettuale descritto nella sezione precedente

\includesvg[inkscapelatex=false, scale=0.425] {schema/concettuale.svg}

\subsection{Vincoli intrarelazionali}

\subsubsection{Carrello}
Ogni prodotto può rimanere all'interno del carrello per un periodo massimo di trenta minuti, a
meno che non sia stata richiesta una proroga. Una volta scaduto il tempo disponibile all'interno
del carrello, questo viene eliminato automaticamente.

\subsubsection{Gerarchia}
Per quanto riguarda la gerarchia delle promozioni abbiamo un vincolo di copertura in quanto ogni
promozione deve essere di uno e uno soltanto dei tre tipi specificati dalle sottoclassi e non può
essere una promozione generica.

\subsection{Vincoli interrelazionali}
\subsubsection{Carrello}
Un modello mantenuto più di due ore nel carrello e dunque rimosso, non può essere reinserito nel
caso in cui sia sotto il livello di scorta. In tal caso si devono attendere altre due ore prima di
poter reinserire il prodotto.

La proroga del tempo di mantenimento all'interno del carrello non può essere effettuata più volte,
una volta richiesta si hanno due ore di tempo prima che il prodotto venga rimosso.

\subsubsection{Personalizzazioni}
L'utente può personalizzare gli occhiali solo con personalizzazioni relative al modello che sta
visionando.

\subsubsection{Segnalazioni vendite}
La notifica che il sistema invia agli amministratori, nel caso ci sia un'impennata di vendite,
deve avvenire nel caso in cui si verifichi un aumento del 20\% in 3 giorni.

\end{document}
