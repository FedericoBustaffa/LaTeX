\documentclass[12pt, a4paper]{article}

\usepackage[T1]{fontenc}
\usepackage[italian]{babel}
\usepackage[hidelinks]{hyperref}
\usepackage[margin=1.25in]{geometry}

\usepackage{amsmath}
\usepackage{amssymb}
\usepackage{amsthm}
\usepackage{amsfonts}
\usepackage{mathtools}

\usepackage{minted}
\usepackage{svg}

\definecolor{minted_bg}{rgb}{0.9, 0.9, 0.9}
\usemintedstyle{colorful}

\setminted[sql]{
	tabsize=4,
	% linenos=true,
	bgcolor=minted_bg,
	fontsize=\small,
	mathescape=true
}

\title{Progetto basi di dati}
\author{Federico Bustaffa 583309}
\date{01/03/2024}

\begin{document}

\maketitle
\tableofcontents

\section{Descrizione del dominio}
Il sistema ha il compito di gestire la vendita online di occhiali. La prima entità rappresenta
gli \textbf{utenti} registrati nel sistema e i loro dati. Quello che interessa principalmente
è l'indirizzo email, diverso per ogni utente.

Ogni utente può visualizzare i vari \textbf{modelli} di occhiali, ciascuno dei quali possiede
delle caratteristiche come la categoria, il marchio, il materiale, prezzo, quantità disponibile
e così via.

L'utente può inoltre applicare delle \textbf{personalizzazioni} al modello che sta visualizzando
specificando colore e forma di lenti e montatura tra quelli disponibili per quel modello. Ogni
modello ha almeno una personalizzazione possibile.

Ogni utente \emph{possiede} un \textbf{carrello}, al quale si possono \emph{aggiungere}, o
\emph{rimuovere} modelli di occhiali (con relativa personalizzazione) e andando a prorogare, se
lo desidera, il tempo in cui un certo prodotto può rimanere nel carrello.

Ogni carrello possiede quindi tre liste da intendere \emph{parallele}: una con i modelli, una con
il timestamp del momento in cui ogni modello è stato inseriti e una di valori booleano che indicano
se è stata specificata la volontà di prorogare il tempo oltre il quale il prodotto verrebbe
eliminato automaticamente dal carrello.

Sono inoltre presenti valori come il costo totale dei prodotti nel carrello e il costo di
spedizione.

L'utente può \emph{effettuare} un \textbf{ordine}, acquistando ciò che ha nel carrello.
Ogni ordine contiene le informazioni relative all'acquisto come un identificatore dell'utente, il
metodo di pagamento, la data, lo stato e la data in cui è avvenuta la consegna.

L'utente è anche in grado di \emph{effettuare} una \textbf{recensione} se lo desidera, dando una
valutazione e scrivendo un commento. Ogni recensione \emph{riguarda} un solo modello di occhiali.

Abbiamo infine un'entità \textbf{amministratore} con cui ci riferiamo alle figure che gestiscono
il sistema e che dunque sono in grado di \emph{gestire} vari tipi di \textbf{promozione}. Le
promozioni possono essere di tre tipologie:
\begin{itemize}
	\item \textbf{Sconti}: quando si vuole applicare una certa percentuale di sconto su uno o più
	      modelli.
	\item \textbf{Offerte}: sono della tipologia in cui se si fanno almeno un certo numero di
	      acquisti si riceve qualcosa in omaggio o un sconto sul totale.
	\item \textbf{Codice promozionale}: sono codici che l'utente può usare in un secondo momento
	      per ottenere vantaggi su futuri acquisti.
\end{itemize}
Non ci possono essere promozioni generiche ma solo di una di queste tre tipologie e ognuna riguarda
uno o più modelli di occhiali.

\section{Schema concettuale}
Di seguito lo schema concettuale descritto nella sezione precedente

\includesvg[inkscapelatex=false, scale=0.475] {schema/concettuale.svg}

\subsection{Vincoli intrarelazionali}

\subsubsection{Carrello}
Ogni prodotto può rimanere all'interno del carrello per un periodo massimo di trenta minuti, a
meno che non sia stata richiesta una proroga. Una volta scaduto il tempo disponibile all'interno
del carrello, questo viene eliminato automaticamente.

\subsubsection{Gerarchia}
Per quanto riguarda la gerarchia delle promozioni abbiamo un vincolo di copertura in quanto ogni
promozione deve essere di uno e uno soltanto dei tre tipi specificati dalle sottoclassi e non può
essere una promozione generica.

\subsection{Vincoli interrelazionali}
\subsubsection{Carrello}
Un modello mantenuto più di due ore nel carrello e dunque rimosso, non può essere reinserito nel
caso in cui sia sotto il livello di scorta. In tal caso si devono attendere altre due ore prima di
poter reinserire il prodotto.

La proroga del tempo di mantenimento all'interno del carrello non può essere effettuata più volte,
una volta richiesta si hanno due ore di tempo prima che il prodotto venga rimosso.

\subsubsection{Personalizzazioni}
L'utente può personalizzare gli occhiali solo con personalizzazioni relative al modello che sta
visionando.

\subsubsection{Segnalazioni vendite}
La notifica che il sistema invia agli amministratori, nel caso ci sia un'impennata di vendite,
deve avvenire nel caso in cui si verifichi un aumento del 20\% in 3 giorni.

\end{document}
