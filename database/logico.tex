\section{Schema logico}
Di seguito lo schema logico derivante dallo schema concettuale proposto nella sezione precedente.

\begin{center}
	\includesvg[inkscapelatex=false, scale=0.6125] {schema/logico.svg}
\end{center}

\subsection{Formato testuale}
Di seguito lo schema logico in formato testuale.
\begin{itemize}
	\item Utente (\underline{IndirizzoEmail}, Nome, Cognome, \underline{carrello*})
	\item Carrello (\underline{IdCarrello})
	\item Modello (\underline{CodiceModello}, Categoria, Marchio, Materiale, Gradazione, Prezzo)
	\item Personalizzazione (\underline{CodicePersonalizzazione}, ColoreMontatura, FormaMontatura,
	      ColoreLenti, FormaLenti, Quantità, Immagini)
	\item ModelloCarrello (\underline{Modello*}, \underline{Carrello*}, TimestampInserimento,
	      Proroga, Quantità)
	\item PersonalizzazioniModello (\underline{Personalizzazione*}, \underline{Modello*})
	\item Recensione (\underline{Utente*}, \underline{Modello*}, Valutazione, Commento)
	\item Ordine (\underline{CodiceOrdine}, \underline{Utente*}, MetodoPagamento, Data, Stato,
	      DataConsegna, CostoSpedizione)
	\item OrdineModello (\underline{Ordine*}, \underline{Modello*}, Quantità)
	\item Responsabile (\underline{Matricola}, Nome, Cognome)
	\item Promozione (\underline{CodicePromozione}, \underline{Responsabile*})
	\item Sconto (\underline{CodicePromozione*}, Percentuale)
	\item Offerta (\underline{CodicePromozione*}, Descrizione)
	\item CodicePromozionale (\underline{CodicePromozione*}, Codice, Descrizione)
\end{itemize}

\subsection{Dipendenze funzionali}
Di seguito sono riportate le dipendenze funzionali ricavate dallo schema logico.

\begin{itemize}
	\item \textbf{Utente}: \underline{IndirizzoEmail} $\rightarrow$ Nome, Cognome,
	      \underline{Carrello*}
	\item \textbf{Modello}:
	\item \textbf{Personalizzazione}:
	\item \textbf{Recensione}:
	\item \textbf{Ordine}:
	\item \textbf{Responsabile}:
	\item \textbf{Promozione}:
	\item \textbf{ModelloCarrello}:
\end{itemize}