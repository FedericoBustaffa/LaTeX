\section{Schema logico}
Di seguito lo schema logico derivante dallo schema concettuale proposto nella sezione precedente.

\begin{center}
	\includesvg[inkscapelatex=false, scale=0.6125] {schema/logico.svg}
\end{center}

\subsection{Formato testuale}
Di seguito lo schema logico in formato testuale.
\begin{itemize}
	\item Utenti (\underline{IndirizzoEmail}, Nome, Cognome, \underline{Carrello*})
	\item Carrelli (\underline{IdCarrello})
	\item Modelli (\underline{CodiceModello}, Categoria, Marchio, Materiale, Gradazione, Prezzo,
	      Quantità)
	\item Personalizzazioni (\underline{CodicePersonalizzazione}, ColoreMontatura, FormaMontatura,
	      ColoreLenti, FormaLenti, Immagini)
	\item ModelliNelCarrello (\underline{Modello*}, \underline{Carrello*}, TimestampInserimento,
	      Proroga, Quantità)
	\item PersonalizzazioniModello (\underline{Personalizzazione*}, \underline{Modello*})
	\item Recensioni (\underline{Utente*}, \underline{Modello*}, Valutazione, Commento)
	\item Ordini (\underline{CodiceOrdine}, \underline{Utente*}, MetodoPagamento, DataOrdine,
	      Stato, DataConsegna, CostoSpedizione)
	\item ModelliOrdinati (\underline{Ordine*}, \underline{Modello*}, Quantità)
	\item Responsabili (\underline{Matricola}, Nome, Cognome)
	\item Promozioni (\underline{CodicePromozione}, \underline{Responsabile*})
	\item Sconti (\underline{CodicePromozione*}, Percentuale)
	\item Offerte (\underline{CodicePromozione*}, Descrizione)
	\item CodiciPromozionali (\underline{CodicePromozione*}, Codice, Descrizione)
\end{itemize}

\subsection{Dipendenze funzionali}
Di seguito sono riportate le dipendenze funzionali ricavate dallo schema logico.
\begin{itemize}
	\item \textbf{Utenti}: \underline{IndirizzoEmail} $\rightarrow$ Nome, Cognome,
	      \underline{Carrello*}
	\item \textbf{Modelli}: \underline{CodiceModello} $\rightarrow$ Categoria, Marchio, Materiale,
	      Gradazione, Prezzo
	\item \textbf{Personalizzazioni}: \underline{CodicePersonalizzazione} $\rightarrow$
	      ColoreMontatura, FormaMontatura, ColoreLenti, FormaLenti, Quantità, Immagini
	\item \textbf{Recensioni}: \underline{Utente*}, \underline{Modello*} $\rightarrow$ Valutazione,
	      Commento
	\item \textbf{Ordini}: \underline{CodiceOrdine} $\rightarrow$ \underline{Utente*},
	      MetodoPagamento, DataOrdine, Stato, DataConsegna, CostoSpedizione
	\item \textbf{Responsabili}: \underline{Matricola} $\rightarrow$ Nome, Cognome
	\item \textbf{Promozioni}: \underline{CodicePromozione} $\rightarrow$ \underline{Responsabile*}
	\item \textbf{Sconti}: \underline{CodicePromozione} $\rightarrow$ Percentuale
	\item \textbf{Offerte}: \underline{CodicePromozione} $\rightarrow$ Descrizione
	\item \textbf{CodiciPromozionali}: \underline{CodicePromozione} $\rightarrow$ Codice,
	      Descrizione
	\item \textbf{ModelliNelCarrello}: \underline{Modello*}, \underline{Carrello*} $\rightarrow$
	      Proroga, Quantità, TimestampInserimento
	\item \textbf{ModelliOrdinati}: \underline{Ordine*}, \underline{Modello*} $\rightarrow$ Quantità
\end{itemize}
Tutte le dipendenze funzionali rispettano la forma normale di Boyce-Codd.
