\section{Schema concettuale}
Di seguito lo schema concettuale descritto nella sezione
precedente.

\begin{figure}[!ht]
	\centering
	\includesvg[inkscapelatex=false, scale=0.6]{concettuale.svg}
	\caption{Schema concettuale}
\end{figure}

\subsection{Vincoli intrarelazionali}
Ogni esame ha un voto che deve essere compreso tra 18 e 32,
ovviamente uno studente non può aver sostenuto e non gli può
essere convalidato un esame con votazione inferiore a 18. In
caso di lode viene invece considerato il voto 32 a prescindere
dal peso che veniva dato alla lode nella precedente università,
corso o ordinamento.

Nel caso si voglia effettuare un ricongiungimento dopo inattività
l'anno di interruzione specificato deve essere maggiore o uguale
a quello di iscrizione.

\subsection{Vincoli interrelazionali}
Gli esami inseriti dallo studente devono rispettare il numero
e la tipologia dei crediti previsti dal corso di studi da cui
proviene.

Gli esami convalidati devono rispettare i vincoli imposti dal
regolamento dell'anno accademico in corso. Non è quindi
possibile convalidare esami che non rispettano i vincoli
riguardanti i crediti (di qualsiasi tipo).
