\section{Navigazione di un robot}
\subsection{Formulazione del problema}
\begin{enumerate}
	\item \textbf{Stato iniziale}: punto $P$.
	\item \textbf{Azioni possibili}: Spostarsi in un punto del piano corrispondente a:
	      \begin{itemize}
		      \item Un vertice di uno tra i poligoni ostacolo.
		      \item Il punto $P$.
		      \item Il punto $A$.
	      \end{itemize}
	\item \textbf{Modello di transizione}: Sia $\alpha$ il punto in cui mi trovo e
	      $\Delta_{\alpha \rightarrow \beta}$ lo spostamento che mi porta da $\alpha$ a $\beta$, allora
	      \[ \alpha \times \Delta_{\alpha \rightarrow \beta} = \beta \]
	\item \textbf{Goal Test}:
	      \begin{itemize}
		      \item $\alpha = A \quad \rightarrow \quad \text{TRUE}$
		      \item $\alpha \neq A \quad \rightarrow \quad \text{FALSE}$
	      \end{itemize}
	\item \textbf{Costo del cammino}: Somma dei costi delle azioni. Il costo di un
	      azione equivale alla distanza tra i due punti $\alpha, \beta$:
	      \[
		      C(\alpha, \Delta_{\alpha \rightarrow \beta}, \beta) =
		      \sqrt{(\alpha.x - \beta.x)^2 + (\alpha.y - \beta.y)^2}
	      \]
\end{enumerate}

\subsection{Definizione dell'euristica}
Prendiamo la \textbf{distanza in linea d'aria}, che, rispettando la
\emph{disuguaglianza triangolare} \`e una \emph{sottostima} ed \`e dunque un'euristica
\emph{ammissibile}.
\[ f(\alpha) = g(\alpha) + h(\alpha) \]